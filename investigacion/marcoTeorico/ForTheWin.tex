\subsection{For The Win} \label{sec:ForTheWin}

 {\em For The Win} es un marco de trabajo realizado por Dan Hunter y Kevin Werbach, de acuerdo
 con los actores una de las motivaciones por las cuales fue creado este marco de trabajo es que
 había una necesidad real de una guía pragmética basada en la investigación que explicara cómo
 implementar gamificación de forma correcta \cite{ForTheWin}.\\
 
 \noindent Las razones principales por la cual se ha utiliza este marco de trabajo es que describe
 el pensamiento de diseñador de juegos, establece una jerarquía entre los distintos elementos de
 juego en la gamificación (ver figura \ref{fig:ForTheWin}), y presenta una serie de pasos para
 su implementación.

    \addfigure[(adaptado de {\it For The Win} \cite{ForTheWin})]%
        {.35}{investigacion/images/forthewin}{fig:ForTheWin}%
        {Jerarquía de elementos de juego segun For The Win}

 \begin{quote}
    A continuación se describe el pensamiento de diseñador de juegos, los distintos elementos
    de juego y los pasos para implementar gamificación.
 \end{quote}

 % \noindent De acuerdo con For The Win, para implementar gamificación se necesitan contemplar
 % los tres tipos de elementos de juego, Dinámicas, Mecánicas y Componentes. Los tipos de
 % elementos son organizados en una pirámide (figura \ref{fig:ForTheWin}) de acuerdo con
 % su nivel de abstracción y el objetivo que tienen \cite[pp. 55-57]{ForTheWin}.
 
\subsubsection{Pensamiento de diseñador de juegos}
   
\subsubsection{Elementos de juego}
\subsubsection{Dinámicas}
\subsubsection{Mecánicas}
\subsubsection{Componentes}
\subsubsection{Integración}
\subsubsection{Pasos para gamificar}

    \begin{multicols}{2}
    \noindent
    {\bf Nivel: Dinámicas}.
        Las dinámicas son lo más abstracto, es la temática que envuelve a todo el sistema.
        Existen 5 dinámicas, las cuales son:
        
        \begin{enumerate}
            \item Restricciones
            \item Emociones
            \item Historia
            \item Progresión
            \item Relaciones sociales
        \end{enumerate}

    \vfill\null
    \columnbreak

    \noindent
    {\bf Nivel: Mecánicas}.
        Las mecánicas son el motivo para que se haga alguna acción, son las que mantienen
        enganchado al jugador. Existen 10 mecánicas, las cuales son:
        
        \begin{enumerate}
            \item Desafíos
            \item Suerte 
            \item Competencia
            \item Cooperación
            \item Retroalimentación 
            \item Obtención de elementos
            \item Recompensas
            \item Transacciones
            \item Turnos
            \item Ganadores y perdedores\\
        \end{enumerate}

    \end{multicols}
\clearpage

        \noindent\textbf{Nivel: Componentes} Los componentes son la forma de implementar las mecánicas y las dinámicas. Existen 15 componentes, los cuales son:
        
    \begin{multicols}{2}
        \begin{enumerate}
            \item Logros
            \item Avatares
            \item Insignias
            \item Peleas de jefes finales
            \item Colecciones
            \item Combates
            \item Desbloqueo de contenido
            \item Regalos e intercambios
            \item Tablas de líderes
            \item Niveles de personaje (Experiencia)
            \item Puntos
            \item Misiones
            \item Esquemas sociales
            \item Equipos
            \item Moneda virtual
        \end{enumerate}
    \end{multicols}
    
    
    \noindent  For The Win establece que para cumplir con gamificación no es necesario tener cada uno de los elementos anteriores, ya que establece que antes de cantidad se necesita calidad, refiriéndose a que los elementos tengan coherencia entre sí.
    
    \subsubsection{ Proceso de implementación}
    
    For the Win indica que el proceso consta de 6 pasos que especifican cómo introducir la gamififación, cada uno de los pasos se describen a continuación \cite[pp. 60-70]{ForTheWin}.\\
    
    \noindent \textbf{1.- Definir los objetivos del negocio}\\
    
    \noindent Los objetivos no se refieren a los planteados en la visión y misión de la empresa, sino al ''¿Por qué?'' se está haciendo este sistema que tiene implementada la gamificación.\\
    
    \noindent \textbf{2.- Delimita las acciones de tus usuarios}\\
    
    \noindent Ya definido el objetivo, se tiene que ver que acciones tus usuarios podrán desarrollar en el sistema, dichas acciones deben ser concretas y específicas. Por ejemplo: Iniciar sesión el la página web, compartir la información del trabajo vía twitter y comentar en una publicación de facebook. 
    
    \noindent Dichas acciones tienen que estar relacionadas con el ''¿Por qué?''.\\
    
    \noindent \textbf{3.- Describe a tus usuarios}\\
    
    \noindent ¿Qué usuarios estarán usando tu sistema? y aún más importante, ¿cuál es tu relación con ellos? y/o ¿qué tanto sabes de ellos? Esto para poder conocer qué podría motivarlos.\\
    
    
    \noindent \textbf{4.- Define tus actividades de inicio a fin}\\
    
    \noindent Conociendo a tus usuarios y tus objetivos ya puedes diseñar que actividades tendrá tu sistema y cómo es el flujo en cada una de ellas. En los juegos siempre las actividades tienen un inicio y a veces tienen un final. Y hay veces que se tienen ciclos antes de llegar al final. Por eso mismo se debe tomar en cuenta que hay 2 posibles formas de crear tu flujo de actividad: de forma de ciclo y forma de escaleras.\\
    
    
    
    
    \noindent \textbf{5.- Nunca olvides la diversión}\\
    
    \noindent Antes de empezar a usar el sistema se tiene que dar un paso atrás y preguntarte si al menos tú consideras que es divertido, si a ti te gustaría probar el hacer dichas actividades voluntariamente.\\
    
    
    \noindent \textbf{6.- Utiliza las herramientas adecuadas para el trabajo}\\
    
    \noindent En esta paso se tiene que especificar qué elementos de juego se utilizarán a lo largo de las actividades diseñadas anteriormente y empezar a codificarlas en tu sistema.\\
    
%\subsection{Marco de trabajo C}
% TODO: Agregar el paper que habla d epapers aquí.

\begin{comment}
\section{Elección de Marco de Trabajo}

%Hemos decidido utilizar  a Yukai-Cho(Octalysis)

    Gracias a que Octalysis divide la implementación de la Gamificación en 8 principios, da una flexibilidad mayor en su implementación puesto que se pueden elegir diferentes herramientas para implementar sus principios, a diferencia de los otros autores que solo enumeran las herramientas más usadas (puntos, insignias y tablas de clasificación.) y no dan cabida al uso de otras distintas decidimos utilizar 0ctalysis por las ventajas.
     
     - Modularidad de principios
     - 
     
     
    Que principios se tendrán.
\end{comment}

