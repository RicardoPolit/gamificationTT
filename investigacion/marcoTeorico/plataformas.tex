

\begin{comment}

\section{Acerca de la investigación}

 La primer etapa de la investigación se centro en investigar acerca de qué es la \gls{gamificacion},
 sus implicaciones y como debería ser implementada la busqueda permitió encontrar diversas
 definiciones de la gamificación, las bases sobre la cual está desarrollada y distintas
 recomendaciones su implementación.\\

 \noindent En segunda instancia, se buscaron diversos casos de estudio y documentos de investigación
 que documentaran el proceso de implementación y los resultados de la investigación realizada, cada
 uno de los documentos encontrados nos sirve para conocer las fortalezas y áreas de oportunidad de
 otras implementación de gamificación.\\

 \noindent Como tercer etapa, se investigaron distintos marcos de trabajo para el diseño e implementación
 de la gamificación, dicha busqueda obtuvo como resultado la elección de dos marcos de trabajo
 {\em ``Octalysis''} \cite{Octalysis} y {\em ``For The Win''} \cite{ForTheWin} (descritos en el
 capítulo \nameref{ch:marcoTeorico}).\\

 \noindent Posteriormente se analizaron los principales sistemas de aprendizaje en linea
 buscando qué elementos de gamificación, las restricciones de uso y características principales.
 Esta etapa de investicación permitió filtrar los sistemas que brindaban soporte a la gamificación,
 de forma nativa o mediante componentes externos (plugins), de los demás sistemas de aprendizaje
 en línea.

\clearpage
\end{comment}


\def\aux{90}

\begin{multicols*}{2}    
\subsection*{Duolingo}
    
 Duolingo es un sistema de aprendizaje dedicado a los idiomas, es un servicio web que
 te brinda la posibilidad de crearte una cuenta y seleccionar entre 9 idiomas para aprender,
 los cuales son: Inglés, guaraní, francés, alemán, catalán, espartano, italiano, portugués y ruso.\\

 \noindent Duolingo divide un idioma en secciones y cada sección contiene sub-secciones,
 que a su vez contienen unidades que se dividen en 5 niveles cada una. Al inicio Duolingo
 solo te permite empezar una unidad.\\

 \noindent Al completar el primer nivel de todas las unidades de una subsección duolingo
 te permite avanzar a la siguiente subsección, para poder acceder a la siguiente es requerido
 completar la cantidad de niveles de unidades especificada.\\

 \noindent Duolingo cuenta con varios módulos orientados a la gamificación, los elementos
 de juego, de acuerdo con {\em For The Win}, que tiene son:

    \begin{itemize}
    \item {\bf Logros:} Cuenta con un sistema de logros o en este caso ``insignias''
        que están divididas en 3 niveles, y cada vez que alcanzas un nivel se 
        desbloquea una estrella que se muestra en el icono del logro.
            
    \item {\bf Desbloqueo de contenido:} Al dividir el contenido de la forma
        anteriormente explicada, Duolingo permite visualizar tu progreso viendo
        la cantidad de unidades completadas y desbloqueadas.
            
    \item {\bf Puntos y niveles de experiencia:} Al completar un nivel de una unidad
        se otorgan puntos de experiencia usados para subir de nivel en un idioma.
        
    \item {\bf Tablas de líderes:} Si agregas a alguien como tu amigo en Duolingo ambos 
        podrán ver su progreso semanal, mensual y total. El resultado de que el sistema
         los compara genera la tabla de líderes. 
         
    \item {\bf Misiones:} Duolingo permite que te pongas una meta diaria y una meta semanal.

    \end{itemize}


\subsection*{Docebo}
    
 Docebo es un servicio web que se enfoca en la creación de dominios donde se brinda
 un sistema gestor de aprendizaje, es decir, que uno pueda tener su página en línea
 donde pueda crear y gestionar sus cursos y los alumnos puedan entrar a tomarlos.\\
    
 \noindent Docebo no cuenta con gamificación de raíz, sino que se necesita instalar
 plugins que se desarrollan con la API de Docebo, dichos plugins hasta el momento 
 solo cuentan con:

    \begin{itemize}
        \item {\bf Logros:} Se cuenta con un sistema de logros,
        que se desbloquean si la persona cumple con sus condiciones.
    \end{itemize}
    
    
    
\subsection*{SAP Litmos}
    
 SAP Litmos es un sistema que te permite crear cursos para tu equipo de trabajo,
 así como delegar tareas y ver el progreso de las mismas. Esta orientado a
 fortalecer el capital humano de una empresa.\\
    
    \noindent SAP Litmos cuenta con 3 módulos de gamificación, los cuales son:
    
    \begin{itemize}
        \item {\bf Insignias:} A diferencia que con los logros, estos
        no son otorgados cuando se cumple una cierta condición, sino 
        que el administrador crea una insignia y se le otorga a un usuario.
        
        \item {\bf Equipos: } Debido a que está orientado al capital humano
         de una empresa, uno puede crear equipos que sean por área de la
          empresa y así ver si las áreas están cumpliendo con sus tareas.
          
        \item {\bf Tablas de líderes y puntos: } SAP Litmos te muestra una
         gráfica de que tanto han avanzado los usuarios en un cierto curso
         o en sus tareas. Esto mediante una gráfica y asignación de puntos.
         
    \end{itemize}


\subsection*{ATutor}

 ATutor es un un sitema gestor de aprendizaje de software libre. Para poder
 utilizarlo se necesita tener un servidor web y montar dicho código en el servidor.\\
    
    \noindent ATutor no cuenta con gamificación de raíz,
    pero cuenta con un plugin llamado \textbf{GameMe} que agrega:
    
    \begin{itemize}
        \item {\bf Logros:} Dichos logros son estáticos y se
        desbloquean cuando se un usuario cumple las condiciones.
        
        \item {\bf Puntos y niveles de experiencia:} Hay definidos 10
        niveles de experiencia y cada que ocurre un evento que tenga
        que ver con un usuario, se le otorga experiencia.
        
    \end{itemize}



\subsection*{ALEKS}

 ALEKS es un servicio web que ofrece un sistema gestor de aprendizaje
 que adapta el contenido al usuario utilizando inteligencia artificial.
 Esto lo mantienen controlado utilizando únicamente ciertos tipos de cursos.\\
    
    \noindent ALEKS cuenta con gamificación de raíz, los elementos con los que cuenta son:
    
    \begin{itemize}
        \item {\bf Progresión:} El fuerte de ALEKS es utilizar la inteligencia
        artificial y algoritmos de predicción así que tiene un montón de datos del
        usuario que aprovecha desplegándolos en gráficos que muestran el progreso
        en diversos temas de un curso, así como el porcentaje del curso que se ha
        tomado, dominado o que falta por revisar. Cabe destacar que un profesor puede
        ver los gráficos de cada alumno, pero los alumnos no pueden ver el de los demás.
        
    \end{itemize}
    
\vfill\null
\columnbreak
\subsection*{Udemy}

 Udemy es un servicio web que te permite tomar cursos y/o subir tus cursos.
 El formato de los cursos es siempre un video. Cuanta con muchos temas
 gracias a que cualquiera puede crear su curso.\\

    \noindent Usando como referencia al marco de trabajo octalysis,
    Udemy cuenta con los siguientes principios de gamificación:
    
    \begin{itemize}
        \item {\bf Creatividad} Debido a que cualquiera puede subir
        sus cursos y recibir retroalimentación de los que lo tomaron,
        se cumple este principio, pero dico principio está orientado
        hacia los creadores de cursos.
        
    \end{itemize}
    
    
    
\subsection*{TalentLMS}

 TalentLMS es un servicio web que se enfoca en la creación de dominios donde se
 brinda un sistema gestor de aprendizaje, es decir, que uno pueda tener su página
 en línea donde pueda crear y gestionar sus cursos y los alumnos puedan entrar a tomarlos.\\
    
    \noindent TalentLMS cuenta con gamificación de raíz,
    y los elementos de juego con los que cuenta, son:
    
    \begin{itemize} 
    
        \item {\bf Logros:} Se cuenta con un sistema de logros o en este caso
        ``insignias'' que están divididas en 8 niveles, y cada vez que alcanzas
        un nivel se desbloquea la insignia en su color correspondiente.
        
        \item {\bf Puntos y niveles de experiencia:} Cada que ocurre un
        determinado evento que tenga que ver con un usuario, se le otorga experiencia.
        
        \item {\bf Tablas de líderes y puntos:} TalentLMS muestra la
        tabla de líderes por categoría de curso, esto a nivel ''plataforma''.
        
    \end{itemize}
    
\end{multicols*}



\clearpage
\subsection*{Moodle}

 Moodle es una plataforma de aprendizaje diseñada para proporcionarle a educadores,
 administradores y estudiantes un sistema integrado único, robusto y seguro para crear
 ambientes de aprendizaje personalizados. Los elementos con los que cuenta moodle sin
 la adición de plugins son los siguientes:

 \begin{quote}
 \begin{itemize}
    \item {\bf Insignias}, pueden ser otorgadas dependiendo de multiples variados
                criterios, existen insignias a nivel plataforma y a nivel curso.

    \item {\bf Desbloqueo de contenido}, los profesores o administradores pueden
                definir que las secciones de un curso se vayan desbloqueando
                conforme se vayan cumpliendo ciertas condiciones.

    \item {\bf Creatividad}, permite a los profesores crear distintos cursos
                experimentando con la inclusión de distintos tipos de ejercicios.
 \end{itemize}
 \end{quote}

 \noindent Una de las fortalezas más grandes de moodle es que fue diseñado para ser altamente
 extensible en cuanto a funcionalidades, dentro del inmenso catálogo de componentes {\it plugins}
 para moodle, se encontraron diez plugins que agregan funcionalidades de gamificación. En la tabla
 \ref{tbl:pluginscreated} se presenta a qué elementos de juego están vinculados estos {\it plugins}.


    \addtable{|l|c|c|c|c|c|c|c|c|c|c|}{tbl:pluginscreated}{%
        Elementos/Plugins &
        \rotatebox[origin=c]{\aux}{LevelUp!         \cite{arte2}}  &
        \rotatebox[origin=c]{\aux}{Ranking block    \cite{arte3}}  &
        \rotatebox[origin=c]{\aux}{Game             \cite{arte4}}  &
        \rotatebox[origin=c]{\aux}{Quizventure      \cite{arte5}}  &
        \rotatebox[origin=c]{\aux}{Stash            \cite{arte6}}  &
        \rotatebox[origin=c]{\aux}{Mootivated       \cite{arte7}}  &
        \rotatebox[origin=c]{\aux}{UNEDrivial       \cite{arte8}}  &
        \rotatebox[origin=c]{\aux}{\ Stamp collection \cite{arte9}\ } &
        \rotatebox[origin=c]{\aux}{Exabis games     \cite{arte10}} &
        \rotatebox[origin=c]{\aux}{Badge leader     \cite{arte11}} \\\hline
        
        Competencias            &   &   & X & X &   &   & X &   & X &   \\\hline
        Niveles                 & X &   &   &   &   &   &   &   &   &   \\\hline
        Desbloqueo de contenido & X &   &   &   & X &   &   &   &   &   \\\hline
        Logros                  & X &   &   &   &   &   & X &   &   & X \\\hline
        Esquema financiero      &   &   &   &   &   &   &   &   &   &   \\\hline
        Cajas de botín          &   &   &   &   &   &   &   &   &   &   \\\hline
        Puntos                  & X & X &   &   &   &   &   &   & X &   \\\hline
        Tienda                  &   &   &   &   &   &   &   &   &   &   \\\hline
        Tabla lideres           & X &   &   &   &   &   & X & X &   & X \\\hline
        Barra de progreso       & X &   &   &   &   &   &   &   &   &   \\\hline
        
    }{Tabla de comparación de componentes (plugins) en Moodle}


\clearpage
\subsection{Elección de la plataforma}
 
 Para poder determinar cual es la plataforma de aprendizaje sobre la cual se trabaja durante
 el desarrollo de este trabajo terminal se realizó una tabla comparativa de las plataformas
 anteriormente mencionadas, los criterios considerados para la elección son los siguientes:
 
 \begin{quote}
    \begin{description}
    \item[] 
    \end{description}
 \end{quote}


    \addtable{| m{13em} |m{1.5cm}|m{1.5cm}|m{1.5cm}|m{1.5cm}|m{1.5cm}|m{1.8cm}|}{tbl:LMSs}{
        {\bf Carácterísticas} & 
        \begin{center}{\bf Moodle}             \cite{PagMoodle}    \end{center} &
        \begin{center}{\bf ATutor}             \cite{PagATutor}    \end{center} &
        \begin{center}{\bf Docebo}             \cite{PagDocebo}    \end{center} & 
        \begin{center}{\bf SAP\newline Litmos} \cite{PagSAPLitmos} \end{center}&
        \begin{center}{\bf Gnosis Connect}     \cite{PagGnosisConnect} \end{center}&
        \begin{center}{\bf TalentLMS}          \cite{PagTalentLMS}\end{center}\\\hline

        % Características       % Moodle  ATutor  Docebo   SAPLit   Gnosis   Talent
        Documentación de código & Sí      & Sí    & Sí     & Sí     & No     & No     \\\hline
        Idioma Español o Inglés & Sí      & Sí    & Sí     & Sí     & Sí     & Sí     \\\hline
        Tipo de Licencia        & GPLv3   & GPL   & Propia & Propia & Propia & Propia \\\hline
        Extensible              & Sí      & Sí    & Sí     & Sí     & No     & No     \\\hline
       %Redistribuye su código fuente       & Sí & Sí & No & No & No & No \\\hline
    }{Comparación de las plataformas de aprendizaje}
    
    \noindent A continuación se presenta el por qué de cada característica:
    \begin{itemize}
        \item Comparte su documentación de código: Debido a que queremos adaptarnos a sistemas existentes, necesitamos entender cómo funciona su estructura, cómo tienen separados sus archivos y cómo se usarían sus funciones. Sin la necesidad de leer todo el código desde el inicio. 
        \item Soporta idioma Español o Inglés: No solo se requiere que la documentación exista, sino que sea entendible para los desarrolladores de este trabajo terminal, y los idiomas que los desarrolladores manejan son inglés y español.
        \item Licencia: Saber qué sistema gestor de aprendizaje pudiera llegar a permitirnos el conocer su código y trabajar con él.
        \item Redistribuye su código fuente: Las licencias GPL tienen como palabra clave la distribución del código binario, es por eso que productos que son servicios de software pueden tener dicha licencia y no distribuir su código fuente.
        \item Permite la incorporación de componentes desarrollados por gente externa: No se quiere modificar directamente el código fuente del sistema, se quiere poder extender sus funcionalidades.
    \end{itemize}
       
 %   Sistema de gestión de aprendizaje es la traducción de Learning Managment System (LMS)
 %   del Inglés. A continuación se enlistan las definiciones de distintos autores acerca
 %   de los sistemas gestores de aprendizaje:
 
 %   Un sistema de gestión de aprendizaje es:
 %
 %      - Un software que incluye  una lista de servicios que le permiten y
 %        ayudan al profesor con la gestión de sus cursos. \cite{LMS_1}
 %
 %      - Una aplicación de software basada en web diseñada para manejar
 %        material didáctico, interacción con el estudiante, herramientas
 %        de evaluación y reportes del progreso de aprendizaje de los estudiantes.
 %        \cite{LMS_2}
 %
 %      - Un software para el manejo y presentación de materiales didácticos en
 %        la internet, además de ofrecer funcionalidades para la colaboración
 %        en línea. \cite{LMS_3}

  
 % \noindent Existen varios sistemas gestores de aprendizaje disponibles para su uso
 % actualmente. Se realizó la siguiente tabla comparativa para poder determinar que
 % sistema gestor de aprendizaje se utilizará.
    
 % Todos los LMS investigados:
 %      Moodle*
 %      Claroline
 %      Docebo*
 %      SAP Litmos*
 %      Gnosis Conect*
 %      TalentLMS*
 %      eFront
 %      Sakai
 %      Edmodo
 %      ATutor*

\begin{comment}
    \noindent Las fuentes para cada columna de la tabla se obtuvieron de las respectivas páginas de los sistemas gestores de aprendizaje. Dichas páginas se pueden encontrar en:
    \begin{itemize}
        \item Moodle: {https://moodle.com/}
        \item ATutor: {https://atutor.github.io/}
        \item Docebo: {https://www.docebo.com/es/}
        \item Litmos: {https://www.litmos.com/es-LA/}
        \item Gnosis Conect: {https://www.gnosisconnect.com/}
        \item TalentLMS: {https://www.talentlms.com/}
    \end{itemize}
\end{comment}   
    
    %Con lo anterior se entiende que Moodle y ATutor cumplen con nuestras necesidades, sin embargo, por preferencia de nuestros directores se decidió que Moodle será el que se usará para este trabajo terminal.
    \noindent Con el cuadro  \ref{tbl:LMSs} se entiende que Moodle y ATutor son las mejores opciones , sin embargo, se decidió utilizar Moodle al final debido a los siguientes motivos:
    
    \begin{itemize}
        \item Moodle está siendo actualmente utilizado por Celex ESCOM, lo cual abre la oportunidad de solicitar soporte a un administrador de Moodle con experiencia valiosa.
        \item ATutor especifica que utiliza la licencia GPL, sin embargo, no especifica ninguna versión. Lo cual provoca no saber a ciencia cierta a que versión se acata.
        \item ATutor cuenta con enlaces rotos asociados a su licencia, lo cual refleja poca importancia en su documentación.
    \end{itemize}
\clearpage    



   % El estado del arte se compone de 2 partes. Una que compara las soluciones con gamificación en sistemas dedicados al aprendizaje y la otra que compara las soluciones con gamificación en plugins de Moodle.
A continuación se describen cada uno de los sistemas:


Con nuestra investigación encontramos varios sistemas dedicados al aprendizaje que cuentan con gamificación. A continuación se presenta una tabla que indica cómo es que dichos sistemas cuentan con gamificación.
   
    \addtable{|c|c|c|c|c|c|c|c|c|}{table:LMS_GMFC}{& %
        
        \rotatebox[origin=c]{\aux}{Duolingo   \cite{PagDuolingo}}  &
        \rotatebox[origin=c]{\aux}{Moodle     \cite{PagMoodle}}    &
        \rotatebox[origin=c]{\aux}{Docebo     \cite{PagDocebo}}    &
        \rotatebox[origin=c]{\aux}{SAP Litmos \cite{PagSAPLitmos}} &
        \rotatebox[origin=c]{\aux}{ATutor     \cite{PagATutor}}    & 
        \rotatebox[origin=c]{\aux}{ALEKS      \cite{PagALEKS}}     &
        \rotatebox[origin=c]{\aux}{Udemy      \cite{PagUdemy}}     &
        \rotatebox[origin=c]{\aux}{TalentLMS  \cite{PagTalentLMS}} \\\hline 
        
        Propia    & X & X &   & X &   & X & X & X \\\hline
        Extendida &   & X & X & X & X &   &   &   \\\hline
        
    }{Implementación de gamificación}
       
\noindent Al decir \textbf{Propia} en el cuadro \ref{table:LMS_GMFC} nos referimos a que el sistema gestor de aprendizaje ya tiene integrado en su funcionalidad la implementación  de gamificación. Y al decir \textbf{Extendida} nos referimos a que existen componentes externos (plugins) que implementan gamificación.\\
    
\noindent A continuación se describen los sistemas en el cuadro \ref{table:LMS_GMFC} y los elementos de gamificación con los que cuentan, de acuerdo con los marcos de referencia \nameref{sec:ForTheWin} y \nameref{sec:octalysis}.
\clearpage
