\subsection{Octalysis}
\label{sec:octalysis}

 {\it Octalysis} es un marco de trabajo realizado por Yu-Kai Chou, el marco de trabajo
 se centra en las diferentes formas en que una persona puede ser motivada para realizar
 una actividad en específico, dichas formas son presentadas en un octágono (ver figura
 \ref{fig:octalysis}) como principios de gamificación.\\
 
 % Gracias a la investigación que realizó Yu-Kai Chou durante 10 años, se dio cuenta que existen
 % 8 ejes que motivan a la gente a realizar cualquier actividad (Figura \ref{fig:octalysis})
 % De estos ejes basó su marco de trabajo $"$Octalysis$"$ para poder implementar de una manera
 % sistemática la Gamificación, y al mismo tiempo obteniendo flexibilidad en la misma.
    
    \addfigure[(adaptado de {\it Octalysis} \cite{Octalysis})]%
        {.95}{investigacion/images/octalysis}{fig:octalysis}{Principios de gamificación según Octalysis}

 \noindent La razón principal por la cual se contempla este marco de trabajo es debido a que presenta
 cómo los principios de gamificación trabajan como un conjunto, y proporciona técnicas específicas
 sobre como brindar soporte a cada uno de estos principios de gamificación. A continuación se
 describe cada uno de los principios junto con algunas técnicas que permiten darles soporte.
    
\subsubsection{1. \principioI} \label{subsec:principioI}
  
 Este principio se ve reflejado cuando las personas estan motivadas debido a que se
 sienten comprometidas en algo más allá que ellos mismos, las personas motivadas no toman acción
 no por su beneficio, sino por ``un bien mayor'' \cite[p. 66, 69]{Octalysis}. Algunas técnicas
 para implementar este principio son:
    
    \begin{itemize}
    \item
    {\bf Narrativa.}
        Brinda el contexto de porqué el jugador debe realizar las actividades dentro del juego o
        del entorno gamificado, generalmente se relata una historia para que el jugador conozca
        cuál es el motivo de su rol o personaje para realizar las actividades diseñadas.
        La narrativa puede ser desde un vídeo introductorio que explique la historia, hasta
        el desarrollo de una temática a lo largo de todo el sistema \cite[p. 81]{Octalysis}.
        
    \item
    {\bf Héroe de la humanidad.}
        Se consigue al hacerle sentir al jugador que pertenece a algo más allá de sí mismo,
        involucrándolo en actividades que conllevan a consecuencias humanitarias buenas y reales
        en el mundo, motivandolo a seguir realizando actividades \cite[p. 82]{Octalysis}.
        
    \item
    {\bf Elitismo.}
        Esta técnica requiere que los jugadores estén organizados en equipos, de tal forma que
        no solo realicen actividades para su propio beneficio, sino también para el beneficio
        de su equipo. Esta técnica requiere de especial cuidado ya que si no existe una competencia
        sana, los podrían ser negativos \cite[p. 83]{Octalysis}.
    
    \end{itemize}
    
\subsubsection{2. \principioII} \label{subsec:principioII}
 
 Este principios se muestra cuando las personas son impulsadas por un sentido de desarrollo continuo
 con el propósito de cumplir un objetivo específico \cite[p. 91]{Octalysis}. Las siguientes técnicas
 permiten brindarle soporte a este principio de gamificación.
    
    \begin{itemize}
    \item
    {\bf Barra de progreso.}
        Esta técnica se beneficia de que los jugadores siempre estan en busca de completar tareas
        que esten incompletas, las barras de progreso permiten mostrar el avance que se tiene de una
        tarea en específico y que tanta cantidad de trabajo es requerida para concluirla. Durante la
        implementación de las barras de progreso es imperativo que el jugador pueda realizar tareas
        significativas para que el esfuerzo de realizar estas se vea reflejado en la barra de progreso
        y así se transmita el sentido de crecimiento \cite[p. 113]{Octalysis}.
        
    \item
    {\bf Insignias.}
        La función de las insignias, logros, medallas y otros elementos de juego parecidos es que el
        jugador pueda mostrar a los demás que realizó una actividad importante y complicada, proporcionando
        un sentido de realización, estos símbolos pueden ser cualquier distintivo cómo: insignias,
        estrellas, sombreros, uniformes, entre otros. Lo importante es el significado y el esfuerzo
        que cada uno representa \cite[p. 117]{Octalysis}.
        
    \item
    {\bf Sistema de Puntuación.} % Sistema de Puntos
        Establece un sistema de puntuación, donde los puntos son obtenidos a través de la realización
        de actividades planeadas. El sistema de puntaje tiene finalidades internas, ayudando al sistema
        a conocer el estado de completitud de las actividades; y externas, brindando al usuario
        retroalimentación acerca del avance de una actividad u objetivo \cite[p. 118]{Octalysis}. 
        
    \item
    {\bf Tabla de líderes.}
        Es una técnica en la que se ordenan a los jugadores en una tabla con base en un criterio específico
        (por ejemplo, el número de nivel) de tal forma que los jugadores puedan subir posiciones en la
        tabla mientras van completando actividades. Al diseñar esta técnica es necesario no hacer
        sentir al usuario frustrado, cuando su posición en la lista es muy baja \cite[p. 121]{Octalysis}.
    \end{itemize}
  

\subsubsection{3. \principioIII} \label{subsec:prinpcioIII}
    
 El principio trata acerca de impulsar la creatividad en las personas, incentivando la toma de
 decisiones, y ayudando al usuario a motivarse por medio del pensamiento creativo \cite[p. 126]{Octalysis}.
 Las siguientes técnicas permiten brindar soporte a este principio.
    
    \begin{itemize}
    \item
    {\bf Amplificadores.}
        Los amplificadores permiten darle una ventaja a los usuarios durante un tiempo limitado,
        lo cual motiva a los usuarios a usar esta ventaja lo más que pueda durante el lapso de tiempo
        en que está activado el amplificador, un ejemplo son las ofertas relampago de la plataforma
        de compras en línea ``Amazon'' \cite[p. 146]{Octalysis}.

        
    \item
    {\bf Percepción de libre albedrío.}
        La percepción de libre albedrío se le brinda al usuario cuando se da a escojer entre distintas
        opciones, lo que le hace sentir que su opinión, experiencia y decisiones son tomadas en cuenta.
        Se le llama percepción de libre albedrío porque, a pesar de que se pueden mostrar varias opciones,
        se guía al usuario a elegir la opción apropiada por medio de incentivos \cite[p. 150]{Octalysis}.
    \end{itemize}
    
\subsubsection{4. \principioIV} \label{subsec:principioIV}
    
 Este principio representa la motivación impulsada por nuestros sentimientos de poseer algo y en
 en consecuencia al conseguirlo tener el deseo de mejorarlo y protegerlo. % Este principio
 % involucra muchos elementos de gamificación, como bienes y dinero virtuales, ambos collecionables % acumulables.
 Este principio está asociado con la perzonalización y el cuidado de aquello que le pertenece al usuario
 \cite[p. 161]{Octalysis}. A continuación se mencionan algunas técnicas que le brindan soporte este
 principio:
    
    \begin{itemize}
    \item
    {\bf Construcción desde cero.}
        Este ejemplo trata acerca de como el usuario siente pertenencia cuando construye
        un objeto desde el inicio, puesto que lo personaliza a su gusto. Es importante que
        el proceso de creación no sea tedioso para evitar el efecto contrario \cite[p. 182]{Octalysis}.
        
    \item
    {\bf Colleccionables.}
        Si se le brindan a los usuarios algunos elementos, personajes o insignias que
        formen parte de una collección específica, ellos intentarán buscar los demás
        elementos faltántes con el propósito de completar la colleción \cite[p. 183]{Octalysis}.
        
    \item
    {\bf Puntos cangeables}
        Los puntos cangeables sirven para obtener bienes en el sistema, los jugadores
        normalmente acumulan los puntos hasta poder cambiarlos por el objeto que deseén.
        La manera en que se obtienen estos puntos es esencial para elegir en que se
        actividades se quieren enfatizar que el usuario realice \cite[p. 187]{Octalysis}.
        
    \item
    {\bf Monitor de accesorios.}
        Es una técnica que le permite a las personas tener un mayor sentido de pertenencia
        hacia algún elemento a través del monitoreo y cuidado constante, al desarrollar un
        mayor sentido de pertenencia las personas buscarán mejorar/desarrollar dicho elemento
        \cite[p. 189]{Octalysis}.

    \end{itemize}
    
\subsubsection{5. \principioV} \label{subsec:principioV}
    
 Este principio incorpora todos los elementos sociales que motivan a la gente, incluyendo
 las tutorias, aceptación social, compañerismo e incluso las competencias; se basa en el
 deseo de conectar y compararnos con otros \cite[pp. 27, 197]{Octalysis}. Algunas técnicas
 vinculadas a este principio son:
    
    \begin{itemize}
        
    \item
    {\bf Tutorías.}
        Las tutorías son una técnica poderosa para mantener motivado a los usuarios
        puesto que les da una experiencia personalizada con el sistema a través de su
        tutor, y les ayuda a superar los obstáculos más comunes que se presenten
        \cite[p. 215]{Octalysis}.
       
    \item
    {\bf Mostrador de trofeos.}
        Este técnica permite a los usuarios mostrar los logros que han obtenido
        por ejemplo, en algunos juegos ciertos avatares o aditamentos obtenidos con
        gran dificultad, se muestran en el perfil del jugador o en su personaje
        \cite[p. 218]{Octalysis}.
 
    \item
    {\bf Actividades grupales.}
        Las actividades en grupo son muy efectivas en incentivar la colaboración
        porque requiere la participación de todos los miembros de un grupo para
        lograr una meta específica. Por ejemplo, cuando una tienda de ropa ofrece
        una promoción aplicable solo a grupos de tres a cinco personas
        \cite[p. 221]{Octalysis}.
        
    \item
    {\bf Rango de aceptación.} % Confirmity Anchor
        El rango de aceptación consiste en motivar a los usuarios mostrandoles
        la diferencias entre sus puntaje, los puntajes de los demás usuarios,
        y el puntaje a alcanzar o métrica objetivo; los usuarios naturalmente
        intentarán sobre pasar el promedio y acercarse a la métrica objetivo
        \cite[p. 226]{Octalysis}.
        
    \item
    {\bf Competencias.} 
        Las competencias son una manera de motivar a los usuarios a ser mejores
        que sus iguales, al comparar constantemente sus habilidades. También ayuda
        a mantener un historial sobre el progreso de los usuarios a lo largo de las
        actividades \cite[p. 210]{Octalysis}.
        
        % Las competencias deben ser diseñadas basadas en un análisis profundo, ya
        % que si es implementarlas en un entorno incorrecto produciría rivalidad o
        % resultados negativos. Por otro lado, en los sistemas de aprendizaje se
        % propone aplicar competencias, ya que se motiva a los usuarios a aprender
        % más y ser mejores que sus compañeros \cite{CompetenciasHelger}.
        
    \end{itemize}
    
\subsubsection{6. \principioVI} \label{subsec:principioVI}

 Este principio esta relacionado con la motivación que se presenta cuando queremos algo
 que no podemos tener de forma inmediata  o porque existe una gran dificultad para obtenerlo
 \cite[p. 233]{Octalysis}. Algunos herramientas que brindan soporte a este principio son:
        
    \begin{itemize}
    \item
    {\bf Siempre visible} % Colgado o mostrar los objetos
        Al mostrarle a los usuarios reiteradamente los objetos que no pueden obtener o que
        son muy difíciles de obtener, comenzarán a desearlos con mayor fuerza. Por ejemplo
        cuando el juguete dificil de obtener es el que viene presente en la caja del cereal
        \cite[p. 252]{Octalysis}.
            
    \item
    {\bf Acciones limitadas.} % Tapas magnéticas
        Esta herramienta consiste en establcer un límite de veces que se puede ejecutar una
        acción especifica, con el motivo de brondar un sentido de abundancia temporal, para
        que después de un tiempo la necesidad o el deséo de ejecutar dicha acción vuelva
        \cite[p. 256]{Octalysis}.
            
    \item
    {\bf Horario establecido} %Dinámica de citas.
        Esta herramienta consiste en establecer un tiempo para la realización de alguna tarea.
        Cuando los usuarios tienen un horario establecido para un tarea planean para acomodar
        sus tiempos, a la par que están más atentos conforme se acercan a la hora estanlecida
        \cite[p. 258]{Octalysis}.
            
    \item
    {\bf Tiempo limitado.} % Descansos de tortura
        Esta herramienta limita al usuario a realizar ciertas acciones durante un tiempo
        específico, cuando el tiempo acabe el usuario tendrá que esperar para volver a
        utilizarlo. Esto hace que el usuario busque algún método para minimar o concluir
        el periodo de espera, los metodos suelen ser invertir dinero o realizar una acción
        a cambio \cite[p. 261]{Octalysis}.
    \end{itemize}

\subsubsection{7. \principioVII}
\label{subsec:principioVII}

 Este principio consiste en mantener atentos a los usuarios debido a que no saben que es lo que
 ocurrirá despues, lo que los motiva y los mantiene a la espera de los nuevps sucesos que puedan
 ocurrir. \cite[pp. 27, 273]{Octalysis}. Algunas herramientas para brindar soporte a este
 principio son:
       
    \begin{itemize}
    \item
    {\bf Elemento sobresaliente} % Elección que brilla
        Este tipo de implementación aborda la curiosidad del usuario al mostrarle una opción
        que se encuentra resaltada en el sistema, lo cual hace que el usuario quiera descubrir
        por qué es que se encuentra brillando y así se puede llegar a guiar al usuario hacia
        acciones determinadas \cite[p. 297]{Octalysis}.
           
    \item
    {\bf Cajas sorpresa.}
        Esta herramienta consiste en recompensar ciertas acciones mediante cajas sorpresa
        las cuales otorgan un elemento de forma aleatorea, los usuarios al darse cuenta que
        la caja puede obtener elementos distintos, buscarán descubrir cuantos elementos diferentes
        puede otorgar \cite[p. 299]{Octalysis}.
           
    \item
    {\bf Elementos inesperados.} % Huevos de Pascua
        Son elementos que aparecen en ubicaciones inesperadas o que aparecen de forma inesperada
        al ejecutarse una acción determinada. A diferencia de las cajas sorpresa, los elementos
        aparecen al ejecutarse acciones que no se sabía que darían el resultado mostrado
        \cite[p. 301]{Octalysis}.
           
    \item
    {\bf Lotería.}
        Este tipo de implementación también está enfocada en las recompensas, sin embargo se les
        recompensa solo a ciertos jugadores ganadores, La probabilidad de obtener la recompensa
        aumenta al mantenerse más tiempo en el sistema, lo cual motiva a los usuarios
        a estar en él y seguir buscando las recompensas \cite[p. 305]{Octalysis}.
    \end{itemize}

\clearpage
\subsubsection{8. \principioVIII} \label{subsec:principioVIII}

 Este principio está relacionado con la motivación presente cuando a las personas realizan acciones
 con el objetivo de evitar perder algo o evitar la ocurrencia de eventos no deseables. Por naturaleza
 los usuarios evitarán en lo mayor posible perder la inversión de tiempo, dinero y esfuerzo
 \cite[p. 311]{Octalysis}. Algunas herramientas relacionadas a este principio son:
        
    \begin{itemize}
    \item
    {\bf Recompensa condicionada.}
        Esto es cuando un sistema primero hace creer al usuario que algo le pertenece de
        manera legítima, y posteriormente los condiciona haciendoles sentir que se los
        van a quitar si no realizan una acción deseada.
        \cite[p. 330]{Octalysis}
            
    \item
    {\bf Oportunidad desvaneciente.}
        Una oportunidad desvaneciente es una oportunidad que desaparecerá si el usuario no
        realiza una acción de forma inmediata. Un ejemplo real son las ofertas limitadas
        que hacen a los clientes decidir si aprovechan la oferta comprando o si la dejan
        pasar la oferta única \cite[p. 333]{Octalysis}.
            
    \item
    {\bf Creación de hábitos.}
        Esta herramienta consiste en generar un estado de uso constante en la etapa final
        de un producto haciendo ciclos de actividades atractivas las cuales permiten que
        los usuarios conviertan en hábitos las actividades específicas.
        \cite[p. 334]{Octalysis}.
            
    \item
    {\bf Pérdida de renunciar a todo.}
        El uso de esta herramienta se ve reflejado cuando los usuarios invierten una gran
        cantidad de tiempo y esfuerzo en algo que aparentemente no brinda frutos y sin
        embargo continuan realizando acciones específicas para que el esfuerzo y tiempo
        no hayan sido en vano \cite[p. 338]{Octalysis}.

    \end{itemize}
    
\clearpage

