\section{Marcos de trabajo para la Gamificación}

 Como se comentó en el capítulo \hyperrefx{ch:introduccion}, el crear una experiencia
 gamificada exitosa no solo consiste en aplicar mecánicas de juegos a una actividad
 específica, tambien requiere del seguimiento de un marco de instrucción apropiado \cite[p. 1110]{GamInE-Learning}.
 Se ha decidido utilizar dos marcos de trabajo como guía en el diseño e implementación
 de los componentes que se desarrollarán. Los marcos de trabajo elegidos son {\it Octalysis}
 \cite{Octalysis} y {\it For The Win} \cite{FrameWorkForTheWin}.

\subsection{Octalysis}
\label{sec:octalysis}

 {\it Octalysis} es un marco de trabajo realizado por Yu-Kai Chou una decada, el marco de
 trabajo se centra en las diferentes formas en que una persona puede ser motivada para
 realizar una actividad en específico, dichas formas son presentadas en un octágono
 (ver figura \ref{fig:octalysis}) como principios de gamificación.\\
 
 % Gracias a la investigación que realizó Yu-Kai Chou durante 10 años, se dio cuenta que existen
 % 8 ejes que motivan a la gente a realizar cualquier actividad (Figura \ref{fig:octalysis})
 % De estos ejes basó su marco de trabajo $"$Octalysis$"$ para poder implementar de una manera
 % sistemática la Gamificación, y al mismo tiempo obteniendo flexibilidad en la misma.
    
    \addfigure[(adaptado de {\it Octalysis} \cite{Octalysis})]%
        {.95}{investigacion/images/octalysis}{fig:octalysis}{Principios de gamificación según Octalysis}

 \noindent La razón principal por la cual se contempla este marco de trabajo es debido a que presenta
 cómo los principios de gamificación trabajan como un conjunto, y proporciona técnicas específicas
 sobre como brindar soporte a cada uno de estos principios de gamificación. A continuación se
 describe cada uno de los principios junto con algunas técnicas que permiten darles soporte.
    
\subsubsection{1. \principioI} \label{subsec:principioI}
  
 Este principio se ve reflejado cuando las personas estan motivadas debido a que se
 sienten comprometidas en algo más allá que ellos mismos, las personas motivadas no toman acción
 no por su beneficio, sino por ``un bien mayor'' \cite[p. 66, 69]{Octalysis}. Algunas técnicas
 para implementar este principio son:
    
    \begin{itemize}
    \item
    {\bf Narrativa.}
        Brinda el contexto de porqué el jugador debe realizar las actividades dentro del juego o
        del entorno gamificado, generalmente se relata una historia para que el jugador conozca
        cuál es el motivo de su rol o personaje para realizar las actividades diseñadas.
        La narrativa puede ser desde un vídeo introductorio que explique la historia, hasta
        el desarrollo de una temática a lo largo de todo el sistema \cite[p. 81]{Octalysis}.
        
    \item
    {\bf Héroe de la humanidad.}
        Se consigue al hacerle sentir al jugador que pertenece a algo más allá de sí mismo,
        involucrándolo en actividades que conllevan a consecuencias humanitarias buenas y reales
        en el mundo, motivandolo a seguir realizando actividades \cite[p. 82]{Octalysis}.
        
    \item
    {\bf Elitismo.}
        Esta técnica requiere que los jugadores estén organizados en equipos, de tal forma que
        no solo realicen actividades para su propio beneficio, sino también para el beneficio
        de su equipo. Esta técnica requiere de especial cuidado ya que si no existe una competencia
        sana, los podrían ser negativos \cite[p. 83]{Octalysis}.
    
    \end{itemize}
    
\subsubsection{2. \principioII} \label{subsec:principioII}
 
 Este principios se muestra cuando las personas son impulsadas por un sentido de desarrollo continuo
 con el propósito de cumplir un objetivo específico \cite[p. 91]{Octalysis}. Las siguientes técnicas
 permiten brindarle soporte a este principio de gamificación.
    
    \begin{itemize}
    \item
    {\bf Barra de progreso.}
        Esta técnica se beneficia de que los jugadores siempre estan en busca de completar tareas
        que esten incompletas, las barras de progreso permiten mostrar el avance que se tiene de una
        tarea en específico y que tanta cantidad de trabajo es requerida para concluirla. Durante la
        implementación de las barras de progreso es imperativo que el jugador pueda realizar tareas
        significativas para que el esfuerzo de realizar estas se vea reflejado en la barra de progreso
        y así se transmita el sentido de crecimiento \cite[p. 113]{Octalysis}.
        
    \item
    {\bf Insignias.}
        La función de las insignias, logros, medallas y otros elementos de juego parecidos es que el
        jugador pueda mostrar a los demás que realizó una actividad importante y complicada, proporcionando
        un sentido de realización, estos símbolos pueden ser cualquier distintivo cómo: insignias,
        estrellas, sombreros, uniformes, entre otros. Lo importante es el significado y el esfuerzo
        que cada uno representa \cite[p. 117]{Octalysis}.
        
    \item
    {\bf Sistema de Puntuación.} % Sistema de Puntos
        Establece un sistema de puntuación, donde los puntos son obtenidos a través de la realización
        de actividades planeadas. El sistema de puntaje tiene finalidades internas, ayudando al sistema
        a conocer el estado de completitud de las actividades; y externas, brindando al usuario
        retroalimentación acerca del avance de una actividad u objetivo \cite[p. 118]{Octalysis}. 
        
    \item
    {\bf Tabla de líderes.}
        Es una técnica en la que se ordenan a los jugadores en una tabla con base en un criterio específico
        (por ejemplo, el número de nivel) de tal forma que los jugadores puedan subir posiciones en la
        tabla mientras van completando actividades. Al diseñar esta técnica es necesario no hacer
        sentir al usuario frustrado, cuando su posición en la lista es muy baja \cite[p. 121]{Octalysis}.
    \end{itemize}
  

\subsubsection{3. \principioIII} \label{subsec:prinpcioIII}
    
 El principio trata acerca de impulsar la creatividad en las personas, incentivando la toma de
 decisiones, y ayudando al usuario a motivarse por medio del pensamiento creativo \cite[p. 126]{Octalysis}.
 Las siguientes técnicas permiten brindar soporte a este principio.
    
    \begin{itemize}
    \item
    {\bf Amplificadores.}
        Los amplificadores permiten darle una ventaja a los usuarios durante un tiempo limitado,
        lo cual motiva a los usuarios a usar esta ventaja lo más que pueda durante el lapso de tiempo
        en que está activado el amplificador, un ejemplo son las ofertas relampago de la plataforma
        de compras en línea ``Amazon'' \cite[p. 146]{Octalysis}.

        
    \item
    {\bf Percepción de libre albedrío.}
        La percepción de libre albedrío se le brinda al usuario cuando se da a escojer entre distintas
        opciones, lo que le hace sentir que su opinión, experiencia y decisiones son tomadas en cuenta.
        Se le llama percepción de libre albedrío porque, a pesar de que se pueden mostrar varias opciones,
        se guía al usuario a elegir la opción apropiada por medio de incentivos \cite[p. 150]{Octalysis}.
    \end{itemize}
    
\subsubsection{4. \principioIV} \label{subsec:principioIV}
    
 Este principio representa la motivación impulsada por nuestros sentimientos de poseer {\em algo} y en
 consecuencia se genera un deseo de mejorarlo, protegerlo y obtener cosas para ese {\em algo}. % Este principio
 % involucra muchos elementos de gamificación, como bienes y dinero virtuales, ambos collecionables % acumulables.
 Este principio está asociado con la perzonalización y el cuidado de aquello que le pertenece al usuario
 \cite[p. 161]{Octalysis}. A continuación se mencionan algunas técnicas que le brindan soporte este
 principio:
    
    \begin{itemize}
    \item
    {\bf Construcción desde cero.}
        Este ejemplo trata acerca de como el usuario siente pertenencia cuando crea
        un objeto desde cero puesto que lo realizó a su gusto. Es importante que el
        proceso de creación no sea tedioso para evitar el efecto contrario
        \cite[p. 182]{Octalysis}.
        
    \item
    {\bf Conjunto de colección.}
        Al darle elementos de personalización a los usuarios como imágenes de perfil,
        o darles logros al volverlos parte de un conjunto de colección los motiva a
        conseguir todos los elementos para obtener todo el conjunto
        \cite[p. 183]{Octalysis}.
        
    \item
    {\bf Puntos intercambiables}
        Estos puntos a diferencia de los puntos de estatus, sirven para obtener bienes
        en el sistema, por lo cual al obtener algún bien, nos motiva a protegerlo y
        mejorarlo. La manera en que se obtienen estos puntos es esencial para elegir
        en que se actividades se quieren enfatizar que el usuario realice
        \cite[p. 187]{Octalysis}.
        
    \item
    {\bf Apego menor.}
        Al estar monitoreando datos o valores constantemente a lo largo de cierto tiempo,
        estos datos empiezan a importarnos y se llega a la motivación de mejorarlos,
        por lo cual se deben de enseñar constantemente los puntos de progreso de los
        usuarios para que se sientan motivados al aumentarlos \cite[p. 189]{Octalysis}.

    \end{itemize}
    
\subsubsection{\principioV} \label{subsec:principioV}
    
 Involucra actividades inspiradas por lo que otras personas piensan, hacen o dicen. Este
 principio es el motor detrás de varios temas como tutorías, competiciones, envidia,
 actividades grupales, tesoro social y compañerismo. Se basa en el deseo de conectar y
 compararse con otros individuos.\cite[p. 197]{Octalysis} Algunos ejemplos son:
    
    \begin{enumerate}
        
    \item
    {\bf Tutorías.} \cite[p. 215]{Pctalysis}
        Las tutorías son una herramienta poderosa para mantener motivado a los usuarios
        puesto que les da una experiencia personalizada con el sistema por medio de su
        tutor, y les ayuda a sobrepasar obstáculos que son comunes en ese entorno.
        
    \item
    {\bf Estante de trofeos.} \cite[p. 218]{Octalysis}
        El estante de trofeos permite al usuario mostrar que cosas ha logrado, lo cual
        se exhibe por si mismo, los estantes de trofeos son vistos cuando se entra en
        la oficina de alguien y en las paredes se ven los premios, certificados y
        credenciales que ha conseguido. En los juegos se puede ver como coronas, logros,
        o avatares. En muchos juegos los elementos y equipo de los avatares solo pueden
        ser conseguidos después de llegar a una dificultad muy grande. Esto permite que
        todos puedan ver que ese usuario ha logrado muchas cosas. 
        
    \item
    {\bf Actividad en grupo.} \cite[p. 221]{Octalysis}
        Las actividades en grupo son muy efectivas en la colaboración así como en propaganda
        viral, porque requiere la participación grupal antes que algún individuo consiga
        el estado ganador. Un ejemplo claro es el de ofrecer descuento en algún producto
        de una tienda, pero ese descuento solo es aplicable si la tienda vende una cierta
        cantidad de el producto en concreto, esto hace que las personas inviten a sus
        conocidos a comprar el producto para poder conseguir el descuento, lo cual logra
        publicidad gratis.
        
    \item
    {\bf Ancla de conformidad.} \cite[p. 226]{Octalysis}
        La ancla de conformidad, habla de motivar a los usuarios al mostrarles las diferencias
        entre sus puntajes, comportamiento o progreso, respecto a los demás que se encuentran
        en el sistema. Esto los hace querer ser parte de la norma, o hasta sobresalir del
        grupo al realizar constantemente actividades mejores y más difíciles.
        
    \item
    {\bf Competencias.} \cite[p. 210]{Octalysis}
        Las competencias son una manera de motivar a los usuarios a ser mejores que sus iguales,
        al estar constantemente comparando sus habilidades con la de los otros. También ayuda a
        mantener un historial sobre el progreso de los usuarios a lo largo de las actividades.
        Es importante recalcar que según el autor Mario Herger \cite{libro25} las competencias
        se deben implementar en casos especiales, de lo contrario obtendrían un resultado negativo
        al esperado, uno de los puntos que propone en el que se debe de aplicar la competencia es
        cuando el sistema es de aprendizaje, puesto que la competencia motiva a los usuarios a
        querer aprender más y ser mejores que sus compañeros.
        
    \end{enumerate}
    
\subsubsection{\principioVI} \label{subsec:principioVI}

 Es el principio que motiva debido a que no podemos tener inmediatamente algún objeto o porque
 existe una gran dificultad para obtenerlo\cite[p. 233]{Octalysis}. Algunos ejemplos de como
 aplicar el principio son:
        
    \begin{enumerate}
    \item \textbf{Colgado o mostrar los objetos.}\cite[p. 252]{Octalysis}
        Al mostrarle a los usuarios los objetos que no pueden obtener o que son muy difíciles
        de obtener, los hace que los deseen con más fuerza. Por ejemplo cuando se muestran en
        una tienda o como objetos bloqueados, esto motiva a los usuarios a querer conseguirlos.
            
    \item \textbf{Tapas magnéticas.}\cite[p. 256]{Octalysis}
        Son limitaciones que se le ponen al número de veces que un usuario puede realizar alguna
        acción, que a su vez lo motiva a querer realizar las acciones más veces. Se habla de que
        nunca se debe de dar al usuario un sentido de abundancia infinita, porque eso hace que
        no se le den importancia a las acciones a realizar.
            
    \item \textbf{Dinámica de citas.}\cite[p. 258]{Octalysis}
        Este ejemplo trata de implementar escasez en el tiempo, al sólo poder realizar ciertas
        acciones en una determinada hora del día, esto motiva al usuario y hace que se esté más
        atento para no perderse el momento del día y poder realizar la acción deseada.
            
    \item \textbf{Descansos de tortura.}\cite[p. 261]{Octalysis}
        Se trata de limitar al usuario a utilizar el sistema solo por cierto tiempo y que tenga
        que esperar para poder volver a utilizarlo. Esto hace que el usuario busque cualquier
        método necesario para terminar el tiempo de espera, esos métodos pueden ser el pagar
        dinero o realizar alguna acción deseada por los dueños del sistema.
    \end{enumerate}

\subsubsection{\principioVII}
\label{subsec:principioVII}

 Se motiva y se mantienen enganchados a los usuarios al no permitirles adivinar cual va a ser
 el siguiente suceso que ocurrirá, esto los hace curiosos y los mantiene atentos a los resultados
 de sus acciones deseadas en el sistema \cite[p. 273]{Octalysis}. Ejemplos de su implementación:
       
    \begin{enumerate}
    \item \textbf{Elección que brilla.}\cite[p. 297]{Octalysis}
        Este tipo de implementación aborda la curiosidad del usuario al mostrarle una opción
        que se encuentra resaltada en el sistema, lo cual hace que el usuario quiera descubrir
        por qué es que se encuentra brillando y así se puede llegar a guiar al usuario hacia
        ciertas acciones deseadas.
           
    \item \textbf{Cajas misteriosas o Cajas de botín.}\cite[p. 299]{IOctalysis}
        Una de las maneras en que se puede implementar este principio es por medio de recompensas
        al realizar ciertas acciones deseadas, pero estas recompensas deben de ser aleatorias
        para mantener interesado al usuario en la posibilidad de recibir cierta recompensa
        que ellos desean.
           
    \item \textbf{Huevos de pascua.}\cite[p. 301]{Octalysis}
        A diferencia de las cajas de botín o cajas misteriosas, las recompensas de tipo huevos
        de pascua no son obtenidas por realizar una acción deseada que el usuario conozca,
        si no que se dan inesperadamente a los usuarios. Esto las hace tener cierto grado
        de sorpresa al ser recibidas.
           
    \item \textbf{Lotería.}\cite[p. 305]{Octalysis}
        Este tipo de implementación también está enfocada en las recompensas, pero en especifico
        se recompensa solo a ciertos usuarios ganadores. Pero esta probabilidad de ganar la
        recompensa aumenta al mantenerse más tiempo en el sistema, lo cual motiva a los usuarios
        a estar en el y seguir obteniendo las recompensas.
    \end{enumerate}

\subsubsection{\principioVIII} \label{subsec:principioVIII}

 Este principio motiva a través del miedo de perder algún objeto o que ocurran eventos indeseables.
 Existen muchas situaciones en las cuales se actúan basados en el miedo de perder algo que representa
 nuestra inversión de tiempo, esfuerzo, dinero o otros recursos \cite{Octalysis}. Ejemplos de su
 implementación:
        
    \begin{enumerate}
    \item \textbf{Herencia legítima.}\cite{Octalysis}
        Esto es cuando un sistema primero hace creer al usuario que algo pertenece a ellos de manera
        legítima, y luego los hace sentir que se los van a quitar si no realizan una acción deseada.
            
    \item \textbf{Oportunidades evanescentes.}\cite{Octalysis}
        Una oportunidad evanescente es una oportunidad que va a desaparecer si el usuario no
        realiza una acción deseada. Un ejemplo real es las ofertas limitadas que te fuerzan
        a decidir si comprar un articulo en ese momento o perder la oferta.
            
    \item \textbf{Estancamiento del status quo.}\cite{Octalysis}
        Este tipo de implementación se realiza al tener que hacer acciones deseadas para
        mantener el status quo que ellos tienen. Esto vuelve en habito el realizar estas
        acciones y se motivan para no perder su status.
            
    \item \textbf{La prisión de costo hundido.}\cite{Octalysis}
        Esto ocurre cuando se invierte tanto tiempo en algo, que aún cuando ya no es disfrutable,
        se continua realizado las acciones deseables e invirtiendo más tiempo porque no se quiere
        sentir la perdida de todo el tiempo invertido.
            
    \end{enumerate}
    
\clearpage


\subsection{For The Win} \label{sec:ForTheWin}

 Dan Hunter y Kevin Werbach crearon un marco de trabajo que se centra en aplicar la gamificación en
 los negocios y empresas. Esto siguiendo 6 pasos y conociendo los elementos de juego. El marco de
 trabajo no tiene un nombre por si mismo, sino que el nombre se lo asignamos utilizando el título
 de su libro ``{\it For The Win:  How game thinking can revolutionize your business}''.
    
\subsubsection{Elementos de juego}
    
    \noindent De acuerdo con For The Win, para implementar gamificación se necesitan contemplar los tres tipos de elementos de juego, Dinámicas, Mecánicas y Componentes. Los tipos de elementos son organizados en una pirámide (figura 
    \ref{fig:FTW_Piramide}) de acuerdo con su nivel de abstracción 
    y el objetivo que tienen \cite[pp. 55-57]{FrameWorkForTheWin}.
    
    \addfigure[(adaptado de {\it For The Win} \cite{FrameWorkForTheWin})]%
        {.35}{investigacion/images/ForTheWin_Piramide}{fig:FTW_Piramide}{Niveles de clasificación de elementos de juego según For The Win}
    
    \begin{multicols}{2}
        \noindent\textbf{Nivel: Dinámicas}. Las dinámicas son lo más abstracto, es la temática que envuelve a todo el sistema. Existen 5 dinámicas, las cuales son:
        
        \begin{enumerate}
            \item Restricciones
            \item Emociones
            \item Historia
            \item Progresión
            \item Relaciones sociales
        \end{enumerate}

\vfill\null
\columnbreak

        \noindent\textbf{Nivel: Mecánicas}. Las mecánicas son el motivo para que se haga alguna acción, son las que mantienen enganchado al jugador. Existen 10 mecánicas, las cuales son:
        
        \begin{enumerate}
            \item Desafíos
            \item Suerte 
            \item Competencia
            \item Cooperación
            \item Retroalimentación 
            \item Obtención de elementos
            \item Recompensas
            \item Transacciones
            \item Turnos
            \item Ganadores y perdedores\\
        \end{enumerate}
        
    \end{multicols}
\clearpage

        \noindent\textbf{Nivel: Componentes} Los componentes son la forma de implementar las mecánicas y las dinámicas. Existen 15 componentes, los cuales son:
        
    \begin{multicols}{2}
        \begin{enumerate}
            \item Logros
            \item Avatares
            \item Insignias
            \item Peleas de jefes finales
            \item Colecciones
            \item Combates
            \item Desbloqueo de contenido
            \item Regalos e intercambios
            \item Tablas de líderes
            \item Niveles de personaje (Experiencia)
            \item Puntos
            \item Misiones
            \item Esquemas sociales
            \item Equipos
            \item Moneda virtual
        \end{enumerate}
    \end{multicols}
    
    
    \noindent  For The Win establece que para cumplir con gamificación no es necesario tener cada uno de los elementos anteriores, ya que establece que antes de cantidad se necesita calidad, refiriéndose a que los elementos tengan coherencia entre sí.
    
    \subsubsection{ Proceso de implementación}
    
    For the Win indica que el proceso consta de 6 pasos que especifican cómo introducir la gamififación, cada uno de los pasos se describen a continuación \cite[pp. 60-70]{FrameWorkForTheWin}.\\
    
    \noindent \textbf{1.- Definir los objetivos del negocio}\\
    
    \noindent Los objetivos no se refieren a los planteados en la visión y misión de la empresa, sino al ''¿Por qué?'' se está haciendo este sistema que tiene implementada la gamificación.\\
    
    \noindent \textbf{2.- Delimita las acciones de tus usuarios}\\
    
    \noindent Ya definido el objetivo, se tiene que ver que acciones tus usuarios podrán desarrollar en el sistema, dichas acciones deben ser concretas y específicas. Por ejemplo: Iniciar sesión el la página web, compartir la información del trabajo vía twitter y comentar en una publicación de facebook. 
    
    \noindent Dichas acciones tienen que estar relacionadas con el ''¿Por qué?''.\\
    
    \noindent \textbf{3.- Describe a tus usuarios}\\
    
    \noindent ¿Qué usuarios estarán usando tu sistema? y aún más importante, ¿cuál es tu relación con ellos? y/o ¿qué tanto sabes de ellos? Esto para poder conocer qué podría motivarlos.\\
    
    
    \noindent \textbf{4.- Define tus actividades de inicio a fin}\\
    
    \noindent Conociendo a tus usuarios y tus objetivos ya puedes diseñar que actividades tendrá tu sistema y cómo es el flujo en cada una de ellas. En los juegos siempre las actividades tienen un inicio y a veces tienen un final. Y hay veces que se tienen ciclos antes de llegar al final. Por eso mismo se debe tomar en cuenta que hay 2 posibles formas de crear tu flujo de actividad: de forma de ciclo y forma de escaleras.\\
    
    
    
    
    \noindent \textbf{5.- Nunca olvides la diversión}\\
    
    \noindent Antes de empezar a usar el sistema se tiene que dar un paso atrás y preguntarte si al menos tú consideras que es divertido, si a ti te gustaría probar el hacer dichas actividades voluntariamente.\\
    
    
    \noindent \textbf{6.- Utiliza las herramientas adecuadas para el trabajo}\\
    
    \noindent En esta paso se tiene que especificar qué elementos de juego se utilizarán a lo largo de las actividades diseñadas anteriormente y empezar a codificarlas en tu sistema.\\
    
%\subsection{Marco de trabajo C}
% TODO: Agregar el paper que habla d epapers aquí.

\begin{comment}
\section{Elección de Marco de Trabajo}

%Hemos decidido utilizar  a Yukai-Cho(Octalysis)

    Gracias a que Octalysis divide la implementación de la Gamificación en 8 principios, da una flexibilidad mayor en su implementación puesto que se pueden elegir diferentes herramientas para implementar sus principios, a diferencia de los otros autores que solo enumeran las herramientas más usadas (puntos, insignias y tablas de clasificación.) y no dan cabida al uso de otras distintas decidimos utilizar 0ctalysis por las ventajas.
     
     - Modularidad de principios
     - 
     
     
    Que principios se tendrán.
\end{comment}

