\section{Metología}

 El desarrollo de este proyecto se realizará mediante un desarrollo iterativo
 utilizando el marco de referencia Scrum. Este aparatado está destinado a
 presentar los roles, eventos y artefactos de scrum, así como a describir
 la forma en que han sido configurado para este proyecto.\\

    \noindent Los creadores Schwaber y Sutherland definen scrum de la siguiente manera:
        \begin{quote}
        Scrum es un marco de trabajo en el cual las personas pueden abordar
        problemas complejos de adaptación mientras que, productiva y creativamente
        desarrollan productos con el mayor valor posible \cite{TheScrumGuide}.
        \end{quote}

    \noindent De acuerdo con Michele Sliger PMP (Project Management Professional) y CST (Certified Scrum Trainer):
        \begin{quote}
        Scrum es un método ágil de entrega iterativa e incremental de productos que
        utiliza comentarios frecuentes y toma de decisiones en colaboración \cite{Sliger1}.
        \end{quote}

    \noindent Pete Deemer, CEO de GoodAgile (Certificadora para Scrum Masters y Product Owners) lo define cómo:% y lider de ScrumPrimer.
        \begin{quote}
        Scrum es un marco de trabajo en el que equipos multifuncionales pueden crear productos
        o desarrollar proyectos de una forma iterativa e incremental \cite{ScrumPrimer}.
        \end{quote}


\subsection{Equipo de Scrum}

 \noindent El equipo de Scrum está conformado por el dueño del producto {\em(Product Owner)},
 el equipo de desarrollo {\em(Development team)} y el maestro scrum {\em(Scrum Master)}.

\subsubsection{Product Owner} 

 El dueño del producto o {\em Product Owner} es responsable de maximizar el valor del producto
 resultante del trabajo del equipo de desarrollo, su principal responsabilidad es la gestión
 del artefacto {\em``Product Backlog''}. Para que las funciones del {\it Product Owner} sean exitosas,
 todos los involucrados en el proyecto deben respetar sus decisiones.\\
 % Nadie puede forzar al equipo de desarrollo a trabajar en requerimientos distintos a los del product owner.\\

 \noindent En este proyecto el rol del Product Owner lo llevarán a cabo los directores del trabajo terminal:
     
    \begin{quote}
    \begin{itemize}
        \item M. en C. Sandra Ivette Bautista, y el
        \item M. en C. Edgar Armando Catalán
    \end{itemize}
    \end{quote}
                                         
 \noindent A pesar de que en la guía oficial de Scrum \cite{TheScrumGuide} se especifica que el {\it Product
 Owner} debe ser una persona, se decidió que este rol fuera llevado a cabo mediante los directores del trabajo
 terminal con la premisa de que para la toma de decisiones ambos directores deben de estar de acuerdo.

\subsubsection{Equipo de desarrollo}

 El equipo de desarrollo consiste en un grupo de profesionales que realizan el trabajo para entregar
 los incrementos del producto al final de cada {\it Sprint}. El equipo de desarrollo es un equipo
 auto-organizado y multifuncional, durante el desarrollo de este proyecto el equipo estará conformado por:

 % \noindent El tamaño del equipo debe ser lo suficientemente pequeño para permanecer ágil y lo
 % suficientemente grande para realizar entregas significativas al final de cada {\em Sprint}.\\
 % Un equipo con menos de tres miembros disminuiría la interacción y por lo tanto la productividad,
 % por otro lado, con más de nueve miembros se requiere mayor coordinación con más complejidad.

 % \noindent El equipo de desarrollo estará conformado por los estudiantes que cursan el trabajo terminal:
        
    \begin{quote}
    \begin{itemize}
        \item David Flores Casanova
        \item Ricardo Naranjo Polit
        \item Daniel Isaí Ortega Zúñiga
    \end{itemize}
    \end{quote}

\subsubsection{Maestro Scrum}

 El maestro Scrum o {\it Scrum Master} es el líder que está al servicio del equipo,
 se encarga de ayudar al equipo scrum a maximizar el valor y mejorar continuamente la
 forma de trabajo, además de guiar a todos los involucrados hacia una mejor implementación
 del marco de trabajo.\\

 \noindent El rol del {\it Scrum Master} durante este proyecto se llevará a cabo mediante
 dos personas con la finalidad de dividir las responsabilidades y no sobrecargar de trabajo
 a una persona. El {\it Scrum Master} estará conformado por:

    \begin{quote}
    \begin{itemize}
        \item M. en C. Edgar Armando Catalán {\it (Responsabilidades hacia el Product Owner)}
        \item Daniel Isaí Ortega {\it(Responsabilidades hacia el equipo de desarrollo)}
    \end{itemize}                        
    \end{quote}

\subsubsection{Stakeholders}

 Los Stakeholders son personas externas al equipo Scrum con un interés y/o conocimientos
 específicos del producto \cite{ScrumGlosary}. Durante el desarrollo del trabajo terminal
 se consideran a los sinodales cómo los Stakeholders oficiales, los cuales son listados a
 continuación:

    \begin{quote}
    \begin{itemize}
        \item Dra. Fabiola Ocampo Botello
        \item M. en C. María del Socorro Téllez Reyes
        \item M. en C. José David Ortega Pacheco
    \end{itemize}
    \end{quote}

\subsection{Eventos}

 Los eventos prescritos en Scrum son usados para crear regularidad en el proceso, además
 minimizan la necesidad de juntas no definidas. Todos los eventos tienen un tiempo establecido, 
 para los {\it sprints} la duración es fija y no puede ser acotada o alargada, los demás eventos
 pueden concluir una vez que su propósito es cumplido.

\subsubsection{Sprint}

 Un {\it Sprint} es el lapso de tiempo en el cual un incremento del producto es creado, los sprints
 son secuenciales, es decir, inician inmediatamente después del término de otro. Internamente 
 cada Sprint consiste en las etapas de planeación, reuniones diarias ({\it Daily Scrum}),
 desarrollo, revisión y retroalimentación.
    
    % \begin{itemize}
    %    \item No se pueden hacer cambios a la definición del Sprint Goal. (objetivo del Sprint)
    %    \item Los objetivos de calidad no disminuyen.
    %    \item El alcance debe ser clarificado por el Product Owner y negociado entre él y el Team.
    % \end{itemize}
    
 % \noindent Solo el Product Owner tiene la autoridad de cancelar un Sprint. Otra situación que puede
 % cancelar el Sprint es que el Sprint Goal se vuelva obsoleto.

 \begin{quote}
    \noindent Para este proyecto los Sprints están configurados a una duración de 14 días con una
    estimación de 18 iteraciones, la duración de dos semanas se estableció con el propósito de:
    
    \begin{itemize}
        \item Incrementar la retroalimentación y detectar los impedimentos
              en la forma de trabajo lo más pronto posible, y

        \item Realizar incrementos más cortos y continuos considerando que el
              equipo de desarrollo está conformado por tres integrantes.
    \end{itemize} 
 \end{quote}
    
\subsubsection{Planeación}

 La planeación del {\it Sprint} se realiza con todos los miembros que forman parte del equipo scrum,
 en dicha reunión se establece cuál será el incremento entregado al final del sprint, y la forma
 en que se logrará el objetivo del mismo.\\

 \noindent El día acordado para llevar a cabo esta reunión son los {\bf martes} cada dos semanas
 {\bf a la 1:30pm} en las instalaciones de la ESCOM. El horario fue acordado tomando en cuenta
 la disponibilidad de todos los miembros del equipo Scrum.
    
    \begin{quote}
    {\bf Nota:} En caso de que, por algun evento extraordinario, no se pueda
                llevar a cabo el Sprint Planning este reunión se reagendará para
                que ocurra lo más pronto posible.
    \end{quote}
  
\subsubsection{Daily Scrum}

 Las reuniones diarias o Daily Scrum se realiza día a día durante la ejecución de los {\it Sprint}, deben
 durar como máximo 15 minutos, en esta reunión el equipo de desarrollo planea como trabajá durante el día.
 En la tabla \ref{tbl:daily} muestra los días acordados, lugar y hora pre-establecidos para la reunión.

 % En la reunión cada miembro debe responder las siguientes preguntas:
     
    % \begin{itemize}
    %    \item ¿Qué hice ayer para ayudar al Team a alcanzar el Sprint Goal?
    %    \item ¿Qué hare hoy para ayudar al Team a lograr el Sprint Goal?
    %    \item ¿Veo algun obstaculo que me impida o impida al Team lograr el Sprint Goal?
    % \end{itemize} 
   
    \addtable{|c|c|c|}{tbl:daily}{
        {\bf Día de Trabajo} & {\bf Lugar} & {\bf Hora Inicio} \\\hline
        Lunes     & ESCOM Sala 21 N & 10:00am \\\hline
        Martes    & ESCOM Sala 21 N & 10:00am \\\hline
        Miércoles & ESCOM Sala 21 N & 10:00am \\\hline
        Jueves    & ESCOM Sala 21 N & 10:00am \\\hline
        Viernes   & ESCOM Sala 21 N & 10:00am \\\hline
        Domingo   & -               & 12:00pm \\\hline
    }{Horario de Daily Scrum}
    
 \noindent Debido a la dificultad de hacer coincidir los horarios del equipo de desarrollo  con los
 del {\it Product Owner}, cuando se requiera de sus decisiones, opinión o retroalimentación se le contactará
 a través de mensajería instantánea.
    
\subsubsection{Revisión}

 Esta reunión se realiza al final de cada {\it Sprint} para de revisar el incremento, discutir los inconvenientes
 encontrados y en dado caso adaptar la planeación del proyecto. El equipo scrum y los stakeholder deben
 revisar el incremento y establecer cuáles serán los cambios a ejecutar.\\

 \noindent Debido a que en este proyecto, los stakeholders y el equipo de scrum tienen distintos horarios
 de disponibilidad, la revisión del {\it sprint} se divide en cuatro fases, aplicando la primer fase a los sprints
 impares y las cuatro fases para sprints pares. Las fases de describen a continuación:
 
    \begin{quote}
    \begin{itemize}
    \item[\it Fase 1]
        Consiste en realizar una primer reunión con el equipo scrum para obtener una
        retroalimentación y revisar el incremento entregado.

    \item[\it Fase 2]
        En esta fase el equipo de desarrollo tiene reuniones con los Stakeholders con la
        finalidad de obtener retroalimentación y observaciones acerca de la forma de
        trabajo y del incremento.

    \item[\it Fase 3] 
        En esta fase los miembros del equipo scrum revisarán las observaciones y
        comentarios de los {\it Stakeholders} para saber cuales proceden.

    \item[\it Fase 4]
        Se avisa a los Stakeholders acerca de cuales observaciones procedieron y cuales no.\\
    \end{itemize}    
    
    {\bf Nota:} Las reuniones de la fase 2, dependen de la disponibilidad que cada {\it stakeholder} tenga,
                en caso de que ningún stakeholder tenga disponibilidad para llevar a cabo la fase 2,
                el proceso de la revisión del {\it Sprint} terminará.
    \end{quote}

\subsubsection{Retroalimentación}

 La etapa de retroalimentación consiste en una reunión del equipo scrum con el objetivo de crear un plan para
 las mejoras en la forma de trabajo, esta reunión ocurre después de la revisión y antes de la planeación del
 siguiente {\it Sprint}. En esta reunión los miembros del equipo scrum ven cómo atender las debilidades y áreas
 de oportunidad.

\subsection{Artefactos}

 Los artefactos en Scrum proporcionan transparencia en toda la aplicación de Scrum
 y además funcionan como herramientas para la inspección de la implementación del
 marco de trabajo y adaptación del mismo. Tener los artefactos organizados brinda
 una mayor visibilidad acerca del avance del proyecto y del producto final.
 
\clearpage

\subsubsection{Product Backlog}

 La cartera de producto o {\it product Backlog} lista todas las características, funcionalidades, requerimientos,
 y mejoras necesarias para la creación del producto. El responsable del contenido, disponibilidad y organización
 del {\it Product Backlog} es el {\it Product Owner}.\\
       
 \noindent Debido a que el proyecto requería una etapa de investigación, se optó por tener dos tipos
 de {\it items} en el product backlog, los items de documentación/preparación del proyecto  y los {\it items}
 para desarrollo del mismo.\\
    
    \noindent{\bf Items de Documentación}\\
    Los items de preparación del proyecto y documentación deben ser especificados
    mediante los atributos presentes en la tabla \ref{attrPBpre}:
    
    \addtable{|l|l|}{attrPBpre}{
        {\bf Atributo} & {\bf Descripción}                                                             \\\hline
        id           &  Es una identificador de la forma ``Ax'' donde {\it x} es un número consecutivo \\\hline
        nombre       &  Nombre representativo de la actividad                                          \\\hline
        descripción  &  Detalle de lo que hay que hacer para llevar a cabo esta actividad.             \\\hline
        sprint       &  Indica el número de Sprint al cual ha sido asignada esta tarea.                \\\hline
        %estado      & Indica el estado ({\it por hacer, en proceso o concluida} de una actividad. \\\hline
        %estimación  & Especifica el periodo de tiempo estimado para la liberación de dicha actividad. \\\hline
    }{Atributos de los Items del P.B de Documentación}


    \noindent{\bf Items de Desarrollo del Proyecto}\\
    Describen las características del software que se desarrollará, estos items deben
    ser redactados de manera objetiva y como requerimientos del sistema, y deben contener
    los atributos presentes en la tabla \ref{attrPB}:
    
    \addtable{|l|p{0.62\textwidth}|}{attrPB}{
        {\bf Atributo} & {\bf Descripción}\\\hline
        id           &  Es una identificador de la forma '{\bf RFx}' o '{\bf RNFx}' para requerimientos    \par
                        funcionales y no funcionales respectivamente. {\it x} es un número consecutivo     \\\hline

        nombre       &  Nombre representativo del requerimiento del sistema.                               \\\hline
        descripción  &  Descripción concisa y objetiva acerca del requerimiento.                           \\\hline
        prioridad    &  Indica la prioridad de un requerimiento, los valores posibles son:                 \par
                        \qquad MA (muy alta), A (alta), M (Media), B (baja) y MB (muy baja)                \\\hline

        sprint       &  Indica el número de Sprint al cual ha sido asignado este requerimiento.            \\\hline
        tipo         &  Tipo de requerimiento no funcional según la clasificación propuesta por Frank Tsui \\\hline
        %estimación  & Especifica el periodo de tiempo estimado para la liberación de dicho requerimiento. \\\hline
    }{Atributos de los Items del P.B de Desarrollo del Proyecto}
    
    \begin{quote}
    {\bf Nota:} El atributo {\it Sprint} debe estar presente en todos los items correspondientes
                al sprint corriente y a los sprints anteriores a este. El atributo {\it Sprint}
                puede no estar presente en los items que no han sido vinculados a un Sprint. 
    \end{quote}
    
\subsubsection{Sprint Backlog}
    
 El {\it Sprint Backlog} está formado por los {\it items} del {\it Product Backlog} seleccionados
 para el {\it Sprint} más el plan para entregarlos, y así cumplir con el Sprint Goal. El {\it Sprint
 Backlog} es una estimación de las funcionalidades que serán entregadas en el siguiente incremento
 del producto.

 % \noindent Conforme los {\it items} del {\it product backlog} vayan siendo seleccionados para
 % tratarse en un sprint, se les añadirá una etiqueta que indique a qué sprint pertenecen.
    
    % \addtable{|l|l|}{SBItems}{
    %    {\bf Atributo} & {\bf Descripción}\\\hline
    %    sprint   &  Indica el número de Sprint al cual ha sido asignado el item.             \\\hline
    %    pruebas  &  (Opcional) Sentencia de cómo se evaluara que dicho ítem esté completado. \\\hline
    %
    % }{Atributos del Sprint Backlog }
