\section{Metodología}

 Este proyecto es realizado mediante un desarrollo iterativo utilizando como referencia
 el marco de trabajo {\it Scrum}. A lo largo de esta sección se presenta el marco de trabajo
 los roles, eventos y artefactos del mismo, a continuación se mencionan tres definiciones
 de distintos autores para brindar una perspectiva más completa acerca del marco de trabajo.\\

    \noindent Sus creadores, K. Schwaber y J. Sutherland lo definen de la siguiente manera:
        \begin{quote}
        Scrum es un marco de trabajo en el cual las personas pueden abordar
        problemas complejos de adaptación mientras que, productiva y creativamente
        desarrollan productos con el mayor valor posible \cite{TheScrumGuide}.
        \end{quote}

    \noindent De acuerdo con Michele Sliger PMP (Project Management Professional) y CST (Certified Scrum Trainer):
        \begin{quote}
        Scrum es un método ágil de entrega iterativa e incremental de productos que
        utiliza comentarios frecuentes y toma de decisiones en colaboración \cite{Sliger1}.
        \end{quote}

    \noindent Pete Deemer, CEO de GoodAgile (Certificadora de Scrum) lo define cómo:
            % y lider de ScrumPrimer.
        \begin{quote}
        Scrum es un marco de trabajo en el que equipos multifuncionales pueden crear productos
        o desarrollar proyectos de una forma iterativa e incremental \cite{ScrumPrimer}.
        \end{quote}


\clearpage

\subsection{Equipo de Scrum}

 El equipo de Scrum está conformado por el dueño del producto {\em(Product Owner)}, el equipo
 de desarrollo {\em(Development team)} y el maestro scrum {\em(Scrum Master)}. A continuación
 se especifican sus responsabilidades.

\subsubsection{Product Owner} 

 \noindent El dueño del producto o {\em Product Owner} es responsable de maximizar el valor
 del producto resultante del trabajo del equipo de desarrollo, su principal responsabilidad
 es la gestión del artefacto llamado {\em``Product Backlog''}, misma que incluye:
   
    \begin{quote}
    \begin{itemize}
        \item Expresar claramente los elementos del {\it Product Backlog}.
        \item Ordenar los elementos del {\it Product Backlog} para lograr mejor las metas y objetivos.
        \item Optimizar el valor del trabajo realizado por el equipo de desarrollo.
        \item Asegurarse de qué el Product Backlog es visible, transparente, y claro para todos.
        \item Asegurarse de que el equipo de desarrollo entienda los elementos del {\it Product Backlog}
                al nivel requerido.\\
    \end{itemize}
    \end{quote}
    
 \noindent El {\it Product Owner} debe trabajar en conjunto con el equipo de desarrollo para cumplir con
 los puntos anteriormente mencionados. Además, el {\it Product Owner} debe ser una persona, no un comité,
 en caso de existir un comité el {\it Product Owner} sera el representante de dicho comité.
    
\subsubsection{Equipo de Desarrollo}

 El equipo de desarrollo consiste en un grupo de profesionales que realizan el trabajo requerido para
 entregar los incrementos del producto al final de cada iteración o {\it Sprint}. El equipo de desarrollo
 debe tener las siguientes características:
    
    \begin{quote}
    \begin{itemize}
        \item Es un equipo auto-organizado, no es requerido que alguien les guie
              acerca de cómo llevar a cabo los incrementos.

        \item El equipo debe ser multi-funcional y contener como equipo todas las
              habilidades necesarias para crear los incrementos del producto.

        \item No hay títulos o etiquetas para los miembros del equipo de desarrollo.

        \item No hay equipos internos en el equipo de desarrollo.

        \item Los miembros del equipo pueden tener habilidades y áreas de conocimiento
              especializadas, pero el conjunto de habilidades pertenecen al equipo.
    \end{itemize}
    \end{quote}
        
 \noindent El Team debe ser lo suficientemente pequeño para permanecer ágil y lo suficientemente grande
 para completar un trabajo significativo dentro de un Sprint.\\
        
 \noindent Un Team con menos de tres miembros disminuiría la interacción, lo cual se traduce como una
 menor productividad, mientras que tener más de nueve miembros requiere demasiada coordinación. Los
 equipos de desarrollo grandes generan demasiada complejidad para que un proceso empírico sea útil.

\clearpage

\subsubsection{Maestro Scrum}

 El maestro Scrum o {\it Scrum Master} es el líder que está al servicio del equipo,
 se encarga de ayudar al equipo scrum a maximizar el valor y mejorar continuamente la
 forma de trabajo, además de guiar a todos los involucrados hacia una mejor implementación
 del marco de trabajo.

    \begin{quote}
        {\bf Responsabilidades relacionadas con el Product Owner }
        \begin{itemize}
        \item Asegurar que los objetivos, alcance y definición del producto sea
              entendido por todos los miembros del equipo de desarrollo.

        \item Encontrar técnicas para una gestión efectiva del artefacto
              {\it Product Backlog}.

        \item Ayudar al equipo de desarrollo a entender la necesidad de la claridad
              y objetividad de los elementos del {\it Product Backlog}.

        \item Entender la planeación del producto en un ambiente empírico.

        \item Asegurarse de que el {\it Product Owner} organice el {\it Product
              Backlog} para maximizar su valor.

        \item Entender y practicar la agilidad.

        \item Facilitar los eventos de Scrum cuando sean solicitados o necesarios.\\
        \end{itemize}

        {\bf Responsabilidades relacionadas con el equipo de desarrollo}
        \begin{itemize}
        \item Entrenar al equipo para ser auto-organizadas y multi-funcionales.
        \item Ayudar al quipo a crear incrementos de alto valor.
        \item Resolver/Remover los impedimientos que frenen el progreso del equipo.
        \item Facilitar los eventos de Scrum cuando sean solicitados o necesarios.
        \item Entrenar al equipo en entornos donde Scrum no puede ser completamente
                adoptado y/o entendido.\\
        \end{itemize}

        %\noindent {\bf Responsabilidades relacionadas con la organización}
        %\begin{itemize}
        %\item Liderar y entrenar a la organización en la adopción del marco de trabajo Scrum.
        %\item Planear las implementaciones de Scrum en la organización.
        %\item Ayudar a los empleados y a los stakeholders a entender Scrum.
        %\item Trabajar con otros Scrum Masters para incrementar la efectividad de la aplicación de Scrum en la organización.
        %\end{itemize}
    \end{quote}

\subsection{Eventos}

 Los eventos prescritos en {\it Scrum} son usados para crear regularidad en el proceso, además
 minimizan la necesidad de juntas no definidas. Todos los eventos tienen un tiempo establecido, 
 para las iteraciones o {\it sprints} la duración es fija y no puede ser acotada o alargada, los
 demás eventos pueden concluir una vez que su propósito es cumplido.

\subsubsection{Sprint}

 Un {\it Sprint} es el lapso de tiempo en el cual un incremento del producto es creado, los sprints
 son secuenciales, es decir, inician inmediatamente después del término de otro. Internamente 
 cada Sprint consiste en las etapas de planeación, reuniones diarias ({\it Daily Scrum}),
 desarrollo, revisión y retroalimentación.\\

 \noindent A continuación se mencionan las características principales que debe tener este evento:

    \begin{itemize}
    \item Los sprints deben tener una duración máxima de un mes.
    \item No se pueden hacer cambios a la definición del objetivo del {\it sprint} ({\it Sprint Goal})
    \item Los objetivos de calidad no disminuyen.
    \item El alcance debe ser clarificado por el {\it Product Owner} y negociado entre él y el equipo
           de desarrollo.
    \end{itemize}
    
 % \noindent Solo el Product Owner tiene la autoridad de cancelar un Sprint. Otra situación que puede
 % cancelar el Sprint es que el Sprint Goal se vuelva obsoleto.

    
\subsubsection{Planeación}

 La planeación del {\it Sprint} se realiza con todos los miembros que forman parte del equipo scrum,
 en dicha reunión se establece cuál será el incremento entregado al final del sprint, y la forma
 en que se logrará el objetivo del mismo.\\

 \noindent La duración de Sprint Planning está relacionada con la duración del {\it Sprint} y la
 cantidad de miembros en el equipo scrum; para {\it Sprints} de un mes la reunión de planeación debe
 durar cómo máximo 8 horas. Durante la reunión se debe acordar el alcance del sprint, el objetivo
 ({\it Sprint Goal}) y las funcionalidades específicas que se desarrollarán.\\

 \noindent El conjunto de funcionalidades a desarrollar deben ser tomadas del artefacto {\it Product
 Backlog} y formarán el artefacto {\it Sprint Backlog} el cual define del alcance del sprint. Como
 requerimiento es necesario contar con el incremento realizado en el {\it sprint} anterior más la base
 de conocimiento acerca cómo ha trabajo el equipo scrum.\\
    
 % El Team se auto-organiza para llevar acabo el trabajo del Sprint Backlog. Las estimaciones
 % de tiempo y esfuerzo se miden en unidades de un día o menos.
    
 % Al finalizar el Sprint Planning, el Team debe ser capaz de explicar al Product Owner y al Scrum
 % Master como se organizarán y la forma de trabajo que ocuparan para alcanzar el Sprint Goal y crear
 % el incremento.
   
 \noindent El objetivo del {\it sprint} o {\it Sprint Goal} debe ser claro y entendible para todos los
 miembros del equipo scrum, además el objetivo debe proporciona orientación al equipo de desarrollo sobre
 el propósito por el cual se está creando el incremento.
  
\subsubsection{Reunión diaria}

 La reunión diaria o {\it Daily Scrum} se realiza día a día durante la ejecución de los {\it Sprint} y debe
 durar como máximo 15 minutos, este reunión tiene la finalidad de optimizar la colaboración y el aprovechamiento
 haciendo una inspección del trabajo realizado desde el anterior {\it daily scrum} y estableciendo qué es lo
 que se trabajará durante el día.\\

 En la reunión cada miembro debe responder las siguientes preguntas:
    
    \begin{quote}
    \begin{itemize}
    \item ¿Qué hice ayer para ayudar al equipo de desarrollo a alcanzar el objetivo del sprint?
    \item ¿Qué hare hoy para ayudar al equipo a lograr el objetivo del sprint?
    \item ¿Veo algún obstaculo que me impida (o al equipo de desarrollo) lograr el objetivo del sprint?
    \end{itemize}
    \end{quote} 

 \noindent El Daily Scrum es una reunión interna del equipo de desarrollo. En caso de que otros estén presentes
 el maestro scrum debe asegurar que la fluidez de la reunión. A menudo los miembros del equipo suelen reunirse
 al termino de la reunión para discutir, adaptar o replantear aspectos mencionados en la reunión.
    
\subsubsection{Revisión}

 Esta reunión se realiza al final de cada {\it Sprint} para de revisar el incremento, discutir los inconvenientes
 encontrados y en dado caso adaptar la planeación del proyecto. El equipo scrum y los stakeholders deben
 revisar el incremento y establecer cuáles serán los cambios a implementar con la finalidad de optimizar el valor
 de los siguientes incrementos.\\

 \noindent La reunión consiste en presentar el incremento realizado para obtener retroalimentación de todos
 los involucrados en el proyecto, para {\it sprints} de 1 mes la reunión debe durar cómo máximo cuatro horas.
 A continuación se detallan los puntos que debe cumplir esta reunión.

    \begin{quote}
    \begin{itemize}
    \item El equipo de desarrollo presenta el incremento, los problemas encontrados y las soluciones
            tomadas.

    \item El {\it Product Owner} confirma cuales elementos del {\it Product Backlog} han sido realizados.
    \item El equipo scrum comenta las cosas a realizar en el siguiente incremento.
    \item Se revisan los tiempos de entrega y presupuestos para las siguientes entregas.
    \end{itemize}
    \end{quote}


\subsubsection{Retroalimentación}

 La etapa de retroalimentación consiste en una reunión del equipo scrum con el objetivo de crear un plan para
 las mejoras en la forma de trabajo, esta reunión ocurre después de la revisión y antes de la planeación del
 siguiente {\it Sprint}. En esta reunión los miembros del equipo scrum ven cómo atender las debilidades y áreas
 de oportunidad.\\

 \noindent Reunión correspondiente a este etapa debe ocurrir después la retroalimentación y antes del inicio del
 nuevo sprint, aproximadamente debe durar tres horas para {\it Sprints} de un mes. Los objetivos principales de
 esta reunión son:

    \begin{quote}
    \begin{itemize}
    \item Inspeccionar el último {\it Sprint} en relación a las personas, relaciones, procesos y herramientas.
    \item Identificar y ordenar las cosas que ocurrieron bien durante el Sprint y las cosas que hay que mejorar.
    \item Crear un plan para la aplicación de las mejoras para tener una mejor implementación de Scrum.
    \end{itemize}
    \end{quote}

\subsection{Artefactos}

 Los artefactos en Scrum proporcionan transparencia en toda la aplicación de Scrum
 y además funcionan como herramientas para la inspección de la implementación del
 marco de trabajo y adaptación del mismo. A continuación se presentan los artefactos
 definidos por el macro de trabajo.

 %Tener los artefactos organizados brinda
 % una mayor visibilidad acerca del avance del proyecto y del producto final. 
 
%\clearpage

\subsubsection{Product Backlog}

 La cartera de producto o {\it product Backlog} lista todas las características, funcionalidades, requerimientos,
 y mejoras necesarias para la creación del producto. El responsable del contenido, disponibilidad y organización
 del {\it Product Backlog} es el {\it Product Owner}.\\

\subsubsection{Sprint Backlog}
    
 El {\it Sprint Backlog} está formado por los {\it items} del {\it Product Backlog} seleccionados
 para el {\it Sprint} más el plan para entregarlos, y así cumplir con el Sprint Goal. El {\it Sprint
 Backlog} es una estimación de las funcionalidades que serán entregadas en el siguiente incremento
 del producto.

\clearpage

% INTRODUCCIÓN

 % la forma en que han sido configurado el marco de trabajo scrum para este proyecto.\\

% ROLES 
% ROLES - PRODUCT OWNER

 % \noindent Para que las funciones del Product Owner sean exitosas, la organización debe respetar sus
 % decisiones. Nadie puede forzar al Development Team para trabajar en un conjunto distinto de requerimientos
 % establecidos en el Product Backlog.

 % \noindent En este proyecto el rol del Product Owner lo llevarán a cabo los directores del trabajo terminal:
     
    % \begin{quote}
    % \begin{itemize}
    %    \item M. en C. Sandra Ivette Bautista, y el
    %    \item M. en C. Edgar Armando Catalán
    % \end{itemize}
    % \end{quote}
                                         
 % \noindent A pesar de que en la guía oficial de Scrum \cite{TheScrumGuide} se especifica que el {\it Product
 % Owner} debe ser una persona, se decidió que este rol fuera llevado a cabo mediante los directores del trabajo
 % terminal con la premisa de que para la toma de decisiones ambos directores deben de estar de acuerdo.

% ROLES - EQUIPO DE DESARROLLO

 % durante el desarrollo de este proyecto el equipo de desarrollo estará conformado por:

 % \noindent El tamaño del equipo debe ser lo suficientemente pequeño para permanecer ágil y lo
 % suficientemente grande para realizar entregas significativas al final de cada {\em Sprint}.\\
 % Un equipo con menos de tres miembros disminuiría la interacción y por lo tanto la productividad,
 % por otro lado, con más de nueve miembros se requiere mayor coordinación con más complejidad.

 % \noindent El equipo de desarrollo estará conformado por los estudiantes que cursan el trabajo terminal:
        
    % \begin{quote}
    % \begin{itemize}
    %    \item David Flores Casanova
    %    \item Ricardo Naranjo Polit
    %    \item Daniel Isaí Ortega Zúñiga
    % \end{itemize}
    % \end{quote}

% ROLES - SCRUM MASTER

 % \noindent El rol del {\it Scrum Master} durante este proyecto se llevará a cabo mediante
 % dos personas con la finalidad de dividir las responsabilidades y no sobrecargar de trabajo
 % a una persona. El {\it Scrum Master} estará conformado por:

    % \begin{quote}
    % \begin{itemize}
    %    \item M. en C. Edgar Armando Catalán {\it (Responsabilidades hacia el Product Owner)}
    %    \item Daniel Isaí Ortega {\it(Responsabilidades hacia el equipo de desarrollo)}
    % \end{itemize}                        
    % \end{quote}

% ROLES - STAKEHOLDERS

 % Los Stakeholders son personas externas al equipo Scrum con un interés y/o conocimientos
 % específicos del producto \cite{ScrumGlosary}. 

 % Durante el desarrollo del trabajo terminal
 % se consideran a los sinodales cómo los Stakeholders oficiales, los cuales son listados a
 % continuación:

    % \begin{quote}
    % \begin{itemize}
    %    \item Dra. Fabiola Ocampo Botello
    %    \item M. en C. María del Socorro Téllez Reyes
    %    \item M. en C. José David Ortega Pacheco
    % \end{itemize}
    % \end{quote}

% EVENTOS
% EVENTOS - SPRINTS

 % \begin{quote}
 %   \noindent Para este proyecto los Sprints están configurados a una duración de 14 días con una
 %   estimación de 18 iteraciones, la duración de dos semanas se estableció con el propósito de:
    
    % \begin{itemize}
    %    \item Incrementar la retroalimentación y detectar los impedimentos
    %          en la forma de trabajo lo más pronto posible, y

    %    \item Realizar incrementos más cortos y continuos considerando que el
    %          equipo de desarrollo está conformado por tres integrantes.
    % \end{itemize} 
 % \end{quote}

% EVENTOS - PLANEACiÓN (SPRINT PLANNING)

 % \noindent El día acordado para llevar a cabo esta reunión son los {\bf martes} cada dos semanas
 % {\bf a la 1:30pm} en las instalaciones de la ESCOM. El horario fue acordado tomando en cuenta
 % la disponibilidad de todos los miembros del equipo Scrum.
    
    % \begin{quote}
    % {\bf Nota:} En caso de que, por algun evento extraordinario, no se pueda
    %            llevar a cabo el Sprint Planning este reunión se reagendará para
    %            que ocurra lo más pronto posible.
    % \end{quote}

% EVENTOS - DAILY SCRUM

 % En la tabla \ref{tbl:daily} muestra los días acordados, lugar y hora pre-establecidos para la reunión.

    % \addtable{|c|c|c|}{tbl:daily}{
    %    {\bf Día de Trabajo} & {\bf Lugar} & {\bf Hora Inicio} \\\hline
    %    Lunes     & ESCOM Sala 21 N & 10:00am \\\hline
    %    Martes    & ESCOM Sala 21 N & 10:00am \\\hline
    %    Miércoles & ESCOM Sala 21 N & 10:00am \\\hline
    %    Jueves    & ESCOM Sala 21 N & 10:00am \\\hline
    %    Viernes   & ESCOM Sala 21 N & 10:00am \\\hline
    %    Domingo   & -               & 12:00pm \\\hline
    %}{Horario de Daily Scrum}
    
 % \noindent Debido a la dificultad de hacer coincidir los horarios del equipo de desarrollo  con los
 % del {\it Product Owner}, cuando se requiera de sus decisiones, opinión o retroalimentación se le contactará
 % a través de mensajería instantánea.

% SPRINT REVIEW

 % \noindent Debido a que en este proyecto, los stakeholders y el equipo de scrum tienen distintos horarios
 % de disponibilidad, la revisión del {\it sprint} se divide en cuatro fases, aplicando la primer fase a los sprints
 % impares y las cuatro fases para sprints pares. Las fases de describen a continuación:
 
 %   \begin{quote}
 %   \begin{itemize}
 %   \item[\it Fase 1]
 %       Consiste en realizar una primer reunión con el equipo scrum para obtener una
 %       retroalimentación y revisar el incremento entregado.

 %   \item[\it Fase 2]
 %       En esta fase el equipo de desarrollo tiene reuniones con los Stakeholders con la
 %       finalidad de obtener retroalimentación y observaciones acerca de la forma de
 %       trabajo y del incremento.

 %   \item[\it Fase 3] 
 %       En esta fase los miembros del equipo scrum revisarán las observaciones y
 %       comentarios de los {\it Stakeholders} para saber cuales proceden.

 %   \item[\it Fase 4]
 %       Se avisa a los Stakeholders acerca de cuales observaciones procedieron y cuales no.\\
 %   \end{itemize}    
    
 %   {\bf Nota:} Las reuniones de la fase 2, dependen de la disponibilidad que cada {\it stakeholder} tenga,
 %               en caso de que ningún stakeholder tenga disponibilidad para llevar a cabo la fase 2,
 %               el proceso de la revisión del {\it Sprint} terminará.
 %   \end{quote}

% ARTEFACTOS 

% PRODUCT BACKLOG

 % \noindent Debido a que el proyecto requería una etapa de investigación, se optó por tener dos tipos
 % de {\it items} en el product backlog, los items de documentación/preparación del proyecto  y los {\it items}
 % para desarrollo del mismo.\\
    
 % \noindent{\bf Items de Documentación}\\
 % Los items de preparación del proyecto y documentación deben ser especificados
 % mediante los atributos presentes en la tabla \ref{attrPBpre}:
    
 %   \addtable{|l|l|}{attrPBpre}{
 %       {\bf Atributo} & {\bf Descripción}                                                             \\\hline
 %       id           &  Es una identificador de la forma ``Ax'' donde {\it x} es un número consecutivo \\\hline
 %       nombre       &  Nombre representativo de la actividad                                          \\\hline
 %       descripción  &  Detalle de lo que hay que hacer para llevar a cabo esta actividad.             \\\hline
 %       sprint       &  Indica el número de Sprint al cual ha sido asignada esta tarea.                \\\hline
 %       %estado      & Indica el estado ({\it por hacer, en proceso o concluida} de una actividad. \\\hline
 %       %estimación  & Especifica el periodo de tiempo estimado para la liberación de dicha actividad. \\\hline
 %   }{Atributos de los Items del P.B de Documentación}


 %   \noindent{\bf Items de Desarrollo del Proyecto}\\
 %   Describen las características del software que se desarrollará, estos items deben
 %   ser redactados de manera objetiva y como requerimientos del sistema, y deben contener
 %   los atributos presentes en la tabla \ref{attrPB}:
 %   
 %   \addtable{|l|p{0.62\textwidth}|}{attrPB}{
 %       {\bf Atributo} & {\bf Descripción}\\\hline
 %       id           &  Es una identificador de la forma '{\bf RFx}' o '{\bf RNFx}' para requerimientos    \par
 %                       funcionales y no funcionales respectivamente. {\it x} es un número consecutivo     \\\hline
%
 %       nombre       &  Nombre representativo del requerimiento del sistema.                               \\\hline
 %       descripción  &  Descripción concisa y objetiva acerca del requerimiento.                           \\\hline
 %       prioridad    &  Indica la prioridad de un requerimiento, los valores posibles son:                 \par
 %                       \qquad MA (muy alta), A (alta), M (Media), B (baja) y MB (muy baja)                \\\hline

 %       sprint       &  Indica el número de Sprint al cual ha sido asignado este requerimiento.            \\\hline
 %       tipo         &  Tipo de requerimiento no funcional según la clasificación propuesta por Frank Tsui \\\hline
        %estimación  & Especifica el periodo de tiempo estimado para la liberación de dicho requerimiento. \\\hline
 %   }{Atributos de los Items del P.B de Desarrollo del Proyecto}
    
  %  \begin{quote}
  %  {\bf Nota:} El atributo {\it Sprint} debe estar presente en todos los items correspondientes
  %              al sprint corriente y a los sprints anteriores a este. El atributo {\it Sprint}
  %              puede no estar presente en los items que no han sido vinculados a un Sprint. 
  %  \end{quote}
 

% SPRINT BACKLOG

 % \noindent Conforme los {\it items} del {\it product backlog} vayan siendo seleccionados para
 % tratarse en un sprint, se les añadirá una etiqueta que indique a qué sprint pertenecen.
    
    % \addtable{|l|l|}{SBItems}{
    %    {\bf Atributo} & {\bf Descripción}\\\hline
    %    sprint   &  Indica el número de Sprint al cual ha sido asignado el item.             \\\hline
    %    pruebas  &  (Opcional) Sentencia de cómo se evaluara que dicho ítem esté completado. \\\hline
    %
    % }{Atributos del Sprint Backlog }

