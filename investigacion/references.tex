
\begin{thebibliography}{9}

    \bibitem{GamByDesign}
        G. Zichermann y C. Cunningham,
        {\it Gamification by design: Implementing game mechanics in web and mobile apps.}
        Sebastopol, CA, USA: O'Reilly Media, 2011.
        % http://storage.libre.life/Gamification_by_Design.pdf
        
    \bibitem{Burke}
        B. Burke,
        {\it Gamify: How gamification motivates people to do extraordinary things.}
        Brookline, MA, USA: Bibliomotion, Garner Inc., 2014.
        
    \bibitem{Kapp}
        K.M. Kapp,
        {\it The gamification of learning and instruction: Game-based methods and strategies for training and education.}
        San Francisco, CA, USA: Pfeiffer, 2012.
        
    \bibitem{Huotari}
        K. Huotari y J. Hamari,
        ``Defining gamification - a service marketing perspective'', en
        {\it Proc. of the 16Th International Academic Mindtrek Conference: ``Envisioning Future Media Environments''},
        2012, pp. 17-22.
        % https://www.researchgate.net/publication/259841647_Defining_Gamification_-_A_Service_Marketing_Perspective
    
    \bibitem{DeterdingDefinition}
        S. Deterding, R. Khaled, L. Nacke y D. Dixon,
        ``Gamification: Toward a definition'', en
        {\it CHI}, Vancouver, BC, Canada, 2011.
        % http://gamification-research.org/wp-content/uploads/2011/04/02-Deterding-Khaled-Nacke-Dixon.pdf
        
    \bibitem{Seaborn}
        K. Seaborn y D.I. Fels
        ``Gamification in theory and action: A survey'',
        {\it International Journal of Human-computer Studies}, vol. 74, no. C, pp. 14-31. Feb. 2015.
        
    \bibitem{DeterdingGamefulness}
        S. Deterding, D. Dixon, R. Khaled, y L. Nacke.
        $"$From game design elements to gamefulness: defining gamification $"$, en {\it Proc. of the 15th
        International Academic MindTrek Conference: Envisioning Future Media Environments} New York, NY, USA, 2011, pp. 9-15. 
        % http://www.rolandhubscher.org/courses/hf765/readings/Deterding_2011.pdf
        
        
% ====================================================
%     GAMIFICATION IN THE EDUCATION
% ====================================================
        
    \bibitem{Aldemir}
        T. Aldemir, B. Celik y G. Kaplan, 
        ``A qualitative investigation of student perceptions of game elements in a gamified course'',
        {\it Comput. Hum. Behav.}, vol. 78, 2018. pp. 235-254.
        % https://daneshyari.com/article/preview/4937006.pdf
        
    \bibitem{BrullFinlayson}
        Brull, S. y S. Finlayson, 
        {\it Importance of Gamification in Increasing Learning},
        doi: 10.3928/00220124-20160715-09,
        J. Contin. Educ. Nursing, 47(8), 372-375 (2016)
        % https://pdfs.semanticscholar.org/fe6b/c81592d63cab8821a95e39a274ed594ff232.pdf
    
    \bibitem{ChuHung}
        Chu, C. y C. H. Hung, 
        {\it Effects of the Digital Game-Development Approach on Elementary School Students’
        Learning Motivation, Problem Solving, and Learning Achievement}, doi: 10.4018/ijdet.2015010105,
        International Journal of Distance Education Technologies (IJDET), 13 (1), 87-102 (2015)
        % https://pdfs.semanticscholar.org/57fd/fe3cea521f0490f263256fe38ed48f85afa8.pdf
        
    \bibitem{UPIICSA}
        I. Hernández-Horta, A. Monroy-Reza y M. Jiménez-García,   
        ``Aprendizaje mediante Juegos basados en Principios de Gamificación en Instituciones de Educación Superior'',
        {\it Formación universitaria}, vol. 11, no. 5, pp. 31-40, 2018.
        % https://scielo.conicyt.cl/pdf/formuniv/v11n5/0718-5006-formuniv-11-05-31.pdf
        
    \bibitem{Admiraal}
        Admiraal, W., Huizenga. J., Heemskerk I., Kuiper, E., Volman, M. y Dam G.t.
        {\it Gender-inclusive game-based learning in secondary education},
        doi: 10.1080/13603116.2014, Int. J. Incl. Educ, 18(11), 1208-1218 (2014)
        % https://www.researchgate.net/publication/265606414_Gender-inclusive_game-based_learning_in_secondary_education
        

% ====================================================
%     PROBLEMATICA
% ====================================================
    
    \bibitem{DropOut}
        Park, J. (2007).
        {\it ``Factors related to learner dropout in online learning''}. In Nafukho, F. M., Chermack, T. H., \& Graham, C. M.
        (Eds.)Proceedings of the 2007 Academy of Human Resource Development Annual Conference (pp. 25-1-25-8). Indianapolis, IN:AHRD.
        % https://files.eric.ed.gov/fulltext/ED504556.pdf
    
    \bibitem{GamInE-Learning}
        Strmečki, D., Bernik, A., Radošević, D. (2015).
        {\it ``Gamification in e-learning: introducing gamified design elements into e-learning systems''}.
        Journal of Computer Science, 11(12), 1108-1117.
        % https://www.researchgate.net/publication/303430512_Gamification_in_E-Learning_Introducing_Gamified_Design_Elements_into_E-Learning_Systems/download

    \bibitem{FrameWorkForTheWin}
        Werbach, Kevin y Dan Hunter.
        {\it ``For the Win: How Game Thinking Can Revolutionize Your Business.''}
        Harrisburg: Wharton Digital Press, 2012.
        % https://ieeexplore.ieee.org/stamp/stamp.jsp?tp=&arnumber=8586766

    \bibitem{Wood-Reiners}
        Wood, L. C. Reiners, T. (2015).
        {\it ``Gamification''}. In M. Khosrow-Pour (Ed.), Encyclopedia of Information Science and Technology (3rd ed., pp. 3039-3047).
        Hershey, PA: Information Science Reference. DOI:10.4018/978-1-4666-5888-2.ch297
        % https://www.researchgate.net/publication/265337179_Gamification

        
% ====================================================
%     PROBLEMATICA
% ====================================================

    %\bibitem{coherencia}
        %M. García-Iruela and R. Hijón-Neira, 
        %\textit{``Proposal of a management interface for gamified environments in Moodle''}
        %International Symposium on Computers in Education (SIIE), Jerez, 2018, p. 1.
        % https://ieeexplore.ieee.org/document/8586766
        
    \bibitem{Lee}
        Lee, J., \& Hammer, J. (2011). 
        {\it Gamification in education: What, how, why bother?}
        Academic Exchange Quarterly, 15(2), p. 146.
        % https://www.researchgate.net/publication/258697764_Gamification_in_Education_What_How_Why_Bother
    
    \bibitem{dropoutOnline}
        Park, J \& Choi, H. (2009). {\it ``Factors Influencing Adult Learners' Decision to Drop Out or Persist in Online Learning. Educational Technology \& Society. 12. 207-217''}. (4), 207-217.
        % https://www.researchgate.net/publication/220374458_Factors_Influencing_Adult_Learners'_Decision_to_Drop_Out_or_Persist_in_Online_Learning

    \bibitem{mappingStudy}
        Dicheva, D., Dichev, C., Agre, G., \& Angelova, G. (2015). {\it``Gamification in Education: A Systematic Mapping Study. Educational Technology \& Society.''} 18. 75-88.
        % https://www.researchgate.net/publication/270273830_Gamification_in_Education_A_Systematic_Mapping_Study
        
    \bibitem{libro1}
        Yu-Kai Chou. (2016). 
        \textit{Actionable Gamification.}
        Malpitas, CA, USA: Octalysis Media.
\begin{comment}
    
    \bibitem{arte1}
        \textit{Moodle Plugin,}
        Febrero 2019, [online] Disponible: 
        \url{https://moodle.org/plugins/?q=gamification}
    
\end{comment}

    \bibitem{PagDuolingo}
    Duolingo. \textit{Página principal de Duolingo} [online] Disponible: \url{https://www.duolingo.com/}

    \bibitem{PagMoodle}
    Moodle. \textit{Página principal de Moodle en español} [online] Disponible:
    \url{https://moodle.org/?lang=es}
    
    \bibitem{PagDocebo}
    Docebo. \textit{Página principal de Docebo en español} [online] Disponible:
    \url{https://www.docebo.com/es/}
    
    \bibitem{PagSAPLitmos}
    SAP. \textit{Página principal de SAP Litmos} [online] Disponible:
    \url{https://www.litmos.com/}

    \bibitem{PagATutor}
    ATutor. \textit{Página principal de ATutor} [online] Disponible:
    \url{https://atutor.github.io/}

    \bibitem{PagALEKS}
    Mac Graw Hill, ALEKS Corporation. \textit{Página principal de ALEKS} [online] Disponible:    
    \url{https://www.aleks.com/}

    \bibitem{PagUdemy}
    Udemy, Inc. \textit{Página principal de Udemy} [online]  Disponible:
    \url{https://www.udemy.com/}
 
    \bibitem{PagTalentLMS}
    Epignosis. \textit{Página principal de TalentLMS} [online]  Disponible:
    \url{https://es.talentlms.com/}



    \bibitem{arte2}
        \textit{Levelup!,}
        Febrero 2019, [online] Disponible:  
        \url{https://levelup.branchup.tech/}
    
    \bibitem{arte3}
        \textit{Ranking block,}
        Febrero 2019, [online] Disponible:  
        \url{https://moodle.org/plugins/block_ranking}
    
    \bibitem{arte4}
        \textit{Game,}
        Febrero 2019, [online] Disponible:  
        \url{https://moodle.org/plugins/mod_game}
        
    \bibitem{arte5}
        \textit{Quizventure,}
        Febrero 2019, [online] Disponible:  
        \url{https://moodle.org/plugins/mod_quizgame}
        
    \bibitem{arte6}
        \textit{Stash,}
        Febrero 2019, [online] Disponible:  
        \url{https://moodle.org/plugins/block_stash}
    
    \bibitem{arte7}
        \textit{Mootivated,}
        Febrero 2019, [online] Disponible:  
        \url{https://moodle.org/plugins/local_mootivated}
    
    \bibitem{arte8}
        \textit{UNEDTrivial,}
        Febrero 2019, [online] Disponible:  
        \url{https://moodle.org/plugins/mod_unedtrivial}
    
    \bibitem{arte9}
        \textit{Stamp collection,}
        Febrero 2019, [online] Disponible:  
        \url{https://moodle.org/plugins/mod_stampcoll}
    
    \bibitem{arte10}
        \textit{Exabis Games,}
        Febrero 2019, [online] Disponible:  
        \url{https://moodle.org/plugins/mod_exagames}
    
    \bibitem{arte11}
        \textit{Badge Ladder,}
        Febrero 2019, [online] Disponible:  
        \url{https://moodle.org/plugins/local_bs_badge_ladder}


    
    \bibitem{scrum1}
        K. Schwaber y J. Sutherland, “The Scrum Guide,” p. 3, Noviembre, 2017. [En linea], Disponible:
        https://www.scrumguides.org/docs/scrumguide/v2017/2017-Scrum-Guide-US.pdf [Accedido Marzo. 28, 2019].
        
    \bibitem{scrum2}
        K. Schwaber y J. Sutherland, “The Scrum Guide,” p. 15, Noviembre, 2017. [En linea], Disponible:
        https://www.scrumguides.org/docs/scrumguide/v2017/2017-Scrum-Guide-US.pdf [Accedido Marzo. 28, 2019].
        
    %%%%%%%%%%%%%%%%%%%%%%%%%%%%%%%%%%%%%%%%%%%%%%%%%%%%%%%%
    %%%%%%%%%%%%%%%%%%%% Referencias relacionadas con conceptos de videojuegos
    %%%%%%%%%%%%%%%%%%%%%%%%%%%%%%%%%%%%%%%%%%%%%%%%%%%%%%%%
\begin{comment}
    \bibitem{conceptoDJLogros}
        Blair, Lucas \& Bowers, Clint \& Cannon-Bowers, Janis \& Gonzalez-Holland, Emily. {\it Understanding the Role of Achievements in Game-Based Learning. International Journal of Serious Games.} , December 2016 
        % https://www.researchgate.net/publication/311960305_Understanding_the_Role_of_Achievements_in_Game-Based_Learning
    
\end{comment}    
    
    \bibitem{conveptosVJNiveles}
        Rogers S. , \textit{Level Up! The guide to great videogame design.}. 1ra edición. Reino Unido : John Wiley \& Sons, 2010. 
        % Ref: http://gen.lib.rus.ec/book/index.php?md5=903CA60564E90813EE2A5DCC22547D46
        
    %%%%%%%%%%%%%%%%%%%%%%%%%%%%%%%%%%%%%%%%%%%%%%%%%%%%%%%%
    %%%%%%%%%%%%%%%%%%%% Referencias relacionadas a Base de datos
    %%%%%%%%%%%%%%%%%%%%%%%%%%%%%%%%%%%%%%%%%%%%%%%%%%%%%%%%
    \bibitem{libroBaseDeDatosEspaniol}
        A. Silberschatz, H. F. Korth, S. Sudarshan. \textit{Fundamentos de Diseño de Bases de Datos}, Cuarta Edición. España, Madrid: McGraw Hill/Interamericana de España,  2007.
        
    \bibitem{libroMatematicasDiscretas}
        Johnsonbaugh R. , \textit{Matemáticas discretas}. Sexta edición. Pearson Prentice Hall, 2005.
        %https://catedras.facet.unt.edu.ar/lad/wp-content/uploads/sites/93/2018/04/Matem%C3%A1ticas-Discretas-6edi-Johnsonbaugh.pdf
        
    \bibitem{libro2}
        Yu-Kai Chou.
        \textit{Actionable Gamification.}
        Malpitas, CA, USA: Octalysis Media. 2016. 
    
    
    \bibitem{libro3}
        Yu-Kai Chou.
        En The Eight core drive.
        \textit{Actionable Gamification.}
        Malpitas, CA, USA: Octalysis Media. 2016. 
    
    \bibitem{libro25}
        Mario Herger. (2014). 
        \textit{Enterprise Gamification.} 
        San Bernandino, CA, USA: EGC Media.
    
        
    \bibitem{LMS_1}
        Ouadoud, M., Chkouri, M. Y., \& Nejjari, A. (2018). {\it''Learning management system and the underlying learning theories: towards a new modeling of an LMS''}. International Journal of Information Science and Technology, 2(1), 25-33.
        
    \bibitem{LMS_2}
        Kasim, N. N. M., \& Khalid, F. (2016). {\it''Choosing the right learning management system (LMS) for the higher education institution context: a systematic review''}. International Journal of Emerging Technologies in Learning (iJET), 11(06), 55-61.
    
    \bibitem{LMS_3}
        Nawang, N. B., \& Darus, M. Y. B. (2012). {\it''Evaluation of an open source learning management system: Claroline''}. Procedia-Social and Behavioral Sciences, 67, 416-426.

    \bibitem{PagGnosisConnect}
    Infopro Learning, Inc. \textit{Página principal de GnosisConnect} [online] Disponible:
    \url{https://www.gnosisconnect.com/}
    
    
    \bibitem{aboutMoodle}
        Moodle. {\it Acerca de Moodle}, 2019. [Online]. Disponible en: \url{https://docs.moodle.org/all/es/Acerca_de_Moodle}. Consultado el 15 de Abril 2019.
        % https://docs.moodle.org/all/es/Acerca_de_Moodle
    
    \bibitem{aboutMoodle19}
        Moodle. {\it 19 / Acerca de Moodle}, 2015. [Online]. Disponible en: \url{https://docs.moodle.org/all/es/19/Acerca_de_Moodle}. Consultado el 15 de Abril 2019.
        % https://docs.moodle.org/all/es/19/Acerca_de_Moodle
    
    \bibitem{moodleHistorial}
        Moodle. {\it dev/Historial de Versiones}, 2019. [Online]. Disponible en: \url{https://docs.moodle.org/all/es/dev/Historia_de_las_versiones}. Consultado el 15 de Abril 2019.
        % https://docs.moodle.org/all/es/dev/Historia_de_las_versiones

    \bibitem{moodleArch}
        Moodle. {\it Moodle architecture}, 2018. [Online]. Disponible en: \url{https://docs.moodle.org/dev/Moodle_architecture}. Consultado el 15 de Abril 2019.
        % https://docs.moodle.org/dev/Moodle_architecture
    
    \bibitem{moodleComponets}
        Moodle. {\it Communication Between Components}, 2017. [Online]. Disponible en: \url{https://docs.moodle.org/dev/Communication_Between_Components}. Consultado el 15 de Abril 2019.
        % https://docs.moodle.org/dev/Communication_Between_Components
    
    \bibitem{moodleCoreAPIs}
        Moodle. {\it Core APIs}, 2019. [Online]. Disponible en: \url{https://docs.moodle.org/dev/Core_APIs}. Consultado el 15 de Abril 2019.
        % https://docs.moodle.org/dev/Core_APIs
    
    \bibitem{moodlePluginfiles}
        Moodle. {\it Plugin files}, 2018. [Online]. Disponible en: \url{https://docs.moodle.org/dev/Plugin_files}. Consultado el 15 de Abril 2019.
        % https://docs.moodle.org/dev/Plugin_files

    \bibitem{moodleInstall}
        Moodle. {\it Installing Moodle}, 2018. [Online]. Disponible en: \url{https://docs.moodle.org/35/en/Installing_Moodle}. Consultado el 15 de Abril 2019.
        % https://docs.moodle.org/35/en/Installing_Moodle
    \bibitem{moodleReleaseNotes}
    Moodle. (2018, noviembre 6) \textit{Notas de Moodle}. [online] Disponible:
    \url{https://docs.moodle.org/dev/Moodle_3.5_release_notes}
    
    
    \bibitem{NetBeans}
        NetBeans. {\it NetBeans IDE Features. NetBeans IDE - The Smarter and Faster Way to Code}, 2016. [Online]. Disponible en: \url{https://netbeans.org/features/}. Consultado el 16 de Abril 2019.
        % https://netbeans.org/features/
    
    \bibitem{moodleNetbeans}
        Moodle. {\it Setting Up NetBeans}, 2017. [Online]. Disponible en: \url{https://docs.moodle.org/dev/Setting_up_Netbeans}. Consultado el 16 de Abril 2019.
        % https://docs.moodle.org/dev/Setting_up_Netbeans
    
    \bibitem{PHPStorm}
        JetBrains. {\it PHPStorm Features}, 2019. [Online]. Disponible en: \url{https://www.jetbrains.com/phpstorm/features/}. Consultado el 16 de Abril 2019.
        % https://www.jetbrains.com/phpstorm/features/
    
    \bibitem{moodlePHPStorm}
        Moodle. {\it Setting Up PHPStorm}, 2018. [Online]. Disponible en: \url{https://docs.moodle.org/dev/Setting_up_PhpStorm}. Consultado el 16 de Abril 2019.
        % https://docs.moodle.org/dev/Setting_up_PhpStorm
    
    

    \bibitem{TiposDeUsuario}
        R. Bartle. Hearts, 
        $"$clubs, diamonds, spades: Players who suit MUDs.$"$
        \textit{Journal of MUD research, vol. 1, no 1, p. 19,} 1996.    
    
    
    
    
    
    \bibitem{defAcoplamiento}
        Roger S. Pressman, Ph.D., \textit{Ingeniería del software. Un enfoque práctico}. 7ma edición. México, D. F. : The McGraw-Hill, 2010. 
        %[X] http://cotana.informatica.edu.bo/downloads/ld-Ingenieria.de.software.enfoque.practico.7ed.Pressman.PDF
    
    %%%%%%%%%%%%%%%%%%%%%%%%%%%%%%%%%%%%%%%%%%%%%%%%%%%%%%%%
    %%%%%%%%%%%%%%%%%%%% Referencias relacionadas a páginas de Moodle
    %%%%%%%%%%%%%%%%%%%%%%%%%%%%%%%%%%%%%%%%%%%%%%%%%%%%%%%%
    \bibitem{moodlePautasBD1}
         Moodle  (2018, Octubre 1).\textit{ Definición de la esctructura XML usando XMLDB Editor}.  [Online]. Disponible: \url{ https://docs.moodle.org/dev/XMLDB_defining_an_XML_structure}   
        %[Z] https://docs.moodle.org/dev/XMLDB_defining_an_XML_structure (1 October 2018)
     
     
    \bibitem{moodlePautasBD2}
          Moodle (2017, Mayo 9). \textit{Pautas para la base de datos.} [Online]. Disponible: \url{https://docs.moodle.org/dev/Database}
        %[Y] https://docs.moodle.org/dev/Database (9 May 2017)
    
    \bibitem{moodleTiposBD}
        Moodle  (2017, Diciembre 8). \textit{Tipos de datos del XMLDB Editor.} [Online]. Disponible: \url{https://docs.moodle.org/dev/XMLDB_column_types}
         %[W] https://docs.moodle.org/dev/XMLDB_column_types
  
    \bibitem{libroBaseDeDatosIngles}
        Davidson L., \textit{''Profesional SQL Server 2000 Database Desing''}, Primera edición. USA: Wrox Press, 2001
  
    \bibitem{libroBaseDeDatosInglesCuarteEnAdelante}
        Jan L., Harrington. \textit{''Relational databse design and implementation''}, Tercera edición. USA: Morgan Kaufmann Publishers is an imprint of Elsevier. 
   
\begin{comment}
    
     
    \bibitem{objetivo2}
        B. Marín, J. Frez, J. Cruz-Lemus, and M. Genero. 2018.
        \textit{An Empirical Investigation on the Benefits of Gamification in Programming Courses.} ACM Trans. Comput. Educ. 19, 1, Article 4 (November 2018), 22 pages.
    
    \bibitem{arte12}
        M. García-Iruela and R. Hijón-Neira, 
        \textit{"Proposal of a management interface for gamified environments in Moodle,"} 2018 International Symposium on Computers in Education (SIIE), Jerez, 2018, p. 2.
        % https://ieeexplore.ieee.org/document/8586766
        
    \bibitem{octa1}
        Matos, P. F. (2018). 
        \textit{Gamification–The power of motivation using Octalysis Framework} (Bachelor's thesis).
        
    \bibitem{Octalysis}
        Y. Chou, 
        \textit{Actionable Gamification.}
        Malpitas, CA, USA: Octalysis Media, 2016.
        
    \bibitem{octa2}
        Dicheva, D., Dichev, C., Agre, G., \& Angelova, G. (2015). 
        \textit{Gamification in Education: A Systematic Mapping Study.} 
        Educational Technology \& Society, 18 (3).
    
    \bibitem{octa3}
        Sanchez-Gordón, M. L., Colomo-Palacios, R., \& Herranz, E. (2016, September). 
        \textit{Gamification and human factors in quality management systems: mapping from octalysis framework to ISO 10018.} 
        In European Conference on Software Process Improvement (pp. 234-241). Springer, Cham.
    
    \bibitem{octa4}
        Ewais, S., \& Alluhaidan, A. (2015). 
        \textit{Classification of stress management mHealth apps based on octalysis framework.}
        
    \bibitem{octa5}
        Cabrera, W. R. R., \& Pech, S. H. Q. 
        \textit{Aprendizaje en un Ambiente Virtual a Través de la Gamificación: Una experiencia en una Universidad en México.}
        Tecnologías y Aprendizaje, 508.
        
    
    
    %%%%%%%%%%%%%%%%%%%%%%%%%%%%%%%%%%%%%%%%%%%%%%%%%%%%%%%%
    %%%%%%%%%%%%%%%%%%%% Referencias relacionadas a la ingeniería de software
    %%%%%%%%%%%%%%%%%%%%%%%%%%%%%%%%%%%%%%%%%%%%%%%%%%%%%%%%
    
    
    
\end{comment}   
\end{thebibliography}
