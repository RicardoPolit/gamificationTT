\chapter{Conclusiones}

\subsection*{David Flores Casanova}

     Realizando un proyecto de una magnitud semajante al TT se requiere de una buena planeación, además de una buena comunicación entre todos los integrantes que participan en él (sinodales, directores, alumnos y profesor de seguimiento).\\
    
    \noindent Es muy importante llevar un registro de las actividades que se hacen diariamente para no perder el seguimiento del trabajo realizado en el trayecto del mismo. La curva de aprendizaje de adaptarse y entender un nuevo sistema no debería ser tomada a la ligera.\\
    
    \noindent Es importante documentar todas las decisiones tomadas del proyecto a pesar de que parezcan triviales. 
    
\subsection*{Ricardo Naranjo Polit}    
    
    Se necesita de una gran experiencia para que un proyecto tenga cero retrasos, y más aún uno del tamaño del trabajo terminal. Es cierto que el proyecto no ha salido como fue planeado desde el principio, pero esto me deja aprendizajes muy importantes para el campo laboral.\\
    
    \noindent El hecho de que el trabajo terminal tenga un tema como la educación me ha hecho replantearme los grandes retos que esta conlleva, las oportunidades en las que se puede mejorar y lo que podemos hacer las generaciones de ahora para ayudar a su mejora.
    
\clearpage
    
\subsection*{Daniel Isaí Ortega Zúñiga}

    Considero que uno de los aspectos más importantes y que es muy fácil descuidar, es la visibilidad del estado en el cual se encuentra el proyecto, El utilizar un marco de trabajo que define iteraciones nos permitió detectar fallos iteración a iteración, mejorando cada vez aspectos del proyecto y de comunicación entre los miembros del equipo, así como los aspectos técnicos del proyecto. Como último punto puedo resaltar que es demasiado importante tener una comunicación activa con cada uno de los miembros interesados en el proyecto, incluyendo a los sinodales.