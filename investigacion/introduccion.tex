\chapter{Introducción}
\label{ch:introduccion}

Este documento tiene la finalidad de establecer formalmente la documentación del trabajo terminal {\numeroTT} que tiene como nombre {\bf\tituloTT}.

\section{Organización del contenido}

Este capítulo, \hyperrefx{ch:introduccion}, tiene como propósito presentar la gamificación, incluyendo sus antecedentes, su uso en la educación y los distintos marcos de trabajo que fueron investigados. En lo referente a la definición del proyecto, se detalla el problema, nuestra propuesta de solución a la misma, el objetivo principal, estado del arte, y los alcances y limitaciones que tiene el trabajo terminal.\\

\noindent La {\pf Parte I: Investigación} contiene a los siguientes capítulos:%

    \begin{itemize}
        \item \hyperrefx{ch:marcoTeorico}, establece el soporte conceptual y documental, 
        especifica los marcos de trabajo usados y además contiene la elección de la 
        plataforma sobre la cual se trabajará.
        
        \item \hyperrefx{ch:curvaAprendizaje}, presenta los resultados obtenidos durante 
        las pruebas de concepto, los problemas encontrados y las soluciones o alternativas 
        propuestas a dichos problemas.
    
        \item \hyperrefx{ch:modAlcance}, especifica a los actores y requerimientos funcionales
        y no funcionales identificados, tambien presenta el diseño modular de la propuesta
        de solución planteada en este capítulo.
        
        \item \hyperrefx{ch:modDominio}, contiene el esquema relacional de la base de 
        datos, contemplando todos los módulos, la relación con las relaciones del core de 
        moodle y la especificación de los atributos.
    \end{itemize}

\noindent Posteriormente a la investigación se decidió destinar una parte de este documento a cada módulo identificado. Los módulos son \nameref{mod:exp}, \nameref{mod:recomp}, \nameref{mod:financ}, \nameref{mod:pers}, \nameref{mod:comp} y \nameref{mod:seguim}. Cada parte contiene el análisis, diseño y pruebas del módulo correspondiente.\\
\clearpage



\section{Antecedentes}
\label{sec:antecedentes}

La idea de utilizar mecánicas de juegos para resolver problemas y atraer distintas audiencias ha sido utilizada a lo largo del tiempo. Por ejemplo, el ejército ha estado usando juegos y simulaciones para el entrenamiento de los soldados, y Estados Unidos ha sido un pionero en el uso de videojuegos en diferentes instalaciones militares \cite{GamByDesign}.


\subsection{Definición de Gamificación}

La Gamificación es un anglicismo proveniente del término en ingles {\it ``Gamification''}, la palabra más cercana en el lenguaje español para referirse a la Gamificación es {\it ''ludificación''}. Sin embargo, ninguna de las dos palabras se encuentra definida en el Diccionario del español de México o en el diccionario de la Real Academia de la lengua Española. A continuación se presentan las definiciones propuestas por distintos autores.

    \begin{itemize}

        \item Gartner define la gamificación como {\it ``El uso de la mecánica del juego y 
        el diseño de la experiencia para involucrar y motivar digitalmente a las personas para 
        que logren sus objetivos''} \cite{Burke}
        
        \item Según Kapp, la {\it`` Gamificacion está utilizando la mecánica basada en juegos, 
        la estética y el pensamiento de juego para involucrar a las personas, motivar la acción, 
        promover el aprendizaje y resolver problemas''} \cite{Kapp} % (2012)
        
        \item Zichermann y Cunningham definen Gamificación como {\it ``El proceso del 
        pensamiento del juego y la mecánica del juego para involucrar a los usuarios y resolver 
        problemas''} \cite{GamByDesign} %(p. 14) % (2011) 
        
        \item Huotari y Hamari proporcionan una definición desde una perspectiva de marketing,
        {\it ``Gamificación es un proceso de mejora de un servicio con posibilidades para 
        experiencias de juego para respaldar la creación de valor''} \cite{Huotari}  %(p. 20). (2012)
        
        \item Deterding, Khaled, Nacke y Dixon, mencionan que la {\it ``Gamificación
        es el uso de elementos presentes en el diseño de juegos en contextos distintos a
        los juegos''} \cite{DeterdingDefinition} %(p. 2). (2011) 
    
    \end{itemize}

\noindent A pesar de las propuestas realizadas, todavía no existe una definición que sea ampliamente aceptada o que esté establecida formalmente \cite{Seaborn}. Durante el desarrollo de este trabajo terminal nos referimos a \hyperlink{tGamificacion}{Gamificación} como ``el uso de mecánicas de juegos en un entorno no lúdico''.


\subsubsection{Inicios de la Gamificación}

El término ``Gamificación'' se originó en la industria de los medios digitales. El primer uso documentado se remonta a 2008, pero el término no tuvo una adopción generalizada antes de la segunda mitad de 2010. Actualmente se siguen introduciendo términos nuevos para referirse a la Gamificación como ``juegos de productividad'', ``entretenimiento de vigilancia'', ``Funware'', ``diseño lúdico'', ``juegos de comportamiento'', ``capa de juego'' o ``juego aplicado''. Sin embargo, el término ``Gamificación'' se ha institucionalizado como el término general \cite{DeterdingGamefulness}.\\
    
\noindent Muchos investigadores creen que la Gamificación tiene el potencial de motivar y activar comportamientos específicos al mismo tiempo que fomenta la lealtad a la experiencia Gamificada. Además, tiene el potencial de hacer las actividades no lúdicas más divertidas, motivar a las personas a realizar tareas y mantenerlas constantemente. \cite{Aldemir}
\clearpage


\subsubsection{Gamificación en la educación}

En la educación, la Gamificación se ha visto como una solución potencial para problemas de participación y motivación en entornos educativos, ya que incorpora una amplia gama de enfoques para la enseñanza y el aprendizaje. La Gamificación educativa utiliza sistemas de reglas similares a los juegos, experiencias de los jugadores y roles culturales con el propósito de moldear el comportamiento del aprendiz \cite{Aldemir}.\\
    
\noindent De acuerdo con Brull y Finlayson, la Gamificación permite que los alumnos participen y creen una comunidad de aprendizaje, donde puedan experimentar emociones como frustración, asombro, misterio y diversión, mismas que permiten crear una conexión personal con el juego educativo y con otros compañeros, disfrutando de la libertad de experimentar y fallar en un entorno agradable \cite{BrullFinlayson}.\\

\noindent Existe evidencia de que los alumnos involucrados en entornos con Gamificación mejoran su aprendizaje e incrementan su motivación y compromiso \cite{ChuHung}; por ejemplo, en Estados Unidos se han aplicado elementos de Gamificación en niveles de primaria y preparatoria, los cuales han propiciado un incremento en la capacidad retención de los alumnos y en el compromiso por parte de los mismos \cite{BrullFinlayson}; la Gamificación también ha sido aplicada con éxito en niveles de secundaria ayudando a mejorar las calificaciones en las pruebas o exámenes de los alumnos involucrados \cite{UPIICSA},\cite{Admiraal}.\\

\noindent{\bf\color{red} INSERTAR TABLA DE RESULTADOS}
%\addtable{definicion}{tbl:mappingStudy}{content}{caption}

\clearpage

\section{Problema}
\label{sec:problematica}

Los sistemas de aprendizaje en línea tienen como reto el retener al estudiante a lo largo de los cursos \cite{DropOut}, y la Gamificación se centra en incrementar la motivación, experiencia y compromiso del usuario, razón por la cual, ha comenzado a ser implementada en los sistemas de aprendizaje en línea. Los resultados muestran que los alumnos inscritos en una versión de un curso con Gamificación obtienen mejores resultados de aprendizaje. \cite{GamInE-Learning}\\

\noindent Sin embargo, el realizar una correcta implementación de gamificación requiere dos tipos de habilidades, el diseño basado en juegos y el entendimiento de las técnicas del entorno bajo el cual se deseá implementar \cite[p. 7]{FrameWorkForTheWin}. Si alguno de estos puntos no es analizado y diseñado propiamente, la implementación de la gamificación fracasará y en el peor de los casos ocasionará resultados negativos. \cite{Wood-Reiners}.\\

\noindent Muchas de las propuestas de gamificación que se pueden encontrar fueron diseñadas pensando en un entorno específico y funcionan en dicho entorno, pero existe la posibilidad de que no funcionen adecuadamente en un entorno diferente o simplemente cuando se traten de utilizar junto con otras propuestas.\\

    \noindent La razon por la cual definimos el problema de la siguiente manera:

    \begin{quote}
    \colorbox{blue!05}{\parbox{\dimexpr\linewidth-2\fboxsep}{\strut%
    %
        Los sistemas de aprendizaje en linea no proporcionan un entorno
        de trabajo donde las funcionalidades (propias o extendidas) dedicadas
        a la gamificación sean lo suficientemente flexibles para brindar un
        mayor soporte a los objetivos del curso.
    %
    \strut}}%
    \end{quote}
    

\section{Propuesta de Solución}
\label{sec:propuesta}

    \noindent Como propuesta de solución ante el problema anteriormente definido, se pretende:
    
    \begin{quote}
    \colorbox{blue!05}{\parbox{\dimexpr\linewidth-2\fboxsep}{\strut%
    %
        Desarrollar componentes que permitan implementar Gamificación dentro
        de una plataforma de aprendizaje en linea, tomando como referencia 
        distintos marcos de trabajo que nos guíen en el diseño e implementación
        de elementos de Gamificación configurables para que se adapten a las 
        necesidades del administrador de la plataforma, los docentes/profesores y 
        los alumnos.
    %
    \strut}}%
    \end{quote}
    
\noindent Como se mencionó anteriormente, el implementar Gamificación requiere tanto habilidades de diseño basado en juegos como conocimiento específico acerca del entorno sobre el cual se trabajará. Para el diseño basado en juegos, se utilizarán como guía distintos marcos de trabajo para el diseño de los elementos de gamificación que desarrollaremos.\\

\noindent Por otra parte, se les proporcionará a los docentes/profesores la flexibilidad suficiente para que ellos puedan organizar el contenido del curso de la forma en que ellos considerén, tambien se les proporcionara en lo mayor posible que puedan elegir los elementos que quieren incluir en su curso de acuerdo con las restricciones y recomendaciones a nivel plataforma en torno a la gamificación.
    
\clearpage


\section{Justificación}
\label{sec:justificacion}

    % -> PROFESORES BUSQUEN NUEVAS ALTERNATIVAS
    % -> GAMIFICACION
    % -> LMS + GAMIFICACION
    % -> INFRAESTRUCTURA + INSTRUCCION
    % -> SOPORTE RESTRINGIDO
    % -> PROYECTO DE INGENIERÍA DE SOFTWARE

La escolarización tradicional es percibida como ineficaz y aburrida por muchos estudiantes. Si bien los maestros buscan continuamente nuevos enfoques de enseñanza, una gran parte de los maestros está de acuerdo en que las escuelas de hoy enfrentan problemas importantes entorno a la motivación y el compromiso de los estudiantes. \cite{objetivo1}.\\
    % DAN: Considero el párrafo muy simple

\noindent Estos problemas están relacionados directamente con el principal reto que los sistemas de aprendizaje en linea afrontan: {\it''el evitar la deserción de los estudiantes a lo largo de los cursos''} \cite{objetivo1}. Existen dos tipos de factores que afectan la decisión de los estudiantes de mantenerse o desertar en sistemas de aprendizaje en linea. Los {\it Factores externos}, relacionados con el trabajo, la salud y las relaciones personales, y los {\it Factores internos} relacionados con la integración social y académica, usabilidad y la motivación \cite{DropOut}, \cite{dropoutOnline}.\\
    
    % Existe una correlación entre los distintos tipos de factores, la presencia de un factor interno puede afectar (positivamente o negativamente) uno o más factores externos, y viceversa. Por ejemplo, cuando los alumnos tienen una gran carga de trabajo y poco tiempo para estudiar, es más probable que abandonen un curso cuando no pueden obtener retroalimentación de sus compañeros o evaluadores.\\ % Si se está utilizando el diseño y la tecnología adecuados, es probable que algunos problemas externos se mitiguen. Por lo tanto, la relación entre los factores internos y los factores externos se expresa como inter-correlación en lugar de como una influencia unilateral. \cite{droupoutOnline}
    
\noindent Muchos maestros buscan continuamente nuevos enfoques de enseñanza los cuales les permitan satisfacer o reducir los problemas de motivación, compromiso y deserción \cite{objetivo1}. Un enfoque capaz de proveer una solución potencial es utilizar la Gamificación ya que permite que los estudiantes mejoren su aprendizaje e incrementen su motivación y compromiso, lo cual brinda soporte a reducir los factores internos que conducen a la deserción \cite{dropoutOnline}.\\
    % DAN: La linea anterior podría se comentada
    
\noindent El considerar la Gamificación como principal enfoque para combatir los problemas de motivación, compromiso y deserción en sistemas de aprendizaje en linea: implica la introducción de marcos de trabajo adecuados, además de una infraestructura tecnológica capaz de soportar las funcionalidades a añadir para su implementación \cite{mappingStudy}.\\
    
\noindent Sin embargo, la falta de los componentes tecnológicos es uno de los principales obstáculos encontrados, por lo cual es necesario el desarrollo de nuevos componentes de software que puedan apoyar de manera eficiente la implementación de la Gamificación en diversos contextos educativos, lo que contribuiría a una adopción a mayor escala, así como a la investigación sobre la viabilidad y eficacia de la Gamificación en la educación \cite{mappingStudy}.\\
    
%\noindent El hecho de implementar Gamificación en un sistema de aprendizaje implica considerar la infraestructura tecnológica y un marco de instrucción apropiado. Sin embargo, los sistemas de gestión de cursos de hoy en día todavía ofrecen soporte restringido para cursos de gamificación. \cite{mappingStudy}\\
    % DAN: El anterior párrafo puede descomentarse (Dan don't recommend it)
    
    %%%%%%%%%%%%%%%%%%%%%%%%%%%%%%%%%%%%%%%%%%%%%%%%%%%%%%%%%%%%%%%%%
    %HABLAR ACERCA DE PORQUE ES COHERENTE LO QUE VAMOS A DESARROLLAR
    %%%%%%%%%%%%%%%%%%%%%%%%%%%%%%%%%%%%%%%%%%%%%%%%%%%%%%%%%%%%%%%%%

\noindent Como se menciona en el problema, los sistemas de aprendizaje en linea contemplados en la investigación realizada no proporcionan un entorno de trabajo en donde las funcionalidades (propias, añadidas o extendidas) dedicadas a la Gamificación trabajen de forma coherente.\\
    
\noindent Por ello se decidió desarrollar componentes de software coherentes que extiendan la funcionalidad de un sistema de aprendizaje en linea, tomando como referencia los 6 pasos propuestos por el marco de trabajo {\it For The Win} \cite{FrameWorkForTheWin}, lo cual da como resultado que los componentes trabajen de forma coherente.
    
\begin{comment}

    % \noindent Los modelos propuestos para introducir Gamificación en los sistemas de aprendizaje en linea pueden están relacionados con los proyectos de diseño de software. Por lo tanto, es necesario seguir todas las fases de desarrollo de sistemas: análisis, diseño, desarrollo, implementación y evaluación. [Y]\\
    % https://www.researchgate.net/publication/303430512_Gamification_in_E-Learning_Introducing_Gamified_Design_Elements_into_E-Learning_Systems
    
    % CONSIDER MOVING TO ESTADO DEL ARTE
    % The  majority  of  the  papers  report  encouraging  results  from  the  experiments,  including  significantly  higher engagement of students in forums, projects and other learning activities ((Anderson et al., 2014),(Caton \& Greenhill, 2013), (Akpolat \& Slany, 2014)), increased attendance, participation, and material downloads (Barata, Gama, Jorge, \& Gonçalves,  2013), positive effect on the quantity of students’ contributions/answers, without a corresponding reduction  in  their  quality (Denny,  2013)]; increasedpercentage  of  passing  students  and  participation  in  voluntary activities and challenging assignments(Iosup \& Epema, 2014), and minimizing the  gap between the lowest and the top  graders (Barata,  Gama,  Jorge,  \&  Gonçalves,  2013).  Hakulinen  et  al. (2014)concludethat  achievement  badges can be  used to affect  the  behavior of students even  when the  badges have  no impact on the  grading.The  papers of this group also report that students considered the gamified instancesto be more motivating, interesting and easier to learn as compared to other courses((Mak, 2013), (Barata et al., 2013), (de Byl \& Hooper, 2013), (Mitchell, Danino, \& May, 2013),(Leong \& Yanjie, 2011)).  [X]
\end{comment}

\clearpage
\section{Objetivos}
\label{sec:objetivos}

\noindent El objetivo de este trabajo terminal es el siguiente:

    \begin{quote}
    \colorbox{blue!05}{\parbox{\dimexpr\linewidth-2\fboxsep}{\strut
    %
        Crear una herramienta que permita implementar los principios
        de Gamificación dentro de una plataforma web de aprendizaje.
    %
    \strut}}
    \end{quote}

\noindent La herramienta que crearemos, estará compuesta por un conjunto de elementos de gamificación los cuales sean configurables con la finalidad de brindar un mayor soporte a las necesidades del administrador de la plataforma, profesores y alumnos, como se mencionó anteriormente en la propuesta de solución. Tambien, para un diseño apropiado de los elementos de gamificación se tomará como guía distintos marcos de trabajo de gamificación.\\

Los objetivos específicos son los siguientes:

    \begin{quote}
    %\colorbox{blue!05}{\parbox{\dimexpr\linewidth-2\fboxsep}{\strut%
    \begin{itemize}%
    %
        \item Elección de la forma de trabajo sobre la cual se
              desarrollará el trabajo terminal.
              
        \item Elección de los marcos de trabajo que se utilizarán como
              guía para el diseño e implementación de elementos de gamificación.
        
        \item Elección de la plataforma de aprendizaje en linea sobre
              la cual se desarrollaran los elementos.
              
        \item Diseño modular del sistema.
        
        \item Diseño, Desarrollo, implementación y pruebas de cada uno de
              los módulos que se planteen.
    %
    \end{itemize}
    %\strut}}
    \end{quote}

    

\section{Estado del Arte}
\label{sec:estadoArte}

   % El estado del arte se compone de 2 partes. Una que compara las soluciones con gamificación en sistemas dedicados al aprendizaje y la otra que compara las soluciones con gamificación en plugins de Moodle.
    
Con nuestra investigación encontramos varios sistemas dedicados al aprendizaje que cuentan con gamificación. A continuación se presenta una tabla que indica cómo es que dichos sistemas cuentan con gamificación.
   
    \def\aux{90}
    \addtable{|c|c|c|c|c|c|c|c|c|}{table:LMS_GMFC}{& %
        
        \rotatebox[origin=c]{\aux}{Duolingo   \cite{PagDuolingo}}  &
        \rotatebox[origin=c]{\aux}{Moodle     \cite{PagMoodle}}    &
        \rotatebox[origin=c]{\aux}{Docebo     \cite{PagDocebo}}    &
        \rotatebox[origin=c]{\aux}{SAP Litmos \cite{PagSAPLitmos}} &
        \rotatebox[origin=c]{\aux}{ATutor     \cite{PagATutor}}    & 
        \rotatebox[origin=c]{\aux}{ALEKS      \cite{PagALEKS}}     &
        \rotatebox[origin=c]{\aux}{Udemy      \cite{PagUdemy}}     &
        \rotatebox[origin=c]{\aux}{TalentLMS  \cite{PagTalentLMS}} \\\hline 
        
        Propia    & X & X &   & X &   & X & X & X \\\hline
        Extendida &   & X & X & X & X &   &   &   \\\hline
        
    }{Implementación de gamificación}
       
\noindent Al decir \textbf{Propia} en el cuadro \ref{table:LMS_GMFC} nos referimos a que el sistema gestor de aprendizaje ya tiene integrado en su funcionalidad la implementación  de gamificación. Y al decir \textbf{Extendida} nos referimos a que existen componentes externos (plugins) que implementan gamificación.\\
    
\noindent A continuación se describen los sistemas en el cuadro \ref{table:LMS_GMFC} y los elementos de gamificación con los que cuentan, de acuerdo con los marcos de referencia \nameref{sec:ForTheWin} y \nameref{sec:octalysis}.
\clearpage

    
\begin{multicols*}{2}    
\subsection*{Duolingo}
    
Duolingo es un sistema de aprendizaje dedicado a los idiomas, es un servicio web que
te brinda la posibilidad de crearte una cuenta y seleccionar entre 9 idiomas para aprender,
los cuales son: Inglés, guaraní, francés, alemán, catalán, espartano, italiano, portugués y ruso.\\

\noindent Duolingo divide un idioma en secciones y cada sección contiene sub-secciones,
que a su vez contienen unidades que se dividen en 5 niveles cada una. Al inicio Duolingo
solo te permite empezar una unidad.
    
\noindent Al completar el primer nivel de todas las unidades de una sub-sección,
Duolingo te permite avanzar a la siguiente sub-sección. Y para poder acceder a la siguiente
sección Duolingo te pide que completes una cierta cantidad de niveles de unidades.\\
    
    \noindent Duolingo cuenta con varios módulos que están orientados a la gamificación, se utilizaron
    los elementos de juego definidos por el marco de trabajo ``For the Win'' para formar la siguiente lista:
    
    \begin{itemize}
        \item {\bf Logros:} Cuenta con un sistema de logros o en este caso ``insignias''
            que están divididas en 3 niveles, y cada vez que alcanzas un nivel se 
            desbloquea una estrella que se muestra en el icono del logro.
            
        \item {\bf Desbloqueo de contenido:} Al dividir el contenido de la forma
            anteriormente explicada, Duolingo permite visualizar tu progreso viendo
            la cantidad de unidades completadas y desbloqueadas.
            
        \item {\bf Puntos y niveles de experiencia:} Cada que completas un nivel de una
        unidad se te otorgan puntos de experiencia y esto te permite subir de nivel.
        Cabe aclarar que la experiencia y el nivel están relacionados con el idioma,
        esto quiere decir que puedo ser nivel 10 en inglés pero también ser nivel 1 en francés.
        
        \item {\bf Tablas de líderes:} Si agregas a alguien como tu amigo en Duolingo ambos 
        podrán ver su progreso semanal, mensual y total. El resultado de que el sistema
         los compara genera la tabla de líderes. 
         
        \item {\bf Misiones:} Duolingo permite que te pongas una meta diaria y una meta semanal.
        
    \end{itemize}



\subsection*{Docebo}
    
Docebo es un servicio web que se enfoca en la creación de dominios donde se brinda
un sistema gestor de aprendizaje, es decir, que uno pueda tener su página en línea
donde pueda crear y gestionar sus cursos y los alumnos puedan entrar a tomarlos.\\
    
\noindent Docebo no cuenta con gamificación de raíz, sino que se necesita instalar
plugins que se desarrollan con la API de Docebo, dichos plugins hasta el momento 
solo cuentan con:\\

    \begin{itemize}
        \item {\bf Logros:} Se cuenta con un sistema de logros,
        que se desbloquean si la persona cumple con sus condiciones.
    \end{itemize}
    
    
    
\subsection*{SAP Litmos}
    
SAP Litmos es un sistema que te permite crear cursos para tu equipo de trabajo,
así como delegar tareas y ver el progreso de las mismas. Esta orientado a
fortalecer el capital humano de una empresa.\\
    
    \noindent SAP Litmos cuenta con 3 módulos de gamificación, los cuales son:
    
    \begin{itemize}
        \item {\bf Insignias:} A diferencia que con los logros, estos
        no son otorgados cuando se cumple una cierta condición, sino 
        que el administrador crea una insignia y se le otorga a un usuario.
        
        \item {\bf Equipos: } Debido a que está orientado al capital humano
         de una empresa, uno puede crear equipos que sean por área de la
          empresa y así ver si las áreas están cumpliendo con sus tareas.
          
        \item {\bf Tablas de líderes y puntos: } SAP Litmos te muestra una
         gráfica de que tanto han avanzado los usuarios en un cierto curso
         o en sus tareas. Esto mediante una gráfica y asignación de puntos.
         
    \end{itemize}


\clearpage
\subsection*{ATutor}

ATutor es un un sitema gestor de aprendizaje de software libre. Para poder
utilizarlo se necesita tener un servidor web y montar dicho código en el servidor.\\
    
    \noindent ATutor no cuenta con gamificación de raíz,
    pero cuenta con un plugin llamado \textbf{GameMe} que agrega:
    
    \begin{itemize}
        \item {\bf Logros:} Dichos logros son estáticos y se
        desbloquean cuando se un usuario cumple las condiciones.
        
        \item {\bf Puntos y niveles de experiencia:} Hay definidos 10
        niveles de experiencia y cada que ocurre un evento que tenga
        que ver con un usuario, se le otorga experiencia.
        
    \end{itemize}



\subsection*{ALEKS}

ALEKS es un servicio web que ofrece un sistema gestor de aprendizaje
que adapta el contenido al usuario utilizando inteligencia artificial.
Esto lo mantienen controlado utilizando únicamente ciertos tipos de cursos.\\
    
    \noindent ALEKS cuenta con gamificación de raíz, los elementos con los que cuenta son:
    
    \begin{itemize}
        \item {\bf Progresión:} El fuerte de ALEKS es utilizar la inteligencia
        artificial y algoritmos de predicción así que tiene un montón de datos del
        usuario que aprovecha desplegándolos en gráficos que muestran el progreso
        en diversos temas de un curso, así como el porcentaje del curso que se ha
        tomado, dominado o que falta por revisar. Cabe destacar que un profesor puede
        ver los gráficos de cada alumno, pero los alumnos no pueden ver el de los demás.
        
    \end{itemize}
    
\vfill\null
\columnbreak
\subsection*{Udemy}

Udemy es un servicio web que te permite tomar cursos y/o subir tus cursos.
El formato de los cursos es siempre un video. Cuanta con muchos temas
gracias a que cualquiera puede crear su curso.\\

    \noindent Usando como referencia al marco de trabajo octalysis,
    Udemy cuenta con los siguientes principios de gamificación:
    
    \begin{itemize}
        \item {\bf \principioIII} Debido a que cualquiera puede subir
        sus cursos y recibir retroalimentación de los que lo tomaron,
        se cumple este principio, pero dico principio está orientado
        hacia los creadores de cursos.
        
    \end{itemize}
    
    
    
\subsection*{TalentLMS}

TalentLMS es un servicio web que se enfoca en la creación de dominios donde se
brinda un sistema gestor de aprendizaje, es decir, que uno pueda tener su página
en línea donde pueda crear y gestionar sus cursos y los alumnos puedan entrar a tomarlos.\\
    
    \noindent TalentLMS cuenta con gamificación de raíz,
    y los elementos de juego con los que cuenta, son:
    
    \begin{itemize} 
    
        \item {\bf Logros:} Se cuenta con un sistema de logros o en este caso
        ``insignias'' que están divididas en 8 niveles, y cada vez que alcanzas
        un nivel se desbloquea la insignia en su color correspondiente.
        
        \item {\bf Puntos y niveles de experiencia:} Cada que ocurre un
        determinado evento que tenga que ver con un usuario, se le otorga experiencia.
        
        \item {\bf Tablas de líderes y puntos:} TalentLMS muestra la
        tabla de líderes por categoría de curso, esto a nivel ''plataforma''.
        
    \end{itemize}
    
\end{multicols*}



\clearpage
\subsection{Moodle}
   
En la sección del marco teórico se especifica Moodle más a fondo y el cómo se desarrollan plugins para el mismo. Es por eso que nos limitamos a hacer comparativas de los elementos de gamificación con los que cuentan los diferentes plugins en la tabla \ref{tbl:pluginscreated}

%\noindent En la tabla \ref{table:pluginComp} se comparan los diferentes plugins que existen en el sistema Moodle.
   
%Antes de poder presentar que otras soluciones existen, se necesita establecer un contexto. Es por ello que se tiene que definir en que sistema gestor de aprendizaje se trabajará en este trabajo terminal.  Existen varios sistemas gestores de aprendizaje disponibles para su uso actualmente, sin embargo, para determinar cuál se usará a lo largo de este trabajo terminal se realizó la siguiente tabla comparativa.

%Ya sabiendo que se utilizará Moodle, podemos fijarnos en las soluciones existentes, que en este caso serían los componentes (traducción de plugins del inglés) que cuentan con gamificación. La tabla \ref{table:1} compara de acuerdo a sus elementos los diferentes  componentes que implementan gamficación en la plataforma Moodle \cite{arte1}.
 

    \addtable{|l|c|c|c|c|c|c|c|c|c|c|}{tbl:pluginscreated}{%
        Elementos/Plugins &
        \rotatebox[origin=c]{\aux}{LevelUp!         \cite{arte2}}  &
        \rotatebox[origin=c]{\aux}{Ranking block    \cite{arte3}}  &
        \rotatebox[origin=c]{\aux}{Game             \cite{arte4}}  &
        \rotatebox[origin=c]{\aux}{Quizventure      \cite{arte5}}  &
        \rotatebox[origin=c]{\aux}{Stash            \cite{arte6}}  &
        \rotatebox[origin=c]{\aux}{Mootivated       \cite{arte7}}  &
        \rotatebox[origin=c]{\aux}{UNEDrivial       \cite{arte8}}  &
        \rotatebox[origin=c]{\aux}{Stamp collection \cite{arte9}}  &
        \rotatebox[origin=c]{\aux}{Exabis games     \cite{arte10}} &
        \rotatebox[origin=c]{\aux}{Badge leader     \cite{arte11}} \\\hline
        
        Competencias            &   &   & X & X &   &   & X &   & X &   \\\hline
        Niveles                 & X &   &   &   &   &   &   &   &   &   \\\hline
        Desbloqueo de contenido & X &   &   &   & X &   &   &   &   &   \\\hline
        Logros                  & X &   &   &   &   &   & X &   &   & X \\\hline
        Esquema financiero      &   &   &   &   &   &   &   &   &   &   \\\hline
        Cajas de botín          &   &   &   &   &   &   &   &   &   &   \\\hline
        Puntos                  & X & X &   &   &   &   &   &   & X &   \\\hline
        Tienda                  &   &   &   &   &   &   &   &   &   &   \\\hline
        Tabla lideres           & X &   &   &   &   &   & X & X &   & X \\\hline
        Barra de progreso       & X &   &   &   &   &   &   &   &   &   \\\hline
        
    }{Tabla de comparación de componentes externos (plugins) en Moodle}


\section{Alcances y Limitaciones}
\label{sec:alcancesLimitaciones}

\subsection{Alcances}
\label{subsec:alcances}
%General

Es importante recalcar que se utiliza el marco de trabajo Scrum \cite{scrum1}, por lo cual se cuenta con un Product Backlog. Según la guía de Scrum \cite{scrum2} el Product Backlog es un documento en el que se encuentran los requerimientos del usuario y es el que se consulta al inicio de cada sprint para obtener los objetivo del sprint en cuestión. También se menciona que este documento siempre se encuentra en constante cambio, agregando, modificando o hasta eliminando requerimientos, por lo cual el alcance del proyecto podría llegar a variar en función de la evolución de éste.

    %A continuación se mencionan los principios a los cuales se decidió dar soporte:
     
        %\begin{quote}
        %\begin{enumerate}
            %\item \principioI
            %\item \principioII
            %\item \principioIII
            %\item \principioIV
            %\item \principioV
            %\item \principioVI
            %\item \principioVII
            %\item \principioVIII
        %\end{enumerate}
        %\end{quote}

\subsection{Limitaciones}
\label{subsec:limitaciones}

{\bf\color{red} NO ESTA DENTRO DE NUESTRO ALCANCE EL DISEÑO DE LOS CURSOS, EL TIPO DE CONTENIDO, DE LA MISMA FORMA PROPORCIONAMOS UNA GUIA Y RECOMENDACIONES SOBRE COMO GAMIFICAR UN CURSO, BRINDAMOS LA HERRAMIENTA PERO NO PODEMOS ASEGURAR QUE UN CURSO ESTE CORRECTAMENTE GAMIFICADO O NO DEBIDO A QUE NO ESTAMOS CAPACITADOS PARA ELLO, NUESTRA PRINCIPAL MOTIVACIÓN CONSISTE EN PROPORCIONAR UNA HERRAMIENTA FLEXIBLE QUE APORTE ELEMENTOS DE GAMIFICACIÓN CONFIGURABLES PARA EL ADMINISTRADOR, PROFESORES Y ALUMNOS}



