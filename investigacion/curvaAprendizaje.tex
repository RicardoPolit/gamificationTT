\chapter{Curva de Aprendizaje}
\label{ch:curvaAprendizaje}

\section{Estableciendo el entorno de desarrollo}

 Al final de este documento se incluye como anexo el documento que detalla el desarrollo de las pruebas de concepto. A continuación se muestran los resultados de dicho documento.

Para llevar a cabo desarrollo sobre la plataforma moodle recomienda considerar el uso de un entorno de desarrollo integrado o IDE (Integrated Development Environment), para facilitar las tareas de programación. Las opciones que brinda moodle en su documentación son los IDEs: Eclipse, Netbeans y PHPStorm.\\

\noindent La primer prueba fue realizada con Eclipse, se incluyeron los archivos del directorio de moodle como parte del proyecto, lamentablemente, los enlaces a los demás archivos, y la depuración de código arrojaban errores debido a que había archivos que no podía vincular correctamente. Por lo que Eclipse fue descartado posterior a la prueba.

\subsection{NetBeans}

Netbeans proporciona un buen soporte a PHP, este IDE tiene integración de un sistema de control de versiones, atajos de teclas, lista de funciones, completación de código, soporte para HTML, CSS y Javascript, renombre de archivos/clases instantáneo, búsqueda rápida, entre otros. \cite{NetBeans}, \cite{moodleNetbeans}
% https://netbeans.org/features/
% https://docs.moodle.org/dev/Setting_up_Netbeans

% En la figura \ref{fig:netbeans} se muestra el entorno instalado con el proyecto de ''moodle'' abierto

\subsection{PHPStorm}

PHPStorm es un IDE comercial desarrollado por JetBrains, es considerado uno de los mejores IDE para desarrolladores que trabajan con PHP, tiene características como completación e inspección de código, soporte para PHPUnit, soporte para BeHat, editor de base de datos, depurador, entre otras funcionalidades \cite{PHPStorm},\cite{  moodlePHPStorm}.
% https://www.jetbrains.com/phpstorm/features/
% https://docs.moodle.org/dev/Setting_up_PhpStorm

%En la figura \ref{fig:PHPStorm} se puede ver una captura del IDE con el proyecto con los archivos de Moodle, abierto.\\

\begin{quote}
Finalmente, después de haber realizado la prueba con los  tres IDEs, se eligió a PHPStorm como entorno de desarrollo considerando los siguientes puntos:
    \begin{itemize}
    \item Moodle considera que PHPStorm es uno de los mejores entornos de desarrollo para PHP.
    \item PHPStorm está diseñado desde un inicio para trabajar con PHP, a diferencia de NetBeans que dan soporte a PHP y a otros lenguajes de programación.
    \item PHPStorm tiene soporte para las versiónes más recientes para PHP, mientras que NetBeans soporta actualmente hasta la versión 5.6 de PHP.
    %\item Por recomendación de un profesor de la ESCOM, debido a su experiencia como usuario.
    % \item Consultando a algunos docentes nos recomendaron utilizar PHPStorm.
    %\item Se cuenta con una licencia gratis por pertenecer al IPN/ESCOM.
    \item El equipo de desarrollo en proyectos anterior ha utilizado anteriormente herramientas de JetBrains y se ha tenido una experiencia agradable. % Android Studio
    \end{itemize}
\end{quote}

\section{Desarrollo de las pruebas}

    De los 54 plugins listados en la sección \hyperrefx{subsec:plugins} se decidió priorizar el desarrollo de aquellos tipos de plugins que nos permitieran extender el esquema de base de datos de moodle, y de aquellos que nos permitieran desplegar la información en la interfaz de usuario, razón por la cual se realizaron las pruebas de concepto de los tipos de plugins Database Fields, Database Presets, User Profile Fields, y Blocks.\\
    
    \noindent A continuación el cuadro \ref{tbl:pruebasC} resume el propósito de cada prueba y los resultados obtenidos.
    
    \addtable{|c|p{0.30\textwidth}|p{0.45\textwidth}|}{tbl:pruebasC}{
        {\bf Tipo de\par Plugin} & {\bf Objetivo} & {\bf Resultados} \\\hline
        
        Database Fields &
            Saber si este plugin nos ayudaría a guardar valores en la base de datos, y si fuera capaz, saber la forma en que lo hace. &
            
            Database Fields nos permite, en caso de que requiriéramos crear un nuevo tipo de dato, que puede ser usado mediante el plugin ''Database Presets''. \\\hline
        
        Database Presets & 
            Saber si este plugin nos permite modificar el esquema de la base de datos, y si fuera capaz, saber la forma en que lo hace. &

            Database Presets nos permite crear y guardar datos en la base de datos, las restricciones es que únicamente nos permite definir formularios. El plugin puede ser usado a  nivel plataforma o a nivel curso. \\\hline
            
        User Profile Fields &
            Saber si este plugin nos permite guardar valores relacionados al usuario, en la base de datos. & 
            
            User Profile Fields permite crear nuestro propio tipo de dato y agregarlo como un campo más a los datos que el usuario debe introducir. Al incluirse un plugin de este tipo todos los usuarios de la plataforma podrán editar el dato especificado por este plugin. \\\hline
            
        Blocks &
            Ver cómo desplegar información mediante el uso de este tipo de plugin y asegurar que un mismo block se pueda ver en las vistas principales de la plataforma. &
            
            Los block/blocks pueden ser instanciados más de una vez y están ligados al usuario.\newline  
            Cada plugin puede definir su propio esquema de tablas, atributos e índices.\newline
            Los plugins pueden habilitar/deshabilitar configuraciones generales para el administrador o locales para el usuario. \newline
            Un plugin puede suscribir una clase para capturar los eventos que arroja moodle. \\\hline

    }{Objetivos y resultados de las pruebas de concepto realizadas}
