
% \ucstEnEdicion     Al terminar una revisión/aprobación con observaciones 
%                    y al inicio del CU.
%
% \ucstEnRevision    Al terminar la edición del CU (version += 0.1).
% \ucstEnAprobacion  Al pasar la revision sin observaciones.
% \ucstAprobado      Al ser aprobado por el usuario (version += 1.0)

\begin{UseCase}[%
Autor/Daniel Ortega,%
Version/0.1,%
Estado/\ucstEnEdicion]%
%
{CU-X0Y}{Titulo}{%
%
 Permite al \refElem{aActor} .}

	\UCitem[control]{Revisor}{ Sin asignar }
	\UCitem[control]{Último cambio}{ \today }

 \UCsection{Atributos}

    \UCitem{Actor(es)}{%
        \refElem{aActor}
    }

	\UCitem{Propósito}{%
        ...
	}
	
	\UCitem{Entradas}{\imprimeUC{entrada}}

	\UCitems{Origen}{%
        \Titem Mouse
        \Titem Teclado
	}

	\UCitem{Salidas}{\imprimeUC{salida}}

	\UCitem{Destino}{%
		\refElem{IU-M02a}
	}
	
	\UCitems{Precondiciones}{%
        \Titem ...
	}

	\UCitem{Postcondiciones}{%
        Ninguna
	}

	\UCitem{Reglas de negocio}{%
		Ninguna
	}

	\UCitems{Errores}{%
        \Titem \UCerr{Err1}{%
        % CAUSA
            ...,}{%
        % EFECTO
            ...}
	}

	% \UCitem{Viene de}{% Indicar si el Caso de uso es primario o se extiende de otro. La mayoría se 
					  % extienden de Login.
		% EJEMPLO: \refIdElem{PY-CU1} o Caso de uso primario.
	% 	\TODO Especificar.
	% }	

 \UCsection[design]{Datos de Diseño}

	\UCitems[design]{Casos de Prueba}{%
        \Titem \refElem{CPC-E0Y}
        \Titem \refElem{CPI-E0Y}
	}

 \UCsection[admin]{Datos de Administración de Requerimiento}

	\UCitem[admin]{Observaciones}{%
        Ninguna
	}

\end{UseCase}

\clearpage
\subsubsection{Trayectorias del caso de uso}

\begin{UCtrayectoria}%
%
 \Actor Presiona el botón \IUMenu en la esquina superior izquierda de la pantalla \refElem{IU-M01}
        para abrir el menu de navegación.

 \Actor Selecciona la opción {\it \IUAdminSitio Administración del sitio}

 \Sistema Carga la pantalla \refElem{IU-M02}

\end{UCtrayectoria}


\subsubsection{Puntos de extensión}

\UCExtensionPoint{Nombre del punto de extensión}{%

    El \refElem{aAdministrador} desea/requiere/necesita ....%
%
    }{En el paso \ref{CU-ET-1x} de la trayectoria principal  ...%
%
    }{\refElem{CU-E2-T}}

