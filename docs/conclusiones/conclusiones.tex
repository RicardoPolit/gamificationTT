\chapter{Conclusiones} \label{ch:conclusiones}

\noindent La gamificación ha sido presentada por diversos artículos como una alternativa potencial para resolver problemas de participación
y motivación en entornos educativos. Como se describió  en el estado del arte, existe evidencia de que la gamificación ha mejorado el aprendizaje, motivación
y compromiso en los estudiantes que utilizan entornos que la aplican.\\

\noindent Con este proyecto terminal se dio solución al problema presentado de que los sistemas de aprendizaje en línea como moodle, no proporcionan
un entorno de trabajo donde las funcionalidades dedicadas a la gamificación, sean lo suficientemente flexibles
para brindar un mayor soporte a los objetivos del curso.\\

\noindent Los módulos de este proyecto han sido programados para poder brindar una flexibilidad suficiente 
a los usuarios de moodle para que estos puedan cumplir con sus objetivos del curso, a la vez
que utilizan las técnicas de gamificación, y así obtener los beneficios que ésta aporta a los estudiantes.\\

\noindent La creación de nuevos complementos para moodle es un proceso más complejo de lo que parece debido a
varias razones, entre las cuales, la principal es la falta de documentación, por esta razón antes de comprometer 
un desarrollo que depende de un servicio externo, hay que conocer bien las dependencias de la funcionalidad propuesta y contrastarlas
con los limitaciones de dicho servicio, en este caso moodle. Al inicio de este trabajo se proponía 
realizar un modo historia que consiste en ir desbloqueando nuevas secciones del curso, pero debido 
a que moodle no permite el desbloqueo dinámico de secciones, se decidió que esta idea
no era viable en el tiempo que se disponía.\\


\noindent Para poder implementar las actividades planeadas en los módulos de competencia y de seguimiento se necesitaba el poder diseñar
cómo un usuario vería la actividad (Definir código html y js propio) y que la actividad puediera ser agregada en un curso de moodle. Moodle soporta 
un tipo de complemento que es denominado 'mod' el cual cumple con los requerimientos anteriores.\\

\noindent Lo anterior brinda la posibilidad de diseñar a conveniencia lo que se deseé, sin embargo, aún se requería utilizar las preguntas
ya creadas por el profesor. Esto se solventó utilizando las funcionalidades que tiene moodle respecto a la obtención de las preguntas
de la base datos y la renderización en pantalla de las mismas.\\

\noindent Además de las actividades, se buscó que el administrador de la plataforma pudiera configurar el comportamiento
de los complementos de este proyecto, esto se logró por medio del archivo 'settings.php' que tiene cada complemento. Dicho archivo
define una sección de configuración que puede editar el administrador.\\

\noindent Otro aspecto importante fue el agregar una página que permitiera alojar los módulos de personalización y financiero, 
debido a que ninguna pantalla de moodle está orientada a lo planteado en dichos módulos. Sin embargo, se quería poder seguir en el entorno de trabajo
de moodle en lugar de diseñar esta única página aparte. Esto fue posible gracias a que moodle soporta la modificación
de su cabecera para incluir páginas externas.\\


\noindent Por otra parte con la gráfica \ref{fig:algoritmo-resultados-grafica} podemos concluir que el algoritmo propuesto
 que utiliza el sistema para responder un cuestionario
está diseñado correctamente, ya que se buscaba cumplir 2 aspectos;
el primero es que cada una de las dificultades del sistema ('Fácil, 'Normal', 'Difícil' e 'Imposible') 
obtuviera en promedio una calificación aprobatoria, mientras que en el segundo aspecto
se buscaba que se notara una diferencia entre las dificultades del sistema.\\


\noindent Todo lo anteriormente dicho fue probado en:
\begin{itemize}
    \item Dos navegadores web (Chrome y Firefox).
    \item Dos sistemas operativos (Ubuntu  y Centos).
    \item Dos versiones de moodle (3.5 y 3.7).
\end{itemize}

\noindent Gracias a lo anterior podemos afirmar, que los módulos desarrollados son soportados en un
servidor linux que tenga instalado moodle 3.5 o 3.7 cumpliendo todos sus requisitos, haciendo de este proyecto y sus módulos una herramienta que se puede utilizar
en más de 5 mil servidores a lo largo del territorio mexicano (según estadísticas proporcionadas por stats.moodle.org) y con la posibilidad,
al ser traducido a otros idiomas, de ampliarlo a más de 100 mil servidores a lo largo del mundo.\\

