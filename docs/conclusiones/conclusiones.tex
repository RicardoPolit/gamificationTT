\chapter{Conclusiones} \label{ch:conclusiones}

\noindent La gamificación ha sido presentada como una alternativa potencial para resolver problemas de participación
y motivación en entornos educativos. Existe evidencia de que la gamificación ha mejorado el aprendizaje, motivación
y compromiso en los estudiantes que utilizan entornos que la aplican.\\

\noindent Con este proyecto terminal se dio solución al problema presentado de que los sistemas de aprendizaje en linea no proporcionan
un entorno de trabajo donde las funcionalidades dedicadas a la gamificación, sean lo suficientemente flexibles
para brindar un mayor soporte a los objetivos del curso.\\

\noindent Los módulos propuesto en este proyecto han sido programados para poder brindar una flexibilidad suficiente a los
creadores de cursos en los sistemas de aprendizaje en linea para cumplir con sus objetivos del curso, a la vez
que utilizan las técnicas de gamificación, y así obtener los beneficios que ésta aporta a los estudiantes.\\

\noindent Gracias a la alta configuración y flexibilidad que tienen los módulos realizados, éstos se pueden implementar en cualquier
servidor que cumpla los requisitos de moodle 3.5 y lo tenga instalado, haciendo de este proyecto y sus módulos una herramienta que se puede utilizar
en más de 5 mil servidores a lo largo del territorio mexicano (según estadísticas proporcionadas por stats.moodle.org) y con la posibilidad,
al ser traducido a otros idiomas, de ampliarlo a más de 100 mil servidores a lo largo del mundo.
