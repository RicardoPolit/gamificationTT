
En este capítulo se especifica cómo se implementó el modelo de información utilizando los complementos de moodle
y la especificación del análisis, diseño, desarrollo y pruebas
por cada uno de los módulos del proyecto. \\

\noindent Moodle permite definir una base de datos mediante los archivos db/install.xml, db/upgrade.php y db/install.php de cada complemento.
Esto permite que cada complemento pueda especificar las entidades que estos utilizen por separado. 
Sin embargo, se optó por crear un complemento cuya función sea la implementación del modelo de información
y que en caso de que un complemento necesite usar alguna entidad, este pueda sin problemas acceder a ella 
sin la necesidad de ocuparse de la creación o las versiones que se tengan
del modelo de información.\\

\noindent En otras palabras, la definciión del modelo de información está centralizada en
un complemento del cual todos los otros complementos dependen. Esto se logra gracias al archivo version.php que establece
las dependencias que tiene un complemento de moodle. Este componente que crea la base de datos,
siguiendo el modelo de infomación es denominado gamedlemaster.\\

\noindent El complemento gamedlemaster también se encarga de definir
los eventos con los que los complementos se puedan comunicar entre si.

