
% \ucstEnEdicion     Al terminar una revisión/aprobación con observaciones 
%                    y al inicio del CU.
%
% \ucstEnRevision    Al terminar la edición del CU (version += 0.1).
% \ucstEnAprobacion  Al pasar la revision sin observaciones.
% \ucstAprobado      Al ser aprobado por el usuario (version += 1.0)

\begin{UseCase}[%
Autor/Daniel Ortega,%
Version/0.1,%
Estado/\ucstEnEdicion]%
%
{CU-M01}{Acceder a la administración del sitio}{% TODO; Deberia se Instalar/Actualizar ???
%
 Permite al \refElem{aAdministrador} acceder a la pantalla \refElem{IU-M01} realizar las
 distintas tareas que incluye la administración del sitio de moodle. Esta caso de uso es realizado
 debido a que es requerido para la ejecución de la mayoría de los casos de uso cuyo actor es el
 administrador.}

	\UCitem[control]{Revisor}{ Sin asignar }
	\UCitem[control]{Último cambio}{ 14/OCT/19 }

 \UCsection{Atributos}

    \UCitem{Actor(es)}{%
        \refElem{aAdministrador}
    }

	\UCitem{Propósito}{%
        Permitir al administrador acceder a la administración del moodle que administra.
	}
	
	\UCitem{Entradas}{-}

	\UCitem{Origen}{-}

	\UCitem{Salidas}{-}

	\UCitems{Destino}{%
        \refElem{IU-M01}
	}
	
	\UCitem{Precondiciones}{-}

	\UCitem{Postcondiciones}{-}

	\UCitem{Reglas de negocio}{-}

	\UCitem{Errores}{-}

	% \UCitem{Viene de}{% Indicar si el Caso de uso es primario o se extiende de otro. La mayoría se 
					  % extienden de Login.
		% EJEMPLO: \refIdElem{PY-CU1} o Caso de uso primario.
	% 	\TODO Especificar.
	% }	

 \UCsection[design]{Datos de Diseño}

	\UCitem[design]{Casos de Prueba}{%
        Incluidos en la ejecución de los casos de uso que incluyen a este caso de uso
    }

 \UCsection[admin]{Datos de Administración de Requerimiento}

	\UCitem[admin]{Observaciones}{-}

\end{UseCase}

\subsubsection{Trayectorias del caso de uso}

\begin{UCtrayectoria}%
%
    \Actor Presiona el botón \IUMenu de la pantalla \refElem{IU-M00}
    \Sistema Despliega el menú de navegación lateral.

    \Actor Selecciona la opción {\bf \IUAdminSitio Administración del sitio}
    \Sistema Carga la pantalla \refElem{IU-M01} con la pestaña de administración del
             sitio preseleccionada.

    \Actor Selecciona la pestaña {\bf plugins}
    \Sistema Carga la pantalla \refElem{IU-M01a}

\end{UCtrayectoria}

\subsubsection{Puntos de extensión}

\UCExtensionPoint{Instalación de un plugin}{%

    El \refElem{aAdministrador} desea extender la funcionalidad
    de moodle mediante la instalación de plugins.%
    }{Al inicio la trayectoria principal}{\refElem{CU-E01}}

\UCExtensionPoint{Configuraciones generales del módulo de experiencia}{%

    El \refElem{aAdministrador} desea cambiar las configuraciones
    generales del módulo de experiencia.%
    }{Al inicio la trayectoria principal}{\refElem{CU-E02}}

\UCExtensionPoint{Configuraciones visuales del módulo de experiencia}{%

    El \refElem{aAdministrador} desea establecer las configuraciones
    de la visualización de los niveles del módulo de experiencia.%
    }{Al inicio la trayectoria principal}{\refElem{CU-E02-1}}

\UCExtensionPoint{Configuraciones de comportamiento del módulo de experiencia}{%

    El \refElem{aAdministrador} desea establecer el comportamiento del
    sistema de experiencia que incluye el módulo de experiencia.%
    }{Al inicio la trayectoria principal}{\refElem{CU-E02-2}}

\UCExtensionPoint{Configuraciones de eventos del módulo de experiencia}{%

    El \refElem{aAdministrador} desea establecer la cantidad de experiencia
    que brindarán los eventos que soporta el módulo de experiencia.%
    }{Al inicio la trayectoria principal}{\refElem{CU-E02-3}}

\UCExtensionPoint{Desinstalación de un plugin}{%

    El \refElem{aAdministrador} desea desinstalar un plugin en
    moodle debido a que ya no requiere de las funcionalidades
    que este brinda%
    }{Al inicio la trayectoria principal}{\refElem{CU-E03}}

