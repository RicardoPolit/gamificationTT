
% \ucstEnEdicion     Al terminar una revisión/aprobación con observaciones
%                    y al inicio del CU.
%
% \ucstEnRevision    Al terminar la edición del CU (version += 0.1).
% \ucstEnAprobacion  Al pasar la revision sin observaciones.
% \ucstAprobado      Al ser aprobado por el usuario (version += 1.0)

\begin{UseCase}[%
Autor/David Flores Casanova,%
Version/0.1,%
Estado/\ucstEnRevision]%
%
{CU-P02}{Modificar perfil gamificado}{%
%
    Permite al usuario (Ya sea un \refElem{aProfesor}, un \refElem{aAdministrador} o un \refElem{aEstudiante})
 de moodle seleccionar qué objetos quiere que se muestren en su perfil.
 Este caso de uso es una extensión del caso de uso {\it \refElem{CU-P01}}.}

	\UCitem[control]{Revisor}{ Sin asignar }
	\UCitem[control]{Último cambio}{ 17/NOV/19 }

 \UCsection{Atributos}

    \UCitem{Actor(es)}{%
        \refElem{aProfesor},
        \refElem{aAdministrador},
        \refElem{aEstudiante}
    }

	\UCitems{Propósito}{%
        \Titem El usuario quiere modificar que objetos se mostrarán cuando se visualice su perfil gamificado en las actividades.
	}

	\UCitem{Entradas}{\imprimeUC{entrada}}

	\UCitems{Origen}{%
        \Titem Mouse
	}

	\UCitem{Salidas}{
        \imprimeUC{salida}
        \Titem ''¡Cambios guardados!''%\refElem{MSG-P01}
        }

	\UCitem{Destino}{%
		\refElem{IU-P01}
	}

	\UCitems{Precondiciones}{%
        \Titem El usuario debió de haber ejecutado el \refElem{CU-P01}.
        \Titem El usuario debe tener al menos un objeto desbloqueado.
        \Titem Los objetos seleccionados deben estar desbloqueado para el usuario.
	}

	\UCitems{Postcondiciones}{%
        \Titem Al mostrar el perfil gamificado del usuario se verán las nuevas opciones que seleccionó.
	}

	\UCitem{Reglas de negocio}{\imprimeUC{regla}}

	\UCitems{Errores}{%
	}

 \UCsection[design]{Datos de Diseño}

	\UCitems[design]{Casos de Prueba}{%
        \Titem \refElem{CPC-P02-1}
        \Titem \refElem{CPI-P02-2}
        \Titem \refElem{CPI-P02-3}
	}

 \UCsection[admin]{Datos de Administración de Requerimiento}

	\UCitem[admin]{Observaciones}{}

\end{UseCase}

\subsubsection{Trayectorias del caso de uso}

\begin{UCtrayectoria}%
%
    \Actor Da clic en algún objeto en la pantalla \refElem{IU-P01}.
    \label{CU-P03-eligiendo-objeto}
    \Sistema Carga la previsualización siguiendo la regla de negocio \regla{BR-P01}.
    \Sistema Comprueba que el actor tenga adquirido el objeto. \refTray{A}
    \Sistema Habilita la opción \textbf{Guardar} de la pantalla \refElem{IU-P01}.
    \Actor Ve el resultado en la previsualización. \refTray{B}
    \Actor Presiona el botón \textbf{Guardar}.
    \Sistema Cambia los objetos seleccionados usando el atributo \refElem{tienda-gmdl-objeto-desbloqueado.elegido}.
    \Sistema Muestra el mensaje ''¡Cambios guardados!''%\refElem{MSG-P01}
\end{UCtrayectoria}

\begin{UCtrayectoriaA}%
  {A}{El actor no tiene el objeto desbloqueado }
    \Sistema Inhabilita la opción \textbf{Guardar} de la pantalla \refElem{IU-P01}.
    \item Se regresa al paso \ref{CU-P03-eligiendo-objeto}.

\end{UCtrayectoriaA}


\begin{UCtrayectoriaA}%
  {B}{El actor desea deshacer los cambios }
    \Actor  Presiona el botón \textbf{Deshacer}.
    \Sistema Carga la configuración actual del actor,
        cargando los objetos  (\salida{tienda-gmdl-objeto}) desbloqueados cuyo
        \refElem{tienda-gmdl-objeto-desbloqueado.elegido} sea igual a 1  y lo muestra en pantalla.
    \Sistema Habilita la opción \textbf{Guardar} de la pantalla \refElem{IU-P01}.
    \item Se regresa al paso \ref{CU-P03-eligiendo-objeto}.

\end{UCtrayectoriaA}
