
% \ucstEnEdicion     Al terminar una revisión/aprobación con observaciones
%                    y al inicio del CU.
%
% \ucstEnRevision    Al terminar la edición del CU (version += 0.1).
% \ucstEnAprobacion  Al pasar la revision sin observaciones.
% \ucstAprobado      Al ser aprobado por el usuario (version += 1.0)

\begin{UseCase}[%
Autor/David Flores Casanova,%
Version/0.1,%
Estado/\ucstEnRevision]%
%
{CU-P03}{Eliminar datos del complemento}{%
%
    Si ya no se quiere contar con el complemento de la tienda, se puede desintalar y se borrarán los datos guardados en la base de datos.
    Este caso de uso es una extensión del caso de uso {\it Desinstalar un complemento} que es propio de moodle.}

	\UCitem[control]{Revisor}{ Sin asignar }
	\UCitem[control]{Último cambio}{ 17/NOV/19 }

 \UCsection{Atributos}

    \UCitem{Actor(es)}{%
        \refElem{aAdministrador}.
    }

	\UCitems{Propósito}{%
        \Titem El actor ya no quiere utilizar el complemento de tienda.
	}

	\UCitem{Entradas}{\imprimeUC{entrada}}

	\UCitems{Origen}{%
        \Titem Mouse
	}

	\UCitem{Salidas}{
        \imprimeUC{salida}}

	\UCitem{Destino}{%
		\refElem{IU-P01}
	}

	\UCitems{Precondiciones}{%
        \Titem se debió de haber ejecutado el CU 'Desinstalar complemento'.
	}

	\UCitems{Postcondiciones}{%
        \Titem Los datos guardados relacionados a qué usuario debloqueó qué objeto estarpan eliminados.
        \Titem La opción Perfil gamificado ahora arrojará un error de pagina no encontrada.
	}

	\UCitem{Reglas de negocio}{\imprimeUC{regla}}

	\UCitems{Errores}{%
	}

 \UCsection[design]{Datos de Diseño}

	\UCitems[design]{Casos de Prueba}{%
	}

 \UCsection[admin]{Datos de Administración de Requerimiento}

	\UCitem[admin]{Observaciones}{La postcondición del error 'paǵina no encontrada' se debe a que el caso de uso que agrega la opción 'Perfil gamificado',
    es soportado únicamente por moodle.}

\end{UseCase}

\subsubsection{Trayectorias del caso de uso}

\begin{UCtrayectoria}%
%

    \Sistema se elimina toda la información de la entidad \refElem{tienda-gmdl-objeto-desbloqueado}. 
    
\end{UCtrayectoria}
