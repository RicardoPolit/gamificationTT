
% \ucstEnEdicion     Al terminar una revisión/aprobación con observaciones
%                    y al inicio del CU.
%
% \ucstEnRevision    Al terminar la edición del CU (version += 0.1).
% \ucstEnAprobacion  Al pasar la revision sin observaciones.
% \ucstAprobado      Al ser aprobado por el usuario (version += 1.0)

\begin{UseCase}[%
Autor/David Flores Casanova,%
Version/0.1,%
Estado/\ucstEnRevision]%
%
{CU-P01}{Ver perfil gamificado}{%
%
Permite a un usuario (Ya sea un \refElem{aProfesor}, un \refElem{aAdministrador} o un \refElem{aEstudiante})
 de moodle ver las configuraciones de personalización que tiene activas en su perfil así como las monedas que tiene disponibles.
 La conclusión de la trayectoria principal de esta caso de uso es una precondición para que
 algunos casos de uso del módulo de personalización puedan ejecutarse.\\%
 Este caso de uso es una extensión del caso de uso {\it Iniciar sesión} que es propio de moodle.}

	\UCitem[control]{Revisor}{ Sin asignar }
	\UCitem[control]{Último cambio}{ 17/NOV/19 }

 \UCsection{Atributos}

    \UCitem{Actor(es)}{%
        \refElem{aProfesor},
        \refElem{aAdministrador},
        \refElem{aEstudiante}
    }

	\UCitems{Propósito}{%
        \Titem El usuario quiere agregar un objeto más a su colección, para poder usarlo en la personalización de si perfil.
	}

	\UCitem{Entradas}{\imprimeUC{entrada}}

	\UCitems{Origen}{%
        \Titem Mouse
	}

	\UCitem{Salidas}{
        \imprimeUC{salida}}

	\UCitem{Destino}{%
		\refElem{IU-P01}
	}

	\UCitems{Precondiciones}{%
        \Titem El usuario debió de haber ejecutado el CU {\it Iniciar sesión} que es propio de moodle.
	}

	\UCitems{Postcondiciones}{%

	}

	\UCitem{Reglas de negocio}{\imprimeUC{regla}}

	\UCitems{Errores}{%
	}

 \UCsection[design]{Datos de Diseño}

	\UCitems[design]{Casos de Prueba}{%
        \Titem \refElem{CPC-P01-1}
        \Titem \refElem{CPC-P01-2}
        \Titem \refElem{CPC-P01-3}
	}

 \UCsection[admin]{Datos de Administración de Requerimiento}

	\UCitem[admin]{Observaciones}{}

\end{UseCase}

\subsubsection{Trayectorias del caso de uso}

\begin{UCtrayectoria}%
%
    \Actor Selecciona dando clic a la opción \textbf{Perfil gamificado} de la pantalla \refElem{IU-M09}.
    \Sistema Corrobora que el actor haya iniciado sesión. \refTray{A}
    \Sistema Corrobora que el actor esté registrado en la entidad \refElem{xp-user}.  \refTray{B}
    \Sistema Redirige a la pantalla \refElem{IU-P01}.
    \label{CU-P01-cargar-informacion}
    \Sistema Carga el nombre del actor (\salida{Nombre de usuario}) y lo muestra en pantalla.
    \Sistema Carga la configuración actual del actor,
    cargando los objetos  (\salida{tienda-gmdl-objeto})
    desbloqueados cuyo \refElem{tienda-gmdl-objeto-desbloqueado.elegido} sea igual a 1  y los muestra en pantalla.
    \Sistema Comprueba que el módulo financiero esté desactivado. \refTray{C}
    \Sistema Carga todos los objetos en \refElem{tienda-gmdl-objeto} y los muestra en pantalla.


\end{UCtrayectoria}

\begin{UCtrayectoriaA}[Fin del caso de uso]%
  {A}{El actor aun no ha iniciado sesión}

  \Sistema Cierra la pantalla \refElem{IU-P01} y redirige a la pantalla \refElem{IU-M00b}.

\end{UCtrayectoriaA}

\begin{UCtrayectoriaA}%
{B}{El actor no está registrado en \refElem{xp-user}}

    \Sistema Registra al actor en \refElem{xp-user}.
    \item Se regresa al paso \ref{CU-P01-cargar-informacion}.

\end{UCtrayectoriaA}


\begin{UCtrayectoriaA}[Fin del caso de uso]%
{C}{El módulo financiero está activado}
    \Sistema Carga todos los objetos en \refElem{tienda-gmdl-objeto} y muestra la opción \textbf{'Comprar'} (utilizando el ícono \IUMonedas{})
     en aquellos objetos que el actor
    aun no tiene en \refElem{tienda-gmdl-objeto-desbloqueado}  y los muestra en pantalla.
    \Sistema Carga (\salida{xp-user.monedas-plata}) y lo muestra en pantalla.

\end{UCtrayectoriaA}




\UCExtensionPoint{Modificar perfil gamificado}{%

    El actor desea modificar qué objetos mostrar en su perfil gamificado.
%
    }{Al final de la trayectoria principal del caso de uso.
%
    }{\refElem{CU-P02}}

\UCExtensionPoint{Comprar objeto}{%

    El actor desea agregar un nuevo objeto a su colección..
  %
      }{Al final de la trayectoria alternativa C
  %
      }{\refElem{CU-F03}}
