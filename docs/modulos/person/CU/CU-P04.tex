
% \ucstEnEdicion     Al terminar una revisión/aprobación con observaciones
%                    y al inicio del CU.
%
% \ucstEnRevision    Al terminar la edición del CU (version += 0.1).
% \ucstEnAprobacion  Al pasar la revision sin observaciones.
% \ucstAprobado      Al ser aprobado por el usuario (version += 1.0)

\begin{UseCase}[%
Autor/David Flores Casanova,%
Version/0.1,%
Estado/\ucstEnRevision]%
%
{CU-P04}{Activar complemento}{%
%
    Para poder empezar a utilizar el plugin, se necesita que el actor\refElem{aAdministrador} haga unas modificaciones
    en la configuración de moodle. Esto ya es soportado por moodle, solo se indica qué es lo que debe de ingresar
    el actor.  Este caso de uso es una extensión del caso de uso {\it Acceder a la administración del sistio} propio de moodle.}

	\UCitem[control]{Revisor}{ Sin asignar }
	\UCitem[control]{Último cambio}{ 17/NOV/19 }

 \UCsection{Atributos}

    \UCitem{Actor(es)}{%
        \refElem{aAdministrador}.
    }

	\UCitems{Propósito}{%
        \Titem El actor quiere utilizar las funciones que le brinda el compelemento de 'tienda'.
	}

	\UCitem{Entradas}{\imprimeUC{entrada}}

	\UCitems{Origen}{%
        \Titem Mouse
        \Titem Teclado
	}

	\UCitem{Salidas}{
        \imprimeUC{salida}}

	\UCitem{Destino}{%
		\refElem{IU-P01}
	}

	\UCitems{Precondiciones}{%
        \Titem se debió de haber ejecutado el CU 'Acceder a la administración del sistio'.
	}

	\UCitems{Postcondiciones}{%
        \Titem La cabezera de moodle contará con la opción de 'Perfil gamificado'.
        \Titem El submenú del usuario de moodle contará con la opción de 'Perfil gamificado'.
	}

	\UCitem{Reglas de negocio}{\imprimeUC{regla}}

	\UCitems{Errores}{%
	}

 \UCsection[design]{Datos de Diseño}

	\UCitems[design]{Casos de Prueba}{%
	}

 \UCsection[admin]{Datos de Administración de Requerimiento}

	\UCitem[admin]{Observaciones}{}

\end{UseCase}

\subsubsection{Trayectorias del caso de uso}

\begin{UCtrayectoria}%
%
    \Actor Selecciona la opción \textbf{Apariencia}  y presiona la opción -> 'Ajustes de temas'.
    \Sistema Redirige a la pantalla \refElem{IU-M10}.
    \Actor Ingresa el texto \textbf{Perfil gamificado|/local/gmtienda/perfilgamificado.php} en la caja de texto \textbf{Ítems del menú personalizado}.
    \Actor Ingresa el texto \textbf{Perfil gamificado|/local/gmtienda/perfilgamificado.php} en la caja de texto \textbf{Ítems del menú de usuario}.
    \Actor Presiona el botón \textbf{Guardar cambios}.
    \Sistema Guarda la configuración.
    \Sistema Recarga la pestaña  \refElem{IU-M10}.
    
\end{UCtrayectoria}
