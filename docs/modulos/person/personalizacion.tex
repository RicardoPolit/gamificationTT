
\subsection{Análisis}

 Este apartado contiene el análisis requerido para la elaboración de módulo de personalización,
 contiene la especificación del alcance de este módulo, la descripción de las funcionalidades
 a desarrollar, la reglas de negocio que rigen el comportamiento del módulo, y por último la
 especificación de los casos de uso a los que brinda soporte.



\subsubsection{Reglas de negocio} %========================================================

 En esta sección se especifican todas las reglas de negocio relevantes para el módulo de
 experiencia. Las reglas de negocio que establece moodle son diferenciadas por tener la letra {\it M}
 antecediendo al número consecutivo en su identificador.

    % No se puede tener seleccionado  más de un objeto de un mismo tipo
    % Los objetos solo pueden ser adquiridos una vez
    % En caso de no tener un objeto seleccionado de un cierto tipo, se debe cargar una configuración por defecto
    %\begin{BusinessRule}[%
Autor/Daniel Isai Ortega Zúñiga,%
Version/0.1,%
Estado/edicion]%
%
{BR-E08}{Valores iniciales de experiencia de un curso}

     \BRitem[control]{Revisada por}{Pendiente.}

 \BRsection[control]{Atributos}
    % Clases: \bcCondition, \bcIntegridad, \bcAutorization o \bcDerivation
    % Tipos: \btEnabler, \btTimer o \btExecutive
    % Niveles: \blControlling o \blInfluencing.

    \BRitem[admin]{Clase}{\bcIntegridad}%

    \BRitem[admin]{Tipo}{\btTimer}%

    \BRitem[admin]{Nivel}{\blControlling}

    \BRitem{Descripción}{%
        Cuando un \refElem{xp-course} es creado la \refElem[experiencia total del curso]%
        {xp-scheme-settings.courseXP} de ser dividida uniformemente entre las
        \refElem[secciones del curso gamificado]{xp-course-section}. Si la división del
        total de experiencia entre el número de secciones genera un residuo entonces este
        se deberá agregan a la última sección del curso.
    }

%   \BRitem{Sentencia}{%
%       Si $fecha$
%   }%

    \BRitem{Ejemplo positivo}{\hfill\par%
        \begin{itemize}
        \item ...
        \end{itemize}
    }

    \BRitem{Ejemplo negativo}{\hfill\par%
        \begin{itemize}
        \item ...
        \end{itemize}
    }

 \end{BusinessRule}
 % Valores iniciales de experiencia del curso

    % INPUT: Cursos Igualitarios.
    % INPUT: Otorgar experiencia
    % INPUT: Administración de experiencia en el curso
    \begin{BusinessRule}[%
Autor/David Flores Casanova,%
Version/0.1,%
Estado/edicion]%
%
{BR-P01}{Solo se puede tener un objeto seleccionado de cada tipo de objeto}

     \BRitem[control]{Revisada por}{Pendiente.}

 \BRsection[control]{Atributos}
    % Clases: \bcCondition, \bcIntegridad, \bcAutorization o \bcDerivation
    % Tipos: \btEnabler, \btTimer o \btExecutive
    % Niveles: \blControlling o \blInfluencing.
    
    \BRitem[admin]{Clase}{\bcIntegridad}%
        
    \BRitem[admin]{Tipo}{\btEnabler}%
        
    \BRitem[admin]{Nivel}{\blControlling}
    
    \BRitem{Descripción}{%
       Un usuario solo puede tener un objeto de un mismo tipo seleccionado para mostrarse en el perfil gamificado.
    }

    \BRitem{Ejemplo positivo}{\hfill\par%
        Un usuario tiene seleccionado para su perfil;
        \begin{itemize}
            \item Un objeto de tipo imagen.
            \item Un objeto de tipo estilo de marco.
            \item Un objeto de tipo color de marco.
        \end{itemize}
    }

    \BRitem{Ejemplo negativo}{\hfill\par%
        Un usuario tiene seleccionado para su perfil;
        \begin{itemize}
            \item Tres objetos de tipo imagen.
        \end{itemize}
        }
    
 \end{BusinessRule}

\clearpage

\subsubsection{Mensajes}
  

    \begin{mensaje2}{MSG-P01}{Cambios guardados}{Operación exitosa}
        \item[Redacción:] ¡Cambios guardados!
    \end{mensaje2}
\subsubsection{Casos de uso} % ============================================================

 En este apartado se especifican todos los casos de usos contemplados para el módulo de
 personalización, para cada caso de uso se especifica su tabla de atributos la cual indica que casos
 de prueba deberán ejecutarse correctamente para corroborar la completitud del caso de uso.

\subsubsection*{Diagrama de casos de uso}

 En la figura \ref{personalizacion:usecases} se detalla el diagrama de casos de uso correspondiente al módulo
 de personalización. Los casos de uso de moodle (en color blanco) son modelados como casos de uso
 abstractos, mientras que los casos de uso del módulo de personalización son diferenciados por el
 color azul, en total el desarrollo de este módulo consiste en 17 casos de uso principales.

    \addfigure{0.6}{modulos/person/diagrams/UseCases}{personalizacion:usecases}{%
        Diagrama de casos de uso del módulo de personalización}

 \noindent
 Debido a que los plugins a desarrollar son elementos opcionales para Moodle, solo se puede
 acceder a los casos de uso del módulo de competencia a través de puntos de extensión de los
 casos de uso de moodle. Por otra parte los casos de uso que serán documentados en esta sección
 serán los del módulo de competencia debido a que Moodle proporciona en su página oficial, guías
 e instructivos como documentación de las funcionalidades que brinda.

    % MODULO DE EXPERIENCIA


% \ucstEnEdicion     Al terminar una revisión/aprobación con observaciones
%                    y al inicio del CU.
%
% \ucstEnRevision    Al terminar la edición del CU (version += 0.1).
% \ucstEnAprobacion  Al pasar la revision sin observaciones.
% \ucstAprobado      Al ser aprobado por el usuario (version += 1.0)

\begin{UseCase}[%
Autor/David Flores Casanova,%
Version/0.1,%
Estado/\ucstEnRevision]%
%
{CU-P01}{Ver perfil gamificado}{%
%
Permite a un usuario (Ya sea un \refElem{aProfesor}, un \refElem{aAdministrador} o un \refElem{aEstudiante})
 de moodle ver las configuraciones de personalización que tiene activas en su perfil así como las monedas que tiene disponibles.
 La conclusión de la trayectoria principal de esta caso de uso es una precondición para que
 algunos casos de uso del módulo de personalización puedan ejecutarse.\\%
 Este caso de uso es una extensión del caso de uso {\it Iniciar sesión} que es propio de moodle.}

	\UCitem[control]{Revisor}{ Sin asignar }
	\UCitem[control]{Último cambio}{ 17/NOV/19 }

 \UCsection{Atributos}

    \UCitem{Actor(es)}{%
        \refElem{aProfesor},
        \refElem{aAdministrador},
        \refElem{aEstudiante}
    }

	\UCitems{Propósito}{%
        \Titem El usuario quiere agregar un objeto más a su colección, para poder usarlo en la personalización de si perfil.
	}

	\UCitem{Entradas}{\imprimeUC{entrada}}

	\UCitems{Origen}{%
        \Titem Mouse
	}

	\UCitem{Salidas}{
        \imprimeUC{salida}}

	\UCitem{Destino}{%
		\refElem{IU-P01}
	}

	\UCitems{Precondiciones}{%
        \Titem El usuario debió de haber ejecutado el CU {\it Iniciar sesión} que es propio de moodle.
	}

	\UCitems{Postcondiciones}{%

	}

	\UCitem{Reglas de negocio}{\imprimeUC{regla}}

	\UCitems{Errores}{%
	}

 \UCsection[design]{Datos de Diseño}

	\UCitems[design]{Casos de Prueba}{%
        \Titem \refElem{CPC-P01-1}
        \Titem \refElem{CPC-P01-2}
        \Titem \refElem{CPC-P01-3}
	}

 \UCsection[admin]{Datos de Administración de Requerimiento}

	\UCitem[admin]{Observaciones}{}

\end{UseCase}

\subsubsection{Trayectorias del caso de uso}

\begin{UCtrayectoria}%
%
    \Actor Selecciona dando clic a la opción \textbf{Perfil gamificado} de la pantalla \refElem{IU-M09}.
    \Sistema Corrobora que el actor haya iniciado sesión. \refTray{A}
    \Sistema Corrobora que el actor esté registrado en la entidad \refElem{xp-user}.  \refTray{B}
    \Sistema Redirige a la pantalla \refElem{IU-P01}.
    \label{CU-P01-cargar-informacion}
    \Sistema Carga el nombre del actor (\salida{Nombre de usuario}) y lo muestra en pantalla.
    \Sistema Carga la configuración actual del actor,
    cargando los objetos  (\salida{tienda-gmdl-objeto})
    desbloqueados cuyo \refElem{tienda-gmdl-objeto-desbloqueado.elegido} sea igual a 1  y los muestra en pantalla.
    \Sistema Comprueba que el módulo financiero esté desactivado. \refTray{C}
    \Sistema Carga todos los objetos en \refElem{tienda-gmdl-objeto} y los muestra en pantalla.


\end{UCtrayectoria}

\begin{UCtrayectoriaA}[Fin del caso de uso]%
  {A}{El actor aun no ha iniciado sesión}

  \Sistema Cierra la pantalla \refElem{IU-P01} y redirige a la pantalla \refElem{IU-M00b}.

\end{UCtrayectoriaA}

\begin{UCtrayectoriaA}%
{B}{El actor no está registrado en \refElem{xp-user}}

    \Sistema Registra al actor en \refElem{xp-user}.
    \item Se regresa al paso \ref{CU-P01-cargar-informacion}.

\end{UCtrayectoriaA}


\begin{UCtrayectoriaA}[Fin del caso de uso]%
{C}{El módulo financiero está activado}
    \Sistema Carga todos los objetos en \refElem{tienda-gmdl-objeto} y muestra la opción \textbf{'Comprar'} (utilizando el ícono \IUMonedas{})
     en aquellos objetos que el actor
    aun no tiene en \refElem{tienda-gmdl-objeto-desbloqueado}  y los muestra en pantalla.
    \Sistema Carga (\salida{xp-user.monedas-plata}) y lo muestra en pantalla.

\end{UCtrayectoriaA}




\UCExtensionPoint{Modificar perfil gamificado}{%

    El actor desea modificar qué objetos mostrar en su perfil gamificado.
%
    }{Al final de la trayectoria principal del caso de uso.
%
    }{\refElem{CU-P02}}

\UCExtensionPoint{Comprar objeto}{%

    El actor desea agregar un nuevo objeto a su colección..
  %
      }{Al final de la trayectoria alternativa C
  %
      }{\refElem{CU-F03}}
   % Ver perfil gamificado

% \ucstEnEdicion     Al terminar una revisión/aprobación con observaciones
%                    y al inicio del CU.
%
% \ucstEnRevision    Al terminar la edición del CU (version += 0.1).
% \ucstEnAprobacion  Al pasar la revision sin observaciones.
% \ucstAprobado      Al ser aprobado por el usuario (version += 1.0)

\begin{UseCase}[%
Autor/David Flores Casanova,%
Version/0.1,%
Estado/\ucstEnRevision]%
%
{CU-P02}{Modificar perfil gamificado}{%
%
    Permite al usuario (Ya sea un \refElem{aProfesor}, un \refElem{aAdministrador} o un \refElem{aEstudiante})
 de moodle seleccionar qué objetos quiere que se muestren en su perfil.
 Este caso de uso es una extensión del caso de uso {\it \refElem{CU-P01}}.}

	\UCitem[control]{Revisor}{ Sin asignar }
	\UCitem[control]{Último cambio}{ 17/NOV/19 }

 \UCsection{Atributos}

    \UCitem{Actor(es)}{%
        \refElem{aProfesor},
        \refElem{aAdministrador},
        \refElem{aEstudiante}
    }

	\UCitems{Propósito}{%
        \Titem El usuario quiere modificar que objetos se mostrarán cuando se visualice su perfil gamificado en las actividades.
	}

	\UCitem{Entradas}{\imprimeUC{entrada}}

	\UCitems{Origen}{%
        \Titem Mouse
	}

	\UCitem{Salidas}{
        \imprimeUC{salida}
        \Titem ''¡Cambios guardados!''%\refElem{MSG-P01}
        }

	\UCitem{Destino}{%
		\refElem{IU-P01}
	}

	\UCitems{Precondiciones}{%
        \Titem El usuario debió de haber ejecutado el \refElem{CU-P01}.
        \Titem El usuario debe tener al menos un objeto desbloqueado.
        \Titem Los objetos seleccionados deben estar desbloqueado para el usuario.
	}

	\UCitems{Postcondiciones}{%
        \Titem Al mostrar el perfil gamificado del usuario se verán las nuevas opciones que seleccionó.
	}

	\UCitem{Reglas de negocio}{\imprimeUC{regla}}

	\UCitems{Errores}{%
	}

 \UCsection[design]{Datos de Diseño}

	\UCitems[design]{Casos de Prueba}{%
        \Titem \refElem{CPC-P02-1}
        \Titem \refElem{CPI-P02-2}
        \Titem \refElem{CPI-P02-3}
	}

 \UCsection[admin]{Datos de Administración de Requerimiento}

	\UCitem[admin]{Observaciones}{}

\end{UseCase}

\subsubsection{Trayectorias del caso de uso}

\begin{UCtrayectoria}%
%
    \Actor Da clic en algún objeto en la pantalla \refElem{IU-P01}.
    \label{CU-P03-eligiendo-objeto}
    \Sistema Carga la previsualización siguiendo la regla de negocio \regla{BR-P01}.
    \Sistema Comprueba que el actor tenga adquirido el objeto. \refTray{A}
    \Sistema Habilita la opción \textbf{Guardar} de la pantalla \refElem{IU-P01}.
    \Actor Ve el resultado en la previsualización. \refTray{B}
    \Actor Presiona el botón \textbf{Guardar}.
    \Sistema Cambia los objetos seleccionados usando el atributo \refElem{tienda-gmdl-objeto-desbloqueado.elegido}.
    \Sistema Muestra el mensaje ''¡Cambios guardados!''%\refElem{MSG-P01}
\end{UCtrayectoria}

\begin{UCtrayectoriaA}%
  {A}{El actor no tiene el objeto desbloqueado }
    \Sistema Inhabilita la opción \textbf{Guardar} de la pantalla \refElem{IU-P01}.
    \item Se regresa al paso \ref{CU-P03-eligiendo-objeto}.

\end{UCtrayectoriaA}


\begin{UCtrayectoriaA}%
  {B}{El actor desea deshacer los cambios }
    \Actor  Presiona el botón \textbf{Deshacer}.
    \Sistema Carga la configuración actual del actor,
        cargando los objetos  (\salida{tienda-gmdl-objeto}) desbloqueados cuyo
        \refElem{tienda-gmdl-objeto-desbloqueado.elegido} sea igual a 1  y lo muestra en pantalla.
    \Sistema Habilita la opción \textbf{Guardar} de la pantalla \refElem{IU-P01}.
    \item Se regresa al paso \ref{CU-P03-eligiendo-objeto}.

\end{UCtrayectoriaA}
   % Modificar perfil gamificado

% \ucstEnEdicion     Al terminar una revisión/aprobación con observaciones
%                    y al inicio del CU.
%
% \ucstEnRevision    Al terminar la edición del CU (version += 0.1).
% \ucstEnAprobacion  Al pasar la revision sin observaciones.
% \ucstAprobado      Al ser aprobado por el usuario (version += 1.0)

\begin{UseCase}[%
Autor/David Flores Casanova,%
Version/0.1,%
Estado/\ucstEnRevision]%
%
{CU-P03}{Eliminar datos del complemento}{%
%
    Si ya no se quiere contar con el complemento de la tienda, se puede desinstalar y se borrarán los datos guardados en la base de datos.
    Este caso de uso es una extensión del caso de uso {\it Desinstalar un complemento} que es propio de moodle.}

	\UCitem[control]{Revisor}{ Sin asignar }
	\UCitem[control]{Último cambio}{ 17/NOV/19 }

 \UCsection{Atributos}

    \UCitem{Actor(es)}{%
        \refElem{aAdministrador}.
    }

	\UCitems{Propósito}{%
        \Titem El actor ya no quiere utilizar el complemento de tienda.
	}

	\UCitem{Entradas}{\imprimeUC{entrada}}

	\UCitems{Origen}{%
        \Titem Mouse
	}

	\UCitem{Salidas}{
        \imprimeUC{salida}}

	\UCitem{Destino}{%
		\refElem{IU-P01}
	}

	\UCitems{Precondiciones}{%
        \Titem se debió de haber ejecutado el CU 'Desinstalar complemento'.
	}

	\UCitems{Postcondiciones}{%
        \Titem Los datos guardados relacionados a qué usuario debloqueó qué objeto estarán eliminados.
        \Titem La opción Perfil gamificado ahora arrojará un error de pagina no encontrada.
	}

	\UCitem{Reglas de negocio}{\imprimeUC{regla}}

	\UCitems{Errores}{%
	}

 \UCsection[design]{Datos de Diseño}

	\UCitems[design]{Casos de Prueba}{%
	}

 \UCsection[admin]{Datos de Administración de Requerimiento}

	\UCitem[admin]{Observaciones}{La postcondición del error 'paǵina no encontrada' se debe a que el caso de uso que agrega la opción 'Perfil gamificado',
    es soportado únicamente por moodle.}

\end{UseCase}

\subsubsection{Trayectorias del caso de uso}

\begin{UCtrayectoria}%
%

    \Sistema Elimina toda la información de la entidad \refElem{tienda-gmdl-objeto-desbloqueado}.

\end{UCtrayectoria}
   % Eliminar datos del complemento

% \ucstEnEdicion     Al terminar una revisión/aprobación con observaciones
%                    y al inicio del CU.
%
% \ucstEnRevision    Al terminar la edición del CU (version += 0.1).
% \ucstEnAprobacion  Al pasar la revision sin observaciones.
% \ucstAprobado      Al ser aprobado por el usuario (version += 1.0)

\begin{UseCase}[%
Autor/David Flores Casanova,%
Version/0.1,%
Estado/\ucstEnRevision]%
%
{CU-P04}{Activar complemento}{%
%
    Para poder empezar a utilizar el plugin, se necesita que el actor\refElem{aAdministrador} haga unas modificaciones
    en la configuración de moodle. Esto ya es soportado por moodle, solo se indica qué es lo que debe de ingresar
    el actor.  Este caso de uso es una extensión del caso de uso {\it Acceder a la administración del sistio} propio de moodle.}

	\UCitem[control]{Revisor}{ Sin asignar }
	\UCitem[control]{Último cambio}{ 17/NOV/19 }

 \UCsection{Atributos}

    \UCitem{Actor(es)}{%
        \refElem{aAdministrador}.
    }

	\UCitems{Propósito}{%
        \Titem El actor quiere utilizar las funciones que le brinda el compelemento de 'tienda'.
	}

	\UCitem{Entradas}{\imprimeUC{entrada}}

	\UCitems{Origen}{%
        \Titem Mouse
        \Titem Teclado
	}

	\UCitem{Salidas}{
        \imprimeUC{salida}}

	\UCitem{Destino}{%
		\refElem{IU-P01}
	}

	\UCitems{Precondiciones}{%
        \Titem se debió de haber ejecutado el CU 'Acceder a la administración del sistio'.
	}

	\UCitems{Postcondiciones}{%
        \Titem La cabezera de moodle contará con la opción de 'Perfil gamificado'.
        \Titem El submenú del usuario de moodle contará con la opción de 'Perfil gamificado'.
	}

	\UCitem{Reglas de negocio}{\imprimeUC{regla}}

	\UCitems{Errores}{%
	}

 \UCsection[design]{Datos de Diseño}

	\UCitems[design]{Casos de Prueba}{%
	}

 \UCsection[admin]{Datos de Administración de Requerimiento}

	\UCitem[admin]{Observaciones}{}

\end{UseCase}

\subsubsection{Trayectorias del caso de uso}

\begin{UCtrayectoria}%
%
    \Actor Selecciona la opción \textbf{Apariencia}  y presiona la opción -> 'Ajustes de temas'.
    \Sistema Redirige a la pantalla \refElem{IU-M10}.
    \Actor Ingresa el texto \textbf{Perfil gamificado|/local/gmtienda/perfilgamificado.php} en la caja de texto \textbf{Ítems del menú personalizado}.
    \Actor Ingresa el texto \textbf{Perfil gamificado|/local/gmtienda/perfilgamificado.php} en la caja de texto \textbf{Ítems del menú de usuario}.
    \Actor Presiona el botón \textbf{Guardar cambios}.
    \Sistema Guarda la configuración.
    \Sistema Recarga la pestaña  \refElem{IU-M10}.
    
\end{UCtrayectoria}
   % Activar complemento
\input{modulos/person/CU/CU-P05}   % Desactivar complemento

% =========================================================
\clearpage
\subsection{Diseño}

\subsubsection{Interfaces del módulo de competencia}

    %
\subsubsection{IU-M00b Ingreso a moodle}

    Esta pantalla es de moodle. Esta pantalla brinda acceso al sistema.

    \IUfig{1}{modulos/moodle/IU/Login}{IU-M00b}{Ingreso a moodle}

\subsubsection{Elementos Relevantes}

    \begin{itemize}
    \item {\bf Lorem ipsum}
        ...
    \end{itemize}

\subsubsection{Acciones relevantes}

    \begin{itemize}
    \item {\bf Lorem ipsum}
        ...
    \end{itemize}

\clearpage

    
\subsubsection{IU-M09 Cabezera de moodle}

    \IUfig{1}{modulos/moodle/IU/cabezera_moodle}{IU-M09}{Cabezera de moodle}

\subsubsection{Elementos Relevantes}

    \begin{itemize}
        \item {\bf Nombre de la plataforma}
            En este caso mostrado como 'Gamedle', es el nombre que recibe la plataforma.
        \item {\bf Botón menú lateral}
            Este botón permite mostrar o ocultar el menú lateral de moodle.
        \item {\bf Idioma preferente}
            Esta opción permite cambiar el idioma con el cual el usuario configura cómo se muestran los textos de moodle en las páginas.
        \item {\bf Notificaciones}
            Esta opción permite ver qué notificaciones aún no se han atendido y ver el historial de las notificaciones.
        \item {\bf Mensajes}
            Esta opción permite ver las conversaciones que se tienen con otros usuarios.
        \item {\bf Perfil moodle}
            Esta opción permite ver el usuario de moodle.
    \end{itemize}


\clearpage

    
\subsubsection{IU-M09a Cabezera de moodle con el perfil gamificado}

    Esta pantalla se muestra en lugar de \refElem{IU-M09}, una vez se haya ejecutado el \refElem{CU-P04}.

    \IUfig{1}{modulos/moodle/IU/cabezera_moodle_gmdl}{IU-M09a}{Cabezera de moodle con perfil gamificado}

\subsubsection{Elementos Relevantes}

    \begin{itemize}
        \item {\bf Nombre de la plataforma}
            En este caso mostrado como 'Gamedle', es el nombre que recibe la plataforma.
        \item {\bf Botón menú lateral}
            Este botón permite mostrar o ocultar el menú lateral de moodle.
        \item {\bf Perfil gamificado}
            Este botón permite acceder ala pantalla \refElem{IU-P01}.
        \item {\bf Idioma preferente}
            Esta opción permite cambiar el idioma con el cual el usuario configura cómo se muestran los textos de moodle en las páginas.
        \item {\bf Notificaciones}
            Esta opción permite ver qué notificaciones aún no se han atendido y ver el historial de las notificaciones.
        \item {\bf Mensajes}
            Esta opción permite ver las conversaciones que se tienen con otros usuarios.
        \item {\bf Perfil moodle}
            Esta opción permite ver el usuario de moodle.
    \end{itemize}


\clearpage

    


\subsubsection{IU-M10 Ajustes de tema - Sección submenús}


    \IUfig{1}{modulos/moodle/IU/elementos_importantes_configuracion}{IU-M10}{Ajustes de tema - sección submenús}

\subsubsection{Elementos Relevantes}

    \begin{itemize}
        \item {\bf Ítems del menú personalizado}
            Campo de texto que establece la cantidad de opciones que están alojados en la cabecera de moodle.
        \item {\bf Ítems del menú del usuario}
            Campo de texto que establece la cantidad de opciones que están alojados en el submenú del usuario de moodle.
    \end{itemize}


\clearpage

    
\subsubsection{IU-P01: Perfil gamificado}

 Pantalla que muestra el estado actual del perfil gamificado del usuario.

    \IUfig{1}{modulos/person/IU/perfil_gamificado}{IU-P01}{%
        Pantalla del perfil gamificado sin módulo financiero}

    %\IUfig{1}{modulos/finan/IU/perfil_gamificado}{IU-F01}{%
    %   Pantalla del perfil gamificado con módulo financiero}

\subsubsection{Elementos Relevantes}

    \begin{itemize}
        \item {\bf Previzualización de configuración}
            En esta sección se muestra cómo se vería el perfil, utilizando los objetos seleccionados por el usuario.
        \item {\bf Monedas usuario}
            En esta sección se muestra la cantida de monedas que tiene a su disposición el usuario.
        \item {\bf Objetos}
            Los objetos a disposición del usuario a ser comprados.
    \end{itemize}

\subsubsection{Acciones relevantes}

    \begin{itemize}
        \item {\bf Deshacer}
            Esta opción borra las configuraciones cambiadas por el usuaario y vuelve a las ya precargadas para la pantalla.
        \item {\bf Guardar}
            Esta opción guarda la configuración que se tiene en la previsualización, para ser  mostrada en las otras activiedades gamificadas.
        \item {\bf Comprar}
            Botón de cada objeto indicado por el ícono \IUMonedas{}, permite comparar el objeto correspondiente.
    \end{itemize}

\clearpage





\subsubsection{Diseño de complementos}



A continuación se presenta cómo los submódulos de competencia
se implementan en moodle.\\


\noindent Resumiendo el módulo de competencia tiene 2 actividades establecidas, llamadas;
competencia uno contra uno y competencia uno contra sistema.
Ambas actividades deben aparecer dentro de la lista de actividades de moodle. Para ello
moodle cuenta con un tipo de complemento que se denomina \textbf{'mod'}, este tipo de complemento al ser instalado
en una plataforma de moodle, crea una nueva opción a la lista de actividades.\\

\noindent Tomando en consideración lo anterior y que existe el complemento gamedlemaster, se presenta en la figura \ref{fig:diseno-comp-comp}
los complementos contemplados y las dependencias entre los mismos.


    \addfigure{1}{modulos/comp/diagrams/diseno_complementos}{fig:diseno-comp-comp}{Implementación del modulo de competencia}


Cada complemento en la figura \ref{fig:diseno-comp-comp} está representado con una cadena que sigue el formato 'tipo\_de\_complemento:nombre\_de\_complemento'. Los tipos de complemento son;
\begin{itemize}
    \item \textbf{mod} - Este complemento permite crear una actividad que aparece en la lista de actividades a agregar a un curso.
    \item \textbf{local} -  Este complemento puede ser usado para múltiples propósitos relacionados con la gestión de la información.
    \item \textbf{block} - Este complemento permite desplegar una sección en la mayoría de las páginas de moodle, la cuál puede representar código html.
\end{itemize}

La función de cada uno de los complementos presentados en la figura \ref{fig:diseno-comp-comp} son:


\begin{itemize}
    \item \textbf{gamedlemaster} Definir la base de datos y los eventos a manejar.
    \item \textbf{gmcompcpu} Definir la competencia uno contra sistema.
    \item \textbf{gmcompvs} Definir la competencia uno contra uno.
    \item \textbf{gmcs} Entregar las monedas por ganar cada una de las competencias anteriores.
\end{itemize}

El complemento de tipo  \textbf{'mod'} tiene un requerimiento en su nombre, el cual es; 'El nombre del complemento a instalar debe ser igual a un nombre
de una de las tablas en la base de datos'. Debido a esto y que moodle no soporta nombres de complementos que contengan guiones bajos, el
nombre de la tabla ya no puede llevarlos.\\



\subsection{Pruebas}

