
\begin{BusinessRule}[%
Autor/Ricard Naranjo Polit,%
Version/0.1,%
Estado/revision]%
%
{BR-C02}{Un usuario no puede desafiar a otro con el que tenga un desafío pendiente}
 % El archivo de instalación debe ser un archivo ZIP, el cual debe contener
 % exactamente un directorio que coincida con el nombre del plugin.
     \BRitem[control]{Revisor}{Sin asignar.}

 \BRsection[control]{Atributos}

    \BRitem[admin]{Clase}{\bcCondition}%
    %\BRitem[admin]{Clase}{\bcIntegridad}%
    %\BRitem[admin]{Clase}{\bcAutorizacion}%
    %\BRitem[admin]{Clase}{\bcDerivacion}%

    \BRitem[admin]{Tipo}{\btEnabler}%
    %\BRitem[admin]{Tipo}{\btTimer}%
    %\BRitem[admin]{Tipo}{\btExecutive}%

    \BRitem[admin]{Nivel}{\blControlling}
    %\BRitem[admin]{Nivel}{\blInfluencing}

    \BRitem{Descripción}{%
        Cuando un usuario desafía a un \refElem{aEstudiante} no lo podrá volver a desafiar hasta que el desafiante
        y desafiado terminen hayan completado la competencia.
        % debido a que se el directorio donde se guardará será el directorio
        % para almacenar las imágenes del plugin.
    }

    \BRitem{Ejemplo positivo}{\hfill\par%
        \begin{itemize}
        \item El usuario desafía a un estudiante, los dos terminan la competencia y el usuario vuelve a desafiar al mismo estudiante.

        \end{itemize}
    }

    \BRitem{Ejemplo negativo}{\hfill\par%
        \begin{itemize}
          \item El usuario desafía a un estudiante y no alguno de los dos no termina la competencia,
          el usuario no puede volver a desafiar al mismo estudiante.

        \end{itemize}
    }%

 \end{BusinessRule}
