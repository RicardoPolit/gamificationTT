

\subsubsection{Diseño de complementos}



A continuación se presenta cómo el módulo personalización
se implmenetan en moodle.\\


\noindent Resumiendo el módulo de personalización, se  tiene una pantalla donde se gestiona qué objetos
desea un usuario en su perfil gamificado
y se debe acceder a ella utilizando una opción en la cabezera de moodle.

La cabezera de moodle no tiene un tipo de complemento asociado como se tuvo con las actividades, sin embargo,
 moodle le da soporte dejando editar al \refElem{aAdministrador}
lo que se muestra en el menú y las acciones que realizan.

\noindent Tomando en consideración lo anterior y que existe el complemento gamedlemaster, se presenta en la figura \ref{fig:diseno-comp-person}
los complementos contemplados y las dependencias entre los mismos.


    \addfigure{1}{modulos/person/diagrams/diseno_complementos}{fig:diseno-comp-person}{Implementación del modulo de competencia}


Cada complemento en la figura \ref{fig:diseno-comp-person} está representado con una cadena que sigue el formato 'tipo\_de\_complemento:nombre\_de\_complemento'.
Los tipos de complemento son;
\begin{itemize}
    \item \textbf{local} -  Este complemento moodle lo iterpreta como un comdín, el cual puede ser usado para múltiples propósitos.
    \item \textbf{block} - Este complemento permite desplegar una sección en la mayoría de las páginas de moodle, la cuál puede representar código html.
\end{itemize}

La función de cada uno de los complementos presentados en la figura \ref{fig:diseno-comp-person} son:


\begin{itemize}
    \item \textbf{gamedlemaster} Definir la base de datos y los eventos a manejar.
    \item \textbf{gmtienda} Permitirle al usuario personalizar su perfil gamificado.
\end{itemize}


