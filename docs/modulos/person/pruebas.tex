
A continuación se enlistan los casos de prueba identificados
correspondientes a cada uno de los casos de uso especificados. Los casos
de prueba listados a continuación son de dos tipos, los correctos e
incorrectos identificados por los prefijos CPC y CPI respectivamente.

\subsubsection{\refElem{CU-P01}}

\begin{itemize}
  \TestCase{CPC-P01-1}{Ver perfil con objetos ya elegidos}
  \TestCase{CPC-P01-2}{Ver perfil con valores por defecto}
  \TestCase{CPC-P01-3}{Usuario no registrado como usuario gamificado}
\end{itemize}


\subsubsection{\refElem{CU-P02}}

\begin{itemize}
  \TestCase{CPC-P02-1}{Modificar perfil con objetos desbloqueados}
  \TestCase{CPI-P02-2}{Intentar modificar pefil con objetos no desbloqueados}
  \TestCase{CPI-P02-3}{Intentar modificar perfil con 2 objetos de un mismo tipo}
\end{itemize}


\subsubsection{\refElem{CU-P03}}

\begin{itemize}
  \TestCase{CPC-P03-1}{Eliminar datos del complemento}
\end{itemize}


\subsubsection{\refElem{CU-P04}}

\begin{itemize}
  \TestCase{CPC-P04-1}{Activar complemento}
  \TestCase{CPI-P04-2}{Activar complemento sin tener el complemento de la tienda}
  \TestCase{CPI-P04-3}{Ingresar valores incorrectos}
\end{itemize}


\subsubsection{\refElem{CU-P05}}

\begin{itemize}
  \TestCase{CPC-P05-1}{Desactivar complemento}
  \TestCase{CPI-P05-2}{Desactivar complemento sin tener el complemento de la tienda}
  \TestCase{CPI-P05-3}{Ingresar valores incorrectos}
\end{itemize}

