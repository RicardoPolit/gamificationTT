
\begin{BusinessRule}[%
Autor/Daniel Isai Ortega Zúñiga,%
Version/0.1,%
Estado/revision]%
%
{BR-E03}{Tipos de Incremento}
    \BRitem[control]{Revisor}{Sin asignar.}

 \BRsection[control]{Atributos}
    
    \BRitem[admin]{Clase}{\bcCondition}%
    %\BRitem[admin]{Clase}{\bcIntegridad}%
    %\BRitem[admin]{Clase}{\bcAutorizacion}%
    %\BRitem[admin]{Clase}{\bcDerivacion}%
        
    \BRitem[admin]{Tipo}{\btEnabler}%
    %\BRitem[admin]{Tipo}{\btTimer}%
    %\BRitem[admin]{Tipo}{\btExecutive}%
        
    \BRitem[admin]{Nivel}{\blControlling}
    %\BRitem[admin]{Nivel}{\blInfluencing}
    
    \BRitem{Descripción}{%
    El tipo de incremento en el sistema de experiencia hace referencia a la forma en que
    aumenta la cantidad de experiencia para pasar de un nivel a otro. Si el incremento es
    de tipo {\bf lineal} entonces el \refElem{xp-scheme-settings.incrementValue} debe ser
    un número positivo mayor a cero, por el contrario si el incremento es {\bf porcentual}
    entonces el valor debe ser un número flotante entre 1.0 y 2.0.
    }

    \BRitem{Ejemplo positivo}{\hfill%
        \begin{itemize}
        \item El \refElem{xp-scheme-settings.increment} es {\bf porcentual} y el
              \refElem{xp-scheme-settings.incrementValue} es 1.3.
        \item El \refElem{xp-scheme-settings.increment} es {\bf lineal} y el
              \refElem{xp-scheme-settings.incrementValue} es 1000.
        \end{itemize}
    }

    \BRitem{Ejemplo negativo}{\hfill%
        \begin{itemize}
        \item El \refElem{xp-scheme-settings.increment} es {\bf porcentual} y el
              \refElem{xp-scheme-settings.incrementValue} es 1000.
        \item El \refElem{xp-scheme-settings.increment} es {\bf lineal} y el
              \refElem{xp-scheme-settings.incrementValue} es 1.1.
        \item El \refElem{xp-scheme-settings.increment} es {\bf lineal} y el
              \refElem{xp-scheme-settings.incrementValue} es -1100.
        \end{itemize}
    }% 
    
 \end{BusinessRule}
