
\begin{BusinessRule}[%
Autor/Daniel Isai Ortega Zúñiga,%
Version/0.1,%
Estado/revision]%
%
{BR-E02}{Permanencia de los puntos de experiencia}
    \BRitem[control]{Revisor}{Sin asignar.}

 \BRsection[control]{Atributos}
    
    %\BRitem[admin]{Clase}{\bcCondition}%
    \BRitem[admin]{Clase}{\bcIntegridad}%
    %\BRitem[admin]{Clase}{\bcAutorizacion}%
    %\BRitem[admin]{Clase}{\bcDerivacion}%
        
    \BRitem[admin]{Tipo}{\btEnabler}%
    %\BRitem[admin]{Tipo}{\btTimer}%
    %\BRitem[admin]{Tipo}{\btExecutive}%
        
    %\BRitem[admin]{Nivel}{\blControlling}
    \BRitem[admin]{Nivel}{\blInfluencing}
    
    \BRitem{Descripción}{%
    Los puntos de experiencia una vez que son obtenidos no pueden ser quitados bajo
    ninguna condicion exceptuando únicamente la acción de eliminación de un usuario y
    la desinstalación de los plugins que formen parte del esquema de experiencia.}

    \BRitem{Ejemplo positivo}{\hfill\par%
        \begin{itemize}
        \item Un usuario conforme va completando las secciones de los cursos obtiene
              puntos de experiencia, si el curso es eliminado entonces el usuario 
              deberá permanecer con los puntos de experiencia obtenidos.

        \item Un usuario con 300 puntos de experiencia es eliminado del sitio, y en
              consecuencia se eliminan sus puntos de experiencia.
        \end{itemize}
    }

    \BRitem{Ejemplo negativo}{\hfill\par%
        \begin{itemize}
        \item Un curso es eliminado y a todos los estudiantes se les resta de sus
              puntos de experiencia la cantidad de experiencia obtenida durante el
              curso.
        \end{itemize}
    }% 
    
 \end{BusinessRule}

