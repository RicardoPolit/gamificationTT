\begin{BusinessRule}[%
Autor/Daniel Isai Ortega Zúñiga,%
Version/0.1,%
Estado/edicion]%
%
{BR-E09}{Secciones editables de un curso con experiencia}

     \BRitem[control]{Revisada por}{Pendiente.}

 \BRsection[control]{Atributos}
    % Clases: \bcCondition, \bcIntegridad, \bcAutorization o \bcDerivation
    % Tipos: \btEnabler, \btTimer o \btExecutive
    % Niveles: \blControlling o \blInfluencing.

    \BRitem[admin]{Clase}{\bcCondition}%

    \BRitem[admin]{Tipo}{\btEnabler}%

    \BRitem[admin]{Nivel}{\blControlling}

    \BRitem{Descripción}{%
        Para todas aquellas secciones de un curso gamificado que ya hayan sido
        completadas por al menos un estudiante, se debe bloquear la edición de los
        puntos de experiencia que otorgan con el propósito de impedir que una sección
        en distintos momentos para distintos alumnos otorge puntos de experiencia.
    }

%   \BRitem{Sentencia}{%
%       Si $fecha$
%   }%

    \BRitem{Ejemplo positivo}{\hfill\par%
        \begin{itemize}
        \item Al abrir la edición de un curso con experiencia con cinco secciones,
              se deshabilita la edición de las primeras tres secciones del curso
              ya que estas han sido completadas por almenos un estudiante.
        \end{itemize}
    }

    \BRitem{Ejemplo negativo}{\hfill\par%
        \begin{itemize}
        \item Al abrir la edición de un curso con experiencia con cinco secciones,
              no se deshabilitan la edición de las secciones del curso ya que hayan
              sido completadas con anterioridad.
        \end{itemize}
    }

 \end{BusinessRule}
