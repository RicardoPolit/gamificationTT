u
% INPUT: Cursos Igualitarios.
% INPUT: Tipos de Incremento lineal en niveles.
% INPUT: Modificación de experiencia de los cursos con experiencia.
% INPUT: Modificacion de la cantidad de experiencia de los alumnos.
% INPUT: Tipos de Incremento lineal en niveles
% INPUT: Otorgar experiencia
% INPUT: Administración de experiencia en el curso

\begin{BusinessRule}[%
Autor/Daniel Isai Ortega Zúñiga,%
Version/0.1,%
Estado/revision]%
%
{BR-E01}{Restricciones del nombre de la imagen de los archivos}
 % El archivo de instalación debe ser un archivo ZIP, el cual debe contener exactamente un
 % directorio que coincida con el nombre del plugin.
     \BRitem[control]{Revisor}{Sin asignar.}

 \BRsection[control]{Atributos}
    
    \BRitem[admin]{Clase}{\bcCondition}%
    %\BRitem[admin]{Clase}{\bcIntegridad}%
    %\BRitem[admin]{Clase}{\bcAutorizacion}%
    %\BRitem[admin]{Clase}{\bcDerivacion}%
        
    \BRitem[admin]{Tipo}{\btEnabler}%
    %\BRitem[admin]{Tipo}{\btTimer}%
    %\BRitem[admin]{Tipo}{\btExecutive}%
        
    \BRitem[admin]{Nivel}{\blControlling}
    %\BRitem[admin]{Nivel}{\blInfluencing}
    
    \BRitem{Descripción}{%
        El archivo seleccionado para la representación visual de los niveles debe
        ser una imagen con las extensiones {\it``png''} o {\it''jgp}, además el
        nombre del archivo que será subido no debe tener el nombre {\it``icon.png''}
        ya que posiblemente colisionaría con el \refElem{Plugin.icon} del plugin.
        % debido a que se el directorio donde se guardará será el directorio 
        % para almacenar las imágenes del plugin.
    }

%   \BRitem{Sentencia}{%
%       Si $fecha$ 
%   }%

    \BRitem{Ejemplo positivo}{\hfill\par%
        \begin{itemize}
        \item El archivo seleccionado para ser la imagen de los niveles tiene
              como nombre {\it``logotipo''} con la extensión {\it png}.

        \item El archivo seleccionado para ser la imagen de los niveles tiene
              como nombre {\it``nivel''} con la extensión {\it jpg}
        \end{itemize}
    }

    \BRitem{Ejemplo negativo}{\hfill\par%
        \begin{itemize}
        \item El archivo seleccionado para ser la imagen de los niveles tiene
              como nombre {\it``icon''} con la extensión {\it png}.

        \item El archivo seleccionado para ser la imagen de los niveles tiene
              como nombre {\it``documento''} con la extensión {\it doc}.
        \end{itemize}
    }% 
    
 \end{BusinessRule}


\begin{BusinessRule}[%
Autor/Daniel Isai Ortega Zúñiga,%
Version/0.1,%
Estado/revision]%
%
{BR-E02}{Permanencia de los puntos de experiencia}
    \BRitem[control]{Revisor}{Sin asignar.}

 \BRsection[control]{Atributos}
    
    %\BRitem[admin]{Clase}{\bcCondition}%
    \BRitem[admin]{Clase}{\bcIntegridad}%
    %\BRitem[admin]{Clase}{\bcAutorizacion}%
    %\BRitem[admin]{Clase}{\bcDerivacion}%
        
    \BRitem[admin]{Tipo}{\btEnabler}%
    %\BRitem[admin]{Tipo}{\btTimer}%
    %\BRitem[admin]{Tipo}{\btExecutive}%
        
    %\BRitem[admin]{Nivel}{\blControlling}
    \BRitem[admin]{Nivel}{\blInfluencing}
    
    \BRitem{Descripción}{%
    Los puntos de experiencia una vez que son obtenidos no pueden ser quitados bajo
    ninguna condicion exceptuando únicamente la acción de eliminación de un usuario y
    la desinstalación de los plugins que formen parte del esquema de experiencia.}

    \BRitem{Ejemplo positivo}{\hfill\par%
        \begin{itemize}
        \item Un usuario conforme va completando las secciones de los cursos obtiene
              puntos de experiencia, si el curso es eliminado entonces el usuario 
              deberá permanecer con los puntos de experiencia obtenidos.

        \item Un usuario con 300 puntos de experiencia es eliminado del sitio, y en
              consecuencia se eliminan sus puntos de experiencia.
        \end{itemize}
    }

    \BRitem{Ejemplo negativo}{\hfill\par%
        \begin{itemize}
        \item Un curso es eliminado y a todos los estudiantes se les resta de sus
              puntos de experiencia la cantidad de experiencia obtenida durante el
              curso.
        \end{itemize}
    }% 
    
 \end{BusinessRule}


\begin{BusinessRule}[%
Autor/Daniel Isai Ortega Zúñiga,%
Version/0.1,%
Estado/revision]%
%
{BR-E03}{Modificaciones en el esquema de experiencia}
    \BRitem[control]{Revisor}{Sin asignar.}

 \BRsection[control]{Atributos}
    
    \BRitem[admin]{Clase}{\bcCondition}%
    %\BRitem[admin]{Clase}{\bcIntegridad}%
    %\BRitem[admin]{Clase}{\bcAutorizacion}%
    %\BRitem[admin]{Clase}{\bcDerivacion}%
        
    \BRitem[admin]{Tipo}{\btEnabler}%
    %\BRitem[admin]{Tipo}{\btTimer}%
    %\BRitem[admin]{Tipo}{\btExecutive}%
        
    \BRitem[admin]{Nivel}{\blControlling}
    %\BRitem[admin]{Nivel}{\blInfluencing}
    
    \BRitem{Descripción}{%
    Cuando se modifiquen el \refElem{xp-scheme-settings} o la \refElem{levelXP} de las
    \refElem{xp-scheme-settings}
    }

    \BRitem{Ejemplo positivo}{\hfill\par%
        \begin{itemize}
        \item ...
        \end{itemize}
    }

    \BRitem{Ejemplo negativo}{\hfill\par%
        \begin{itemize}
        \item ...
        \end{itemize}
    }% 
    
 \end{BusinessRule}

\begin{BusinessRule}[%
Autor/Daniel Isai Ortega Zúñiga,%
Version/0.1,%
Estado/revision]%
%
{BR-E04}{Modificacion la cantidad que brindan los cursos} % Cuando están iniciados
    \BRitem[control]{Revisor}{Sin asignar.}

 \BRsection[control]{Atributos}
    
    \BRitem[admin]{Clase}{\bcCondition}%
    %\BRitem[admin]{Clase}{\bcIntegridad}%
    %\BRitem[admin]{Clase}{\bcAutorizacion}%
    %\BRitem[admin]{Clase}{\bcDerivacion}%
        
    \BRitem[admin]{Tipo}{\btEnabler}%
    %\BRitem[admin]{Tipo}{\btTimer}%
    %\BRitem[admin]{Tipo}{\btExecutive}%
        
    \BRitem[admin]{Nivel}{\blControlling}
    %\BRitem[admin]{Nivel}{\blInfluencing}
    
    \BRitem{Descripción}{%
    }

%   \BRitem{Sentencia}{%
%       Si $fecha$ 
%   }%

    \BRitem{Ejemplo positivo}{\hfill\par%
        \begin{itemize}
        \item ...
        \end{itemize}
    }

    \BRitem{Ejemplo negativo}{\hfill\par%
        \begin{itemize}
        \item ...
        \end{itemize}
    }% 
    
\end{BusinessRule}
