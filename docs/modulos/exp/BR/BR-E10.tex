\begin{BusinessRule}[%
Autor/Daniel Isai Ortega Zúñiga,%
Version/0.1,%
Estado/edicion]%
%
{BR-E10}{Administración de la experiencia de un curso}

     \BRitem[control]{Revisada por}{Pendiente.}

 \BRsection[control]{Atributos}
    % Clases: \bcCondition, \bcIntegridad, \bcAutorization o \bcDerivation
    % Tipos: \btEnabler, \btTimer o \btExecutive
    % Niveles: \blControlling o \blInfluencing.

    \BRitem[admin]{Clase}{\bcCondition}%

    \BRitem[admin]{Tipo}{\btEnabler}%

    \BRitem[admin]{Nivel}{\blControlling}

    \BRitem{Descripción}{%
       La suma de los puntos de experiencia que brindan las distintas secciones de
       un curso con soporte para brindar puntos de experiencia debe ser exactamente
       igual a la experiencia total que brindan los cursos gamificados (este valor
       de experiencia es establecido por el \refElem{aAdministrador}).
    }

%   \BRitem{Sentencia}{%
%       Si $fecha$
%   }%

    \BRitem{Ejemplo positivo}{\hfill\par%
        \begin{itemize}
        \item La cantidad total de experiencia que deben brindar los cursos es de 
              1500, por lo tanto la suma de las secciones de todos los cursos es
              exactamente igual a 1500 puntos.
        \end{itemize}
    }

    \BRitem{Ejemplo negativo}{\hfill\par%
        \begin{itemize}
        \item La cantidad total de experiencia que deben brindar los cursos es de 
              1500 y la suma de las secciones de un curso $A$ en particular es de
              3000 puntos.
        \end{itemize}
    }

 \end{BusinessRule}
