\begin{BusinessRule}[%
Autor/Daniel Isai Ortega Zúñiga,%
Version/0.1,%
Estado/revision]%
%
{BR-E06}{Eliminación de cursos gamificados} % Cuando están iniciados
    \BRitem[control]{Revisor}{Sin asignar.}

 \BRsection[control]{Atributos}
    
    \BRitem[admin]{Clase}{\bcCondition}%
    %\BRitem[admin]{Clase}{\bcIntegridad}%
    %\BRitem[admin]{Clase}{\bcAutorizacion}%
    %\BRitem[admin]{Clase}{\bcDerivacion}%
        
    \BRitem[admin]{Tipo}{\btEnabler}%
    %\BRitem[admin]{Tipo}{\btTimer}%
    %\BRitem[admin]{Tipo}{\btExecutive}%
        
    \BRitem[admin]{Nivel}{\blControlling}
    %\BRitem[admin]{Nivel}{\blInfluencing}
    
    \BRitem{Descripción}{% 
        Debido a que el \refElem{mdl-course.format} por defecto para los cursos de
        moodle es el formato de tópicos/temas El formato de curso gamificado extiende
        las funcionalidades de este formato para facilitar la migración de un curso
        gamificado a uno no gamificado y viceversa. % TODO Pasar a analisis.
        La desinstalación del módulo de exériencia implica que los cursos con el
        \refElem{xp-course.format} gamificado (gamedle) se migren a cursos no
        gamificados, por compatibilidad en este migración se deben realizar las 
        siguientes acciones de forma transaccional:

        \begin{itemize}
        \item El formato de los \refElem[cursos gamificados]{xp-course} debe 
              cambiarse por el formato por defecto que tienen los cursos en moodle
              el cual es el de tópicos/temas.

        \item Se deben eliminar las \refElem[opciones del formato del curso]%
              {mdl-course-format-options} gamificado (gamedle).

        \item Se deben establecer las opciones de \refElem[secciones ocultas]%
              {xp-course.name} y aspecto del curso equivalentes en el formato por 
              defecto (tópicos/temas).
        \end{itemize}
    }

%   \BRitem{Sentencia}{%
%       Si $fecha$ 
%   }%

    \BRitem{Ejemplo positivo}{\hfill\par%
        \begin{itemize}
        \item ...
        \end{itemize}
    }

    \BRitem{Ejemplo negativo}{\hfill\par%
        \begin{itemize}
        \item ...
        \end{itemize}
    }% 
    
\end{BusinessRule}
