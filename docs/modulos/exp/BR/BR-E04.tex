\begin{BusinessRule}[%
Autor/Daniel Isai Ortega Zúñiga,%
Version/0.1,%
Estado/revision]%
%
{BR-E04}{Calculo de experiencia del nivel con incremento porcentual}
    \BRitem[control]{Revisor}{Sin asignar.}

 \BRsection[control]{Atributos}
    
    \BRitem[admin]{Clase}{\bcCondition}%
    %\BRitem[admin]{Clase}{\bcIntegridad}%
    %\BRitem[admin]{Clase}{\bcAutorizacion}%
    %\BRitem[admin]{Clase}{\bcDerivacion}%
        
    \BRitem[admin]{Tipo}{\btEnabler}%
    %\BRitem[admin]{Tipo}{\btTimer}%
    %\BRitem[admin]{Tipo}{\btExecutive}%
        
    \BRitem[admin]{Nivel}{\blControlling}
    %\BRitem[admin]{Nivel}{\blInfluencing}
    
    \BRitem{Descripción}{%
        El calculo para obtener la experiencia del nivel $i$ uando el tipo de
        incremento es porcentual está dado por la siguiente fórmula: Sea {\it exp()}
        la función que optiene la experiencia de un nivel en específico, sea tambien
        $i$ el nivel del cual se calcula la experiencia, sea $inc$ el factor de
        incremento de nivel a nivel, y finalmente sea $round()$ una función de
        redondeo a números enteros, entonces:

            $$ exp(i) = round( exp(1) * (inc)^{(i-1)})$$
    }

%   \BRitem{Sentencia}{%
%       Si $fecha$ 
%   }%

    \BRitem{Ejemplo positivo}{\hfill\par%
        \begin{itemize}
        \item La experiencia requerida para superar el nivel 1 es de 2000 puntos y el
              factor de incremento entre los niveles es 1.1, entonces la experiencia
              requerida para pasar el nivel 5 es de 2928 puntos.
        \end{itemize}
    }

    \BRitem{Ejemplo negativo}{\hfill\par%
        \begin{itemize}
        \item La experinecia requerida para superar el nivel 1 es de 2000 puntos y el
              factor de incremento entre los niveles es 1.1, entonces la experiencia
              requerida para pasar el nivel 5 es de 2300 puntos.
        \end{itemize}
    }% 
    
\end{BusinessRule}
