
\subsection*{Interfaces de Moodle}

 Como primera instancia se presentan las tanto las interfaces utilizadas de moodle
 que son utilizadas a lo largo de los distintos casos de uso del módulo de
 experiencia. La especificación de las interfaces de moodle contiene una imagen de
 la interfaz y una descripción de los elementos y acciones más relevantes por cada
 interfaz.

    % INTERFACES DE MOODLE
    
\subsubsection{IU-M00 Pantalla principal}

 El tablero o {\it Dashboard} (ver figura~\ref{IU-M02}) es la primer página que ve un usuario de
 inmediatamente despues iniciar sesión, esta página muestra a los usuarios detalles de su progreso
 y fechas límite próximas \cite{MoodleTablero} . Los elementos que tiene esta página con las demás
 páginas del sitio de moodle es el menú de navegación a la izquierda y la columna derecha de
 bloques.

    \IUfig{1}{modulos/moodle/IU/Dashboard.png}{IU-M00}{Pantalla Principal de Moodle}

\subsubsection{Elementos relevantes}

    \begin{itemize}
    \item
    {\bf Menú Superior}
        Como su nombre lo indica se encuentra en la parte superior, este elemento se
        encuentra en la mayoría de las pantallas de moodle.

    \item
    {\bf Menú de Navegación}
        Cuando esta visible se encuentra en la parte izquierda de la parte izquierda
        de la mayoría de las pantallas de moodle. Se puede ocultar o mostrar con la
        acción \IUMenu[].

    \item
    {\bf Contenido}
        Tiene todos los demás elementos que conforman el contenido de la pantalla.

    \end{itemize}

\subsubsection{Acciones relevantes}

    \begin{itemize}
    
    \item
    {\bf \IUMenu{} (Desplegar el menú)}
        Si el menú está oculto, cuando el usuario presione el botón \IUMenu{} el menú de
        navegación se desplegará.

    \item {\bf \IUMenu{} (Ocultar Menu)}
        Si el menú está visible, cuando el usuario presione el botón \IUMenu{} el menú de
        navegación se ocultará.

    \item {\bf \IUAdminSitio{} Administración del sitio }
        Cuando el menú está visible, el botón de administración del sitio nos permitirá
        navegar a la pantalla \refElem{IU-M01}

    \end{itemize}
  % Tablero (Dashboard)
    
\subsubsection{IU-M00a Selector de archivos}

 El selector de archivos permite a los usuarios de moodle seleccionar un archivo desde los archivos
 del servidor, archivos recientes, archivos privados, desde la computadora o incluso buscar imágenes
 para ser seleccionadas \cite{MoodleSelectorArchivos}.

    \IUfig{1}{modulos/moodle/IU/PopUpArchivos.png}{IU-M00a}{Selector de archivos}

\subsubsection{Elementos relevantes}

    \begin{itemize}
    \item {\bf Menú izquierdo}
        Permite al usuario escoger desde que medio seleccionará el archivo a elegir.
    \end{itemize}

\subsubsection{Acciones relevantes}

    \begin{itemize}
    \item {\bf Browse (Subir un archivo)}
        Cuando se presione el botón \fbox{Browse}, el navegador desplegará una ventana
        emergente para seleccionar un archivo desde el sistema de archivos.

    \item {\bf Subir este archivo}
        Cuando el usuario presione este botón el usuario confirmará la acción de subir
        el archivo que previamente a seleccionado.
    \end{itemize}
 % Form: Selector de archivos
    
\subsubsection{IU-M00b Ingreso a moodle}

    Esta pantalla es de moodle. Esta pantalla brinda acceso al sistema.

    \IUfig{1}{modulos/moodle/IU/Login}{IU-M00b}{Ingreso a moodle}

\subsubsection{Elementos Relevantes}

    \begin{itemize}
    \item {\bf Lorem ipsum}
        ...
    \end{itemize}

\subsubsection{Acciones relevantes}

    \begin{itemize}
    \item {\bf Lorem ipsum}
        ...
    \end{itemize}

\clearpage
 % Login

    
\subsection{IU-M01 Pantalla principal}

 La página de portada, o página principal mostrada en la figura \ref{IU-M01}, es la
 página inicial que ve alguien que llega a un sitio Moodle antes o después de entrar al sitio.
 Típicamente un estudiante verá los cursos, algunos bloques de información, mostrados en un tema.
 En la Barra de navegación y en el menú de navegación (esquina superior izquierda).\\

 \noindent 
 La combinación de las políticas del sitio, autenticación del usuario y configuraciones de la
 portada determinan quién puede llegar a la portada, los elementos que pueden ver y acciones
 que pueden realizar \cite{MoodlePortada}.
    % https://docs.moodle.org/all/es/Portada

    \IUfig{1}{modulos/IUMoodle/IU-M01-moodle.png}{IU-M01}{Pantalla Principal de Moodle}

\subsubsection{Elementos Relevantes}

    \begin{itemize}
    \item
    {\bf Menú Superior}
        Como su nombre lo indica se encuentra en la parte superior, este elemento se
        encuentra en la mayoría de las pantallas de moodle.

    \item
    {\bf Menú de Navegación}
        Cuando esta visible se encuentra en la parte izquierda de la parte izquierda
        de la mayoría de las pantallas de moodle. Se puede ocultar o mostrar con la
        acción \IUMenu[].

    \item
    {\bf Contenido}
        Tiene todos los demás elementos que conforman el contenido de la pantalla.

    \end{itemize}

\subsubsection{Acciones relevantes}

    \begin{itemize}
    
    \item
    {\bf \IUMenu (Desplegar el menú)}
        Si el menú está oculto, cuando el usuario presione el botón \IUMenu el menú de
        navegación se desplegará.

    \item {\bf \IUMenu (Ocultar Menu)}
        Si el menú está visible, cuando el usuario presione el botón \IUMenu el menú de
        navegación se ocultará.

    \item {\bf \IUAdminSitio Administración del sitio }
        Cuando el menú está visible, el botón de administración del sitio nos permitirá
        navegar a la pantalla \refElem{IU-M02}

    \end{itemize}
  % Administración del sitio
    
\subsection{IU-M01a: Resultado de instalación del plugin}

 Esta pantalla muestra el resultado de la instalación del plugin. Si los plugins
 que son instalados vienen de una fuente confiable, esta pantalla siempre debería
 de aparecer mostrando un resultado existoso, en caso contrario dira que la instalación
 no pudo llevarse a cabo de forma exitosa.

    \IUfig{1}{modulos/IUMoodle/InstallResult.png}{IU-M01a}{Resultado de instalación del plugin}

\subsubsection{Elementos relevantes}

    \begin{itemize}
    \item {\bf Mensaje de estado de instalación}
        El mensaje se pinta de color verde o color rojo dependiendo si el 
    \end{itemize}

\subsubsection{Acciones relevantes}

    \begin{itemize}
    \item {\bf Aceptar}
        Si el plugin possé configuraciones para el administrador entonces esta acción
        redirigirá a la pantalla de configuración correspondiente al plugin, en caso
        contrario redirigirá a la anterior página del sitio de moodle mostrada.
    \end{itemize}
 % Administración del sitio plugins
    
\subsubsection{IU-M01b Administración del sitio (Usuarios)}

 Esta pantalla permite al \refElem{aAdministrador} acceder a las configuraciones
 específicas de usuarios dentro del moodle que administra. En esta pantalla se
 encuentran tres categorías principales de configuraciones, dichas categorías con
 usuarios, cuentas y permisos, cada una con sus correspondientes configuraciones
 específicas.

    \IUfig{1}{modulos/moodle/IU/AdministracionSitioUsers}%
        {IU-M01b}{Administración del sitio (Usuarios)}

\subsubsection{Elementos Relevantes}

    \begin{itemize}
    \item {\bf Menu de opciones}
        En la parte principal de la pantalla se encuentra la lista de las
        opciones agrupadas en las tres categorías de usuarios, cuentas y permisos.
    \end{itemize}

\subsubsection{Acciones relevantes}

    \begin{itemize}
    \item {\bf Agregar un usuario}
        Permite acceder al formulario para crear cuentas para distintos usuarios.

    \item {\bf Mirar la lista de usuarios}
        Permite acceder a la pantalla para gestionar las distintas cuentas de usuarios
        de la plataforma.
    \end{itemize}

\clearpage
 % Administración del sitio usuarios
    
\subsubsection{IU-M01c Administración del sitio (Cursos)}

 Esta pantalla permite al \refElem{aProfesor} acceder a las configuraciones
 específicas de los cursos. El acceso a esta pantalla es indispensable para que el 
 profesor cree un curso, lo edite y tambien administre su contenido.

    \IUfig{1}{modulos/moodle/IU/AdministracionSitioCursos}%
        {IU-M01c}{Administración del sitio (módulos)}

\subsubsection{Elementos Relevantes}

    \begin{itemize}
    \item {\bf Menú de opciones}
        Para la cuenta del profesor se muestran las opciones mínimas requeridas
        para acceder a la gestión de los cursos y de las categorías a las que
        están vinculadas los cursos.
    \end{itemize}

\subsubsection{Acciones relevantes}

    \begin{itemize}
    \item {\bf Selección de la opción gestionar cursos y categorías}
        Redirige a la pantalla para la gestión de cursos y categorías.
    \end{itemize}

\clearpage
 % Administración del sitio Cursos

    
\subsection{IU-M02: Instalador de plugin}

 La página del instalador de plugins permite al \refElem{aAdministrador} instalar nuevos plugins al 
 moodle que administra de una forma sencilla y sin tener que manipular los archivos en el servidor
 donde se tenga moodle instalado, para ello cada \refElem[plugin]{Plugin} a instalar debe estar
 compresos en un archivo {\it ZIP} cumpliendo con la regla \refElem{BR-M1}.

 % TODO: BR-M1: Restricciones el archivo de instalación.
 % El archivo de instalación debe ser un archivo ZIP, el cual debe contener exactamente un
 % directorio que coincida con el nombre del plugin.

    \IUfig{1}{modulos/moodleIU/InstallPlugin.png}{IU-M02}{Instalador de plugin}

\subsubsection{Elementos relevantes}

    \begin{itemize}
    \item {\bf Selector de archivos.}
        Permite elegir un archivo y prepararlo para subirlo a moodle
        y realizar las acciones correspondientes.
    \end{itemize}

\subsubsection{Acciones relevantes}

    \begin{itemize}
    \item {\bf Selección de un archivo}
        Permite seleccionar un \refElem{Plugin} compreso en un archivo {\it ZIP} para
        ser instalado en moodle.

    \item {\bf Instalar plugin desde un archivo ZIP}
        Confirma el envió del formulario que contiene principalmente al archivo compreso con 
        el plugin que será instalado. Redirige a la pantalla \refElem{IU-M02a}.
    \end{itemize}

\clearpage
  % Instalación de Plugin
    
\subsection{IU-M02a Validación del plugin a instalar}

 La pantalla de validación del plugin a instalar presenta el resultado de la validación de un
 archivo de plugin compreso con base en la regla \refElem{BR-M01}. Esta pantalla dira Si el archivo
 esta formado correctamente o no, además de las acciones adicionales que se llevarán a cabo.

    \IUfig{1}{modulos/moodleIU/PluginZIPValidacion}{IU-M02a}{Validación del plugin a instalar}

\subsubsection{Elementos Relevantes}

    \begin{itemize}
    \item {\bf Validación del plugin}
        Contiene el resultado de la validación del plugin más las acciones
        a realizar para proceder con la instalación del plugin.
        
    \end{itemize}

\subsubsection{Acciones relevantes}

    \begin{itemize}
    \item {\bf Continuar}
        En caso correcto de la validación, este botón permite continuar con la instalación
        rediriginedo a la página \refElem{IU-M02b}.

    \item {\bf Cancelar}
        El botón de cancelar interrumpe el proceso de instalación del plugin, redirigiendo
        a la anterior pantalla \refElem{IU-M01}.
    \end{itemize}

\clearpage
 % Validación de archivo ZIP
    
\subsubsection{IU-M02b Comprobación de plugins}

 Esta página muestra los plugins que pueden requerir su atención durante un actualización al sitio
 de moodle, como una la instalación o actualización de plugins. La documentación de esta pantalla
 contempla únicamente el caso de instalación/actualización de un plugin a la vez.

    \IUfig{1}{modulos/moodle/IU/InstallConfirm}{IU-M02b}{Comprobación de plugins}

\subsubsection{Elementos relevantes}

   \begin{itemize}
   \item {\bf Lista de plugins a instalar}
        La lista de \refElem[Plugins]{Plugin} que se van a instalar, incluyendo
        sus atributos.
   \end{itemize}

\subsubsection{Acciones relevantes}

    \begin{itemize}
    \item {\bf Actualizar base de datos de Moodle ahora}
        Esta acción permite instalar el plugin en Moodle y correr la secuencia de
        instrucciones establecida por el plugin a instalar. Redirige a la pantalla
        \refElem{IU-M02d}.

    \item {\bf Cancelar las nuevas instalaciones}
        Esta acción permite cancelar las instalaciones o actualizaciones de los plugins,
        redirige a la pantalla \refElem{IU-M02c}

    \item {\bf Cancelar esta instalación}
        A diferencia de la acción anterior, esta acción permite cancelar la instalación
        o actualización de un plugin en particular anterior. Redirige a la pantalla
        \refElem{IU-M02c}.
    \end{itemize}

\clearpage
 % Comprobación de plugibs
    
\subsubsection{IU-M02c: Cancelación de instalación de plugins}

 Esta pantalla tiene el propósito de notificar al \refElem{aAdministrador} de las acciones a
 llevar a cabo en caso de proseguir con la cancelación de la instalación de los plugins. Esta es
 la última confirmación que se le pregunta al administrador antes de la cancelación.

    \IUfig{1}{modulos/moodle/IU/InstallCancelled.png}{IU-M02c}{Comprobación de pluginc}

\subsubsection{Elementos relevantes}

    \begin{itemize}
    \item {\bf Lista de los plugins}
        Contiene la lista de los plugins y la ubicación absoluta de la carpeta
        que contiene todos los archivos de cada plugin que será eliminado.
    \end{itemize}

\subsubsection{Acciones relevantes}

    \begin{itemize}
    \item {\bf Continuar}
        Esta acción confirma la cancelación de los plugins y eliminación de los
        archivos de los mismos. Redirige a la pantalla \refElem{IU-M02}.

    \item {\bf Cancelar}
        Esta acción regresa a la pantalla \refElem{IU-M01} para continuar con la instalación
        de plugins.
    \end{itemize}

\clearpage
 % Cancelación Instalación Plugin
    
\subsubsection{IU-M02d: Resultado de instalación del plugin}

 Esta pantalla muestra el resultado de la instalación del plugin. Si los plugins
 que son instalados vienen de una fuente confiable, esta pantalla siempre debería
 de aparecer mostrando un resultado existoso, en caso contrario dira que la instalación
 no pudo llevarse a cabo de forma exitosa.

    \IUfig{1}{modulos/moodle/IU/InstallResult.png}{IU-M02d}{Resultado de instalación del plugin}

\subsubsection{Elementos relevantes}

    \begin{itemize}
    \item {\bf Mensaje de estado de instalación}
        El mensaje se pinta de color verde o color rojo dependiendo si el 
    \end{itemize}

\subsubsection{Acciones relevantes}

    \begin{itemize}
    \item {\bf Aceptar}
        Si el plugin poseé configuraciones para el administrador entonces esta acción
        redirigirá a la pantalla de configuración correspondiente al \refElem{Plugin},
        en caso contrario redirigirá a la anterior página del sitio de moodle mostrada.
    \end{itemize}

\clearpage
 % Resultado Instalación Plugin

    
\subsubsection{IU-M03 Vista General de Plugins}

 Esta pantalla permite al \refElem{aAdministrador} visualizar la lista de los plugins
 instalados en moodle incluyen los que vienen incluidos en modole, aśi como los
 adicionales, tambien permite acceder a la pantalla \refElem{IU-M03a} para ver la
 lista de todos los plugins.

    \IUfig{1}{modulos/moodle/IU/VistaGeneralPlugins}{IU-M03}{Vista General de Plugins}

\subsubsection{Elementos Relevantes}

    \begin{itemize}
    \item {\bf Lista de todos los plugins}
        Contiene la lista de todos los plugins agrupados por el tipo de plugin, 
        se pueden visualizar y realizar acciones de configuración y eliminación
        a cada elemento de lista.
    \end{itemize}

\subsubsection{Acciones relevantes}

    \begin{itemize}
    \item {\bf Cambiar la lista mostrada de los plugins}
        Permite seleccionar la pestaña correspondiente a la lista de plugins
        que se desea ver.
    \item {\bf Acceder a la configuración de un plugin}
        Si el plugin tiene configuración mediante este enlace se puede acceder
        a la configuración del plugin correspondiente.
    \item {\bf Desinstalar}
        Permite la desinstalación de los plugins que no sean requeridos en la
        plataforma.
    \end{itemize}

\clearpage
  % 
    
\subsubsection{IU-M03a Vista de Plugins Adicionales}

 Esta pantalla permite al \refElem{aAdministrador} visualizar la lista de los plugins
 instalados adicionales instalados, tambien permite acceder a la pantalla
 \refElem{IU-M03} para ver la lista de todos los plugins.

    \IUfig{1}{modulos/moodle/IU/VistaPluginsAdicionales}%
        {IU-M03a}{Vista de Plugins Adicionales}

\subsubsection{Elementos Relevantes}

    \begin{itemize}
    \item {\bf Lista de plugins adicionales}
        Contiene la lista de todos los plugins externos a moodle, instalados por
        el \refElem{aAdministrador} agrupados por el tipo de plugin, de la misma
        forma se pueden visualizar y realizar acciones de configuración y eliminación
        a cada elemento de lista.
    \end{itemize}

\subsubsection{Acciones relevantes}

    \begin{itemize}
    \item {\bf Cambiar la lista mostrada de los plugins}
        Permite seleccionar la pestaña correspondiente a la lista de plugins
        que se desea ver.
    \item {\bf Acceder a la configuración de un plugin}
        Si el plugin tiene configuración mediante este enlace se puede acceder
        a la configuración del plugin correspondiente.
    \item {\bf Desinstalar}
        Permite la desinstalación de los plugins que no sean requeridos en la
        plataforma.
    \end{itemize}

\clearpage
 % 

    
\subsubsection{IU-M04 Desinstalar plugin}

 Esta pantalla muestra un mensaje de confirmación para la acción de eliminar un
 plugin en específico.

    \IUfig{1}{modulos/moodle/IU/DesinstalarPluginConfirm}{IU-M04}{Desinstalar plugin}

\subsubsection{Elementos Relevantes}

    \begin{itemize}
    \item {\bf Mensaje de confirmación}
        Muestra un mensaje solicitando la confirmación del usuario acerca de la
        desinstalación de un plugin.
    \end{itemize}

\subsubsection{Acciones relevantes}

    \begin{itemize}
    \item {\bf Continuar}
        Permite al usuario confirmar a operación de desinstalar un plugin.
    \item {\bf Cancelar}
        Permite al usuario cancelar la operación de desinstalación de un plugin.
    \end{itemize}

\clearpage
  % 
    
\subsubsection{IU-E04a Desinstalando plugin}

 Cuando se desinstala un plugin moodle pide la confirmación del usuario para eliminar
 los archivos correspondientes al plugin. Si el usuario decide no eliminar los
 archivos entonces el plugin se detectará como una plugin para ser instalado desde
 cero.

    \IUfig{1}{modulos/moodle/IU/DesinstalandoPlugin}%
        {IU-M04a}{Desinstalando plugin}

\subsubsection{Elementos Relevantes}

    \begin{itemize}
    \item {\bf Mensaje de confirmación}
        Contiene un mensaje que pide la confirmación del usuario para proceder con
        la eliminación de los archivos de un plugin y si dejar los archivos para que
        este sea instalado de nuevo.
    \end{itemize}

\subsubsection{Acciones relevantes}

    \begin{itemize}
    \item {\bf Continuar}
        Permite confirmar la eliminación de los archivos del plugin que fue
        previamente desinstalado.

    \item {\bf Cancelar}
        Permite indicarle a moodle que no elimine los archivos correspondientes al
        plugin se acaba de eliminar, esto permite que el plugin se pueda instalar
        desde cero.
    \end{itemize}

\clearpage
 % 

    
\subsubsection{IU-M05 Creación de usuario}

 Descripción ...

    \IUfig{1}{modulos/moodle/IU/CreateUser}{IU-M05}{Creación de usuario}

\subsubsection{Elementos Relevantes}

    \begin{itemize}
    \item {\bf Lorem ipsum}
        ...
    \end{itemize}

\subsubsection{Acciones relevantes}

    \begin{itemize}
    \item {\bf Lorem ipsum}
        ...
    \end{itemize}

\clearpage
  % 
    
\subsubsection{IU-M05a Lista de usuarios}

 Descripción ...

    \IUfig{1}{modulos/moodle/IU/ListUsers}{IU-M05a}{Lista de Usuarios}

\subsubsection{Elementos Relevantes}

    \begin{itemize}
    \item {\bf Lorem ipsum}
        ...
    \end{itemize}

\subsubsection{Acciones relevantes}

    \begin{itemize}
    \item {\bf Lorem ipsum}
        ...
    \end{itemize}

\clearpage
 % 
    
\subsubsection{IU-M05b Confirmación acerca de eliminar un usuario}

 Mensaje de confirmación acerca de la eliminación de un usuario.

    \IUfig{1}{modulos/moodle/IU/DeleteUser}%
        {IU-M05b}{Confirmación acerca de eliminar un usuario}

\subsubsection{Elementos Relevantes}

    \begin{itemize}
    \item {\bf Mensaje de confirmación}
        Presenta el mensaje de confirmación para continuar con la operación de
        eliminación de un usuario.
    \end{itemize}

\subsubsection{Acciones relevantes}

    \begin{itemize}
    \item {\bf Eliminar}
        Permite continuar con las operaciones de eliminación de un usuario
        de moodle.
        
    \item {\bf Cancelar}
        Permite cancelar las operaciones de eliminación de un usuario de moodle.
    \end{itemize}

\clearpage
 % 
    
\subsubsection{IU-M05c Pantalla de auto-registro}

 Ofrece a los usuarios que no estan registrados en moodle un mecanismo mediante
 el cual pueden acceder a crear su usuario de moodle de forma por su propia cuenta.

    \IUfig{1}{modulos/moodle/IU/CreateOwnUser}{IU-M05c}{Pantalla de auto-registro}

\subsubsection{Elementos Relevantes}

    \begin{itemize}
    \item {\bf Formulario de registro}
        Contiene los campos para el registro de un usuario de moodle
    \end{itemize}

\subsubsection{Acciones relevantes}

    \begin{itemize}
    \item {\bf Crear nueva cuenta}
        Permite al usuario confirmar la creación de se nueva cuenta en el
        sitio de moodle

    \item {\bf Cancelar}
        Permite al usuario cancelar el registro de su cuenta de usuario.
    \end{itemize}

\clearpage
 % 

    
\subsubsection{IU-M06 Cursos y categorías}

 Esta pantalla permite acceder a la gestión de cursos y categorías existentes en
 moodle, partiendo de este punto se pueden editar, configurar, eliminar y crear
 cursos y categorías presentes en moodle. Los actores que pueden acceder a esta
 pantalla son el \refElem{aProfesor} y el \refElem{aAdministrador}.

    \IUfig{1}{modulos/moodle/IU/Cursos-Categories}{IU-M06}{Cursos y categorías}

\subsubsection{Elementos Relevantes}

    \begin{itemize}
    \item {\bf Lista de categorías}
        Muestra la lista de categorías existentes en el sitio de moodle.
    \item {\bf Lista de cursos}
        Contiene la lista de los cursos correspondientes a la categoría
        seleccionada, sobre esta lista se puden ejecutar las acciones
        relevantes..
    \end{itemize}

\subsubsection{Acciones relevantes}

    \begin{itemize}
    \item {\bf Crear un nuevo curso}
        Permite crear un nuevo curso con la categoría seleccionada como
        la categoría por defecto.
    \item {\bf Visualizar curso}
        Permite acceder a la pantalla de la visualización de un nuevo curso
        para administrarlo y/o editarlo.
        la categoría por defecto.
    \item {\bf Eliminar un curso}
        Permite seleccionar un curso que se deseé eliminar.
    \end{itemize}

\clearpage
  % Cursos y Categorías
    
\subsubsection{IU-M06a Confirmación para eliminar un curso}

 %Descripción ...

    \IUfig{1}{modulos/moodle/IU/Cursos-Delete}%
        {IU-M06a}{Confirmación para eliminar un curso}

\begin{comment}
\subsubsection{Elementos Relevantes}

    \begin{itemize}
    \item {\bf Lorem ipsum}
        ...
    \end{itemize}

\subsubsection{Acciones relevantes}

    \begin{itemize}
    \item {\bf Lorem ipsum}
        ...
    \end{itemize}
\end{comment}

\clearpage
 % Confirmación eliminar curso
    
\subsubsection{IU-M06b Estado de la eliminación de un curso}

% Descripción ...

    \IUfig{1}{modulos/moodle/IU/Cursos-Delete-Status}%
        {IU-M06b}{Estado de la eliminación de un curso}

\begin{comment}
\subsubsection{Elementos Relevantes}

    \begin{itemize}
    \item {\bf Lorem ipsum}
        ...
    \end{itemize}

\subsubsection{Acciones relevantes}

    \begin{itemize}
    \item {\bf Lorem ipsum}
        ...
    \end{itemize}
\end{comment}

\clearpage
 % Estado de eliminación de un curso
    
\subsubsection{IU-M06c Curso de moodle}

 Descripción ...

    \IUfig{1}{modulos/moodle/IU/Cursos-normal}{IU-M06c}{Curso de moodle}

\subsubsection{Elementos Relevantes}

    \begin{itemize}
    \item {\bf Lorem ipsum}
        ...
    \end{itemize}

\subsubsection{Acciones relevantes}

    \begin{itemize}
    \item {\bf Lorem ipsum}
        ...
    \end{itemize}

\clearpage
 % Curso de moodle

    
\subsubsection{IU-M07 Pantalla principal del curso}

 Descripción ...

    \IUfig{1}{modulos/moodle/IU/p_principal_curso}{IU-M07}{Pantalla principal del curso}

\subsubsection{Elementos Relevantes}

    \begin{itemize}
    \item {\bf Lorem ipsum}
        ...
    \end{itemize}

\subsubsection{Acciones relevantes}

    \begin{itemize}
    \item {\bf Lorem ipsum}
        ...
    \end{itemize}

\clearpage
 % Pagina Inicial del Sitio

\subsection*{Interfaces diseñadas}

 A continuación se presentan todas las interfaces que fueron requeridas agregar
 a moodle para brindar las funcionalidades planteadas por los casos de uso. Al igual
 que la especificación de las interfaces de moodle, la especificación de estas
 interfaces contiene la imagen de la interfaz y una descripción de los elementos
 y acciones más relevantes por cada interfaz.

    
\subsubsection{IU-E01 Configuraciones del módulo de experiencia}

 Descripción ...

    \IUfig{1}{modulos/exp/IU/Settings}{IU-E01}{Configuraciones del módulo de experiencia}

\subsubsection{Elementos Relevantes}

    \begin{itemize}
    \item {\bf Lorem ipsum}
        ...
    \end{itemize}

\subsubsection{Acciones relevantes}

    \begin{itemize}
    \item {\bf Lorem ipsum}
        ...
    \end{itemize}

\clearpage
  % Configuraciones
    
\subsection{IU-E02 Configuraciones generales}

 Descripción ...

    \IUfig{1}{modulos/modExpIU/SettingsGenerales}{IU-E02}{Configuraciones generales}

\subsubsection{Elementos Relevantes}

    \begin{itemize}
    \item {\bf Lorem ipsum}
        ...
    \end{itemize}

\subsubsection{Acciones relevantes}

    \begin{itemize}
    \item {\bf Lorem ipsum}
        ...
    \end{itemize}

\clearpage
  % Configuraciones Generales
    
\subsection{IU-E03: Configuraciones Visuales del módulo de experiencia}

 Descripción ...

    \IUfig{1}{modulos/exp/IU/settingsVisuales}{IU-E03}{%
        configuraciones Visuales del módulo de experiencia}

\subsubsection{Elementos Relevantes}

    \begin{itemize}
    \item {\bf Lorem ipsum}
        ...
    \end{itemize}

\subsubsection{Acciones relevantes}

    \begin{itemize}
    \item {\bf Guardar cambios}
        ...

    \item {\bf Cancelar}
        ...
    \end{itemize}

\clearpage
  % Configuraciones Visuales
    
\subsection{IU-E03a Módulo de experiencia desactivado}

 Descripción ...

    \IUfig{1}{modulos/modExpIU/settingsExperienceDisabled}{IU-E03a}{%
        Módulo de experiencia desactivado}

\subsubsection{Elementos Relevantes}

    \begin{description}
    \bTerm{tSelectColor}{panel de colores} ...
    \end{description}

\subsubsection{Acciones relevantes}

    \begin{itemize}
    \item {\bf Lorem ipsum}
        ...
    \end{itemize}

\clearpage
 % Configuraciones Mod Exp desactivado
    
\subsubsection{IU-E04: Configuraciones del sistema de experiencia}

 Descripción ...

    \IUfig{1}{modulos/exp/IU/settingsEsquema}{IU-E04}{%
        Configuraciones del sistema de experiencia}

\subsubsection{Elementos Relevantes}

    \begin{itemize}
    \item {\bf Lorem ipsum}
        ...
    \end{itemize}

\subsubsection{Acciones relevantes}

    \begin{itemize}
    \item {\bf Lorem ipsum}
        ...
    \end{itemize}

\clearpage
  % Configuraciones Esquema
    
\subsubsection{IU-E05 Configuración de eventos con experiencia}

% Descripción ...

    \IUfig{1}{modulos/exp/IU/SettingsEventos}{IU-E05}%
        {Configuración de eventos con experiencia}

\begin{comment}
\subsubsection{Elementos Relevantes}

    \begin{itemize}
    \item {\bf Lorem ipsum}
        ...
    \end{itemize}

\subsubsection{Acciones relevantes}

    \begin{itemize}
    \item {\bf Lorem ipsum}
        ...
    \end{itemize}
\end{comment}

\clearpage
  % Configuraciones de Eventos
    
\subsection{IU-E05a Eventos con experiencia desactivados}

 Descripción ...

    \IUfig{1}{modulos/exp/IU/SettingsEventsDisabled}{IU-E05a}%
    {Eventos con experiencia desactivados}

\subsubsection{Elementos Relevantes}

    \begin{itemize}
    \item {\bf Lorem ipsum}
        ...
    \end{itemize}

\subsubsection{Acciones relevantes}

    \begin{itemize}
    \item {\bf Lorem ipsum}
        ...
    \end{itemize}

\clearpage
 % Configuraciones Events desativados
    
\subsubsection{IU-E06 Creación de un curso gamificado}

 Descripción ...

    \IUfig{1}{modulos/exp/IU/CursoCreate}{IU-E06}{Creación curso gamificado}

\subsubsection{Elementos Relevantes}

    \begin{itemize}
    \item {\bf Lorem ipsum}
        ...
    \end{itemize}

\subsubsection{Acciones relevantes}

    \begin{itemize}
    \item {\bf Lorem ipsum}
        ...
    \end{itemize}

\clearpage
  % Curso Gamificado
    
\subsubsection{IU-E06a Curso Gamificado}

 Descripción ...

    \IUfig{1}{modulos/exp/IU/CursoView}{IU-E06a}{Curso Gamificado}

\subsubsection{Elementos Relevantes}

    \begin{itemize}
    \item {\bf Lorem ipsum}
        ...
    \end{itemize}

\subsubsection{Acciones relevantes}

    \begin{itemize}
    \item {\bf Lorem ipsum}
        ...
    \end{itemize}

\clearpage
  % Curso Gamificado
    
\subsection{IU-E06b Curso gamificado en edición}

 Descripción ...

    \IUfig{1}{modulos/exp/IU/CursoEdit}{IU-E06b}{Curso gamificado en edición}

\subsubsection{Elementos Relevantes}

    \begin{itemize}
    \item {\bf Lorem ipsum}
        ...
    \end{itemize}

\subsubsection{Acciones relevantes}

    \begin{itemize}
    \item {\bf Lorem ipsum}
        ...
    \end{itemize}

\clearpage
 % Curso Gamificado (editing)

    %
\subsubsection{IU-E04: Configuraciones del sistema de experiencia}

 Descripción ...

    \IUfig{1}{modulos/exp/IU/settingsEsquema}{IU-E04}{%
        Configuraciones del sistema de experiencia}

\subsubsection{Elementos Relevantes}

    \begin{itemize}
    \item {\bf Lorem ipsum}
        ...
    \end{itemize}

\subsubsection{Acciones relevantes}

    \begin{itemize}
    \item {\bf Lorem ipsum}
        ...
    \end{itemize}

\clearpage
 % Configuraciones de Comportamiento
    %
\subsubsection{IU-E05 Configuración de eventos con experiencia}

% Descripción ...

    \IUfig{1}{modulos/exp/IU/SettingsEventos}{IU-E05}%
        {Configuración de eventos con experiencia}

\begin{comment}
\subsubsection{Elementos Relevantes}

    \begin{itemize}
    \item {\bf Lorem ipsum}
        ...
    \end{itemize}

\subsubsection{Acciones relevantes}

    \begin{itemize}
    \item {\bf Lorem ipsum}
        ...
    \end{itemize}
\end{comment}

\clearpage
 % Configuraciones de eventos

\subsection*{Diseño de plugins} % TODO CHANGE FOR INPUTS

 Para poder desarrollar las funcionalidades especificadas mediante los casos de uso
 es requerido el desarrollo de tres plugins distintos. Dichos plugins son listados a
 continuación:

    \begin{itemize}
    \item {\bf\color{primary}%
    \hypertarget{local:gamedlemaster}{Gamedlemaster} (local)}

        Este plugin ayudará a extender el esquema de la base de datos de moodle
        para que los plugins de experiencia y los de los demás módulos puedan
        acceder a los datos sin la necesidad de depender explícitamente de otro
        módulo.

    \item {\bf\color{primary}%
    \hypertarget{format:gamedle}{Curso con puntos de experiencia} (course format)}

        Este es un plugin de tipo {\it course format} lo cual permite cambiar
        la estructura de cómo se presentan las secciones y actividades del curso
        \cite{MoodleCourseFormat}. Este tipo de plugin fue escogido para que los
        cursos en moodle puedan otorgar puntos de experiencia.
        % https://docs.moodle.org/dev/Course_formats

    \item {\bf\color{primary}%
    \hypertarget{block:gmxp}{Nivel Gamedle} (block)}

        Los plugins de tipo {\it block} permiten desplegar información y/o brindar una 
        funcionalidad a lo largo de distintas páginas de moodle \cite{MoodleBlocks},
        además permite a cualquier usuario agregarlo u ocultarlo en las distintas
        pantallas a las que tiene acceso, debido a estar razones este plugin se
        encargará de mostrar el nivel y puntos de experiencia de dicho nivel que
        tiene un usuario.
        % https://docs.moodle.org/dev/Plugin_types
    \end{itemize}

\subsubsection{Diagrama de componentes} % TODO CHANGE FOR INPUTS
\subsubsection{Diagrama de clases} % TODO CHANGE FOR INPUTS
