
% \ucstEnEdicion     Al terminar una revisión/aprobación con observaciones
%                    y al inicio del CU.
%
% \ucstEnRevision    Al terminar la edición del CU (version += 0.1).
% \ucstEnAprobacion  Al pasar la revision sin observaciones.
% \ucstAprobado      Al ser aprobado por el usuario (version += 1.0)

\begin{UseCase}[%
Autor/Daniel Ortega,%
Version/0.1,%
Estado/\ucstEnEdicion]%
%
{CU-E03}{Desinstalar plugins del esquema de experiencia}{%
%
 Permite al \refElem{aAdministrador} desinstalar el o los plugins correspondientes
 al módulo de experiencia cuando desee remover las funcionalidades que estos otorgan
 y eliminar los registros de los puntos de experiencia de los usuarios y cursos.}

	\UCitem[control]{Revisor}{ Sin asignar }
	\UCitem[control]{Último cambio}{ \today }

 \UCsection{Atributos}

    \UCitem{Actor(es)}{%
        \refElem{aAdministrador}
    }

	\UCitem{Propósito}{%
        Remover el módulo de experiencia junto con las funcionalidades que
        brinda y los datos generados por el mismo.
	}

	\UCitem{Entradas}{\imprimeUC{entrada}}

	\UCitems{Origen}{%
        \Titem Mouse
	}

	\UCitem{Salidas}{Ninguna}

	\UCitem{Destino}{%
		\refElem{IU-M03}
	}

	\UCitem{Precondiciones}{%
        Los plugins que se desean instalar deben estar previamente
        instalados.
	}

	\UCitem{Postcondiciones}{%
        Los datos de generados por el esquema de experiencia deben ser
        eliminados una vez concluida la desinstalación del o de los plugins.
	}

	\UCitem{Reglas de negocio}{\imprimeUC{regla}}

	\UCitems{Errores}{%
        \Titem \UCerr{Err1}{%
        % CAUSA
            El plugin a desinstalar no se encuentra instalado}{%
        % EFECTO
            no se puede proceder con la ejecución debido a que ya está
            desinstalado}
	}

 \UCsection[design]{Datos de Diseño}

	\UCitems[design]{Casos de Prueba}{%
        \Titem \refElem{CPC-E03}
	}

 \UCsection[admin]{Datos de Administración de Requerimiento}

	\UCitem[admin]{Observaciones}{%
        Como se explica en el modelo de información la eliminación de las entidades
        \refElem{xp-section-reward}, \refElem{xp-user}, \refElem{xp-course-section} y
        \refElem{xp-course} no afectan las funcionalidades de moodle.
	}

\end{UseCase}

\subsubsection{Trayectorias del caso de uso}

\begin{UCtrayectoria}%
%
  \includeUC{CU-M01}
  \Actor Presiona la opción {\bf Vista General de Plugins}.
  \Sistema Carga la pantalla \refElem{IU-M03}.

  \Actor Pide consultar únicamente los plugins adicionales instalados presionando
         la opción {\bf Plugins adicionales}..
  \Sistema Carga la pantalla \refElem{IU-M03a}.

  \Actor Presiona la opción {\bf Desinstalar} correspondiente al
         \entrada[plugin]{Plugin} del módulo de experiencia que desea desinstalar.
  \Sistema Muestra la pantalla \refElem{IU-M04}.
  \Sistema Pide la confirmación del usuario para continuar con la desinstalación.

  \Actor Presiona el botón de {\bf Aceptar}. \refTray{A}

  \Sistema Obtiene la lista de los \entrada[cursos gamificados]{xp-course} y por cada curso
           realiza las siguientes acciones, y continúaen el \label{CU-E03-loop-delete-course}

  \Sistema Cambia los \entrada[formatos]{mdl-course.format} de los cursos por el formato de
           secciones aplicando la regla \regla{BR-E06}.

  \Sistema Obtiene las \entrada[secciones de los curso gamificados]{xp-course-section}
           y las elimina.

  \Sistema Elimina los datos de todos las \entrada[recompensas de experiencia]%
           {xp-section-reward} y \entrada[usuarios gamificados]{xp-user}
           registrados en el sistema.

  \Sistema Muestra la pantalla \refElem{IU-M04a} pidiendo al usuario la confirmación
           para eliminar los archivos del plugin.

  \Actor Presiona el botón {\bf Continuar}. \refTray{B}


  \Sistema Elimina los archivos correspondientes al plugin.

  \Sistema Redirige a la pantalla \refElem{IU-M03}.

\end{UCtrayectoria}

\begin{UCtrayectoriaA}{A}{%
El \refElem{aAdministrador} desea cancelar la desinstalación del plugin
}
  \Actor Presiona el botón de {\bf Cancelar}.

  \Sistema Cancela las operaciones de desinstalación y redirige al usuario a la
           pantalla \refElem{IU-M04}.
\end{UCtrayectoriaA}

\begin{UCtrayectoriaA}{B}{%
El \refElem{aAdministrador} no desea eliminar los archivos del plugin}

  \Actor Presiona el botón {\bf Cancelar}.

  \Sistema Detectando que hay archivos de un plugin que no está instalado.

  \Sistema Carga la pantalla \refElem{IU-M02b} detectando el plugin del módulo de experiencia como
           plugin a instalar.

\end{UCtrayectoriaA}
