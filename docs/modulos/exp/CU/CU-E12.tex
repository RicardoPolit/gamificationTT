
% \ucstEnEdicion     Al terminar una revisión/aprobación con observaciones 
%                    y al inicio del CU.
%
% \ucstEnRevision    Al terminar la edición del CU (version += 0.1).
% \ucstEnAprobacion  Al pasar la revision sin observaciones.
% \ucstAprobado      Al ser aprobado por el usuario (version += 1.0)

\begin{UseCase}[%
Autor/Daniel Ortega,%
Version/0.1,%
Estado/\ucstEnEdicion]%
%
{CU-E12}{Crear cuenta de usuario gamificado}{%
%
 Permite al módulo de experiencia recibir el evento, emitido por moodle, de cuando
 un usuario es creado, con base en este evento el módulo de experiencia asignará los
 datos por defecto de un usuario gamificado al usuario que se acaba de crear.}

	\UCitem[control]{Revisor}{ Sin asignar }
	\UCitem[control]{Último cambio}{ \today }

 \UCsection{Atributos}

    \UCitem{Actor(es)}{%
        \refElem{aAdministrador}.
    }

	\UCitem{Propósito}{%
        Asociar una cuenta de un \refElem{xp-user} a un \refElem{mdl-user}
        cuando es creado por el administrador.
	}
	
	\UCitems{Entradas}{
        \Titem Evento emitido por moodle de que un usuario a sido creado.
        \Titem \refElem{mdl-user.id} de \refElem{mdl-user}.
    }

	\UCitems{Origen}{%
        \Titem API de eventos de Moodle.
	}

	\UCitem{Salidas}{\imprimeUC{salida}}

	\UCitem{Destino}{Ninguno}
	
	\UCitems{Precondiciones}{%
        \Titem Los plugins del módulo de experiencia deben de estar instalados.
	}

	\UCitems{Postcondiciones}{%
        \Titem Los datos de un usuario gamificado deben permanecer vinculados con
               el usuario de moodle que se creó.
	}

	\UCitem{Reglas de negocio}{Ninguna}

	\UCitems{Errores}{%
        \Titem \UCerr{Err1}{%
        % CAUSA
            Los plugins del módulo de experiencia no se encuentran instalados y}{%
        % EFECTO (inconsistencia)
            no se puede crear un usuario gamificado, termina el caso de uso}
	}

	% \UCitem{Viene de}{% Indicar si el Caso de uso es primario o se extiende de otro. La mayoría se 
					  % extienden de Login.
		% EJEMPLO: \refIdElem{PY-CU1} o Caso de uso primario.
	% 	\TODO Especificar.
	% }	

 \UCsection[design]{Datos de Diseño}

	\UCitems[design]{Casos de Prueba}{%
        \Titem \refElem{CPC-E12}
	}

 \UCsection[admin]{Datos de Administración de Requerimiento}

	\UCitem[admin]{Observaciones}{%
        Ninguna
	}

\end{UseCase}

\clearpage
\subsubsection{Trayectorias del caso de uso}

\begin{UCtrayectoria}%
%
  \includeUC{CU-M01} ejecutando la trayectoria alternativa B
  \Actor Presiona la opción {\bf Agregar un usuario}.
  \Sistema Carga la pantalla \refElem{IU-M05}.
  \Actor Ingresa el \entrada{mdl-user.username}, \entrada{mdl-user.auth}, 
         \entrada{mdl-user.password}, \entrada{mdl-user.firstname} y
         \entrada{mdl-user.lastname}, que son los atributos requeridos para el nuevo
         usuario.

  \Actor Opcional: Ingresa los demás valores presentes en la pantalla \refElem{IU-M05}.
  \Actor Presiona el botón de {\bf Crear Usuario}. \refTray{A}

  \Sistema Obtiene lo valores ingresados para el nuevo usuario y crea la cuenta de un
           \refElem{mdl-user}. 
  \Sistema Obtiene el \entrada{mdl-user.id} del usuario recientemente creado 

  \Sistema Vincula los datos de un \salida{xp-user} con el id obtenido, estableciendo los
           valores por defecto para el \refElem{xp-user.level}, \refElem{xp-user.levelxp}
           y \refElem{xp-user.xp} definidos en el modelo de información. \refErr{Err1}

  \Sistema Obtiene el \salida{xp-user.firstname}, \salida{xp-user.lastname}, 
           \salida{xp-user.email}, \salida{xp-user.lastaccess}, \salida{xp-user.city},
           \salida{xp-user.country} de los usuarios presentes en moodle.

  \Sistema Despliega la información de los usuarios en la pantalla \refElem{IU-M05a}.
\end{UCtrayectoria}


\begin{UCtrayectoriaA}{A}{%
El \refElem{aAdministrador} desea cancelar la creación de un nuevo usuario
}
  \Actor Presiona el botón {\bf Cancelar}.
  \Sistema Redirige a la pantalla \refElem{IU-M01b}
\end{UCtrayectoriaA}


