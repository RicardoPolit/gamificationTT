
% \ucstEnEdicion     Al terminar una revisión/aprobación con observaciones 
%                    y al inicio del CU.
%
% \ucstEnRevision    Al terminar la edición del CU (version += 0.1).
% \ucstEnAprobacion  Al pasar la revision sin observaciones.
% \ucstAprobado      Al ser aprobado por el usuario (version += 1.0)

\begin{UseCase}[%
Autor/Daniel Ortega,%
Version/0.1,%
Estado/\ucstEnEdicion]%
%
{CU-E12}{Crear cuenta de usuario gamificado}{%
%
 Permite al módulo de experiencia recibir el evento, emitido por moodle, de cuando
 un usuario es creado, con base en este evento el módulo de experiencia asignará los
 datos por defecto de un usuario gamificado al usuario que se acaba de crear.}

	\UCitem[control]{Revisor}{ Sin asignar }
	\UCitem[control]{Último cambio}{ \today }

 \UCsection{Atributos}

    \UCitem{Actor(es)}{%
        \refElem{aAdministrador} o \refElem{aAlumno}.
    }

	\UCitem{Propósito}{%
        Asociar una cuenta de un \refElem{xp-user} a un \refElem{mdl-user}
        cuando es creado por el administrador.
	}
	
	\UCitem{Entradas}{\imprimeUC{entrada}}

	\UCitems{Origen}{%
        \Titem Teclado
        \Titem Mouse
    }

	\UCitem{Salidas}{\imprimeUC{salida}}

	\UCitem{Destino}{\refElem{IU-M05a}}
	
	\UCitems{Precondiciones}{%
        \Titem Los plugins del módulo de experiencia deben de estar instalados.
        \Titem Cuando el actor es un \refElem{aAlumno} \refTray{A}, la opción
               opción de permitir autoregistros debe estar habilitada y las
               configuraciones para poder realizar envío de correos mediante SMTP
               deben ser las correctas.
	}

	\UCitems{Postcondiciones}{%
        \Titem Los datos de un usuario gamificado deben permanecer vinculados con
               el usuario de moodle que se creó.
	}

	\UCitem{Reglas de negocio}{\imprimeUC{regla}}

	\UCitems{Errores}{%
        \Titem \UCerr{Err1}{%
        % CAUSA
            Los plugins del módulo de experiencia no se encuentran instalados y}{%
        % EFECTO
            no se puede crear un usuario gamificado, termina el caso de uso}
        \Titem \UCerr{Err2}{%
        % CAUSA
            La opción de creación de cuentas mediante autoregistro se encuentra
            deshabilitada}{%
        % EFECTO
            no se muestra la opción para que el alumno cree su propia contraseña,
            termina el caso de uso.}
        \Titem \UCerr{Err3}{%
        % CAUSA
            La configuración de envío de correos para autoregistrarse no permite
            enviar correos}{%
        % EFECTO
            no se puede enviar el correo de confirmación, pero la cuenta ha sido 
            creada, el \refElem{aAlumno} contacta con soporte para que validen su
            cuenta manualmente}
	}

	% \UCitem{Viene de}{% Indicar si el Caso de uso es primario o se extiende de 
    % otro. La mayoría se extienden de Login.
		% EJEMPLO: \refIdElem{PY-CU1} o Caso de uso primario.
	% 	\TODO Especificar.
	% }	

 \UCsection[design]{Datos de Diseño}

	\UCitems[design]{Casos de Prueba}{%
        \Titem \refElem{CPC-E12}
        \Titem \refElem{CPC-E12a}
	}

 \UCsection[admin]{Datos de Administración de Requerimiento}

	\UCitem[admin]{Observaciones}{%
        Ninguna
	}

\end{UseCase}

\subsubsection{Trayectorias del caso de uso}

\begin{UCtrayectoria}%
%
  \includeUC{CU-M01} presionando la pestaña {\bf Usuarios}. \refTray{A}
  \Actor Presiona la opción {\bf Agregar un usuario} de la pantalla \refElem{IU-M01b}.
  \Sistema Carga la pantalla \refElem{IU-M05}.
  \Actor Ingresa el \entrada{mdl-user.username}, \entrada{mdl-user.password}, 
         \entrada{mdl-user.firstname} y \entrada{mdl-user.lastname}, que son los 
         atributos requeridos para el nuevo usuario. \label{IU-E12-input-data}

  \Actor Opcionalmente ingresa los valores de los demás campos.

  \Actor Presiona el botón de {\bf Crear Usuario}. \refTray{B} \refTray{C}
         \label{CU-E12-submit}.

  \Sistema Obtiene lo valores ingresados para el nuevo usuario y crea la cuenta de un
           \refElem{mdl-user}. 

  \Sistema Vincula los datos de un \salida{xp-user} con el \refElem{mdl-user} obtenido,
           estableciendo los valores iniciales para el \refElem{xp-user.level},
           \refElem{xp-user.levelxp} y \refElem{xp-user.xp} de acuerdo con la regla
           \regla{BR-E07}. \refErr{Err1} \label{IU-E12}

  \Sistema Obtiene el \salida{mdl-user.firstname}, \salida{mdl-user.lastname}, 
           \salida{mdl-user.email}, \salida{mdl-user.lastaccess},
           \salida{mdl-user.city} y \salida{mdl-user.country} de los usuarios 
           presentes en moodle.

  \Sistema Despliega la información de los usuarios en la pantalla \refElem{IU-M05a}.
\end{UCtrayectoria}


\begin{UCtrayectoriaA}{A}{%
El usuario que ejecuta este caso de uso es el \refElem{aAlumno} para registrarse
así mismo.
}
  \Actor Se encuentra en la pantalla \refElem{IU-M00b}. \refErr{Err2}
  \Actor Presiona el botón {\bf Comience ahora creando una cuenta}.
  \Sistema Carga la pantalla \refElem{IU-M05c}.
  \Sistema Ejecuta del paso \ref{IU-E12-input-data} al 
  \Sistema Envía el correo de confirmación al \refElem{aAlumno}. \refErr{Err3}
  \Actor Revisa su correo y accede al enlace para confirmar el registro de su cuenta.
\end{UCtrayectoriaA}

\begin{UCtrayectoriaA}{B}{%
Algunos de los campos ingresados por el \refElem{aAdministrador} son erroneos
}
  \Sistema Imprime los mensaje de error debajo de los campos con valores incorrectos.
  \Actor Ingresa nuevamente los valores en los campos marcados como incorrectos.
  \Sistema Regresa al paso \ref{CU-E12-submit}.
\end{UCtrayectoriaA}


\begin{UCtrayectoriaA}{C}{%
El \refElem{aAdministrador} desea cancelar la creación de un nuevo usuario
}
  \Actor Presiona el botón {\bf Cancelar}.
  \Sistema Redirige a la pantalla \refElem{IU-M01b}
\end{UCtrayectoriaA}
