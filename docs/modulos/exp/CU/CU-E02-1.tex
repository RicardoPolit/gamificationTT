
% \ucstEnEdicion     Al terminar una revisión/aprobación con observaciones
%                    y al inicio del CU.
%
% \ucstEnRevision    Al terminar la edición del CU (version += 0.1).
% \ucstEnAprobacion  Al pasar la revision sin observaciones.
% \ucstAprobado      Al ser aprobado por el usuario (version += 1.0)

\begin{UseCase}[%
Autor/Daniel Ortega,%
Version/0.1,%
Estado/\ucstEnRevision]%
%
{CU-E02-1}{Configurar visualización de niveles}{%
%
 Permite al \refElem{aAdministrador} establecer y modificar los colores, textos e
 imagenes que se muestran en la visualización del nivel actual y en las ventanas
 emergentes de un nuevo nivel alcanzado. Cuando un administrador actualice los valores
 de la configuración los usuarios serán capaces de ver los cambios al renderizarse la
 siguiente página.}

	\UCitem[control]{Revisor}{ Sin asignar }
	\UCitem[control]{Último cambio}{ \today }

 \UCsection{Atributos}

    \UCitem{Actor(es)}{%
        \refElem{aAdministrador}
    }

	\UCitem{Propósito}{%
      \Titem Cambiar las \refElem{xp-visual-settings} del módulo de experiencia.
      \Titem Cambiar el color de la barra de progreso del nivel actual que ve cada
             usuario.
      \Titem Establecer la imagen que se presenta en cada nivel de la plataforma.
      \Titem El mensaje de felicitaciones que se presenta al alcanzar un nuevo nivel
      \Titem Establecer el color que tendrá el número del nivel
      \Titem Cambiar la descripción de los mensajes visible cuando se alcanza un
             nuevo nivel.
      \Titem Cambiar el nombre de los niveles.
	}

	\UCitem{Entradas}{\imprimeUC{entrada}}

	\UCitems{Origen}{%
        \Titem Mouse
        \Titem Teclado
	}

	\UCitem{Salidas}{\imprimeUC{salida}}

	\UCitems{Destino}{%
		\Titem \refElem{IU-M01}
	}

	\UCitems{Precondiciones}{%
        \Titem Que los plugins del módulo de experiencia se encuentren instalados,
        \Titem El módulo de experiencia haya sido habilitando en el caso uso
               \refElem{CU-E02}
	}

	\UCitem{Postcondiciones}{%
        Los nuevos valores de las \refElem{xp-visual-settings} deben ser actualizados
        para todos los usuarios y además deben persistirse en el sistema.
	}

	\UCitem{Reglas de negocio}{\imprimeUC{regla}}

	\UCitems{Errores}{%
        \Titem \UCerr{Err1}{%
        % CAUSA
            Los plugins del módulo de experiencia no se encuentran instalados,}{%
        % EFECTO
            No se presentan en el menú las opciones para acceder a la pantalla
            de la configuración y no se puede llevar a cabo el caso de uso.}

        \Titem \UCerr{Err2}{%
        % CAUSA
            La imagen no cumple con las restricciones de nombre de la regla de negocio
            \refElem{BR-E01},}{%
        % EFECTO
            No se remplaza la imagen de la configuración actual, se realizan
            las demás actualizaciones con los datos ingresados por el usuario,
            y se emite el mensage que indica que el nombre es inválido.}

        \Titem \UCerr{Err3}{%
        % CAUSA
            Ocurre un fallo durante la persistencia de la imagen en el sistema,}{%
        % EFECTO
            Se interrumpe la actualización de la imagen, se procede con las
            demás actualizaciones y se emite el mensaje de error conrrespondiente}
	}

 \UCsection[design]{Datos de Diseño}

	\UCitems[design]{Casos de Prueba}{%
        \Titem \refElem{CPC-E02-1}
        \Titem \refElem{CPI-E02-1a}
        \Titem \refElem{CPI-E02-1b}
	}

 \UCsection[admin]{Datos de Administración de Requerimiento}

	\UCitem[admin]{Observaciones}{%
        Ninguna
	}

\end{UseCase}

\subsubsection{Trayectorias del caso de uso}

\begin{UCtrayectoria}%
  \includeUC{CU-M01} \refErr{Err1}

  \Actor Presiona la opción {\bf \refElem{tExpSettingsVisual}} en la categoría
         \refElem{tExpCategoria}.
  \Sistema Obtiene el valor de si el módulo de experiencia está \refElem[activado]%
           {xp-general-settings.activated} o no. \refTray{A}
  \Sistema Obtiene los valores actuales de la configuración:
           \salida{xp-visual-settings.title},
           \salida{xp-visual-settings.description},
           \salida{xp-visual-settings.message},
           \salida{xp-visual-settings.colorLvl} y
           \salida{xp-visual-settings.colorBar}.

  \Sistema Carga la pantalla \refElem{IU-E03} estableciendo como valores por defecto
           las \refElem{tExpSettingsVisual} actuales obtenidas.

  \Actor Ingresa los valores de \entrada{xp-visual-settings.title},
         \entrada{xp-visual-settings.description} y
         \entrada{xp-visual-settings.message} para los campos requeridos.
         \label{CU-E02-1.formulario}
  \Actor Ingresa los valores para el \entrada{xp-visual-settings.colorLvl} y
         \entrada{xp-visual-settings.colorBar} seleccionando el color y la
         tonalidad mediante el \refElem{tSelectColor}. \refTray{B}
         \label{CU-E02-1.color}
  \Actor Presiona la opción {\bf Seleccione un archivo}. \refTray{C}
  \Sistema Despliega la pantalla \refElem{IU-M00a} como pantalla emergente
           \label{CU-E02-1.seleccion-archivo}

  \Actor Selecciona la opción {\it Subir un archivo} en el menu izquierdo de la
         pantalla emergente y posteriormente presiona el botón {\it Browse}.
  \Actor Selecciona el archivo de la \entrada{xp-visual-settings.image}.
  \Actor Presiona el botón {\bf Subir este archivo}.
  \Sistema Valida que el archivo tenga alguna de las extensiones de indicadas
           por la regla \regla{BR-E01}. \refTray{D}
  \Sistema Cierra la pantalla emergente y muestra el nombre del archivo seleccionado
           en la pantalla \refElem{IU-E03}.


  \Actor Presiona el botón {\bf Guardar Cambios}. \refTray{E} \label{CU-E02-1.validacion}

  \Sistema Valida que las opciones ingresadas cumplan con las restricciones
           especificadas en el modelo de información. \refTray{F}.

  \Sistema Valida que la \refElem{xp-visual-settings.image} proporcionada cumpla con
           las restricciones de nombre de archivo establecida por la regla
           \refElem{BR-E01}. \refErr{Err2}
  \Sistema Remplaza la imagen de los niveles por la imagen propocionada por el usuario,
           actualiza los demás valores de las configuraciones. \refErr{Err3}
  \Sistema Despliega la pantalla \refElem{IU-E03} con el mensaje que indica que
           los datos han sido actualizados exitosamente.
\end{UCtrayectoria}

\begin{UCtrayectoriaA}{A}{El módulo de experiencia no se encuentra activado}

  \Sistema Carga la pantalla \refElem{IU-E03a}.

  \Actor Presiona el botón {\bf Activar módulo de experiencia}
  \includeUC{CU-E02} a partir del paso \ref{CU-E02-ir-a-formulario},
                     para activar el módulo de experiencia.

  \Sistema Regresa al inicio de la trayectoria principal.

\end{UCtrayectoriaA}

\begin{UCtrayectoriaA}{B}{%
El \refElem{aAdministrador} desea especificar el valor del \refElem{tColor}
directamente}

    \Actor Ingresa el valor hexadecimal del color para el
           \refElem{xp-visual-settings.colorLvl} o
           \refElem{xp-visual-settings.colorBar}.

    \Sistema Continua en el paso \ref{CU-E02-1.color} de la trayectoria principal.

\end{UCtrayectoriaA}

\begin{UCtrayectoriaA}{C}{%
No desea cambiar la \refElem{xp-visual-settings.image} actual de los niveles}

    \Sistema Continua en el paso \ref{CU-E02-1.validacion} de la trayectoria principal
\end{UCtrayectoriaA}

\begin{UCtrayectoriaA}{D}{Cuando el archivo seleccionado es distinto de {\it png o jpg}}

  \Sistema Emite en una ventana emergente el mensaje {\it Error: ``El tipo de
           archivo \$EXT no se acepta.''} siendo {\it\$EXT} la extensión del
           archivo seleccionado.
  \Sistema Regresa al paso \ref{CU-E02-1.seleccion-archivo}.

\end{UCtrayectoriaA}

\begin{UCtrayectoriaA}[Fin del caso de uso]{E}{%
Desea cancelar la actualización de las configuraciones}

  \Actor Presiona el botón cancelar.
  \Sistema Redirige a la pantala \refElem{IU-M01}.

\end{UCtrayectoriaA}

\begin{UCtrayectoriaA}{F}{%
Alguno de los datos ingresados no cumple con las restricciones en el modelo de
información}

    \Sistema Imprime los mensajes de error abajo de los campos con valores incorrectos.

    \Actor Ingresa nuevamente los valores para los campos marcados como incorrectos.
    \Sistema Regresa al paso \ref{CU-E02-1.formulario}

\end{UCtrayectoriaA}

