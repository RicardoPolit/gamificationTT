
% \ucstEnEdicion     Al terminar una revisión/aprobación con observaciones 
%                    y al inicio del CU.
%
% \ucstEnRevision    Al terminar la edición del CU (version += 0.1).
% \ucstEnAprobacion  Al pasar la revision sin observaciones.
% \ucstAprobado      Al ser aprobado por el usuario (version += 1.0)

\begin{UseCase}[%
Autor/Daniel Ortega,%
Version/0.1,%
Estado/\ucstEnEdicion]%
%
{CU-E02-2}{Configurar sistema de experiencia}{%
%
 Permite al \refElem{aAdministrador} establecer y modificar las cantidades de puntos
 de experiencia que brindan los cursos en la plataforma y la forma en que aumenta
 la cantidad de experiencia requerida para pasar de un nivel al siguiente. Los cambios
 sobre estas configuraciones pueden afectar el nivel en que se encuentran los usuarios
 y cambiar la cantidad de experiencia que los cursos brindan.}

	\UCitem[control]{Revisor}{ Sin asignar }
	\UCitem[control]{Último cambio}{ \today }

 \UCsection{Atributos}

    \UCitem{Actor(es)}{%
        \refElem{aAdministrador}
    }

	\UCitems{Propósito}{%
        \Titem Permitir al administrador configurar el sistema de experiencia.
        \Titem Establecer o modificar la cantidad de experiencia que brindan los
               cursos.
        \Titem Establecer la cantidad de experiencia requerida para pasar el primer
               nivel usada como calculo para los demás niveles.
        \Titem Cambiar la forma en cómo se incrementa la cantidad de experiencia
               requerida para avanzar de un nivel a otro.
	}
	
	\UCitem{Entradas}{\imprimeUC{entrada}}

	\UCitems{Origen}{%
        \Titem Mouse
        \Titem Teclado
        \Titem Sistema (para los datos de los \refElem[usuarios gamificados]{xp-user})
	}

	\UCitem{Salidas}{\imprimeUC{salida}}

	\UCitems{Destino}{%
		\Titem \refElem{IU-E04}
	}
	
	\UCitems{Precondiciones}{%
        \Titem Que los plugins del módulo de experiencia se encuentren instalados
        \Titem El módulo de experiencia debe estár habilitado en el caso de uso
               \refElem{CU-E02}.
	}

	\UCitem{Postcondiciones}{%
        Los nuevos valores de las \refElem{xp-scheme-settings} deber ser
        estár actualizados para todos los usuarios, además de persistirse en el
        sistema.
	}

	\UCitem{Reglas de negocio}{\imprimeUC{regla}}

	\UCitems{Errores}{%
        \Titem \UCerr{Err1}{%
        % CAUSA
            Los plugins del módulo de experiencia no se encuentran instalados,}{%
        % EFECTO
            no se presentan las opciones en el menú y por lo tanto no se puede
            acceder a las configuraciones}
	}

 \UCsection[design]{Datos de Diseño}

	\UCitems[design]{Casos de Prueba}{%
        \Titem \refElem{CPC-E02-2a}
        \Titem \refElem{CPC-E02-2b}
        \Titem \refElem{CPC-E02-2c}
        \Titem \refElem{CPI-E02-2}
	}

 \UCsection[admin]{Datos de Administración de Requerimiento}

	\UCitem[admin]{Observaciones}{%
        Los cambios en las configuraciones del sistema de experiencia podrían hacer
        que los usuarios aumenten de nivel si la cantidad de experiencia acumulada del
        nivel es mayor a la nueva cantidad de experiencia correspondiente a dicho nivel.
        Se recomienda que los cambios se realicen cuando la cantidad alumnos que usen el
        sistema sea mínima.}

\end{UseCase}

\subsubsection{Trayectorias del caso de uso}

\begin{UCtrayectoria}%
%
  \includeUC{CU-M01} \refErr{Err1}

  \Actor Presiona la opción {\bf \refElem{tExpSettingsComportamiento}} en la categoría
         \refElem{tExpCategoria}. \refTray{A} 
  \Sistema Obtiene el valor de si el módulo de experiencia está \refElem[activado]%
           {xp-general-settings.activated} o no. \refTray{B} \label{CU-E02-2-loading}
  \Sistema Obtiene los valores actuales de la configuración del sistema de experiencia:
           \salida{xp-scheme-settings.increment},
           \salida{xp-scheme-settings.incrementValue},
           \salida{xp-scheme-settings.levelXP} y
           \salida{xp-scheme-settings.courseXP}.
  \Sistema Carga la pantalla \refElem{IU-E04} estableciendo como valores por defecto
           las \refElem{xp-scheme-settings} obtenidas en el anterior paso.
  %\sistema muestra un mensaje informando al \refelem{aadministrador} de que si 
  %         modifica la experiencia el nivel 1 o el tipo de incremento se alterarían
  %         la cantidad de experiencia requerida para subir de nivel.

  \Actor Especifica si el \entrada{xp-scheme-settings.increment} en la cantidad de
         experiencia de los niveles será {\it Lineal} o {\it Porcentual}.
  \Actor Ingresa el valor para el \entrada{xp-scheme-settings.incrementValue} con
         base en la regla \regla{BR-E03}.
  \Actor Ingresa los valores para la \entrada{xp-scheme-settings.levelXP} y la
         \entrada{xp-scheme-settings.courseXP}.
  \Actor Presiona la opción {\bf Guardar Cambios}. \refTray{C} \label{CU-E02-2-submit}

  \Sistema Valida que los valores ingresados por el usuario cumplan con las
           restricciones especificadas en el modelo de información.
  \Sistema Verifica que el \refElem{xp-scheme-settings.incrementValue} cumpla
           con la regla \refElem{BR-E03}. \refTray{D}
  \Sistema Actualiza los valores de las \refElem{xp-scheme-settings} con los
           ingresados por el usuario.

  % ACTUALIZACION DE NIVEL DE LOS USUARIOS
  \Sistema Obtiene la lista de los \refElem[usuarios gamificados]{xp-user} y por cada uno
           realiza las siguientes acciones.
  \Sistema Obtiene el \entrada{xp-user.level} y la cantidad de \entrada{xp-user.levelxp}.
  \Sistema Calcula la nueva cantidad de experiencia correspondiente al nivel del usuario con
           base en la regla \regla{BR-E04} o \regla{BR-E05} si el incremento es lineal o
           porcentual respectivamente.
  \Sistema - Si la cantidad de experiencia del usuario es menor a la experiencia correspondiente
           al nivel, entonces se procede al siguiente usuario. \refTray{E} \label{CU-E02-2-Usuarios}

  \Sistema Despliega la pantalla \refElem{IU-E04} con el mensaje de que los datos
           han sido actualizados exitosamente.

\end{UCtrayectoria}

\begin{UCtrayectoriaA}{A}{
El \refElem{aAdministrador} selecciona la categoría \refElem{tExpCategoria}}
  \Sistema Carga la pantalla \refElem{IU-E01}
  \Actor Regresa al paso \ref{CU-E02-2-loading}
\end{UCtrayectoriaA}

\begin{UCtrayectoriaA}{B}{
El módulo de experiencia no se encuentra activado}
  \Sistema Carga la pantalla \refElem{IU-E03a}.
  \Actor Presiona el botón {\bf Activar módulo de experiencia}
  \includeUC{CU-E02} a partir del paso \ref{CU-E02-ir-a-formulario},
                     para activar el módulo de experiencia.

  \Sistema Regresa al inicio de la trayectoria principal.

\end{UCtrayectoriaA}

\begin{UCtrayectoriaA}{C}{
El \refElem{aAdministrador} desea cancelar la modificación en el sistema de
experiencia}

  \Actor Presiona el botón {\bf Cancelar}.
  \Sistema Redirige a la pantalla \refElem{IU-M01}.
\end{UCtrayectoriaA}

\begin{UCtrayectoriaA}{D}{
Alguno de los valores ingresados por el usuario son incorrectos.}
  \Sistema Imprime los mensajes de error abajo de los campos con los valores
           incorrectos.
  \Actor Ingresa nuevamente los valores en los campos marcados como incorrectos.
  \Sistema Regresa al paso \ref{CU-E02-2-submit}.

\end{UCtrayectoriaA}

\begin{UCtrayectoriaA}{E}{%
La cantidad de \refElem{xp-user.levelxp} es mayor a la experiencia del nivel en el
que se encuentra}

  \Sistema Avanza al \refElem{xp-user} al siguiente nivel, usando los
           puntos de experiencia del mismo y establece el sobrante como
           la \refElem{xp-user.levelxp} correspondiente al nuevo nivel.
  \Sistema Repite el paso anterior hasta que la cantidad de experiencia
           del nivel del usuario sea menor que la del nivel.
  \Sistema Regresa al paso \ref{CU-E02-2-Usuarios}

\end{UCtrayectoriaA}
