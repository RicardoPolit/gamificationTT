
% \ucstEnEdicion     Al terminar una revisión/aprobación con observaciones 
%                    y al inicio del CU.
%
% \ucstEnRevision    Al terminar la edición del CU (version += 0.1).
% \ucstEnAprobacion  Al pasar la revision sin observaciones.
% \ucstAprobado      Al ser aprobado por el usuario (version += 1.0)

\begin{UseCase}[%
Autor/Daniel Ortega,%
Version/0.1,%
Estado/\ucstEnEdicion]%
%
{CU-E04}{Crear curso con experiencia}{%
%
 Permite al \refElem{aProfesor} .}

	\UCitem[control]{Revisor}{ Sin asignar }
	\UCitem[control]{Último cambio}{ \today }

 \UCsection{Atributos}

    \UCitem{Actor(es)}{%
        \refElem{aProfesor}, \refElem{aAdministrador}.
    }

	\UCitem{Propósito}{%
        Crear un curso con soporte para brindar puntos de experiencia conforme
        van completando las secciones del curso para que los alumnos los acumulen
        y para subir de nivel.
	}
	
	\UCitem{Entradas}{\imprimeUC{entrada}}

	\UCitems{Origen}{%
        \Titem Mouse
        \Titem Teclado
	}

	\UCitem{Salidas}{\imprimeUC{salida}}

	\UCitem{Destino}{%
		\refElem{IU-E06a}
	}
	
	\UCitems{Precondiciones}{%
        \Titem Los plugins correspondientes al módulo de experiencia deben de estar
               habilitados.
        \Titem El módulo de experiencia debe de estar habilitado por el
               \refElem{aAdministrador}.
	}

	\UCitem{Postcondiciones}{%
        Las configuraciones particulares para este curso deben ser almacenadas
        en el sistema.
	}

	\UCitem{Reglas de negocio}{\imprimeUC{regla}}

	\UCitems{Errores}{%
        \Titem \UCerr{Err1}{%
        % CAUSA
            Los plugins correspondientes al módulo de experiencia no se encuentran
            instalados,}{%
        % EFECTO
            no aparece la opción para el formato gamificado, termina el caso de uso.}

        \Titem \UCerr{Err2}{%
        % CAUSA
            El módulo de experiencia se encuentra deshabilitado en la plataforma,}{%
        % EFECTO
            se le solicita al \refElem{aAdministrador} que habilite la experiencia}
	}

	% \UCitem{Viene de}{% Indicar si el Caso de uso es primario o se extiende de otro. La mayoría se 
					  % extienden de Login.
		% EJEMPLO: \refIdElem{PY-CU1} o Caso de uso primario.
	% 	\TODO Especificar.
	% }	

 \UCsection[design]{Datos de Diseño}

	\UCitems[design]{Casos de Prueba}{%
        \Titem \refElem{CPC-E04} % Creado chingon
        \Titem \refElem{CPI-E04} % Módulo de experiencia dehabilitado
	}

 \UCsection[admin]{Datos de Administración de Requerimiento}

	\UCitem[admin]{Observaciones}{%
        Ninguna
	}

\end{UseCase}

\subsubsection{Trayectorias del caso de uso}

\begin{UCtrayectoria}%
  \includeUC{CU-M01} presionando la pestaña {\bf Cursos}

  \Actor Presiona la opción {\bf Gestionar cursos y categorías}.
  \Sistema Obtiene las \refElem[categorias]{mdl-course-category} y 
           \refElem{mdl-course.fullname} de los cursos presentes en la plataforma.
  \Sistema Muestra la pantalla \refElem{IU-M06}.

  \Actor Selecciona el botón nuevo curso de la categoría seleccionada por defecto.
  \Sistema Muestra la pantalla \refElem{IU-E06}.

  \Actor Introduce el \entrada{mdl-course.fullname}, \entrada{mdl-course.shortname} y
         la categoría a la cual pertenecerá el curso.

  \Actor Despliega la sección para el formato del curso.
  \Actor Selecciona el \entrada{xp-course.format} de curso gamificado {\it(gamedle)}.
         \refErr{Err1} \label{CU-E04-format}

  \Actor Especifica las opciones \entrada{xp-course.sections},
         \entrada{xp-course.hiddensections} y  \entrada{xp-course.coursedisplay}
         del formato de curso gamificado. \refTray{A} \refErr{Err2}

  \Actor Opcionalmente incluye valores en los demás campos opcionales del formulario.
  \Actor Presiona el botón {\bf Guardar cambios y mostrar}. \refTray{B} \refTray{C}
         \refTray{D} \label{CU-E04-submit}.

  \Sistema Obtiene los datos ingresados por el usuario.
  \Sistema Crea un curso \salida{mdl-course} junto con la cantidad de
           \salida[secciones]{mdl-course-sections} especificadas por el usuario.
  \Sistema Crea un \salida{xp-course} con los valores de  \salida{xp-course.hiddensections}
           y \salida{xp-course.coursedisplay} ingresados por el usuario.
  \Sistema Por cada una de las \salida[secciones]{xp-course.sections} del curso gamificado
           crea las \salida[secciones gamificadas]{xp-course-section} correspondientes.
  \Sistema Establece la \salida{xp-course-section.xp} de cada seccion del curso con base 
           en la regla de negocio \regla{BR-E08}.
  \Sistema Muestra la pantalla con \refElem{IU-E06a} con los datos del curso recien
           creado.

\end{UCtrayectoria}

\begin{UCtrayectoriaA}{A}{%
El módulo de experiencia se encuentra deshabilitado
}
  \Sistema Muestra el mensaje informando que el módulo de experiencia está 
           deshabilitado.
  \Actor Si es que desea continuar y crear un curso gamificado sin la opción
         de brindar experiencia, prosigue en el paso \ref{CU-E04-format}

\end{UCtrayectoriaA}

\begin{UCtrayectoriaA}{B}{%
Alguno de los campos introducidos por el usuario es erroneo.
}

  \Sistema Muestra los mensajes de error correspondientes en los campos que
           con datos inválidos.
  \Actor Ingresa de nuevo los campos que contiene errores.
  \Sistema Regresa al paso \ref{CU-E04-submit}

\end{UCtrayectoriaA}

\begin{UCtrayectoriaA}{C}{%
El usuario desea regresar a la pantalla \refElem{IU-M06}.
}

  \Actor Presiona el botón {\bf Guardar y regresar}

  \Sistema Redirige a la pantalla \refElem{IU-M06}

\end{UCtrayectoriaA}

\begin{UCtrayectoriaA}{D}{%
El usuario desea cancelar la creación del curso
}

  \Actor Presiona el botón {\bf Cancelar}
  \Sistema Redirige a la pantalla \refElem{IU-M06}

\end{UCtrayectoriaA}
