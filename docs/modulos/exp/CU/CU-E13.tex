
% \ucstEnEdicion     Al terminar una revisión/aprobación con observaciones
%                    y al inicio del CU.
%
% \ucstEnRevision    Al terminar la edición del CU (version += 0.1).
% \ucstEnAprobacion  Al pasar la revision sin observaciones.
% \ucstAprobado      Al ser aprobado por el usuario (version += 1.0)

\begin{UseCase}[%
Autor/Daniel Ortega,%
Version/0.1,%
Estado/\ucstEnEdicion]%
%
{CU-E13}{Eliminar usuario gamificado}{%
%
 Permite al \refElem{aAdministrador} que cuando se elimine a algún estudiante de moodle
 también se eliminen los datos correspondientes al módulo de experiencia de dicho
 usuario.}

	\UCitem[control]{Revisor}{ Sin asignar }
	\UCitem[control]{Último cambio}{ \today }

 \UCsection{Atributos}

    \UCitem{Actor(es)}{%
        \refElem{aAdministrador}
    }

	\UCitem{Propósito}{%
        Eliminar los datos de experiencia de un usuario gamificado cuando se vaya al
        usuario de moodle vinculado a este.
	}

	\UCitem{Entradas}{\imprimeUC{entrada}}

	\UCitems{Origen}{%
        \Titem Mouse
	}

	\UCitem{Salidas}{\imprimeUC{salida}}

	\UCitem{Destino}{%
		\refElem{IU-M05a}
	}

	\UCitems{Precondiciones}{%
        \Titem Los plugins correspondientes al módulo de experiencia deben de estar
               instalados.
        \Titem El \refElem{mdl-user} que se desea eliminar no debe haber sido
               eliminado previamente.
	}

	\UCitem{Postcondiciones}{%
        \Titem Los datos de experiencia vinculados al \refElem{xp-user} deben de ser
               removidos del sistema.
	}

	\UCitem{Reglas de negocio}{\imprimeUC{regla}}

	\UCitems{Errores}{%
        \Titem \UCerr{Err1}{%
        % CAUSA
            Los plugins correspondientes al módulo de experiencia no se encuentran
            instalados,}{%
        % EFECTO
            continúaen el penúltimo paso de la trayectoria principal}

        \Titem \UCerr{Err2}{%
        % CAUSA
            El usuario que se desea eliminar ha sido eliminado previamente,}{%
        % EFECTO
            no se muestra la entrada del estudiante en la pantalla \refElem{IU-M05a},
            termina el caso de uso}
	}

	% \UCitem{Viene de}{% Indicar si el Caso de uso es primario o se extiende de otro. La mayoría se
					  % extienden de Login.
		% EJEMPLO: \refIdElem{PY-CU1} o Caso de uso primario.
	% 	\TODO Especificar.
	% }

 \UCsection[design]{Datos de Diseño}

	\UCitems[design]{Casos de Prueba}{%
        \Titem \refElem{CPC-E13}
	}

 \UCsection[admin]{Datos de Administración de Requerimiento}

	\UCitem[admin]{Observaciones}{%
        Ninguna
	}

\end{UseCase}

\subsubsection{Trayectorias del caso de uso}

\begin{UCtrayectoria}%
%
  \includeUC{CU-M01} presionando la pestaña {\bf Usuarios}

  \Actor Presiona la opción {\bf Mirar lista de usuarios}
  \Sistema Obtiene el \salida{mdl-user.firstname}, \salida{mdl-user.lastname},
           \salida{mdl-user.email}, \salida{mdl-user.lastaccess},
           \salida{mdl-user.city} y \salida{mdl-user.country} de los usuarios
           presentes en moodle.

  \Sistema Despliega la información de los usuarios en la pantalla \refElem{IU-M05a}.

  \Actor Presiona el botón \IUEliminar correspondiente al usuario que desea eliminar.
         \refErr{Err2}.

  \Sistema Pide la confirmación para eliminar al usuario seleccionado mediante la
           pantalla \refElem{IU-M05b}.
  \Actor Presiona el botón {\bf Eliminar} \refTray{A}

  \Sistema Elimina los datos de experiencia obtenidos en los cursos gamificados en
           los que el usuario ha sido \entrada{xp-section-reward}.

  \Sistema Elimina los datos del \entrada{xp-user}.

  \Sistema Elimina al \entrada{mdl-user}.

  \Sistema Obtiene el \refElem{mdl-user.firstname}, \refElem{mdl-user.lastname},
           \refElem{mdl-user.email}, \refElem{mdl-user.lastaccess},
           \refElem{mdl-user.city} y \refElem{mdl-user.country} de los usuarios
           presentes en moodle.

  \Sistema Despliega la información de los usuarios en la pantalla \refElem{IU-M05a}.


\end{UCtrayectoria}

\begin{UCtrayectoriaA}{A}{%
El \refElem{aAdministrador} desea cancelar la eliminación del usuario seleccionado
}%

  \Actor Presiona el botón {\bf Cancelar}
  \Sistema Obtiene el \refElem{mdl-user.firstname}, \refElem{mdl-user.lastname},
           \refElem{mdl-user.email}, \refElem{mdl-user.lastaccess},
           \refElem{mdl-user.city} y \refElem{mdl-user.country} de los usuarios
           presentes en moodle.
  \Sistema Redirige a la pantalla \refElem{IU-M05a}.
\end{UCtrayectoriaA}
