
% \ucstEnEdicion     Al terminar una revisión/aprobación con observaciones
%                    y al inicio del CU.
%
% \ucstEnRevision    Al terminar la edición del CU (version += 0.1).
% \ucstEnAprobacion  Al pasar la revision sin observaciones.
% \ucstAprobado      Al ser aprobado por el usuario (version += 1.0)

\begin{UseCase}[%
Autor/Daniel Ortega,%
Version/0.1,%
Estado/\ucstEnRevision]%
%
{CU-E02}{Realizar configuraciones del módulo de experiencia}{%
%
 Permite al \refElem{aAdministrador} acceder a las configuraciones del esquema de
 experiencia para consultar y cambiar los aspectos generales del módulo de experiencia,
 los cuales consisten en habilitar/deshabilitar el esquema de experiencia además de
 habilitar/deshabilitar que los eventos proporcionen experiencia.}

	\UCitem[control]{Revisor}{ Sin asignar }
	\UCitem[control]{Último cambio}{ 21/OCT/19 }

 \UCsection{Atributos}

    \UCitem{Actor(es)}{%
        \refElem{aAdministrador}
    }

	\UCitems{Propósito}{%
        \Titem Habilitar/deshabilitar el módulo de experiencia.
        \Titem Habilitar/deshabilitar que los eventos proporcionen experiencia.
	}

%% BEGIN-BLOQUE PARA AGREGAR UNA REVISION ------------------------------------->
%% Copiar y descomentar este bloque por cada revision que se realice
%	\UCsection[control]{% Indicar la versión objeto de la revisión.
%		Revisión de la Versión \TODO X.X
%	}
%	\UCitem[control]{Revisó}{% Coloque el nombre de quien realizó la revisión
%		\TODO Especificar
%	}
%	\UCitem[control]{Fecha}{% Coloque la fecha de la revisión
%		% EJEMPLO: 21 de Septiembre de 2019.
%		\TODO Especificar
%	}
%	\UCitem[control]{Resultado}{% Las opciones son:
%								% Pendiente: se pasa a EnEdicion y se agregan las observaciones
%								% Aprobado: Se pasa a EnAprobacion.
%		\TODO Especificar
%	}
%	\UCitems[control]{Observaciones}{
%		% Agregar las observaciones resultado de la revision o la palabra ``Ninguna''
%		\Titem \TODO Agregar observaciones en cada viñeta, usar el comando \TODO %\TOCHK \DONE.
%	}
%% <------------------------------------------ END-BLOQUE PARA AGREGAR UNA REVISION

	\UCitems{Entradas}{\imprimeUC{entrada}}

	\UCitem{Origen}{%
        Mouse
	}

	\UCitems{Salidas}{\imprimeUC{salida}}

	\UCitem{Destino}{%
		\refElem{IU-M01}
	}

	\UCitem{Precondiciones}{%
        Que los plugins del módulo de experiencia se encuentren instalados
	}

	\UCitem{Postcondiciones}{%
        Los nuevos valores de las \refElem{xp-general-settings} deben de ser
        almacenados en el sistema.
	}

	\UCitem{Reglas de negocio}{%
		Ninguna
	}

	\UCitems{Errores}{%
        \Titem \UCerr{Err1}{%
        % CAUSA
            Los plugins del módulo de experiencia no se encuentran instalados,}{%
        % EFECTO
            No se presentan en el menu las opciones para modificar el esquema de %
            experiencia. No se puede llevar a cabo el caso de uso}
	}

	% \UCitem{Viene de}{% Indicar si el Caso de uso es primario o se extiende de otro. La mayoría se
					  % extienden de Login.
		% EJEMPLO: \refIdElem{PY-CU1} o Caso de uso primario.
	% 	\TODO Especificar.
	% }

 \UCsection[design]{Datos de Diseño}

	\UCitems[design]{Casos de Prueba}{%
        \Titem \refElem{CPC-E02}
	}

 \UCsection[admin]{Datos de Administración de Requerimiento}

	\UCitem[admin]{Observaciones}{%
        Ninguna
	}

\end{UseCase}

\subsubsection{Trayectorias del caso de uso}

\begin{UCtrayectoria}%
%
  \includeUC{CU-M01} \refErr{Err1}

  \Actor Presiona la opción {\bf \refElem{tExpSettingsGeneral}} en la categoría
         \refElem{tExpCategoria}. \refTray{A} \label{CU-E02-ir-a-formulario}
  \Sistema Obtiene el valor establecido de los \salida[eventos activados]%
           {xp-general-settings.events} que determina si los eventos brindan
           experiencia o no. \label{CU-E03-formulario} \refTray{B}
  \Sistema Obtiene el valor \salida{xp-general-settings.activated} que define si el
           módulo de experiencia está activado.
  \Sistema Carga la pantalla \refElem{IU-E02} estableciendo los valores por defecto del
           formulario con los valores obtenidos.

  \Actor Establece el valor \entrada{xp-general-settings.activated} para definir si
         el sistema de experiencia estará habilitado o deshabilitado.
         \label{CU-E02-settings}
  \Actor Establece el valor para los \entrada[eventos activados]%
         {xp-general-settings.events}, indicando si se les brindará soporte a que
         los eventos brinden experiencia.
  \Actor Presiona el botón {\bf Guardar cambios}.
  \Sistema Obtiene los valores indicados por el usuario y actualiza las
           \refElem{xp-general-settings}.
  \Sistema Carga la pantalla \refElem{IU-E02} con los valores actualizados.

\end{UCtrayectoria}

\begin{UCtrayectoriaA}{A}{%
El \refElem{aAdministrador} selecciona la categoría \refElem{tExpCategoria}}

    \Sistema Carga la pantalla \refElem{IU-E01}.
    \Sistema Regresa al paso \ref{CU-E02-ir-a-formulario}.
\end{UCtrayectoriaA}

\begin{UCtrayectoriaA}{B}{%
El \refElem{aAdministrador} está instalando el plugin y no hay configuración previa }

  \Sistema Obtiene los valores por defecto \refElem{xp-general-settings.activated} y
           \refElem{xp-general-settings.events} indicados en los archivos de instalación del
           plugin.

  \Sistema Carga la pantalla \refElem{IU-E02} estableciendo los valores obtenidos desde los
           archivos del plugin.
  \Sistema Regresa al paso \ref{CU-E02-settings}.
\end{UCtrayectoriaA}
