
% \ucstEnEdicion     Al terminar una revisión/aprobación con observaciones
%                    y al inicio del CU.
%
% \ucstEnRevision    Al terminar la edición del CU (version += 0.1).
% \ucstEnAprobacion  Al pasar la revision sin observaciones.
% \ucstAprobado      Al ser aprobado por el usuario (version += 1.0)

\begin{UseCase}[%
Autor/Daniel Ortega,%
Version/0.1,%
Estado/\ucstEnEdicion]%
%
{CU-E07}{Administrar experiencia de un curso}{%
%
 Permite al \refElem{aProfesor} establecer la cantidad de experiencia que cada una de las
 secciones de un curso gamificado brindará a los alumnos cuando estos la hayan completado}

	\UCitem[control]{Revisor}{ Sin asignar }
	\UCitem[control]{Último cambio}{ \today }

 \UCsection{Atributos}

    \UCitem{Actor(es)}{%
        \refElem{aProfesor}
    }

	\UCitems{Propósito}{%
        \Titem Permitirle al profesor especificar la cantidad de experiencia que cada
               sección del curso brindará.
        \Titem Permitirle al profesor ponderar de acuerdo a su criterio que secciones
               del curso deben brindar mayor o menor experiencia.
	}

	\UCitem{Entradas}{\imprimeUC{entrada}}

	\UCitems{Origen}{%
        \Titem Mouse
        \Titem Teclado
	}

	\UCitem{Salidas}{\imprimeUC{salida}}

	\UCitem{Destino}{%
		\refElem{IU-E06b}
	}

	\UCitems{Precondiciones}{%
        \Titem Los plugins del módulo de experiencia deben de estar habilitados
        \Titem El curso debe debe de tener el \refElem{xp-course.format} gamificado.
	}

	\UCitem{Postcondiciones}{%
        Los valores de experiencia para cada sección del curso deben ser
        almacenados en el sistema.
	}

	\UCitem{Reglas de negocio}{\imprimeUC{regla}}

	\UCitems{Errores}{%
        \Titem \UCerr{Err1}{%
        % CAUSA
            El curso elegido no es un curso gamificado con soporte para experiencia,}{%
        % EFECTO
            no se puede administrar la experiencia y termina el caso de uso}
	}

	% \UCitem{Viene de}{% Indicar si el Caso de uso es primario o se extiende de otro. La mayoría se
					  % extienden de Login.
		% EJEMPLO: \refIdElem{PY-CU1} o Caso de uso primario.
	% 	\TODO Especificar.
	% }

 \UCsection[design]{Datos de Diseño}

	\UCitems[design]{Casos de Prueba}{%
        \Titem \refElem{CPC-E07}
        \Titem \refElem{CPC-E07a}
        \Titem \refElem{CPI-E07}
	}

 \UCsection[admin]{Datos de Administración de Requerimiento}

	\UCitem[admin]{Observaciones}{%
        Ninguna
	}

\end{UseCase}

\subsubsection{Trayectorias del caso de uso}

\begin{UCtrayectoria}%
%
  \Actor Presiona el botón \IUMenu de la pantalla \refElem{IU-M00}
  \Sistema Despliega el menú de navegación lateral
  \Actor Selecciona la opción \IUHome {\bf Página Inicial del Sitio}

  \Sistema Obtiene el \salida{mdl-course.fullname} de los cursos disponibles en la
           plataforma.

  \Sistema Muestra la lista de cursos disponibles en la pantalla \refElem{IU-M07}.

  \Actor Selecciona el \entrada{mdl-course} del cual desea administrar su experiencia.
  \Sistema Obtiene el \entrada{mdl-course.fullname}, \entrada{mdl-course.shortname}
           así como el \salida[secciones]{mdl-course-section.name} de las secciones
           del curso junto con las \salida[actividades]{mdl-course-module}
           correspondientes a cada sección y el estado de \refElem[completitud]%
           {mdl-course-module-completion.completionstate}.

  \Sistema Muestra los datos obtenidos en la pantalla \refElem{IU-M07}.
           \label{CU-E07-pantalla}

  \Actor Presiona el botón \IUAdminSitio en la parte superior izquierda de la pantalla
  \Sistema Muestra el menú desplegable de la administración del curso

  \Actor Presiona el botón \IUEditar {\bf Activar Edición}. \refErr{Err1}
  \Sistema Obtiene la \salida{xp-course-section.xp} de la secciones gamificadas del
           curso y revisa hay \refElem[recompensas]{xp-section-reward} que hayan sido
           entregadas correspondientes a dicha sección.
  \Sistema Obtiene la cantidad total de \salida{xp-scheme-settings.courseXP}.
  \Sistema Muestra la experiencia de cada sección y el total de experiencia
           en la pantalla \refElem{IU-E06b}.
  \Sistema Habilita o deshabilita los campos para editar la experiencia de cada
           sección con base en la regla \regla{BR-E09}.

  \Actor Introduce la experiencia de las secciones de acuerdo con la regla
         \regla{BR-E10}. \refTray{A}
  \Actor Presiona el botón {\bf Guardar cambios}. \refTray{B} \label{CU-E07-submit}
  \Sistema Obtiene los valores de experiencia ingresados por el usuario
           correspondientes a las secciones del curso.
  \Sistema Valida que los valores de experiencia ingresados cumplan con las
           restricciones especificadas en el modelo de información. \refTray{C}
  \Sistema Valida que los valores de experiencia cumplan con la regla
           \refElem{BR-E10} \refTray{D}.
  \Sistema Actualiza los valores de experiencia correspondientes a cada sección.
           \label{CU-E07-finish}
  \Sistema Muestra la pantalla \refElem{IU-E06b} con la \refElem{xp-course-section.xp}
           actualizada de las secciones del curso.


\end{UCtrayectoria}

\begin{UCtrayectoriaA}{A}{%
Alguna de las secciones gamificadas ha sido completada por al menos un alumno.
}
  \Sistema Detecta que todas las \refElem{mdl-course-module} de una sección
           han sido completadas por al menos un \refElem{aEstudiante}.
  \Sistema Deshabilita los campos de edición en todas las secciones que entre
           en el anterior supuesto.
  \Sistema Continúa en el paso \ref{CU-E07-pantalla}.
\end{UCtrayectoriaA}

\begin{UCtrayectoriaA}{B}{%
El \refElem{aProfesor} desea distribuir uniformemente la experiencia del curso
disponible entre las distintas secciones de acuerdo con la regla.}

  \Actor Presiona el botón {\bf Distribuir uniformemente}
  \Sistema Divide la cantidad de experiencia entre las secciones editables
           de acuerdo con las reglas \refElem{BR-E10} y \regla{BR-E08}.
  \Sistema Continua en el paso \ref{CU-E07-finish} de la trayectoria principal.
\end{UCtrayectoriaA}

\begin{UCtrayectoriaA}{C}{%
Alguno de los campos para establecer la experiencia de las %
\refElem[secciones del curso]{xp-course-section} es incorrecto.}

  \Sistema Muestra el mensaje de error debajo de los campos con un valor de
           experiencia incorrecto.

  \Actor Ingresa de nuevo los valores de experiencia de los campos de experiencia
         incorrectos.
  \Sistema Regresa la paso \ref{CU-E07-submit} de la trayectoria principal
\end{UCtrayectoriaA}

\begin{UCtrayectoriaA}{D}{%
Los valores de experiencia introducidos por el \refElem{aProfesor} no cumplen con
la regla \refElem{BR-E09}.}

  \Sistema Muestra el mensaje de error de que la suma de la experiencia no es igual
           al total de experiencia del curso.
  \Actor Ingresa de nuevo los valores de experiencia correspondientes a cada sección
         del curso.
  \Sistema Regresa la paso \ref{CU-E07-submit} de la trayectoria principal
\end{UCtrayectoriaA}
