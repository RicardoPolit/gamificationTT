
% \ucstEnEdicion     Al terminar una revisión/aprobación con observaciones 
%                    y al inicio del CU.
%
% \ucstEnRevision    Al terminar la edición del CU (version += 0.1).
% \ucstEnAprobacion  Al pasar la revision sin observaciones.
% \ucstAprobado      Al ser aprobado por el usuario (version += 1.0)

\begin{UseCase}[%
Autor/Daniel Ortega,%
Version/0.1,%
Estado/\ucstEnEdicion]%
%
{CU-E05}{Habilitar el soporte para experiencia en un curso}{%
%
 Permite al \refElem{aProfesor} incluir el soporte para que las secciones del curso
 que administra brinden experiencia a los alumnos conforme estos las vayan completando.}

	\UCitem[control]{Revisor}{ Sin asignar }
	\UCitem[control]{Último cambio}{ \today }

 \UCsection{Atributos}

    \UCitem{Actor(es)}{%
        \refElem{aProfesor}
    }

	\UCitem{Propósito}{%
        Agregar soporte para brindar experiencia en cualquier curso que haya sido creado
        en moodle.
	}
	
	\UCitem{Entradas}{\imprimeUC{entrada}}

	\UCitems{Origen}{%
        \Titem Mouse
	}

	\UCitem{Salidas}{\imprimeUC{salida}}

	\UCitem{Destino}{%
		\refElem{IU-E06a}
	}
	
	\UCitems{Precondiciones}{%
        \Titem Los plugins correspondientes al módulo de experiencia deben de estar
               habilitados.
	}

	\UCitem{Postcondiciones}{%
        \Titem Se debe de brindar la experiencia correspondiente de las secciones que
               hayan completado los alumnos inscritos en el curso.
	}

	\UCitem{Reglas de negocio}{\imprimeUC{regla}}

	\UCitems{Errores}{%
        \Titem \UCerr{Err1}{%
        % CAUSA
            Los plugins corespondientes al módulo de experiencia no se encuentran
            instalados,}{%
        % EFECTO
            no aparece la opción para cambiar el formato del curso al formato gamificado}

        \Titem \UCerr{Err2}{%
        % CAUSA
            El módulo de experiencia se encuentra deshabilitado en la plataforma,}{%
        % EFECTO
            se le solicita al \refElem{aAdministrador} que habilite la experiencia}
    }

	% \UCitem{Viene de}{% Indicar si el Caso de uso es primario o se extiende de otro. La mayoría se 
					  % extienden de Login.
		% EJEMPLO: \refIdElem{PY-CU1} o Caso de uso primario.
	% 	\TODO Especificar.
	% }	

 \UCsection[design]{Datos de Diseño}

	\UCitems[design]{Casos de Prueba}{%
        \Titem \refElem{CPC-E05}
        \Titem \refElem{CPC-E05a}
	}

 \UCsection[admin]{Datos de Administración de Requerimiento}

	\UCitem[admin]{Observaciones}{%
        Hay distintas formas de llegar a la pantalla para editar las configuraciones de
        un curso, sin embargo este caso de uso solo contempla el acceso mediante la gestión
        de usuarios y categorias.
	}

\end{UseCase}

\subsubsection{Trayectorias del caso de uso}

\begin{UCtrayectoria}%
%
 \includeUC{CU-M01} presionando la pestaña {\bf Cursos}.

  \Actor Presiona la opción {\bf Gestionar cursos y categorías}.
  \Sistema Obtiene las \refElem[categorias]{mdl-course-category} y 
           \refElem{mdl-course.fullname} de los cursos presentes en la plataforma.
  \Sistema Muestra en la pantalla \refElem{IU-M06} la lista de cursos.
           \label{CU-E05-course-list}

  \Actor Presiona el botón \IUConfigurar del curso que desea editar.
  \Sistema Obtiene el \salida{mdl-course.fullname}, \salida{mdl-course.shortname} y
           \salida{mdl-course.format} y demás datos del curso que se desea editar.
  \Sistema Muestra los datos obtenidos del curso en la pantalla \refElem{IU-M05}.


  \Actor Despliega la sección para el formato del curso.
  \Actor Selecciona el \entrada{xp-course.format} de curso gamificado.
         \refErr{Err1}
  \Sistema Carga la información correspondiente al formato del curso gamificado
           como se muestra en la pantalla \refElem{IU-E06}.

  \Actor Especifica las opciones \entrada{xp-course.hiddensections} y
         \entrada{xp-course.coursedisplay} del formato de curso gamificado. \refTray{A}
         \refErr{Err2}

  \Actor Opcionalmente edita los demás campos opcionales del formulario.
  \Actor Presiona el botón {\bf Guardar Cambios y mostrar}. \refTray{B}, \refTray{C},
         \refTray{D} \label{CU-E05-submit}

  \Sistema Obtiene los datos ingresados por el usuario.
  \Sistema Crea un \salida{xp-course} con los valores de  \salida{xp-course.hiddensections}
           y \salida{xp-course.coursedisplay} ingresados por el usuario.
  \Sistema Actualiza los demas datos del \refElem{mdl-course}.
  \Sistema Por cada una de las \entrada[secciones]{xp-course.sections} del curso crea las
           \salida[secciones gamificadas]{xp-course-section} correspondientes.
  \Sistema Establece la \salida{xp-course-section.xp} de cada seccion del curso con base 
           en la regla de negocio \regla{BR-E08}.
  \Sistema Brinda la \refElem{xp-course-section.xp} de las secciones del curso
           gamificadas a los alumnos que las hayan completado. \label{CU-E05-finish}
  \Sistema Muestra la pantalla con \refElem{IU-E06a} con los datos del curso recien
           creado.

\end{UCtrayectoria}

\begin{UCtrayectoriaA}{A}{%
El módulo de experiencia se encuentra deshabilitado
}
  \Sistema Muestra el mensaje informando que el módulo de experiencia está deshabilitado.
  \Actor Si es que desea continuar cambiando el formato del curso por uno gamificado,
         prosigue en el paso \ref{CU-E05-submit}

\end{UCtrayectoriaA}

\begin{UCtrayectoriaA}{B}{%
Alguno de los campos introducidos por el usuario es erroneo.
}

  \Sistema Muestra los mensajes de error correspondientes en los campos que
           contienen datos inválidos.
  \Actor Ingresa de nuevo los campos que contiene errores.
  \Sistema Regresa al paso \ref{CU-E05-submit}

\end{UCtrayectoriaA}

\begin{UCtrayectoriaA}{C}{%
El usuario desea regresar a la pantalla \refElem{IU-M06}.
}

  \Actor Presiona el botón {\bf Guardar y regresar}
  \Sistema Ejecuta los pasos a partir del paso \ref{CU-E05-submit} hasta el
           paso \ref{CU-E05-finish}.
  \Sistema Redirige a la pantalla \refElem{IU-M06}

\end{UCtrayectoriaA}

\begin{UCtrayectoriaA}{D}{%
El usuario desea cancelar la creación del curso
}
  \Actor Presiona el botón {\bf Cancelar}
  \Sistema Redirige a la pantalla \refElem{IU-M06}

\end{UCtrayectoriaA}
