
% \ucstEnEdicion     Al terminar una revisión/aprobación con observaciones 
%                    y al inicio del CU.
%
% \ucstEnRevision    Al terminar la edición del CU (version += 0.1).
% \ucstEnAprobacion  Al pasar la revision sin observaciones.
% \ucstAprobado      Al ser aprobado por el usuario (version += 1.0)

\begin{UseCase}[%
Autor/Daniel Ortega,%
Version/0.1,%
Estado/\ucstEnEdicion]%
%
{CU-E05}{Habilitar el soporte para experiencia en un curso}{%
%
 Permite al \refElem{aProfesor} incluir el soporte para que las secciones del curso
 que administra brinden experiencia a los alumnos conforme estos las vayan completando.}

	\UCitem[control]{Revisor}{ Sin asignar }
	\UCitem[control]{Último cambio}{ \today }

 \UCsection{Atributos}

    \UCitem{Actor(es)}{%
        \refElem{aProfesor}
    }

	\UCitem{Propósito}{%
        Agregar soporte para brindar experiencia en cualquier curso que haya sido creado
        en moodle.
	}
	
	\UCitem{Entradas}{\imprimeUC{entrada}}

	\UCitems{Origen}{%
        \Titem Mouse
	}

	\UCitem{Salidas}{\imprimeUC{salida}}

	\UCitem{Destino}{%
		\refElem{IU-E06a}
	}
	
	\UCitems{Precondiciones}{%
        \Titem Los plugins correspondientes al módulo de experiencia deben de estar
               habilitados.
	}

	\UCitem{Postcondiciones}{%
        \Titem Se debe de brindar la experiencia correspondiente de las secciones que
               hayan completado los alumnos inscritos en el curso.
	}

	\UCitem{Reglas de negocio}{\imprimeUC{regla}}

	\UCitems{Errores}{%
        \Titem \UCerr{Err1}{%
        % CAUSA
            Los plugins corespondientes al módulo de experiencia no se encuentran
            instalados,}{%
        % EFECTO
            no aparece la opción para cambiar el formato del curso al formato gamificado}

        \Titem \UCerr{Err2}{%
        % CAUSA
            El módulo de experiencia se encuentra deshabilitado en la plataforma,}{%
        % EFECTO
            se le solicita al \refElem{aAdministrador} que habilite la experiencia}
    }

	% \UCitem{Viene de}{% Indicar si el Caso de uso es primario o se extiende de otro. La mayoría se 
					  % extienden de Login.
		% EJEMPLO: \refIdElem{PY-CU1} o Caso de uso primario.
	% 	\TODO Especificar.
	% }	

 \UCsection[design]{Datos de Diseño}

	\UCitems[design]{Casos de Prueba}{%
        \Titem \refElem{CPC-E05}
        \Titem \refElem{CPC-E05a}
	}

 \UCsection[admin]{Datos de Administración de Requerimiento}

	\UCitem[admin]{Observaciones}{%
        Ninguna
	}

\end{UseCase}

\clearpage
\subsubsection{Trayectorias del caso de uso}

\begin{UCtrayectoria}%
%
 \includeUC{CU-M01} presionando la pestaña {\bf Cursos}.

  \Actor Presiona la opción {\bf Gestionar cursos y categorías}.
  \Sistema Obtiene las \refElem[categorias]{mdl-course-category} y 
           \refElem{mdl-course.fullname} de los cursos presentes en la plataforma.
  \Sistema Muestra en la pantalla \refElem{IU-M06} la lista de cursos. \refTray{A}.
           \label{CU-E05-course-list}

  \Actor Presiona el botón  

\end{UCtrayectoria}

\begin{UCtrayectoriaA}{A}{%
El curso al cual el \refElem{aProfesor} desea brindar soporte para experiencia 
se encuentra en una categoria distinta a la mostrada por defecto.
}
  \Actor Selecciona la \refElem{course.category} a la que pertenece el curso
         que desea agregarle experiencia.

  \Sistema Obtiene las \refElem[categorias]{mdl-course-category} y 
           \refElem{mdl-course.fullname} de los cursos presentes en la plataforma.

  \Sistema Redirige al paso \ref{CU-E05-course-list} de la trayectoria principal.

\end{UCtrayectoriaA}

