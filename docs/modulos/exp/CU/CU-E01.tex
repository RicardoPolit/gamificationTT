
% \ucstEnEdicion     Al terminar una revisión/aprobación con observaciones
%                    y al inicio del CU.
%
% \ucstEnRevision    Al terminar la edición del CU (version += 0.1).
% \ucstEnAprobacion  Al pasar la revision sin observaciones.
% \ucstAprobado      Al ser aprobado por el usuario (version += 1.0)

\begin{UseCase}[%
Autor/Daniel Ortega,%
Version/0.1,%
Estado/\ucstEnRevision]%
%
{CU-E01}{Instalar plugin del módulo de experiencia}{% TODO; Deberia se Instalar/Actualizar ???
%
 Permite al \refElem{aAdministrador} incluir todas las funcionalidades que brinda el módulo de
 experiencia al moodle que administra mediante la instalación de los plugins correspondientes.
 La conclusión de la trayectoria principal de esta caso de uso es una precondición para que los
 demás casos de uso del módulo de experiencia puedan ejecutarse.}

	\UCitem[control]{Revisor}{ Sin asignar }
	\UCitem[control]{Último cambio}{ 13/OCT/19 }

 \UCsection{Atributos}

    \UCitem{Actor(es)}{%
        \refElem{aAdministrador}
    }

	\UCitems{Propósito}{%
        \Titem Permitir al administrador incluir todas las funcionalidades que brinda el módulo de
        experiencia al moodle que administra.

        \Titem Permitir a los usuarios de moodle ver su progreso en la plataforma mediante puntos
        de experiencia.
	}

	\UCitem{Entradas}{\imprimeUC{entrada}}

	\UCitems{Origen}{%
        \Titem Mouse
	}

	\UCitem{Salidas}{\imprimeUC{salida}}

	\UCitem{Destino}{%
		\refElem{IU-M01}
	}

	\UCitems{Precondiciones}{%
        \Titem La carpeta comprimida que contiene los archivos del plugin
        \Titem El plugin debe cumplir con la regla \refElem{BR-M01} para poder ser
               instalado.
        % \Titem Si se trata de una actualización de un plugin la versión de este debe
               % cumplir con la regla \refElem{BR-M02}.
	}

	\UCitems{Postcondiciones}{%
        \Titem El plugin debe permanecer instalado en moodle.%
        \Titem La actualización de las \refElem{xp-general-settings} del módulo de experiencia
               deben persistirse en el sistema.
        \Titem Los \refElem[usuarios]{mdl-user} registrados en moodle deberán tener la
               asociada la información de un \refElem{xp-user}.
	}

	\UCitem{Reglas de negocio}{\imprimeUC{regla}}

	\UCitems{Errores}{%
        \Titem \UCerr{Err1}{%
            El archivo zip del plugin seleccionado está roto,}{% CAUSA
            no se puede continuar con la instalación del plugin}% EFECTO

        \Titem \UCerr{Err2}{%
            Alguna de las dependencias del plugin no se satisface con
            los plugins instalados}{% CAUSE
            no se puede continuar con las instalación del plugin}% EFECTO

        \Titem \UCerr{Err3}{%
        % CAUSA
            Durante la ejecución de las tareas de instalación del nuevo plugin ocurre
            un error,}{%
        % EFECTO
            las tareas de instalación del propio plugin no pudieron concluir apropiadamente,
            el plugin sigue }
	}

	% \UCitem{Viene de}{% Indicar si el Caso de uso es primario o se extiende de otro. La mayoría se
					  % extienden de Login.
		% EJEMPLO: \refIdElem{PY-CU1} o Caso de uso primario.
	% 	\TODO Especificar.
	% }

 \UCsection[design]{Datos de Diseño}

	\UCitems[design]{Casos de Prueba}{%
        \Titem \refElem{CPC-E01}
	}

 \UCsection[admin]{Datos de Administración de Requerimiento}

	\UCitem[admin]{Observaciones}{}

\end{UseCase}

\subsubsection{Trayectorias del caso de uso}

\begin{UCtrayectoria}%
%
    \includeUC{CU-M01}

    \Actor Selecciona la opción {\bf Instalar plugins}
    \Sistema Carga la pantalla \refElem{IU-M02} con el formulario para seleccionar el
             plugin a instalar. \label{CU-E01-formulario-instalacion}

    \Actor Presiona la opción {\bf Seleccione un archivo}
    \Sistema Despliega la pantalla \refElem{IU-M00a} como pantalla emergente
             \label{CU-E01-seleccion-archivo}

    \Actor Selecciona la opción {\it Subir un archivo} en el menu izquierdo de la pantalla
           emergente y posteriormente presiona el botón {\it Browse}.
    \Actor Selecciona el archivo que contiene al \entrada{Plugin} del módulo de experiencia.
    \Actor Presiona el botón {\bf Subir este archivo}.
    \Sistema Valida que el archivo del plugin sea de tipo {\it ZIP}. \refTray{A}
    \Sistema Cierra la pantalla emergente y muestra el nombre del archivo seleccionado en la
             pantalla \refElem{IU-M02}

    \Actor Presiona el botón {\bf Instalar plugin desde archivo ZIP}
    \Sistema Valida que el archivo {\it ZIP} cumpla con las restricciones dictadas por la
             \regla{BR-M01}. \refErr{Err1}
    \Sistema Obtiene el \salida{Plugin.componente}, la \salida{Plugin.fullpath} y el
             \salida{Plugin.pluginname} del plugin a ser instalado.
    \Sistema Carga la pantalla \refElem{IU-M02a}, mostrando los datos anteriormente obtenidos.

    \Actor Continua con la instalación de plugin presionando la opción {\bf continuar}. \refTray{B}

    \Sistema Obtiene tambien la \salida{Plugin.moodle}, la lista de \salida{Plugin.dependencies}
             y la \salida{Plugin.version} del plugin a instalar. \refErr{Err2}
             \label{CU-E01-comprobacion}
    \Sistema Despliega los datos obtenidos en la pantalla \refElem{IU-M02b}

    \Actor Presiona el botón {\bf Actualizar base de datos Moodle ahora}. \refTray{C}
    \Sistema Procesa las tareas de instalación de moodle.

    \Sistema Obtiene la lista de los \refElem[identificadores]{mdl-user.id} de los
             usuarios de moodle.
    \Sistema Asocia mediante los identificadores datos de un \refElem{xp-user}
             estableciendo el \refElem{xp-user.level} actual igual a $1$,
             la \refElem{xp-user.xp} igual a $0$.
    \Sistema Establece los valores por defecto para las \refElem{xp-visual-settings} (
              \entrada{xp-visual-settings.title},
              \entrada{xp-visual-settings.description},
              \entrada{xp-visual-settings.message},
              \entrada{xp-visual-settings.colorLvl},
              \entrada{xp-visual-settings.colorBar} e
              \entrada{xp-visual-settings.image}), especificadas en el modelo de información.

    \Sistema Carga la interfaz \refElem{IU-M02d} informando que la instalación
             ha sido llevaba a cabo de forma correcta. \refErr{Err3}

    \Actor Presiona el botón {\bf continuar}.

    \includeUC{CU-E02} a partir del paso \ref{CU-E03-formulario}
\end{UCtrayectoria}

\begin{UCtrayectoriaA}{A}{Cuando el archivo seleccionado es distinto de un ZIP}

  \Sistema Emite en una ventana emergente el mensaje {\it Error: ``El tipo de
           archivo \$EXT no se acepta.''} siendo {\it\$EXT} la extensión del
           archivo seleccionado.
  \Sistema Regresa al paso \ref{CU-E01-seleccion-archivo}.

\end{UCtrayectoriaA}

\begin{UCtrayectoriaA}[Fin del caso de uso]%
{B}{El \refElem{aAdministrador} desea cancelar la instalación después de la validación del archivo ZIP}

    \Actor Presiona el botón {\bf cancelar} de la pantalla \refElem{IU-M02a}.
    \Sistema Cancela la instalación del plugin y redirige a la pantalla \refElem{IU-M02}

\end{UCtrayectoriaA}

\begin{UCtrayectoriaA}[Fin del caso de uso]%
{C}{El \refElem{aAdministrador} desea cancelar la instalación después de ver la comprobación de plugins a instalar}

    \Actor Presiona el botón {\bf cancelar esta instalación} o {\bf cancelar las nuevas instalaciones}
    \Sistema Redirige a la pantalla \refElem{IU-M02c}

    \Actor Si el actor presiona el botón {\bf Continuar} entonces
    \UCpaso[--] el caso de uso terminará, (en caso contrario)

    \Actor Si el actor presiona el botón {\bf Cancelar}
    \Sistema Regresa al paso \ref{CU-E01-comprobacion}
\end{UCtrayectoriaA}

%\subsubsection{Puntos de extensión}

%\UCExtensionPoint{Nombre del punto de extensión}{%

%    El \refElem{aAdministrador} desea/requiere/necesita ....%
%
%    }{En el paso \ref{CU-ET-1x} de la trayectoria principal  ...%
%
%    }{\refElem{CU-E2-T}}
