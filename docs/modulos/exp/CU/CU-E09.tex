% \ucstEnEdicion     Al terminar una revisión/aprobación con observaciones
%                    y al inicio del CU.
%
% \ucstEnRevision    Al terminar la edición del CU (version += 0.1).
% \ucstEnAprobacion  Al pasar la revision sin observaciones.
% \ucstAprobado      Al ser aprobado por el usuario (version += 1.0)

\begin{UseCase}[%
Autor/Daniel Ortega,%
Version/0.1,%
Estado/\ucstEnEdicion]%
%
{CU-E09}{Eliminar sección con experiencia}{%
%
 Permite al \refElem{aProfesor} eliminar una seccion de un curso gamificado con soporte
 para experiencia con base en la regla \refElem{BR-E11}, la cual indica para eliminar una
 sección de un curso gamificado esta debe brindar 0 puntos de experiencia.}

	\UCitem[control]{Revisor}{ Sin asignar }
	\UCitem[control]{Último cambio}{ \today }

 \UCsection{Atributos}

    \UCitem{Actor(es)}{%
        \refElem{aProfesor}
    }

	\UCitems{Propósito}{%
        \Titem Permitirle al profesor eliminar una sección de un curso con soporte
               para experiencia.
        \Titem Permitirle al profesor eliminar secciones del curso de manera controlada
               sin afectar la cantidad de experiencia total que brinda el curso.
	}

	\UCitem{Entradas}{\imprimeUC{entrada}}

	\UCitems{Origen}{%
        \Titem Mouse
	}

	\UCitem{Salidas}{\imprimeUC{salida}}

	\UCitem{Destino}{%
		\refElem{IU-E06b}
	}

	\UCitems{Precondiciones}{%
        \Titem El curso debe debe de tener el \refElem{xp-course.format} gamificado.
	}

	\UCitem{Postcondiciones}{%
        Las eliminación de las secciones de experiencia debe permanecer en el sistema.
	}

	\UCitem{Reglas de negocio}{\imprimeUC{regla}}

	\UCitems{Errores}{%
        \Titem \UCerr{Err1}{%
        % CAUSA
            El curso elegido no es un curso gamificado con soporte para experiencia,}{%
        % EFECTO
            termina el caso de uso.}
        \Titem \UCerr{Err2}{%
        % CAUSE
            La sección que se desea eliminar ya ha sido completada por alguno
            de los \refElem{aAlumno},}{%
        % EFECTO
            la opción de eliminar sección no se muestra, termina el caso de uso.}
	}

	% \UCitem{Viene de}{% Indicar si el Caso de uso es primario o se extiende de otro. La mayoría se
					  % extienden de Login.
		% EJEMPLO: \refIdElem{PY-CU1} o Caso de uso primario.
	% 	\TODO Especificar.
	% }

 \UCsection[design]{Datos de Diseño}

	\UCitems[design]{Casos de Prueba}{%
        \Titem \refElem{CPC-E08}
        \Titem \refElem{CPI-E08}
	}

 \UCsection[admin]{Datos de Administración de Requerimiento}

	\UCitem[admin]{Observaciones}{%
        Ninguna
	}

\end{UseCase}

\subsubsection{Trayectorias del caso de uso}

\begin{UCtrayectoria}%
%
  \Actor Presiona el botón \IUMenu de la pantalla \refElem{IU-M00}
  \Sistema Despliega el menú de navegación lateral
  \Actor Selecciona la opción \IUHome {\bf Página Inicial del Sitio}

  \Sistema Obtiene el \salida{mdl-course.fullname} de los cursos disponibles en la
           plataforma.

  \Sistema Muestra la lista de cursos disponibles en la pantalla \refElem{IU-M07}.

  \Actor Selecciona el \entrada{mdl-course} al cual desea agregar una sección con
         soporte de gamificación.
  \Sistema Obtiene el \entrada{mdl-course.fullname}, \entrada{mdl-course.shortname}
           así como el \salida[secciones]{mdl-course-section.name} de las secciones
           del curso junto con las \salida[actividades]{mdl-course-module}
           correspondientes a cada sección y el estado de \refElem[completitud]%
           {mdl-course-module.completionstate}.

  \Sistema Muestra los datos obtenidos en la pantalla \refElem{IU-M07}.
           \label{CU-E07-pantalla}

  \Actor Presiona el botón \IUAdminSitio en la parte superior izquierda de la pantalla
  \Sistema Muestra el menú desplegable de la administración del curso

  \Actor Presiona el botón \IUEditar {\bf Activar Edición}. \refErr{Err1}
  \Sistema Obtiene la \salida{xp-course-section.xp} de la secciones gamificadas del
           curso y revisa hay \refElem[recompensas]{xp-section-reward} que hayan sido
           entregadas correspondientes a dicha sección.
  \Sistema Obtiene la cantidad total de \salida{xp-scheme-settings.courseXP}.
  \Sistema Muestra la experiencia de cada sección y el total de experiencia
           en la pantalla \refElem{IU-E06b}.
  \Sistema Habilita o deshabilita los campos para editar la experiencia de cada
           sección con base en la regla \regla{BR-E09}.

  \Actor Presiona la opción {\bf Editar} correspondiente a la \entrada{xp-course-section}
         sección que desea eliminar. \label{CU-E09-options}.

  \Sistema Muestra el menu acciones de edición que se pueden realizar a dicha sección de
           acuerdo con la regla \regla{BR-E11}. \refTray{A} \refErr{Err2}

  \Actor Selecciona la opción \IUEliminar {\bf Eliminar sección}.

  \Sistema Elimina la \refElem{mdl-course-section} especificada, así como la
           \refElem{xp-course-section}.
  \Sistema Muestra la pantalla \refElem{IU-E06b} con los datos de las secciones restantes
           del curso.

\end{UCtrayectoria}

\begin{UCtrayectoriaA}{A}{%
La cantidad de \refElem{xp-course-section.xp} de la sección a eliminar es distinta de cero.
}
  \Sistema Redirige al paso \ref{CU-E09-options} de la trayectoria principal.
\end{UCtrayectoriaA}
