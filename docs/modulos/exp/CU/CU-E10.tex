
% \ucstEnEdicion     Al terminar una revisión/aprobación con observaciones 
%                    y al inicio del CU.
%
% \ucstEnRevision    Al terminar la edición del CU (version += 0.1).
% \ucstEnAprobacion  Al pasar la revision sin observaciones.
% \ucstAprobado      Al ser aprobado por el usuario (version += 1.0)

\begin{UseCase}[%
Autor/Daniel Ortega,%
Version/0.1,%
Estado/\ucstEnEdicion]%
%
{CU-E10}{Eliminar un curso gamificado}{%
%
 Permite al \refElem{aAdministrador} eliminar un curso gamificado ...}

	\UCitem[control]{Revisor}{ Sin asignar }
	\UCitem[control]{Último cambio}{ \today }

 \UCsection{Atributos}

    \UCitem{Actor(es)}{%
        \refElem{aAdministrador}
    }

	\UCitem{Propósito}{%
        Permitir al administrador eliminar un curso que está gamificado junto
        con los datos de gamificación que este ha otorgado.
	}
	
	\UCitem{Entradas}{\imprimeUC{entrada}}

	\UCitems{Origen}{%
        \Titem Mouse
	}

	\UCitem{Salidas}{\imprimeUC{salida}}

	\UCitem{Destino}{%
		\refElem{IU-M02a}
	}
	
	\UCitems{Precondiciones}{%
        \Titem El curso que se desea eliminar no debe haber sido eliminado
               anteriormente.
	}

	\UCitem{Postcondiciones}{%
        Ninguna
	}

	\UCitem{Reglas de negocio}{\imprimeUC{regla}}

	\UCitems{Errores}{%
        \Titem \UCerr{Err1}{%
        % CAUSA
            El curso que se desea eliminar ha sido eliminado previamente,}{%
        % EFECTO
            termina el caso de uso.}
	}

	% \UCitem{Viene de}{% Indicar si el Caso de uso es primario o se extiende de otro. La mayoría se 
					  % extienden de Login.
		% EJEMPLO: \refIdElem{PY-CU1} o Caso de uso primario.
	% 	\TODO Especificar.
	% }	

 \UCsection[design]{Datos de Diseño}

	\UCitems[design]{Casos de Prueba}{%
        \Titem \refElem{CPC-E05}
        \Titem \refElem{CPC-E05a}
	}

 \UCsection[admin]{Datos de Administración de Requerimiento}

	\UCitem[admin]{Observaciones}{%
        Ninguna
	}

\end{UseCase}

\clearpage
\subsubsection{Trayectorias del caso de uso}

\begin{UCtrayectoria}%
%
  \Actor Presiona la opción {\bf Gestionar cursos y categorías}.
  \Sistema Obtiene las \salida[categorias]{mdl-course-category} y 
           \salida{mdl-course.fullname} de los cursos presentes en la plataforma.
  \Sistema Muestra la pantalla \refElem{IU-M06}. \refErr{Err1}

  \Actor Presiona el botón \IUEliminar correspondiente al curso que desea eliminar.
         \label{CU-E05-delete-button}. \refTray{A}
  \Sistema Despliega la pantalla \refElem{IU-M06a} pidiendo la confirmación del 
           usuario para continuar con la eliminación.

  \Actor Presiona la opción {\bf Eliminar}. \refTray{B}
  \Sistema Elimina las \entrada{xp-section-reward} de experiencia que se han 
           entregado respetando la regla \regla{BR-E02}.
  \Sistema Elimina las \entrada[secciones gamificadas]{xp-course-section} del curso.
  \Sistema Elimina las opciones del \entrada{xp-course.format}.
  \Sistema Procede con las demás tareas de eliminación para concluir con la 
           eliminación del curso.
  \Sistema Muestra la pantalla \refElem{IU-M06b}.

  \Actor Presiona el botón {\bf Continuar}.
  \Sistema Obtiene las \refElem[categorias]{mdl-course-category} y 
           \refElem{mdl-course.fullname} de los cursos restantes en la plataforma.
  \Sistema Muestra la pantalla \refElem{IU-M06}.
\end{UCtrayectoria}

\begin{UCtrayectoriaA}{A}{%
El \refElem{mdl-course} que el \refElem{aAdministrador} desea eliminar se encuentra 
en otra categoría diferente a la seleccionada por defecto.}

  \Actor Selecciona la \refElem{mdl-course-category} donde se encuentra el curso
         que desea eliminar.

  \Sistema Obtiene las \refElem{mdl-course.fullname} de los cursos pertenecientes a
           dichas categorías.
  \Sistema Regresa al paso \ref{CU-E05-delete-button}.
\end{UCtrayectoriaA}

\begin{UCtrayectoriaA}{B}{%
El \refElem{aAdministrador} desea cancelar la eliminación del curso.}

  \Actor Presiona el botón {\bf Cancelar}
  \Sistema Obtiene las \refElem[categorias]{mdl-course-category} y 
           \refElem{mdl-course.fullname} de los cursos presentes en la plataforma.
\end{UCtrayectoriaA}
