
% \ucstEnEdicion     Al terminar una revisión/aprobación con observaciones 
%                    y al inicio del CU.
%
% \ucstEnRevision    Al terminar la edición del CU (version += 0.1).
% \ucstEnAprobacion  Al pasar la revision sin observaciones.
% \ucstAprobado      Al ser aprobado por el usuario (version += 1.0)

\begin{UseCase}[%
Autor/Daniel Ortega,%
Version/0.1,%
Estado/\ucstEnEdicion]%
%
{CU-E08}{Administración }{%
%
 Permite al \refElem{aProfesor} agregar una sección con soporte para experiencia
 en un curso gamificado y que pueda configurar la cantidad de experiencia que brinda
 dicha sección mediante el caso de uso \refElem{CU-E07}.}

	\UCitem[control]{Revisor}{ Sin asignar }
	\UCitem[control]{Último cambio}{ \today }

 \UCsection{Atributos}

    \UCitem{Actor(es)}{%
        \refElem{aProfesor}
    }

	\UCitems{Propósito}{%
        \Titem Permitirle al profesor agregar una sección con soporte para experiencia
               en curso gamificado.
        \Titem Permitirle al profesor tener la misma funcionalidad de agregar nuevas
               secciones en como en un curso normal en moodle.
	}
	
	\UCitem{Entradas}{\imprimeUC{entrada}}

	\UCitems{Origen}{%
        \Titem Mouse
        \Titem Teclado
	}

	\UCitem{Salidas}{\imprimeUC{salida}}

	\UCitem{Destino}{%
		\refElem{IU-E06b}
	}
	
	\UCitems{Precondiciones}{%
        \Titem El módulo de experiencia debe estar habilitado.
        \Titem El curso debe debe de tener el \refElem{xp-course.format} gamificado.
	}

	\UCitem{Postcondiciones}{%
        Los valores de experiencia para cada sección del curso deben ser
        almacenados en el sistema.
	}

	\UCitem{Reglas de negocio}{\imprimeUC{regla}}

	\UCitems{Errores}{%
        \Titem \UCerr{Err1}{%
        % CAUSA
            El curso elegido no es un curso gamificado con soporte para experiencia,}{%
        % EFECTO
            no se puede crear una sección con experiencia y termina el caso de uso}
	}

	% \UCitem{Viene de}{% Indicar si el Caso de uso es primario o se extiende de otro. La mayoría se 
					  % extienden de Login.
		% EJEMPLO: \refIdElem{PY-CU1} o Caso de uso primario.
	% 	\TODO Especificar.
	% }	

 \UCsection[design]{Datos de Diseño}

	\UCitems[design]{Casos de Prueba}{%
        \Titem \refElem{CPC-E08}
	}

 \UCsection[admin]{Datos de Administración de Requerimiento}

	\UCitem[admin]{Observaciones}{%
        Ninguna
	}

\end{UseCase}

\subsubsection{Trayectorias del caso de uso}

\begin{UCtrayectoria}%
%
  \Actor Presiona el botón \IUMenu de la pantalla \refElem{IU-M00}
  \Sistema Despliega el menú de navegación lateral
  \Actor Selecciona la opción \IUHome {\bf Página Inicial del Sitio}

  \Sistema Obtiene el \salida{mdl-course.fullname} de los cursos disponibles en la
           plataforma.

  \Sistema Muestra la lista de cursos disponibles en la pantalla \refElem{IU-M07}.

  \Actor Selecciona el \entrada{mdl-course} al cual desea agregar una sección con
         soporte de gamificación.
  \Sistema Obtiene el \entrada{mdl-course.fullname}, \entrada{mdl-course.shortname}
           así como el \salida[secciones]{mdl-course-section.name} de las secciones 
           del curso junto con las \salida[actividades]{mdl-course-module} 
           correspondients a cada sección y el estado de \refElem[completitud]%
           {mdl-course-module.completionstate}.

  \Sistema Muestra los datos obtenidos en la pantalla \refElem{IU-M07}.
           \label{CU-E07-pantalla}

  \Actor Presiona el botón \IUAdminSitio en la parte superior izquierda de la pantalla
  \Sistema Muestra el menu desplegable de la administración del curso

  \Actor Presiona el botón \IUEditar {\bf Activar Edición}. \refErr{Err1}
  \Sistema Obtiene la \salida{xp-course-section.xp} de la secciones gamificadas del 
           curso y revisa hay \refElem[recompensas]{xp-section-reward} que hayan sido
           entregadas correspondientes a dicha sección.
  \Sistema Obtiene la cantidad total de \salida{xp-scheme-settings.courseXP}.
  \Sistema Muestra la experiencia de cada sección y el total de experiencia 
           en la pantalla \refElem{IU-E06b}.
  \Sistema Habilita o deshabilita los campos para editar la experiencia de cada 
           sección con base en la regla \regla{BR-E09}.

  \Actor Presiona el botón \IUAnadir {\bf Misión}.
  \Sistema Crea una sección \entrada{mdl-course-section} perteneciente al
           \refElem{mdl-course}.
  \Sistema Crea uan sección \entrada{xp-course-section} asociada a la śección
           recientemente creada, indicando que dicha sección tiene cero puntos de
           \refElem{xp-course.section.xp}.
  \Sistema Muestra la pantalla \refElem{IU-E06b} con la \refElem{xp-course-section.xp}
           actualizada incluyendo los datos de la nueva sección creada.

\end{UCtrayectoria}

