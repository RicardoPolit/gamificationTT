
% \ucstEnEdicion     Al terminar una revisión/aprobación con observaciones
%                    y al inicio del CU.
%
% \ucstEnRevision    Al terminar la edición del CU (version += 0.1).
% \ucstEnAprobacion  Al pasar la revision sin observaciones.
% \ucstAprobado      Al ser aprobado por el usuario (version += 1.0)

\begin{UseCase}[%
Autor/Daniel Ortega,%
Version/0.1,%
Estado/\ucstEnEdicion]%
%
{CU-E02-3}{Configurar los eventos que entregan experiencia}{%
%
 Permite al \refElem{aAdministrador} elegir cuales eventos de los soportados
 brindarán experiencia y cuales no. Además de aquellos eventos que brindan
 experiencia puede especificar la cantidad de experiencia que estos entregan.}

	\UCitem[control]{Revisor}{ Sin asignar }
	\UCitem[control]{Último cambio}{ \today }

 \UCsection{Atributos}

    \UCitem{Actor(es)}{%
        \refElem{aAdministrador}
    }

	\UCitems{Propósito}{%
        \Titem Configurar que eventos proporcionan experiencia.
        \Titem Establecer la cantidad de experiencia de los eventos
               otorgarán.
        \Titem Deshabilitar eventos para que dejen de brindar
               experiencia.
	}

	\UCitem{Entradas}{\imprimeUC{entrada}}

	\UCitems{Origen}{%
        \Titem Mouse
        \Titem Teclado
	}

	\UCitem{Salidas}{\imprimeUC{salida}}

	\UCitems{Destino}{%
		\Titem \refElem{IU-E05}
	}

	\UCitems{Precondiciones}{%
        \Titem Los plugins correspondientes al módulo de experiencia deben
               de estar previamente instalados.
        \Titem El módulo de experiencia debe estar habilitado mediante el caso
               de uso \refElem{CU-E02}.
        \Titem Los eventos con experiencia deben estar habilitados mediante el caso
               de uso \refElem{CU-E02}.
	}

	\UCitem{Postcondiciones}{%
        Los nuevos valores para los eventos habilitados para dar experiencia deben
        de ser actualizados y persistirse en el sistema.
	}

	\UCitem{Reglas de negocio}{Ninguna}

	\UCitems{Errores}{%
        \Titem \UCerr{Err1}{%
        % CAUSA
            Los plugins del módulo de experiencia no se encuentran instalados,}{%
        % EFECTO
            no se presentan las opciones en el menú y por lo tanto no se acceder
            a las configuraciones}
	}

	% \UCitem{Viene de}{% Indicar si el Caso de uso es primario o se extiende de otro. La mayoría se
					  % extienden de Login.
		% EJEMPLO: \refIdElem{PY-CU1} o Caso de uso primario.
	% 	\TODO Especificar.
	% }

 \UCsection[design]{Datos de Diseño}

	\UCitems[design]{Casos de Prueba}{%
        \Titem \refElem{CPC-E02-3}
        \Titem \refElem{CPI-E02-3}
	}

 \UCsection[admin]{Datos de Administración de Requerimiento}

	\UCitem[admin]{Observaciones}{%
        Ninguna
	}

\end{UseCase}

\clearpage
\subsubsection{Trayectorias del caso de uso}

\begin{UCtrayectoria}%
%
  \includeUC{CU-M01} \refErr{Err1}

  \Actor Presiona la opción {\bf\refElem{tExpSettingsEventos}} en la categoría
         \refElem{tExpCategoria}. \refTray{A}

  \Sistema Obtiene el valor de si el módulo de experiencia está \refElem[activado]%
           {xp-general-settings.activated} o no. \refTray{B} \label{CU-E02-3-activated}

  \Sistema Obtiene el valor de si la funcionalidad de que los \refElem[eventos]%
           {xp-general-settings.events} otorgen experiencia está activada o no.
           \refTray{C} \label{CU-E02-3-events}

  \Sistema Obtiene los valores actuales de la configuración de los eventos con experiencia:
           \salida{xp-events-settings.competence} y \salida{xp-events-settings.competenceXP}
  \Sistema Carga la pantalla \refElem{IU-E05} estableciendo como valores por defecto
           las \refElem{xp-events-settings} obtenidas en el anterior paso.

  \Actor Habilita los eventos que desea que otorgen experiencia:
         \entrada{xp-events-settings.competence}
  \Actor Especifica la \entrada[experiencia del evento de competencia ganada]%
         {xp-events-settings.competence}
  \Actor Presiona el botón {\bf Guardar Cambios} \refTray{D} \label{CU-E02-3-submit}

  \Sistema Valida que los datos ingresados por el usuario cumplan con las restricciones
           de acuerdo con el modelo de información. \refTray{E}
  \Sistema Actualiza los datos de la configuración de eventos en el sistema
  \Sistema Muestra la pantalla \refElem{IU-E05} con el mensaje de que los datos han sido
           actualizados correctamente.

\end{UCtrayectoria}


\begin{UCtrayectoriaA}{A}{%
El \refElem{aAdministrador} selecciona la categoría \refElem{tExpCategoria}.
}
  \Sistema Carga a pantalla \refElem{IU-E01}.
  \Actor Regresa a paso \ref{CU-E02-3-activated}
\end{UCtrayectoriaA}

\begin{UCtrayectoriaA}{B}{%
El módulo de experiencia no se encuentra activado.
}
  \Sistema Carga la pantalla \refElem{IU-E03a}.
  \Actor Presiona el botón {\bf Activar módulo de experiencia}
  \includeUC{CU-E02} a partir del paso \ref{CU-E02-ir-a-formulario},
                     para activar el módulo de experiencia.

  \Sistema Regresa al inicio de la trayectoria principal.
\end{UCtrayectoriaA}

\begin{UCtrayectoriaA}{C}{%
La opción de que los eventos brinden experiencia no se encuentra activada.
}
  \Sistema Carga la pantalla \refElem{IU-E05a}.
  \Actor Presiona el botón {\bf Habilitar Eventos}
  \includeUC{CU-E02} a partir del paso \ref{CU-E02-ir-a-formulario},
                     para activar el módulo de experiencia.

  \Sistema Regresa al inicio de la trayectoria principal.
\end{UCtrayectoriaA}

\begin{UCtrayectoriaA}{D}{%
El \refElem{aAdministrador} desea cancelar la modificación de la configuración
de los eventos.
}
  \Actor Presiona el botón {\bf Cancelar}
  \Sistema Redirige a la pantalla \refElem{IU-M01}
\end{UCtrayectoriaA}

\begin{UCtrayectoriaA}{E}{%
Alguno de los campos ingresados por el \refElem{aAdministrador} son incorrectos.
}
  \Sistema Imprime los mensajes de error debajo de los con los valores incorrectos
  \Actor Ingresa nuevamente los valores en los campos marcados como incorrectos.
  \Sistema Regresa al paso \ref{CU-E02-3-submit}
\end{UCtrayectoriaA}

