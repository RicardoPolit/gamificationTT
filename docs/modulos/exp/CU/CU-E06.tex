
% \ucstEnEdicion     Al terminar una revisión/aprobación con observaciones 
%                    y al inicio del CU.
%
% \ucstEnRevision    Al terminar la edición del CU (version += 0.1).
% \ucstEnAprobacion  Al pasar la revision sin observaciones.
% \ucstAprobado      Al ser aprobado por el usuario (version += 1.0)

\begin{UseCase}[%
Autor/Daniel Ortega,%
Version/0.1,%
Estado/\ucstEnEdicion]%
%
{CU-E06}{Quitar el soporte de brindar experiencia a un curso gamificado}{%
%
 Permite al \refElem{aProfesor} quitar el soporte para brindar experiencia a un
 curso gamificado, se recomienda que para que no cambie la organización del curso
 se ocupe el formato de tópicos/temas debido a que en este formato esta basado el
 formato gamificado.}

	\UCitem[control]{Revisor}{ Sin asignar }
	\UCitem[control]{Último cambio}{ \today }

 \UCsection{Atributos}

    \UCitem{Actor(es)}{%
        \refElem{aProfesor}
    }

	\UCitem{Propósito}{%
        Permitir al profesor remover el soporte brindar puntos de experiencia de
        un curso que previamente gamificado.
	}
	
	\UCitem{Entradas}{\imprimeUC{entrada}}

	\UCitems{Origen}{%
        \Titem Mouse
	}

	\UCitem{Salidas}{\imprimeUC{salida}}

	\UCitem{Destino}{%
		\refElem{IU-M02a}
	}
	
	\UCitems{Precondiciones}{%
        \Titem El curso al que se le debe quitar el soporte de experiencia debe tener
               el formato gamificado.
        \Titem Los plugins del módulo de experiencia deben de estar instalados.
	}

	\UCitem{Postcondiciones}{%
        Ninguna
	}

	\UCitem{Reglas de negocio}{\imprimeUC{regla}}

	\UCitems{Errores}{%
        \Titem \UCerr{Err1}{%
        % CAUSA
            El curso elegido no es un curso gamificado con soporte para experiencia,}{%
        % EFECTO
            termina el caso de uso}
	}

	% \UCitem{Viene de}{% Indicar si el Caso de uso es primario o se extiende de otro. La mayoría se 
					  % extienden de Login.
		% EJEMPLO: \refIdElem{PY-CU1} o Caso de uso primario.
	% 	\TODO Especificar.
	% }	

 \UCsection[design]{Datos de Diseño}

	\UCitems[design]{Casos de Prueba}{%
        \Titem \refElem{CPC-E06}
        \Titem \refElem{CPC-E06a}
	}

 \UCsection[admin]{Datos de Administración de Requerimiento}

	\UCitem[admin]{Observaciones}{%
        Ninguna
	}

\end{UseCase}

\clearpage
\subsubsection{Trayectorias del caso de uso}

\begin{UCtrayectoria}%
%
 \includeUC{CU-M01} presionando la pestaña {\bf Cursos}.

  \Actor Presiona la opción {\bf Gestionar cursos y categorías}.
  \Sistema Obtiene las \refElem[categorias]{mdl-course-category} y 
           \refElem{mdl-course.fullname} de los cursos presentes en la plataforma.
  \Sistema Muestra en la pantalla \refElem{IU-M06} la lista de cursos.

  \Actor Presiona el botón \IUConfigurar del curso que desea editar.
  \Sistema Obtiene el \salida{mdl-course.fullname}, \salida{mdl-course.shortname} y
           \salida{mdl-course.format}.
  \Sistema Obtiene el valor de \salida{xp-course.hiddensections} y
           \salida{xp-course.coursedisplay}. \refErr{Err1}
  \Sistema Obtiene los demás datos del curso.
  \Sistema Muestra los datos obtenidos del curso en la pantalla \refElem{IU-E05}.

  \Actor Despliega la sección para el formato del curso.
  \Actor Selecciona un \entrada{mdl-course.format} distinto al 
         \refElem[formato gamificado]{xp-course.format}.
  \Sistema Carga la información correspondiente al formato del curso elegido.

  \Actor Opcionalmente edita los demás campos opcionales del formulario.
  \Actor Presiona el botón {\bf Guardar Cambios y mostrar}. \refTray{A}, \refTray{B},
         \refTray{C} \label{CU-E06-submit}
  \Sistema Actualiza los datos del curso de moodle con los datos ingresados por el usuario.
  \Sistema Elimina el \entrada{xp-course} vinculado al \refElem{mdl-course} y las
           configuraciones del mismo.
  \Sistema Elimina las \entrada[secciones del curso gamificado]{xp-course-section} 
           vinculadas al curso.
  \Sistema Elimina las \entrada[recompensas]{xp-section-reward} de las secciones del 
           curso sin alterar la experiencia recibida por los alumnos de acuerdo con 
           la regla \regla{BR-E02}. \label{CU-E06-finish}
  \Sistema Muestra la pantalla \refElem{IU-M06c}.

\end{UCtrayectoria}

\begin{UCtrayectoriaA}{A}{%
Alguno de los campos introducidos por el usuario es erroneo.
}

  \Sistema Muestra los mensajes de error correspondientes debajo de cada campo inválido.
  \Actor Ingresa de nuevo el valor para los campos que contienen errores.
  \Sistema Regresa al paso \ref{CU-E06-submit}

\end{UCtrayectoriaA}

\begin{UCtrayectoriaA}{B}{%
El usuario desea regresar a la pantalla \refElem{IU-M06}.}

  \Actor Presiona el botón {\bf Guardar y regresar}
  \Sistema Ejecuta los pasos a partir del paso \ref{CU-E06-submit} hasta el
           paso \ref{CU-E06-finish}.
  \Sistema Redirige a la pantalla \refElem{IU-M06}

\end{UCtrayectoriaA}

\begin{UCtrayectoriaA}{C}{%
El usuario desea cancelar la eliminación del soporte para brindar experiencia en el curso}

  \Actor Presiona el botón {\bf Cancelar}
  \Sistema Redirige a la pantalla \refElem{IU-M06}

\end{UCtrayectoriaA}
