
\subsection{Análisis}

 Este apartado contiene el análisis requerido para la elaboración de módulo de experiencia,
 contiene la especificación del alcance de este módulo, la descripción de las funcionalidades
 a desarrollar, la reglas de negocio que rigen el comportamiento del módulo, y por último la
 especificación de los casos de uso a los que brinda soporte.

\subsubsection{Esquema de experiencia}

 El esquema de experiencia le proporciona al \refElem{aAdministrador} y a los \refElem[Profesores]%
 {aProfesor} un mecanismo mediante el cual pueden configurar la forma en que se obtienen los puntos
 de experiencia, la cantidad a otorgar, el número de puntos de cada nivel y finalmente la
 visualización del nivel y de los puntos de cada usuario.\\

 \noindent
 Las configuraciones fueron organizadas en dos grupos: las {\it configuraciones a
 nivel plataforma} las cuales definen valores de forma global, y las {\it configuraciones a nivel
 curso} las cuales definen valores para un curso en específico. A continuación se describen cada
 uno de los grupos.\\

\subsubsection{Submódulo de niveles}
\subsubsection{Funcionalidades}

% \subsubsection{Esquema de experiencia.}

 % El esquema de experiencia es la configuración sobre cómo funciona el sistema
 % de puntos de experiencia, incluyendo la cantidad de experiencia que tiene
 % cada nivel, el tipo de incremento en los puntos de experiencia nivel a nivel,
 % las restricciones sobre la experiencia y la forma en que se otorgarán los puntos.

% \subsubsection{Niveles.}

 % Es el mecanismo que permite mostrarle a los estudiantes el progreso que han tenido
 % a nivel plataforma mediante el nivel y los puntos de experiencia obtenidos en
 % los cursos, además contiene la configuración para establecer el cómo se verá el
 % nivel y la experiencia obtenida de dicho nivel.

 % Los principios de gamificación a los cuales permite brindarles soporte son:
 % \principioII
 % \principioVI
%
\subsubsection{Reglas de negocio} %========================================================

 En esta sección se especifican todas las reglas de negocio relevantes para el módulo de
 experiencia. Las reglas de negocio que establece moodle son diferenciadas por tener la letra {\it M}
 antecediendo al número consecutivo en su identificador.

    
\subsection{Entidades de moodle}

Debido a que moodle cuenta con más de 400 entidades en su versión 3.5, se opta
por mostrar 2 subconjuntos que muestren las entidades que se utilizan para el proyecto.\\

\noindent El primer subconjunto es aquel que explica la forma en que moodle implementa los cursos,
secciones de curso, actividades, usuarios y roles (el cual se presenta en la figura \ref{fig:BD-ER-M1}),
mientras que el segundo subconjunto muestra como moodle maneja toda la
estructura de las preguntas creadas por el profesor y respondidas por el estudiante
(el cual se presenta en la figura \ref{fig:BD-ER-M2}).

\noindent El objetivo de ambos esquemas (\ref{fig:BD-ER-M1} y \ref{fig:BD-ER-M2}) es expresar la idea general que abarcan ambos subconjuntos.

\clearpage
\addfigure{0.7}{analisis/diagrams/db_module_structure}{fig:BD-ER-M1}{Esquema de la base de datos de moodle 'Cursos'}


\noindent Utilizando la figura \ref{fig:BD-ER-M1}, se obtuvieron las siguientes reglas y características que tiene moodle respecto a los usuarios en un curso y a la estructura de los cursos.
\begin{enumerate}
    \item Un usuario -{\it mdl\_user}- tiene un rol -{\it mdl\_role}- en un cierto contexto -{\it mdl\_context}-, cuyo  '{\it context\_level}' sea igual a cincuenta(50).
    \item Si el contexto '{\it context\_level}' es de 50, el atributo '{\it instance\_id}' hace referencia al atributo '{\it id}' de un curso -{\it mdl\_course}-.
    \item El curso -{\it mdl\_course}- tiene varias secciones -{\it mdl\_course\_sections}-.
    \item Cada seccion -{\it mdl\_course\_sections}- tiene varias actividades -{\it mdl\_course\_modules}- que pertenecen a un tipo de actividad -{\it mdl\_modules}-.
    \item Por cada registro en tipo de actividad -{\it mdl\_modules}-, se tiene una entidad que lleva el mismo nombre.
    \item El atributo '{\it instance\_id}' de una actividad  -{\it mdl\_course\_modules}- apunta a diferentes entidades. La entidad a la que apunta depende del nombre del tipo de actividad -{\it mdl\_modules}-.
    \item Un usuario -{\it mdl\_user}- se inscribió -{\it mdl\_user\_enrolments}- a un curso -{\it mdl\_course}-, por medio de un formato soportado de inscripción -{\it mdl\_enrol}-.
\end{enumerate}

\clearpage

 \addfigure{0.7}{analisis/diagrams/db_module_questions}{fig:BD-ER-M2}{Esquema de la base de datos de moodle 'Preguntas' }



\noindent Utilizando la figura \ref{fig:BD-ER-M2}, se obtuvieron las siguientes reglas y características que tiene moodle respecto a las preguntas.
\begin{enumerate}
    \item Las preguntas -{\it mdl\_question}- tienen versiones -{\it mdl\_question\_attempts}-.
    \item Una pregunta -{\it mdl\_question}- pertenece a un banco de preguntas -{\it mdl\_question\_categories}-.
    \item La versión de una pregunta -{\it mdl\_question\_attempts}- es contestada -{\it mdl\_question\_usages}- en un determinado contexto -{\it mdl\_context}-.
    \item Un usuario -{\it mdl\_user}- responde una versión de una pregunta -{\it mdl\_question\_attempt\_stepts}-.
    \item El responder una versión de una pregunta -{\it mdl\_question\_attempt\_stepts}- conlleva pasos\\ -{\it mdl\_question\_attempt\_stept\_data}-, los cuales son: cómo se muestra, si ya se terminó de responder y qué se respondió.
\end{enumerate}


 A continuación se presenta la especificación de las entidades del esquema de base
 de datos de moodle que son relevantes para el desarrollo de los módulos y submódulos
 de proyecto.

    \begin{cdtEntidad}{mdl-config-plugins}{Configuración de Plugin}{%
    Es una tabla del núcleo de moodle que almacena todas las configuraciones globales
    relacionadas a los plugins instalados, al iniciar moodle las configuraciones de los
    plugins instalados y habilitados se cargan en memoria.}

	    \brAttr{id}{Id}{tInt}{%
	        Es el dígito que representa el identificador único para una configuración
            específica de un plugin.\par

            \it Restricciones:
            \refElem{tPrimaryKey},
            \refElem{tAutoIncrement}.
        }

        \brAttr{plugin}{Plugin}{tVarchar}{%
            Cadena de caracteres del nombre identificador del plugin al cual pertenece
            la configuración.\par

            \it Restricciones:
            \refElem{tRequired},
            \refElem{tRange} (0,100),
            \refElem{tUniqueKey}
        }

        \brAttr{name}{Nombre}{tVarchar}{%
            Cadena de caracteres que representa el nombre de la configuración de un
            plugin en específico.\par

            \it Restricciones:
            \refElem{tUniqueKey},
            \refElem{tRange} (0,100),
            \refElem{tRequired}
        }

        \brAttr{value}{Valor}{tVarchar}{%
            Cadena que almacena el valor de una configuración perteneciente a alguno
            de los plugins instalados.\par

            \it Restricciones:
            \refElem{tRange} (0,4294967295),
            \refElem{tRequired}
        }
    \end{cdtEntidad}\schemeName{config\_plugins}

    \begin{cdtEntidad}{mdl-user}{Usuario de moodle}{%
    Es una tabla del núcleo de moodle que contiene toda la información que se
    almacena de los usuarios en la plataforma, independientemente del rol que
    estos contenga, esta relación contiene más de 53 atributos, sin embargo solo
    se detallan aquellos relevantes.}

	    \brAttr{id}{Id}{tInt}{%
	        Es el dígito que representa el identificador único para cada uno
            de los usuarios en moodle.\par

            \it Restricciones:
            \refElem{tPrimaryKey},
            \refElem{tAutoIncrement}.
        }
	    \brAttr{username}{nombre de usuario}{tVarchar}{%
	        .\par

            \it Restricciones:
            \refElem{tRequired},
            \refElem{tLength} 0-100
        }
	    \brAttr{password}{contraseña}{tVarchar}{%
	        .\par

            \it Restricciones:
            \refElem{tRequired},
            \refElem{tLength} 0-255.
        }
	    \brAttr{firstname}{nombre}{tVarchar}{%
	        .\par

            \it Restricciones:
            \refElem{tRequired},
            \refElem{tLength} 0-100
        }
	    \brAttr{lastname}{apellido}{tVarchar}{%
	        .\par

            \it Restricciones:
            \refElem{tRequired},
            \refElem{tLength} 0-100
        }
	    \brAttr{email}{correo}{tVarchar}{%
	        .\par

            \it Restricciones:
            \refElem{tRequired},
            \refElem{tLength} 0-100
        }
	    \brAttr{lastaccess}{último registro}{tInt}{%
	        .\par

            \it Restricciones:
            \refElem{tRequired},
            \refElem{tLength} 10
        }
	    \brAttr{city}{ciudad}{tVarchar}{%
	        .\par

            \it Restricciones:
            \refElem{tRequired},
            \refElem{tLength} 0-120
        }
	    \brAttr{country}{pais}{tVarchar}{%
	        .\par

            \it Restricciones:
            \refElem{tRequired},
            \refElem{tLength} 2
        }

    \end{cdtEntidad}\schemeName{user}

    \begin{cdtEntidad}{mdl-course}{Curso de moodle}{%
    Es una tabla del núcleo de moodle que contiene la información principal de cada
    curso registrado en moodle. Esta entidad contiene 31 atributos, a continuación se
    detallan los atributos relevantes para la especificación de este proyecto.}

	    \brAttr{id}{Id}{tInt}{%
	        Es el dígito que representa al identificador único para cada uno
            de los cursos en moodle.\par

            \it Restricciones:
            \refElem{tPrimaryKey},
            \refElem{tAutoIncrement}.
        }

	    \brAttr{format}{formato}{tVarchar}{%
	        Es el dígito que representa al identificador único para cada uno
            de los cursos en moodle.\par

            \it Restricciones:
            \refElem{tRequired}.
            \refElem{tDefault} topics,
            \refElem{tLength} 0-21.
        }

	    \brAttr{fullname}{nombre completo}{tVarchar}{%
	        Es el nombre completo que se le asigna al curso.\par

            \it Restricciones:
            \refElem{tRequired}.
            \refElem{tLength} 0-21.
        }

	    \brAttr{shortname}{nombre corto}{tVarchar}{%
            Es el nombre corto que se le asigna al curso.\par

            \it Restricciones:
            \refElem{tRequired}.
            \refElem{tLength} 0-21.
        }

    \end{cdtEntidad}\schemeName{course}

    \begin{cdtEntidad}{mdl-course-section}{Sección del curso de moodle}{%
    }
	    \brAttr{id}{Id}{tInt}{%
	        Es el dígito que representa al identificador único para cada sección
            de los cursos en moodle.\par

            \it Restricciones:
            \refElem{tPrimaryKey},
            \refElem{tAutoIncrement}.
        }

        \brAttr{name}{nombre}{tVarchar}{%
	        Es el dígito nombre que permite identificar a una sección dentro de un curso
            en moodle.\par

            \it Restricciones: ...
        }
    \end{cdtEntidad}\schemeName{course\_sections}

    \begin{cdtEntidad}{mdl-course-format-options}{Opciones del formato del curso}{%
    }
	    \brAttr{id}{Id}{tInt}{%
	        Es el dígito que representa al identificador único para cada uno
            de los cursos en moodle.\par

            \it Restricciones:
            \refElem{tPrimaryKey},
            \refElem{tAutoIncrement}.
        }

	    \brAttr{courseid}{Id}{tInt}{%
	        Es el dígito que representa al identificador único para cada uno
            de los cursos en moodle.\par

            \it Restricciones:
            \refElem{tForeignKey},
            \refElem{tRequired}
        }

	    \brAttr{format}{formato}{tVarchar}{%
	        Es el dígito que representa al identificador único para cada uno
            de los cursos en moodle.\par

            \it Restricciones:
            \refElem{tRequired}.
            \refElem{tDefault} topics,
            \refElem{tLength} 0-21.
        }

	    \brAttr{name}{opcion}{tVarchar}{%
	        Es el dígito que representa al identificador único para cada uno
            de los cursos en moodle.\par

            \it Restricciones:
            \refElem{tPrimaryKey},
            \refElem{tLength} 0-100
        }

	    \brAttr{value}{valor}{tVarchar}{%
	        Es el dígito que representa al identificador único para cada uno
            de los cursos en moodle.\par

            \it Restricciones:
            \refElem{tRequired}
        }

    \end{cdtEntidad}\schemeName{course\_format\_options}

    \begin{cdtEntidad}{mdl-course-category}{Categoria de curso}{%
      .}
        \brAttr{id}{id}{tInt}{%
        .}
        \brAttr{name}{nombre}{tInt}{%
        .}

    \end{cdtEntidad}\schemeName{course\_category}

    \begin{cdtEntidad}{mdl-course-module}{Actividad del curso}{%
    .}
	    \brAttr{id}{Id}{tInt}{%
	        Es el dígito que representa al identificador único para cada uno
            de los cursos en moodle.\par

            \it Restricciones:
            \refElem{tPrimaryKey},
            \refElem{tAutoIncrement}.
        }
        \brAttr{course}{curso}{tInt}{%
        .}
        \brAttr{module}{actividad}{tInt}{%
        .}
        \brAttr{section}{sección}{tInt}{%
        .}
    \end{cdtEntidad}\schemeName{course\_module}

    \begin{cdtEntidad}{mdl-course-module-completion}{Actividad del curso para alumno}{%
    .}
	    \brAttr{id}{Id}{tInt}{%
	        Es el dígito que representa al identificador único para cada uno
            de los cursos en moodle.\par

            \it Restricciones:
            \refElem{tPrimaryKey},
            \refElem{tAutoIncrement}.
        }
        \brAttr{coursemoduleid}{actividad}{tInt}{%
        .}
        \brAttr{userid}{usuario}{tInt}{%
        .}
        \brAttr{completionstate}{completitud}{tBoolean}{%
        .}
    \end{cdtEntidad}\schemeName{course\_module}

    \begin{cdtEntidad}{Plugin}{Plugin}{%
    La forma en que moodle obtiene información acerca de los plugins es analizando
    los archivos internos de cada uno, a pesar de que los plugins no forman parte
    del esquema de base de datos, si forman parte del modelo de información que
    utiliza Moodle.}

	    \brAttr{componente}{Componente}{tVarchar}{%
	        Cadena compuesta por el tipo de plugin y el nombre del mismo, que
            representa a la clase principal del plugin que contiene los métodos
            principales del plugin.\par

            \it Restricciones: Ninguna
        }

	    \brAttr{pluginname}{Nombre}{tVarchar}{%
	        Es el nombre del plugin obtenido de los archivos de
            internacionalización presentes en el plugin, el valor de esta cadena
            depende del lenguaje seleccionado en moodle.\par

            \it Restricciones: Ninguna
        }

	    \brAttr{fullpath}{Ruta absoluta}{tPath}{%
	        La ruta absoluta de un plugin denota la ubicación del plugin en el
            sistema de archivos, esta ruta está compuesta por la ruta absoluta
            de la instalación de moodle, la carpeta correspondiente al tipo del
            plugin y el nombre del plugin.\par

            \it Restricciones: Formato ``/path/to/moodle/plugintype/pluginname''
        }

	    \brAttr{path}{Ruta relativa}{tPath}{%
	        La ruta relativa denota la ubicación del plugin dentro de la carpeta
            donde se encuentran los archivos de moodle, esta ruta está compuesta
            por la carpeta correspondiente al tipo del plugin y el nombre del
            plugin.\par

            \it Restricciones: Formato ``plugintype/pluginname''
        }

	    \brAttr{version}{Versión}{tVersion}{%
	        Numero entero de longitud de 10 dígitos que representa la versión del
            plugin.\par

            \it Restricciones: Ninguna adicional al tipo de dato
        }

	    \brAttr{moodle}{Versión de Moodle}{tVersion}{%
	        Número entero de longitud de 10 dígitos que representa la versión de
            moodle en la que se puede instalar el plugin.\par

            \it Restricciones: Ninguna adicional al tipo de dato
        }

        \brAttr{dependencies}{Dependencias}{tObject}{%
            Objeto que almacena un conjunto de claves con sus respectivos valores,
            donde cada clave representa el nombre del componente del plugin y el valor
            es la \refElem{Plugin.version} requerida del mismo.

            \it Restricciones: Ninguna
        }

        \brAttr{icon}{ícono}{tImage}{%
            Imagen para el ícono del plugin, debe estar contenida en el directorio
            {\it pix/} del plugin y tener como nombre {\it icon.png} o {\it icon.svg},
            moodle recomienda tener ambos archivos por si los navegadores no soportan
            algún tipo de archivo \cite{moodlePluginfiles}.\par

            \it Restricciones: El nombre debe ser icono con extensiones png o svg
        }

    \end{cdtEntidad}
 % Archivo de plugin
    \input{modulos/exp/BR/BR-E01} % Restricciones sobre de imagen del nivel.
    
\begin{BusinessRule}[%
Autor/Daniel Isai Ortega Zúñiga,%
Version/0.1,%
Estado/revision]%
%
{BR-E02}{Permanencia de los puntos de experiencia}
    \BRitem[control]{Revisor}{Sin asignar.}

 \BRsection[control]{Atributos}
    
    %\BRitem[admin]{Clase}{\bcCondition}%
    \BRitem[admin]{Clase}{\bcIntegridad}%
    %\BRitem[admin]{Clase}{\bcAutorizacion}%
    %\BRitem[admin]{Clase}{\bcDerivacion}%
        
    \BRitem[admin]{Tipo}{\btEnabler}%
    %\BRitem[admin]{Tipo}{\btTimer}%
    %\BRitem[admin]{Tipo}{\btExecutive}%
        
    %\BRitem[admin]{Nivel}{\blControlling}
    \BRitem[admin]{Nivel}{\blInfluencing}
    
    \BRitem{Descripción}{%
    Los puntos de experiencia una vez que son obtenidos no pueden ser quitados bajo
    ninguna condicion exceptuando únicamente la acción de eliminación de un usuario y
    la desinstalación de los plugins que formen parte del esquema de experiencia.}

    \BRitem{Ejemplo positivo}{\hfill\par%
        \begin{itemize}
        \item Un usuario conforme va completando las secciones de los cursos obtiene
              puntos de experiencia, si el curso es eliminado entonces el usuario 
              deberá permanecer con los puntos de experiencia obtenidos.

        \item Un usuario con 300 puntos de experiencia es eliminado del sitio, y en
              consecuencia se eliminan sus puntos de experiencia.
        \end{itemize}
    }

    \BRitem{Ejemplo negativo}{\hfill\par%
        \begin{itemize}
        \item Un curso es eliminado y a todos los estudiantes se les resta de sus
              puntos de experiencia la cantidad de experiencia obtenida durante el
              curso.
        \end{itemize}
    }% 
    
 \end{BusinessRule}

 % Permanencia del nivel de experiencia.
    
\begin{BusinessRule}[%
Autor/Daniel Isai Ortega Zúñiga,%
Version/0.1,%
Estado/revision]%
%
{BR-E03}{Tipos de Incremento}
    \BRitem[control]{Revisor}{Sin asignar.}

 \BRsection[control]{Atributos}
    
    \BRitem[admin]{Clase}{\bcCondition}%
    %\BRitem[admin]{Clase}{\bcIntegridad}%
    %\BRitem[admin]{Clase}{\bcAutorizacion}%
    %\BRitem[admin]{Clase}{\bcDerivacion}%
        
    \BRitem[admin]{Tipo}{\btEnabler}%
    %\BRitem[admin]{Tipo}{\btTimer}%
    %\BRitem[admin]{Tipo}{\btExecutive}%
        
    \BRitem[admin]{Nivel}{\blControlling}
    %\BRitem[admin]{Nivel}{\blInfluencing}
    
    \BRitem{Descripción}{%
    Cuando se modifiquen el \refElem{xp-scheme-settings} o la \refElem{levelXP} de las
    \refElem{xp-scheme-settings}
    }

    \BRitem{Ejemplo positivo}{\hfill\par%
        \begin{itemize}
        \item ...
        \end{itemize}
    }

    \BRitem{Ejemplo negativo}{\hfill\par%
        \begin{itemize}
        \item ...
        \end{itemize}
    }% 
    
 \end{BusinessRule}
 % Tipos de incremento
    \begin{BusinessRule}[%
Autor/Daniel Isai Ortega Zúñiga,%
Version/0.1,%
Estado/revision]%
%
{BR-E04}{Calculo de experiencia del nivel con incremento porcentual}
    \BRitem[control]{Revisor}{Sin asignar.}

 \BRsection[control]{Atributos}
    
    \BRitem[admin]{Clase}{\bcCondition}%
    %\BRitem[admin]{Clase}{\bcIntegridad}%
    %\BRitem[admin]{Clase}{\bcAutorizacion}%
    %\BRitem[admin]{Clase}{\bcDerivacion}%
        
    \BRitem[admin]{Tipo}{\btEnabler}%
    %\BRitem[admin]{Tipo}{\btTimer}%
    %\BRitem[admin]{Tipo}{\btExecutive}%
        
    \BRitem[admin]{Nivel}{\blControlling}
    %\BRitem[admin]{Nivel}{\blInfluencing}
    
    \BRitem{Descripción}{%
        El calculo para obtener la experiencia del nivel $i$ uando el tipo de
        incremento es porcentual está dado por la siguiente fórmula: Sea {\it exp()}
        la función que optiene la experiencia de un nivel en específico, sea tambien
        $i$ el nivel del cual se calcula la experiencia, sea $inc$ el factor de
        incremento de nivel a nivel, y finalmente sea $round()$ una función de
        redondeo a números enteros, entonces:

            $$ exp(i) = round( exp(1) * (inc)^{(i-1)})$$
    }

%   \BRitem{Sentencia}{%
%       Si $fecha$ 
%   }%

    \BRitem{Ejemplo positivo}{\hfill\par%
        \begin{itemize}
        \item La experiencia requerida para superar el nivel 1 es de 2000 puntos y el
              factor de incremento entre los niveles es 1.1, entonces la experiencia
              requerida para pasar el nivel 5 es de 2928 puntos.
        \end{itemize}
    }

    \BRitem{Ejemplo negativo}{\hfill\par%
        \begin{itemize}
        \item La experinecia requerida para superar el nivel 1 es de 2000 puntos y el
              factor de incremento entre los niveles es 1.1, entonces la experiencia
              requerida para pasar el nivel 5 es de 2300 puntos.
        \end{itemize}
    }% 
    
\end{BusinessRule}
 % Incremento porcentual
    \begin{BusinessRule}[%
Autor/Daniel Isai Ortega Zúñiga,%
Version/0.1,%
Estado/revision]%
%
{BR-E05}{Cálculo de experiencia del nivel con incremento linea} % Cuando están iniciados
    \BRitem[control]{Revisor}{Sin asignar.}

 \BRsection[control]{Atributos}
    
    \BRitem[admin]{Clase}{\bcCondition}%
    %\BRitem[admin]{Clase}{\bcIntegridad}%
    %\BRitem[admin]{Clase}{\bcAutorizacion}%
    %\BRitem[admin]{Clase}{\bcDerivacion}%
        
    \BRitem[admin]{Tipo}{\btEnabler}%
    %\BRitem[admin]{Tipo}{\btTimer}%
    %\BRitem[admin]{Tipo}{\btExecutive}%
        
    \BRitem[admin]{Nivel}{\blControlling}
    %\BRitem[admin]{Nivel}{\blInfluencing}
    
    \BRitem{Descripción}{%
    }

%   \BRitem{Sentencia}{%
%       Si $fecha$ 
%   }%

    \BRitem{Ejemplo positivo}{\hfill\par%
        \begin{itemize}
        \item ...
        \end{itemize}
    }

    \BRitem{Ejemplo negativo}{\hfill\par%
        \begin{itemize}
        \item ...
        \end{itemize}
    }% 
    
\end{BusinessRule}
 % Incremento lineal
    \begin{BusinessRule}[%
Autor/Daniel Isai Ortega Zúñiga,%
Version/0.1,%
Estado/revision]%
%
{BR-E06}{Eliminación de cursos gamificados} % Cuando están iniciados
    \BRitem[control]{Revisor}{Sin asignar.}

 \BRsection[control]{Atributos}

    \BRitem[admin]{Clase}{\bcCondition}%
    %\BRitem[admin]{Clase}{\bcIntegridad}%
    %\BRitem[admin]{Clase}{\bcAutorizacion}%
    %\BRitem[admin]{Clase}{\bcDerivacion}%

    \BRitem[admin]{Tipo}{\btEnabler}%
    %\BRitem[admin]{Tipo}{\btTimer}%
    %\BRitem[admin]{Tipo}{\btExecutive}%

    \BRitem[admin]{Nivel}{\blControlling}
    %\BRitem[admin]{Nivel}{\blInfluencing}

    \BRitem{Descripción}{%
        Debido a que el \refElem{mdl-course.format} por defecto para los cursos de
        moodle es el formato de tópicos/temas El formato de curso gamificado extiende
        las funcionalidades de este formato para facilitar la migración de un curso
        gamificado a uno no gamificado y viceversa. % TODO Pasar a analisis.
        La desinstalación del módulo de experiencia implica que los cursos con el
        \refElem{xp-course.format} gamificado (gamedle) se migren a cursos no
        gamificados, por compatibilidad en este migración se deben realizar las
        siguientes acciones de forma transaccional:

        \begin{itemize}
        \item El formato de los \refElem[cursos gamificados]{xp-course} debe
              cambiarse por el formato por defecto que tienen los cursos en moodle
              el cual es el de tópicos/temas.

        \item Se deben eliminar las \refElem[opciones del formato del curso]%
              {mdl-course-format-options} gamificado (gamedle).

        \item Se deben establecer los valores para las opciones del formato de
              tópicos/temas, que tienen por \refElem[nombre]%
              {mdl-course-format-options.name} secciones ocultas y aspecto del curso.
        \end{itemize}
    }

%   \BRitem{Sentencia}{%
%       Si $fecha$
%   }%

    \BRitem{Ejemplo positivo}{\hfill\par%
        \begin{itemize}
        \item Cuando se desinstala los plugins correspondientes al módulo de
              experiencia los cursos que estan vinculados con el formato gamificado
              son cambiados al formato de topicos/temas (formato por defecto de
              moodle), y también se establecen las opciones equivalentes al formato
              gamificado.
        \end{itemize}
    }

    \BRitem{Ejemplo negativo}{\hfill\par%
        \begin{itemize}
        \item Cuando se desinstala los plugins correspondientes al módulo de
              experiencia los cursos que estan vinculados con el formato gamificado
              no son cambiados al formato de topicos/temas ocasionando inconsistencia
              entre los cursos y los formatos.
        \end{itemize}
    }%

\end{BusinessRule}
 % Eliminación de cursos gamificados
    \begin{BusinessRule}[%
Autor/Daniel Isai Ortega Zúñiga,%
Version/0.1,%
Estado/edicion]%
%
{BR-E07}{Valores iniciales de experiencia}

     \BRitem[control]{Revisada por}{Pendiente.}

 \BRsection[control]{Atributos}
    % Clases: \bcCondition, \bcIntegridad, \bcAutorization o \bcDerivation
    % Tipos: \btEnabler, \btTimer o \btExecutive
    % Niveles: \blControlling o \blInfluencing.

    \BRitem[admin]{Clase}{\bcIntegridad}%

    \BRitem[admin]{Tipo}{\btTimer}%

    \BRitem[admin]{Nivel}{\blControlling}

    \BRitem{Descripción}{%
        Cuando un \refElem{xp-user} es creado este debe de empezar a ganar puntos
        de experiencia a partir del \refElem{xp-user.level} uno, tenido cero puntos
        de experiencia en la \refElem{xp-user.levelxp} y \refElem{xp-user.xp}. Ningún
        usuario puede comenzar con valores distintos a los indicados anteriormente.
    }

%   \BRitem{Sentencia}{%
%       Si $fecha$ 
%   }%

    \BRitem{Ejemplo positivo}{\hfill\par%
        \begin{itemize}
        \item ...
        \end{itemize}
    }

    \BRitem{Ejemplo negativo}{\hfill\par%
        \begin{itemize}
        \item ...
        \end{itemize}
    }

 \end{BusinessRule}
 % Valores iniciales de experiencia
    \begin{BusinessRule}[%
Autor/Daniel Isai Ortega Zúñiga,%
Version/0.1,%
Estado/edicion]%
%
{BR-E08}{Valores iniciales de experiencia de un curso}

     \BRitem[control]{Revisada por}{Pendiente.}

 \BRsection[control]{Atributos}
    % Clases: \bcCondition, \bcIntegridad, \bcAutorization o \bcDerivation
    % Tipos: \btEnabler, \btTimer o \btExecutive
    % Niveles: \blControlling o \blInfluencing.

    \BRitem[admin]{Clase}{\bcIntegridad}%

    \BRitem[admin]{Tipo}{\btTimer}%

    \BRitem[admin]{Nivel}{\blControlling}

    \BRitem{Descripción}{%
        Cuando un \refElem{xp-course} es creado la \refElem[experiencia total del curso]%
        {xp-scheme-settings.courseXP} de ser dividida uniformemente entre las
        \refElem[secciones del curso gamificado]{xp-course-section}. Si la división del
        total de experiencia entre el número de secciones genera un residuo entonces este
        se deberá agregan a la última sección del curso.
    }

%   \BRitem{Sentencia}{%
%       Si $fecha$
%   }%

    \BRitem{Ejemplo positivo}{\hfill\par%
        \begin{itemize}
        \item ...
        \end{itemize}
    }

    \BRitem{Ejemplo negativo}{\hfill\par%
        \begin{itemize}
        \item ...
        \end{itemize}
    }

 \end{BusinessRule}
 % Valores iniciales de experiencia del curso
    \begin{BusinessRule}[%
Autor/Daniel Isai Ortega Zúñiga,%
Version/0.1,%
Estado/edicion]%
%
{BR-E09}{Secciones editables de un curso con experiencia}

     \BRitem[control]{Revisada por}{Pendiente.}

 \BRsection[control]{Atributos}
    % Clases: \bcCondition, \bcIntegridad, \bcAutorization o \bcDerivation
    % Tipos: \btEnabler, \btTimer o \btExecutive
    % Niveles: \blControlling o \blInfluencing.

    \BRitem[admin]{Clase}{\bcCondition}%

    \BRitem[admin]{Tipo}{\btEnabler}%

    \BRitem[admin]{Nivel}{\blControlling}

    \BRitem{Descripción}{%
        Para todas aquellas secciones de un curso gamificado que ya hayan sido
        completadas por al menos un estudiante, se debe bloquear la edición de los
        puntos de experiencia que otorgan con el propósito de impedir que una sección
        en distintos momentos para distintos alumnos otorge puntos de experiencia.
    }

%   \BRitem{Sentencia}{%
%       Si $fecha$
%   }%

    \BRitem{Ejemplo positivo}{\hfill\par%
        \begin{itemize}
        \item Al abrir la edición de un curso con experiencia con cinco secciones,
              se deshabilita la edición de las primeras tres secciones del curso
              ya que estas han sido completadas por almenos un estudiante.
        \end{itemize}
    }

    \BRitem{Ejemplo negativo}{\hfill\par%
        \begin{itemize}
        \item Al abrir la edición de un curso con experiencia con cinco secciones,
              no se deshabilitan la edición de las secciones del curso ya que hayan
              sido completadas con anterioridad.
        \end{itemize}
    }

 \end{BusinessRule}
 % Secciones editables de un curso con experiencia
    \begin{BusinessRule}[%
Autor/Daniel Isai Ortega Zúñiga,%
Version/0.1,%
Estado/edicion]%
%
{BR-E10}{Administración de la experiencia de un curso}

     \BRitem[control]{Revisada por}{Pendiente.}

 \BRsection[control]{Atributos}
    % Clases: \bcCondition, \bcIntegridad, \bcAutorization o \bcDerivation
    % Tipos: \btEnabler, \btTimer o \btExecutive
    % Niveles: \blControlling o \blInfluencing.

    \BRitem[admin]{Clase}{\bcCondition}%

    \BRitem[admin]{Tipo}{\btEnabler}%

    \BRitem[admin]{Nivel}{\blControlling}

    \BRitem{Descripción}{%
        Un evento puede ...
    }

%   \BRitem{Sentencia}{%
%       Si $fecha$
%   }%

    \BRitem{Ejemplo positivo}{\hfill\par%
        \begin{itemize}
        \item ...
        \end{itemize}
    }

    \BRitem{Ejemplo negativo}{\hfill\par%
        \begin{itemize}
        \item ...
        \end{itemize}
    }

 \end{BusinessRule}
 % Administración de la experiencia en un curso
    \begin{BusinessRule}[%
Autor/Daniel Isai Ortega Zúñiga,%
Version/0.1,%
Estado/edicion]%
%
{BR-E11}{Eliminación de las secciones con experiencia}

     \BRitem[control]{Revisada por}{Pendiente.}

 \BRsection[control]{Atributos}
    % Clases: \bcCondition, \bcIntegridad, \bcAutorization o \bcDerivation
    % Tipos: \btEnabler, \btTimer o \btExecutive
    % Niveles: \blControlling o \blInfluencing.

    \BRitem[admin]{Clase}{\bcCondition}%

    \BRitem[admin]{Tipo}{\btEnabler}%

    \BRitem[admin]{Nivel}{\blControlling}

    \BRitem{Descripción}{%
        Para poder eliminar una sección de un curso con soporte para experiencia
        esta sección debe configurar con un valor de cero puntos de experiencia.
        con propósito de que la eliminación de una sección no colisione con el
        cumplimiento de la regla \refElem{BR-E10}.
    }

%   \BRitem{Sentencia}{%
%       Si $fecha$
%   }%

    \BRitem{Ejemplo positivo}{\hfill\par%
        \begin{itemize}
        \item Se procede a eliminar un sección que tiene cero puntos de experiencia
              en un curso gamificado.
        \end{itemize}
    }

    \BRitem{Ejemplo negativo}{\hfill\par%
        \begin{itemize}
        \item Se procede a eliminar un sección que tiene 300 puntos de experiencia
              en un curso gamificado.
        \end{itemize}
    }

 \end{BusinessRule}
 % Eliminación de secciones con experiencia

    % INPUT: Cursos Igualitarios.
    % INPUT: Otorgar experiencia

\clearpage
\subsubsection{Casos de uso} % ============================================================

 En este apartado se especifican todos los casos de usos contemplados para el módulo de
 experiencia, para cada caso de uso se especifica su tabla de atributos la cual indica que casos
 de prueba deberán ejecutarse correctamente para corroborar la completitud del caso de uso.

\subsubsection*{Diagrama de casos de uso}

 En la figura \ref{exp:usecases} se detalla el diagrama de casos de uso correspondiente al módulo
 de experiencia. Los casos de uso de moodle (en color blanco) son modelados como casos de uso
 abstractos, mientras que los casos de uso del módulo de experiencia son diferenciados por el
 color azul, en total el desarrollo de este módulo consiste en 13 casos de uso principales.

    \addfigure{0.9}{modulos/UseCases}{exp:usecases}{%
        Diagrama de casos de uso del módulo de experiencia}

 \noindent
 Debido a que los plugins a desarrollar son elementos opcionales para Moodle, solo se puede
 acceder a los casos de uso del módulo de experiencia a través de puntos de extensión de los
 casos de uso de moodle. Por otra parte los casos de uso que serán documentados en esta sección
 serán los del módulo de experiencia debido a que Moodle proporciona en su página oficial, guías
 e instructivos como documentación de las funcionalidades que brinda.

    % CASOS DE USO DE MOODLE
    
% \ucstEnEdicion     Al terminar una revisión/aprobación con observaciones
%                    y al inicio del CU.
%
% \ucstEnRevision    Al terminar la edición del CU (version += 0.1).
% \ucstEnAprobacion  Al pasar la revision sin observaciones.
% \ucstAprobado      Al ser aprobado por el usuario (version += 1.0)

\begin{UseCase}[%
Autor/Daniel Ortega,%
Version/0.1,%
Estado/\ucstEnEdicion]%
%
{CU-M01}{Acceder a la administración del sitio}{% TODO; Deberia se Instalar/Actualizar ???
%
 Permite al \refElem{aAdministrador} acceder a la pantalla \refElem{IU-M01} realizar las
 distintas tareas que incluye la administración del sitio de moodle. Esta caso de uso es realizado
 debido a que es requerido para la ejecución de la mayoría de los casos de uso cuyo actor es el
 administrador.}

	\UCitem[control]{Revisor}{ Sin asignar }
	\UCitem[control]{Último cambio}{ 14/OCT/19 }

 \UCsection{Atributos}

    \UCitem{Actor(es)}{%
        \refElem{aAdministrador}
    }

	\UCitem{Propósito}{%
        Permitir al administrador acceder a la administración del moodle que administra.
	}

	\UCitem{Entradas}{-}

	\UCitem{Origen}{-}

	\UCitem{Salidas}{-}

	\UCitems{Destino}{%
        \refElem{IU-M01}
	}

	\UCitem{Precondiciones}{-}

	\UCitem{Postcondiciones}{-}

	\UCitem{Reglas de negocio}{-}

	\UCitem{Errores}{-}

	% \UCitem{Viene de}{% Indicar si el Caso de uso es primario o se extiende de otro. La mayoría se
					  % extienden de Login.
		% EJEMPLO: \refIdElem{PY-CU1} o Caso de uso primario.
	% 	\TODO Especificar.
	% }

 \UCsection[design]{Datos de Diseño}

	\UCitem[design]{Casos de Prueba}{%
        Incluidos en la ejecución de los casos de uso que incluyen a este caso de uso
    }

 \UCsection[admin]{Datos de Administración de Requerimiento}

	\UCitem[admin]{Observaciones}{-}

\end{UseCase}

\subsubsection{Trayectorias del caso de uso}

\begin{UCtrayectoria}%
%
    \Actor Presiona el botón \IUMenu de la pantalla \refElem{IU-M00}
    \Sistema Despliega el menú de navegación lateral.

    \Actor Selecciona la opción {\bf \IUAdminSitio Administración del sitio}
    \Sistema Carga la pantalla \refElem{IU-M01} con la pestaña de administración del
             sitio preseleccionada.

    \Actor Selecciona la pestaña {\bf plugins}. \refTray{A}
    \Sistema Carga la pantalla \refElem{IU-M01a}

\end{UCtrayectoria}

\begin{UCtrayectoriaA}{A}{El \refElem{aAdministrador} desea administrar los usuarios
dentro de la plataforma}

    \Actor Selecciona la pestaña {\bf Usuarios}.
    \Sistema Carga la pantalla \refElem{IU-M01b}.
\end{UCtrayectoriaA}

\subsubsection{Puntos de extensión}

\UCExtensionPoint{Instalación de un plugin}{%

    El \refElem{aAdministrador} desea extender la funcionalidad
    de moodle mediante la instalación de plugins.%
    }{Al inicio la trayectoria principal}{\refElem{CU-E01}}

\UCExtensionPoint{Configuraciones generales del módulo de experiencia}{%

    El \refElem{aAdministrador} desea cambiar las configuraciones
    generales del módulo de experiencia.%
    }{Al inicio la trayectoria principal}{\refElem{CU-E02}}

\UCExtensionPoint{Configuraciones visuales del módulo de experiencia}{%

    El \refElem{aAdministrador} desea establecer las configuraciones
    de la visualización de los niveles del módulo de experiencia.%
    }{Al inicio la trayectoria principal}{\refElem{CU-E02-1}}

\UCExtensionPoint{Configuraciones de comportamiento del módulo de experiencia}{%

    El \refElem{aAdministrador} desea establecer el comportamiento del
    sistema de experiencia que incluye el módulo de experiencia.%
    }{Al inicio la trayectoria principal}{\refElem{CU-E02-2}}

\UCExtensionPoint{Configuraciones de eventos del módulo de experiencia}{%

    El \refElem{aAdministrador} desea establecer la cantidad de experiencia
    que brindarán los eventos que soporta el módulo de experiencia.%
    }{Al inicio la trayectoria principal}{\refElem{CU-E02-3}}

\UCExtensionPoint{Desinstalación de un plugin}{%

    El \refElem{aAdministrador} desea desinstalar un plugin en
    moodle debido a que ya no requiere de las funcionalidades
    que este brinda%
    }{Al inicio la trayectoria principal}{\refElem{CU-E03}}

\UCExtensionPoint{Crear usuario gamificado}{%

    El \refElem{aAdministrador} desea crear un usuario en moodle
    para otorgarle acceso a las distintas funcionalidades que brinda
    moodle.%
    }{Al inicio la trayectoria principal}{\refElem{CU-E12}}

 % Acceder a la administración del sitio

    % MODULO DE EXPERIENCIA
    
% \ucstEnEdicion     Al terminar una revisión/aprobación con observaciones
%                    y al inicio del CU.
%
% \ucstEnRevision    Al terminar la edición del CU (version += 0.1).
% \ucstEnAprobacion  Al pasar la revision sin observaciones.
% \ucstAprobado      Al ser aprobado por el usuario (version += 1.0)

\begin{UseCase}[%
Autor/Daniel Ortega,%
Version/0.1,%
Estado/\ucstEnRevision]%
%
{CU-E01}{Instalar plugin del módulo de experiencia}{% TODO; Deberia se Instalar/Actualizar ???
%
 Permite al \refElem{aAdministrador} incluir todas las funcionalidades que brinda el módulo de
 experiencia al moodle que administra mediante la instalación de los plugins correspondientes.
 La conclusión de la trayectoria principal de esta caso de uso es una precondición para que los
 demás casos de uso del módulo de experiencia puedan ejecutarse.}

	\UCitem[control]{Revisor}{ Sin asignar }
	\UCitem[control]{Último cambio}{ 13/OCT/19 }

 \UCsection{Atributos}

    \UCitem{Actor(es)}{%
        \refElem{aAdministrador}
    }

	\UCitems{Propósito}{%
        \Titem Permitir al administrador incluir todas las funcionalidades que brinda el módulo de
        experiencia al moodle que administra.

        \Titem Permitir a los usuarios de moodle ver su progreso en la plataforma mediante puntos
        de experiencia.
	}

	\UCitem{Entradas}{\imprimeUC{entrada}}

	\UCitems{Origen}{%
        \Titem Mouse
	}

	\UCitem{Salidas}{\imprimeUC{salida}}

	\UCitem{Destino}{%
		\refElem{IU-M01}
	}

	\UCitems{Precondiciones}{%
        \Titem La carpeta comprimida que contiene los archivos del plugin
        \Titem El plugin debe cumplir con la regla \refElem{BR-M01} para poder ser
               instalado.
        % \Titem Si se trata de una actualización de un plugin la versión de este debe
               % cumplir con la regla \refElem{BR-M02}.
	}

	\UCitems{Postcondiciones}{%
        \Titem El plugin debe permanecer instalado en moodle.%
        \Titem La actualización de las \refElem{xp-general-settings} del módulo de experiencia
               deben persistirse en el sistema.
        \Titem Los \refElem[usuarios]{mdl-user} registrados en moodle deberán tener la
               asociada la información de un \refElem{xp-user}.
	}

	\UCitem{Reglas de negocio}{\imprimeUC{regla}}

	\UCitems{Errores}{%
        \Titem \UCerr{Err1}{%
            El archivo zip del plugin seleccionado está roto,}{% CAUSA
            no se puede continuar con la instalación del plugin}% EFECTO

        \Titem \UCerr{Err2}{%
            Alguna de las dependencias del plugin no se satisface con
            los plugins instalados}{% CAUSE
            no se puede continuar con las instalación del plugin}% EFECTO

        \Titem \UCerr{Err3}{%
        % CAUSA
            Durante la ejecución de las tareas de instalación del nuevo plugin ocurre
            un error,}{%
        % EFECTO
            las tareas de instalación del propio plugin no pudieron concluir apropiadamente,
            el plugin sigue }
	}

	% \UCitem{Viene de}{% Indicar si el Caso de uso es primario o se extiende de otro. La mayoría se
					  % extienden de Login.
		% EJEMPLO: \refIdElem{PY-CU1} o Caso de uso primario.
	% 	\TODO Especificar.
	% }

 \UCsection[design]{Datos de Diseño}

	\UCitems[design]{Casos de Prueba}{%
        \Titem \refElem{CPC-E01}
	}

 \UCsection[admin]{Datos de Administración de Requerimiento}

	\UCitem[admin]{Observaciones}{}

\end{UseCase}

\subsubsection{Trayectorias del caso de uso}

\begin{UCtrayectoria}%
%
    \includeUC{CU-M01}

    \Actor Selecciona la opción {\bf Instalar plugins}
    \Sistema Carga la pantalla \refElem{IU-M02} con el formulario para seleccionar el
             plugin a instalar. \label{CU-E01-formulario-instalacion}

    \Actor Presiona la opción {\bf Seleccione un archivo}
    \Sistema Despliega la pantalla \refElem{IU-M00a} como pantalla emergente
             \label{CU-E01-seleccion-archivo}

    \Actor Selecciona la opción {\it Subir un archivo} en el menu izquierdo de la pantalla
           emergente y posteriormente presiona el botón {\it Browse}.
    \Actor Selecciona el archivo que contiene al \entrada{Plugin} del módulo de experiencia.
    \Actor Presiona el botón {\bf Subir este archivo}.
    \Sistema Valida que el archivo del plugin sea de tipo {\it ZIP}. \refTray{A}
    \Sistema Cierra la pantalla emergente y muestra el nombre del archivo seleccionado en la
             pantalla \refElem{IU-M02}

    \Actor Presiona el botón {\bf Instalar plugin desde archivo ZIP}
    \Sistema Valida que el archivo {\it ZIP} cumpla con las restricciones dictadas por la
             \regla{BR-M01}. \refErr{Err1}
    \Sistema Obtiene el \salida{Plugin.componente}, la \salida{Plugin.fullpath} y el
             \salida{Plugin.pluginname} del plugin a ser instalado.
    \Sistema Carga la pantalla \refElem{IU-M02a}, mostrando los datos anteriormente obtenidos.

    \Actor Continua con la instalación de plugin presionando la opción {\bf continuar}. \refTray{B}

    \Sistema Obtiene tambien la \salida{Plugin.moodle}, la lista de \salida{Plugin.dependencies}
             y la \salida{Plugin.version} del plugin a instalar. \refErr{Err2}
             \label{CU-E01-comprobacion}
    \Sistema Despliega los datos obtenidos en la pantalla \refElem{IU-M02b}

    \Actor Presiona el botón {\bf Actualizar base de datos Moodle ahora}. \refTray{C}
    \Sistema Procesa las tareas de instalación de moodle.

    \Sistema Obtiene la lista de los \refElem[identificadores]{mdl-user.id} de los
             usuarios de moodle.
    \Sistema Asocia mediante los identificadores datos de un \refElem{xp-user}
             estableciendo el \refElem{xp-user.level} actual igual a $1$,
             la \refElem{xp-user.xp} igual a $0$.
    \Sistema Establece los valores por defecto para las \refElem{xp-visual-settings} (
              \entrada{xp-visual-settings.title},
              \entrada{xp-visual-settings.description},
              \entrada{xp-visual-settings.message},
              \entrada{xp-visual-settings.colorLvl},
              \entrada{xp-visual-settings.colorBar} e
              \entrada{xp-visual-settings.image}), especificadas en el modelo de información.

    \Sistema Carga la interfaz \refElem{IU-M02d} informando que la instalación
             ha sido llevaba a cabo de forma correcta. \refErr{Err3}

    \Actor Presiona el botón {\bf continuar}.

    \includeUC{CU-E02} a partir del paso \ref{CU-E03-formulario}
\end{UCtrayectoria}

\begin{UCtrayectoriaA}{A}{Cuando el archivo seleccionado es distinto de un ZIP}

  \Sistema Emite en una ventana emergente el mensaje {\it Error: ``El tipo de
           archivo \$EXT no se acepta.''} siendo {\it\$EXT} la extensión del
           archivo seleccionado.
  \Sistema Regresa al paso \ref{CU-E01-seleccion-archivo}.

\end{UCtrayectoriaA}

\begin{UCtrayectoriaA}[Fin del caso de uso]%
{B}{El \refElem{aAdministrador} desea cancelar la instalación después de la validación del archivo ZIP}

    \Actor Presiona el botón {\bf cancelar} de la pantalla \refElem{IU-M02a}.
    \Sistema Cancela la instalación del plugin y redirige a la pantalla \refElem{IU-M02}

\end{UCtrayectoriaA}

\begin{UCtrayectoriaA}[Fin del caso de uso]%
{C}{El \refElem{aAdministrador} desea cancelar la instalación después de ver la comprobación de plugins a instalar}

    \Actor Presiona el botón {\bf cancelar esta instalación} o {\bf cancelar las nuevas instalaciones}
    \Sistema Redirige a la pantalla \refElem{IU-M02c}

    \Actor Si el actor presiona el botón {\bf Continuar} entonces
    \UCpaso[--] el caso de uso terminará, (en caso contrario)

    \Actor Si el actor presiona el botón {\bf Cancelar}
    \Sistema Regresa al paso \ref{CU-E01-comprobacion}
\end{UCtrayectoriaA}

%\subsubsection{Puntos de extensión}

%\UCExtensionPoint{Nombre del punto de extensión}{%

%    El \refElem{aAdministrador} desea/requiere/necesita ....%
%
%    }{En el paso \ref{CU-ET-1x} de la trayectoria principal  ...%
%
%    }{\refElem{CU-E2-T}}
   % Instalar plugin del esquema de experiencia
    
% \ucstEnEdicion     Al terminar una revisión/aprobación con observaciones 
%                    y al inicio del CU.
%
% \ucstEnRevision    Al terminar la edición del CU (version += 0.1).
% \ucstEnAprobacion  Al pasar la revision sin observaciones.
% \ucstAprobado      Al ser aprobado por el usuario (version += 1.0)

\begin{UseCase}[%
Autor/Daniel Ortega,%
Version/0.1,%
Estado/\ucstEnEdicion]%
%
{CU-E02}{Configurar esquema de experiencia}{%
%
 Permite al \refElem{aAdministrador} configurar el esquema de experiencia, especificando
 la forma en que se obtienen los puntos de experiencia, la cantidad de puntos a otorgar, 
 el número de puntos de cada nivel y finalmente la visualización del nivel y de los puntos
 de cada usuario.}

	\UCitem[control]{Revisor}{ Sin asignar }
	\UCitem[control]{Último cambio}{ \today }

 \UCsection{Atributos}

    \UCitem{Actor(es)}{%
        \refElem{aAdministrador}
    }

	\UCitem{Propósito}{%
        Acceder a las menús de configuración del esquema de experiencia.
		% \Titem Establecer la cantidad de experiencia que brindarán los cursos.
        % \Titem Establecer la forma en que incrementa la cantidad de experiencia de cada nivel con respecto al anterior.
        % \Titem Configurar la visualización de la pantalla emergente al subir de nivel
        % \Titem Configurar la visualización del bloque que muestra el nivel actual del usuario.
        % \Titem Establecer visualización agrupando niveles
	}
	
%% BEGIN-BLOQUE PARA AGREGAR UNA REVISION ------------------------------------->
%% Copiar y descomentar este bloque por cada revision que se realice
%	\UCsection[control]{% Indicar la versión objeto de la revisión.
%		Revisión de la Versión \TODO X.X
%	}
%	\UCitem[control]{Revisó}{% Coloque el nombre de quien realizó la revisión
%		\TODO Especificar
%	}
%	\UCitem[control]{Fecha}{% Coloque la fecha de la revisión
%		% EJEMPLO: 21 de Septiembre de 2019.
%		\TODO Especificar
%	}
%	\UCitem[control]{Resultado}{% Las opciones son: 
%								% Pendiente: se pasa a EnEdicion y se agregan las observaciones
%								% Aprobado: Se pasa a EnAprobacion.
%		\TODO Especificar
%	}
%	\UCitems[control]{Observaciones}{
%		% Agregar las observaciones resultado de la revision o la palabra ``Ninguna''
%		\Titem \TODO Agregar observaciones en cada viñeta, usar el comando \TODO %\TOCHK \DONE.
%	}
%% <------------------------------------------ END-BLOQUE PARA AGREGAR UNA REVISION
	
	\UCitems{Entradas}{%
		\Titem Selección de la opción \IUMenu menu.
        \Titem Selección de la opción \IUAdminSitio Administración del Sitio.
        \Titem Selección de la pestaña {\it Plugins}.
	}

	\UCitems{Origen}{%
        \Titem Mouse
	}

	\UCitems{Salidas}{%
		\imprimeUC{salida}
	}

	\UCitems{Destino}{%
		\Titem \refElem{IU-M02a}
	}
	
	\UCitems{Precondiciones}{%
        % TODO: ¿Se deben especificar aunque eso sea diseño?
        \Titem Que los plugins del módulo de experiencia se encuentren instalados
	}

	\UCitem{Postcondiciones}{%
        Ninguna
	}

	\UCitem{Reglas de negocio}{%
		Ninguna
	}

	\UCitems{Errores}{%
        \Titem \UCerr{Err1}{%
        % CAUSA
            Los plugins del módulo de experiencia no se encuentran instalados,}{%
        % EFECTO
            No se presenta en el menu de opciones las opciones para modificar %
            el esquema de experiencia.}
	}

	% \UCitem{Viene de}{% Indicar si el Caso de uso es primario o se extiende de otro. La mayoría se 
					  % extienden de Login.
		% EJEMPLO: \refIdElem{PY-CU1} o Caso de uso primario.
	% 	\TODO Especificar.
	% }	

 \UCsection[design]{Datos de Diseño}

	\UCitems[design]{Casos de Prueba}{%
        \Titem \refElem{CPC-E02}
        \Titem \refElem{CPI-E02}
	}

 \UCsection[admin]{Datos de Administración de Requerimiento}

	\UCitem[admin]{Observaciones}{%
        Ninguna
	}

\end{UseCase}

\clearpage
\subsubsection{Trayectorias del caso de uso}

\begin{UCtrayectoria}%
%
 \Actor Presiona el botón \IUMenu en la esquina superior izquierda de la pantalla \refElem{IU-M01}
        para abrir el menu de navegación.

 \Actor Selecciona la opción {\it \IUAdminSitio Administración del sitio}

 \Sistema Carga la pantalla \refElem{IU-M02}

 \Actor Selecciona ver las opciones para administrar plugins presionando la pestaña
        {\it Plugins}

 \Sistema Obtiene las categorias de las opciones que se muestran en la pantalla.

 \Sistema En la categoría {\it Bloques} agrega la subcategoría \salida{tExpSettingsGeneral} con
          los enlaces a las configuraciones \salida{tExpSettingsVisual} y 
          \salida{tExpSettingsComportamiento} del módulo de experiencia. \refErr{Err1}

 \Sistema Carga la pantalla \refElem{IU-M02a}.

    % \Sistema ... \label{CU-E2-Menu}
    % \Actor ...  \refTray{B} \ref{CU-E2-Menu}
\end{UCtrayectoria}


\subsubsection{Puntos de extensión}

\UCExtensionPoint{Configuraciones generales}{%

    El \refElem{aAdministrador} desea cambiar las configuraciones generales del módulo de 
    experiencia, las cuales incluyen habilitar/deshabilitar el módulo de experiencia o
    la entrega de recompensa a otros eventos de Gamedle.%
%
    }{En el paso \ref{CU-E2-1x} de la trayectoria principal%
%
    }{\refElem{CU-E02-1}}

\UCExtensionPoint{Configuraciones Visuales}{%

    El \refElem{aAdministrador} desea cambiar las configuraciones generales del módulo de 
    experiencia, las cuales incluyen habilitar/deshabilitar el módulo de experiencia o
    la entrega de recompensa a otros eventos de Gamedle.%
%
    }{En el paso \ref{CU-E2-1x} de la trayectoria principal%
%
    }{\refElem{CU-E02-2}}

\UCExtensionPoint{Configuraciones de Comportamiento}{%

    El \refElem{aAdministrador} desea cambiar las configuraciones generales del módulo de 
    experiencia, las cuales incluyen habilitar/deshabilitar el módulo de experiencia o
    la entrega de recompensa a otros eventos de Gamedle.%
%
    }{En el paso \ref{CU-E2-1x} de la trayectoria principal%
%
    }{\refElem{CU-E02-3}}

   % Configuraciones generales
    
% \ucstEnEdicion     Al terminar una revisión/aprobación con observaciones
%                    y al inicio del CU.
%
% \ucstEnRevision    Al terminar la edición del CU (version += 0.1).
% \ucstEnAprobacion  Al pasar la revision sin observaciones.
% \ucstAprobado      Al ser aprobado por el usuario (version += 1.0)

\begin{UseCase}[%
Autor/Daniel Ortega,%
Version/0.1,%
Estado/\ucstEnRevision]%
%
{CU-E02-1}{Configurar visualización de niveles}{%
%
 Permite al \refElem{aAdministrador} establecer y modificar los colores, textos e
 imagenes que se muestran en la visualización del nivel actual y en las ventanas
 emergentes de un nuevo nivel alcanzado. Cuando un administrador actualice los valores
 de la configuración los usuarios serán capaces de ver los cambios al renderizarse la
 siguiente página.}

	\UCitem[control]{Revisor}{ Sin asignar }
	\UCitem[control]{Último cambio}{ \today }

 \UCsection{Atributos}

    \UCitem{Actor(es)}{%
        \refElem{aAdministrador}
    }

	\UCitem{Propósito}{%
      \Titem Cambiar las \refElem{xp-visual-settings} del módulo de experiencia.
      \Titem Cambiar el color de la barra de progreso del nivel actual que ve cada
             usuario.
      \Titem Establecer la imagen que se presenta en cada nivel de la plataforma.
      \Titem El mensaje de felicitaciones que se presenta al alcanzar un nuevo nivel
      \Titem Establecer el color que tendrá el número del nivel
      \Titem Cambiar la descripción de los mensajes visible cuando se alcanza un
             nuevo nivel.
      \Titem Cambiar el nombre de los niveles.
	}

	\UCitem{Entradas}{\imprimeUC{entrada}}

	\UCitems{Origen}{%
        \Titem Mouse
        \Titem Teclado
	}

	\UCitem{Salidas}{\imprimeUC{salida}}

	\UCitems{Destino}{%
		\Titem \refElem{IU-M01}
	}

	\UCitems{Precondiciones}{%
        \Titem Que los plugins del módulo de experiencia se encuentren instalados,
        \Titem El módulo de experiencia haya sido habilitando en el caso uso
               \refElem{CU-E02}
	}

	\UCitem{Postcondiciones}{%
        Los nuevos valores de las \refElem{xp-visual-settings} deben ser actualizados
        para todos los usuarios y además deben persistirse en el sistema.
	}

	\UCitem{Reglas de negocio}{\imprimeUC{regla}}

	\UCitems{Errores}{%
        \Titem \UCerr{Err1}{%
        % CAUSA
            Los plugins del módulo de experiencia no se encuentran instalados,}{%
        % EFECTO
            No se presentan en el menú las opciones para acceder a la pantalla
            de la configuración y no se puede llevar a cabo el caso de uso.}

        \Titem \UCerr{Err2}{%
        % CAUSA
            La imagen no cumple con las restricciones de nombre de la regla de negocio
            \refElem{BR-E01},}{%
        % EFECTO
            No se remplaza la imagen de la configuración actual, se realizan
            las demás actualizaciones con los datos ingresados por el usuario,
            y se emite el mensage que indica que el nombre es inválido.}

        \Titem \UCerr{Err3}{%
        % CAUSA
            Ocurre un fallo durante la persistencia de la imagen en el sistema,}{%
        % EFECTO
            Se interrumpe la actualización de la imagen, se procede con las
            demás actualizaciones y se emite el mensaje de error conrrespondiente}
	}

 \UCsection[design]{Datos de Diseño}

	\UCitems[design]{Casos de Prueba}{%
        \Titem \refElem{CPC-E02-1}
        \Titem \refElem{CPI-E02-1a}
        \Titem \refElem{CPI-E02-1b}
	}

 \UCsection[admin]{Datos de Administración de Requerimiento}

	\UCitem[admin]{Observaciones}{%
        Ninguna
	}

\end{UseCase}

\subsubsection{Trayectorias del caso de uso}

\begin{UCtrayectoria}%
  \includeUC{CU-M01} \refErr{Err1}

  \Actor Presiona la opción {\bf \refElem{tExpSettingsVisual}} en la categoría
         \refElem{tExpCategoria}.
  \Sistema Obtiene el valor de si el módulo de experiencia está \refElem[activado]%
           {xp-general-settings.activated} o no. \refTray{A}
  \Sistema Obtiene los valores actuales de la configuración:
           \salida{xp-visual-settings.title},
           \salida{xp-visual-settings.description},
           \salida{xp-visual-settings.message},
           \salida{xp-visual-settings.colorLvl} y
           \salida{xp-visual-settings.colorBar}.

  \Sistema Carga la pantalla \refElem{IU-E03} estableciendo como valores por defecto
           las \refElem{tExpSettingsVisual} actuales obtenidas.

  \Actor Ingresa los valores de \entrada{xp-visual-settings.title},
         \entrada{xp-visual-settings.description} y
         \entrada{xp-visual-settings.message} para los campos requeridos.
         \label{CU-E02-1.formulario}
  \Actor Ingresa los valores para el \entrada{xp-visual-settings.colorLvl} y
         \entrada{xp-visual-settings.colorBar} seleccionando el color y la
         tonalidad mediante el \refElem{tSelectColor}. \refTray{B}
         \label{CU-E02-1.color}
  \Actor Presiona la opción {\bf Seleccione un archivo}. \refTray{C}
  \Sistema Despliega la pantalla \refElem{IU-M00a} como pantalla emergente
           \label{CU-E02-1.seleccion-archivo}

  \Actor Selecciona la opción {\it Subir un archivo} en el menu izquierdo de la
         pantalla emergente y posteriormente presiona el botón {\it Browse}.
  \Actor Selecciona el archivo de la \entrada{xp-visual-settings.image}.
  \Actor Presiona el botón {\bf Subir este archivo}.
  \Sistema Valida que el archivo tenga alguna de las extensiones de indicadas
           por la regla \regla{BR-E01}. \refTray{D}
  \Sistema Cierra la pantalla emergente y muestra el nombre del archivo seleccionado
           en la pantalla \refElem{IU-E03}.


  \Actor Presiona el botón {\bf Guardar Cambios}. \refTray{E} \label{CU-E02-1.validacion}

  \Sistema Valida que las opciones ingresadas cumplan con las restricciones
           especificadas en el modelo de información. \refTray{F}.

  \Sistema Valida que la \refElem{xp-visual-settings.image} proporcionada cumpla con
           las restricciones de nombre de archivo establecida por la regla
           \refElem{BR-E01}. \refErr{Err2}
  \Sistema Remplaza la imagen de los niveles por la imagen propocionada por el usuario,
           actualiza los demás valores de las configuraciones. \refErr{Err3}
  \Sistema Despliega la pantalla \refElem{IU-E03} con el mensaje que indica que
           los datos han sido actualizados exitosamente.
\end{UCtrayectoria}

\begin{UCtrayectoriaA}{A}{El módulo de experiencia no se encuentra activado}

  \Sistema Carga la pantalla \refElem{IU-E03a}.

  \Actor Presiona el botón {\bf Activar módulo de experiencia}
  \includeUC{CU-E02} a partir del paso \ref{CU-E02-ir-a-formulario},
                     para activar el módulo de experiencia.

  \Sistema Regresa al inicio de la trayectoria principal.

\end{UCtrayectoriaA}

\begin{UCtrayectoriaA}{B}{%
El \refElem{aAdministrador} desea especificar el valor del \refElem{tColor}
directamente}

    \Actor Ingresa el valor hexadecimal del color para el
           \refElem{xp-visual-settings.colorLvl} o
           \refElem{xp-visual-settings.colorBar}.

    \Sistema Continua en el paso \ref{CU-E02-1.color} de la trayectoria principal.

\end{UCtrayectoriaA}

\begin{UCtrayectoriaA}{C}{%
No desea cambiar la \refElem{xp-visual-settings.image} actual de los niveles}

    \Sistema Continua en el paso \ref{CU-E02-1.validacion} de la trayectoria principal
\end{UCtrayectoriaA}

\begin{UCtrayectoriaA}{D}{Cuando el archivo seleccionado es distinto de {\it png o jpg}}

  \Sistema Emite en una ventana emergente el mensaje {\it Error: ``El tipo de
           archivo \$EXT no se acepta.''} siendo {\it\$EXT} la extensión del
           archivo seleccionado.
  \Sistema Regresa al paso \ref{CU-E02-1.seleccion-archivo}.

\end{UCtrayectoriaA}

\begin{UCtrayectoriaA}[Fin del caso de uso]{E}{%
Desea cancelar la actualización de las configuraciones}

  \Actor Presiona el botón cancelar.
  \Sistema Redirige a la pantala \refElem{IU-M01}.

\end{UCtrayectoriaA}

\begin{UCtrayectoriaA}{F}{%
Alguno de los datos ingresados no cumple con las restricciones en el modelo de
información}

    \Sistema Imprime los mensajes de error abajo de los campos con valores incorrectos.

    \Actor Ingresa nuevamente los valores para los campos marcados como incorrectos.
    \Sistema Regresa al paso \ref{CU-E02-1.formulario}

\end{UCtrayectoriaA}

 % Configurar visualización de niveles
    
% \ucstEnEdicion     Al terminar una revisión/aprobación con observaciones 
%                    y al inicio del CU.
%
% \ucstEnRevision    Al terminar la edición del CU (version += 0.1).
% \ucstEnAprobacion  Al pasar la revision sin observaciones.
% \ucstAprobado      Al ser aprobado por el usuario (version += 1.0)

\begin{UseCase}[%
Autor/Daniel Ortega,%
Version/0.1,%
Estado/\ucstEnEdicion]%
%
{CU-E02-2}{Configurar sistema de experiencia}{%
%
 Permite al \refElem{aAdministrador} establecer y modificar las cantidades de puntos
 de experiencia que brindan los cursos en la plataforma y la forma en que aumenta
 la cantidad de experiencia requerida para pasar de un nivel al siguiente. Se 
 definieron dos formas en la que la experiencia de los niveles se determina, ambas
 estan definidas por la regla de negocio \refElem{BR-E02}}

	\UCitem[control]{Revisor}{ Sin asignar }
	\UCitem[control]{Último cambio}{ \today }

 \UCsection{Atributos}

    \UCitem{Actor(es)}{%
        \refElem{aAdministrador}
    }

	\UCitems{Propósito}{%
        \Titem Permitir al administrador configurar el sistema de experiencia.
        \Titem Establecer o modificar la cantidad de experiencia que brindan los
               cursos.
        \Titem Establecer la cantidad de experiencia requerida para pasar el primer
               nivel usada como calculo para los demás niveles.
        \Titem Cambiar la forma en cómo se incrementa la cantidad de experiencia
               requerida para avanzar de un nivel a otro.
	}
	
	\UCitem{Entradas}{\imprimeUC{entrada}}

	\UCitems{Origen}{%
        \Titem Mouse
        \Titem Teclado
	}

	\UCitem{Salidas}{\imprimeUC{salida}}

	\UCitems{Destino}{%
		\Titem \refElem{IU-E04}
	}
	
	\UCitems{Precondiciones}{%
        \Titem Que los plugins del módulo de experiencia se encuentren instalados
        \Titem El módulo de experiencia debe estár habilitado en el caso de uso
               \refElem{CU-E02}.
	}

	\UCitem{Postcondiciones}{%
        Los nuevos valores de las \refElem{xp-scheme-settings} deber ser
        estár actualizados para todos los usuarios, además de persistirse en el
        sistema.
	}

	\UCitem{Reglas de negocio}{\imprimeUC{regla}}

	\UCitems{Errores}{%
        \Titem \UCerr{Err1}{%
        % CAUSA
            Los plugins del módulo de experiencia no se encuentran instalados,}{%
        % EFECTO
            no se presentan las opciones en el menú y por lo tanto no se puede
            acceder a las configuraciones}
	}

	% \UCitem{Viene de}{% Indicar si el Caso de uso es primario o se extiende de otro. La mayoría se 
					  % extienden de Login.
		% EJEMPLO: \refIdElem{PY-CU1} o Caso de uso primario.
	% 	\TODO Especificar.
	% }	

 \UCsection[design]{Datos de Diseño}

	\UCitems[design]{Casos de Prueba}{%
        \Titem \refElem{CPC-E02-2}
        \Titem \refElem{CPI-E02-2a}
        \Titem \refElem{CPI-E02-2b}
        \Titem \refElem{CPI-E02-2c}
	}

 \UCsection[admin]{Datos de Administración de Requerimiento}

	\UCitem[admin]{Observaciones}{%
        Ninguna
	}

\end{UseCase}

\subsubsection{Trayectorias del caso de uso}

\begin{UCtrayectoria}%
%
  \includeUC{CU-M01} \refErr{Err1}

  \Actor Presiona la opción {\bf \refElem{tExpSettingsComportamiento}} en la categoría
         \refElem{tExpCategoria}. \refTray{A} 
  \Sistema Obtiene el valor de si el módulo de experiencia está \refElem[activado]%
           {xp-general-settings.activated} o no. \refTray{B} \label{CU-E02-2-loading}
  \Sistema Obtiene los valores actuales de la configuración del sistema de experiencia:
           \salida{xp-scheme-settings.increment},
           \salida{xp-scheme-settings.incrementValue},
           \salida{xp-scheme-settings.levelXP} y
           \salida{xp-scheme-settings.courseXP}.
  \Sistema Carga la pantalla \refElem{IU-E04} estableciendo como valores por defecto
           las \refElem{xp-scheme-settings} obtenidas en el anterior paso.
  \Sistema Muestra la descripción de la regla \regla{BR-E03} para informar al 
           \refElem{aAdministrador} del comportamiento que tendrá el sistema en 
           caso de modificarse la cantidad de experiencia correspondiente al primer
           nivel.

  \Sistema Muestra la descripción de la regla \regla{BR-E04} para informar al 
           \refElem{aAdministrador} del comportamiento que tendrá el sistema si se 
           modificarse la cantidad de experiencia que otorgan los cursos.

  \Actor Especifica si el \entrada{xp-scheme-settings.increment} en la cantidad de
         experiencia de los niveles será {\it Lineal} o {\it Percentual}.
  \Actor Ingresa el valor para el \entrada{xp-scheme-settings.incrementValue} con
         base en la regla \regla{BR-E02}.
  \Actor Ingresa los valores para la \entrada{xp-scheme-settings.levelXP} y la
         \entrada{xp-scheme-settings.courseXP}.
  \Actor Presiona la opción {\bf Guardar Cambios}. \refTray{C} \label{CU-E02-2-submit}

  \Sistema Valida que los valores ingresados por el usuario cumplan con las
           restricciones especificadas en el modelo de información.
  \Sistema Verifica que el \refElem{xp-scheme-settings.incrementValue} cumpla
           con la regla \refElem{BR-E02}. \refTray{D}
  \Sistema Actualiza los valores de las \refElem{xp-scheme-settings} con los
           ingresados por el usuario.
  \Sistema Despliega la pantalla \refElem{IU-E04} con el mensaje de que los datos
           han sido actualizados exitosamente.

\end{UCtrayectoria}

\begin{UCtrayectoriaA}{A}{
El \refElem{aAdministrador} selecciona la categoría \refElem{tExpCategoria}}
  \Sistema Carga la pantalla \refElem{IU-E01}
  \Actor Regresa al paso \ref{CU-E02-2-loading}
\end{UCtrayectoriaA}

\begin{UCtrayectoriaA}{B}{
El módulo de experiencia no se encuentra activado}
  \Sistema Carga la pantalla \refElem{IU-E03a}.
  \Actor Presiona el botón {\bf Activar módulo de experiencia}
  \includeUC{CU-E02} a partir del paso \ref{CU-E02-ir-a-formulario},
                     para activar el módulo de experiencia.

  \Sistema Regresa al inicio de la trayectoria principal.

\end{UCtrayectoriaA}

\begin{UCtrayectoriaA}{C}{
El \refElem{aAdministrador} desea cancelar la modificación en el sistema de
experiencia}

  \Actor Presiona el botón {\bf Cancelar}.
  \Sistema Redirige a la pantalla \refElem{IU-M01}.
\end{UCtrayectoriaA}

\begin{UCtrayectoriaA}{D}{
Alguno de los valores ingresados por el usuario son incorrectos.}
  \Sistema Imprime los mensajes de error abajo de los campos con los valores
           incorrectos.
  \Actor Ingresa nuevamente los valores en los campos marcados como incorrectos.
  \Sistema Regresa al paso \ref{CU-E02-2-submit}.

\end{UCtrayectoriaA}
 % Configurar esquema de experiencia
    
% \ucstEnEdicion     Al terminar una revisión/aprobación con observaciones 
%                    y al inicio del CU.
%
% \ucstEnRevision    Al terminar la edición del CU (version += 0.1).
% \ucstEnAprobacion  Al pasar la revision sin observaciones.
% \ucstAprobado      Al ser aprobado por el usuario (version += 1.0)

\begin{UseCase}[%
Autor/Daniel Ortega,%
Version/0.1,%
Estado/\ucstEnEdicion]%
%
{CU-E02-3}{Configurar esquema de experiencia}{%
%
 Permite al \refElem{aActor} .}

	\UCitem[control]{Revisor}{ Sin asignar }
	\UCitem[control]{Último cambio}{ \today }

 \UCsection{Atributos}

    \UCitem{Actor(es)}{%
        \refElem{aActor}
    }

	\UCitem{Propósito}{%
        ...
	}
	
	\UCitem{Entradas}{\imprimeUC{entrada}}

	\UCitems{Origen}{%
        \Titem Mouse
        \Titem Teclado
	}

	\UCitem{Salidas}{\imprimeUC{salida}}

	\UCitems{Destino}{%
		\Titem \refElem{IU-M02a}
	}
	
	\UCitems{Precondiciones}{%
        \Titem ...
	}

	\UCitem{Postcondiciones}{%
        Ninguna
	}

	\UCitem{Reglas de negocio}{%
		Ninguna
	}

	\UCitems{Errores}{%
        \Titem \UCerr{Err1}{%
        % CAUSA
            ...,}{%
        % EFECTO
            ...}
	}

	% \UCitem{Viene de}{% Indicar si el Caso de uso es primario o se extiende de otro. La mayoría se 
					  % extienden de Login.
		% EJEMPLO: \refIdElem{PY-CU1} o Caso de uso primario.
	% 	\TODO Especificar.
	% }	

 \UCsection[design]{Datos de Diseño}

	\UCitems[design]{Casos de Prueba}{%
        \Titem \refElem{CPC-E0Y}
        \Titem \refElem{CPI-E0Y}
	}

 \UCsection[admin]{Datos de Administración de Requerimiento}

	\UCitem[admin]{Observaciones}{%
        Ninguna
	}

\end{UseCase}

\clearpage
\subsubsection{Trayectorias del caso de uso}

\begin{UCtrayectoria}%
%
 \Actor Presiona el botón \IUMenu en la esquina superior izquierda de la pantalla \refElem{IU-M01}
        para abrir el menu de navegación.

 \Actor Selecciona la opción {\it \IUAdminSitio Administración del sitio}

 \Sistema Carga la pantalla \refElem{IU-M02}

\end{UCtrayectoria}


\subsubsection{Puntos de extensión}

\UCExtensionPoint{Nombre del punto de extensión}{%

    El \refElem{aAdministrador} desea/requiere/necesita ....%
%
    }{En el paso \ref{CU-ET-1x} de la trayectoria principal  ...%
%
    }{\refElem{CU-E2-T}}

 % Configurar Eventos de experiencia
    
% \ucstEnEdicion     Al terminar una revisión/aprobación con observaciones 
%                    y al inicio del CU.
%
% \ucstEnRevision    Al terminar la edición del CU (version += 0.1).
% \ucstEnAprobacion  Al pasar la revision sin observaciones.
% \ucstAprobado      Al ser aprobado por el usuario (version += 1.0)

\begin{UseCase}[%
Autor/Daniel Ortega,%
Version/0.1,%
Estado/\ucstEnEdicion]%
%
{CU-E03}{Desinstalar plugins del esquema de experiencia}{%
%
 Permite al \refElem{aAdministrador} desinstalr el o los plugins correspondientes
 al módulo de experiencia cuando desee remover las funcionalidades que estos otorgan
 y eliminar los registros de los puntos de experiencia de los usuarios y cursos.}

	\UCitem[control]{Revisor}{ Sin asignar }
	\UCitem[control]{Último cambio}{ \today }

 \UCsection{Atributos}

    \UCitem{Actor(es)}{%
        \refElem{aAdministrador}
    }

	\UCitem{Propósito}{%
        Remover el módulo de experiencia junto con las funcionalidades que
        brinda y los datos generados por el mismo.
	}
	
	\UCitem{Entradas}{\imprimeUC{entrada}}

	\UCitems{Origen}{%
        \Titem Mouse
	}

	\UCitem{Salidas}{Ninguna}

	\UCitem{Destino}{%
		\refElem{IU-M03}
	}
	
	\UCitem{Precondiciones}{%
        Los plugins que se desean instalar deben estar previamente 
        instalados.
	}

	\UCitem{Postcondiciones}{%
        Los datos de generados por el esquema de experiencia deben ser 
        eliminados una vez concluida la desinstalación del o de los plugins.
	}

	\UCitem{Reglas de negocio}{Ninguna}

	\UCitems{Errores}{%
        \Titem \UCerr{Err1}{%
        % CAUSA
            El plugin a desinstalar no se encuentra instalado}{%
        % EFECTO
            no se puede proceder con la ejecución debido a que ya está 
            desinstalado}
	}

 \UCsection[design]{Datos de Diseño}

	\UCitems[design]{Casos de Prueba}{%
        \Titem \refElem{CPC-E03}
	}

 \UCsection[admin]{Datos de Administración de Requerimiento}

	\UCitem[admin]{Observaciones}{%
        Ninguna
	}

\end{UseCase}

\subsubsection{Trayectorias del caso de uso}

\begin{UCtrayectoria}%
%
  \includeUC{CU-M01}
  \Actor Presiona la opción {\bf Vista General de Plugins}.
  \Sistema Carga la pantalla \refElem{IU-M03}.

  \Actor Pide consultar únicamente los plugins adicionales instalados presionando
         la opción {\bf Plugins adicionales}..
  \Sistema Carga la pantalla \refElem{IU-M03a}.

  \Actor Presiona la opción {\bf Desinstalar} correspondiente al
         \entrada[plugin]{Plugin} del módulo de experiencia que desea desinstalar.
  \Sistema Muestra la pantalla \refElem{IU-M04}.
  \Sistema Pide la confirmación del usuario para continuar con la desinstalación.

  \Actor Presiona el botón de {\bf Aceptar}. \refTray{A}
  \Sistema ...
  \Sistema ...
  \Sistema ...
  \Sistema ...

  \Sistema Muestra la pantalla \refElem{IU-M04a} pidiendo al usuario la confirmación
           para eliminar los archivos del plugin. \refTray{B}
  \Sistema Elimina los archivos correspondientes al plugin.

\end{UCtrayectoria}

\begin{UCtrayectoriaA}{A}{%
El \refElem{aAdministrador} desea cancelar la desinstalación del plugin
}
\end{UCtrayectoriaA}

\begin{UCtrayectoriaA}{B}{%
El \refElem{aAdministrador} no desea eliminar los archivos del plugin}
\end{UCtrayectoriaA}

   % Desinstalar plugin del esquema de experiencia
    
% \ucstEnEdicion     Al terminar una revisión/aprobación con observaciones
%                    y al inicio del CU.
%
% \ucstEnRevision    Al terminar la edición del CU (version += 0.1).
% \ucstEnAprobacion  Al pasar la revision sin observaciones.
% \ucstAprobado      Al ser aprobado por el usuario (version += 1.0)

\begin{UseCase}[%
Autor/Daniel Ortega,%
Version/0.1,%
Estado/\ucstEnEdicion]%
%
{CU-E04}{Crear curso con experiencia}{%
%
 Permite al \refElem{aProfesor} .}

	\UCitem[control]{Revisor}{ Sin asignar }
	\UCitem[control]{Último cambio}{ \today }

 \UCsection{Atributos}

    \UCitem{Actor(es)}{%
        \refElem{aProfesor}, \refElem{aAdministrador}.
    }

	\UCitem{Propósito}{%
        Crear un curso con soporte para brindar puntos de experiencia conforme
        van completando las secciones del curso para que los estudiantes los acumulen
        y para subir de nivel.
	}

	\UCitem{Entradas}{\imprimeUC{entrada}}

	\UCitems{Origen}{%
        \Titem Mouse
        \Titem Teclado
	}

	\UCitem{Salidas}{\imprimeUC{salida}}

	\UCitem{Destino}{%
		\refElem{IU-E06a}
	}

	\UCitems{Precondiciones}{%
        \Titem Los plugins correspondientes al módulo de experiencia deben de estar
               habilitados.
        \Titem El módulo de experiencia debe de estar habilitado por el
               \refElem{aAdministrador}.
	}

	\UCitem{Postcondiciones}{%
        Las configuraciones particulares para este curso deben ser almacenadas
        en el sistema.
	}

	\UCitem{Reglas de negocio}{\imprimeUC{regla}}

	\UCitems{Errores}{%
        \Titem \UCerr{Err1}{%
        % CAUSA
            Los plugins correspondientes al módulo de experiencia no se encuentran
            instalados,}{%
        % EFECTO
            no aparece la opción para el formato gamificado, termina el caso de uso.}

        \Titem \UCerr{Err2}{%
        % CAUSA
            El módulo de experiencia se encuentra deshabilitado en la plataforma,}{%
        % EFECTO
            se le solicita al \refElem{aAdministrador} que habilite la experiencia}
	}

	% \UCitem{Viene de}{% Indicar si el Caso de uso es primario o se extiende de otro. La mayoría se
					  % extienden de Login.
		% EJEMPLO: \refIdElem{PY-CU1} o Caso de uso primario.
	% 	\TODO Especificar.
	% }

 \UCsection[design]{Datos de Diseño}

	\UCitems[design]{Casos de Prueba}{%
        \Titem \refElem{CPC-E04} % Creado chingon
        \Titem \refElem{CPI-E04} % Módulo de experiencia dehabilitado
	}

 \UCsection[admin]{Datos de Administración de Requerimiento}

	\UCitem[admin]{Observaciones}{%
        Ninguna
	}

\end{UseCase}

\subsubsection{Trayectorias del caso de uso}

\begin{UCtrayectoria}%
  \includeUC{CU-M01} presionando la pestaña {\bf Cursos}

  \Actor Presiona la opción {\bf Gestionar cursos y categorías}.
  \Sistema Obtiene las \refElem[categorias]{mdl-course-category} y
           \refElem{mdl-course.fullname} de los cursos presentes en la plataforma.
  \Sistema Muestra la pantalla \refElem{IU-M06}.

  \Actor Selecciona el botón nuevo curso de la categoría seleccionada por defecto.
  \Sistema Muestra la pantalla \refElem{IU-E06}.

  \Actor Introduce el \entrada{mdl-course.fullname}, \entrada{mdl-course.shortname} y
         la categoría a la cual pertenecerá el curso.

  \Actor Despliega la sección para el formato del curso.
  \Actor Selecciona el \entrada{xp-course.format} de curso gamificado {\it(gamedle)}.
         \refErr{Err1} \label{CU-E04-format}

  \Actor Especifica las opciones \entrada{xp-course.sections},
         \entrada{xp-course.hiddensections} y  \entrada{xp-course.coursedisplay}
         del formato de curso gamificado. \refTray{A} \refErr{Err2}

  \Actor Opcionalmente incluye valores en los demás campos opcionales del formulario.
  \Actor Presiona el botón {\bf Guardar cambios y mostrar}. \refTray{B} \refTray{C}
         \refTray{D} \label{CU-E04-submit}.

  \Sistema Obtiene los datos ingresados por el usuario.
  \Sistema Crea un curso \salida{mdl-course} junto con la cantidad de
           \salida[secciones]{mdl-course-section} especificadas por el usuario.
  \Sistema Crea un \salida{xp-course} con los valores de  \salida{xp-course.hiddensections}
           y \salida{xp-course.coursedisplay} ingresados por el usuario.
  \Sistema Por cada una de las \salida[secciones]{xp-course.sections} del curso gamificado
           crea las \salida[secciones gamificadas]{xp-course-section} correspondientes.
  \Sistema Establece la \salida{xp-course-section.xp} de cada seccion del curso con base
           en la regla de negocio \regla{BR-E08}. \label{CU-E04-finish}
  \Sistema Muestra la pantalla con \refElem{IU-E06a} con los datos del curso recién
           creado.

\end{UCtrayectoria}

\begin{UCtrayectoriaA}{A}{%
El módulo de experiencia se encuentra deshabilitado
}
  \Sistema Muestra el mensaje informando que el módulo de experiencia está
           deshabilitado.
  \Actor Si es que desea continuar y crear un curso gamificado sin la opción
         de brindar experiencia, prosigue en el paso \ref{CU-E04-format}

\end{UCtrayectoriaA}

\begin{UCtrayectoriaA}{B}{%
Alguno de los campos introducidos por el usuario es erróneo.
}

  \Sistema Muestra los mensajes de error correspondientes en los campos que
           contienen datos inválidos.
  \Actor Ingresa de nuevo los campos que contiene errores.
  \Sistema Regresa al paso \ref{CU-E04-submit}

\end{UCtrayectoriaA}

\begin{UCtrayectoriaA}{C}{%
El usuario desea regresar a la pantalla \refElem{IU-M06}.
}

  \Actor Presiona el botón {\bf Guardar y regresar}
  \Sistema Ejecuta los pasos a partir del paso \ref{CU-E04-submit} hasta el
           paso \ref{CU-E04-finish}.
  \Sistema Redirige a la pantalla \refElem{IU-M06}

\end{UCtrayectoriaA}

\begin{UCtrayectoriaA}{D}{%
El usuario desea cancelar la creación del curso
}

  \Actor Presiona el botón {\bf Cancelar}
  \Sistema Redirige a la pantalla \refElem{IU-M06}

\end{UCtrayectoriaA}
   % Crear un curso gamificado
    
% \ucstEnEdicion     Al terminar una revisión/aprobación con observaciones 
%                    y al inicio del CU.
%
% \ucstEnRevision    Al terminar la edición del CU (version += 0.1).
% \ucstEnAprobacion  Al pasar la revision sin observaciones.
% \ucstAprobado      Al ser aprobado por el usuario (version += 1.0)

\begin{UseCase}[%
Autor/Daniel Ortega,%
Version/0.1,%
Estado/\ucstEnEdicion]%
%
{CU-E05}{Habilitar el soporte para experiencia en un curso}{%
%
 Permite al \refElem{aProfesor} incluir el soporte para que las secciones del curso
 que administra brinden experiencia a los alumnos conforme estos las vayan completando.}

	\UCitem[control]{Revisor}{ Sin asignar }
	\UCitem[control]{Último cambio}{ \today }

 \UCsection{Atributos}

    \UCitem{Actor(es)}{%
        \refElem{aProfesor}
    }

	\UCitem{Propósito}{%
        Agregar soporte para brindar experiencia en cualquier curso que haya sido creado
        en moodle.
	}
	
	\UCitem{Entradas}{\imprimeUC{entrada}}

	\UCitems{Origen}{%
        \Titem Mouse
	}

	\UCitem{Salidas}{\imprimeUC{salida}}

	\UCitem{Destino}{%
		\refElem{IU-E06a}
	}
	
	\UCitems{Precondiciones}{%
        \Titem Los plugins correspondientes al módulo de experiencia deben de estar
               habilitados.
	}

	\UCitem{Postcondiciones}{%
        \Titem Se debe de brindar la experiencia correspondiente de las secciones que
               hayan completado los alumnos inscritos en el curso.
	}

	\UCitem{Reglas de negocio}{\imprimeUC{regla}}

	\UCitems{Errores}{%
        \Titem \UCerr{Err1}{%
        % CAUSA
            Los plugins corespondientes al módulo de experiencia no se encuentran
            instalados,}{%
        % EFECTO
            no aparece la opción para cambiar el formato del curso al formato gamificado}

        \Titem \UCerr{Err2}{%
        % CAUSA
            El módulo de experiencia se encuentra deshabilitado en la plataforma,}{%
        % EFECTO
            se le solicita al \refElem{aAdministrador} que habilite la experiencia}
    }

	% \UCitem{Viene de}{% Indicar si el Caso de uso es primario o se extiende de otro. La mayoría se 
					  % extienden de Login.
		% EJEMPLO: \refIdElem{PY-CU1} o Caso de uso primario.
	% 	\TODO Especificar.
	% }	

 \UCsection[design]{Datos de Diseño}

	\UCitems[design]{Casos de Prueba}{%
        \Titem \refElem{CPC-E05}
        \Titem \refElem{CPC-E05a}
	}

 \UCsection[admin]{Datos de Administración de Requerimiento}

	\UCitem[admin]{Observaciones}{%
        Ninguna
	}

\end{UseCase}

\clearpage
\subsubsection{Trayectorias del caso de uso}

\begin{UCtrayectoria}%
%
 \includeUC{CU-M01} presionando la pestaña {\bf Cursos}.

  \Actor Presiona la opción {\bf Gestionar cursos y categorías}.
  \Sistema Obtiene las \refElem[categorias]{mdl-course-category} y 
           \refElem{mdl-course.fullname} de los cursos presentes en la plataforma.
  \Sistema Muestra en la pantalla \refElem{IU-M06} la lista de cursos. \refTray{A}.
           \label{CU-E05-course-list}

  \Actor Presiona el botón  

\end{UCtrayectoria}

\begin{UCtrayectoriaA}{A}{%
El curso al cual el \refElem{aProfesor} desea brindar soporte para experiencia 
se encuentra en una categoria distinta a la mostrada por defecto.
}
  \Actor Selecciona la \refElem{course.category} a la que pertenece el curso
         que desea agregarle experiencia.

  \Sistema Obtiene las \refElem[categorias]{mdl-course-category} y 
           \refElem{mdl-course.fullname} de los cursos presentes en la plataforma.

  \Sistema Redirige al paso \ref{CU-E05-course-list} de la trayectoria principal.

\end{UCtrayectoriaA}

   % Convertir un curso ordinario a uno gamificado
    
% \ucstEnEdicion     Al terminar una revisión/aprobación con observaciones 
%                    y al inicio del CU.
%
% \ucstEnRevision    Al terminar la edición del CU (version += 0.1).
% \ucstEnAprobacion  Al pasar la revision sin observaciones.
% \ucstAprobado      Al ser aprobado por el usuario (version += 1.0)

\begin{UseCase}[%
Autor/Daniel Ortega,%
Version/0.1,%
Estado/\ucstEnEdicion]%
%
{CU-E06}{Quitar el soporte de brindar experiencia a un curso gamificado}{%
%
 Permite al \refElem{aProfesor} quitar el soporte para brindar experiencia a un
 curso gamificado, se recomienda que para que no cambie la organización del curso
 se ocupe el formato de tópicos/temas debido a que en este formato esta basado el
 formato gamificado.}

	\UCitem[control]{Revisor}{ Sin asignar }
	\UCitem[control]{Último cambio}{ \today }

 \UCsection{Atributos}

    \UCitem{Actor(es)}{%
        \refElem{aProfesor}
    }

	\UCitem{Propósito}{%
        Permitir al profesor remover el soporte brindar puntos de experiencia de
        un curso que previamente gamificado.
	}
	
	\UCitem{Entradas}{\imprimeUC{entrada}}

	\UCitems{Origen}{%
        \Titem Mouse
	}

	\UCitem{Salidas}{\imprimeUC{salida}}

	\UCitem{Destino}{%
		\refElem{IU-M02a}
	}
	
	\UCitems{Precondiciones}{%
        \Titem El curso al que se le debe quitar el soporte de experiencia debe tener
               el formato gamificado.
        \Titem Los plugins del módulo de experiencia deben de estar instalados.
	}

	\UCitem{Postcondiciones}{%
        Ninguna
	}

	\UCitem{Reglas de negocio}{\imprimeUC{regla}}

	\UCitems{Errores}{%
        \Titem \UCerr{Err1}{%
        % CAUSA
            El curso elegido no es un curso gamificado con soporte para experiencia,}{%
        % EFECTO
            termina el caso de uso}
	}

	% \UCitem{Viene de}{% Indicar si el Caso de uso es primario o se extiende de otro. La mayoría se 
					  % extienden de Login.
		% EJEMPLO: \refIdElem{PY-CU1} o Caso de uso primario.
	% 	\TODO Especificar.
	% }	

 \UCsection[design]{Datos de Diseño}

	\UCitems[design]{Casos de Prueba}{%
        \Titem \refElem{CPC-E0Y}
        \Titem \refElem{CPI-E0Y}
	}

 \UCsection[admin]{Datos de Administración de Requerimiento}

	\UCitem[admin]{Observaciones}{%
        Ninguna
	}

\end{UseCase}

\clearpage
\subsubsection{Trayectorias del caso de uso}

\begin{UCtrayectoria}%
%
 \includeUC{CU-M01} presionando la pestaña {\bf Cursos}.

  \Actor Presiona la opción {\bf Gestionar cursos y categorías}.
  \Sistema Obtiene las \refElem[categorias]{mdl-course-category} y 
           \refElem{mdl-course.fullname} de los cursos presentes en la plataforma.
  \Sistema Muestra en la pantalla \refElem{IU-M06} la lista de cursos.
           \label{CU-E05-course-list}

  \Actor Presiona el botón \IUConfigurar del curso que desea editar.
  \Sistema Obtiene el \salida{mdl-course.fullname}, \salida{mdl-course.shortname} y
           \salida{mdl-course.format}.
  \Sistema Obtiene el valor de \salida{xp-course.hiddensections} y
           \salida{xp-course.coursedisplay}. \refErr{Err1}
  \Sistema Obtiene los demás datos del curso.
  \Sistema Muestra los datos obtenidos del curso en la pantalla \refElem{IU-E05}.

  \Actor Despliega la sección para el formato del curso.
  \Actor Selecciona UN \entrada{mdl-course.format} distinto al 
         \refElem[formato gamificado]{xp-course.format}.
  \Sistema Carga la información correspondiente al formato del curso elegido.

  \Actor Opcionalmente edita los demás campos opcionales del formulario.
  \Actor Presiona el botón {\bf Guardar Cambios y mostrar}. \refTray{B}, \refTray{C},
         \refTray{D} \label{CU-E05-submit}
  \Sistema Actualiza los datos del curso de moodle con los ingresados por el usuario.
  \Sistema Elimina el \entrada{xp-course} vinculado al \refElem{mdl-course} y las
           configuraciones del mismo.
  \Sistema Elimina las \entrada{xp-course-section} vinculadas al curso.
  \Sistema Elimina las \entrada[recompensas]{xp-section-reward} de las secciones del 
           curso sin alterar la experiencia recibida por los alumnos de acuerdo con 
           la regla \regla{BR-E02}.
  \Sistema Muestra la pantalla \refElem{IU-M06a}.

\end{UCtrayectoria}


\subsubsection{Puntos de extensión}

\UCExtensionPoint{Nombre del punto de extensión}{%

    El \refElem{aAdministrador} desea/requiere/necesita ....%
%
    }{En el paso \ref{CU-ET-1x} de la trayectoria principal  ...%
%
    }{\refElem{CU-E2-T}}

   % Convertir Curso gamificado -> curso ordinario
    
% \ucstEnEdicion     Al terminar una revisión/aprobación con observaciones 
%                    y al inicio del CU.
%
% \ucstEnRevision    Al terminar la edición del CU (version += 0.1).
% \ucstEnAprobacion  Al pasar la revision sin observaciones.
% \ucstAprobado      Al ser aprobado por el usuario (version += 1.0)

\begin{UseCase}[%
Autor/Daniel Ortega,%
Version/0.1,%
Estado/\ucstEnEdicion]%
%
{CU-E07}{Administrar experiencia de un curso}{%
%
 Permite al \refElem{aProfesor} establecer la cantidad de experiencia que cada una de las
 secciones de un curso gamificado brindará a los alunos cuando estos la hayan completado}

	\UCitem[control]{Revisor}{ Sin asignar }
	\UCitem[control]{Último cambio}{ \today }

 \UCsection{Atributos}

    \UCitem{Actor(es)}{%
        \refElem{aProfesor}
    }

	\UCitems{Propósito}{%
        \Titem Permitirle al profesor especificar la cantidad de experiencia que cada
               sección del curso brindará.
        \Titem Permitirle al profesor ponderar de acuerdo a su criterio que secciones
               del curso deben brindar mayor o menor experiencia.
	}
	
	\UCitem{Entradas}{\imprimeUC{entrada}}

	\UCitems{Origen}{%
        \Titem Mouse
        \Titem Teclado
	}

	\UCitem{Salidas}{\imprimeUC{salida}}

	\UCitem{Destino}{%
		\refElem{IU-E06b}
	}
	
	\UCitems{Precondiciones}{%
        \Titem Los plugins del módulo de experiencia deben de estar habilitados
        \Titem El curso debe debe de tener el \refElem{xp-course.format} gamificado.
	}

	\UCitem{Postcondiciones}{%
        Los valores de experiencia para cada sección del curso deben ser
        almacenados en el sistema.
	}

	\UCitem{Reglas de negocio}{\imprimeUC{regla}}

	\UCitems{Errores}{%
        \Titem \UCerr{Err1}{%
        % CAUSA
            El curso elegido no es un curso gamificado con soporte para experiencia,}{%
        % EFECTO
            no se puede administrar la experiencia y termina el caso de uso}
	}

	% \UCitem{Viene de}{% Indicar si el Caso de uso es primario o se extiende de otro. La mayoría se 
					  % extienden de Login.
		% EJEMPLO: \refIdElem{PY-CU1} o Caso de uso primario.
	% 	\TODO Especificar.
	% }	

 \UCsection[design]{Datos de Diseño}

	\UCitems[design]{Casos de Prueba}{%
        \Titem \refElem{CPC-E07}
        \Titem \refElem{CPC-E07a}
        \Titem \refElem{CPI-E07}
	}

 \UCsection[admin]{Datos de Administración de Requerimiento}

	\UCitem[admin]{Observaciones}{%
        Ninguna
	}

\end{UseCase}

\subsubsection{Trayectorias del caso de uso}

\begin{UCtrayectoria}%
%
  \Actor Presiona el botón \IUMenu de la pantalla \refElem{IU-M00}
  \Sistema Despliega el menú de navegación lateral
  \Actor Selecciona la opción \IUHome {\bf Página Inicial del Sitio}

  \Sistema Obtiene el \salida{mdl-course.fullname} de los cursos disponibles en la
           plataforma.

  \Sistema Muestra la lista de cursos disponibles en la pantalla \refElem{IU-M07}.

  \Actor Selecciona el \entrada{mdl-course} del cual desea administrar su experiencia.
  \Sistema Obtiene el \entrada{mdl-course.fullname}, \entrada{mdl-course.shortname}
           así como el \salida[secciones]{mdl-course-section.name} de las secciones 
           del curso junto con las \salida[actividades]{mdl-course-module} 
           correspondients a cada sección y el estado de \refElem[completitud]%
           {mdl-course-module.completionstate}.

  \Sistema Muestra los datos obtenidos en la pantalla \refElem{IU-M07}.
           \label{CU-E07-pantalla}

  \Actor Presiona el botón \IUAdminSitio en la parte superior izquierda de la pantalla
  \Sistema Muestra el menu desplegable de la administración del curso

  \Actor Presiona el botón \IUEditar {\bf Activar Edición}. \refErr{Err1}
  \Sistema Obtiene la \salida{xp-course-section.xp} de la secciones gamificadas del 
           curso y revisa hay \refElem[recompensas]{xp-section-reward} que hayan sido
           entregadas correspondientes a dicha sección.
  \Sistema Obtiene la cantidad total de \salida{xp-scheme-settings.courseXP}.
  \Sistema Muestra la experiencia de cada sección y el total de experiencia 
           en la pantalla \refElem{IU-E06b}.
  \Sistema Habilita o deshabilita los campos para editar la experiencia de cada 
           sección con base en la regla \regla{BR-E09}.

  \Actor Introduce la experiencia de las secciones de acuerdo con la regla
         \regla{BR-E10}. \refTray{A}
  \Actor Presiona el botón {\bf Guardar cambios}. \refTray{B} \label{CU-E07-submit}
  \Sistema Obtiene los valores de experiencia ingresados por el usuario
           correspondientes a las secciones del curso.
  \Sistema Valida que los valores de experiencia ingresados cumplan con las
           restricciones especificadas en el modelo de información. \refTray{C}
  \Sistema Valida que los valores de experiencia cumplan con la regla
           \refElem{BR-E10} \refTray{D}.
  \Sistema Actualiza los valores de experiencia correspondientes a cada sección.
           \label{CU-E07-finish}
  \Sistema Muestra la pantalla \refElem{IU-E06b} con la \refElem{xp-course-section.xp}
           actualizada de las secciones del curso.
           

\end{UCtrayectoria}

\begin{UCtrayectoriaA}{A}{%
Alguna de las secciones gamificadas ha sido completada por almenos un alumno.
}
  \Sistema Detecta que todas las \refElem{mdl-course-module} de una sección
           han sido completadas por al menos un \refElem{aAlumno}.
  \Sistema Deshabilita los campos de edición en todas las secciones que entre
           en el anterior supuesto.
  \Sistema Continua en el paso \ref{CU-E07-pantalla}.
\end{UCtrayectoriaA}

\begin{UCtrayectoriaA}{B}{%
El \refElem{aProfesor} desea distribuir uniformemente la experiencia del curso
disponible entre las distintas secciones de acuerdo con la regla.}

  \Actor Presiona el botón {\bf Distribuir uniformemente}
  \Sistema Divide la cantidad de experiencia entre las secciones editables
           de acuerdo con las reglas \refElem{BR-E10} y \regla{BR-E08}.
  \Sistema Continua en el paso \ref{CU-E07-finish} de la trayectoria principal.
\end{UCtrayectoriaA}

\begin{UCtrayectoriaA}{C}{%
Alguno de los campos para establecer la experiencia de las %
\refElem[secciones del curso]{xp-course-section} es incorrecto.}

  \Sistema Muestra el mensaje de error debajo de los campos con un valor de
           experiencia incorrecto.

  \Actor Ingresa de nuevo los valores de experiencia de los campos de experiencia
         incrrectos.
  \Sistema Regresa la paso \ref{CU-E07-submit} de la trayectoria principal
\end{UCtrayectoriaA}

\begin{UCtrayectoriaA}{D}{%
Los valores de experiencia introducidos por el \refElem{aProfesor} no cumplen con
la regla \refElem{BR-E09}.}

  \Sistema Muestra el mensaje de error de que la suma de la experiencia no es igual
           al total de experiencia del curso.
  \Actor Ingresa de nuevo los valores de experiencia correspondientes a cada sección
         del curso.
  \Sistema Regresa la paso \ref{CU-E07-submit} de la trayectoria principal
\end{UCtrayectoriaA}
   % Administrar Curso gamificado
    
% \ucstEnEdicion     Al terminar una revisión/aprobación con observaciones
%                    y al inicio del CU.
%
% \ucstEnRevision    Al terminar la edición del CU (version += 0.1).
% \ucstEnAprobacion  Al pasar la revision sin observaciones.
% \ucstAprobado      Al ser aprobado por el usuario (version += 1.0)

\begin{UseCase}[%
Autor/Daniel Ortega,%
Version/0.1,%
Estado/\ucstEnEdicion]%
%
{CU-E08}{Agregar sección con experiencia}{%
%
 Permite al \refElem{aProfesor} agregar una sección con soporte para experiencia
 en un curso gamificado y que pueda configurar la cantidad de experiencia que brinda
 dicha sección mediante el caso de uso \refElem{CU-E07}.}

	\UCitem[control]{Revisor}{ Sin asignar }
	\UCitem[control]{Último cambio}{ \today }

 \UCsection{Atributos}

    \UCitem{Actor(es)}{%
        \refElem{aProfesor}
    }

	\UCitems{Propósito}{%
        \Titem Permitirle al profesor agregar una sección con soporte para experiencia
               en curso gamificado.
        \Titem Permitirle al profesor tener la misma funcionalidad de agregar nuevas
               secciones en como en un curso normal en moodle.
	}

	\UCitem{Entradas}{\imprimeUC{entrada}}

	\UCitems{Origen}{%
        \Titem Mouse
	}

	\UCitem{Salidas}{\imprimeUC{salida}}

	\UCitem{Destino}{%
		\refElem{IU-E06b}
	}

	\UCitems{Precondiciones}{%
        \Titem El módulo de experiencia debe estar habilitado.
        \Titem El curso debe debe de tener el \refElem{xp-course.format} gamificado.
	}

	\UCitem{Postcondiciones}{%
        Los valores de experiencia para cada sección del curso deben ser
        almacenados en el sistema.
	}

	\UCitem{Reglas de negocio}{\imprimeUC{regla}}

	\UCitems{Errores}{%
        \Titem \UCerr{Err1}{%
        % CAUSA
            El curso elegido no es un curso gamificado con soporte para experiencia,}{%
        % EFECTO
            no se puede crear una sección con experiencia y termina el caso de uso}
	}

	% \UCitem{Viene de}{% Indicar si el Caso de uso es primario o se extiende de otro. La mayoría se
					  % extienden de Login.
		% EJEMPLO: \refIdElem{PY-CU1} o Caso de uso primario.
	% 	\TODO Especificar.
	% }

 \UCsection[design]{Datos de Diseño}

	\UCitems[design]{Casos de Prueba}{%
        \Titem \refElem{CPC-E08}
	}

 \UCsection[admin]{Datos de Administración de Requerimiento}

	\UCitem[admin]{Observaciones}{%
        Ninguna
	}

\end{UseCase}

\subsubsection{Trayectorias del caso de uso}

\begin{UCtrayectoria}%
%
  \Actor Presiona el botón \IUMenu de la pantalla \refElem{IU-M00}
  \Sistema Despliega el menú de navegación lateral
  \Actor Selecciona la opción \IUHome {\bf Página Inicial del Sitio}

  \Sistema Obtiene el \salida{mdl-course.fullname} de los cursos disponibles en la
           plataforma.

  \Sistema Muestra la lista de cursos disponibles en la pantalla \refElem{IU-M07}.

  \Actor Selecciona el \entrada{mdl-course} al cual desea agregar una sección con
         soporte de gamificación.
  \Sistema Obtiene el \entrada{mdl-course.fullname}, \entrada{mdl-course.shortname}
           así como el \salida[secciones]{mdl-course-section.name} de las secciones
           del curso junto con las \salida[actividades]{mdl-course-module}
           correspondientes a cada sección y el estado de \refElem[completitud]%
           {mdl-course-module-completion.completionstate}.

  \Sistema Muestra los datos obtenidos en la pantalla \refElem{IU-M07}.
           \label{CU-E07-pantalla}

  \Actor Presiona el botón \IUAdminSitio en la parte superior izquierda de la pantalla
  \Sistema Muestra el menú desplegable de la administración del curso

  \Actor Presiona el botón \IUEditar {\bf Activar Edición}. \refErr{Err1}
  \Sistema Obtiene la \salida{xp-course-section.xp} de la secciones gamificadas del
           curso y revisa hay \refElem[recompensas]{xp-section-reward} que hayan sido
           entregadas correspondientes a dicha sección.
  \Sistema Obtiene la cantidad total de \salida{xp-scheme-settings.courseXP}.
  \Sistema Muestra la experiencia de cada sección y el total de experiencia
           en la pantalla \refElem{IU-E06b}.
  \Sistema Habilita o deshabilita los campos para editar la experiencia de cada
           sección con base en la regla \regla{BR-E09}.

  \Actor Presiona el botón \IUAnadir {\bf Misión}.
  \Sistema Crea una sección \entrada{mdl-course-section} perteneciente al
           \refElem{mdl-course}.
  \Sistema Crea una sección \entrada{xp-course-section} asociada a la sección
           recientemente creada, indicando que dicha sección tiene cero puntos de
           \refElem{xp-course-section.xp}.
  \Sistema Muestra la pantalla \refElem{IU-E06b} con los datos de la nueva sección creada.
\end{UCtrayectoria}
   % Agregar seccion con experiencia
    %
% \ucstEnEdicion     Al terminar una revisión/aprobación con observaciones 
%                    y al inicio del CU.
%
% \ucstEnRevision    Al terminar la edición del CU (version += 0.1).
% \ucstEnAprobacion  Al pasar la revision sin observaciones.
% \ucstAprobado      Al ser aprobado por el usuario (version += 1.0)

\begin{UseCase}[%
Autor/Daniel Ortega,%
Version/0.1,%
Estado/\ucstEnEdicion]%
%
{CU-E09}{Eliminar sección con experiencia}{%
%
 Permite al \refElem{aProfesor} eliminar una seccion de un curso gamificado con soporte
 para experiencia con base en la regla \refElem{BR-E11}, la cual indica para eliminar una
 sección de un curso gamificado esta debe brindar 0 puntos de experiencia.}

	\UCitem[control]{Revisor}{ Sin asignar }
	\UCitem[control]{Último cambio}{ \today }

 \UCsection{Atributos}

    \UCitem{Actor(es)}{%
        \refElem{aProfesor}
    }

	\UCitems{Propósito}{%
        \Titem Permitirle al profesor eliminar una sección de un curso con soporte 
               para experiencia.
        \Titem Permitirle al profesor eliminar secciones del curso de manera controlada
               sin afectar la cantidad de experiencia total que brinda el curso.
	}
	
	\UCitem{Entradas}{\imprimeUC{entrada}}

	\UCitems{Origen}{%
        \Titem Mouse
	}

	\UCitem{Salidas}{\imprimeUC{salida}}

	\UCitem{Destino}{%
		\refElem{IU-E06b}
	}
	
	\UCitems{Precondiciones}{%
        \Titem El curso debe debe de tener el \refElem{xp-course.format} gamificado.
	}

	\UCitem{Postcondiciones}{%
        Las eliminación de las secciones de experiencia debe permanecer en el sistema.
	}

	\UCitem{Reglas de negocio}{\imprimeUC{regla}}

	\UCitems{Errores}{%
        \Titem \UCerr{Err1}{%
        % CAUSA
            El curso elegido no es un curso gamificado con soporte para experiencia,}{%
        % EFECTO
            termina el caso de uso.}
        \Titem \UCerr{Err2}{%
        % CAUSE
            La sección que se desea eliminar ya ha sido completada por alguno
            de los \refElem{aAlumno},}{%
        % EFECTO
            la opción de eliminar sección no se muestra, termina el caso de uso.}
	}

	% \UCitem{Viene de}{% Indicar si el Caso de uso es primario o se extiende de otro. La mayoría se 
					  % extienden de Login.
		% EJEMPLO: \refIdElem{PY-CU1} o Caso de uso primario.
	% 	\TODO Especificar.
	% }	

 \UCsection[design]{Datos de Diseño}

	\UCitems[design]{Casos de Prueba}{%
        \Titem \refElem{CPC-E08}
        \Titem \refElem{CPI-E08}
	}

 \UCsection[admin]{Datos de Administración de Requerimiento}

	\UCitem[admin]{Observaciones}{%
        Ninguna
	}

\end{UseCase}

\subsubsection{Trayectorias del caso de uso}

\begin{UCtrayectoria}%
%
  \Actor Presiona el botón \IUMenu de la pantalla \refElem{IU-M00}
  \Sistema Despliega el menú de navegación lateral
  \Actor Selecciona la opción \IUHome {\bf Página Inicial del Sitio}

  \Sistema Obtiene el \salida{mdl-course.fullname} de los cursos disponibles en la
           plataforma.

  \Sistema Muestra la lista de cursos disponibles en la pantalla \refElem{IU-M07}.

  \Actor Selecciona el \entrada{mdl-course} al cual desea agregar una sección con
         soporte de gamificación.
  \Sistema Obtiene el \entrada{mdl-course.fullname}, \entrada{mdl-course.shortname}
           así como el \salida[secciones]{mdl-course-section.name} de las secciones 
           del curso junto con las \salida[actividades]{mdl-course-module} 
           correspondients a cada sección y el estado de \refElem[completitud]%
           {mdl-course-module.completionstate}.

  \Sistema Muestra los datos obtenidos en la pantalla \refElem{IU-M07}.
           \label{CU-E07-pantalla}

  \Actor Presiona el botón \IUAdminSitio en la parte superior izquierda de la pantalla
  \Sistema Muestra el menu desplegable de la administración del curso

  \Actor Presiona el botón \IUEditar {\bf Activar Edición}. \refErr{Err1}
  \Sistema Obtiene la \salida{xp-course-section.xp} de la secciones gamificadas del 
           curso y revisa hay \refElem[recompensas]{xp-section-reward} que hayan sido
           entregadas correspondientes a dicha sección.
  \Sistema Obtiene la cantidad total de \salida{xp-scheme-settings.courseXP}.
  \Sistema Muestra la experiencia de cada sección y el total de experiencia 
           en la pantalla \refElem{IU-E06b}.
  \Sistema Habilita o deshabilita los campos para editar la experiencia de cada 
           sección con base en la regla \regla{BR-E09}.

  \Actor Presiona la opción {\bf Editar} correspondiente a la \entrada{xp-course-section}
         seccion que desea eliminar. \label{CU-E09-options}.

  \Sistema Muestra el menu acciones de edición que se pueden realizar a dicha sección de
           acuerdo con la regla \regla{BR-E11}. \refTray{A} \refErr{Err2}

  \Actor Selecciona la opción \IUEliminar {\bf Eliminar sección}.

  \Sistema Elimina la \refElem{mdl-course-section} especificada, así como la
           \refElem{xp-course-section}.
  \Sistema Muestra la pantalla \refElem{IU-E06b} con los datos de las secciones restantes
           del curso.

\end{UCtrayectoria}

\begin{UCtrayectoriaA}{A}{%
La cantidad de \refElem{xp-course-section.xp} de la sección a eliminar es distinta de cero.
}
  \Sistema Redirige al paso \ref{CU-E09-options} de la trayectoria principal.
\end{UCtrayectoriaA}

   % Eliminar seccion con experiencia
    
% \ucstEnEdicion     Al terminar una revisión/aprobación con observaciones
%                    y al inicio del CU.
%
% \ucstEnRevision    Al terminar la edición del CU (version += 0.1).
% \ucstEnAprobacion  Al pasar la revision sin observaciones.
% \ucstAprobado      Al ser aprobado por el usuario (version += 1.0)

\begin{UseCase}[%
Autor/Daniel Ortega,%
Version/0.1,%
Estado/\ucstEnEdicion]%
%
{CU-E10}{Eliminar un curso gamificado}{%
%
 Permite al \refElem{aAdministrador} eliminar un curso gamificado. }

	\UCitem[control]{Revisor}{ Sin asignar }
	\UCitem[control]{Último cambio}{ \today }

 \UCsection{Atributos}

    \UCitem{Actor(es)}{%
        \refElem{aAdministrador}
    }

	\UCitem{Propósito}{%
        Permitir al administrador eliminar un curso que está gamificado junto
        con los datos de gamificación que este ha otorgado.
	}

	\UCitem{Entradas}{\imprimeUC{entrada}}

	\UCitems{Origen}{%
        \Titem Mouse
	}

	\UCitem{Salidas}{\imprimeUC{salida}}

	\UCitem{Destino}{%
		\refElem{IU-M02a}
	}

	\UCitems{Precondiciones}{%
        \Titem El curso que se desea eliminar no debe haber sido eliminado
               anteriormente.
	}

	\UCitem{Postcondiciones}{%
        Ninguna
	}

	\UCitem{Reglas de negocio}{\imprimeUC{regla}}

	\UCitems{Errores}{%
        \Titem \UCerr{Err1}{%
        % CAUSA
            El curso que se desea eliminar ha sido eliminado previamente,}{%
        % EFECTO
            termina el caso de uso.}
	}

	% \UCitem{Viene de}{% Indicar si el Caso de uso es primario o se extiende de otro. La mayoría se
					  % extienden de Login.
		% EJEMPLO: \refIdElem{PY-CU1} o Caso de uso primario.
	% 	\TODO Especificar.
	% }

 \UCsection[design]{Datos de Diseño}

	\UCitems[design]{Casos de Prueba}{%
        \Titem \refElem{CPC-E05}
        \Titem \refElem{CPC-E05a}
	}

 \UCsection[admin]{Datos de Administración de Requerimiento}

	\UCitem[admin]{Observaciones}{%
        Ninguna
	}

\end{UseCase}

\clearpage
\subsubsection{Trayectorias del caso de uso}

\begin{UCtrayectoria}%
%
  \Actor Presiona la opción {\bf Gestionar cursos y categorías}.
  \Sistema Obtiene las \salida[categorias]{mdl-course-category} y
           \salida{mdl-course.fullname} de los cursos presentes en la plataforma.
  \Sistema Muestra la pantalla \refElem{IU-M06}. \refErr{Err1}

  \Actor Presiona el botón \IUEliminar correspondiente al curso que desea eliminar.
         \label{CU-E05-delete-button}. \refTray{A}
  \Sistema Despliega la pantalla \refElem{IU-M06a} pidiendo la confirmación del
           usuario para continuar con la eliminación.

  \Actor Presiona la opción {\bf Eliminar}. \refTray{B}
  \Sistema Elimina las \entrada{xp-section-reward} de experiencia que se han
           entregado respetando la regla \regla{BR-E02}.
  \Sistema Elimina las \entrada[secciones gamificadas]{xp-course-section} del curso.
  \Sistema Elimina las opciones del \entrada{xp-course.format}.
  \Sistema Procede con las demás tareas de eliminación para concluir con la
           eliminación del curso.
  \Sistema Muestra la pantalla \refElem{IU-M06b}.

  \Actor Presiona el botón {\bf Continuar}.
  \Sistema Obtiene las \refElem[categorias]{mdl-course-category} y
           \refElem{mdl-course.fullname} de los cursos restantes en la plataforma.
  \Sistema Muestra la pantalla \refElem{IU-M06}.
\end{UCtrayectoria}

\begin{UCtrayectoriaA}{A}{%
El \refElem{mdl-course} que el \refElem{aAdministrador} desea eliminar se encuentra
en otra categoría diferente a la seleccionada por defecto.}

  \Actor Selecciona la \refElem{mdl-course-category} donde se encuentra el curso
         que desea eliminar.

  \Sistema Obtiene las \refElem{mdl-course.fullname} de los cursos pertenecientes a
           dichas categorías.
  \Sistema Regresa al paso \ref{CU-E05-delete-button}.
\end{UCtrayectoriaA}

\begin{UCtrayectoriaA}{B}{%
El \refElem{aAdministrador} desea cancelar la eliminación del curso.}

  \Actor Presiona el botón {\bf Cancelar}
  \Sistema Obtiene las \refElem[categorias]{mdl-course-category} y
           \refElem{mdl-course.fullname} de los cursos presentes en la plataforma.
\end{UCtrayectoriaA}
   % Eliminar un curso gamificado
    %\input{modulos/exp/CU/CU-E11}   % Recibir Experiencia                                  %RICARDO: Por qué está comentado este caso de uso?
    
% \ucstEnEdicion     Al terminar una revisión/aprobación con observaciones
%                    y al inicio del CU.
%
% \ucstEnRevision    Al terminar la edición del CU (version += 0.1).
% \ucstEnAprobacion  Al pasar la revision sin observaciones.
% \ucstAprobado      Al ser aprobado por el usuario (version += 1.0)

\begin{UseCase}[%
Autor/Daniel Ortega,%
Version/0.1,%
Estado/\ucstEnEdicion]%
%
{CU-E12}{Crear cuenta de usuario gamificado}{%
%
 Permite al módulo de experiencia recibir el evento, emitido por moodle, de cuando
 un usuario es creado, con base en este evento el módulo de experiencia asignará los
 datos por defecto de un usuario gamificado al usuario que se acaba de crear.}

	\UCitem[control]{Revisor}{ Sin asignar }
	\UCitem[control]{Último cambio}{ \today }

 \UCsection{Atributos}

    \UCitem{Actor(es)}{%
        \refElem{aAdministrador} o \refElem{aEstudiante}.
    }

	\UCitem{Propósito}{%
        Asociar una cuenta de un \refElem{xp-user} a un \refElem{mdl-user}
        cuando es creado por el administrador.
	}

	\UCitem{Entradas}{\imprimeUC{entrada}}

	\UCitems{Origen}{%
        \Titem Teclado
        \Titem Mouse
    }

	\UCitem{Salidas}{\imprimeUC{salida}}

	\UCitem{Destino}{\refElem{IU-M05a}}

	\UCitems{Precondiciones}{%
        \Titem Los plugins del módulo de experiencia deben de estar instalados.
        \Titem Cuando el actor es un \refElem{aEstudiante} \refTray{A}, la opción
               de permitir autoregistros debe estar habilitada y las
               configuraciones para poder realizar envío de correos mediante SMTP
               deben ser las correctas.
	}

	\UCitems{Postcondiciones}{%
        \Titem Los datos de un usuario gamificado deben permanecer vinculados con
               el usuario de moodle que se creó.
	}

	\UCitem{Reglas de negocio}{\imprimeUC{regla}}

	\UCitems{Errores}{%
        \Titem \UCerr{Err1}{%
        % CAUSA
            Los plugins del módulo de experiencia no se encuentran instalados y}{%
        % EFECTO
            no se puede crear un usuario gamificado, termina el caso de uso}
        \Titem \UCerr{Err2}{%
        % CAUSA
            La opción de creación de cuentas mediante autoregistro se encuentra
            deshabilitada}{%
        % EFECTO
            no se muestra la opción para que el estudiante cree su propia contraseña,
            termina el caso de uso.}
        \Titem \UCerr{Err3}{%
        % CAUSA
            La configuración de envío de correos para autoregistrarse no permite
            enviar correos}{%
        % EFECTO
            no se puede enviar el correo de confirmación, pero la cuenta ha sido
            creada, el \refElem{aEstudiante} contacta con soporte para que validen su
            cuenta manualmente}
	}

	% \UCitem{Viene de}{% Indicar si el Caso de uso es primario o se extiende de
    % otro. La mayoría se extienden de Login.
		% EJEMPLO: \refIdElem{PY-CU1} o Caso de uso primario.
	% 	\TODO Especificar.
	% }

 \UCsection[design]{Datos de Diseño}

	\UCitems[design]{Casos de Prueba}{%
        \Titem \refElem{CPC-E12}
        \Titem \refElem{CPC-E12a}
	}

 \UCsection[admin]{Datos de Administración de Requerimiento}

	\UCitem[admin]{Observaciones}{%
        Ninguna
	}

\end{UseCase}

\subsubsection{Trayectorias del caso de uso}

\begin{UCtrayectoria}%
%
  \includeUC{CU-M01} presionando la pestaña {\bf Usuarios}. \refTray{A}
  \Actor Presiona la opción {\bf Agregar un usuario} de la pantalla \refElem{IU-M01b}.
  \Sistema Carga la pantalla \refElem{IU-M05}.
  \Actor Ingresa el \entrada{mdl-user.username}, \entrada{mdl-user.password},
         \entrada{mdl-user.firstname} y \entrada{mdl-user.lastname}, que son los
         atributos requeridos para el nuevo usuario. \label{IU-E12-input-data}

  \Actor Opcionalmente ingresa los valores de los demás campos.

  \Actor Presiona el botón de {\bf Crear Usuario}. \refTray{B} \refTray{C}
         \label{CU-E12-submit}.

  \Sistema Obtiene lo valores ingresados para el nuevo usuario y crea la cuenta de un
           \refElem{mdl-user}.

  \Sistema Vincula los datos de un \salida{xp-user} con el \refElem{mdl-user} obtenido,
           estableciendo los valores iniciales para el \refElem{xp-user.level},
           \refElem{xp-user.levelxp} y \refElem{xp-user.xp} de acuerdo con la regla
           \regla{BR-E07}. \refErr{Err1} \label{IU-E12}

  \Sistema Obtiene el \salida{mdl-user.firstname}, \salida{mdl-user.lastname},
           \salida{mdl-user.email}, \salida{mdl-user.lastaccess},
           \salida{mdl-user.city} y \salida{mdl-user.country} de los usuarios
           presentes en moodle.

  \Sistema Despliega la información de los usuarios en la pantalla \refElem{IU-M05a}.
\end{UCtrayectoria}


\begin{UCtrayectoriaA}{A}{%
El usuario que ejecuta este caso de uso es el \refElem{aEstudiante} para registrarse
así mismo.
}
  \Actor Se encuentra en la pantalla \refElem{IU-M00b}. \refErr{Err2}
  \Actor Presiona el botón {\bf Comience ahora creando una cuenta}.
  \Sistema Carga la pantalla \refElem{IU-M05c}.
  \Sistema Ejecuta del paso \ref{IU-E12-input-data} al
  \Sistema Envía el correo de confirmación al \refElem{aEstudiante}. \refErr{Err3}
  \Actor Revisa su correo y accede al enlace para confirmar el registro de su cuenta.
\end{UCtrayectoriaA}

\begin{UCtrayectoriaA}{B}{%
Algunos de los campos ingresados por el \refElem{aAdministrador} son erróneos
}
  \Sistema Imprime los mensaje de error debajo de los campos con valores incorrectos.
  \Actor Ingresa nuevamente los valores en los campos marcados como incorrectos.
  \Sistema Regresa al paso \ref{CU-E12-submit}.
\end{UCtrayectoriaA}


\begin{UCtrayectoriaA}{C}{%
El \refElem{aAdministrador} desea cancelar la creación de un nuevo usuario
}
  \Actor Presiona el botón {\bf Cancelar}.
  \Sistema Redirige a la pantalla \refElem{IU-M01b}
\end{UCtrayectoriaA}
   % Crear cuenta de un usuario gamificado
    
% \ucstEnEdicion     Al terminar una revisión/aprobación con observaciones
%                    y al inicio del CU.
%
% \ucstEnRevision    Al terminar la edición del CU (version += 0.1).
% \ucstEnAprobacion  Al pasar la revision sin observaciones.
% \ucstAprobado      Al ser aprobado por el usuario (version += 1.0)

\begin{UseCase}[%
Autor/Daniel Ortega,%
Version/0.1,%
Estado/\ucstEnEdicion]%
%
{CU-E13}{Eliminar usuario gamificado}{%
%
 Permite al \refElem{aAdministrador} que cuando se elimine a algún estudiante de moodle
 también se eliminen los datos correspondientes al módulo de experiencia de dicho
 usuario.}

	\UCitem[control]{Revisor}{ Sin asignar }
	\UCitem[control]{Último cambio}{ \today }

 \UCsection{Atributos}

    \UCitem{Actor(es)}{%
        \refElem{aAdministrador}
    }

	\UCitem{Propósito}{%
        Eliminar los datos de experiencia de un usuario gamificado cuando se vaya al
        usuario de moodle vinculado a este.
	}

	\UCitem{Entradas}{\imprimeUC{entrada}}

	\UCitems{Origen}{%
        \Titem Mouse
	}

	\UCitem{Salidas}{\imprimeUC{salida}}

	\UCitem{Destino}{%
		\refElem{IU-M05a}
	}

	\UCitems{Precondiciones}{%
        \Titem Los plugins correspondientes al módulo de experiencia deben de estar
               instalados.
        \Titem El \refElem{mdl-user} que se desea eliminar no debe haber sido
               eliminado previamente.
	}

	\UCitem{Postcondiciones}{%
        \Titem Los datos de experiencia vinculados al \refElem{xp-user} deben de ser
               removidos del sistema.
	}

	\UCitem{Reglas de negocio}{\imprimeUC{regla}}

	\UCitems{Errores}{%
        \Titem \UCerr{Err1}{%
        % CAUSA
            Los plugins correspondientes al módulo de experiencia no se encuentran
            instalados,}{%
        % EFECTO
            continúaen el penúltimo paso de la trayectoria principal}

        \Titem \UCerr{Err2}{%
        % CAUSA
            El usuario que se desea eliminar ha sido eliminado previamente,}{%
        % EFECTO
            no se muestra la entrada del estudiante en la pantalla \refElem{IU-M05a},
            termina el caso de uso}
	}

	% \UCitem{Viene de}{% Indicar si el Caso de uso es primario o se extiende de otro. La mayoría se
					  % extienden de Login.
		% EJEMPLO: \refIdElem{PY-CU1} o Caso de uso primario.
	% 	\TODO Especificar.
	% }

 \UCsection[design]{Datos de Diseño}

	\UCitems[design]{Casos de Prueba}{%
        \Titem \refElem{CPC-E13}
	}

 \UCsection[admin]{Datos de Administración de Requerimiento}

	\UCitem[admin]{Observaciones}{%
        Ninguna
	}

\end{UseCase}

\subsubsection{Trayectorias del caso de uso}

\begin{UCtrayectoria}%
%
  \includeUC{CU-M01} presionando la pestaña {\bf Usuarios}

  \Actor Presiona la opción {\bf Mirar lista de usuarios}
  \Sistema Obtiene el \salida{mdl-user.firstname}, \salida{mdl-user.lastname},
           \salida{mdl-user.email}, \salida{mdl-user.lastaccess},
           \salida{mdl-user.city} y \salida{mdl-user.country} de los usuarios
           presentes en moodle.

  \Sistema Despliega la información de los usuarios en la pantalla \refElem{IU-M05a}.

  \Actor Presiona el botón \IUEliminar correspondiente al usuario que desea eliminar.
         \refErr{Err2}.

  \Sistema Pide la confirmación para eliminar al usuario seleccionado mediante la
           pantalla \refElem{IU-M05b}.
  \Actor Presiona el botón {\bf Eliminar} \refTray{A}

  \Sistema Elimina los datos de experiencia obtenidos en los cursos gamificados en
           los que el usuario ha sido \entrada{xp-section-reward}.

  \Sistema Elimina los datos del \entrada{xp-user}.

  \Sistema Elimina al \entrada{mdl-user}.

  \Sistema Obtiene el \refElem{mdl-user.firstname}, \refElem{mdl-user.lastname},
           \refElem{mdl-user.email}, \refElem{mdl-user.lastaccess},
           \refElem{mdl-user.city} y \refElem{mdl-user.country} de los usuarios
           presentes en moodle.

  \Sistema Despliega la información de los usuarios en la pantalla \refElem{IU-M05a}.


\end{UCtrayectoria}

\begin{UCtrayectoriaA}{A}{%
El \refElem{aAdministrador} desea cancelar la eliminación del usuario seleccionado
}%

  \Actor Presiona el botón {\bf Cancelar}
  \Sistema Obtiene el \refElem{mdl-user.firstname}, \refElem{mdl-user.lastname},
           \refElem{mdl-user.email}, \refElem{mdl-user.lastaccess},
           \refElem{mdl-user.city} y \refElem{mdl-user.country} de los usuarios
           presentes en moodle.
  \Sistema Redirige a la pantalla \refElem{IU-M05a}.
\end{UCtrayectoriaA}
   % Eliminar cuenta de un usuario gamificado

% =========================================================
\clearpage
\subsection{Diseño}

 En esta sección se presenta los distintos diagramas que ayudan a modelar el
 software a desarrollar, de la misma forma todas las decisiones y consideraciones
 relevantes para la implementación de las funcionalidades se encuentra documentada
 en esta sección.

    
\subsection*{Interfaces de Moodle}

 Como primera instancia se presentan las tanto las interfaces utilizadas de moodle
 que son utilizadas a lo largo de los distintos casos de uso del módulo de
 experiencia. La especificación de las interfaces de moodle contiene una imagen de
 la interfaz y una descripción de los elementos y acciones más relevantes por cada
 interfaz.

    % INTERFACES DE MOODLE
    
\subsubsection{IU-M00 Pantalla principal}

 El tablero o {\it Dashboard} (ver figura~\ref{IU-M02}) es la primer página que ve un usuario de
 inmediatamente despues iniciar sesión, esta página muestra a los usuarios detalles de su progreso
 y fechas límite próximas \cite{MoodleTablero} . Los elementos que tiene esta página con las demás
 páginas del sitio de moodle es el menú de navegación a la izquierda y la columna derecha de
 bloques.

    \IUfig{1}{modulos/moodle/IU/Dashboard.png}{IU-M00}{Pantalla Principal de Moodle}

\subsubsection{Elementos relevantes}

    \begin{itemize}
    \item
    {\bf Menú Superior}
        Como su nombre lo indica se encuentra en la parte superior, este elemento se
        encuentra en la mayoría de las pantallas de moodle.

    \item
    {\bf Menú de Navegación}
        Cuando esta visible se encuentra en la parte izquierda de la parte izquierda
        de la mayoría de las pantallas de moodle. Se puede ocultar o mostrar con la
        acción \IUMenu[].

    \item
    {\bf Contenido}
        Tiene todos los demás elementos que conforman el contenido de la pantalla.

    \end{itemize}

\subsubsection{Acciones relevantes}

    \begin{itemize}
    
    \item
    {\bf \IUMenu{} (Desplegar el menú)}
        Si el menú está oculto, cuando el usuario presione el botón \IUMenu{} el menú de
        navegación se desplegará.

    \item {\bf \IUMenu{} (Ocultar Menu)}
        Si el menú está visible, cuando el usuario presione el botón \IUMenu{} el menú de
        navegación se ocultará.

    \item {\bf \IUAdminSitio{} Administración del sitio }
        Cuando el menú está visible, el botón de administración del sitio nos permitirá
        navegar a la pantalla \refElem{IU-M01}

    \end{itemize}
  % Tablero (Dashboard)
    
\subsubsection{IU-M00a Selector de archivos}

 El selector de archivos permite a los usuarios de moodle seleccionar un archivo desde los archivos
 del servidor, archivos recientes, archivos privados, desde la computadora o incluso buscar imágenes
 para ser seleccionadas \cite{MoodleSelectorArchivos}.

    \IUfig{1}{modulos/moodle/IU/PopUpArchivos.png}{IU-M00a}{Selector de archivos}

\subsubsection{Elementos relevantes}

    \begin{itemize}
    \item {\bf Menú izquierdo}
        Permite al usuario escoger desde que medio seleccionará el archivo a elegir.
    \end{itemize}

\subsubsection{Acciones relevantes}

    \begin{itemize}
    \item {\bf Browse (Subir un archivo)}
        Cuando se presione el botón \fbox{Browse}, el navegador desplegará una ventana
        emergente para seleccionar un archivo desde el sistema de archivos.

    \item {\bf Subir este archivo}
        Cuando el usuario presione este botón el usuario confirmará la acción de subir
        el archivo que previamente a seleccionado.
    \end{itemize}
 % Form: Selector de archivos
    
\subsubsection{IU-M00b Ingreso a moodle}

    Esta pantalla es de moodle. Esta pantalla brinda acceso al sistema.

    \IUfig{1}{modulos/moodle/IU/Login}{IU-M00b}{Ingreso a moodle}

\subsubsection{Elementos Relevantes}

    \begin{itemize}
    \item {\bf Lorem ipsum}
        ...
    \end{itemize}

\subsubsection{Acciones relevantes}

    \begin{itemize}
    \item {\bf Lorem ipsum}
        ...
    \end{itemize}

\clearpage
 % Login

    
\subsection{IU-M01 Pantalla principal}

 La página de portada, o página principal mostrada en la figura \ref{IU-M01}, es la
 página inicial que ve alguien que llega a un sitio Moodle antes o después de entrar al sitio.
 Típicamente un estudiante verá los cursos, algunos bloques de información, mostrados en un tema.
 En la Barra de navegación y en el menú de navegación (esquina superior izquierda).\\

 \noindent 
 La combinación de las políticas del sitio, autenticación del usuario y configuraciones de la
 portada determinan quién puede llegar a la portada, los elementos que pueden ver y acciones
 que pueden realizar \cite{MoodlePortada}.
    % https://docs.moodle.org/all/es/Portada

    \IUfig{1}{modulos/IUMoodle/IU-M01-moodle.png}{IU-M01}{Pantalla Principal de Moodle}

\subsubsection{Elementos Relevantes}

    \begin{itemize}
    \item
    {\bf Menú Superior}
        Como su nombre lo indica se encuentra en la parte superior, este elemento se
        encuentra en la mayoría de las pantallas de moodle.

    \item
    {\bf Menú de Navegación}
        Cuando esta visible se encuentra en la parte izquierda de la parte izquierda
        de la mayoría de las pantallas de moodle. Se puede ocultar o mostrar con la
        acción \IUMenu[].

    \item
    {\bf Contenido}
        Tiene todos los demás elementos que conforman el contenido de la pantalla.

    \end{itemize}

\subsubsection{Acciones relevantes}

    \begin{itemize}
    
    \item
    {\bf \IUMenu (Desplegar el menú)}
        Si el menú está oculto, cuando el usuario presione el botón \IUMenu el menú de
        navegación se desplegará.

    \item {\bf \IUMenu (Ocultar Menu)}
        Si el menú está visible, cuando el usuario presione el botón \IUMenu el menú de
        navegación se ocultará.

    \item {\bf \IUAdminSitio Administración del sitio }
        Cuando el menú está visible, el botón de administración del sitio nos permitirá
        navegar a la pantalla \refElem{IU-M02}

    \end{itemize}
  % Administración del sitio
    
\subsection{IU-M01a: Resultado de instalación del plugin}

 Esta pantalla muestra el resultado de la instalación del plugin. Si los plugins
 que son instalados vienen de una fuente confiable, esta pantalla siempre debería
 de aparecer mostrando un resultado existoso, en caso contrario dira que la instalación
 no pudo llevarse a cabo de forma exitosa.

    \IUfig{1}{modulos/IUMoodle/InstallResult.png}{IU-M01a}{Resultado de instalación del plugin}

\subsubsection{Elementos relevantes}

    \begin{itemize}
    \item {\bf Mensaje de estado de instalación}
        El mensaje se pinta de color verde o color rojo dependiendo si el 
    \end{itemize}

\subsubsection{Acciones relevantes}

    \begin{itemize}
    \item {\bf Aceptar}
        Si el plugin possé configuraciones para el administrador entonces esta acción
        redirigirá a la pantalla de configuración correspondiente al plugin, en caso
        contrario redirigirá a la anterior página del sitio de moodle mostrada.
    \end{itemize}
 % Administración del sitio plugins
    
\subsubsection{IU-M01b Administración del sitio (Usuarios)}

 Esta pantalla permite al \refElem{aAdministrador} acceder a las configuraciones
 específicas de usuarios dentro del moodle que administra. En esta pantalla se
 encuentran tres categorías principales de configuraciones, dichas categorías con
 usuarios, cuentas y permisos, cada una con sus correspondientes configuraciones
 específicas.

    \IUfig{1}{modulos/moodle/IU/AdministracionSitioUsers}%
        {IU-M01b}{Administración del sitio (Usuarios)}

\subsubsection{Elementos Relevantes}

    \begin{itemize}
    \item {\bf Menu de opciones}
        En la parte principal de la pantalla se encuentra la lista de las
        opciones agrupadas en las tres categorías de usuarios, cuentas y permisos.
    \end{itemize}

\subsubsection{Acciones relevantes}

    \begin{itemize}
    \item {\bf Agregar un usuario}
        Permite acceder al formulario para crear cuentas para distintos usuarios.

    \item {\bf Mirar la lista de usuarios}
        Permite acceder a la pantalla para gestionar las distintas cuentas de usuarios
        de la plataforma.
    \end{itemize}

\clearpage
 % Administración del sitio usuarios
    
\subsubsection{IU-M01c Administración del sitio (Cursos)}

 Esta pantalla permite al \refElem{aProfesor} acceder a las configuraciones
 específicas de los cursos. El acceso a esta pantalla es indispensable para que el 
 profesor cree un curso, lo edite y tambien administre su contenido.

    \IUfig{1}{modulos/moodle/IU/AdministracionSitioCursos}%
        {IU-M01c}{Administración del sitio (módulos)}

\subsubsection{Elementos Relevantes}

    \begin{itemize}
    \item {\bf Menú de opciones}
        Para la cuenta del profesor se muestran las opciones mínimas requeridas
        para acceder a la gestión de los cursos y de las categorías a las que
        están vinculadas los cursos.
    \end{itemize}

\subsubsection{Acciones relevantes}

    \begin{itemize}
    \item {\bf Selección de la opción gestionar cursos y categorías}
        Redirige a la pantalla para la gestión de cursos y categorías.
    \end{itemize}

\clearpage
 % Administración del sitio Cursos

    
\subsection{IU-M02: Instalador de plugin}

 La página del instalador de plugins permite al \refElem{aAdministrador} instalar nuevos plugins al 
 moodle que administra de una forma sencilla y sin tener que manipular los archivos en el servidor
 donde se tenga moodle instalado, para ello cada \refElem[plugin]{Plugin} a instalar debe estar
 compresos en un archivo {\it ZIP} cumpliendo con la regla \refElem{BR-M1}.

 % TODO: BR-M1: Restricciones el archivo de instalación.
 % El archivo de instalación debe ser un archivo ZIP, el cual debe contener exactamente un
 % directorio que coincida con el nombre del plugin.

    \IUfig{1}{modulos/moodleIU/InstallPlugin.png}{IU-M02}{Instalador de plugin}

\subsubsection{Elementos relevantes}

    \begin{itemize}
    \item {\bf Selector de archivos.}
        Permite elegir un archivo y prepararlo para subirlo a moodle
        y realizar las acciones correspondientes.
    \end{itemize}

\subsubsection{Acciones relevantes}

    \begin{itemize}
    \item {\bf Selección de un archivo}
        Permite seleccionar un \refElem{Plugin} compreso en un archivo {\it ZIP} para
        ser instalado en moodle.

    \item {\bf Instalar plugin desde un archivo ZIP}
        Confirma el envió del formulario que contiene principalmente al archivo compreso con 
        el plugin que será instalado. Redirige a la pantalla \refElem{IU-M02a}.
    \end{itemize}

\clearpage
  % Instalación de Plugin
    
\subsection{IU-M02a Validación del plugin a instalar}

 La pantalla de validación del plugin a instalar presenta el resultado de la validación de un
 archivo de plugin compreso con base en la regla \refElem{BR-M01}. Esta pantalla dira Si el archivo
 esta formado correctamente o no, además de las acciones adicionales que se llevarán a cabo.

    \IUfig{1}{modulos/moodleIU/PluginZIPValidacion}{IU-M02a}{Validación del plugin a instalar}

\subsubsection{Elementos Relevantes}

    \begin{itemize}
    \item {\bf Validación del plugin}
        Contiene el resultado de la validación del plugin más las acciones
        a realizar para proceder con la instalación del plugin.
        
    \end{itemize}

\subsubsection{Acciones relevantes}

    \begin{itemize}
    \item {\bf Continuar}
        En caso correcto de la validación, este botón permite continuar con la instalación
        rediriginedo a la página \refElem{IU-M02b}.

    \item {\bf Cancelar}
        El botón de cancelar interrumpe el proceso de instalación del plugin, redirigiendo
        a la anterior pantalla \refElem{IU-M01}.
    \end{itemize}

\clearpage
 % Validación de archivo ZIP
    
\subsubsection{IU-M02b Comprobación de plugins}

 Esta página muestra los plugins que pueden requerir su atención durante un actualización al sitio
 de moodle, como una la instalación o actualización de plugins. La documentación de esta pantalla
 contempla únicamente el caso de instalación/actualización de un plugin a la vez.

    \IUfig{1}{modulos/moodle/IU/InstallConfirm}{IU-M02b}{Comprobación de plugins}

\subsubsection{Elementos relevantes}

   \begin{itemize}
   \item {\bf Lista de plugins a instalar}
        La lista de \refElem[Plugins]{Plugin} que se van a instalar, incluyendo
        sus atributos.
   \end{itemize}

\subsubsection{Acciones relevantes}

    \begin{itemize}
    \item {\bf Actualizar base de datos de Moodle ahora}
        Esta acción permite instalar el plugin en Moodle y correr la secuencia de
        instrucciones establecida por el plugin a instalar. Redirige a la pantalla
        \refElem{IU-M02d}.

    \item {\bf Cancelar las nuevas instalaciones}
        Esta acción permite cancelar las instalaciones o actualizaciones de los plugins,
        redirige a la pantalla \refElem{IU-M02c}

    \item {\bf Cancelar esta instalación}
        A diferencia de la acción anterior, esta acción permite cancelar la instalación
        o actualización de un plugin en particular anterior. Redirige a la pantalla
        \refElem{IU-M02c}.
    \end{itemize}

\clearpage
 % Comprobación de plugibs
    
\subsubsection{IU-M02c: Cancelación de instalación de plugins}

 Esta pantalla tiene el propósito de notificar al \refElem{aAdministrador} de las acciones a
 llevar a cabo en caso de proseguir con la cancelación de la instalación de los plugins. Esta es
 la última confirmación que se le pregunta al administrador antes de la cancelación.

    \IUfig{1}{modulos/moodle/IU/InstallCancelled.png}{IU-M02c}{Comprobación de pluginc}

\subsubsection{Elementos relevantes}

    \begin{itemize}
    \item {\bf Lista de los plugins}
        Contiene la lista de los plugins y la ubicación absoluta de la carpeta
        que contiene todos los archivos de cada plugin que será eliminado.
    \end{itemize}

\subsubsection{Acciones relevantes}

    \begin{itemize}
    \item {\bf Continuar}
        Esta acción confirma la cancelación de los plugins y eliminación de los
        archivos de los mismos. Redirige a la pantalla \refElem{IU-M02}.

    \item {\bf Cancelar}
        Esta acción regresa a la pantalla \refElem{IU-M01} para continuar con la instalación
        de plugins.
    \end{itemize}

\clearpage
 % Cancelación Instalación Plugin
    
\subsubsection{IU-M02d: Resultado de instalación del plugin}

 Esta pantalla muestra el resultado de la instalación del plugin. Si los plugins
 que son instalados vienen de una fuente confiable, esta pantalla siempre debería
 de aparecer mostrando un resultado existoso, en caso contrario dira que la instalación
 no pudo llevarse a cabo de forma exitosa.

    \IUfig{1}{modulos/moodle/IU/InstallResult.png}{IU-M02d}{Resultado de instalación del plugin}

\subsubsection{Elementos relevantes}

    \begin{itemize}
    \item {\bf Mensaje de estado de instalación}
        El mensaje se pinta de color verde o color rojo dependiendo si el 
    \end{itemize}

\subsubsection{Acciones relevantes}

    \begin{itemize}
    \item {\bf Aceptar}
        Si el plugin poseé configuraciones para el administrador entonces esta acción
        redirigirá a la pantalla de configuración correspondiente al \refElem{Plugin},
        en caso contrario redirigirá a la anterior página del sitio de moodle mostrada.
    \end{itemize}

\clearpage
 % Resultado Instalación Plugin

    
\subsubsection{IU-M03 Vista General de Plugins}

 Esta pantalla permite al \refElem{aAdministrador} visualizar la lista de los plugins
 instalados en moodle incluyen los que vienen incluidos en modole, aśi como los
 adicionales, tambien permite acceder a la pantalla \refElem{IU-M03a} para ver la
 lista de todos los plugins.

    \IUfig{1}{modulos/moodle/IU/VistaGeneralPlugins}{IU-M03}{Vista General de Plugins}

\subsubsection{Elementos Relevantes}

    \begin{itemize}
    \item {\bf Lista de todos los plugins}
        Contiene la lista de todos los plugins agrupados por el tipo de plugin, 
        se pueden visualizar y realizar acciones de configuración y eliminación
        a cada elemento de lista.
    \end{itemize}

\subsubsection{Acciones relevantes}

    \begin{itemize}
    \item {\bf Cambiar la lista mostrada de los plugins}
        Permite seleccionar la pestaña correspondiente a la lista de plugins
        que se desea ver.
    \item {\bf Acceder a la configuración de un plugin}
        Si el plugin tiene configuración mediante este enlace se puede acceder
        a la configuración del plugin correspondiente.
    \item {\bf Desinstalar}
        Permite la desinstalación de los plugins que no sean requeridos en la
        plataforma.
    \end{itemize}

\clearpage
  % 
    
\subsubsection{IU-M03a Vista de Plugins Adicionales}

 Esta pantalla permite al \refElem{aAdministrador} visualizar la lista de los plugins
 instalados adicionales instalados, tambien permite acceder a la pantalla
 \refElem{IU-M03} para ver la lista de todos los plugins.

    \IUfig{1}{modulos/moodle/IU/VistaPluginsAdicionales}%
        {IU-M03a}{Vista de Plugins Adicionales}

\subsubsection{Elementos Relevantes}

    \begin{itemize}
    \item {\bf Lista de plugins adicionales}
        Contiene la lista de todos los plugins externos a moodle, instalados por
        el \refElem{aAdministrador} agrupados por el tipo de plugin, de la misma
        forma se pueden visualizar y realizar acciones de configuración y eliminación
        a cada elemento de lista.
    \end{itemize}

\subsubsection{Acciones relevantes}

    \begin{itemize}
    \item {\bf Cambiar la lista mostrada de los plugins}
        Permite seleccionar la pestaña correspondiente a la lista de plugins
        que se desea ver.
    \item {\bf Acceder a la configuración de un plugin}
        Si el plugin tiene configuración mediante este enlace se puede acceder
        a la configuración del plugin correspondiente.
    \item {\bf Desinstalar}
        Permite la desinstalación de los plugins que no sean requeridos en la
        plataforma.
    \end{itemize}

\clearpage
 % 

    
\subsubsection{IU-M04 Desinstalar plugin}

 Esta pantalla muestra un mensaje de confirmación para la acción de eliminar un
 plugin en específico.

    \IUfig{1}{modulos/moodle/IU/DesinstalarPluginConfirm}{IU-M04}{Desinstalar plugin}

\subsubsection{Elementos Relevantes}

    \begin{itemize}
    \item {\bf Mensaje de confirmación}
        Muestra un mensaje solicitando la confirmación del usuario acerca de la
        desinstalación de un plugin.
    \end{itemize}

\subsubsection{Acciones relevantes}

    \begin{itemize}
    \item {\bf Continuar}
        Permite al usuario confirmar a operación de desinstalar un plugin.
    \item {\bf Cancelar}
        Permite al usuario cancelar la operación de desinstalación de un plugin.
    \end{itemize}

\clearpage
  % 
    
\subsubsection{IU-E04a Desinstalando plugin}

 Cuando se desinstala un plugin moodle pide la confirmación del usuario para eliminar
 los archivos correspondientes al plugin. Si el usuario decide no eliminar los
 archivos entonces el plugin se detectará como una plugin para ser instalado desde
 cero.

    \IUfig{1}{modulos/moodle/IU/DesinstalandoPlugin}%
        {IU-M04a}{Desinstalando plugin}

\subsubsection{Elementos Relevantes}

    \begin{itemize}
    \item {\bf Mensaje de confirmación}
        Contiene un mensaje que pide la confirmación del usuario para proceder con
        la eliminación de los archivos de un plugin y si dejar los archivos para que
        este sea instalado de nuevo.
    \end{itemize}

\subsubsection{Acciones relevantes}

    \begin{itemize}
    \item {\bf Continuar}
        Permite confirmar la eliminación de los archivos del plugin que fue
        previamente desinstalado.

    \item {\bf Cancelar}
        Permite indicarle a moodle que no elimine los archivos correspondientes al
        plugin se acaba de eliminar, esto permite que el plugin se pueda instalar
        desde cero.
    \end{itemize}

\clearpage
 % 

    
\subsubsection{IU-M05 Creación de usuario}

 Descripción ...

    \IUfig{1}{modulos/moodle/IU/CreateUser}{IU-M05}{Creación de usuario}

\subsubsection{Elementos Relevantes}

    \begin{itemize}
    \item {\bf Lorem ipsum}
        ...
    \end{itemize}

\subsubsection{Acciones relevantes}

    \begin{itemize}
    \item {\bf Lorem ipsum}
        ...
    \end{itemize}

\clearpage
  % 
    
\subsubsection{IU-M05a Lista de usuarios}

 Descripción ...

    \IUfig{1}{modulos/moodle/IU/ListUsers}{IU-M05a}{Lista de Usuarios}

\subsubsection{Elementos Relevantes}

    \begin{itemize}
    \item {\bf Lorem ipsum}
        ...
    \end{itemize}

\subsubsection{Acciones relevantes}

    \begin{itemize}
    \item {\bf Lorem ipsum}
        ...
    \end{itemize}

\clearpage
 % 
    
\subsubsection{IU-M05b Confirmación acerca de eliminar un usuario}

 Mensaje de confirmación acerca de la eliminación de un usuario.

    \IUfig{1}{modulos/moodle/IU/DeleteUser}%
        {IU-M05b}{Confirmación acerca de eliminar un usuario}

\subsubsection{Elementos Relevantes}

    \begin{itemize}
    \item {\bf Mensaje de confirmación}
        Presenta el mensaje de confirmación para continuar con la operación de
        eliminación de un usuario.
    \end{itemize}

\subsubsection{Acciones relevantes}

    \begin{itemize}
    \item {\bf Eliminar}
        Permite continuar con las operaciones de eliminación de un usuario
        de moodle.
        
    \item {\bf Cancelar}
        Permite cancelar las operaciones de eliminación de un usuario de moodle.
    \end{itemize}

\clearpage
 % 
    
\subsubsection{IU-M05c Pantalla de auto-registro}

 Ofrece a los usuarios que no estan registrados en moodle un mecanismo mediante
 el cual pueden acceder a crear su usuario de moodle de forma por su propia cuenta.

    \IUfig{1}{modulos/moodle/IU/CreateOwnUser}{IU-M05c}{Pantalla de auto-registro}

\subsubsection{Elementos Relevantes}

    \begin{itemize}
    \item {\bf Formulario de registro}
        Contiene los campos para el registro de un usuario de moodle
    \end{itemize}

\subsubsection{Acciones relevantes}

    \begin{itemize}
    \item {\bf Crear nueva cuenta}
        Permite al usuario confirmar la creación de se nueva cuenta en el
        sitio de moodle

    \item {\bf Cancelar}
        Permite al usuario cancelar el registro de su cuenta de usuario.
    \end{itemize}

\clearpage
 % 

    
\subsubsection{IU-M06 Cursos y categorías}

 Esta pantalla permite acceder a la gestión de cursos y categorías existentes en
 moodle, partiendo de este punto se pueden editar, configurar, eliminar y crear
 cursos y categorías presentes en moodle. Los actores que pueden acceder a esta
 pantalla son el \refElem{aProfesor} y el \refElem{aAdministrador}.

    \IUfig{1}{modulos/moodle/IU/Cursos-Categories}{IU-M06}{Cursos y categorías}

\subsubsection{Elementos Relevantes}

    \begin{itemize}
    \item {\bf Lista de categorías}
        Muestra la lista de categorías existentes en el sitio de moodle.
    \item {\bf Lista de cursos}
        Contiene la lista de los cursos correspondientes a la categoría
        seleccionada, sobre esta lista se puden ejecutar las acciones
        relevantes..
    \end{itemize}

\subsubsection{Acciones relevantes}

    \begin{itemize}
    \item {\bf Crear un nuevo curso}
        Permite crear un nuevo curso con la categoría seleccionada como
        la categoría por defecto.
    \item {\bf Visualizar curso}
        Permite acceder a la pantalla de la visualización de un nuevo curso
        para administrarlo y/o editarlo.
        la categoría por defecto.
    \item {\bf Eliminar un curso}
        Permite seleccionar un curso que se deseé eliminar.
    \end{itemize}

\clearpage
  % Cursos y Categorías
    
\subsubsection{IU-M06a Confirmación para eliminar un curso}

 %Descripción ...

    \IUfig{1}{modulos/moodle/IU/Cursos-Delete}%
        {IU-M06a}{Confirmación para eliminar un curso}

\begin{comment}
\subsubsection{Elementos Relevantes}

    \begin{itemize}
    \item {\bf Lorem ipsum}
        ...
    \end{itemize}

\subsubsection{Acciones relevantes}

    \begin{itemize}
    \item {\bf Lorem ipsum}
        ...
    \end{itemize}
\end{comment}

\clearpage
 % Confirmación eliminar curso
    
\subsubsection{IU-M06b Estado de la eliminación de un curso}

% Descripción ...

    \IUfig{1}{modulos/moodle/IU/Cursos-Delete-Status}%
        {IU-M06b}{Estado de la eliminación de un curso}

\begin{comment}
\subsubsection{Elementos Relevantes}

    \begin{itemize}
    \item {\bf Lorem ipsum}
        ...
    \end{itemize}

\subsubsection{Acciones relevantes}

    \begin{itemize}
    \item {\bf Lorem ipsum}
        ...
    \end{itemize}
\end{comment}

\clearpage
 % Estado de eliminación de un curso
    
\subsubsection{IU-M06c Curso de moodle}

 Descripción ...

    \IUfig{1}{modulos/moodle/IU/Cursos-normal}{IU-M06c}{Curso de moodle}

\subsubsection{Elementos Relevantes}

    \begin{itemize}
    \item {\bf Lorem ipsum}
        ...
    \end{itemize}

\subsubsection{Acciones relevantes}

    \begin{itemize}
    \item {\bf Lorem ipsum}
        ...
    \end{itemize}

\clearpage
 % Curso de moodle

    
\subsubsection{IU-M07 Pantalla principal del curso}

 Descripción ...

    \IUfig{1}{modulos/moodle/IU/p_principal_curso}{IU-M07}{Pantalla principal del curso}

\subsubsection{Elementos Relevantes}

    \begin{itemize}
    \item {\bf Lorem ipsum}
        ...
    \end{itemize}

\subsubsection{Acciones relevantes}

    \begin{itemize}
    \item {\bf Lorem ipsum}
        ...
    \end{itemize}

\clearpage
 % Pagina Inicial del Sitio

\subsection*{Interfaces diseñadas}

 A continuación se presentan todas las interfaces que fueron requeridas agregar
 a moodle para brindar las funcionalidades planteadas por los casos de uso. Al igual
 que la especificación de las interfaces de moodle, la especificación de estas
 interfaces contiene la imagen de la interfaz y una descripción de los elementos
 y acciones más relevantes por cada interfaz.

    
\subsubsection{IU-E01 Configuraciones del módulo de experiencia}

 Descripción ...

    \IUfig{1}{modulos/exp/IU/Settings}{IU-E01}{Configuraciones del módulo de experiencia}

\subsubsection{Elementos Relevantes}

    \begin{itemize}
    \item {\bf Lorem ipsum}
        ...
    \end{itemize}

\subsubsection{Acciones relevantes}

    \begin{itemize}
    \item {\bf Lorem ipsum}
        ...
    \end{itemize}

\clearpage
  % Configuraciones
    
\subsection{IU-E02 Configuraciones generales}

 Descripción ...

    \IUfig{1}{modulos/modExpIU/SettingsGenerales}{IU-E02}{Configuraciones generales}

\subsubsection{Elementos Relevantes}

    \begin{itemize}
    \item {\bf Lorem ipsum}
        ...
    \end{itemize}

\subsubsection{Acciones relevantes}

    \begin{itemize}
    \item {\bf Lorem ipsum}
        ...
    \end{itemize}

\clearpage
  % Configuraciones Generales
    
\subsection{IU-E03: Configuraciones Visuales del módulo de experiencia}

 Descripción ...

    \IUfig{1}{modulos/exp/IU/settingsVisuales}{IU-E03}{%
        configuraciones Visuales del módulo de experiencia}

\subsubsection{Elementos Relevantes}

    \begin{itemize}
    \item {\bf Lorem ipsum}
        ...
    \end{itemize}

\subsubsection{Acciones relevantes}

    \begin{itemize}
    \item {\bf Guardar cambios}
        ...

    \item {\bf Cancelar}
        ...
    \end{itemize}

\clearpage
  % Configuraciones Visuales
    
\subsection{IU-E03a Módulo de experiencia desactivado}

 Descripción ...

    \IUfig{1}{modulos/modExpIU/settingsExperienceDisabled}{IU-E03a}{%
        Módulo de experiencia desactivado}

\subsubsection{Elementos Relevantes}

    \begin{description}
    \bTerm{tSelectColor}{panel de colores} ...
    \end{description}

\subsubsection{Acciones relevantes}

    \begin{itemize}
    \item {\bf Lorem ipsum}
        ...
    \end{itemize}

\clearpage
 % Configuraciones Mod Exp desactivado
    
\subsubsection{IU-E04: Configuraciones del sistema de experiencia}

 Descripción ...

    \IUfig{1}{modulos/exp/IU/settingsEsquema}{IU-E04}{%
        Configuraciones del sistema de experiencia}

\subsubsection{Elementos Relevantes}

    \begin{itemize}
    \item {\bf Lorem ipsum}
        ...
    \end{itemize}

\subsubsection{Acciones relevantes}

    \begin{itemize}
    \item {\bf Lorem ipsum}
        ...
    \end{itemize}

\clearpage
  % Configuraciones Esquema
    
\subsubsection{IU-E05 Configuración de eventos con experiencia}

% Descripción ...

    \IUfig{1}{modulos/exp/IU/SettingsEventos}{IU-E05}%
        {Configuración de eventos con experiencia}

\begin{comment}
\subsubsection{Elementos Relevantes}

    \begin{itemize}
    \item {\bf Lorem ipsum}
        ...
    \end{itemize}

\subsubsection{Acciones relevantes}

    \begin{itemize}
    \item {\bf Lorem ipsum}
        ...
    \end{itemize}
\end{comment}

\clearpage
  % Configuraciones de Eventos
    
\subsection{IU-E05a Eventos con experiencia desactivados}

 Descripción ...

    \IUfig{1}{modulos/exp/IU/SettingsEventsDisabled}{IU-E05a}%
    {Eventos con experiencia desactivados}

\subsubsection{Elementos Relevantes}

    \begin{itemize}
    \item {\bf Lorem ipsum}
        ...
    \end{itemize}

\subsubsection{Acciones relevantes}

    \begin{itemize}
    \item {\bf Lorem ipsum}
        ...
    \end{itemize}

\clearpage
 % Configuraciones Events desativados
    
\subsubsection{IU-E06 Creación de un curso gamificado}

 Descripción ...

    \IUfig{1}{modulos/exp/IU/CursoCreate}{IU-E06}{Creación curso gamificado}

\subsubsection{Elementos Relevantes}

    \begin{itemize}
    \item {\bf Lorem ipsum}
        ...
    \end{itemize}

\subsubsection{Acciones relevantes}

    \begin{itemize}
    \item {\bf Lorem ipsum}
        ...
    \end{itemize}

\clearpage
  % Curso Gamificado
    
\subsubsection{IU-E06a Curso Gamificado}

 Descripción ...

    \IUfig{1}{modulos/exp/IU/CursoView}{IU-E06a}{Curso Gamificado}

\subsubsection{Elementos Relevantes}

    \begin{itemize}
    \item {\bf Lorem ipsum}
        ...
    \end{itemize}

\subsubsection{Acciones relevantes}

    \begin{itemize}
    \item {\bf Lorem ipsum}
        ...
    \end{itemize}

\clearpage
  % Curso Gamificado
    
\subsection{IU-E06b Curso gamificado en edición}

 Descripción ...

    \IUfig{1}{modulos/exp/IU/CursoEdit}{IU-E06b}{Curso gamificado en edición}

\subsubsection{Elementos Relevantes}

    \begin{itemize}
    \item {\bf Lorem ipsum}
        ...
    \end{itemize}

\subsubsection{Acciones relevantes}

    \begin{itemize}
    \item {\bf Lorem ipsum}
        ...
    \end{itemize}

\clearpage
 % Curso Gamificado (editing)

    %
\subsubsection{IU-E04: Configuraciones del sistema de experiencia}

 Descripción ...

    \IUfig{1}{modulos/exp/IU/settingsEsquema}{IU-E04}{%
        Configuraciones del sistema de experiencia}

\subsubsection{Elementos Relevantes}

    \begin{itemize}
    \item {\bf Lorem ipsum}
        ...
    \end{itemize}

\subsubsection{Acciones relevantes}

    \begin{itemize}
    \item {\bf Lorem ipsum}
        ...
    \end{itemize}

\clearpage
 % Configuraciones de Comportamiento
    %
\subsubsection{IU-E05 Configuración de eventos con experiencia}

% Descripción ...

    \IUfig{1}{modulos/exp/IU/SettingsEventos}{IU-E05}%
        {Configuración de eventos con experiencia}

\begin{comment}
\subsubsection{Elementos Relevantes}

    \begin{itemize}
    \item {\bf Lorem ipsum}
        ...
    \end{itemize}

\subsubsection{Acciones relevantes}

    \begin{itemize}
    \item {\bf Lorem ipsum}
        ...
    \end{itemize}
\end{comment}

\clearpage
 % Configuraciones de eventos

\subsection*{Diseño de plugins} % TODO CHANGE FOR INPUTS

 Para poder desarrollar las funcionalidades especificadas mediante los casos de uso
 es requerido el desarrollo de tres plugins distintos. Dichos plugins son listados a
 continuación:

    \begin{itemize}
    \item {\bf\color{primary}%
    \hypertarget{local:gamedlemaster}{Gamedlemaster} (local)}

        Este plugin ayudará a extender el esquema de la base de datos de moodle
        para que los plugins de experiencia y los de los demás módulos puedan
        acceder a los datos sin la necesidad de depender explícitamente de otro
        módulo.

    \item {\bf\color{primary}%
    \hypertarget{format:gamedle}{Curso con puntos de experiencia} (course format)}

        Este es un plugin de tipo {\it course format} lo cual permite cambiar
        la estructura de cómo se presentan las secciones y actividades del curso
        \cite{MoodleCourseFormat}. Este tipo de plugin fue escogido para que los
        cursos en moodle puedan otorgar puntos de experiencia.
        % https://docs.moodle.org/dev/Course_formats

    \item {\bf\color{primary}%
    \hypertarget{block:gmxp}{Nivel Gamedle} (block)}

        Los plugins de tipo {\it block} permiten desplegar información y/o brindar una 
        funcionalidad a lo largo de distintas páginas de moodle \cite{MoodleBlocks},
        además permite a cualquier usuario agregarlo u ocultarlo en las distintas
        pantallas a las que tiene acceso, debido a estar razones este plugin se
        encargará de mostrar el nivel y puntos de experiencia de dicho nivel que
        tiene un usuario.
        % https://docs.moodle.org/dev/Plugin_types
    \end{itemize}

\subsubsection{Diagrama de componentes} % TODO CHANGE FOR INPUTS
\subsubsection{Diagrama de clases} % TODO CHANGE FOR INPUTS


\begin{comment} % ==================================================================
\subsubsection{Interfaces de Moodle}

    % INTERFACES DE MOODLE
    
\subsubsection{IU-M00 Pantalla principal}

 El tablero o {\it Dashboard} (ver figura~\ref{IU-M02}) es la primer página que ve un usuario de
 inmediatamente despues iniciar sesión, esta página muestra a los usuarios detalles de su progreso
 y fechas límite próximas \cite{MoodleTablero} . Los elementos que tiene esta página con las demás
 páginas del sitio de moodle es el menú de navegación a la izquierda y la columna derecha de
 bloques.

    \IUfig{1}{modulos/moodle/IU/Dashboard.png}{IU-M00}{Pantalla Principal de Moodle}

\subsubsection{Elementos relevantes}

    \begin{itemize}
    \item
    {\bf Menú Superior}
        Como su nombre lo indica se encuentra en la parte superior, este elemento se
        encuentra en la mayoría de las pantallas de moodle.

    \item
    {\bf Menú de Navegación}
        Cuando esta visible se encuentra en la parte izquierda de la parte izquierda
        de la mayoría de las pantallas de moodle. Se puede ocultar o mostrar con la
        acción \IUMenu[].

    \item
    {\bf Contenido}
        Tiene todos los demás elementos que conforman el contenido de la pantalla.

    \end{itemize}

\subsubsection{Acciones relevantes}

    \begin{itemize}
    
    \item
    {\bf \IUMenu{} (Desplegar el menú)}
        Si el menú está oculto, cuando el usuario presione el botón \IUMenu{} el menú de
        navegación se desplegará.

    \item {\bf \IUMenu{} (Ocultar Menu)}
        Si el menú está visible, cuando el usuario presione el botón \IUMenu{} el menú de
        navegación se ocultará.

    \item {\bf \IUAdminSitio{} Administración del sitio }
        Cuando el menú está visible, el botón de administración del sitio nos permitirá
        navegar a la pantalla \refElem{IU-M01}

    \end{itemize}
  % Tablero (Dashboard)
    
\subsubsection{IU-M00a Selector de archivos}

 El selector de archivos permite a los usuarios de moodle seleccionar un archivo desde los archivos
 del servidor, archivos recientes, archivos privados, desde la computadora o incluso buscar imágenes
 para ser seleccionadas \cite{MoodleSelectorArchivos}.

    \IUfig{1}{modulos/moodle/IU/PopUpArchivos.png}{IU-M00a}{Selector de archivos}

\subsubsection{Elementos relevantes}

    \begin{itemize}
    \item {\bf Menú izquierdo}
        Permite al usuario escoger desde que medio seleccionará el archivo a elegir.
    \end{itemize}

\subsubsection{Acciones relevantes}

    \begin{itemize}
    \item {\bf Browse (Subir un archivo)}
        Cuando se presione el botón \fbox{Browse}, el navegador desplegará una ventana
        emergente para seleccionar un archivo desde el sistema de archivos.

    \item {\bf Subir este archivo}
        Cuando el usuario presione este botón el usuario confirmará la acción de subir
        el archivo que previamente a seleccionado.
    \end{itemize}
 % Form: Selector de archivos
    
\subsubsection{IU-M00b Ingreso a moodle}

    Esta pantalla es de moodle. Esta pantalla brinda acceso al sistema.

    \IUfig{1}{modulos/moodle/IU/Login}{IU-M00b}{Ingreso a moodle}

\subsubsection{Elementos Relevantes}

    \begin{itemize}
    \item {\bf Lorem ipsum}
        ...
    \end{itemize}

\subsubsection{Acciones relevantes}

    \begin{itemize}
    \item {\bf Lorem ipsum}
        ...
    \end{itemize}

\clearpage
 % Login

    
\subsection{IU-M01 Pantalla principal}

 La página de portada, o página principal mostrada en la figura \ref{IU-M01}, es la
 página inicial que ve alguien que llega a un sitio Moodle antes o después de entrar al sitio.
 Típicamente un estudiante verá los cursos, algunos bloques de información, mostrados en un tema.
 En la Barra de navegación y en el menú de navegación (esquina superior izquierda).\\

 \noindent 
 La combinación de las políticas del sitio, autenticación del usuario y configuraciones de la
 portada determinan quién puede llegar a la portada, los elementos que pueden ver y acciones
 que pueden realizar \cite{MoodlePortada}.
    % https://docs.moodle.org/all/es/Portada

    \IUfig{1}{modulos/IUMoodle/IU-M01-moodle.png}{IU-M01}{Pantalla Principal de Moodle}

\subsubsection{Elementos Relevantes}

    \begin{itemize}
    \item
    {\bf Menú Superior}
        Como su nombre lo indica se encuentra en la parte superior, este elemento se
        encuentra en la mayoría de las pantallas de moodle.

    \item
    {\bf Menú de Navegación}
        Cuando esta visible se encuentra en la parte izquierda de la parte izquierda
        de la mayoría de las pantallas de moodle. Se puede ocultar o mostrar con la
        acción \IUMenu[].

    \item
    {\bf Contenido}
        Tiene todos los demás elementos que conforman el contenido de la pantalla.

    \end{itemize}

\subsubsection{Acciones relevantes}

    \begin{itemize}
    
    \item
    {\bf \IUMenu (Desplegar el menú)}
        Si el menú está oculto, cuando el usuario presione el botón \IUMenu el menú de
        navegación se desplegará.

    \item {\bf \IUMenu (Ocultar Menu)}
        Si el menú está visible, cuando el usuario presione el botón \IUMenu el menú de
        navegación se ocultará.

    \item {\bf \IUAdminSitio Administración del sitio }
        Cuando el menú está visible, el botón de administración del sitio nos permitirá
        navegar a la pantalla \refElem{IU-M02}

    \end{itemize}
  % Administración del sitio
    
\subsection{IU-M01a: Resultado de instalación del plugin}

 Esta pantalla muestra el resultado de la instalación del plugin. Si los plugins
 que son instalados vienen de una fuente confiable, esta pantalla siempre debería
 de aparecer mostrando un resultado existoso, en caso contrario dira que la instalación
 no pudo llevarse a cabo de forma exitosa.

    \IUfig{1}{modulos/IUMoodle/InstallResult.png}{IU-M01a}{Resultado de instalación del plugin}

\subsubsection{Elementos relevantes}

    \begin{itemize}
    \item {\bf Mensaje de estado de instalación}
        El mensaje se pinta de color verde o color rojo dependiendo si el 
    \end{itemize}

\subsubsection{Acciones relevantes}

    \begin{itemize}
    \item {\bf Aceptar}
        Si el plugin possé configuraciones para el administrador entonces esta acción
        redirigirá a la pantalla de configuración correspondiente al plugin, en caso
        contrario redirigirá a la anterior página del sitio de moodle mostrada.
    \end{itemize}
 % Administración del sitio plugins
    
\subsubsection{IU-M01b Administración del sitio (Usuarios)}

 Esta pantalla permite al \refElem{aAdministrador} acceder a las configuraciones
 específicas de usuarios dentro del moodle que administra. En esta pantalla se
 encuentran tres categorías principales de configuraciones, dichas categorías con
 usuarios, cuentas y permisos, cada una con sus correspondientes configuraciones
 específicas.

    \IUfig{1}{modulos/moodle/IU/AdministracionSitioUsers}%
        {IU-M01b}{Administración del sitio (Usuarios)}

\subsubsection{Elementos Relevantes}

    \begin{itemize}
    \item {\bf Menu de opciones}
        En la parte principal de la pantalla se encuentra la lista de las
        opciones agrupadas en las tres categorías de usuarios, cuentas y permisos.
    \end{itemize}

\subsubsection{Acciones relevantes}

    \begin{itemize}
    \item {\bf Agregar un usuario}
        Permite acceder al formulario para crear cuentas para distintos usuarios.

    \item {\bf Mirar la lista de usuarios}
        Permite acceder a la pantalla para gestionar las distintas cuentas de usuarios
        de la plataforma.
    \end{itemize}

\clearpage
 % Administración del sitio usuarios
    
\subsubsection{IU-M01c Administración del sitio (Cursos)}

 Esta pantalla permite al \refElem{aProfesor} acceder a las configuraciones
 específicas de los cursos. El acceso a esta pantalla es indispensable para que el 
 profesor cree un curso, lo edite y tambien administre su contenido.

    \IUfig{1}{modulos/moodle/IU/AdministracionSitioCursos}%
        {IU-M01c}{Administración del sitio (módulos)}

\subsubsection{Elementos Relevantes}

    \begin{itemize}
    \item {\bf Menú de opciones}
        Para la cuenta del profesor se muestran las opciones mínimas requeridas
        para acceder a la gestión de los cursos y de las categorías a las que
        están vinculadas los cursos.
    \end{itemize}

\subsubsection{Acciones relevantes}

    \begin{itemize}
    \item {\bf Selección de la opción gestionar cursos y categorías}
        Redirige a la pantalla para la gestión de cursos y categorías.
    \end{itemize}

\clearpage
 % Administración del sitio Cursos

    
\subsection{IU-M02: Instalador de plugin}

 La página del instalador de plugins permite al \refElem{aAdministrador} instalar nuevos plugins al 
 moodle que administra de una forma sencilla y sin tener que manipular los archivos en el servidor
 donde se tenga moodle instalado, para ello cada \refElem[plugin]{Plugin} a instalar debe estar
 compresos en un archivo {\it ZIP} cumpliendo con la regla \refElem{BR-M1}.

 % TODO: BR-M1: Restricciones el archivo de instalación.
 % El archivo de instalación debe ser un archivo ZIP, el cual debe contener exactamente un
 % directorio que coincida con el nombre del plugin.

    \IUfig{1}{modulos/moodleIU/InstallPlugin.png}{IU-M02}{Instalador de plugin}

\subsubsection{Elementos relevantes}

    \begin{itemize}
    \item {\bf Selector de archivos.}
        Permite elegir un archivo y prepararlo para subirlo a moodle
        y realizar las acciones correspondientes.
    \end{itemize}

\subsubsection{Acciones relevantes}

    \begin{itemize}
    \item {\bf Selección de un archivo}
        Permite seleccionar un \refElem{Plugin} compreso en un archivo {\it ZIP} para
        ser instalado en moodle.

    \item {\bf Instalar plugin desde un archivo ZIP}
        Confirma el envió del formulario que contiene principalmente al archivo compreso con 
        el plugin que será instalado. Redirige a la pantalla \refElem{IU-M02a}.
    \end{itemize}

\clearpage
  % Instalación de Plugin
    
\subsection{IU-M02a Validación del plugin a instalar}

 La pantalla de validación del plugin a instalar presenta el resultado de la validación de un
 archivo de plugin compreso con base en la regla \refElem{BR-M01}. Esta pantalla dira Si el archivo
 esta formado correctamente o no, además de las acciones adicionales que se llevarán a cabo.

    \IUfig{1}{modulos/moodleIU/PluginZIPValidacion}{IU-M02a}{Validación del plugin a instalar}

\subsubsection{Elementos Relevantes}

    \begin{itemize}
    \item {\bf Validación del plugin}
        Contiene el resultado de la validación del plugin más las acciones
        a realizar para proceder con la instalación del plugin.
        
    \end{itemize}

\subsubsection{Acciones relevantes}

    \begin{itemize}
    \item {\bf Continuar}
        En caso correcto de la validación, este botón permite continuar con la instalación
        rediriginedo a la página \refElem{IU-M02b}.

    \item {\bf Cancelar}
        El botón de cancelar interrumpe el proceso de instalación del plugin, redirigiendo
        a la anterior pantalla \refElem{IU-M01}.
    \end{itemize}

\clearpage
 % Validación de archivo ZIP
    
\subsubsection{IU-M02b Comprobación de plugins}

 Esta página muestra los plugins que pueden requerir su atención durante un actualización al sitio
 de moodle, como una la instalación o actualización de plugins. La documentación de esta pantalla
 contempla únicamente el caso de instalación/actualización de un plugin a la vez.

    \IUfig{1}{modulos/moodle/IU/InstallConfirm}{IU-M02b}{Comprobación de plugins}

\subsubsection{Elementos relevantes}

   \begin{itemize}
   \item {\bf Lista de plugins a instalar}
        La lista de \refElem[Plugins]{Plugin} que se van a instalar, incluyendo
        sus atributos.
   \end{itemize}

\subsubsection{Acciones relevantes}

    \begin{itemize}
    \item {\bf Actualizar base de datos de Moodle ahora}
        Esta acción permite instalar el plugin en Moodle y correr la secuencia de
        instrucciones establecida por el plugin a instalar. Redirige a la pantalla
        \refElem{IU-M02d}.

    \item {\bf Cancelar las nuevas instalaciones}
        Esta acción permite cancelar las instalaciones o actualizaciones de los plugins,
        redirige a la pantalla \refElem{IU-M02c}

    \item {\bf Cancelar esta instalación}
        A diferencia de la acción anterior, esta acción permite cancelar la instalación
        o actualización de un plugin en particular anterior. Redirige a la pantalla
        \refElem{IU-M02c}.
    \end{itemize}

\clearpage
 % Comprobación de plugibs
    
\subsubsection{IU-M02c: Cancelación de instalación de plugins}

 Esta pantalla tiene el propósito de notificar al \refElem{aAdministrador} de las acciones a
 llevar a cabo en caso de proseguir con la cancelación de la instalación de los plugins. Esta es
 la última confirmación que se le pregunta al administrador antes de la cancelación.

    \IUfig{1}{modulos/moodle/IU/InstallCancelled.png}{IU-M02c}{Comprobación de pluginc}

\subsubsection{Elementos relevantes}

    \begin{itemize}
    \item {\bf Lista de los plugins}
        Contiene la lista de los plugins y la ubicación absoluta de la carpeta
        que contiene todos los archivos de cada plugin que será eliminado.
    \end{itemize}

\subsubsection{Acciones relevantes}

    \begin{itemize}
    \item {\bf Continuar}
        Esta acción confirma la cancelación de los plugins y eliminación de los
        archivos de los mismos. Redirige a la pantalla \refElem{IU-M02}.

    \item {\bf Cancelar}
        Esta acción regresa a la pantalla \refElem{IU-M01} para continuar con la instalación
        de plugins.
    \end{itemize}

\clearpage
 % Cancelación Instalación Plugin
    
\subsubsection{IU-M02d: Resultado de instalación del plugin}

 Esta pantalla muestra el resultado de la instalación del plugin. Si los plugins
 que son instalados vienen de una fuente confiable, esta pantalla siempre debería
 de aparecer mostrando un resultado existoso, en caso contrario dira que la instalación
 no pudo llevarse a cabo de forma exitosa.

    \IUfig{1}{modulos/moodle/IU/InstallResult.png}{IU-M02d}{Resultado de instalación del plugin}

\subsubsection{Elementos relevantes}

    \begin{itemize}
    \item {\bf Mensaje de estado de instalación}
        El mensaje se pinta de color verde o color rojo dependiendo si el 
    \end{itemize}

\subsubsection{Acciones relevantes}

    \begin{itemize}
    \item {\bf Aceptar}
        Si el plugin poseé configuraciones para el administrador entonces esta acción
        redirigirá a la pantalla de configuración correspondiente al \refElem{Plugin},
        en caso contrario redirigirá a la anterior página del sitio de moodle mostrada.
    \end{itemize}

\clearpage
 % Resultado Instalación Plugin

    
\subsubsection{IU-M03 Vista General de Plugins}

 Esta pantalla permite al \refElem{aAdministrador} visualizar la lista de los plugins
 instalados en moodle incluyen los que vienen incluidos en modole, aśi como los
 adicionales, tambien permite acceder a la pantalla \refElem{IU-M03a} para ver la
 lista de todos los plugins.

    \IUfig{1}{modulos/moodle/IU/VistaGeneralPlugins}{IU-M03}{Vista General de Plugins}

\subsubsection{Elementos Relevantes}

    \begin{itemize}
    \item {\bf Lista de todos los plugins}
        Contiene la lista de todos los plugins agrupados por el tipo de plugin, 
        se pueden visualizar y realizar acciones de configuración y eliminación
        a cada elemento de lista.
    \end{itemize}

\subsubsection{Acciones relevantes}

    \begin{itemize}
    \item {\bf Cambiar la lista mostrada de los plugins}
        Permite seleccionar la pestaña correspondiente a la lista de plugins
        que se desea ver.
    \item {\bf Acceder a la configuración de un plugin}
        Si el plugin tiene configuración mediante este enlace se puede acceder
        a la configuración del plugin correspondiente.
    \item {\bf Desinstalar}
        Permite la desinstalación de los plugins que no sean requeridos en la
        plataforma.
    \end{itemize}

\clearpage
  %
    
\subsubsection{IU-M03a Vista de Plugins Adicionales}

 Esta pantalla permite al \refElem{aAdministrador} visualizar la lista de los plugins
 instalados adicionales instalados, tambien permite acceder a la pantalla
 \refElem{IU-M03} para ver la lista de todos los plugins.

    \IUfig{1}{modulos/moodle/IU/VistaPluginsAdicionales}%
        {IU-M03a}{Vista de Plugins Adicionales}

\subsubsection{Elementos Relevantes}

    \begin{itemize}
    \item {\bf Lista de plugins adicionales}
        Contiene la lista de todos los plugins externos a moodle, instalados por
        el \refElem{aAdministrador} agrupados por el tipo de plugin, de la misma
        forma se pueden visualizar y realizar acciones de configuración y eliminación
        a cada elemento de lista.
    \end{itemize}

\subsubsection{Acciones relevantes}

    \begin{itemize}
    \item {\bf Cambiar la lista mostrada de los plugins}
        Permite seleccionar la pestaña correspondiente a la lista de plugins
        que se desea ver.
    \item {\bf Acceder a la configuración de un plugin}
        Si el plugin tiene configuración mediante este enlace se puede acceder
        a la configuración del plugin correspondiente.
    \item {\bf Desinstalar}
        Permite la desinstalación de los plugins que no sean requeridos en la
        plataforma.
    \end{itemize}

\clearpage
 %

    
\subsubsection{IU-M04 Desinstalar plugin}

 Esta pantalla muestra un mensaje de confirmación para la acción de eliminar un
 plugin en específico.

    \IUfig{1}{modulos/moodle/IU/DesinstalarPluginConfirm}{IU-M04}{Desinstalar plugin}

\subsubsection{Elementos Relevantes}

    \begin{itemize}
    \item {\bf Mensaje de confirmación}
        Muestra un mensaje solicitando la confirmación del usuario acerca de la
        desinstalación de un plugin.
    \end{itemize}

\subsubsection{Acciones relevantes}

    \begin{itemize}
    \item {\bf Continuar}
        Permite al usuario confirmar a operación de desinstalar un plugin.
    \item {\bf Cancelar}
        Permite al usuario cancelar la operación de desinstalación de un plugin.
    \end{itemize}

\clearpage
  %
    
\subsubsection{IU-E04a Desinstalando plugin}

 Cuando se desinstala un plugin moodle pide la confirmación del usuario para eliminar
 los archivos correspondientes al plugin. Si el usuario decide no eliminar los
 archivos entonces el plugin se detectará como una plugin para ser instalado desde
 cero.

    \IUfig{1}{modulos/moodle/IU/DesinstalandoPlugin}%
        {IU-M04a}{Desinstalando plugin}

\subsubsection{Elementos Relevantes}

    \begin{itemize}
    \item {\bf Mensaje de confirmación}
        Contiene un mensaje que pide la confirmación del usuario para proceder con
        la eliminación de los archivos de un plugin y si dejar los archivos para que
        este sea instalado de nuevo.
    \end{itemize}

\subsubsection{Acciones relevantes}

    \begin{itemize}
    \item {\bf Continuar}
        Permite confirmar la eliminación de los archivos del plugin que fue
        previamente desinstalado.

    \item {\bf Cancelar}
        Permite indicarle a moodle que no elimine los archivos correspondientes al
        plugin se acaba de eliminar, esto permite que el plugin se pueda instalar
        desde cero.
    \end{itemize}

\clearpage
 %

    
\subsubsection{IU-M05 Creación de usuario}

 Descripción ...

    \IUfig{1}{modulos/moodle/IU/CreateUser}{IU-M05}{Creación de usuario}

\subsubsection{Elementos Relevantes}

    \begin{itemize}
    \item {\bf Lorem ipsum}
        ...
    \end{itemize}

\subsubsection{Acciones relevantes}

    \begin{itemize}
    \item {\bf Lorem ipsum}
        ...
    \end{itemize}

\clearpage
  %
    
\subsubsection{IU-M05a Lista de usuarios}

 Descripción ...

    \IUfig{1}{modulos/moodle/IU/ListUsers}{IU-M05a}{Lista de Usuarios}

\subsubsection{Elementos Relevantes}

    \begin{itemize}
    \item {\bf Lorem ipsum}
        ...
    \end{itemize}

\subsubsection{Acciones relevantes}

    \begin{itemize}
    \item {\bf Lorem ipsum}
        ...
    \end{itemize}

\clearpage
 %
    
\subsubsection{IU-M05b Confirmación acerca de eliminar un usuario}

 Mensaje de confirmación acerca de la eliminación de un usuario.

    \IUfig{1}{modulos/moodle/IU/DeleteUser}%
        {IU-M05b}{Confirmación acerca de eliminar un usuario}

\subsubsection{Elementos Relevantes}

    \begin{itemize}
    \item {\bf Mensaje de confirmación}
        Presenta el mensaje de confirmación para continuar con la operación de
        eliminación de un usuario.
    \end{itemize}

\subsubsection{Acciones relevantes}

    \begin{itemize}
    \item {\bf Eliminar}
        Permite continuar con las operaciones de eliminación de un usuario
        de moodle.
        
    \item {\bf Cancelar}
        Permite cancelar las operaciones de eliminación de un usuario de moodle.
    \end{itemize}

\clearpage
 %
    
\subsubsection{IU-M05c Pantalla de auto-registro}

 Ofrece a los usuarios que no estan registrados en moodle un mecanismo mediante
 el cual pueden acceder a crear su usuario de moodle de forma por su propia cuenta.

    \IUfig{1}{modulos/moodle/IU/CreateOwnUser}{IU-M05c}{Pantalla de auto-registro}

\subsubsection{Elementos Relevantes}

    \begin{itemize}
    \item {\bf Formulario de registro}
        Contiene los campos para el registro de un usuario de moodle
    \end{itemize}

\subsubsection{Acciones relevantes}

    \begin{itemize}
    \item {\bf Crear nueva cuenta}
        Permite al usuario confirmar la creación de se nueva cuenta en el
        sitio de moodle

    \item {\bf Cancelar}
        Permite al usuario cancelar el registro de su cuenta de usuario.
    \end{itemize}

\clearpage
 %

    
\subsubsection{IU-M06 Cursos y categorías}

 Esta pantalla permite acceder a la gestión de cursos y categorías existentes en
 moodle, partiendo de este punto se pueden editar, configurar, eliminar y crear
 cursos y categorías presentes en moodle. Los actores que pueden acceder a esta
 pantalla son el \refElem{aProfesor} y el \refElem{aAdministrador}.

    \IUfig{1}{modulos/moodle/IU/Cursos-Categories}{IU-M06}{Cursos y categorías}

\subsubsection{Elementos Relevantes}

    \begin{itemize}
    \item {\bf Lista de categorías}
        Muestra la lista de categorías existentes en el sitio de moodle.
    \item {\bf Lista de cursos}
        Contiene la lista de los cursos correspondientes a la categoría
        seleccionada, sobre esta lista se puden ejecutar las acciones
        relevantes..
    \end{itemize}

\subsubsection{Acciones relevantes}

    \begin{itemize}
    \item {\bf Crear un nuevo curso}
        Permite crear un nuevo curso con la categoría seleccionada como
        la categoría por defecto.
    \item {\bf Visualizar curso}
        Permite acceder a la pantalla de la visualización de un nuevo curso
        para administrarlo y/o editarlo.
        la categoría por defecto.
    \item {\bf Eliminar un curso}
        Permite seleccionar un curso que se deseé eliminar.
    \end{itemize}

\clearpage
  % Cursos y Categorías
    
\subsubsection{IU-M06a Confirmación para eliminar un curso}

 %Descripción ...

    \IUfig{1}{modulos/moodle/IU/Cursos-Delete}%
        {IU-M06a}{Confirmación para eliminar un curso}

\begin{comment}
\subsubsection{Elementos Relevantes}

    \begin{itemize}
    \item {\bf Lorem ipsum}
        ...
    \end{itemize}

\subsubsection{Acciones relevantes}

    \begin{itemize}
    \item {\bf Lorem ipsum}
        ...
    \end{itemize}
\end{comment}

\clearpage
 % Confirmación eliminar curso
    
\subsubsection{IU-M06b Estado de la eliminación de un curso}

% Descripción ...

    \IUfig{1}{modulos/moodle/IU/Cursos-Delete-Status}%
        {IU-M06b}{Estado de la eliminación de un curso}

\begin{comment}
\subsubsection{Elementos Relevantes}

    \begin{itemize}
    \item {\bf Lorem ipsum}
        ...
    \end{itemize}

\subsubsection{Acciones relevantes}

    \begin{itemize}
    \item {\bf Lorem ipsum}
        ...
    \end{itemize}
\end{comment}

\clearpage
 % Estado de eliminación de un curso
    
\subsubsection{IU-M06c Curso de moodle}

 Descripción ...

    \IUfig{1}{modulos/moodle/IU/Cursos-normal}{IU-M06c}{Curso de moodle}

\subsubsection{Elementos Relevantes}

    \begin{itemize}
    \item {\bf Lorem ipsum}
        ...
    \end{itemize}

\subsubsection{Acciones relevantes}

    \begin{itemize}
    \item {\bf Lorem ipsum}
        ...
    \end{itemize}

\clearpage
 % Curso de moodle

    
\subsubsection{IU-M07 Pantalla principal del curso}

 Descripción ...

    \IUfig{1}{modulos/moodle/IU/p_principal_curso}{IU-M07}{Pantalla principal del curso}

\subsubsection{Elementos Relevantes}

    \begin{itemize}
    \item {\bf Lorem ipsum}
        ...
    \end{itemize}

\subsubsection{Acciones relevantes}

    \begin{itemize}
    \item {\bf Lorem ipsum}
        ...
    \end{itemize}

\clearpage
 % Pagina Inicial del Sitio

\subsubsection{Interfaces del módulo de experiencia}

    
\subsubsection{IU-E01 Configuraciones del módulo de experiencia}

 Descripción ...

    \IUfig{1}{modulos/exp/IU/Settings}{IU-E01}{Configuraciones del módulo de experiencia}

\subsubsection{Elementos Relevantes}

    \begin{itemize}
    \item {\bf Lorem ipsum}
        ...
    \end{itemize}

\subsubsection{Acciones relevantes}

    \begin{itemize}
    \item {\bf Lorem ipsum}
        ...
    \end{itemize}

\clearpage
  % Configuraciones
    
\subsection{IU-E02 Configuraciones generales}

 Descripción ...

    \IUfig{1}{modulos/modExpIU/SettingsGenerales}{IU-E02}{Configuraciones generales}

\subsubsection{Elementos Relevantes}

    \begin{itemize}
    \item {\bf Lorem ipsum}
        ...
    \end{itemize}

\subsubsection{Acciones relevantes}

    \begin{itemize}
    \item {\bf Lorem ipsum}
        ...
    \end{itemize}

\clearpage
  % Configuraciones Generales
    
\subsection{IU-E03: Configuraciones Visuales del módulo de experiencia}

 Descripción ...

    \IUfig{1}{modulos/exp/IU/settingsVisuales}{IU-E03}{%
        configuraciones Visuales del módulo de experiencia}

\subsubsection{Elementos Relevantes}

    \begin{itemize}
    \item {\bf Lorem ipsum}
        ...
    \end{itemize}

\subsubsection{Acciones relevantes}

    \begin{itemize}
    \item {\bf Guardar cambios}
        ...

    \item {\bf Cancelar}
        ...
    \end{itemize}

\clearpage
  % Configuraciones Visuales
    
\subsection{IU-E03a Módulo de experiencia desactivado}

 Descripción ...

    \IUfig{1}{modulos/modExpIU/settingsExperienceDisabled}{IU-E03a}{%
        Módulo de experiencia desactivado}

\subsubsection{Elementos Relevantes}

    \begin{description}
    \bTerm{tSelectColor}{panel de colores} ...
    \end{description}

\subsubsection{Acciones relevantes}

    \begin{itemize}
    \item {\bf Lorem ipsum}
        ...
    \end{itemize}

\clearpage
 % Configuraciones Mod Exp desactivado
    
\subsubsection{IU-E04: Configuraciones del sistema de experiencia}

 Descripción ...

    \IUfig{1}{modulos/exp/IU/settingsEsquema}{IU-E04}{%
        Configuraciones del sistema de experiencia}

\subsubsection{Elementos Relevantes}

    \begin{itemize}
    \item {\bf Lorem ipsum}
        ...
    \end{itemize}

\subsubsection{Acciones relevantes}

    \begin{itemize}
    \item {\bf Lorem ipsum}
        ...
    \end{itemize}

\clearpage
  % Configuraciones Esquema
    
\subsubsection{IU-E05 Configuración de eventos con experiencia}

% Descripción ...

    \IUfig{1}{modulos/exp/IU/SettingsEventos}{IU-E05}%
        {Configuración de eventos con experiencia}

\begin{comment}
\subsubsection{Elementos Relevantes}

    \begin{itemize}
    \item {\bf Lorem ipsum}
        ...
    \end{itemize}

\subsubsection{Acciones relevantes}

    \begin{itemize}
    \item {\bf Lorem ipsum}
        ...
    \end{itemize}
\end{comment}

\clearpage
  % Configuraciones de Eventos
    
\subsection{IU-E05a Eventos con experiencia desactivados}

 Descripción ...

    \IUfig{1}{modulos/exp/IU/SettingsEventsDisabled}{IU-E05a}%
    {Eventos con experiencia desactivados}

\subsubsection{Elementos Relevantes}

    \begin{itemize}
    \item {\bf Lorem ipsum}
        ...
    \end{itemize}

\subsubsection{Acciones relevantes}

    \begin{itemize}
    \item {\bf Lorem ipsum}
        ...
    \end{itemize}

\clearpage
 % Configuraciones Events desativados
    
\subsubsection{IU-E06 Creación de un curso gamificado}

 Descripción ...

    \IUfig{1}{modulos/exp/IU/CursoCreate}{IU-E06}{Creación curso gamificado}

\subsubsection{Elementos Relevantes}

    \begin{itemize}
    \item {\bf Lorem ipsum}
        ...
    \end{itemize}

\subsubsection{Acciones relevantes}

    \begin{itemize}
    \item {\bf Lorem ipsum}
        ...
    \end{itemize}

\clearpage
  % Curso Gamificado
    
\subsubsection{IU-E06a Curso Gamificado}

 Descripción ...

    \IUfig{1}{modulos/exp/IU/CursoView}{IU-E06a}{Curso Gamificado}

\subsubsection{Elementos Relevantes}

    \begin{itemize}
    \item {\bf Lorem ipsum}
        ...
    \end{itemize}

\subsubsection{Acciones relevantes}

    \begin{itemize}
    \item {\bf Lorem ipsum}
        ...
    \end{itemize}

\clearpage
  % Curso Gamificado
    
\subsection{IU-E06b Curso gamificado en edición}

 Descripción ...

    \IUfig{1}{modulos/exp/IU/CursoEdit}{IU-E06b}{Curso gamificado en edición}

\subsubsection{Elementos Relevantes}

    \begin{itemize}
    \item {\bf Lorem ipsum}
        ...
    \end{itemize}

\subsubsection{Acciones relevantes}

    \begin{itemize}
    \item {\bf Lorem ipsum}
        ...
    \end{itemize}

\clearpage
 % Curso Gamificado (editing)

    %
\subsubsection{IU-E04: Configuraciones del sistema de experiencia}

 Descripción ...

    \IUfig{1}{modulos/exp/IU/settingsEsquema}{IU-E04}{%
        Configuraciones del sistema de experiencia}

\subsubsection{Elementos Relevantes}

    \begin{itemize}
    \item {\bf Lorem ipsum}
        ...
    \end{itemize}

\subsubsection{Acciones relevantes}

    \begin{itemize}
    \item {\bf Lorem ipsum}
        ...
    \end{itemize}

\clearpage
 % Configuraciones de Comportamiento
    %
\subsubsection{IU-E05 Configuración de eventos con experiencia}

% Descripción ...

    \IUfig{1}{modulos/exp/IU/SettingsEventos}{IU-E05}%
        {Configuración de eventos con experiencia}

\begin{comment}
\subsubsection{Elementos Relevantes}

    \begin{itemize}
    \item {\bf Lorem ipsum}
        ...
    \end{itemize}

\subsubsection{Acciones relevantes}

    \begin{itemize}
    \item {\bf Lorem ipsum}
        ...
    \end{itemize}
\end{comment}

\clearpage
 % Configuraciones de eventos

\subsubsection{Diseño de plugins} % TODO CHANGE FOR INPUTS
\subsubsection{Diagrama de componentes} % TODO CHANGE FOR INPUTS
\subsubsection{Diagrama de clases} % TODO CHANGE FOR INPUTS

\end{comment} % >>>>>>> d14b3c5df86665f3cc11f090d68bc3001772b4f5 ============asdsds

\subsection{Pruebas}

    
A continuación se enlistan los casos de prueba identificados
correspondientes a cada uno de los casos de uso especificados. Los casos
de prueba listados a continuación son de dos tipos, los correctos e
incorrectos identificados por los prefijos CPC y CPI respectivamente.

\subsubsection{\refElem{CU-P01}}

\begin{itemize}
  \TestCase{CPC-P01-1}{Ver perfil con objetos ya elegidos}
  \TestCase{CPC-P01-2}{Ver perfil con valores por defecto}
  \TestCase{CPC-P01-3}{Usuario no registrado como usuario gamificado}
\end{itemize}


\subsubsection{\refElem{CU-P02}}

\begin{itemize}
  \TestCase{CPC-P02-1}{Modificar perfil con objetos desbloqueados}
  \TestCase{CPI-P02-2}{Intentar modificar pefil con objetos no desbloqueados}
  \TestCase{CPI-P02-3}{Intentar modificar perfil con 2 objetos de un mismo tipo}
\end{itemize}


\subsubsection{\refElem{CU-P03}}

\begin{itemize}
  \TestCase{CPC-P03-1}{Eliminar datos del complemento}
\end{itemize}


\subsubsection{\refElem{CU-P04}}

\begin{itemize}
  \TestCase{CPC-P04-1}{Activar complemento}
  \TestCase{CPI-P04-2}{Activar complemento sin tener el complemento de la tienda}
  \TestCase{CPI-P04-3}{Ingresar valores incorrectos}
\end{itemize}


\subsubsection{\refElem{CU-P05}}

\begin{itemize}
  \TestCase{CPC-P05-1}{Desactivar complemento}
  \TestCase{CPI-P05-2}{Desactivar complemento sin tener el complemento de la tienda}
  \TestCase{CPI-P05-3}{Ingresar valores incorrectos}
\end{itemize}



\begin{comment}
\section{Submódulos}

\subsubsection{Esquema configurable}

 Se quiere que el administrador de la página pueda configurar:
 \begin{quote}
 \begin{itemize}
    \item{La ''\hyperref[table:METerminosExperiencia1]{experiencia del nivel}'' del nivel 1.}
    \item {El tipo de incremento.}
    \item {La cantidad de los puntos de experiencia en el incremento.}
    \item {La ''\hyperref[table:METerminosExperiencia1]{experiencia otorgada}'' que da resolver cualquier actividad.}
 \end{itemize}
 \end{quote}

\subsection{Submódulo de Niveles}

 Presenta a los estudiantes su progreso utilizando un sistema de niveles que se van alcanzado
 obteniendo puntos de experiencia. Al alcanzar un nuevo nivel la barra que muestra la
 cantidad de experiencia del nivel se actualizará.
 % y cada vez que se alcanza un nivel, los puntos de experiencia se regresan a cero.

    \begin{quote}
    \begin{description}
    \item[Objetivo] \hfill\\
        Mostrar a los estudiantes el nivel actual de experiencia que tienen y el avance que tienen de ese mismo nivel.
        %Mostrar el nivel actual que tienen los estudiantes, así como el avance que tienen en ese mismo nivel.

        %Proveer información al estudiante que indique la cantidad de tiempo y esfuerzo que le ha dedicado a la plataforma.

    \item[Principios a los que brinda soporte:] \hfill
        \begin{itemize}
            \item 2 \principioII
            \item 6 \principioVI
        \end{itemize}
    \end{description}
    \end{quote}

%\begin{comment}%

\subsection{Submódulo de Barra de Progreso}

Muestra al estudiante el progreso que lleva en un curso usando un valor de 0\% a 100\% dependiendo de los ejercicios que haya hecho del curso o del tiempo que haya transcurrido.

    \begin{quote}
    \begin{description}
        \item[Objetivo] \hfill\\
            Proveer información al estudiante que indique el tiempo y esfuerzo que le ha dedicado a un curso, así como el que le falta por dedicar.

        \item[Principios a los que brinda soporte:] \hfill
        \begin{itemize}
            \item 2 \principioII
        \end{itemize}
    \end{description}
    \end{quote}
%\end{comment}%

\subsection{Diagrama de Clases}

    En la figura \ref{fig:classesXP} se muestra el diagrama de clases, los archivos {\it lib, events, settings, version} y los {\it módulos AMD} son representados mediante el uso de clases. Para facilitar la lectura del diagrama se representa a moodle como un paquete completo, el cual lee los distintos archivos y clases que requiere el plugin para funcionar.

%    \addfigure{1}{diagrams/classesExp}{fig:classesXP}{Diagrama de clases del Módulo de Experiencia}
\clearpage

\subsection{Diagrama de componentes}

    En la figura \ref{fig:bloques1} se muestra el diagrama de componentes del Módulo de experiencia que contiene como interactúa el Módulo con la plataforma Moodle.

%    \addfigure{1}{diagrams/bloques1}{fig:bloques1}{Diagrama de componentes del Módulo de Experiencia}

\clearpage
\subsection*{DS-E2: Crear curso con experiencia}

    Para diseñar la forma en que se ejecuta el caso de uso CU-E2, se tomó en consideración el flujo normal de eventos emitidos cuando se crea un curso en moodle. Los eventos emitidos en orden cronológico son {\it course\_created}, {\it course\_section\_created} y {\it enrol\_instance\_created}.\\

    \noindent En la figura \ref{ds:e2} se detalla la interacción entre el core de moodle, los eventos emitidos, y las clases del plugin {\bf Format Gamedle}.

\end{comment}
