
\chapter{Análisis}

 Este apartado contiene el análisis requerido para la elaboración de módulo de experiencia,
 contiene la especificación del alcance de este módulo, la descripción de las funcionalidades
 a desarrollar, la reglas de negocio que rigen el comportamiento del módulo, y por último la
 especificación de los casos de uso a los que brinda soporte.

\section{Esquema de experiencia}

 El esquema de experiencia le proporciona al \refElem{aAdministrador} y a los \refElem[Profesores]%
 {aProfesor} un mecanismo mediante el cual pueden configurar la forma en que se obtienen los puntos
 de experiencia, la cantidad a otorgar, el número de puntos de cada nivel y finalmente la
 visualización del nivel y de los puntos de cada usuario.\\

 \noindent
 Las configuraciones fueron organizadas en dos grupos: las {\it configuraciones a
 nivel plataforma} las cuales definen valores de forma global, y las {\it configuraciones a nivel
 curso} las cuales definen valores para un curso en específico. A continuación se describen cada
 uno de los grupos.\\

\section{Submódulo de niveles}

\section{Funcionalidades}

\section{Reglas de negocio}

 En esta sección se especifican todas las reglas de negocio relevantes para el módulo de
 experiencia. Las reglas de negocio que establece moodle son diferenciadas por tener la letra {\it M}
 antecediendo al número consecutivo en su identificador.

    
\subsection{Entidades de moodle}

Debido a que moodle cuenta con más de 400 tablas en su versión 3.5, se opta
por mostrar 2 subconjuntos que muestren las tablas que se utilizan para el proyecto.\\

\noindent El primer subconjunto es aquel que explica la forma en que moodle implementa los cursos, 
secciones de curso, actividades, usuarios y roles (el cual se presenta en la figura \ref{fig:BD-ER-M1}), 
mientras que el segundo conjunto muestra como moodle maneja toda la 
estructura de las preguntas creadas por el profesor y respondidas por el usuario
(el cual se presenta en la figura \ref{fig:BD-ER-M2}).  



En lugar de describir y mostrar cada uno de los campos de cada una de las entidades de moodle que se contemplan,
lo que se quiere lograr con ambos esquemas (\ref{fig:BD-ER-M1}) y \ref{fig:BD-ER-M2}))
es expresar la idea general del comportamiento.

\clearpage
\addfigure{0.7}{analisis/diagrams/db_module_structure}{fig:BD-ER-M1}{Esquema de la base de datos de moodle 'Cursos'}


\noindent Utilizando la figura \ref{fig:BD-ER-M1}, se obtuvieron las siguientes reglas y caracteristicas que contiene moodle respecto a los usuarios en un curso y a la estructura de los cursos.
\begin{enumerate}
    \item Un usuario -{\it mdl\_user}- tiene un rol -{\it mdl\_role}- en un cierto contexto -{\it mdl\_context}-, cuyo  '{\it context\_level}' sea igual a cincuenta(50).
    \item Si el contexto '{\it context\_level}' es de 50, el atributo '{\it instance\_id}' hace referencia al atributo '{\it id}' de un curso -{\it mdl\_course}-.
    \item El curso -{\it mdl\_course}- tiene varias secciones -{\it mdl\_course\_sections}-.
    \item Cada seccion -{\it mdl\_course\_sections}- tiene varias actividades -{\it mdl\_course\_modules}- que pertenecen a un tipo de actividad -{\it mdl\_modules}-.
    \item Por cada registro en tipo de actividad -{\it mdl\_modules}-, se tiene una entidad que lleva el mismo nombre.
    \item El atributo '{\it instance\_id}' de una actividad  -{\it mdl\_course\_modules}- apunta a diferentes entidades. La entidad a la que apunta depende del nombre del tipo de actividad -{\it mdl\_modules}-.
    \item Un usuario -{\it mdl\_user}- se inscribió -{\it mdl\_user\_enrolments}- a un curso -{\it mdl\_course}-, por medio de un formato soportado de inscripción -{\it mdl\_enrol}-.
\end{enumerate}

\clearpage

 \addfigure{0.7}{analisis/diagrams/db_module_questions}{fig:BD-ER-M2}{Esquema de la base de datos de moodle 'Preguntas' }



\noindent Utilizando la figura \ref{fig:BD-ER-M2}, se obtuvieron las siguientes reglas y caracteristicas que contiene moodle respecto a las preguntas.
\begin{enumerate}
    \item Las preguntas -{\it mdl\_question}- tienen versiones -{\it mdl\_question\_attempts}-.
    \item Una pregunta -{\it mdl\_question}- pertenece a un banco de preguntas -{\it mdl\_question\_categories}-.
    \item La versión de una pregunta -{\it mdl\_question\_attempts}- es contestada -{\it mdl\_question\_usages}- en un determinado contexto -{\it mdl\_context}-.
    \item Un usuario -{\it mdl\_user}- responde una versión de una pregunta -{\it mdl\_question\_attempt\_stepts}-.
    \item El responder una versión de una pregunta -{\it mdl\_question\_attempt\_stepts}- conlleva pasos -{\it mdl\_question\_attempt\_stept\_data}-, los cuales son: cómo se muestra, si ya se terminó de responder y qué se respondió.
\end{enumerate}


 A continuación se presenta la especificación de las entidades del esquema de base
 de datos de moodle que son relevantes para el desarrollo de los módulos y submódulos
 de proyecto.

    \begin{cdtEntidad}{mdl-config-plugins}{Configuración de Plugin}{%
    Es una tabla del núcleo de moodle que almacena todas las configuraciones globales
    relacionadas a los plugins instalados, al iniciar moodle las configuraciones de los
    plugins instalados y habilitados se cargan en memoria.}

	    \brAttr{id}{Id}{tInt}{%
	        Es el dígito que representa el identificador único para una configuración
            específica de un plugin.\par

            \it Restricciones:
            \refElem{tPrimaryKey},
            \refElem{tAutoIncrement}.
        }

        \brAttr{plugin}{Plugin}{tVarchar}{%
            Cadena de caracteres del nombre identificador del plugin al cual pertenece
            la configuración.\par

            \it Restricciones:
            \refElem{tRequired},
            \refElem{tRange} (0,100),
            \refElem{tUniqueKey}
        }

        \brAttr{name}{Nombre}{tVarchar}{%
            Cadena de caracteres que representa el nombre de la configuración de un
            plugin en específico.\par

            \it Restricciones:
            \refElem{tUniqueKey},
            \refElem{tRange} (0,100),
            \refElem{tRequired}
        }

        \brAttr{value}{Valor}{tVarchar}{%
            Cadena que almacena el valor de una configuración perteneciente a alguno
            de los plugins instalados.\par

            \it Restricciones:
            \refElem{tRange} (0,4294967295),
            \refElem{tRequired}
        }
    \end{cdtEntidad}\schemeName{config\_plugins}

    \begin{cdtEntidad}{mdl-user}{Usuario de moodle}{%
    Es una tabla del núcleo de moodle que contiene toda la información que se
    almacena de los usuarios en la plataforma, independientemente del rol que
    estos contenga, esta relación contiene más de 53 atributos, sin embargo solo
    se detallan aquellos relevantes.}

	    \brAttr{id}{Id}{tInt}{%
	        Es el dígito que representa el identificador único para cada uno
            de los usuarios en moodle.\par

            \it Restricciones:
            \refElem{tPrimaryKey},
            \refElem{tAutoIncrement}.
        }
	    \brAttr{username}{nombre de usuario}{tVarchar}{%
	        .\par

            \it Restricciones:
            \refElem{tRequired},
            \refElem{tLength} 0-100
        }
	    \brAttr{password}{contraseña}{tVarchar}{%
	        .\par

            \it Restricciones:
            \refElem{tRequired},
            \refElem{tLength} 0-255.
        }
	    \brAttr{firstname}{nombre}{tVarchar}{%
	        .\par

            \it Restricciones:
            \refElem{tRequired},
            \refElem{tLength} 0-100
        }
	    \brAttr{lastname}{apellido}{tVarchar}{%
	        .\par

            \it Restricciones:
            \refElem{tRequired},
            \refElem{tLength} 0-100
        }
	    \brAttr{email}{correo}{tVarchar}{%
	        .\par

            \it Restricciones:
            \refElem{tRequired},
            \refElem{tLength} 0-100
        }
	    \brAttr{lastaccess}{último registro}{tInt}{%
	        .\par

            \it Restricciones:
            \refElem{tRequired},
            \refElem{tLength} 10
        }
	    \brAttr{city}{ciudad}{tVarchar}{%
	        .\par

            \it Restricciones:
            \refElem{tRequired},
            \refElem{tLength} 0-120
        }
	    \brAttr{country}{pais}{tVarchar}{%
	        .\par

            \it Restricciones:
            \refElem{tRequired},
            \refElem{tLength} 2
        }

    \end{cdtEntidad}\schemeName{user}

    \begin{cdtEntidad}{mdl-course}{Curso de moodle}{%
    Es una tabla del núcleo de moodle que contiene la información principal de cada 
    curso registrado en moodle. Esta entidad contiene 31 atributos, a continuación se
    detallan los atributos relevantes para la especificación de este proyecto.}

	    \brAttr{id}{Id}{tInt}{%
	        Es el dígito que representa al identificador único para cada uno
            de los cursos en moodle.\par

            \it Restricciones:
            \refElem{tPrimaryKey},
            \refElem{tAutoIncrement}.
        }

	    \brAttr{format}{formato}{tVarchar}{%
	        Es el dígito que representa al identificador único para cada uno
            de los cursos en moodle.\par

            \it Restricciones:
            \refElem{tRequired}.
            \refElem{tDefault} topics,
            \refElem{tLength} 0-21.
        }

	    \brAttr{fullname}{nombre completo}{tVarchar}{%
	        Es el nombre completo que se le asigna al curso.\par

            \it Restricciones:
            \refElem{tRequired}.
            \refElem{tLength} 0-21.
        }

	    \brAttr{shortname}{nombre corto}{tVarchar}{%
            Es el nombre corto que se le asigna al curso.\par

            \it Restricciones:
            \refElem{tRequired}.
            \refElem{tLength} 0-21.
        }

    \end{cdtEntidad}\schemeName{course}

    \begin{cdtEntidad}{mdl-course-sections}{Secciones del curso de moodle}{%
    }
	    \brAttr{id}{Id}{tInt}{%
	        Es el dígito que representa al identificador único para cada seccion
            de los cursos en moodle.\par

            \it Restricciones:
            \refElem{tPrimaryKey},
            \refElem{tAutoIncrement}.
        }
    \end{cdtEntidad}\schemeName{course\_sections}

    \begin{cdtEntidad}{mdl-course-format-options}{Opciones del formato del curso}{%
    }

	    \brAttr{id}{Id}{tInt}{%
	        Es el dígito que representa al identificador único para cada uno
            de los cursos en moodle.\par

            \it Restricciones:
            \refElem{tPrimaryKey},
            \refElem{tAutoIncrement}.
        }

	    \brAttr{courseid}{Id}{tInt}{%
	        Es el dígito que representa al identificador único para cada uno
            de los cursos en moodle.\par

            \it Restricciones:
            \refElem{tForeignKey},
            \refElem{tRequired}
        }

	    \brAttr{format}{formato}{tVarchar}{%
	        Es el dígito que representa al identificador único para cada uno
            de los cursos en moodle.\par

            \it Restricciones:
            \refElem{tRequired}.
            \refElem{tDefault} topics,
            \refElem{tLength} 0-21.
        }

	    \brAttr{name}{opcion}{tVarchar}{%
	        Es el dígito que representa al identificador único para cada uno
            de los cursos en moodle.\par

            \it Restricciones:
            \refElem{tPrimaryKey},
            \refElem{tLength} 0-100
        }

	    \brAttr{value}{valor}{tVarchar}{%
	        Es el dígito que representa al identificador único para cada uno
            de los cursos en moodle.\par

            \it Restricciones:
            \refElem{tRequired}
        }

    \end{cdtEntidad}\schemeName{course\_format\_options}

    \begin{cdtEntidad}{mdl-course-category}{Categoria de curso}{%
    .}
    \end{cdtEntidad}\schemeName{course\_category}

    \begin{cdtEntidad}{Plugin}{Plugin}{%
    La forma en que moodle obtiene información acerca de los plugins es analizando
    los archivos internos de cada uno, a pesar de que los plugins no forman parte
    del esquema de base de datos, si forman parte del modelo de información que
    utiliza Moodle.}

	    \brAttr{componente}{Componente}{tVarchar}{%
	        Cadena compuesta por el tipo de plugin y el nombre del mismo, que
            representa a la clase principal del plugin que contiene los métodos
            principales del plugin.\par

            \it Restricciones: Ninguna
        }

	    \brAttr{pluginname}{Nombre}{tVarchar}{%
	        Es el nombre del plugin obtenido de los archivos de
            internacionalización presentes en el plugin, el valor de esta cadena
            depende del lenguaje seleccionado en moodle.\par

            \it Restricciones: Ninguna
        }

	    \brAttr{fullpath}{Ruta absoluta}{tPath}{%
	        La ruta absoluta de un plugin denota la ubicación del plugin en el
            sistema de archivos, esta ruta está compuesta por la ruta absoluta
            de la instalación de moodle, la carpeta correspondiente al tipo del
            plugin y el nombre del plugin.\par

            \it Restricciones: Formato ``/path/to/moodle/plugintype/pluginname''
        }

	    \brAttr{path}{Ruta relativa}{tPath}{%
	        La ruta relativa denota la ubicación del plugin dentro de la carpeta 
            donde se encuentran los archivos de moodle, esta ruta está compuesta
            por la carpeta correspondiente al tipo del plugin y el nombre del
            plugin.\par

            \it Restricciones: Formato ``plugintype/pluginname''
        }

	    \brAttr{version}{Versión}{tVersion}{%
	        Numero entero de longitud de 10 dígitos que representa la versión del 
            plugin.\par

            \it Restricciones: Ninguna adicional al tipo de dato
        }

	    \brAttr{moodle}{Versión de Moodle}{tVersion}{%
	        Número entero de longitud de 10 dígitos que representa la versión de 
            moodle en la que se puede instalar el plugin.\par

            \it Restricciones: Ninguna adicional al tipo de dato
        }

        \brAttr{dependencies}{Dependencias}{tObject}{%
            Objeto que almacena un conjunto de claves con sus respectivos valores, 
            donde cada clave representa el nombre del componente del plugin y el valor 
            es la \refElem{Plugin.version} requerida del mismo.

            \it Restricciones: Ninguna
        }

        \brAttr{icon}{ícono}{tImage}{%
            Imagen para el ícono del plugin, debe estar contenida en el directorio
            {\it pix/} del plugin y tener como nombre {\it icon.png} o {\it icon.svg},
            moodle recomienda tener ambos archivos por si los navegadores no soportan
            algun tipo de archivo \cite{moodlePluginfiles}.\par 

            \it Restricciones: El nombre debe ser icon con extensiones png o svg
        }

    \end{cdtEntidad}
 % Archivo de plugin
    
\begin{BusinessRule}[%
Autor/Daniel Isai Ortega Zúñiga,%
Version/0.1,%
Estado/revision]%
%
{BR-E01}{Restricciones del nombre de la imagen de los archivos}
 % El archivo de instalación debe ser un archivo ZIP, el cual debe contener
 % exactamente un directorio que coincida con el nombre del plugin.
     \BRitem[control]{Revisor}{Sin asignar.}

 \BRsection[control]{Atributos}
    
    \BRitem[admin]{Clase}{\bcCondition}%
    %\BRitem[admin]{Clase}{\bcIntegridad}%
    %\BRitem[admin]{Clase}{\bcAutorizacion}%
    %\BRitem[admin]{Clase}{\bcDerivacion}%
        
    \BRitem[admin]{Tipo}{\btEnabler}%
    %\BRitem[admin]{Tipo}{\btTimer}%
    %\BRitem[admin]{Tipo}{\btExecutive}%
        
    \BRitem[admin]{Nivel}{\blControlling}
    %\BRitem[admin]{Nivel}{\blInfluencing}
    
    \BRitem{Descripción}{%
        El archivo seleccionado para la representación visual de los niveles debe
        ser una imagen con las extensiones {\it``png''} o {\it''jgp}, además el
        nombre del archivo que será subido no debe tener el nombre {\it``icon.png''}
        ya que posiblemente colisionaría con el \refElem{Plugin.icon} del plugin.
        % debido a que se el directorio donde se guardará será el directorio 
        % para almacenar las imágenes del plugin.
    }

    \BRitem{Ejemplo positivo}{\hfill\par%
        \begin{itemize}
        \item El archivo seleccionado para ser la imagen de los niveles tiene
              como nombre {\it``logotipo''} con la extensión {\it png}.

        \item El archivo seleccionado para ser la imagen de los niveles tiene
              como nombre {\it``nivel''} con la extensión {\it jpg}
        \end{itemize}
    }

    \BRitem{Ejemplo negativo}{\hfill\par%
        \begin{itemize}
        \item El archivo seleccionado para ser la imagen de los niveles tiene
              como nombre {\it``icon''} con la extensión {\it png}.

        \item El archivo seleccionado para ser la imagen de los niveles tiene
              como nombre {\it``documento''} con la extensión {\it doc}.
        \end{itemize}
    }% 
    
 \end{BusinessRule}

 % Restricciones sobre de imagen del nivel.
    
\begin{BusinessRule}[%
Autor/Daniel Isai Ortega Zúñiga,%
Version/0.1,%
Estado/revision]%
%
{BR-E02}{Permanencia de los puntos de experiencia}
    \BRitem[control]{Revisor}{Sin asignar.}

 \BRsection[control]{Atributos}

    %\BRitem[admin]{Clase}{\bcCondition}%
    \BRitem[admin]{Clase}{\bcIntegridad}%
    %\BRitem[admin]{Clase}{\bcAutorizacion}%
    %\BRitem[admin]{Clase}{\bcDerivacion}%

    \BRitem[admin]{Tipo}{\btEnabler}%
    %\BRitem[admin]{Tipo}{\btTimer}%
    %\BRitem[admin]{Tipo}{\btExecutive}%

    %\BRitem[admin]{Nivel}{\blControlling}
    \BRitem[admin]{Nivel}{\blInfluencing}

    \BRitem{Descripción}{%
    Los puntos de experiencia una vez que son obtenidos no pueden ser quitados bajo
    ninguna condición exceptuando únicamente la acción de eliminación de un usuario y
    la desinstalación de los plugins que formen parte del esquema de experiencia.}

    \BRitem{Ejemplo positivo}{\hfill\par%
        \begin{itemize}
        \item Un usuario conforme va completando las secciones de los cursos obtiene
              puntos de experiencia, si el curso es eliminado entonces el usuario
              deberá permanecer con los puntos de experiencia obtenidos.

        \item Un usuario con 300 puntos de experiencia es eliminado del sitio, y en
              consecuencia se eliminan sus puntos de experiencia.
        \end{itemize}
    }

    \BRitem{Ejemplo negativo}{\hfill\par%
        \begin{itemize}
        \item Un curso es eliminado y a todos los estudiantes se les resta de sus
              puntos de experiencia la cantidad de experiencia obtenida durante el
              curso.
        \end{itemize}
    }%

 \end{BusinessRule}
 % Permanencia del nivel de experiencia.
    
\begin{BusinessRule}[%
Autor/Daniel Isai Ortega Zúñiga,%
Version/0.1,%
Estado/revision]%
%
{BR-E03}{Tipos de Incremento}
    \BRitem[control]{Revisor}{Sin asignar.}

 \BRsection[control]{Atributos}
    
    \BRitem[admin]{Clase}{\bcCondition}%
    %\BRitem[admin]{Clase}{\bcIntegridad}%
    %\BRitem[admin]{Clase}{\bcAutorizacion}%
    %\BRitem[admin]{Clase}{\bcDerivacion}%
        
    \BRitem[admin]{Tipo}{\btEnabler}%
    %\BRitem[admin]{Tipo}{\btTimer}%
    %\BRitem[admin]{Tipo}{\btExecutive}%
        
    \BRitem[admin]{Nivel}{\blControlling}
    %\BRitem[admin]{Nivel}{\blInfluencing}
    
    \BRitem{Descripción}{%
    Cuando se modifiquen el \refElem{xp-scheme-settings} o la \refElem{levelXP} de las
    \refElem{xp-scheme-settings}
    }

    \BRitem{Ejemplo positivo}{\hfill\par%
        \begin{itemize}
        \item ...
        \end{itemize}
    }

    \BRitem{Ejemplo negativo}{\hfill\par%
        \begin{itemize}
        \item ...
        \end{itemize}
    }% 
    
 \end{BusinessRule}
 % Tipos de incremento
    \begin{BusinessRule}[%
Autor/Daniel Isai Ortega Zúñiga,%
Version/0.1,%
Estado/revision]%
%
{BR-E04}{Calculo de experiencia del nivel con incremento porcentual}
    \BRitem[control]{Revisor}{Sin asignar.}

 \BRsection[control]{Atributos}
    
    \BRitem[admin]{Clase}{\bcCondition}%
    %\BRitem[admin]{Clase}{\bcIntegridad}%
    %\BRitem[admin]{Clase}{\bcAutorizacion}%
    %\BRitem[admin]{Clase}{\bcDerivacion}%
        
    \BRitem[admin]{Tipo}{\btEnabler}%
    %\BRitem[admin]{Tipo}{\btTimer}%
    %\BRitem[admin]{Tipo}{\btExecutive}%
        
    \BRitem[admin]{Nivel}{\blControlling}
    %\BRitem[admin]{Nivel}{\blInfluencing}
    
    \BRitem{Descripción}{%
        El calculo para obtener la experiencia del nivel $i$ uando el tipo de
        incremento es porcentual está dado por la siguiente fórmula: Sea {\it exp()}
        la función que optiene la experiencia de un nivel en específico, sea tambien
        $i$ el nivel del cual se calcula la experiencia, sea $inc$ el factor de
        incremento de nivel a nivel, y finalmente sea $round()$ una función de
        redondeo a números enteros, entonces:

            $$ exp(i) = round( exp(1) * (inc)^{(i-1)})$$
    }

%   \BRitem{Sentencia}{%
%       Si $fecha$ 
%   }%

    \BRitem{Ejemplo positivo}{\hfill\par%
        \begin{itemize}
        \item La experiencia requerida para superar el nivel 1 es de 2000 puntos y el
              factor de incremento entre los niveles es 1.1, entonces la experiencia
              requerida para pasar el nivel 5 es de 2928 puntos.
        \end{itemize}
    }

    \BRitem{Ejemplo negativo}{\hfill\par%
        \begin{itemize}
        \item La experinecia requerida para superar el nivel 1 es de 2000 puntos y el
              factor de incremento entre los niveles es 1.1, entonces la experiencia
              requerida para pasar el nivel 5 es de 2300 puntos.
        \end{itemize}
    }% 
    
\end{BusinessRule}
 % Incremento porcentual
    \begin{BusinessRule}[%
Autor/Daniel Isai Ortega Zúñiga,%
Version/0.1,%
Estado/revision]%
%
{BR-E05}{Cálculo de experiencia del nivel con incremento linea} % Cuando están iniciados
    \BRitem[control]{Revisor}{Sin asignar.}

 \BRsection[control]{Atributos}
    
    \BRitem[admin]{Clase}{\bcCondition}%
    %\BRitem[admin]{Clase}{\bcIntegridad}%
    %\BRitem[admin]{Clase}{\bcAutorizacion}%
    %\BRitem[admin]{Clase}{\bcDerivacion}%
        
    \BRitem[admin]{Tipo}{\btEnabler}%
    %\BRitem[admin]{Tipo}{\btTimer}%
    %\BRitem[admin]{Tipo}{\btExecutive}%
        
    \BRitem[admin]{Nivel}{\blControlling}
    %\BRitem[admin]{Nivel}{\blInfluencing}
    
    \BRitem{Descripción}{%
    }

%   \BRitem{Sentencia}{%
%       Si $fecha$ 
%   }%

    \BRitem{Ejemplo positivo}{\hfill\par%
        \begin{itemize}
        \item ...
        \end{itemize}
    }

    \BRitem{Ejemplo negativo}{\hfill\par%
        \begin{itemize}
        \item ...
        \end{itemize}
    }% 
    
\end{BusinessRule}
 % Incremento lineal
    % INPUT: Cursos Igualitarios.
    % INPUT: Otorgar experiencia
    % INPUT: Administración de experiencia en el curso

\clearpage
\section{Casos de uso}

 En este apartado se especifican todos los casos de usos contemplados para el módulo de
 experiencia, para cada caso de uso se especifica su tabla de atributos la cual indica que casos
 de prueba deberán ejecutarse correctamente para corrobarar la completitud del caso de uso.

\subsection{Diagrama de casos de uso}

 En la figura \ref{exp:usecases} se detalla el diagrama de casos de uso correspondiente al módulo
 de experiencia. Los casos de uso de moodle (en color blanco) son modelados como casos de uso
 abstractos, mientras que los casos de uso del módulo de experiencia son diferenciados por el
 color azul, en total el desarrollo de este módulo consiste en 13 casos de uso principales.

    \addfigure{0.9}{modulos/UseCases}{exp:usecases}{%
        Diagrama de casos de uso del módulo de experiencia}

 \noindent
 Debido a que los plugins a desarrollar son elementos opcionales para Moodle, solo se puede
 acceder a los casos de uso del módulo de experiencia a través de puntos de extensión de los
 casos de uso de moodle. Por otra parte los casos de uso que serán documentados en esta sección
 serán los del módulo de experiencia debido a que Moodle proporciona en su página oficial, guías
 e instructivos como documentación de las funcionalidades que brinda.

    % CASOS DE USO DE MOODLE
    
% \ucstEnEdicion     Al terminar una revisión/aprobación con observaciones 
%                    y al inicio del CU.
%
% \ucstEnRevision    Al terminar la edición del CU (version += 0.1).
% \ucstEnAprobacion  Al pasar la revision sin observaciones.
% \ucstAprobado      Al ser aprobado por el usuario (version += 1.0)

\begin{UseCase}[%
Autor/Daniel Ortega,%
Version/0.1,%
Estado/\ucstEnEdicion]%
%
{CU-M01}{Acceder a la administración del sitio}{% TODO; Deberia se Instalar/Actualizar ???
%
 Permite al \refElem{aAdministrador} acceder a la pantalla \refElem{IU-M01} realizar las
 distintas tareas que incluye la administración del sitio de moodle. Esta caso de uso es realizado
 debido a que es requerido para la ejecución de la mayoría de los casos de uso cuyo actor es el
 administrador.}

	\UCitem[control]{Revisor}{ Sin asignar }
	\UCitem[control]{Último cambio}{ 14/OCT/19 }

 \UCsection{Atributos}

    \UCitem{Actor(es)}{%
        \refElem{aAdministrador}
    }

	\UCitem{Propósito}{%
        Permitir al administrador acceder a la administración del moodle que administra.
	}
	
	\UCitem{Entradas}{-}

	\UCitem{Origen}{-}

	\UCitem{Salidas}{-}

	\UCitems{Destino}{%
        \refElem{IU-M01}
	}
	
	\UCitem{Precondiciones}{-}

	\UCitem{Postcondiciones}{-}

	\UCitem{Reglas de negocio}{-}

	\UCitem{Errores}{-}

	% \UCitem{Viene de}{% Indicar si el Caso de uso es primario o se extiende de otro. La mayoría se 
					  % extienden de Login.
		% EJEMPLO: \refIdElem{PY-CU1} o Caso de uso primario.
	% 	\TODO Especificar.
	% }	

 \UCsection[design]{Datos de Diseño}

	\UCitem[design]{Casos de Prueba}{%
        Incluidos en la ejecución de los casos de uso que incluyen a este caso de uso
    }

 \UCsection[admin]{Datos de Administración de Requerimiento}

	\UCitem[admin]{Observaciones}{-}

\end{UseCase}

\subsubsection{Trayectorias del caso de uso}

\begin{UCtrayectoria}%
%
    \Actor Presiona el botón \IUMenu de la pantalla \refElem{IU-M00}
    \Sistema Despliega el menú de navegación lateral.

    \Actor Selecciona la opción {\bf \IUAdminSitio Administración del sitio}
    \Sistema Carga la pantalla \refElem{IU-M01} con la pestaña de administración del
             sitio preseleccionada.

    \Actor Selecciona la pestaña {\bf plugins}
    \Sistema Carga la pantalla \refElem{IU-M01a}

\end{UCtrayectoria}

\subsubsection{Puntos de extensión}

\UCExtensionPoint{Instalación de un plugin}{%

    El \refElem{aAdministrador} desea extender la funcionalidad
    de moodle mediante la instalación de plugins.%
    }{Al inicio la trayectoria principal}{\refElem{CU-E01}}

\UCExtensionPoint{Configuraciones generales del módulo de experiencia}{%

    El \refElem{aAdministrador} desea cambiar las configuraciones
    generales del módulo de experiencia.%
    }{Al inicio la trayectoria principal}{\refElem{CU-E02}}

\UCExtensionPoint{Configuraciones visuales del módulo de experiencia}{%

    El \refElem{aAdministrador} desea establecer las configuraciones
    de la visualización de los niveles del módulo de experiencia.%
    }{Al inicio la trayectoria principal}{\refElem{CU-E02-1}}

\UCExtensionPoint{Configuraciones de comportamiento del módulo de experiencia}{%

    El \refElem{aAdministrador} desea establecer el comportamiento del
    sistema de experiencia que incluye el módulo de experiencia.%
    }{Al inicio la trayectoria principal}{\refElem{CU-E02-2}}

\UCExtensionPoint{Configuraciones de eventos del módulo de experiencia}{%

    El \refElem{aAdministrador} desea establecer la cantidad de experiencia
    que brindarán los eventos que soporta el módulo de experiencia.%
    }{Al inicio la trayectoria principal}{\refElem{CU-E02-3}}

\UCExtensionPoint{Desinstalación de un plugin}{%

    El \refElem{aAdministrador} desea desinstalar un plugin en
    moodle debido a que ya no requiere de las funcionalidades
    que este brinda%
    }{Al inicio la trayectoria principal}{\refElem{CU-E03}}

 % Acceder a la administración del sitio

    % MODULO DE EXPERIENCIA
    
% \ucstEnEdicion     Al terminar una revisión/aprobación con observaciones 
%                    y al inicio del CU.
%
% \ucstEnRevision    Al terminar la edición del CU (version += 0.1).
% \ucstEnAprobacion  Al pasar la revision sin observaciones.
% \ucstAprobado      Al ser aprobado por el usuario (version += 1.0)

\begin{UseCase}[%
Autor/Daniel Ortega,%
Version/0.1,%
Estado/\ucstEnEdicion]%
%
{CU-E01}{Instalar plugin del módulo de experiencia}{% TODO; Deberia se Instalar/Actualizar ???
%
 Permite al \refElem{aAdministrador} incluir todas las funcionalidades que brinda el módulo de
 experiencia al moodle que administra mediante la instalación de los plugins correspondientes.
 La conclusión de la trayectoria principal de esta caso de uso es una precondición para que los
 demás casos de uso puedan ejecutarse.}

	\UCitem[control]{Revisor}{ Sin asignar }
	\UCitem[control]{Último cambio}{ 13/OCT/19 }

 \UCsection{Atributos}

    \UCitem{Actor(es)}{%
        \refElem{aAdministrador}
    }

	\UCitems{Propósito}{%
        \Titem Permitir al administrador incluir todas las funcionalidades que brinda el módulo de
        experiencia al moodle que administra.

        \Titem Permitir a los usuarios de moodle ver su progreso en la plataforma mediante puntos
        de experiencia.
	}
	
	\UCitem{Entradas}{\imprimeUC{entrada}}

	\UCitems{Origen}{%
        \Titem Mouse
	}

	\UCitem{Salidas}{\imprimeUC{salida}}

	\UCitem{Destino}{%
		\refElem{IU-M01}
	}
	
	\UCitems{Precondiciones}{%
        \Titem La caperta comprimida que contiene los archivos del plugin
        \Titem El plugin debe cumplir con la regla \refElem{BR-M01} para poder ser
               instalado.
        % \Titem Si se trata de una actualización de un plugin la versión de este debe
               % cumplir con la regla \refElem{BR-M02}.
	}

	\UCitems{Postcondiciones}{%
        \Titem El plugin debe permanecer instalado en moodle.%
        \Titem La actualización de las \refElem{xp-general-settings} del módulo de experiencia
               deben persistirse en el sistema.
        \Titem Los \refElem[usuarios]{mdl-user} registrados en moodle deberán tener la
               asociada la información de un \refElem{xp-user}.
	}

	\UCitem{Reglas de negocio}{\imprimeUC{regla}}

	\UCitems{Errores}{%
        \Titem \UCerr{Err1}{%
            El archivo zip del plugin seleccionado está corroto,}{% CAUSA
            no se puede continuar con la instalación del plugin}% EFECTO

        \Titem \UCerr{Err2}{%
            Alguna de las dependencias del plugin no se satisface con
            los plugins instalados}{% CAUSE
            no se puede continuar con las instalación del plugin}% EFECTO

        \Titem \UCerr{Err3}{%
        % CAUSA
            Durante la ejecución de las tareas de instalación del nuevo plugin ocurre
            un error,}{%
        % EFECTO
            las tareas de instalación del propio plugin no pudieron concluir apropiadamente,
            el plugin sigue }
	}

	% \UCitem{Viene de}{% Indicar si el Caso de uso es primario o se extiende de otro. La mayoría se 
					  % extienden de Login.
		% EJEMPLO: \refIdElem{PY-CU1} o Caso de uso primario.
	% 	\TODO Especificar.
	% }	

 \UCsection[design]{Datos de Diseño}

	\UCitems[design]{Casos de Prueba}{%
        \Titem \refElem{CPC-E01}
	}

 \UCsection[admin]{Datos de Administración de Requerimiento}

	\UCitem[admin]{Observaciones}{%
	}

\end{UseCase}

\subsubsection{Trayectorias del caso de uso}

\begin{UCtrayectoria}%
%
    \includeUC{CU-M01}

    \Actor Selecciona la opción {\bf Instalar plugins}
    \Sistema Carga la pantalla \refElem{IU-M02} con el formulario para seleccionar el
             plugin a instalar. \label{CU-E01-formulario-instalacion}

    \Actor Presiona la opción {\bf Seleccione un archivo} 
    \Sistema Despliega la pantalla \refElem{IU-M00a} como pantalla emergente 
             \label{CU-E01-seleccion-archivo}

    \Actor Selecciona la opción {\it Subir un archivo} en el menu izquierdo de la pantalla
           emergente y posteriormente presiona el botón {\it Browse}.
    \Actor Selecciona el archivo que contiene al \entrada{Plugin} del módulo de experiencia.
    \Actor Presiona el botón {\bf Subir este archivo}.
    \Sistema Valida que el archivo del plugin sea de tipo {\it ZIP}. \refTray{A}
    \Sistema Cierra la pantalla emergente y muestra el nombre del archivo seleccionado en la
             pantalla \refElem{IU-M02}

    \Actor Presiona el botón {\bf Instalar plugin desde archivo ZIP}
    \Sistema Valida que el archivo {\it ZIP} cumpla con las restricciones dictadas por la
             \regla{BR-M01}. \refErr{Err1}
    \Sistema Obtiene el \salida{Plugin.componente}, la \salida{Plugin.fullpath} y el
             \salida{Plugin.pluginname} del plugin a ser instalado.
    \Sistema Carga la pantalla \refElem{IU-M02a}, mostrando los datos anteriormente obtenidos.

    \Actor Continua con la instalación de plugin presionando la opción {\bf continuar}. \refTray{B}

    \Sistema Obtiene tambien la \salida{Plugin.moodle}, la lista de \salida{Plugin.dependencies}
             y la \salida{Plugin.version} del plugin a instalar. \refErr{Err2}
             \label{CU-E01-comprobacion}
    \Sistema Despliega los datos obtenidos en la pantalla \refElem{IU-M02b}

    \Actor Presiona el botón {\bf Actualizar base de datos Moodle ahora}. \refTray{C}
    \Sistema Procesa las tareas de instalación de moodle.

    \Sistema Obtiene la lista de los \refElem[identificadores]{mdl-user.id} de los
             usuarios de moodle.
    \Sistema Asocia mediante los identificadores datos de un \refElem{xp-user}
             estableciendo el \refElem{xp-user.level} actual igual a $1$,
             la \refElem{xp-user.xp} igual a $0$.
    \Sistema Establece los valores por defecto para las \refElem{xp-visual-settings} (
              \entrada{xp-visual-settings.title}, 
              \entrada{xp-visual-settings.description}, 
              \entrada{xp-visual-settings.message}, 
              \entrada{xp-visual-settings.colorLvl}, 
              \entrada{xp-visual-settings.colorBar} e
              \entrada{xp-visual-settings.image}), especificadas en el modelo de información.

    \Sistema Carga la interfaz \refElem{IU-M02d} informando que la instalación
             ha sido llevaba a cabo de forma correcta. \refErr{Err3i}

    \Actor Presiona el botón {\bf continuar}.

    \includeUC{CU-E02} a partir del paso \ref{CU-E03-formulario}
\end{UCtrayectoria}

\begin{UCtrayectoriaA}{A}{Cuando el archivo seleccionado es distinto de un ZIP}

  \Sistema Emite en una ventana emergente el mensaje {\it Error: ``El tipo de
           archivo \$EXT no se acepta.''} siendo {\it\$EXT} la extensión del
           archivo seleccionado.
  \Sistema Regresa al paso \ref{CU-E01-seleccion-archivo}.

\end{UCtrayectoriaA}

\begin{UCtrayectoriaA}[Fin del caso de uso]%
{B}{El \refElem{aAdministrador} desea cancelar la instalación despues de la validación del archivo ZIP}

    \Actor Presiona el botón {\bf cancelar} de la pantalla \refElem{IU-M02a}.
    \Sistema Cancela la instalación del plugin y redirige a la pantalla \refElem{IU-M02}

\end{UCtrayectoriaA}

\begin{UCtrayectoriaA}[Fin del caso de uso]%
{C}{El \refElem{aAdministrador} desea cancelar la instalación despues de ver la comprobación de plugins a instalar}

    \Actor Presiona el botón {\bf cancelar esta instalación} o {\bf cancelar las nuevas instalaciones}
    \Sistema Redirige a la pantalla \refElem{IU-M02c}

    \Actor Si el actor presiona el botón {\bf Continuar} entonces 
    \UCpaso[--] el caso de uso terminará, (en caso contrario)

    \Actor Si el actor presiona el botón {\bf Cancelar}
    \Sistema Regresa al paso \ref{CU-E01-comprobacion}
\end{UCtrayectoriaA}

%\subsubsection{Puntos de extensión}

%\UCExtensionPoint{Nombre del punto de extensión}{%

%    El \refElem{aAdministrador} desea/requiere/necesita ....%
%
%    }{En el paso \ref{CU-ET-1x} de la trayectoria principal  ...%
%
%    }{\refElem{CU-E2-T}}

   % Instalar plugin del esquema de experiencia
    
% \ucstEnEdicion     Al terminar una revisión/aprobación con observaciones 
%                    y al inicio del CU.
%
% \ucstEnRevision    Al terminar la edición del CU (version += 0.1).
% \ucstEnAprobacion  Al pasar la revision sin observaciones.
% \ucstAprobado      Al ser aprobado por el usuario (version += 1.0)

\begin{UseCase}[%
Autor/Daniel Ortega,%
Version/0.1,%
Estado/\ucstEnEdicion]%
%
{CU-E02}{Realizar configuraciones del módulo de experiencia}{%
%
 Permite al \refElem{aAdministrador} acceder a las configuraciones del esquema de
 experiencia para consultar y cambiar los aspectos generales del módulo de experiencia, los cuales consisten en habilitar/deshabilitar el esquema de experiencia además de
 habilitar/deshabilitar que los eventos proporcionen experiencia.}

	\UCitem[control]{Revisor}{ Sin asignar }
	\UCitem[control]{Último cambio}{ \today }

 \UCsection{Atributos}

    \UCitem{Actor(es)}{%
        \refElem{aAdministrador}
    }

	\UCitems{Propósito}{%
        \Titem Habilitar/deshabilitar el módulo de experiencia.
        \Titem Habilitar/deshabilitar que los eventos proporcionen experiencia.
	}
	
%% BEGIN-BLOQUE PARA AGREGAR UNA REVISION ------------------------------------->
%% Copiar y descomentar este bloque por cada revision que se realice
%	\UCsection[control]{% Indicar la versión objeto de la revisión.
%		Revisión de la Versión \TODO X.X
%	}
%	\UCitem[control]{Revisó}{% Coloque el nombre de quien realizó la revisión
%		\TODO Especificar
%	}
%	\UCitem[control]{Fecha}{% Coloque la fecha de la revisión
%		% EJEMPLO: 21 de Septiembre de 2019.
%		\TODO Especificar
%	}
%	\UCitem[control]{Resultado}{% Las opciones son: 
%								% Pendiente: se pasa a EnEdicion y se agregan las observaciones
%								% Aprobado: Se pasa a EnAprobacion.
%		\TODO Especificar
%	}
%	\UCitems[control]{Observaciones}{
%		% Agregar las observaciones resultado de la revision o la palabra ``Ninguna''
%		\Titem \TODO Agregar observaciones en cada viñeta, usar el comando \TODO %\TOCHK \DONE.
%	}
%% <------------------------------------------ END-BLOQUE PARA AGREGAR UNA REVISION
	
	\UCitems{Entradas}{\imprimeUC{entrada}}

	\UCitem{Origen}{%
        Mouse
	}

	\UCitems{Salidas}{\imprimeUC{salida}}

	\UCitem{Destino}{%
		\refElem{IU-M01}
	}
	
	\UCitem{Precondiciones}{%
        Que los plugins del módulo de experiencia se encuentren instalados
	}

	\UCitem{Postcondiciones}{%
        Los nuevos valores de las \refElem{xp-general-settings} deben de ser
        almacenados en el sistema.
	}

	\UCitem{Reglas de negocio}{%
		Ninguna
	}

	\UCitems{Errores}{%
        \Titem \UCerr{Err1}{%
        % CAUSA
            Los plugins del módulo de experiencia no se encuentran instalados,}{%
        % EFECTO
            No se presenta en el menu de opciones las opciones para modificar %
            el esquema de experiencia.}
	}

	% \UCitem{Viene de}{% Indicar si el Caso de uso es primario o se extiende de otro. La mayoría se 
					  % extienden de Login.
		% EJEMPLO: \refIdElem{PY-CU1} o Caso de uso primario.
	% 	\TODO Especificar.
	% }	

 \UCsection[design]{Datos de Diseño}

	\UCitems[design]{Casos de Prueba}{%
        \Titem \refElem{CPC-E02}
	}

 \UCsection[admin]{Datos de Administración de Requerimiento}

	\UCitem[admin]{Observaciones}{%
        Ninguna
	}

\end{UseCase}

\clearpage
\subsubsection{Trayectorias del caso de uso}

\begin{UCtrayectoria}%
%
  \includeUC{CU-M01} \refErr{Err1}

  \Actor Presiona la opción {\bf \refElem{tExpSettingsGeneral}} en la categoría
         \refElem{tExpCategoria}. \refTray{A} \label{CU-E02-ir-a-formulario}
  \Sistema Obtiene el valor establecido de los \salida[eventos activados]%
           {xp-general-settings.events} que determina si los eventos brindan
           experiencia o no. \label{CU-E03-formulario}
  \Sistema Obiene el valor \salida{xp-general-settings.activated} que define si el
           módulo de experiencia está activado.
  \Sistema Carga la pantalla \refElem{IU-E02} estableciendo los valores por defecto del
           formulario con los valores obtenidos.

  \Actor Establece el valor \entrada{xp-general-settings.activated} para definir si
         el sistema de experiencia estará habilitado o deshabilitado. \refTray{B}
  \Actor Establece el valor \entrada[eventos activados]{xp-general-settings.events}
         para definir si se les brindará soporte a los eventos dependiendo si estos
         están activados o no.
  \Actor Presiona el botón {\bf Guardar cambios}.
  \Sistema Obtiene el valor \refElem{xp-general-settings.activated} y de
           \refElem[eventos activados]{xp-general-settings.events} y los valores
           establecidos por el \refElem{aActor}.
  \Sistema Carga la pantalla \refElem{IU-M01}.

\end{UCtrayectoria}

\begin{UCtrayectoriaA}{A}{%
El \refElem{aAdministrador} selecciona la categoría \refElem{tExpCategoria}}

    \Sistema Carga la pantalla \refElem{IU-E01}.
    \Sistema Regresa al paso \ref{CU-E02-ir-a-formulario}.
\end{UCtrayectoriaA}

   % Configuraciones generales
    
% \ucstEnEdicion     Al terminar una revisión/aprobación con observaciones 
%                    y al inicio del CU.
%
% \ucstEnRevision    Al terminar la edición del CU (version += 0.1).
% \ucstEnAprobacion  Al pasar la revision sin observaciones.
% \ucstAprobado      Al ser aprobado por el usuario (version += 1.0)

\begin{UseCase}[%
Autor/Daniel Ortega,%
Version/0.1,%
Estado/\ucstEnEdicion]%
%
{CU-E02-1}{Habilitar/Deshabilitar el módulo de experiencia}{%
%
 Permite al \refElem{aActor} .}

	\UCitem[control]{Revisor}{ Sin asignar }
	\UCitem[control]{Último cambio}{ \today }

 \UCsection{Atributos}

    \UCitem{Actor(es)}{%
        \refElem{aActor}
    }

	\UCitem{Propósito}{%
        ...
	}
	
	\UCitem{Entradas}{\imprimeUC{entrada}}

	\UCitems{Origen}{%
        \Titem Mouse
        \Titem Teclado
	}

	\UCitem{Salidas}{\imprimeUC{salida}}

	\UCitems{Destino}{%
		\Titem \refElem{IU-M02a}
	}
	
	\UCitems{Precondiciones}{%
        \Titem ...
	}

	\UCitem{Postcondiciones}{%
        Ninguna
	}

	\UCitem{Reglas de negocio}{%
		Ninguna
	}

	\UCitems{Errores}{%
        \Titem \UCerr{Err1}{%
        % CAUSA
            ...,}{%
        % EFECTO
            ...}
	}

	% \UCitem{Viene de}{% Indicar si el Caso de uso es primario o se extiende de otro. La mayoría se 
					  % extienden de Login.
		% EJEMPLO: \refIdElem{PY-CU1} o Caso de uso primario.
	% 	\TODO Especificar.
	% }	

 \UCsection[design]{Datos de Diseño}

	\UCitems[design]{Casos de Prueba}{%
        \Titem \refElem{CPC-E0Y}
        \Titem \refElem{CPI-E0Y}
	}

 \UCsection[admin]{Datos de Administración de Requerimiento}

	\UCitem[admin]{Observaciones}{%
        Ninguna
	}

\end{UseCase}

\clearpage
\subsubsection{Trayectorias del caso de uso}

\begin{UCtrayectoria}%
%
 \Actor Presiona el botón \IUMenu en la esquina superior izquierda de la pantalla \refElem{IU-M01}
        para abrir el menu de navegación.

 \Actor Selecciona la opción {\it \IUAdminSitio Administración del sitio}

 \Sistema Carga la pantalla \refElem{IU-M02}

\end{UCtrayectoria}


\subsubsection{Puntos de extensión}

\UCExtensionPoint{Nombre del punto de extensión}{%

    El \refElem{aAdministrador} desea/requiere/necesita ....%
%
    }{En el paso \ref{CU-ET-1x} de la trayectoria principal  ...%
%
    }{\refElem{CU-E2-T}}

 % Configurar visualización de niveles
    
% \ucstEnEdicion     Al terminar una revisión/aprobación con observaciones
%                    y al inicio del CU.
%
% \ucstEnRevision    Al terminar la edición del CU (version += 0.1).
% \ucstEnAprobacion  Al pasar la revision sin observaciones.
% \ucstAprobado      Al ser aprobado por el usuario (version += 1.0)

\begin{UseCase}[%
Autor/Daniel Ortega,%
Version/0.1,%
Estado/\ucstEnEdicion]%
%
{CU-E02-2}{Configurar sistema de experiencia}{%
%
 Permite al \refElem{aAdministrador} establecer y modificar las cantidades de puntos
 de experiencia que brindan los cursos en la plataforma y la forma en que aumenta
 la cantidad de experiencia requerida para pasar de un nivel al siguiente. Los cambios
 sobre estas configuraciones pueden afectar el nivel en que se encuentran los usuarios
 y cambiar la cantidad de experiencia que los cursos brindan.}

	\UCitem[control]{Revisor}{ Sin asignar }
	\UCitem[control]{Último cambio}{ \today }

 \UCsection{Atributos}

    \UCitem{Actor(es)}{%
        \refElem{aAdministrador}
    }

	\UCitems{Propósito}{%
        \Titem Permitir al administrador configurar el sistema de experiencia.
        \Titem Establecer o modificar la cantidad de experiencia que brindan los
               cursos.
        \Titem Establecer la cantidad de experiencia requerida para pasar el primer
               nivel usada como calculo para los demás niveles.
        \Titem Cambiar la forma en cómo se incrementa la cantidad de experiencia
               requerida para avanzar de un nivel a otro.
	}

	\UCitem{Entradas}{\imprimeUC{entrada}}

	\UCitems{Origen}{%
        \Titem Mouse
        \Titem Teclado
        \Titem Sistema (para los datos de los \refElem[usuarios gamificados]{xp-user})
	}

	\UCitem{Salidas}{\imprimeUC{salida}}

	\UCitems{Destino}{%
		\Titem \refElem{IU-E04}
	}

	\UCitems{Precondiciones}{%
        \Titem Que los plugins del módulo de experiencia se encuentren instalados
        \Titem El módulo de experiencia debe estár habilitado en el caso de uso
               \refElem{CU-E02}.
	}

	\UCitem{Postcondiciones}{%
        Los nuevos valores de las \refElem{xp-scheme-settings} deber ser
        estár actualizados para todos los usuarios, además de persistirse en el
        sistema.
	}

	\UCitem{Reglas de negocio}{\imprimeUC{regla}}

	\UCitems{Errores}{%
        \Titem \UCerr{Err1}{%
        % CAUSA
            Los plugins del módulo de experiencia no se encuentran instalados,}{%
        % EFECTO
            no se presentan las opciones en el menú y por lo tanto no se puede
            acceder a las configuraciones}
	}

 \UCsection[design]{Datos de Diseño}

	\UCitems[design]{Casos de Prueba}{%
        \Titem \refElem{CPC-E02-2a}
        \Titem \refElem{CPC-E02-2b}
        \Titem \refElem{CPC-E02-2c}
        \Titem \refElem{CPI-E02-2}
	}

 \UCsection[admin]{Datos de Administración de Requerimiento}

	\UCitem[admin]{Observaciones}{%
        Los cambios en las configuraciones del sistema de experiencia podrían hacer
        que los usuarios aumenten de nivel si la cantidad de experiencia acumulada del
        nivel es mayor a la nueva cantidad de experiencia correspondiente a dicho nivel.
        Se recomienda que los cambios se realicen cuando la cantidad estudiantes que usen el
        sistema sea mínima.}

\end{UseCase}

\subsubsection{Trayectorias del caso de uso}

\begin{UCtrayectoria}%
%
  \includeUC{CU-M01} \refErr{Err1}

  \Actor Presiona la opción {\bf \refElem{tExpSettingsComportamiento}} en la categoría
         \refElem{tExpCategoria}. \refTray{A}
  \Sistema Obtiene el valor de si el módulo de experiencia está \refElem[activado]%
           {xp-general-settings.activated} o no. \refTray{B} \label{CU-E02-2-loading}
  \Sistema Obtiene los valores actuales de la configuración del sistema de experiencia:
           \salida{xp-scheme-settings.increment},
           \salida{xp-scheme-settings.incrementValue},
           \salida{xp-scheme-settings.levelXP} y
           \salida{xp-scheme-settings.courseXP}.
  \Sistema Carga la pantalla \refElem{IU-E04} estableciendo como valores por defecto
           las \refElem{xp-scheme-settings} obtenidas en el anterior paso.
  %\sistema muestra un mensaje informando al \refelem{aadministrador} de que si
  %         modifica la experiencia el nivel 1 o el tipo de incremento se alterarían
  %         la cantidad de experiencia requerida para subir de nivel.

  \Actor Especifica si el \entrada{xp-scheme-settings.increment} en la cantidad de
         experiencia de los niveles será {\it Lineal} o {\it Porcentual}.
  \Actor Ingresa el valor para el \entrada{xp-scheme-settings.incrementValue} con
         base en la regla \regla{BR-E03}.
  \Actor Ingresa los valores para la \entrada{xp-scheme-settings.levelXP} y la
         \entrada{xp-scheme-settings.courseXP}.
  \Actor Presiona la opción {\bf Guardar Cambios}. \refTray{C} \label{CU-E02-2-submit}

  \Sistema Valida que los valores ingresados por el usuario cumplan con las
           restricciones especificadas en el modelo de información.
  \Sistema Verifica que el \refElem{xp-scheme-settings.incrementValue} cumpla
           con la regla \refElem{BR-E03}. \refTray{D}
  \Sistema Actualiza los valores de las \refElem{xp-scheme-settings} con los
           ingresados por el usuario.

  % ACTUALIZACION DE NIVEL DE LOS USUARIOS
  \Sistema Obtiene la lista de los \refElem[usuarios gamificados]{xp-user} y por cada uno
           realiza las siguientes acciones.
  \Sistema Obtiene el \entrada{xp-user.level} y la cantidad de \entrada{xp-user.levelxp}.
  \Sistema Calcula la nueva cantidad de experiencia correspondiente al nivel del usuario con
           base en la regla \regla{BR-E04} o \regla{BR-E05} si el incremento es lineal o
           porcentual respectivamente.
  \Sistema - Si la cantidad de experiencia del usuario es menor a la experiencia correspondiente
           al nivel, entonces se procede al siguiente usuario. \refTray{E} \label{CU-E02-2-Usuarios}

  \Sistema Despliega la pantalla \refElem{IU-E04} con el mensaje de que los datos
           han sido actualizados exitosamente.

\end{UCtrayectoria}

\begin{UCtrayectoriaA}{A}{
El \refElem{aAdministrador} selecciona la categoría \refElem{tExpCategoria}}
  \Sistema Carga la pantalla \refElem{IU-E01}
  \Actor Regresa al paso \ref{CU-E02-2-loading}
\end{UCtrayectoriaA}

\begin{UCtrayectoriaA}{B}{
El módulo de experiencia no se encuentra activado}
  \Sistema Carga la pantalla \refElem{IU-E03a}.
  \Actor Presiona el botón {\bf Activar módulo de experiencia}
  \includeUC{CU-E02} a partir del paso \ref{CU-E02-ir-a-formulario},
                     para activar el módulo de experiencia.

  \Sistema Regresa al inicio de la trayectoria principal.

\end{UCtrayectoriaA}

\begin{UCtrayectoriaA}{C}{
El \refElem{aAdministrador} desea cancelar la modificación en el sistema de
experiencia}

  \Actor Presiona el botón {\bf Cancelar}.
  \Sistema Redirige a la pantalla \refElem{IU-M01}.
\end{UCtrayectoriaA}

\begin{UCtrayectoriaA}{D}{
Alguno de los valores ingresados por el usuario son incorrectos.}
  \Sistema Imprime los mensajes de error abajo de los campos con los valores
           incorrectos.
  \Actor Ingresa nuevamente los valores en los campos marcados como incorrectos.
  \Sistema Regresa al paso \ref{CU-E02-2-submit}.

\end{UCtrayectoriaA}

\begin{UCtrayectoriaA}{E}{%
La cantidad de \refElem{xp-user.levelxp} es mayor a la experiencia del nivel en el
que se encuentra}

  \Sistema Avanza al \refElem{xp-user} al siguiente nivel, usando los
           puntos de experiencia del mismo y establece el sobrante como
           la \refElem{xp-user.levelxp} correspondiente al nuevo nivel.
  \Sistema Repite el paso anterior hasta que la cantidad de experiencia
           del nivel del usuario sea menor que la del nivel.
  \Sistema Regresa al paso \ref{CU-E02-2-Usuarios}

\end{UCtrayectoriaA}
 % Configurar esquema de experiencia

    %
% \ucstEnEdicion     Al terminar una revisión/aprobación con observaciones 
%                    y al inicio del CU.
%
% \ucstEnRevision    Al terminar la edición del CU (version += 0.1).
% \ucstEnAprobacion  Al pasar la revision sin observaciones.
% \ucstAprobado      Al ser aprobado por el usuario (version += 1.0)

\begin{UseCase}[%
Autor/Daniel Ortega,%
Version/0.1,%
Estado/\ucstEnEdicion]%
%
{CU-E02-1}{Habilitar/Deshabilitar el módulo de experiencia}{%
%
 Permite al \refElem{aActor} .}

	\UCitem[control]{Revisor}{ Sin asignar }
	\UCitem[control]{Último cambio}{ \today }

 \UCsection{Atributos}

    \UCitem{Actor(es)}{%
        \refElem{aActor}
    }

	\UCitem{Propósito}{%
        ...
	}
	
	\UCitem{Entradas}{\imprimeUC{entrada}}

	\UCitems{Origen}{%
        \Titem Mouse
        \Titem Teclado
	}

	\UCitem{Salidas}{\imprimeUC{salida}}

	\UCitems{Destino}{%
		\Titem \refElem{IU-M02a}
	}
	
	\UCitems{Precondiciones}{%
        \Titem ...
	}

	\UCitem{Postcondiciones}{%
        Ninguna
	}

	\UCitem{Reglas de negocio}{%
		Ninguna
	}

	\UCitems{Errores}{%
        \Titem \UCerr{Err1}{%
        % CAUSA
            ...,}{%
        % EFECTO
            ...}
	}

	% \UCitem{Viene de}{% Indicar si el Caso de uso es primario o se extiende de otro. La mayoría se 
					  % extienden de Login.
		% EJEMPLO: \refIdElem{PY-CU1} o Caso de uso primario.
	% 	\TODO Especificar.
	% }	

 \UCsection[design]{Datos de Diseño}

	\UCitems[design]{Casos de Prueba}{%
        \Titem \refElem{CPC-E0Y}
        \Titem \refElem{CPI-E0Y}
	}

 \UCsection[admin]{Datos de Administración de Requerimiento}

	\UCitem[admin]{Observaciones}{%
        Ninguna
	}

\end{UseCase}

\clearpage
\subsubsection{Trayectorias del caso de uso}

\begin{UCtrayectoria}%
%
 \Actor Presiona el botón \IUMenu en la esquina superior izquierda de la pantalla \refElem{IU-M01}
        para abrir el menu de navegación.

 \Actor Selecciona la opción {\it \IUAdminSitio Administración del sitio}

 \Sistema Carga la pantalla \refElem{IU-M02}

\end{UCtrayectoria}


\subsubsection{Puntos de extensión}

\UCExtensionPoint{Nombre del punto de extensión}{%

    El \refElem{aAdministrador} desea/requiere/necesita ....%
%
    }{En el paso \ref{CU-ET-1x} de la trayectoria principal  ...%
%
    }{\refElem{CU-E2-T}}

 % Configuraciones generales
    %
% \ucstEnEdicion     Al terminar una revisión/aprobación con observaciones
%                    y al inicio del CU.
%
% \ucstEnRevision    Al terminar la edición del CU (version += 0.1).
% \ucstEnAprobacion  Al pasar la revision sin observaciones.
% \ucstAprobado      Al ser aprobado por el usuario (version += 1.0)

\begin{UseCase}[%
Autor/Daniel Ortega,%
Version/0.1,%
Estado/\ucstEnEdicion]%
%
{CU-E02-2}{Configurar sistema de experiencia}{%
%
 Permite al \refElem{aAdministrador} establecer y modificar las cantidades de puntos
 de experiencia que brindan los cursos en la plataforma y la forma en que aumenta
 la cantidad de experiencia requerida para pasar de un nivel al siguiente. Los cambios
 sobre estas configuraciones pueden afectar el nivel en que se encuentran los usuarios
 y cambiar la cantidad de experiencia que los cursos brindan.}

	\UCitem[control]{Revisor}{ Sin asignar }
	\UCitem[control]{Último cambio}{ \today }

 \UCsection{Atributos}

    \UCitem{Actor(es)}{%
        \refElem{aAdministrador}
    }

	\UCitems{Propósito}{%
        \Titem Permitir al administrador configurar el sistema de experiencia.
        \Titem Establecer o modificar la cantidad de experiencia que brindan los
               cursos.
        \Titem Establecer la cantidad de experiencia requerida para pasar el primer
               nivel usada como calculo para los demás niveles.
        \Titem Cambiar la forma en cómo se incrementa la cantidad de experiencia
               requerida para avanzar de un nivel a otro.
	}

	\UCitem{Entradas}{\imprimeUC{entrada}}

	\UCitems{Origen}{%
        \Titem Mouse
        \Titem Teclado
        \Titem Sistema (para los datos de los \refElem[usuarios gamificados]{xp-user})
	}

	\UCitem{Salidas}{\imprimeUC{salida}}

	\UCitems{Destino}{%
		\Titem \refElem{IU-E04}
	}

	\UCitems{Precondiciones}{%
        \Titem Que los plugins del módulo de experiencia se encuentren instalados
        \Titem El módulo de experiencia debe estár habilitado en el caso de uso
               \refElem{CU-E02}.
	}

	\UCitem{Postcondiciones}{%
        Los nuevos valores de las \refElem{xp-scheme-settings} deber ser
        estár actualizados para todos los usuarios, además de persistirse en el
        sistema.
	}

	\UCitem{Reglas de negocio}{\imprimeUC{regla}}

	\UCitems{Errores}{%
        \Titem \UCerr{Err1}{%
        % CAUSA
            Los plugins del módulo de experiencia no se encuentran instalados,}{%
        % EFECTO
            no se presentan las opciones en el menú y por lo tanto no se puede
            acceder a las configuraciones}
	}

 \UCsection[design]{Datos de Diseño}

	\UCitems[design]{Casos de Prueba}{%
        \Titem \refElem{CPC-E02-2a}
        \Titem \refElem{CPC-E02-2b}
        \Titem \refElem{CPC-E02-2c}
        \Titem \refElem{CPI-E02-2}
	}

 \UCsection[admin]{Datos de Administración de Requerimiento}

	\UCitem[admin]{Observaciones}{%
        Los cambios en las configuraciones del sistema de experiencia podrían hacer
        que los usuarios aumenten de nivel si la cantidad de experiencia acumulada del
        nivel es mayor a la nueva cantidad de experiencia correspondiente a dicho nivel.
        Se recomienda que los cambios se realicen cuando la cantidad estudiantes que usen el
        sistema sea mínima.}

\end{UseCase}

\subsubsection{Trayectorias del caso de uso}

\begin{UCtrayectoria}%
%
  \includeUC{CU-M01} \refErr{Err1}

  \Actor Presiona la opción {\bf \refElem{tExpSettingsComportamiento}} en la categoría
         \refElem{tExpCategoria}. \refTray{A}
  \Sistema Obtiene el valor de si el módulo de experiencia está \refElem[activado]%
           {xp-general-settings.activated} o no. \refTray{B} \label{CU-E02-2-loading}
  \Sistema Obtiene los valores actuales de la configuración del sistema de experiencia:
           \salida{xp-scheme-settings.increment},
           \salida{xp-scheme-settings.incrementValue},
           \salida{xp-scheme-settings.levelXP} y
           \salida{xp-scheme-settings.courseXP}.
  \Sistema Carga la pantalla \refElem{IU-E04} estableciendo como valores por defecto
           las \refElem{xp-scheme-settings} obtenidas en el anterior paso.
  %\sistema muestra un mensaje informando al \refelem{aadministrador} de que si
  %         modifica la experiencia el nivel 1 o el tipo de incremento se alterarían
  %         la cantidad de experiencia requerida para subir de nivel.

  \Actor Especifica si el \entrada{xp-scheme-settings.increment} en la cantidad de
         experiencia de los niveles será {\it Lineal} o {\it Porcentual}.
  \Actor Ingresa el valor para el \entrada{xp-scheme-settings.incrementValue} con
         base en la regla \regla{BR-E03}.
  \Actor Ingresa los valores para la \entrada{xp-scheme-settings.levelXP} y la
         \entrada{xp-scheme-settings.courseXP}.
  \Actor Presiona la opción {\bf Guardar Cambios}. \refTray{C} \label{CU-E02-2-submit}

  \Sistema Valida que los valores ingresados por el usuario cumplan con las
           restricciones especificadas en el modelo de información.
  \Sistema Verifica que el \refElem{xp-scheme-settings.incrementValue} cumpla
           con la regla \refElem{BR-E03}. \refTray{D}
  \Sistema Actualiza los valores de las \refElem{xp-scheme-settings} con los
           ingresados por el usuario.

  % ACTUALIZACION DE NIVEL DE LOS USUARIOS
  \Sistema Obtiene la lista de los \refElem[usuarios gamificados]{xp-user} y por cada uno
           realiza las siguientes acciones.
  \Sistema Obtiene el \entrada{xp-user.level} y la cantidad de \entrada{xp-user.levelxp}.
  \Sistema Calcula la nueva cantidad de experiencia correspondiente al nivel del usuario con
           base en la regla \regla{BR-E04} o \regla{BR-E05} si el incremento es lineal o
           porcentual respectivamente.
  \Sistema - Si la cantidad de experiencia del usuario es menor a la experiencia correspondiente
           al nivel, entonces se procede al siguiente usuario. \refTray{E} \label{CU-E02-2-Usuarios}

  \Sistema Despliega la pantalla \refElem{IU-E04} con el mensaje de que los datos
           han sido actualizados exitosamente.

\end{UCtrayectoria}

\begin{UCtrayectoriaA}{A}{
El \refElem{aAdministrador} selecciona la categoría \refElem{tExpCategoria}}
  \Sistema Carga la pantalla \refElem{IU-E01}
  \Actor Regresa al paso \ref{CU-E02-2-loading}
\end{UCtrayectoriaA}

\begin{UCtrayectoriaA}{B}{
El módulo de experiencia no se encuentra activado}
  \Sistema Carga la pantalla \refElem{IU-E03a}.
  \Actor Presiona el botón {\bf Activar módulo de experiencia}
  \includeUC{CU-E02} a partir del paso \ref{CU-E02-ir-a-formulario},
                     para activar el módulo de experiencia.

  \Sistema Regresa al inicio de la trayectoria principal.

\end{UCtrayectoriaA}

\begin{UCtrayectoriaA}{C}{
El \refElem{aAdministrador} desea cancelar la modificación en el sistema de
experiencia}

  \Actor Presiona el botón {\bf Cancelar}.
  \Sistema Redirige a la pantalla \refElem{IU-M01}.
\end{UCtrayectoriaA}

\begin{UCtrayectoriaA}{D}{
Alguno de los valores ingresados por el usuario son incorrectos.}
  \Sistema Imprime los mensajes de error abajo de los campos con los valores
           incorrectos.
  \Actor Ingresa nuevamente los valores en los campos marcados como incorrectos.
  \Sistema Regresa al paso \ref{CU-E02-2-submit}.

\end{UCtrayectoriaA}

\begin{UCtrayectoriaA}{E}{%
La cantidad de \refElem{xp-user.levelxp} es mayor a la experiencia del nivel en el
que se encuentra}

  \Sistema Avanza al \refElem{xp-user} al siguiente nivel, usando los
           puntos de experiencia del mismo y establece el sobrante como
           la \refElem{xp-user.levelxp} correspondiente al nuevo nivel.
  \Sistema Repite el paso anterior hasta que la cantidad de experiencia
           del nivel del usuario sea menor que la del nivel.
  \Sistema Regresa al paso \ref{CU-E02-2-Usuarios}

\end{UCtrayectoriaA}
 % Visualización de niveles
    %
% \ucstEnEdicion     Al terminar una revisión/aprobación con observaciones
%                    y al inicio del CU.
%
% \ucstEnRevision    Al terminar la edición del CU (version += 0.1).
% \ucstEnAprobacion  Al pasar la revision sin observaciones.
% \ucstAprobado      Al ser aprobado por el usuario (version += 1.0)

\begin{UseCase}[%
Autor/Daniel Ortega,%
Version/0.1,%
Estado/\ucstEnEdicion]%
%
{CU-E02-3}{Configurar los eventos que entregan experiencia}{%
%
 Permite al \refElem{aAdministrador} elegir cuales eventos de los soportados
 brindarán experiencia y cuales no. Además de aquellos eventos que brindan
 experiencia puede especificar la cantidad de experiencia que estos entregan.}

	\UCitem[control]{Revisor}{ Sin asignar }
	\UCitem[control]{Último cambio}{ \today }

 \UCsection{Atributos}

    \UCitem{Actor(es)}{%
        \refElem{aAdministrador}
    }

	\UCitems{Propósito}{%
        \Titem Configurar que eventos proporcionan experiencia.
        \Titem Establecer la cantidad de experiencia de los eventos
               otorgarán.
        \Titem Deshabilitar eventos para que dejen de brindar
               experiencia.
	}

	\UCitem{Entradas}{\imprimeUC{entrada}}

	\UCitems{Origen}{%
        \Titem Mouse
        \Titem Teclado
	}

	\UCitem{Salidas}{\imprimeUC{salida}}

	\UCitems{Destino}{%
		\Titem \refElem{IU-E05}
	}

	\UCitems{Precondiciones}{%
        \Titem Los plugins correspondientes al módulo de experiencia deben
               de estar previamente instalados.
        \Titem El módulo de experiencia debe estar habilitado mediante el caso
               de uso \refElem{CU-E02}.
        \Titem Los eventos con experiencia deben estar habilitados mediante el caso
               de uso \refElem{CU-E02}.
	}

	\UCitem{Postcondiciones}{%
        Los nuevos valores para los eventos habilitados para dar experiencia deben
        de ser actualizados y persistirse en el sistema.
	}

	\UCitem{Reglas de negocio}{Ninguna}

	\UCitems{Errores}{%
        \Titem \UCerr{Err1}{%
        % CAUSA
            Los plugins del módulo de experiencia no se encuentran instalados,}{%
        % EFECTO
            no se presentan las opciones en el menú y por lo tanto no se acceder
            a las configuraciones}
	}

	% \UCitem{Viene de}{% Indicar si el Caso de uso es primario o se extiende de otro. La mayoría se
					  % extienden de Login.
		% EJEMPLO: \refIdElem{PY-CU1} o Caso de uso primario.
	% 	\TODO Especificar.
	% }

 \UCsection[design]{Datos de Diseño}

	\UCitems[design]{Casos de Prueba}{%
        \Titem \refElem{CPC-E02-3}
        \Titem \refElem{CPI-E02-3}
	}

 \UCsection[admin]{Datos de Administración de Requerimiento}

	\UCitem[admin]{Observaciones}{%
        Ninguna
	}

\end{UseCase}

\subsubsection{Trayectorias del caso de uso}

\begin{UCtrayectoria}%
%
  \includeUC{CU-M01} \refErr{Err1}

  \Actor Presiona la opción {\bf\refElem{tExpSettingsEventos}} en la categoría
         \refElem{tExpCategoria}. \refTray{A}

  \Sistema Obtiene el valor de si el módulo de experiencia está \refElem[activado]%
           {xp-general-settings.activated} o no. \refTray{B} \label{CU-E02-3-activated}

  \Sistema Obtiene el valor de si la funcionalidad de que los \refElem[eventos]%
           {xp-general-settings.events} otorgen experiencia está activada o no.
           \refTray{C} \label{CU-E02-3-events}

  \Sistema Obtiene los valores actuales de la configuración de los eventos con experiencia:
           \salida{xp-events-settings.competencecpuevent},
           \salida{xp-events-settings.competencecpuxp},
           \salida{xp-events-settings.competencevsevent},
           \salida{xp-events-settings.competencevsxp},
           \salida{xp-events-settings.preguntadiariaevento} y
           \salida{xp-events-settings.preguntadiariaxp}.

  \Sistema Carga la pantalla \refElem{IU-E05} estableciendo como valores por defecto
           las \refElem{xp-events-settings} obtenidas en el anterior paso.

  \Actor Habilita los eventos que desea que otorguen experiencia:
           \entrada{xp-events-settings.competencecpuevent},
           \entrada{xp-events-settings.competencevsevent} y
           \entrada{xp-events-settings.preguntadiariaevento}.

  \Sistema Habilita los campos para especificar la experiencia correspondientes a aquellos
           eventos que el usuario haya habilitado.

  \Actor Especifica los valores para la experiencia correspondientes a los eventos que haya
         habilitado, en los campos:
           \entrada{xp-events-settings.competencecpuxp},
           \entrada{xp-events-settings.competencevsxp} y
           \entrada{xp-events-settings.preguntadiariaxp}.

  \Actor Presiona el botón {\bf Guardar Cambios} \refTray{D} \label{CU-E02-3-submit}

  \Sistema Valida que los datos ingresados por el usuario cumplan con las restricciones
           de acuerdo con el modelo de información. \refTray{E}
  \Sistema Actualiza los datos de la configuración de eventos en el sistema
  \Sistema Muestra la pantalla \refElem{IU-E05} con el mensaje de que los datos han sido
           actualizados correctamente.

\end{UCtrayectoria}


\begin{UCtrayectoriaA}{A}{%
El \refElem{aAdministrador} selecciona la categoría \refElem{tExpCategoria}.
}
  \Sistema Carga a pantalla \refElem{IU-E01}.
  \Actor Regresa a paso \ref{CU-E02-3-activated}
\end{UCtrayectoriaA}

\begin{UCtrayectoriaA}{B}{%
El módulo de experiencia no se encuentra activado.
}
  \Sistema Carga la pantalla \refElem{IU-E03a}.
  \Actor Presiona el botón {\bf Activar módulo de experiencia}
  \includeUC{CU-E02} a partir del paso \ref{CU-E02-ir-a-formulario},
                     para activar el módulo de experiencia.

  \Sistema Regresa al inicio de la trayectoria principal.
\end{UCtrayectoriaA}

\begin{UCtrayectoriaA}{C}{%
La opción de que los eventos brinden experiencia no se encuentra activada.
}
  \Sistema Carga la pantalla \refElem{IU-E05a}.
  \Actor Presiona el botón {\bf Habilitar Eventos}
  \includeUC{CU-E02} a partir del paso \ref{CU-E02-ir-a-formulario},
                     para activar el módulo de experiencia.

  \Sistema Regresa al inicio de la trayectoria principal.
\end{UCtrayectoriaA}

\begin{UCtrayectoriaA}{D}{%
El \refElem{aAdministrador} desea cancelar la modificación de la configuración
de los eventos.
}
  \Actor Presiona el botón {\bf Cancelar}
  \Sistema Redirige a la pantalla \refElem{IU-M01}
\end{UCtrayectoriaA}

\begin{UCtrayectoriaA}{E}{%
Alguno de los campos ingresados por el \refElem{aAdministrador} son incorrectos.
}
  \Sistema Imprime los mensajes de error debajo de los con los valores incorrectos
  \Actor Ingresa nuevamente los valores en los campos marcados como incorrectos.
  \Sistema Regresa al paso \ref{CU-E02-3-submit}
\end{UCtrayectoriaA}
 % Esquema de experiencia
    %
\begin{UseCase}{CU-E2}{Crear curso con experiencia}{%
El profesor desea crear un nuevo curso en moodle con soporte para brindar puntos de experiencia a los alumnos, partiendo de la interfaz \IUref{moodle:nuevoCurso} llena los campos del curso, escoge el formato {\it Gamedle} y habilita la opción de experiencia, finalmente presiona el botón para crear el curso.}

	\UCrow{Versión}{\color{gray} 0.1 (Edición)}
    \UCrow{Autor}{\color{gray}	Daniel Ortega}
    \UCrow{Supervisa}{\color{gray}}
    \UCrow{Actor}{Profesor}
    \UCrow{Propósito}{Que el profesor pueda crear un curso que tenga soporte para brindar puntos de experiencia en las distintas secciones del curso.}
    \UCrow{Entradas}{
		\begin{Titemize}
		    \Titem{ Nombre completo y nombre corto del curso }
		    \Titem{ Datos generales y de descripción del curso }
		    \Titem{ Elección del formato del curso }
		    \Titem{ Numero de secciones }
		    \Titem{ Visibilidad de secciones ocultas }
		    \Titem{ Aspecto del curso }
		    \Titem{ Casilla de experiencia }
		    \Titem{ Conjunto de datos restantes }
		    \Titem{ Botón de confirmación \textit{Guardar y regresar} o \textit{Guardar cambios y mostrar} }
		\end{Titemize}
   	}
    \UCrow{Origen}{ Ratón para las acciones y elecciones, teclado para los campos de texto}
	\UCrow{Salidas}{ \IUref{moodle:} o \IUref{moodle:} } %\begin{Titemize}\Titem{Ninguna}\end{Titemize}
    \UCrow{Destino}{ Pantalla }
    \UCrow{Precondiciones}{
		\begin{Titemize}
	        \Titem{ Contar con los permisos necesarios para crear cursos }
	        \Titem{ Tener instalado el plugin ''Gamedle Level'' }
	        \Titem{ Tener instalado el plugin ''Gamedle Format'' }
	        \Titem{ Que el actor haya elegido el formato de curso {\it Gamedle} en el caso de uso {\it Crear curso} }
		\end{Titemize}
    }
    \UCrow{Postcondiciones}{%
        \begin{Titemize}
            \Titem{ Se crea un curso con soporte para brindar experiencia. }
            \Titem{ Las secciones del curso tienen experiencia predeterminada }
            \Titem{ El curso y las secciones se muestran de acuerdo a las configuraciones realizadas }
        \end{Titemize}
    }
	\UCrow{Errores}{ No se encuentra la opción ''Gamedle Format'' en los formatos del curso, debido a que los plugins no han sido instalados }
    \UCrow{Observaciones}{  }
\end{UseCase}
\clearpage

%\textbullet{Trayectorias}

\begin{UCtrayectoria}{Principal}
    \moodle Muestra la interfaz \IUref{moodle:nuevoCurso}. \UCnote{CU: Crear curso}
    \actor Especifica el ''nombre completo'' y ''nombre corto'' además de la ''descripción'' y los ''datos generales'' del curso.\\
    
    \setcounter{enumi}{0}
    \actor Selecciona el {\it formato del curso} {\bf Curso Gamedle}. \IUref{exp:format} \UCnote{CU: Crear curso con experiencia}
    \sistema Carga los nuevos datos para el formulario del formato: Curso Gamedle.
    \actor Selecciona el ''número de secciones'' que tendrá por defecto el curso.
    \actor Selecciona la ''visibilidad'' de forma colapsada u no visible de las secciones ocultas.
    \actor Especifica si el ''aspecto del curso'' es mostrar una sección por página o mostrar todas.
    \actor Habilita la ''casilla de experiencia''. \UCnote{\bf Trayectoria A}
    \actor Especifica el conjunto de datos restantes para la configuración del curso.
    \actor Presiona botón \fbox{Guardar y regresar} \UCnote{\bf Trayectoria B} \UCnote{\bf Trayectoria C} %\fbox{Guardar cambios}.
    \sistema Crea el curso y las secciones del mismo.
    \sistema Obtiene del esquema de experiencia la cantidad de experiencia para los cursos.
    \sistema Divide la cantidad de experiencia del curso entre las secciones creadas.
    \sistema Guarda los valores de experiencia que le corresponden a cada sección.
    \sistema Muestra la pantalla \IUref{moodle:}
        % en caso de que la división no sea entera, la última sección tendrá la cantidad para completar
    \item[- -] - - {\em Fin del caso de uso.}
\end{UCtrayectoria}

%\begin{UCtrayectoria}[Formato distinto a Gamedle.]{Alternativa A}
    %\actor Selecciona un formato de curso distinto a ''Gamedle Format''
    %\actor Especifica el conjunto de datos restantes para la configuración del curso.
    %\actor Presiona botón \fbox{Guardar y regresar} o \fbox{Guardar cambios y mostrar}.
    %\item[- -] - - {\em Fin del caso de uso}
%\end{UCtrayectoria}

\begin{UCtrayectoria}[Formato Gamedle sin experiencia]{Alternativa A}
    \actor Deshabilita la ''casilla de experiencia''
    \actor Especifica el conjunto de datos restantes para la configuración del curso.
    \item[- -] - - {\em Fin del caso de uso}
\end{UCtrayectoria}

\begin{UCtrayectoria}[Guardar cambios y mostrar]{Alternativa B}
    \actor Presiona botón \fbox{Guardar cambios y mostrar}.
    \sistema Crea el curso y las secciones del mismo.
    \sistema Obtiene del esquema de experiencia la cantidad de experiencia para los cursos.
    \sistema Divide la cantidad de experiencia del curso entre las secciones creadas.
    \sistema Guarda los valores de experiencia que le corresponden a cada sección.
    \sistema Muestra la pantalla \IUref{moodle:}
    \item[- -] - - {\em Fin del caso de uso}
\end{UCtrayectoria}

\begin{UCtrayectoria}[Cancelar]{Alternativa C}
    \actor Presiona botón \fbox{Cancelar}.
    \item[- -] - - {\em Fin del caso de uso}
\end{UCtrayectoria}

%\UserStory{Crear curso con experiencia}{Como {\bf administrador} me gustaría que la instalación de un ... por lo que ...}

\clearpage\clearpage % Crear curso con experiencia
    %\begin{UseCase}{CU-E9}{Recibir experiencia}{
Cuando un alumno conteste un ejercicio y suba un intento para revisión, el sistema le otorgará la experiencia correspondiente. Si recibe experiencia suficiente, el usuario subirá de nivel.
}
	\UCrow{Versión}{\color{gray} 0.2 (Revisado)}
    \UCrow{Autor}{\color{gray}	David Flores Casanova}
    \UCrow{Supervisa}{\color{gray}	Daniel Isaí Ortega Zúñiga}
    \UCrow{Actor}{Alumno} % Gerente, Instructor
    \UCrow{Propósito}{Otorgarle experiencia al actor por completar una actividad.}
    \UCrow{Entradas}{
        Selección en el botón \#2 ''Enviar todo y terminar'' de la interfaz \hyperref[IUM01]{IU-M01 Ver intento de examen} .\newline
        Selección en el botón \#1 ''Enviar todo y terminar'' de la interfaz \hyperref[IUM02]{IU-M02 Confirmación de envío de intento} .\newline
        Presión de la tecla ''Enter'' o la tecla ''espacio''.
   	}
    \UCrow{Origen}{Ratón y teclado de la computadora }
	\UCrow{Salidas}{
	    \begin{Titemize}
        \Titem{''\hyperref[table:METerminosExperiencia1]{Experiencia otorgada}''.}
        \Titem{''\hyperref[table:METerminosExperiencia1]{Experiencia actual}'' del actor.}
        \Titem{''\hyperref[table:METerminosExperiencia1]{Experiencia del nivel}'' del nivel actual del actor.}
        \Titem{Barra de progreso, mostrada según el \hyperref[table:METerminosExperiencia1]{porcentaje actual}.}
        \Titem{Nombre del nivel (Cadena de caracteres, longitud $\leq$ 60)}
        \Titem{Mensaje de felicitaciones del nivel (Cadena de caracteres, longitud $\leq$ 50)}
        \Titem{Imagen del nivel (Imagen, formato '.png')}
        \Titem{Descripción del nivel (Cadena de caracteres, longitud $\leq$ 200)}
	    \end{Titemize}
    }
    \UCrow{Destino}{Pantalla}
    \UCrow{Precondiciones}{
		\begin{CUTitemize}
	        \CUTitem{El actor está registrado como alumno del curso.}
            \CUTitem{El actor no tiene registrados intentos anteriores.}
			\CUTitem{La actividad del curso está creada.} 
            \CUTitem{El módulo de experiencia está habilitado.}
		\end{CUTitemize}
    }
    \UCrow{Postcondiciones}{
		\begin{CUTitemize}
	        \CUTitem{Al actor se le registra su nueva cantidad de experiencia.}
			\CUTitem{Se actualiza la cantidad de experiencia que ha recibido el actor en ese curso.}
		\end{CUTitemize}
    }
	\UCrow{Errores}{E1: El actor no está registrado como alumno del curso}
    \UCrow{Observaciones}{}
\end{UseCase}

%\textbullet{Trayectorias}

\begin{UCtrayectoria}{Principal}
    \actor se encuentra en la interfaz \hyperref[IUM01]{IU-M01 Ver intento de examen}.
    \actor selecciona el botón \#2 ''Enviar todo y terminar'' .
    \sistema muestra la ventana emergente \hyperref[IUM02]{IU-M02 Confirmación de envío de intento}.
    \actor selecciona el botón \#1 ''Enviar todo y terminar'' ({\it Trayectoria alternativa A})
    \sistema comprueba que el módulo de experiencia esté habilitado ({\it Trayectoria alternativa B}).
    \sistema comprueba que el actor que subió el intento es un alumno del curso ({\it Trayectoria alternativa C}).
    \sistema comprueba que el actor no tenga registrados intentos anteriores ({\it Trayectoria alternativa D}).
    \sistema calcula si la experiencia que se le dará al actor provoca que este ''suba de nivel'' ({\it Trayectoria alternativa E}).
    \sistema carga la interfaz \hyperref[IUM03]{IU-M03 Intento calificado}.
    \item[- -] - - {\em El caso de uso termina.}
\end{UCtrayectoria}

\begin{UCtrayectoria}{alternativa A}
    \item[- -] - - {\em El usuario presionó el botón \#2 \fbox{Cancelar} o  el botón \#3 \fbox{X}.}
    \sistema cierra la interfaz \hyperref[IUM02]{IU-M02 Confirmación de envío de intento}.
    \item[- -] - - {\em El caso de uso termina.}
\end{UCtrayectoria}

\begin{UCtrayectoria}{alternativa B}
    \item[- -] - - {\em El módulo de experiencia no está habilitado.}
    \item[- -] - - {\em El caso de uso termina.}
\end{UCtrayectoria}

\begin{UCtrayectoria}{alternativa C}
    \item[- -] - - {\em El actor no está registrado como alumno del curso.}
    \item[- -] - - {\em El caso de uso termina.}
\end{UCtrayectoria}

\begin{UCtrayectoria}{alternativa D}
    \item[- -] - - {\em El actor ya había hecho intentos anteriores.}
    \item[- -] - - {\em El caso de uso termina.}
\end{UCtrayectoria}

\begin{UCtrayectoria}{alternativa E}
    \item[- -] - - {\em La ''\hyperref[table:METerminosExperiencia1]{experiencia otorgada}'' que recibe el actor es suficiente para ''subir de nivel''.}
    %\item[- -] - - {\em Se extiende al CU-E02.}
    \sistema calcula la ''\hyperref[table:METerminosExperiencia2]{experiencia sobrante}''.
    \sistema incrementa el nivel actual del actor en una unidad.
    \sistema cambia el valor de la ''\hyperref[table:METerminosExperiencia1]{experiencia actual}'' por la de la ''\hyperref[table:METerminosExperiencia1]{experiencia sobrante}''
    \sistema guarda el nivel actual del actor y la ''\hyperref[table:METerminosExperiencia1]{experiencia actual}''.
    \sistema comprueba si el nivel actual del actor existe dentro de un rango de niveles ({\it Trayectoria alternativa F}). 
    \sistema carga la interfaz \hyperref[IUE02]{IU-E02 Subir de nivel} con la información por defecto de niveles.
    \actor presiona la tecla ''enter'' o ''espacio''.
    \sistema cierra la interfaz \hyperref[IUE02]{IU-E02 Subir de nivel}.
    \item[- -] - - {\em Se continua en el paso \#9 de la trayectoria principal.}
\end{UCtrayectoria}


\begin{UCtrayectoria}{alternativa F}
    \item[- -] - - {\em El nivel actual del actor está dentro de un rango de niveles.}
    \sistema carga la interfaz \hyperref[IUE02]{IU-E02 Subir de nivel}  con la información especificada en el rango de niveles.
    \item[- -] - - {\em Se continua en el paso \#7 de la trayectoria alternativa E.}
\end{UCtrayectoria}

\vfill\clearpage\clearpage % Recibir experiencia

% =========================================================
\clearpage
\section{Interfaces de Moodle}

    % INTERFACES DE MOODLE 
    
\subsubsection{IU-M00 Pantalla principal}

 El tablero o {\it Dashboard} (ver figura~\ref{IU-M02}) es la primer página que ve un usuario de
 inmediatamente despues iniciar sesión, esta página muestra a los usuarios detalles de su progreso
 y fechas límite próximas \cite{MoodleTablero} . Los elementos que tiene esta página con las demás
 páginas del sitio de moodle es el menú de navegación a la izquierda y la columna derecha de
 bloques.

    \IUfig{1}{modulos/moodle/IU/Dashboard.png}{IU-M00}{Pantalla Principal de Moodle}

\subsubsection{Elementos relevantes}

    \begin{itemize}
    \item
    {\bf Menú Superior}
        Como su nombre lo indica se encuentra en la parte superior, este elemento se
        encuentra en la mayoría de las pantallas de moodle.

    \item
    {\bf Menú de Navegación}
        Cuando esta visible se encuentra en la parte izquierda de la parte izquierda
        de la mayoría de las pantallas de moodle. Se puede ocultar o mostrar con la
        acción \IUMenu[].

    \item
    {\bf Contenido}
        Tiene todos los demás elementos que conforman el contenido de la pantalla.

    \end{itemize}

\subsubsection{Acciones relevantes}

    \begin{itemize}
    
    \item
    {\bf \IUMenu{} (Desplegar el menú)}
        Si el menú está oculto, cuando el usuario presione el botón \IUMenu{} el menú de
        navegación se desplegará.

    \item {\bf \IUMenu{} (Ocultar Menu)}
        Si el menú está visible, cuando el usuario presione el botón \IUMenu{} el menú de
        navegación se ocultará.

    \item {\bf \IUAdminSitio{} Administración del sitio }
        Cuando el menú está visible, el botón de administración del sitio nos permitirá
        navegar a la pantalla \refElem{IU-M01}

    \end{itemize}
  % Tablero (Dashboard)
    
\subsection{IU-M00a Selector de archivos}

 El selector de archivos permite a los usuarios de moodle seleccionar un archivo desde los archivos
 del servidor, archivos recientes, archivos privados, desde la computadora o incluso buscar imagenes
 para ser seleccionadas \cite{MoodleSelectorArchivos}.

    \IUfig{1}{modulos/moodleIU/PopUpArchivos.png}{IU-M00a}{Selector de archivos}

\subsubsection{Elementos relevantes}

    \begin{itemize}
    \item {\bf Menú izquierdo}
        Permite al usuario escojer desde que medio seleccionará el archivo a elegir.
    \end{itemize}

\subsubsection{Acciones relevantes}

    \begin{itemize}
    \item {\bf Browse (Subir un archivo)}
        Cuando se presione el botón \fbox{Browse}, el navegador desplegará una ventana
        emergente para seleccionar un archivo desde el sistema de archivos.

    \item {\bf Subir este archivo}
        Cuando el usuario presione este botón el usuario confirmará la acción de subir
        el archivo que previamente a seleccionado.
    \end{itemize}
  % Form: Selector de archivos

    
\subsubsection{IU-M01: Administración del sitio}

 La página de administración del sitio permite al \refElem{aAdministrador} acceder a todas las
 opciones para administrar la apariencia, seguridad, usuarios, permisos, cursos, plugins y demás
 funcionalidades que brinda moodle. La amplia cantidad de configuraciones están agrupadas en nueve
 categorías principales: {\it administración del sitio, usuarios, grupos, calificaciones, plugins,
 apariencia, servidor, reportes y desarrollo}.

    \IUfig{1}{modulos/moodle/IU/AdministracionSitio.png}{IU-M01}{Administración del sitio}

\subsubsection{Elementos relevantes}

    \begin{itemize}
    \item {\bf Pestañas}
        Permiten acceder al conjunto de herramientas y configuraciones que brindan
        cada una de las categorías principales para la administración del sitio.
    \end{itemize}

\subsubsection{Acciones relevantes}

    \begin{itemize}
    \item {\bf Plugins (pestaña)}
        Permite administrar los plugins así como acceder a las configuraciones particulares de
        cada plugin, redirige a la pantalla \refElem{IU-M01a}.
    \end{itemize}

\clearpage
  % Administración del sitio
    \subsection{IU-M01a: Administración del sitio (Plugins)}

 La página de administración del sitio con la pestaña de plugins seleccionada (ver figura
 \ref{IU-M01a}) permite al \refElem{aAdministrador} instalar, desinstalar y realizar las
 configuraciones que tienen los distintos plugins instalados en moodle. Las configuraciones se
 encuentran organizadas por tipos de plugins.

    \IUfig{1}{modulos/moodleIU/AdministracionSitioPlugins.png}{IU-M01a}{%
        Administración del sitio (plugins)}
 
\subsubsection{Acciones relevantes}

    \begin{itemize}
    \item {\bf Instalar plugins}
        Permite navegar a la pantalla \refElem{IU-M02} para acceder al formulario de
        instalación de plugins e instalar plugins de forma sencilla.

    \item {\bf Vista general de plugins}
        Permite acceder a la lista de plugins instalados en moodle, presente en la pantalla
        \refElem{IU-M03}.
    \end{itemize}

 \noindent 
 Los distintos plugins que se instalarán correspondientes a cada módulo añadirán distintas
 configuraciones las cuales se podrán acceder mediante esta página. A continuación se presenta
 una lista de las opciones que se añadirán.

    \begin{description}[font=\color{primary}]

    \bTerm{tExpCategoria}{Gamedle: Módulo de Experiencia}
            Representa una categoría de configuraciones del módulo de experiencia. Si se oprime
            este enlace entonces redirigirá a la pantalla \refElem{IU-E01}\hfill

        \begin{description}[font={\labelitemi\ \color{black}}]

        \bTerm{tExpSettingsGeneral}{Configuraciones Generales}
            Es el enlace a las configuraciones para habilitar o deshabilitar complementamente
            el sistema de experiencia, redirige a la pantalla \refElem{IU-E02}.

        \bTerm{tExpSettingsVisual}{Configuraciones Visuales}
            Es el enlace a las configuraciones visuales acerca de cómo se visualizan las
            pantallas emergentes al subir de nivel y el nivel actual en el que se encuentra
            un usuario, redirige a la pantalla \refElem{IU-E03}.

        \bTerm{tExpSettingsComportamiento}{Configuraciones de Comportamiento}
            Es el enlace a las configuraciones que específican la cantidad de experiencia que
            los cursos brindarán y la cantidad de experiencia que requiere cada nivel, redirige
            a la pantalla \refElem{IU-E04}.

        \bTerm{tExpSettingsEventos}{Configuraciones de Eventos}
            Es el enlace a las configuración que especifican a que eventos otorgarán experiencia
            y la cantidad de experiencia que entregarán \refElem{IU-E05}.
        \end{description}

    \end{description}

\clearpage
 % Administración del sitio plugins

    
\subsection{IU-M02 Pantalla principal}

 La página de portada, o página principal mostrada en la figura \ref{IU-M02}, es la
 página inicial que ve alguien que llega a un sitio Moodle antes o después de entrar al sitio.
 Típicamente un estudiante verá los cursos, algunos bloques de información, mostrados en un tema.
 En la Barra de navegación y en el menú de navegación (esquina superior izquierda).\\

 \noindent 
 La combinación de las políticas del sitio, autenticación del usuario y configuraciones de la
 portada determinan quién puede llegar a la portada, los elementos que pueden ver y acciones
 que pueden realizar \cite{MoodlePortada}.
    % https://docs.moodle.org/all/es/Portada

    \IUfig{1}{modulos/IUMoodle/Dashboard.png}{IU-M02}{Pantalla Principal de Moodle}

\subsubsection{Elementos relevantes}

    \begin{itemize}
    \item
    {\bf Menú Superior}
        Como su nombre lo indica se encuentra en la parte superior, este elemento se
        encuentra en la mayoría de las pantallas de moodle.

    \item
    {\bf Menú de Navegación}
        Cuando esta visible se encuentra en la parte izquierda de la parte izquierda
        de la mayoría de las pantallas de moodle. Se puede ocultar o mostrar con la
        acción \IUMenu[].

    \item
    {\bf Contenido}
        Tiene todos los demás elementos que conforman el contenido de la pantalla.

    \end{itemize}

\subsubsection{Acciones relevantes}

    \begin{itemize}
    
    \item
    {\bf \IUMenu (Desplegar el menú)}
        Si el menú está oculto, cuando el usuario presione el botón \IUMenu el menú de
        navegación se desplegará.

    \item {\bf \IUMenu (Ocultar Menu)}
        Si el menú está visible, cuando el usuario presione el botón \IUMenu el menú de
        navegación se ocultará.

    \item {\bf \IUAdminSitio Administración del sitio }
        Cuando el menú está visible, el botón de administración del sitio nos permitirá
        navegar a la pantalla \refElem{IU-M03}

    \end{itemize}
  % Instalación de Plugin
    
\subsection{IU-M02a Validación del plugin a instalar}

 La pantalla de validación del plugin a instalar presenta el resultado de la validación de un
 archivo de plugin compreso con base en la regla \refElem{BR-M01}. Esta pantalla dira Si el archivo
 esta formado correctamente o no, además de las acciones adicionales que se llevarán a cabo.

    \IUfig{1}{modulos/moodleIU/InstallZIPValidacion}{IU-M02a}{Validación del plugin a instalar}

\subsubsection{Elementos Relevantes}

    \begin{itemize}
    \item {\bf Validación del plugin}
        Contiene el resultado de la validación del plugin más las acciones
        a realizar para proceder con la instalación del plugin.
        
    \end{itemize}

\subsubsection{Acciones relevantes}

    \begin{itemize}
    \item {\bf Continuar}
        En caso correcto de la validación, este botón permite continuar con la instalación
        rediriginedo a la página \refElem{IU-M02b}.

    \item {\bf Cancelar}
        El botón de cancelar interrumpe el proceso de instalación del plugin, redirigiendo
        a la anterior pantalla \refElem{IU-M01}.
    \end{itemize}

\clearpage
 % Validación de archivo ZIP
    
\subsection{IU-M02b Comprobación de plugins}

 Esta página muestra los plugins que pueden requerir su atención durante un actualización al sitio
 de moodle, como una la instalación o actualización de plugins. La documentación de esta pantalla
 contempla únicamente el caso de instalación/actualización de un plugin a la vez.

    \IUfig{1}{modulos/moodle/IU/InstallConfirm}{IU-M02b}{Comprobación de plugins}

\subsubsection{Elementos relevantes}

   \begin{itemize}
   \item {\bf Lista de plugins a instalar}
        La lista de \refElem[Plugins]{Plugin} que se van a instalar, incluyendo
        sus atributos.
   \end{itemize}

\subsubsection{Acciones relevantes}

    \begin{itemize}
    \item {\bf Actualizar base de datos de Moodle ahora}
        Esta acción permite instalar el plugin en Moodle y correr la secuencia de
        instrucciones establecida por el plugin a instalar. Redirige a la pantalla
        \refElem{IU-M02d}.

    \item {\bf Cancelar las nuevas instalaciones}
        Esta acción permite cancelar las instalaciones o actualizaciones de los plugins,
        redirige a la pantalla \refElem{IU-M02c}

    \item {\bf Cancelar esta instalación}
        A diferencia de la acción anterior, esta acción permite cancelar la instalación
        o actualización de un plugin en particular anterior. Redirige a la pantalla
        \refElem{IU-M02c}.
    \end{itemize}

\clearpage
 % Comprobación de plugibs
    
\subsubsection{IU-M02c: Cancelación de instalación de plugins}

 Esta pantalla tiene el proposito de notificar al \refElem{aAdministrador} de las acciones a
 llevar a cabo en caso de proseguir con la cancelación de la instalación de los plugins. Esta es
 la última confirmación que se le pregunta al administrador antes de la cancelación.

    \IUfig{1}{modulos/moodle/IU/InstallCancelled.png}{IU-M02c}{Comprobación de pluginc}

\subsubsection{Elementos relevantes}

    \begin{itemize}
    \item {\bf Lista de los plugins}
        Contiene la lista de los plugins y la ubicación absoluta de la carpeta
        que contiene todos los archivos de cada plugin que será eliminado.
    \end{itemize}

\subsubsection{Acciones relevantes}

    \begin{itemize}
    \item {\bf Continuar}
        Esta acción confirma la cancelación de los plugins y eliminación de los
        archivos de los mismos. Redirige a la pantalla \refElem{IU-M02}.

    \item {\bf Cancelar}
        Esta acción regresa a la pantalla \refElem{IU-M01} para continuar con la instalación
        de plugins.
    \end{itemize}

\clearpage
 % Cancelación Instalación Plugin
    
\subsection{IU-M02d: Resultado de instalación del plugin}

 Esta pantalla muestra el resultado de la instalación del plugin. Si los plugins
 que son instalados vienen de una fuente confiable, esta pantalla siempre debería
 de aparecer mostrando un resultado existoso, en caso contrario dira que la instalación
 no pudo llevarse a cabo de forma exitosa.

    \IUfig{1}{modulos/moodleIU/InstallResult.png}{IU-M02d}{Resultado de instalación del plugin}

\subsubsection{Elementos relevantes}

    \begin{itemize}
    \item {\bf Mensaje de estado de instalación}
        El mensaje se pinta de color verde o color rojo dependiendo si el 
    \end{itemize}

\subsubsection{Acciones relevantes}

    \begin{itemize}
    \item {\bf Aceptar}
        Si el plugin poseé configuraciones para el administrador entonces esta acción
        redirigirá a la pantalla de configuración correspondiente al \refElem{Plugin},
        en caso contrario redirigirá a la anterior página del sitio de moodle mostrada.
    \end{itemize}

\clearpage
 % Resultado Instalación Plugin

\section{Interfaces del módulo de experiencia}

    
\subsubsection{IU-E01 Configuraciones del módulo de experiencia}

% Descripción ...

    \IUfig{1}{modulos/exp/IU/Settings}{IU-E01}{Configuraciones del módulo de experiencia}

\begin{comment}
\subsubsection{Elementos Relevantes}

    \begin{itemize}
    \item {\bf Lorem ipsum}
        ...
    \end{itemize}

\subsubsection{Acciones relevantes}

    \begin{itemize}
    \item {\bf Lorem ipsum}
        ...
    \end{itemize}
\end{comment}

\clearpage
  % Configuraciones
    
\subsubsection{IU-E02 Configuraciones generales del módulo de experiencia}

% Descripción ...

    \IUfig{1}{modulos/exp/IU/SettingsGenerales}{IU-E02}{%
        Configuraciones generales del módulo de experiencia}

\begin{comment}
\subsubsection{Elementos Relevantes}

    \begin{itemize}
    \item {\bf Lorem ipsum}
        ...
    \end{itemize}

\subsubsection{Acciones relevantes}

    \begin{itemize}
    \item {\bf Lorem ipsum}
        ...
    \end{itemize}
\end{comment}

\clearpage
  % Configuraciones Generales
    
\subsection{IU-E03: Configuraciones Visuales del módulo de experiencia}

 Descripción ...

    \IUfig{1}{modulos/modExpIU/settingsVisuales}{IU-E03}{%
        configuraciones Visuales del módulo de experiencia}

\subsubsection{Elementos Relevantes}

    \begin{itemize}
    \item {\bf Lorem ipsum}
        ...
    \end{itemize}

\subsubsection{Acciones relevantes}

    \begin{itemize}
    \item {\bf Guardar cambios}
        ...

    \item {\bf Cancelar}
        ...
    \end{itemize}

\clearpage
  % Configuraciones Visuales
    
\subsubsection{IU-E03a Módulo de experiencia desactivado}

 Descripción ...

    \IUfig{1}{modulos/exp/IU/settingsExperienceDisabled}{IU-E03a}{%
        Módulo de experiencia desactivado}

\subsubsection{Elementos Relevantes}

    \begin{description}
    \bTerm{tSelectColor}{panel de colores} ...
    \end{description}

\subsubsection{Acciones relevantes}

    \begin{itemize}
    \item {\bf Lorem ipsum}
        ...
    \end{itemize}

\clearpage
 % Configuraciones Mod Exp desactivado
    
\subsection{IU-E04: Configuraciones del sistema de experiencia}

 Descripción ...

    \IUfig{1}{modulos/exp/IU/settingsEsquema}{IU-E04}{%
        Configuraciones del sistema de experiencia}

\subsubsection{Elementos Relevantes}

    \begin{itemize}
    \item {\bf Lorem ipsum}
        ...
    \end{itemize}

\subsubsection{Acciones relevantes}

    \begin{itemize}
    \item {\bf Lorem ipsum}
        ...
    \end{itemize}

\clearpage
  % Configuraciones Esquema

    %
\subsection{IU-E04: Configuraciones del sistema de experiencia}

 Descripción ...

    \IUfig{1}{modulos/exp/IU/settingsEsquema}{IU-E04}{%
        Configuraciones del sistema de experiencia}

\subsubsection{Elementos Relevantes}

    \begin{itemize}
    \item {\bf Lorem ipsum}
        ...
    \end{itemize}

\subsubsection{Acciones relevantes}

    \begin{itemize}
    \item {\bf Lorem ipsum}
        ...
    \end{itemize}

\clearpage
 % Configuraciones de Comportamiento
    %
\subsection{IU-E05 Configuración de eventos con experiencia}

 Descripción ...

    \IUfig{1}{modulos/exp/IU/SettingsEventos}{IU-E05}%
        {Configuracion de eventos con experiencia}

\subsubsection{Elementos Relevantes}

    \begin{itemize}
    \item {\bf Lorem ipsum}
        ...
    \end{itemize}

\subsubsection{Acciones relevantes}

    \begin{itemize}
    \item {\bf Lorem ipsum}
        ...
    \end{itemize}

\clearpage
 % Configuraciones de eventos

\chapter{Diseño}
\section{Diseño de plugins}
\section{Diagrama de componentes}
\section{Diagrama de clases}

\chapter{Pruebas}

    
\TestCase{CPC-E01}{Instalar plugins del esquema de experiencia}

    
\TestCase{CPC-E02}{Realizar configuraciones del módulo de experiencia}

    
\TestCase{CPC-E02-1}{Realizar configuración de visualización de niveles}

    
\TestCase{CPI-E02-1a}{Realizar configuraciones visuales con todos los datos erroneos}

    
\TestCase{CPI-E02-1b}{Configuraciones visuales con formato y nombre de imagen inválidos}

    
\TestCase{CPC-E02-2}{Realizar configuraciones del sistema de experiencia}

    
\TestCase{CPI-E02-2a}{Realizar configuraciones del sistema de experiencia con todos los
datos inválidos}

    
\TestCase{CPI-E02-2b}{Realizar configuraciones del sistema de experiencia con cursos iniciados}

    
\TestCase{CPI-E02-2c}{Realizar configuraciones del sistema de experiencia con alumnos con experiencia establecida}


\begin{comment}
\section{Submódulos}


%\begin{comment} % TERMINOS Y EJEMPLOS DE EXPERIENCIAi
\subsection{Esquema de Experiencia}

 Es la especificación de los conceptos relacionados con los puntos de experiencia, cuales
 son los tipos de incremento y cómo se usan y cuales restricciones se aplican para la
 implementación de los puntos de experiencia, niveles y la acción subir de nivel.
 % y cuántos niveles hay.\\

\subsubsection{Conceptos de puntos de experiencia}
\noindent Como se especificó en el marco teórico, los puntos de experiencia son una unidad que representa la cantidad de actividades completadas por un usuario, sin embargo, se puede referir en diversos contextos a estos puntos. Es por ello que se utilizarán conceptos definidos por los cuadros  \hyperref[table:METerminosExperiencia1]{6.1} y  \hyperref[table:METerminosExperiencia2]{6.2} .\\

\noindent Para poder explicar mejor los conceptos que se tienen para los puntos de experiencia, se utilizará un ejemplo y cada concepto en los cuadros  \hyperref[table:METerminosExperiencia1]{6.1} y  \hyperref[table:METerminosExperiencia2]{6.2} , tendrá su relación con este ejemplo.\\

\noindent Se tiene un usuario ''\textit{U}'', y dicho usuario está actualmente en el nivel ''5'' con ''1000'' puntos de experiencia, y además los puntos de experiencia relacionados con los niveles son los siguientes:
\begin{itemize}
    \item Estando en el nivel 1 se necesitan 1000 puntos de experiencia para subir al nivel 2.
    \item Estando en el nivel 2 se necesitan 1250 puntos de experiencia para subir al nivel 3.
    \item Estando en el nivel 3 se necesitan 1500 puntos de experiencia para subir al nivel 4.
    \item Estando en el nivel 4 se necesitan 1750 puntos de experiencia para subir al nivel 5.
    \item Estando en el nivel 5 se necesitan 2000 puntos de experiencia para subir al nivel 6.
\end{itemize}

\noindent Agreguemos que existe una actividad ''\textit{A}'' que al ser completada otorga 1050 puntos de experiencia.\\

\noindent Con el ejemplo anterior, podemos proceder a definir nuestros conceptos y relacionarlos con el ejemplo para que quede todo más claro.

\begin{table}[h!]
    \label{table:METerminosExperiencia1}
    \centering
        \begin{tabular}{|m{0.2 \textwidth}|m{0.32 \textwidth}|m { 0.42\textwidth}|}\hline
        \textbf{Nombre} & \textbf{Definición} & \textbf{Ejemplo} \\\hline

        Experiencia actual  &
        Es la cantidad de puntos de experiencia que un usuario ha conseguido mientras tiene asociado un cierto nivel. &
        ''\textit{U}'' tiene una \textbf{experiencial actual} de 1000 puntos de experiencia. Y el nivel al cual está asociado es el nivel 5.
        \\\hline

        Experiencia del nivel &
        Es la cantidad de puntos de experiencia que se requieren para subir de un nivel ''\textit{A}'' a un nivel ''\textit{B}'', sin contemplar la \textbf{experiencia actual} del usuario.&
        Con nuestro ejemplo la \textbf{experiencia del nivel} 1, es 1000 mientras que la del nivel 5 es 2000. \\\hline

        Porcentaje actual &
        Es un número entero positivo con rango del 1 al 100, que se calcula de la forma $ \frac{experiencia\_actual}{experiencia\_del\_nivel} * 100 $ &
        ''\textit{U}'' actualmente está en el nivel 5, por lo tanto el cálculo sería: $ \frac{1000}{2000} * 100 = 50 $
        \\\hline

        Experiencia otorgada &
        Es la cantidad de puntos de experiencia que se otorgan al completar una actividad. &
        La actividad ''\textit{A}'' tiene una \textbf{experiencia otorgada} de 1050.
        \\\hline

        Experiencia acumulada &
        Es la cantidad de puntos de experiencia que un usuario ha conseguido a través de los niveles. &
        ''\textit{U}'' está actualmente en el nivel 5, esto quiere decir que ha pasado por los niveles 1, 2, 3 y 4. Cada uno de estos últimos tiene su \textbf{experiencia del nivel}, por lo tanto, nuestro usuario ''\textit{U}'' a conseguido $ 1000 + 1250 + 1500 + 1750 $ puntos para llegar al nivel 5. Además, ''\textit{U}'' tiene una \textbf{experiencia actual} de 1000 puntos.\newline

        Sumando todo lo anterior $ 1000 + 1250 + 1500 + 1750 + 1000  = 6500 $, por lo tanto el usuario ''\textit{U}'' tiene una \textbf{experiencia acumulada} de 6500 \\\hline

        \end{tabular}
    \caption{Conceptos referentes a los puntos de experiencia (parte 1).}
\end{table}
\clearpage
\begin{table}[h!]
    \label{table:METerminosExperiencia2}
    \centering
        \begin{tabular}{|m{0.2 \textwidth}|m{0.32 \textwidth}|m { 0.42\textwidth}|}\hline
        \textbf{Nombre} & \textbf{Definición} & \textbf{Ejemplo} \\\hline

        Experiencia necesaria &
        Es la cantidad de puntos de experiencia que un usuario necesita para que su \textbf{experiencia actual} alcance la \textbf{experiencia del nivel}. Expresado de forma matemática:
        \begin{center}
            experiencia del nivel\newline
            \underline{ - experiencia actual} \newline
            experiencia necesaria \newline
        \end{center}&

        Para  ''\textit{U}'' su \textbf{experiencia necesaria} sería.

        \begin{center}
              2000\newline
            \underline{ - 1000} \newline
            1000\newline
        \end{center}
        \\\hline

        Experiencia sobrante &
        Es la cantidad de puntos de experiencia que rebasan la \textbf{experiencia del nivel} cuando a un usuario recibe \textbf{experiencia otorgada}. Expresado de forma matemática:
        \begin{center}
            experiencia otorgada\newline
            + experiencia actual\newline
            \underline{ - experiencia del nivel} \newline
            experiencia sobrante \newline
        \end{center}&
        Si el alumno ''U'' realiza la actividad ''A'', recibiría una \textbf{experiencia otorgada} de 1050 , sin embargo, su \textbf{experiencia actual} es de 1000. Por lo tanto, tendría una \textbf{experiencia sobrante} de 50 puntos de experiencia.

        \begin{center}
            1050\newline
            + 1000\newline
            \underline{ - 2000} \newline
            50 \newline
        \end{center}\\\hline
        \end{tabular}
    \caption{Conceptos referentes a los puntos de experiencia (parte 2).}
\end{table}

\noindent A lo largo de este capítulo se utilizaran los conceptos anteriores.\\
%\end{comment}

\subsubsection{Tipos de incremento}

 Un tipo de incremento define cómo se calcula la diferencia entre los valores de
 ''\hyperref[table:METerminosExperiencia1]{experiencia del nivel}'' de un nivel
 \textit{$n_i$} y el siguiente a él \textit{$n_(i+1)$}.\\

 \noindent
 Se da soporte a los siguientes tipos de incremento entre los niveles:

    \begin{quote}
    \begin{description}
        \item[Lineal] Establece que la diferencia es una cantidad fija.\\
        Sea $f(n_i)$ una función la cual indica la ''\hyperref[table:METerminosExperiencia1]{experiencia del nivel}'' de un nivel $n_i$. Y sea $e$ una constante que representa la diferencia de la ''\hyperref[table:METerminosExperiencia1]{experiencia del nivel}'' entre 2 niveles continuos. Entonces.
            $$\forall n_i \in Nivel\ | \left(\ f(n_{i+1}) - f(n_i)\ \right) = e$$

        \item[Porcentual] Establece que la diferencia está regida por la siguiente regla:\\
        Sea $n_i$ un nivel de experiencia,  $f(n_i)$ una función la cual indica la ''\hyperref[table:METerminosExperiencia1]{experiencia del nivel}'' de un nivel $n_i$ y  $c$ una constante tal que $1 \leq c \leq 2$ , entonces
            $$\forall n_i \in Nivel\ |\ (c)f(n_i) = f(n_{i+1}).$$
    \end{description}
    \end{quote}


%\begin{comment}
    \begin{quote}
    \begin{description}
        \item[Lineal] Establece que la diferencia es una cantidad fija.\\
        Sea $f(n_i)$ una función la cual indica la cantidad de experiencia requerida para subir al nivel $n_i$. y sea $e$ una constante que representa el incremento de la cantidad de experiencia requerida entre niveles. Entonces.
            $$\forall n_i \in Nivel\ | \left(\ f(n_{i+1}) - f(n_i)\ \right) = e$$

        \item[Porcentual] Establece que la diferencia entre cantidad necesaria de experiencia para subir del nivel $i$ al nivel $i+1$ está regido por la siguiente regla:\\
        Sea $n_i$ un nivel de experiencia y $n_{i+1}$ el siguiente nivel, y sea $c$ una constante tal que $1 \leq c \leq 2$ , entonces
            $$\forall n_i \in Nivel\ |\ (c)f(n_i) = f(n_{i+1}).$$
    \end{description}
    \end{quote}
%\end{comment}

\subsubsection{Esquema configurable}

 Se quiere que el administrador de la página pueda configurar:
 \begin{quote}
 \begin{itemize}
    \item{La ''\hyperref[table:METerminosExperiencia1]{experiencia del nivel}'' del nivel 1.}
    \item {El tipo de incremento.}
    \item {La cantidad de los puntos de experiencia en el incremento.}
    \item {La ''\hyperref[table:METerminosExperiencia1]{experiencia otorgada}'' que da resolver cualquier actividad.}
 \end{itemize}
 \end{quote}

\subsection{Submódulo de Niveles}

 Presenta a los alumnos su progreso utilizando un sistema de niveles que se van alcanzado
 obteniendo puntos de experiencia. Al alcanzar un nuevo nivel la barra que muestra la
 cantidad de experiencia del nivel se actualizará.
 % y cada vez que se alcanza un nivel, los puntos de experiencia se regresan a cero.

    \begin{quote}
    \begin{description}
    \item[Objetivo] \hfill\\
        Mostrar a los alumnos el nivel actual de experiencia que tienen y el avance que tienen de ese mismo nivel.
        %Mostrar el nivel actual que tienen los alumnos, así como el avance que tienen en ese mismo nivel.

        %Proveer información al alumno que indique la cantidad de tiempo y esfuerzo que le ha dedicado a la plataforma.

    \item[Principios a los que brinda soporte:] \hfill
        \begin{itemize}
            \item 2 \principioII
            \item 6 \principioVI
        \end{itemize}
    \end{description}
    \end{quote}

%\begin{comment}%

\subsection{Submódulo de Barra de Progreso}

Muestra al alumno el progreso que lleva en un curso usando un valor de 0\% a 100\% dependiendo de los ejercicios que haya hecho del curso o del tiempo que haya transcurrido.

    \begin{quote}
    \begin{description}
        \item[Objetivo] \hfill\\
            Proveer información al alumno que indique el tiempo y esfuerzo que le ha dedicado a un curso, así como el que le falta por dedicar.

        \item[Principios a los que brinda soporte:] \hfill
        \begin{itemize}
            \item 2 \principioII
        \end{itemize}
    \end{description}
    \end{quote}
%\end{comment}%

%\subsection{Comportamiento en Moodle} ESTO ES DISEÑO

%\subsubsection{Plugin}
%\subsubsection{Opciones para el administrador}

\clearpage
%\subsection{Reglas de Uso} % Reglas de Negocio | But there's no business

\section{Interfaces}

\subsection*{IU-E01 Bloque de experiencia}
\label{IUE01}

    Visualización en Moodle del bloque de experiencia.

%    \addfigure{1}{IU/IU_E01_Bloque_Experiencia}{fig:IUE01}{IU-E01: Bloque de experiencia.}

    \noindent {\bf Elementos:}
    \begin{quote}
    \begin{description}
    	\item[Nombre del bloque] Nombre para diferenciar los bloques en las interfaces de Moodle.
    	\item[Imagen del nivel] Imagen del nivel configurada por el administrador.
    	\item[Número del nivel] Número entero positivo que representa el nivel actual del usuario.
    	\item[Barra de progreso] Barra que se llena según el ''\hyperref[table:METerminosExperiencia1]{porcentaje actual}'' del usuario
    	\item[Experiencia actual del nivel] Número entero positivo que representa la ''\hyperref[table:METerminosExperiencia1]{experiencia actual}'' del usuario.
    	\item[Experiencia total del nivel] Número entero positivo que representa la ''\hyperref[table:METerminosExperiencia1]{experiencia del nivel}''.
    	\item[Experiencia acumulada] Número entero positivo que representa la cantidad de ''\hyperref[table:METerminosExperiencia1]{experiencia acumulada}''.
    \end{description}
    \end{quote}
	\clearpage

\subsection*{IU-E02 Subir de nivel}
\label{IUE02}

    Esta interfaz es una ventana emergente que sale siempre que el usuario tenga ''\hyperref[table:METerminosExperiencia1]{experiencia sobrante}'' al ejecutar el \hyperref[CU-E01]{CU-E01 Recibir experiencia}.

%    \addfigure{1}{IU/IU_E02_PopUp_SubirNivel}{fig:IUE02}{IU-E02: Ventana emergente que aparece al subir de nivel.}

    \noindent {\bf Elementos:}
    \begin{quote}
    \begin{description}
    	\item[Mensaje] Mensaje de felicitaciones configurado por el administrador.
    	\item[Imagen del nivel] Imagen del nivel configurada por el administrador.
    	\item[Número del nivel] Número entero positivo que representa el nivel actual del usuario.
    	\item[Nombre] Nombre que reciben los niveles, que es configurado por el administrador.
    	\item[Descripción] Descripción asociada a los niveles, que es configurada por el administrador.
    \end{description}
    \end{quote}
	\clearpage

\subsection*{IU-E03 Configuración del esquema de experiencia}
\label{IUE03}
    Esta interfaz es definida por el archivo \textbf{settings.php}, sin embargo, está es generada por Moodle.\\
    En esta interfaz el administrador puede modificar los aspectos visuales que tienen las interfaces \hyperref[IUE01]{IU-E01 Bloque de experiencia} y \hyperref[IUE02]{IU-E02 Subir de nivel} , y configurar el tipo de incremento, cuanto incremento hay por nivel, la ''\hyperref[table:METerminosExperiencia1]{experiencia del nivel}'' del nivel 1 y la ''\hyperref[table:METerminosExperiencia1]{experiencia otorgada}'' que darán todas las actividades.\\
    Esta interfaz también activa y desactiva el funcionamiento del módulo de experiencia.\\

    %\addfigure{1}{IU/IU_E03_config_parte1}{fig:IUE03_1}{Interfaz donde se  configura el esquema de experiencia parte 1.}


    \noindent {\bf Elementos importantes:}
    \begin{quote}
    \begin{description}
    	\item[Opción \#1] Permite activar y desactivar el módulo de experiencia.
    	\item[Opción \#2] Permite seleccionar entre los 2 tipos de incremento 'Lineal' y 'Porcentual'.
    	\item[Opción \#3] Permite asignar cuanta experiencia habrá de diferencia entre la ''\hyperref[table:METerminosExperiencia1]{experiencia del nivel}'' del nivel \textit{A} y la ''\hyperref[table:METerminosExperiencia1]{experiencia del nivel}'' del nivel \textit{B}.
    	\item[Opción \#4] Permite asignar la ''\hyperref[table:METerminosExperiencia1]{experiencia del nivel}'' 1.
    	\item[Opción \#5] Permite asignar la ''\hyperref[table:METerminosExperiencia1]{experiencia otorgada}'' que entregarán todas las actividades.
    \end{description}
    \end{quote}

\clearpage
    %\addfigure{1}{IU/IU_E03_config_parte2}{fig:IUE03_2}{Interfaz donde se  configura el esquema de experiencia parte 2.}


    \noindent {\bf Elementos importantes:}
    \begin{quote}
    \begin{description}
    	\item[Opción \#6] Permite asignar un nombre por defecto a los niveles.
    	\item[Opción \#7] Permite asignar un mensaje de felicitaciones por defecto al subir de nivel.
    	\item[Opción \#8] Permite asignar una descripción por defecto al subir de nivel.
    	\item[Opción \#9] Permite asignar el color por defecto del número de nivel.
    \end{description}
    \end{quote}


\clearpage

    %\addfigure{1}{IU/IU_E03_config_parte3}{fig:IUE03_3}{Interfaz donde se  configura el esquema de experiencia parte 3.}

    \noindent {\bf Elementos importantes:}
    \begin{quote}
    \begin{description}
    	\item[Opción \#10] Permite asignar el color por defecto de la barra de progreso.
    	\item[Opción \#11] Permite asignar una imagen por defecto a los niveles.
    	\item[Botón \#1 'Guardar cambios'] Con este botón el administrador puede guardar los cambios que haya hecho en las opciones de la interfaz.
    \end{description}
    \end{quote}

\clearpage

\subsection*{IU-M01 Ver intento de examen}
\label{IUM01}

    Esta interfaz es de Moodle, sin embargo, es utilizada para el caso de uso \hyperref[CU-E01]{CU-E01 Recibir experiencia}.\\

    \noindent En esta interfaz los alumnos pueden ver un resumen de su intento para resolver el examen, así como poder reanudar el intento para terminar o corregir, o para enviar las respuestas para su revisión.

    %\addfigure{1}{IU/IU_M01_VerIntento}{fig:IUM01}{IU-M01: Ver intento.}

    \noindent {\bf Elementos importantes:}
    \begin{quote}
    \begin{description}
    	\item[Botón \# 1 'Regresar al intento'] Con este botón el usuario puede seguir contestando el examen.
    	\item[Botón \# 2 'Enviar todo y terminar'] Con este botón el usuario indica que terminó de responder el examen y que quiere enviar las respuestas para su revisión.
    \end{description}
    \end{quote}
	\clearpage

\subsection*{IU-M02 Confirmación de envío de intento}
\label{IUM02}

    Esta interfaz es de Moodle, sin embargo, es utilizada para el caso de uso  \hyperref[CU-E01]{CU-E01 Recibir experiencia}.\\

    \noindent Esta interfaz es una ventana emergente que permite al usuario pensárselo una segunda vez antes de subir sus respuestas para ser calificadas. Esto inclusive por si le da por error.\\

    %\addfigure{1}{IU/IU_M02_PopUp_Confirmacion}{fig:IUM02}{IU-M02: Confirmación de envío de intento.}

    \noindent {\bf Elementos importantes:}
    \begin{quote}
    \begin{description}
    	\item[Botón \#1 'Enviar todo y terminar'] Con este botón el usuario puede reafirmar que quiere mandar sus respuestas para ser calificadas.
    	\item[Botón \#2 'Cancelar'] Con este botón el usuario puede cancelar el subir sus respuestas.
    	\item[Botón \#3 'X'] Este botón tiene el mismo efecto que el Botón \# 2 'Cancelar'.
    \end{description}
    \end{quote}
	\clearpage

\subsection*{IU-M03 Intento calificado}
\label{IUM03}

    Esta interfaz es de Moodle, sin embargo, es utilizada para el caso de uso   \hyperref[CU-E01]{CU-E01 Recibir experiencia}.\\

    \noindent Si un usuario envía su intento para ser calificado
    %y dicho intento puede ser calificado por el sistema,
    se muestra esta interfaz con la calificación de su intento. Algo importante es que la interfaz  \textbf{IU-E01 Bloque de experiencia} está visible, haciendo posible mostrar la actualización de los datos del usuario.\\

    %\addfigure{1}{IU/IU_M03_FinalizarIntento}{fig:IUM03}{IU-M03: Intento calificado.}


	\clearpage

\subsection*{IU-M04 Sección de plugins}
\label{IUM04}

    Esta interfaz es de Moodle, sin embargo, es utilizada para el caso de uso \hyperref[CU-E02]{CU-E02 Configuración esquema de experiencia} .\\

    \noindent Esta interfaz contiene la lista de todos los plugins que tienen configuraciones globales, donde cada uno de sus elementos es un enlace a la página de configuración respectiva a cada plugin. Además en su primera sección tiene las opciones para ver, manejar e instalar plugins.\\

%    \addfigure{1}{IU/IU_M04_SeccionPlugins}{fig:IUM04}{IU-M04: Sección donde están todos los plugins con configuraciones globales.}
	\clearpage


\subsection*{Moodle: IU-M05 Crear Curso}

 El objetivo de esta pantalla (Figura \ref{moodle:nuevoCurso}) es permitirle al profesor crear un nuevo curso, especificando los datos generales del curso, la descripción, apariencia, tamaño de los archivos, el seguimiento de finalización, los grupos, renombre de roles y las marcas vinculadas al curso.

%    \addfigureB{1}{IU/mCrearCurso}{moodle:nuevoCurso}{Moodle IU-M05 Crear curso}

    {\bf Elementos}
    \begin{itemize}
        \item \bf{Datos Generales} es un formulario que contiene el nombre completo y corto del curso (obligatorios), su identificador, categoría, visibilidad, así como las fechas de inicio y término del curso.
        \item \bf{Descripción}
        \item \bf{Formato de curso}
        \item \bf{Apariencia}
        \item \bf{Archivos y subidas}
        \item \bf{Seguimiento de finalización}
        \item \bf{Grupos}
        \item \bf{Renombrar rol}
        \item \bf{Marcas}
    \end{itemize}


\chapter{Diseño}

\subsection{Diagrama de Clases}

    En la figura \ref{fig:classesXP} se muestra el diagrama de clases, los archivos {\it lib, events, settings, version} y los {\it módulos AMD} son representados mediante el uso de clases. Para facilitar la lectura del diagrama se representa a móodle como un paquete completo, el cual lee los distintos archivos y clases que requiere el plugin para funcionar.

%    \addfigure{1}{diagrams/classesExp}{fig:classesXP}{Diagrama de clases del Módulo de Experiencia}
\clearpage

\subsection{Diagrama de componentes}

    En la figura \ref{fig:bloques1} se muestra el diagrama de componentes del Módulo de experiencia que contiene como interactúa el Módulo con la plataforma Moodle.

%    \addfigure{1}{diagrams/bloques1}{fig:bloques1}{Diagrama de componentes del Módulo de Experiencia}

\clearpage
\subsection{Diagramas de Secuencia}
\subsection*{DS-E2: Crear curso con experiencia}

    Para diseñar la forma en que se ejecuta el caso de uso CU-E2, se tomó en consideración el flujo normal de eventos emitidos cuando se crea un curso en moodle. Los eventos emitidos en orden cronológico son {\it course\_created}, {\it course\_section\_created} y {\it enrol\_instance\_created}.\\

    \noindent En la figura \ref{ds:e2} se detalla la interacción entre el core de moodle, los eventos emitidos, y las clases del plugin {\bf Format Gamedle}.

\chapter{Pruebas}
\end{comment}
