
% \ucstEnEdicion     Al terminar una revisión/aprobación con observaciones
%                    y al inicio del CU.
%
% \ucstEnRevision    Al terminar la edición del CU (version += 0.1).
% \ucstEnAprobacion  Al pasar la revision sin observaciones.
% \ucstAprobado      Al ser aprobado por el usuario (version += 1.0)

%\addfigure[(adaptado de {\it For The Win} \cite{ForTheWin})]%
%    {.4}{investigacion/images/forthewin}{fig:ForTheWin}%
%    {Jerarquía de elementos de juego segun For The Win}

\begin{UseCase}[%
Autor/Ricardo Naranjo,%
Version/0.1,%
Estado/\ucstEnRevision]%
%
{CU-C07}{Ver estado de instancia de actividad (Competencia uno contra uno)}{%
%
 Permite al \refElem{aEstudiante}, \refElem{aProfesor} y al \refElem{aAdministrador} ver el estado actual de una instancia de la actividad competencia uno contra uno en el curso.
 Este caso de uso es una extensión del caso de uso {\it Ver curso} que es propio de moodle.}

	\UCitem[control]{Revisor}{ Sin asignar }
	\UCitem[control]{Último cambio}{ 13/NOV/19 }

 \UCsection{Atributos}

    \UCitem{Actor(es)}{%
        \refElem{aEstudiante},
        \refElem{aProfesor},
        \refElem{aAdministrador}
    }

	\UCitems{Propósito}{%
        \Titem Permitir al \refElem{aEstudiante}, \refElem{aProfesor} y al \refElem{aAdministrador} ver el estado actual de una instancia de la actividad de competencia uno contra uno.
	}

	\UCitem{Entradas}{\imprimeUC{entrada}}

	\UCitems{Origen}{%
        \Titem Mouse
	}

	\UCitem{Salidas}{\imprimeUC{salida}}

	\UCitem{Destino}{%
		\refElem{IU-C02}
	}

	\UCitems{Precondiciones}{%
        \Titem El plugin de competencia uno contra uno debe estar instalado en moodle.
        \Titem La instancia de la actividad de competencia uno contra uno debe estar creada.
        % Realizar el caso de uso "listar actividades disponibles?"
        % \Titem Si se trata de una actualización de un plugin la versión de este debe
               % cumplir con la regla \refElem{BR-M02}.
	}

	\UCitems{Postcondiciones}{%
        \Titem La pantalla principal de la instancia de la actividad de competencia uno contra uno \refElem{IU-C02} debe mostrar los datos pertinentes al usuario que realizó el caso de uso.%

	}

	\UCitem{Reglas de negocio}{\imprimeUC{regla}}

	\UCitems{Errores}{%
	}

	% \UCitem{Viene de}{% Indicar si el Caso de uso es primario o se extiende de otro. La mayoría se
					  % extienden de Login.
		% EJEMPLO: \refIdElem{PY-CU1} o Caso de uso primario.
	% 	\TODO Especificar.
	% }

 \UCsection[design]{Datos de Diseño}

	\UCitems[design]{Casos de Prueba}{%
        \Titem \refElem{CPC-C07}
	}

 \UCsection[admin]{Datos de Administración de Requerimiento}

	\UCitem[admin]{Observaciones}{}

\end{UseCase}

\subsubsection{Trayectorias del caso de uso}

\begin{UCtrayectoria}%
%

    \Actor Presiona el nombre de la instancia a la que quiere acceder en la pantalla \refElem{IU-M07}.

    \Sistema Valida que las partidas en curso cumplan con la \regla{BR-C01} y las partidas que no la cumplen son terminadas.

    \Sistema Verifica que el actor esté dado de alto como usuario gamificado. \refTray{B}.

    \Sistema Valida que la bandera \refElem{comp-1v1-gmcompvs.apuestas-activas} esté activa. \refTray{A}.
    \label{CU-C07-mostrar informacion}
    \Sistema Redirige a la pantalla principal de la instancia \refElem{IU-C02}.

    \Sistema Muestra el \salida{número de victorias}.

    \Sistema Muestra en el área {\bf Compañeros del curso} un bloque compañero por cada \refElem{aEstudiante} que esté inscrito en el curso y que cumpla con la \regla{BR-C02}, dicho bloque se compone de la \salida{imagen de perfil compañero}, \salida{nombre compañero}, campo {\bf Monedas a apostar} y el botón {\bf Desafiar}.

    \Sistema Muestra en el área {\bf Desafíos pendientes} un bloque desafiante por cada \refElem{aEstudiante}, \refElem{aProfesor} o \refElem{aAdministrador} que haya desafiado al \refElem{aEstudiante} que está realizando el caso de uso, dicho bloque se compone de la \salida{imagen de perfil desafiante}, \salida{nombre desafiante}, campo {\bf Monedas a apostar} y el botón {\bf Aceptar desafío}.

\end{UCtrayectoria}

\begin{UCtrayectoriaA}[Fin del caso de uso]{A}{La bandera \refElem{comp-1v1-gmcompvs.apuestas-activas} está inactiva}

  \Sistema Redirige a la pantalla principal de la instancia \refElem{IU-C02a}.

  \Sistema Muestra el número de victorias.

  \Sistema Muestra en el área {\bf Compañeros del curso} un bloque por cada \refElem{aEstudiante} que esté inscrito en el curso y que cumpla con la \regla{BR-C02}, dicho bloque se compone de la imagen de perfil compañero, nombre compañero y el botón {\bf Desafiar}.

  \Sistema Muestra en el área {\bf Desafíos pendientes} un bloque desafiante por cada \refElem{aEstudiante}, \refElem{aProfesor} o \refElem{aAdministrador} que haya desafiado al \refElem{aEstudiante} que está realizando el caso de uso, dicho bloque se compone de la imagen de perfil desafiante, nombre desafiante y el botón {\bf Aceptar desafío}.

\end{UCtrayectoriaA}


\begin{UCtrayectoriaA}{B}{El actor no está dado de alta como usuario gamificado (\refElem{xp-user})}

  \Sistema Registra al actor en la entidad (\refElem{xp-user}).
    \item Se regresa al paso \ref{CU-C07-mostrar informacion} de la trayectoria principal.


\end{UCtrayectoriaA}

\subsubsection{Puntos de extensión}

\UCExtensionPoint{Ver historial de las partidas}{%

    El \refElem{aAdministrador}, \refElem{aProfesor} o \refElem{aEstudiante} desea ver su historial de las partidas de competencia uno contra uno.
%
    }{Al final de la trayectoria principal del caso de uso.
%
    }{\refElem{CU-C08}}


\UCExtensionPoint{Ver tabla de posiciones}{%

    El \refElem{aAdministrador}, \refElem{aProfesor} o \refElem{aEstudiante} desea ver la tabla de posiciones de una instancia de competencia uno contra uno.
%
    }{Al final de la trayectoria principal del caso de uso.
%
    }{\refElem{CU-C09}}

  \UCExtensionPoint{Desafiar a un estudiante con apuestas}{%

      El \refElem{aAdministrador}, \refElem{aProfesor} o \refElem{aEstudiante} desafiar a un estudiante y apostar una cantidad de monedas.
  %
      }{Al final de la trayectoria principal del caso de uso.
  %
      }{\refElem{CU-C10}}


\UCExtensionPoint{Desafiar a un estudiante sin apuestas}{%

    El \refElem{aAdministrador}, \refElem{aProfesor} o \refElem{aEstudiante} desafiar a un estudiante.
%
    }{Al final de la \refTray{A}.
%
    }{\refElem{CU-C11}}
