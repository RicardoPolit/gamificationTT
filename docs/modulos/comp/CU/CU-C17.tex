
% \ucstEnEdicion     Al terminar una revisión/aprobación con observaciones
%                    y al inicio del CU.
%
% \ucstEnRevision    Al terminar la edición del CU (version += 0.1).
% \ucstEnAprobacion  Al pasar la revision sin observaciones.
% \ucstAprobado      Al ser aprobado por el usuario (version += 1.0)

%\addfigure[(adaptado de {\it For The Win} \cite{ForTheWin})]%
%    {.4}{investigacion/images/forthewin}{fig:ForTheWin}%
%    {Jerarquía de elementos de juego segun For The Win}

\begin{UseCase}[%
Autor/Ricardo Naranjo,%
Version/0.1,%
Estado/\ucstEnRevision]%
%
{CU-C17}{Aceptar un desafío sin apostar (Competencia uno contra uno)}{%
%
 Permite al \refElem{aEstudiante} aceptar un desafío.
 Este caso de uso es una extensión del caso de uso \refElem{CU-C07} en la trayectoria alternativa A.}

	\UCitem[control]{Revisor}{ Sin asignar }
	\UCitem[control]{Último cambio}{ 13/NOV/19 }

 \UCsection{Atributos}

    \UCitem{Actor(es)}{%
        \refElem{aEstudiante}
    }

	\UCitems{Propósito}{%
        \Titem Permitir al \refElem{aEstudiante} aceptar un desafío.
        \Titem Permitir al \refElem{aEstudiante} terminar un desafío.
	}

	\UCitem{Entradas}{\imprimeUC{entrada}}

	\UCitems{Origen}{%
        \Titem Mouse
        \Titem Teclado
	}

	\UCitem{Salidas}{\imprimeUC{salida}}

	\UCitem{Destino}{%
		\refElem{IU-C11}
	}

	\UCitems{Precondiciones}{%
        \Titem El plugin de competencia uno contra uno debe estar instalado en moodle.
        \Titem La instancia de la actividad de competencia uno contra uno debe estar creada.
        % Realizar el caso de uso "listar actividades disponibles?"
        % \Titem Si se trata de una actualización de un plugin la versión de este debe
               % cumplir con la regla \refElem{BR-M02}.
	}

	\UCitems{Postcondiciones}{%
        \Titem Se muestra en la pantalla de historial del usuario \refElem{IU-C07} la nueva partida.%

	}

	\UCitem{Reglas de negocio}{\imprimeUC{regla}}

	\UCitems{Errores}{%
	}

	% \UCitem{Viene de}{% Indicar si el Caso de uso es primario o se extiende de otro. La mayoría se
					  % extienden de Login.
		% EJEMPLO: \refIdElem{PY-CU1} o Caso de uso primario.
	% 	\TODO Especificar.
	% }

 \UCsection[design]{Datos de Diseño}

	\UCitems[design]{Casos de Prueba}{%
        \Titem \refElem{CPC-C17}
	}

 \UCsection[admin]{Datos de Administración de Requerimiento}

	\UCitem[admin]{Observaciones}{}

\end{UseCase}

\subsubsection{Trayectorias del caso de uso}

\begin{UCtrayectoria}%
%

    \Actor Presiona el botón {\bf Aceptar desafío} en el bloque del usuario del que quiere aceptar el desafío en la pantalla \refElem{IU-C02a} en la sección "Desafíos pendientes"..

    \Sistema Inicia la partida y establece los valores correspondientes en \refElem{comp-1v1-gmdl-participacion} ( \entrada{comp-1v1-gmdl-participacion.fecha-inicio} ).

    \Sistema Redirige a la pantalla del cuestionario del desafío \refElem{IU-C09}.

    \Actor Contesta las preguntas mostradas. \refTray{A}
    \label{CU-C10-contesta-cuestionario}
    \Actor Presiona el botón {\bf Terminar}.

    \Sistema Evalúa las respuestas ingresadas y calcula su puntuación.

    \Sistema Establece los valores correspondientes de la participación \refElem{comp-1v1-gmdl-participacion} (
    \entrada{comp-1v1-gmdl-participacion.fecha-fin},
    \entrada{comp-1v1-gmdl-participacion.puntuacion}).

    \Sistema Valida que el usuario que lo desafió haya terminado de contestar su cuestionario. \refTray{B}

    \Sistema Calcula y otorga puntaje extra de acuerdo al tiempo tardado en contestar el cuestionario.
    \Sistema Comprueba que el actor haya obtenido un mayor puntaje que el usuario desafiante. \refTray{C}
    \Sistema Lanza el evento para dar las monedas que le corresponden al actor.
    \Sistema Lanza el evento para dar la experiencia que le corresponde al actor.
    \Sistema Redirige a la pantalla \refElem{IU-C11}.
    \Sistema Muestra el puntaje de cada participante \salida{comp-1v1-gmdl-participacion.puntuacion}.

\end{UCtrayectoria}

\begin{UCtrayectoriaA}[Fin del caso de uso]%
  {A}{El actor desea salir del cuestionario sin terminarlo}

  \Actor Presiona cualquier botón, excepto el botón de {\bf Terminar}, que haga que abandone la pantalla.
  \Sistema Muestra mensaje de alerta \refElem{IU-C13}.
  \Actor Presiona el botón {\bf Abandonar}. \refTray{D}
  \Sistema Establece el valor correspondiente de la participación \refElem{comp-1v1-gmdl-participacion} (
  \refElem{comp-1v1-gmdl-participacion.fecha-fin}).
  \Sistema Redirige a la pantalla elegida por el actor.

\end{UCtrayectoriaA}

\begin{UCtrayectoriaA}[Fin del caso de uso]%
  {B}{El usuario que lo desafió no ha terminado de contestar su cuestionario}

  \Sistema Redirige a la pantalla \refElem{IU-C10}

\end{UCtrayectoriaA}

\begin{UCtrayectoriaA}[Fin del caso de uso]%
  {C}{El usuario desafiante obtuvo un mayor puntaje que el actor}

  \Sistema Lanza el evento para dar las monedas que le corresponden al usuario desafiante.
  \Sistema Lanza el evento para dar la experiencia que le corresponde al usuario desafiante.
  \Sistema Redirige a la pantalla \refElem{IU-C12}.
  \Sistema Muestra el puntaje de cada participante \refElem{comp-1v1-gmdl-participacion.puntuacion}.

\end{UCtrayectoriaA}

\begin{UCtrayectoriaA}{D}{El actor no quiere abandonar el cuestionario}

  \Actor Presiona el botón {\bf Cancelar}.
  \Sistema Cierra el mensaje de alerta.
  \Sistema Regresa al paso \ref{CU-C10-contesta-cuestionario}

\end{UCtrayectoriaA}
