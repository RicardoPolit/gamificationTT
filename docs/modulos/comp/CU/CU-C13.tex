
% \ucstEnEdicion     Al terminar una revisión/aprobación con observaciones
%                    y al inicio del CU.
%
% \ucstEnRevision    Al terminar la edición del CU (version += 0.1).
% \ucstEnAprobacion  Al pasar la revision sin observaciones.
% \ucstAprobado      Al ser aprobado por el usuario (version += 1.0)

%\addfigure[(adaptado de {\it For The Win} \cite{ForTheWin})]%
%    {.4}{investigacion/images/forthewin}{fig:ForTheWin}%
%    {Jerarquía de elementos de juego segun For The Win}

\begin{UseCase}[%
Autor/Ricardo Naranjo,%
Version/0.1,%
Estado/\ucstEnRevision]%
%
{CU-C13}{Desafiar al sistema (Competencia uno contra sistema)}{%
%
 Permite al \refElem{aEstudiante}, \refElem{aProfesor} y al \refElem{aAdministrador} desafiar al sistema.
 Este caso de uso es una extensión del caso de uso \refElem{CU-C12}.}

	\UCitem[control]{Revisor}{ Sin asignar }
	\UCitem[control]{Último cambio}{ 13/NOV/19 }

 \UCsection{Atributos}

    \UCitem{Actor(es)}{%
        \refElem{aEstudiante},
        \refElem{aProfesor},
        \refElem{aAdministrador}
    }

	\UCitems{Propósito}{%
        \Titem Permitir al \refElem{aEstudiante}, \refElem{aProfesor} y al \refElem{aAdministrador} desafiar al sistema.
        \Titem Permitir al \refElem{aEstudiante}, \refElem{aProfesor} y al \refElem{aAdministrador} terminar un desafío.
	}

	\UCitem{Entradas}{\imprimeUC{entrada}}

	\UCitems{Origen}{%
        \Titem Mouse
        \Titem Teclado
	}

	\UCitem{Salidas}{\imprimeUC{salida}}

	\UCitem{Destino}{%
		\refElem{IU-C10}
	}

	\UCitems{Precondiciones}{%
        \Titem El plugin de competencia uno contra sistema debe estar instalado en moodle.
        \Titem La instancia de la actividad de competencia uno contra sistema debe estar creada.
        % Realizar el caso de uso "listar actividades disponibles?"
        % \Titem Si se trata de una actualización de un plugin la versión de este debe
               % cumplir con la regla \refElem{BR-M02}.
	}

	\UCitems{Postcondiciones}{%
        \Titem Se muestra en la pantalla de historial del usuario \refElem{IU-C14} el desafío nuevo.%
	}

	\UCitem{Reglas de negocio}{\imprimeUC{regla}}

	\UCitems{Errores}{%
	}

	% \UCitem{Viene de}{% Indicar si el Caso de uso es primario o se extiende de otro. La mayoría se
					  % extienden de Login.
		% EJEMPLO: \refIdElem{PY-CU1} o Caso de uso primario.
	% 	\TODO Especificar.
	% }

 \UCsection[design]{Datos de Diseño}

	\UCitems[design]{Casos de Prueba}{%
        \Titem \refElem{CPC-C13}
	}

 \UCsection[admin]{Datos de Administración de Requerimiento}

	\UCitem[admin]{Observaciones}{}

\end{UseCase}

\subsubsection{Trayectorias del caso de uso}

\begin{UCtrayectoria}%
%

    \Actor Selecciona la dificultad deseada en el campo {\bf Seleccione una dificultad} de la pantalla \refElem{IU-C04}.

    \Actor Presiona el botón {\bf Empezar}.

    \Sistema Inicia la partida y establece los valores correspondientes \refElem{comp-cpu-gmdl-intento}( \entrada{comp-cpu-gmdl-intento.gmdl-dificultad-cpu-id}, \entrada{comp-cpu-gmdl-intento.gmdl-comp-cpu-id}, \entrada{comp-cpu-gmdl-intento.gmdl-usuario-id}, \entrada{comp-cpu-gmdl-intento.fecha-inicio} ).

    \Sistema Redirige a la pantalla del cuestionario del desafío \refElem{IU-C09}.

    \Actor Contesta las preguntas mostradas.

    \Actor Presiona el botón {\bf Terminar}.

    \Sistema Evalúa las respuestas ingresadas y calcula su puntuación.

    \Sistema Contesta el cuestionario y calcula su puntuación.

    \Sistema Establece los valores correspondientes del intento \refElem{comp-cpu-gmdl-intento} (
    \entrada{comp-cpu-gmdl-intento.fecha-fin},
    \entrada{comp-cpu-gmdl-intento.puntuacion-cpu},
    \entrada{comp-cpu-gmdl-intento.puntuacion-usuario}).

    \Sistema Valida que el actor haya sacado más o igual puntuación que el sistema. \refTray{A}

    \Sistema Lanza el evento para dar las monedas que le corresponden al actor.
    \Sistema Lanza el evento para dar la experiencia que le corresponde al actor.

    \Sistema Redirige a la pantalla \refElem{IU-C11}.
    \Sistema Muestra el puntaje del actor y del sistema \salida{comp-cpu-gmdl-intento.puntuacion-usuario}, \salida{comp-cpu-gmdl-intento.puntuacion-cpu}.

\end{UCtrayectoria}

\begin{UCtrayectoriaA}[Fin del caso de uso]%
  {A}{El sistema obtuvo un mayor puntaje que el actor}

  \Sistema Redirige a la pantalla \refElem{IU-C12}.
  \Sistema Muestra el puntaje del actor y del sistema \refElem{comp-cpu-gmdl-intento.puntuacion-usuario}, \refElem{comp-cpu-gmdl-intento.puntuacion-cpu}.

\end{UCtrayectoriaA}
