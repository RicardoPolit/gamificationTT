
% \ucstEnEdicion     Al terminar una revisión/aprobación con observaciones
%                    y al inicio del CU.
%
% \ucstEnRevision    Al terminar la edición del CU (version += 0.1).
% \ucstEnAprobacion  Al pasar la revision sin observaciones.
% \ucstAprobado      Al ser aprobado por el usuario (version += 1.0)

%\addfigure[(adaptado de {\it For The Win} \cite{ForTheWin})]%
%    {.4}{investigacion/images/forthewin}{fig:ForTheWin}%
%    {Jerarquía de elementos de juego segun For The Win}

\begin{UseCase}[%
Autor/Ricardo Naranjo,%
Version/0.1,%
Estado/\ucstEnRevision]%
%
{CU-C12}{Ver estado de instancia de actividad (Competencia uno contra sistema)}{%
%
 Permite al \refElem{aEstudiante}, \refElem{aProfesor} y al \refElem{aAdministrador} ver el estado actual de una instancia de la actividad competencia uno contra sistema en el curso.
 Este caso de uso es una extensión del caso de uso {\it Ver curso} que es propio de moodle.}

	\UCitem[control]{Revisor}{ Sin asignar }
	\UCitem[control]{Último cambio}{ 13/NOV/19 }

 \UCsection{Atributos}

    \UCitem{Actor(es)}{%
        \refElem{aEstudiante},
        \refElem{aProfesor},
        \refElem{aAdministrador}
    }

	\UCitems{Propósito}{%
        \Titem Permitir al \refElem{aEstudiante}, \refElem{aProfesor} y al \refElem{aAdministrador} ver el estado actual de una instancia de la actividad de competencia uno contra sistema.
	}

	\UCitem{Entradas}{\imprimeUC{entrada}}

	\UCitems{Origen}{%
        \Titem Mouse
	}

	\UCitem{Salidas}{\imprimeUC{salida} \begin{itemize}
    \item {\bf Icono de dificultad vencida}\IUCpuvencida
    \item {\bf Icono de dificultad no vencida}\IUCpunovencida
  \end{itemize} }

	\UCitem{Destino}{%
		\refElem{IU-C04}
	}

	\UCitems{Precondiciones}{%
        \Titem El plugin de competencia uno contra sistema debe estar instalado en moodle.
        \Titem La instancia de la actividad de competencia uno contra sistema debe estar creada.
        % Realizar el caso de uso "listar actividades disponibles?"
        % \Titem Si se trata de una actualización de un plugin la versión de este debe
               % cumplir con la regla \refElem{BR-M02}.
	}

	\UCitems{Postcondiciones}{%
        \Titem La pantalla principal de la instancia de la actividad de competencia uno contra sistema \refElem{IU-C04} debe mostrar los datos pertinentes al usuario que realizó el caso de uso.%

	}

	\UCitem{Reglas de negocio}{\imprimeUC{regla}}

	\UCitems{Errores}{%
	}

	% \UCitem{Viene de}{% Indicar si el Caso de uso es primario o se extiende de otro. La mayoría se
					  % extienden de Login.
		% EJEMPLO: \refIdElem{PY-CU1} o Caso de uso primario.
	% 	\TODO Especificar.
	% }

 \UCsection[design]{Datos de Diseño}

	\UCitems[design]{Casos de Prueba}{%
        \Titem \refElem{CPC-C12}
	}

 \UCsection[admin]{Datos de Administración de Requerimiento}

	\UCitem[admin]{Observaciones}{}

\end{UseCase}

\subsubsection{Trayectorias del caso de uso}

\begin{UCtrayectoria}%
%

    \Actor Presiona el nombre de la instancia a la que quiere acceder en la pantalla \refElem{IU-M07}.

    \Sistema Redirige a la pantalla principal de la instancia \refElem{IU-C04}.

    \Sistema Muestra las dificultades del sistema vencido por medio del icono \IUCpuvencida.

    \Sistema Muestra las dificultades del sistema que no han sido vencidas por medio del icono \IUCpunovencida.

    \Sistema Muestra en la sección {\bf Desafiar computadora} las dificultades que se pueden desafiar \salida{comp-cpu-gmdl-dificultad-cpu}

\end{UCtrayectoria}

\subsubsection{Puntos de extensión}

\UCExtensionPoint{Ver historial de sus partidas}{%

    El \refElem{aAdministrador}, \refElem{aProfesor} o \refElem{aEstudiante} desea ver su historial de las partidas de competencia uno contra sistema.
%
    }{Al final de la trayectoria principal del caso de uso.
%
    }{\refElem{CU-C15}}


\UCExtensionPoint{Ver tabla de puntuaciones}{%

    El \refElem{aAdministrador}, \refElem{aProfesor} o \refElem{aEstudiante} desea ver la tabla de puntuaciones de una instancia de competencia uno contra sistema.
%
    }{Al final de la trayectoria principal del caso de uso.
%
    }{\refElem{CU-C14}}

  \UCExtensionPoint{Desafiar al sistema}{%

      El \refElem{aAdministrador}, \refElem{aProfesor} o \refElem{aEstudiante} desea desafiar al sistema en alguna de sus dificultades.
  %
      }{Al final de la trayectoria principal del caso de uso.
  %
      }{\refElem{CU-C13}}
