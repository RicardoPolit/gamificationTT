
% \ucstEnEdicion     Al terminar una revisión/aprobación con observaciones
%                    y al inicio del CU.
%
% \ucstEnRevision    Al terminar la edición del CU (version += 0.1).
% \ucstEnAprobacion  Al pasar la revision sin observaciones.
% \ucstAprobado      Al ser aprobado por el usuario (version += 1.0)

\begin{UseCase}[%
Autor/Ricardo Naranjo,%
Version/0.1,%
Estado/\ucstEnRevision]%
%
{CU-C06}{Eliminar instancia (Competencia uno contra sistema)}{%
%
 Permite al \refElem{aProfesor} y al \refElem{aAdministrador} eliminar una instancia de la actividad competencia uno contra sistema en su curso.
 Este caso de uso es una extensión del caso de uso {\it Ver curso} que es propio de moodle.}

	\UCitem[control]{Revisor}{ Sin asignar }
	\UCitem[control]{Último cambio}{ 13/NOV/19 }

 \UCsection{Atributos}

    \UCitem{Actor(es)}{%
        \refElem{aProfesor},
        \refElem{aAdministrador}
    }

	\UCitems{Propósito}{%
        \Titem Permitir al \refElem{aProfesor} y al \refElem{aAdministrador} eliminar una instancia de la actividad de competencia uno contra sistema.
	}

	\UCitem{Entradas}{\imprimeUC{entrada}}

	\UCitems{Origen}{%
        \Titem Mouse
	}

	\UCitem{Salidas}{\imprimeUC{salida}}

	\UCitem{Destino}{%
		\refElem{IU-M07}
	}

	\UCitems{Precondiciones}{%
        \Titem El plugin de competencia uno contra sistema debe estar instalado en moodle.
        \Titem La instancia de la actividad de competencia uno contra sistema debe estar creada.
        % Realizar el caso de uso "listar actividades disponibles?"
        % \Titem Si se trata de una actualización de un plugin la versión de este debe
               % cumplir con la regla \refElem{BR-M02}.
	}

	\UCitems{Postcondiciones}{%
        \Titem La instancia de la actividad eliminada no debe mostrarse en la pantalla \refElem{IU-M07}.%

	}

	\UCitem{Reglas de negocio}{\imprimeUC{regla}}

	\UCitems{Errores}{%
	}

	% \UCitem{Viene de}{% Indicar si el Caso de uso es primario o se extiende de otro. La mayoría se
					  % extienden de Login.
		% EJEMPLO: \refIdElem{PY-CU1} o Caso de uso primario.
	% 	\TODO Especificar.
	% }

 \UCsection[design]{Datos de Diseño}

	\UCitems[design]{Casos de Prueba}{%
        \Titem \refElem{CPC-C01}
	}

 \UCsection[admin]{Datos de Administración de Requerimiento}

	\UCitem[admin]{Observaciones}{}

\end{UseCase}

\subsubsection{Trayectorias del caso de uso}

\begin{UCtrayectoria}%
%

    \Actor Activa la edición del curso en la pantalla \refElem{IU-M07}.

    \Sistema Redirige a la pantalla de edición del curso \refElem{IU-M07aa}.

    \Actor Presiona el botón {\bf Editar} de la instancia que desea eliminar.

    \Sistema Despliega el menú \refElem{IU-M07b}.

    \Actor Presiona el botón {\bf Eliminar} del menú desplegable \refElem{IU-M07b}.

    \Sistema Despliega mensaje de confirmación de eliminación. \refElem{IU-M07c}

    \Actor Presiona el botón {\bf Si}. \refTray{A}

    \Sistema Redirige a la pantalla \refElem{IU-M07} y elimina la instancia y los valores de la instancia \refElem{comp-cpu-gmcompcpu}, así como los datos que dependen de la instancia en las siguientes entidades: \refElem{comp-cpu-gmdl-intento} y \refElem{comp-cpu-gmdl-respuesta-cpu}.

\end{UCtrayectoria}

\begin{UCtrayectoriaA}[Fin del caso de uso]{A}{El \refElem{aProfesor} o \refElem{aAdministrador} desea cancelar la eliminación después de mostrar el mensaje de confirmación}

  \Actor Presiona el botón {\bf No} en la mensaje de confirmación \refElem{IU-M07C}.
  \Sistema Redirige a la pantalla \refElem{IU-M07}.

\end{UCtrayectoriaA}
