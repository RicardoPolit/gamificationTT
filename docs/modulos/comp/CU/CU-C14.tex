
% \ucstEnEdicion     Al terminar una revisión/aprobación con observaciones
%                    y al inicio del CU.
%
% \ucstEnRevision    Al terminar la edición del CU (version += 0.1).
% \ucstEnAprobacion  Al pasar la revision sin observaciones.
% \ucstAprobado      Al ser aprobado por el usuario (version += 1.0)

%\addfigure[(adaptado de {\it For The Win} \cite{ForTheWin})]%
%    {.4}{investigacion/images/forthewin}{fig:ForTheWin}%
%    {Jerarquía de elementos de juego segun For The Win}

\begin{UseCase}[%
Autor/Ricardo Naranjo,%
Version/0.1,%
Estado/\ucstEnRevision]%
%
{CU-C14}{Ver tabla de puntuaciones (Competencia uno contra sistema)}{%
%
 Permite al \refElem{aEstudiante}, \refElem{aProfesor} y al \refElem{aAdministrador} ver la tabla de puntuaciones de una instancia de la actividad competencia uno contra sistema en el curso.
 Este caso de uso es una extensión del caso de uso \refElem{CU-C12}.}

	\UCitem[control]{Revisor}{ Sin asignar }
	\UCitem[control]{Último cambio}{ 13/NOV/19 }

 \UCsection{Atributos}

    \UCitem{Actor(es)}{%
        \refElem{aEstudiante},
        \refElem{aProfesor},
        \refElem{aAdministrador}
    }

	\UCitems{Propósito}{%
        \Titem Permitir al \refElem{aEstudiante}, \refElem{aProfesor} y al \refElem{aAdministrador} ver la tabla de puntuaciones de una instancia de la actividad de competencia uno contra sistema.
	}

	\UCitem{Entradas}{\imprimeUC{entrada}}

	\UCitems{Origen}{%
        \Titem Mouse
	}

	\UCitem{Salidas}{\imprimeUC{salida}}

	\UCitem{Destino}{%
		\refElem{IU-C15}
	}

	\UCitems{Precondiciones}{%
        \Titem El plugin de competencia uno contra sistema debe estar instalado en moodle.
        \Titem La instancia de la actividad de competencia uno contra sistema debe estar creada.
        % Realizar el caso de uso "listar actividades disponibles?"
        % \Titem Si se trata de una actualización de un plugin la versión de este debe
               % cumplir con la regla \refElem{BR-M02}.
	}

	\UCitems{Postcondiciones}{%
        \Titem Se muestra la pantalla de tabla de puntuaciones de la instancia de la actividad de competencia uno contra sistema \refElem{IU-C15} de acuerdo a la dificultad seleccionada.%

	}

	\UCitem{Reglas de negocio}{\imprimeUC{regla}}

	\UCitems{Errores}{%
	}

	% \UCitem{Viene de}{% Indicar si el Caso de uso es primario o se extiende de otro. La mayoría se
					  % extienden de Login.
		% EJEMPLO: \refIdElem{PY-CU1} o Caso de uso primario.
	% 	\TODO Especificar.
	% }

 \UCsection[design]{Datos de Diseño}

	\UCitems[design]{Casos de Prueba}{%
        \Titem \refElem{CPC-C14}
	}

 \UCsection[admin]{Datos de Administración de Requerimiento}

	\UCitem[admin]{Observaciones}{}

\end{UseCase}

\subsubsection{Trayectorias del caso de uso}

\begin{UCtrayectoria}%
%

    \Actor Presiona el botón {\bf Puntuaciones} de la pantalla \refElem{IU-C04}.

    \Sistema Redirige a la pantalla de la instancia \refElem{IU-C15}.

    \Sistema Muestra un campo {\bf Seleccione una dificultad} que está por defecto en fácil. \refTray{A}

    \Sistema Muestra el \salida{comp-cpu-gmdl-dificultad-cpu.nombre} seleccionada en el campo {\bf Seleccione una dificultad}
    \label{CU-C14-muestra-informacion}

    \Sistema Muestra una tabla que presenta una fila, que contiene \salida{Posición del usuario}, \salida{Imagen de perfil}, \salida{Nombre de usuario}, \salida{comp-cpu-gmdl-intento.puntuacion-usuario} , por cada primer intento de cada usuario en esta instancia de la competencia en la dificultad seleccionada.

    \Sistema Muestra una tabla que presenta una fila, que contiene {\bf Posición del usuario}, {\bf Imagen de perfil}, {\bf Nombre de usuario}, \refElem{comp-cpu-gmdl-intento.puntuacion-usuario} , por cada mejor intento de cada usuario en esta instancia de la competencia en la dificultad seleccionada.

\end{UCtrayectoria}

\begin{UCtrayectoriaA}{A}{El actor desea seleccionar otra dificultad a ver}

  \Actor Selecciona en el campo {\bf Seleccione una dificultad} la dificultad deseada.
  \Sistema Regresa al paso \ref{CU-C14-muestra-informacion}

\end{UCtrayectoriaA}
