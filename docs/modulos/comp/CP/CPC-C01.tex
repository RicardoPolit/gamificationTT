
\TestCase{CPC-C01}{Crear nueva instancia de la actividad de competencia uno contra uno}

\begin{quote} %Contendrá la descripción del guión
	\textbf{ID:} C01.\\
    \textbf{Autor: } Ricardo Naranjo Polit\\
	\textbf{Alcance:}  \refElem{CU-C01}.\\
    \textbf{Preparación:}\\
    	%Contendrá todos los distintos datos que deberán estar preparados
      -Que se encuentre instalado el plugin {\bf gamedlemaster}.\\
      -Que se encuentre instalado el plugin {\bf gmcompvs}.\\
      -Que esté creado un curso.
      \begin{quote}
      -Que exista al menos un banco de preguntas.\\
      \textbf{Banco de preguntas 1}:
        	\begin{itemize} %Contendrá todos los atributos de dicho ejemplo
                \item \textbf{Nombre:} nueva.
                \item \textbf{Pregunta 1:}
                \begin{itemize}
                  \item \textbf{Tipo:} Opción múltiple
                  \item \textbf{Nombre:} ''Esta prueba es para ver si agrega solo la pregunta correcta''
                  \item \textbf{Texto de la pregunta:} ''Correctamente''
                  \item \textbf{Respuestas:}
                  \begin{itemize}
                    \item \textbf{si}, porcentaje: 100
                    \item \textbf{no}, porcentaje: 0
                    \item \textbf{definitivamente si}, porcentaje: 70
                  \end{itemize}
                \end{itemize}

                \item \textbf{Pregunta 2:}
                \begin{itemize}
                  \item \textbf{Tipo:} Opción múltiple
                  \item \textbf{Nombre:} ''nuevaPRegunta''
                  \item \textbf{Texto de la pregunta:} ''Esto funciona?''
                  \item \textbf{Respuestas:}
                  \begin{itemize}
                    \item \textbf{si}, porcentaje: 100
                    \item \textbf{no}, porcentaje: 0
                    \item \textbf{tal vez}, porcentaje: 60
                  \end{itemize}
                \end{itemize}

                \item \textbf{Pregunta 3:}
                \begin{itemize}
                  \item \textbf{Tipo:} Numérica
                  \item \textbf{Nombre:} ''Años''
                  \item \textbf{Texto de la pregunta:} ''¿Cuantos años tengo?''
                  \item \textbf{Respuestas:}
                  \begin{itemize}
                    \item \textbf{23}, porcentaje: 100
                    \item \textbf{22}, porcentaje: 60
                    \item \textbf{21}, porcentaje: 0
                  \end{itemize}
                \end{itemize}

                \item \textbf{Pregunta 4:}
                \begin{itemize}
                  \item \textbf{Tipo:} Respuesta corta
                  \item \textbf{Nombre:} ''PrimerShortAnswer''
                  \item \textbf{Texto de la pregunta:} ''¿Cómo me llamo?''
                  \item \textbf{Respuestas:}
                  \begin{itemize}
                    \item \textbf{Ricardo}, porcentaje: 100
                    \item \textbf{David}, porcentaje: 0
                    \item \textbf{Daniel}, porcentaje: 0
                  \end{itemize}
                \end{itemize}

                \item \textbf{Pregunta 5:}
                \begin{itemize}
                  \item \textbf{Tipo:} Falso/Verdadero
                  \item \textbf{Nombre:} ''Horario''
                  \item \textbf{Texto de la pregunta:} ''Son las dos de la mañana''
                  \item \textbf{Respuesta correcta:} Verdadero
                \end{itemize}

            \end{itemize}
    \end{quote}


\end{quote}

    \textbf{Entradas:}\\
    \begin{enumerate}
        \item \textbf{Nombre de actividad:} ''Nueva competencia uno contra uno''
        \item \textbf{Categoría de preguntas:} ''nueva''
        \item \textbf{Apuestas:} ''Activada''
        \item \textbf{Seguimiento de finalización:} ''Mostrar la actividad como completada cuando se cumplan las condiciones''
        \item \textbf{Requerir ver:} Desactivado
        \item \textbf{Competencias ganadas requeridas:} Activado
        \item \textbf{Los estudiantes deben de haber ganado un mínimo de competencias de:} 1
    \end{enumerate}
    \textbf{Pasos:}\\

    Trayectoria principal de \refElem{CU-C01}\\

    \textbf{Salida:}\\

     En la pantalla \refElem{IU-M08} la nueva instancia de la competencia uno contra uno llamada ''Nueva competencia uno contra uno''.
