

A continuación se presenta de manera general el algoritmo que sigue el
sistema para responder un cuestionario.

\noindent Al diseñar el algoritmo se tuvo en cuenta que no respondiera únicamente
de manera correcta o incorrecta determinado por una probabilidad.
Se diseño para que cada nivel de dificultad del sistema hiciera las mismas acciones,
pero con más oportunidades o más probabilidades de contestar correctamente.
Es por ello que se consideran 4 factores:

\begin{enumerate}
    \item Saber cuántas respuestas se deben elegir para responder correctamente la pregunta.
    \item El número de intentos para responder la pregunta correctamente.
    \item La probabilidad de que una respuesta sea elegida ante las otras respuestas.
    \item La posibilidad de reducir el número de respuestas que se tienen.
\end{enumerate}

El flujo general que no depende de la dificultad del sistema está representado en la figura \ref{fig:algoritmo-cpu-1},
en dicha figura se presentan cuadros azules cuya especificación se encuentran en las figuras:


\begin{enumerate}
    \item 'Calcular intentos [i]' está representado en la figura \ref{fig:algoritmo-cpu-2}.
    \item 'Calcular [c]' está representado en la figura \ref{fig:algoritmo-cpu-3} .
    \item 'Ponderar las respuestas' está representado en la figura \ref{fig:algoritmo-cpu-4} .
    \item 'Intentar descartar la respuesta [r] como una opción' está representado en la figura \ref{fig:algoritmo-cpu-5} .
\end{enumerate}

\clearpage
    \addfigure{0.8}{modulos/comp/diagrams/algoritmo_parte_1}{fig:algoritmo-cpu-1}{Diagrama de flujo del algoritmo, General}

\clearpage
    \addfigure{0.8}{modulos/comp/diagrams/algoritmo_parte_2}{fig:algoritmo-cpu-2}{Diagrama de flujo del algoritmo, 'Obtener intentos'}


\clearpage
    El algoritmo debe tomar en cuenta los casos en que las preguntas tienen más de una respuesta correcta
    y los casos en que la respuesta correcta a una pregunta es una combinación de 2 o más respuestas.\\
    \addfigure{0.8}{modulos/comp/diagrams/algoritmo_parte_3}{fig:algoritmo-cpu-3}{Diagrama de flujo del algoritmo, 'Respuestas a elegir'}
\clearpage
    Debido a que el sistema elige una pregunta de manera al azar,
    dependiendo de la dificultad se hace más probable que elija una respuesta correcta o que se elija una respuesta incorrecta.
    \addfigure{0.8}{modulos/comp/diagrams/algoritmo_parte_4}{fig:algoritmo-cpu-4}{Diagrama de flujo del algoritmo, 'Ponderación de respuestas'}
\clearpage
    Para aprovechar los intentos que tiene un sistema,
    se hace que cada nivel de dificultad pueda reducir las respuestas posibles una vez haya seleccionado una respuesta.
    \addfigure{0.8}{modulos/comp/diagrams/algoritmo_parte_5}{fig:algoritmo-cpu-5}{Diagrama de flujo del algoritmo, 'Reducir opciones'}
\clearpage


\noindent Para corroborar el funcionamiento del algoritmo y que cada dificultad tenga más probabilidad de obtener una mejor calificación se hicieron pruebas.
Dichas pruebas consistían en eligir una dificultad y que el sistema contestara 10,000 veces el mismo cuestionario y con ello obtener
el promedio de su calificación en ese cuestionario \ref{table:resultados-calificaciones-algoritmo-sistema},
el promedio de obtención de cada una de las calificaciones  \ref{table:resultados-calificacion-promedio-algoritmo-sistema}
y la desviación estándar que se tiene \ref{table:resultados-desviacion-algoritmo-sistema}.\\

\begin{table}[h!]
    \centering
    \begin{tabular}{|c|c|c|c|c|} \hline
        Dificultad del sistema & Fácil &     Normal &    Difícil &   Imposible \\\hline
        Calificación de 0 &  0.73\% &    0.01\% &   0.0\%  &  0.0\%   \\\hline
        Calificación de 1 &  3.76\% &    0.14\% &   0.0\%  &  0.0\%\\\hline
        Calificación de 2 &  5.6\% &    0.26\% &   0.0\%  &  0.0\%\\\hline
        Calificación de 3 &  12.33\% &    1.55\% &   0.0\%  &  0.0\%\\\hline
        Calificación de 4 &  20.79\% &    4.68\% &   0.07\%  &  0.0\%\\\hline
        Calificación de 5 &  17.69\% &    8.93\% &   0.31\%  &  0.01\%\\\hline
        Calificación de 6 &  21.55\% &    21.32\% &   3.15\%  &  0.56\%\\\hline
        Calificación de 7 &  10.44\% &    19.99\% &   3.68\%  &  0.63\%\\\hline
        Calificación de 8 &  5.01\% &    21.95\% &   21.86\%  &  10.35\%\\\hline
        Calificación de 9 &  1.81\% &    12.08\% &   11.87\%  &  5.99\%\\\hline
        Calificación de 10 &  0.29\% &    9.09\% &   59.06\%  &  82.46\%\\\hline
    \end{tabular}
    \caption{Tabla de resultados- Prueba algoritmo del sistema 'Porcentaje de obtención de calificación'}
    \label{table:resultados-calificaciones-algoritmo-sistema}
\end{table}


\begin{table}[h!]
    \centering
    \begin{tabular}{|c|c|c|c|c|} \hline
        Dificultad del sistema &                 Fácil &     Normal &    Difícil &   Imposible \\\hline
        Promedio de calificación &  6.0142 &    7.8878 &    9.4805 &    9.8186 \\ \hline
    \end{tabular}
    \caption{Tabla de resultados- Prueba algoritmo del sistema 'Calificación promedio'}
    \label{table:resultados-calificacion-promedio-algoritmo-sistema}
\end{table}


\begin{table}[h!]
    \centering
    \begin{tabular}{|c|c|c|c|c|} \hline
        Dificultad del sistema &                 Fácil &     Normal &    Difícil &   Imposible \\\hline
        Porcentaje de confianza &  2.0210884097436 &    3.2341593158042 &    4.7286951952952 &    5.1423616092209 \\\hline
    \end{tabular}
    \caption{Tabla de resultados- Prueba algoritmo del sistema 'Desviación estándar'}
    \label{table:resultados-desviacion-algoritmo-sistema}
\end{table}

Se grafica los resultados de las dificultades para una mejor evaluación de rendimiento en la imagen \ref{fig:algoritmo-resultados-grafica}.

\addfigure{0.8}{modulos/comp/diagrams/grafica_dificultades}{fig:algoritmo-resultados-grafica}{Gráfica de resultados de probabilidad}
