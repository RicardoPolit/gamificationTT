
\subsection{Análisis}

 Este apartado contiene el análisis requerido para la elaboración de módulo de competencia,
 contiene la especificación del alcance de este módulo, la descripción de las funcionalidades
 a desarrollar, la reglas de negocio que rigen el comportamiento del módulo, y por último la
 especificación de los casos de uso a los que brinda soporte.

%\subsubsection{Submódulo de competencia 1 contra 1}
%\subsubsection{Funcionalidades}

\subsubsection{Reglas de negocio} %========================================================

 En esta sección se especifican todas las reglas de negocio relevantes para el módulo de
 experiencia. Las reglas de negocio que establece moodle son diferenciadas por tener la letra {\it M}
 antecediendo al número consecutivo en su identificador.

    %
\subsection{Entidades de moodle}

Debido a que moodle cuenta con más de 400 tablas en su versión 3.5, se opta
por mostrar 2 subconjuntos que muestren las tablas que se utilizan para el proyecto.\\

\noindent El primer subconjunto es aquel que explica la forma en que moodle implementa los cursos, 
secciones de curso, actividades, usuarios y roles (el cual se presenta en la figura \ref{fig:BD-ER-M1}), 
mientras que el segundo conjunto muestra como moodle maneja toda la 
estructura de las preguntas creadas por el profesor y respondidas por el usuario
(el cual se presenta en la figura \ref{fig:BD-ER-M2}).  



En lugar de describir y mostrar cada uno de los campos de cada una de las entidades de moodle que se contemplan,
lo que se quiere lograr con ambos esquemas (\ref{fig:BD-ER-M1}) y \ref{fig:BD-ER-M2}))
es expresar la idea general del comportamiento.

\clearpage
\addfigure{0.7}{analisis/diagrams/db_module_structure}{fig:BD-ER-M1}{Esquema de la base de datos de moodle 'Cursos'}


\noindent Utilizando la figura \ref{fig:BD-ER-M1}, se obtuvieron las siguientes reglas y caracteristicas que contiene moodle respecto a los usuarios en un curso y a la estructura de los cursos.
\begin{enumerate}
    \item Un usuario -{\it mdl\_user}- tiene un rol -{\it mdl\_role}- en un cierto contexto -{\it mdl\_context}-, cuyo  '{\it context\_level}' sea igual a cincuenta(50).
    \item Si el contexto '{\it context\_level}' es de 50, el atributo '{\it instance\_id}' hace referencia al atributo '{\it id}' de un curso -{\it mdl\_course}-.
    \item El curso -{\it mdl\_course}- tiene varias secciones -{\it mdl\_course\_sections}-.
    \item Cada seccion -{\it mdl\_course\_sections}- tiene varias actividades -{\it mdl\_course\_modules}- que pertenecen a un tipo de actividad -{\it mdl\_modules}-.
    \item Por cada registro en tipo de actividad -{\it mdl\_modules}-, se tiene una entidad que lleva el mismo nombre.
    \item El atributo '{\it instance\_id}' de una actividad  -{\it mdl\_course\_modules}- apunta a diferentes entidades. La entidad a la que apunta depende del nombre del tipo de actividad -{\it mdl\_modules}-.
    \item Un usuario -{\it mdl\_user}- se inscribió -{\it mdl\_user\_enrolments}- a un curso -{\it mdl\_course}-, por medio de un formato soportado de inscripción -{\it mdl\_enrol}-.
\end{enumerate}

\clearpage

 \addfigure{0.7}{analisis/diagrams/db_module_questions}{fig:BD-ER-M2}{Esquema de la base de datos de moodle 'Preguntas' }



\noindent Utilizando la figura \ref{fig:BD-ER-M2}, se obtuvieron las siguientes reglas y caracteristicas que contiene moodle respecto a las preguntas.
\begin{enumerate}
    \item Las preguntas -{\it mdl\_question}- tienen versiones -{\it mdl\_question\_attempts}-.
    \item Una pregunta -{\it mdl\_question}- pertenece a un banco de preguntas -{\it mdl\_question\_categories}-.
    \item La versión de una pregunta -{\it mdl\_question\_attempts}- es contestada -{\it mdl\_question\_usages}- en un determinado contexto -{\it mdl\_context}-.
    \item Un usuario -{\it mdl\_user}- responde una versión de una pregunta -{\it mdl\_question\_attempt\_stepts}-.
    \item El responder una versión de una pregunta -{\it mdl\_question\_attempt\_stepts}- conlleva pasos -{\it mdl\_question\_attempt\_stept\_data}-, los cuales son: cómo se muestra, si ya se terminó de responder y qué se respondió.
\end{enumerate}


 A continuación se presenta la especificación de las entidades del esquema de base
 de datos de moodle que son relevantes para el desarrollo de los módulos y submódulos
 de proyecto.

    \begin{cdtEntidad}{mdl-config-plugins}{Configuración de Plugin}{%
    Es una tabla del núcleo de moodle que almacena todas las configuraciones globales
    relacionadas a los plugins instalados, al iniciar moodle las configuraciones de los
    plugins instalados y habilitados se cargan en memoria.}

	    \brAttr{id}{Id}{tInt}{%
	        Es el dígito que representa el identificador único para una configuración
            específica de un plugin.\par

            \it Restricciones:
            \refElem{tPrimaryKey},
            \refElem{tAutoIncrement}.
        }

        \brAttr{plugin}{Plugin}{tVarchar}{%
            Cadena de caracteres del nombre identificador del plugin al cual pertenece
            la configuración.\par

            \it Restricciones:
            \refElem{tRequired},
            \refElem{tRange} (0,100),
            \refElem{tUniqueKey}
        }

        \brAttr{name}{Nombre}{tVarchar}{%
            Cadena de caracteres que representa el nombre de la configuración de un
            plugin en específico.\par

            \it Restricciones:
            \refElem{tUniqueKey},
            \refElem{tRange} (0,100),
            \refElem{tRequired}
        }

        \brAttr{value}{Valor}{tVarchar}{%
            Cadena que almacena el valor de una configuración perteneciente a alguno
            de los plugins instalados.\par

            \it Restricciones:
            \refElem{tRange} (0,4294967295),
            \refElem{tRequired}
        }
    \end{cdtEntidad}\schemeName{config\_plugins}

    \begin{cdtEntidad}{mdl-user}{Usuario de moodle}{%
    Es una tabla del núcleo de moodle que contiene toda la información que se
    almacena de los usuarios en la plataforma, independientemente del rol que
    estos contenga, esta relación contiene más de 53 atributos, sin embargo solo
    se detallan aquellos relevantes.}

	    \brAttr{id}{Id}{tInt}{%
	        Es el dígito que representa el identificador único para cada uno
            de los usuarios en moodle.\par

            \it Restricciones:
            \refElem{tPrimaryKey},
            \refElem{tAutoIncrement}.
        }
	    \brAttr{username}{nombre de usuario}{tVarchar}{%
	        .\par

            \it Restricciones:
            \refElem{tRequired},
            \refElem{tLength} 0-100
        }
	    \brAttr{password}{contraseña}{tVarchar}{%
	        .\par

            \it Restricciones:
            \refElem{tRequired},
            \refElem{tLength} 0-255.
        }
	    \brAttr{firstname}{nombre}{tVarchar}{%
	        .\par

            \it Restricciones:
            \refElem{tRequired},
            \refElem{tLength} 0-100
        }
	    \brAttr{lastname}{apellido}{tVarchar}{%
	        .\par

            \it Restricciones:
            \refElem{tRequired},
            \refElem{tLength} 0-100
        }
	    \brAttr{email}{correo}{tVarchar}{%
	        .\par

            \it Restricciones:
            \refElem{tRequired},
            \refElem{tLength} 0-100
        }
	    \brAttr{lastaccess}{último registro}{tInt}{%
	        .\par

            \it Restricciones:
            \refElem{tRequired},
            \refElem{tLength} 10
        }
	    \brAttr{city}{ciudad}{tVarchar}{%
	        .\par

            \it Restricciones:
            \refElem{tRequired},
            \refElem{tLength} 0-120
        }
	    \brAttr{country}{pais}{tVarchar}{%
	        .\par

            \it Restricciones:
            \refElem{tRequired},
            \refElem{tLength} 2
        }

    \end{cdtEntidad}\schemeName{user}

    \begin{cdtEntidad}{mdl-course}{Curso de moodle}{%
    Es una tabla del núcleo de moodle que contiene la información principal de cada 
    curso registrado en moodle. Esta entidad contiene 31 atributos, a continuación se
    detallan los atributos relevantes para la especificación de este proyecto.}

	    \brAttr{id}{Id}{tInt}{%
	        Es el dígito que representa al identificador único para cada uno
            de los cursos en moodle.\par

            \it Restricciones:
            \refElem{tPrimaryKey},
            \refElem{tAutoIncrement}.
        }

	    \brAttr{format}{formato}{tVarchar}{%
	        Es el dígito que representa al identificador único para cada uno
            de los cursos en moodle.\par

            \it Restricciones:
            \refElem{tRequired}.
            \refElem{tDefault} topics,
            \refElem{tLength} 0-21.
        }

	    \brAttr{fullname}{nombre completo}{tVarchar}{%
	        Es el nombre completo que se le asigna al curso.\par

            \it Restricciones:
            \refElem{tRequired}.
            \refElem{tLength} 0-21.
        }

	    \brAttr{shortname}{nombre corto}{tVarchar}{%
            Es el nombre corto que se le asigna al curso.\par

            \it Restricciones:
            \refElem{tRequired}.
            \refElem{tLength} 0-21.
        }

    \end{cdtEntidad}\schemeName{course}

    \begin{cdtEntidad}{mdl-course-sections}{Secciones del curso de moodle}{%
    }
	    \brAttr{id}{Id}{tInt}{%
	        Es el dígito que representa al identificador único para cada seccion
            de los cursos en moodle.\par

            \it Restricciones:
            \refElem{tPrimaryKey},
            \refElem{tAutoIncrement}.
        }
    \end{cdtEntidad}\schemeName{course\_sections}

    \begin{cdtEntidad}{mdl-course-format-options}{Opciones del formato del curso}{%
    }

	    \brAttr{id}{Id}{tInt}{%
	        Es el dígito que representa al identificador único para cada uno
            de los cursos en moodle.\par

            \it Restricciones:
            \refElem{tPrimaryKey},
            \refElem{tAutoIncrement}.
        }

	    \brAttr{courseid}{Id}{tInt}{%
	        Es el dígito que representa al identificador único para cada uno
            de los cursos en moodle.\par

            \it Restricciones:
            \refElem{tForeignKey},
            \refElem{tRequired}
        }

	    \brAttr{format}{formato}{tVarchar}{%
	        Es el dígito que representa al identificador único para cada uno
            de los cursos en moodle.\par

            \it Restricciones:
            \refElem{tRequired}.
            \refElem{tDefault} topics,
            \refElem{tLength} 0-21.
        }

	    \brAttr{name}{opcion}{tVarchar}{%
	        Es el dígito que representa al identificador único para cada uno
            de los cursos en moodle.\par

            \it Restricciones:
            \refElem{tPrimaryKey},
            \refElem{tLength} 0-100
        }

	    \brAttr{value}{valor}{tVarchar}{%
	        Es el dígito que representa al identificador único para cada uno
            de los cursos en moodle.\par

            \it Restricciones:
            \refElem{tRequired}
        }

    \end{cdtEntidad}\schemeName{course\_format\_options}

    \begin{cdtEntidad}{mdl-course-category}{Categoria de curso}{%
    .}
    \end{cdtEntidad}\schemeName{course\_category}

    \begin{cdtEntidad}{Plugin}{Plugin}{%
    La forma en que moodle obtiene información acerca de los plugins es analizando
    los archivos internos de cada uno, a pesar de que los plugins no forman parte
    del esquema de base de datos, si forman parte del modelo de información que
    utiliza Moodle.}

	    \brAttr{componente}{Componente}{tVarchar}{%
	        Cadena compuesta por el tipo de plugin y el nombre del mismo, que
            representa a la clase principal del plugin que contiene los métodos
            principales del plugin.\par

            \it Restricciones: Ninguna
        }

	    \brAttr{pluginname}{Nombre}{tVarchar}{%
	        Es el nombre del plugin obtenido de los archivos de
            internacionalización presentes en el plugin, el valor de esta cadena
            depende del lenguaje seleccionado en moodle.\par

            \it Restricciones: Ninguna
        }

	    \brAttr{fullpath}{Ruta absoluta}{tPath}{%
	        La ruta absoluta de un plugin denota la ubicación del plugin en el
            sistema de archivos, esta ruta está compuesta por la ruta absoluta
            de la instalación de moodle, la carpeta correspondiente al tipo del
            plugin y el nombre del plugin.\par

            \it Restricciones: Formato ``/path/to/moodle/plugintype/pluginname''
        }

	    \brAttr{path}{Ruta relativa}{tPath}{%
	        La ruta relativa denota la ubicación del plugin dentro de la carpeta 
            donde se encuentran los archivos de moodle, esta ruta está compuesta
            por la carpeta correspondiente al tipo del plugin y el nombre del
            plugin.\par

            \it Restricciones: Formato ``plugintype/pluginname''
        }

	    \brAttr{version}{Versión}{tVersion}{%
	        Numero entero de longitud de 10 dígitos que representa la versión del 
            plugin.\par

            \it Restricciones: Ninguna adicional al tipo de dato
        }

	    \brAttr{moodle}{Versión de Moodle}{tVersion}{%
	        Número entero de longitud de 10 dígitos que representa la versión de 
            moodle en la que se puede instalar el plugin.\par

            \it Restricciones: Ninguna adicional al tipo de dato
        }

        \brAttr{dependencies}{Dependencias}{tObject}{%
            Objeto que almacena un conjunto de claves con sus respectivos valores, 
            donde cada clave representa el nombre del componente del plugin y el valor 
            es la \refElem{Plugin.version} requerida del mismo.

            \it Restricciones: Ninguna
        }

        \brAttr{icon}{ícono}{tImage}{%
            Imagen para el ícono del plugin, debe estar contenida en el directorio
            {\it pix/} del plugin y tener como nombre {\it icon.png} o {\it icon.svg},
            moodle recomienda tener ambos archivos por si los navegadores no soportan
            algun tipo de archivo \cite{moodlePluginfiles}.\par 

            \it Restricciones: El nombre debe ser icon con extensiones png o svg
        }

    \end{cdtEntidad}
 % Archivo de plugin
    
\begin{BusinessRule}[%
Autor/Ricard Naranjo Polit,%
Version/0.1,%
Estado/revision]%
%
{BR-C01}{Restricciones del tiempo que tiene }
 % El archivo de instalación debe ser un archivo ZIP, el cual debe contener
 % exactamente un directorio que coincida con el nombre del plugin.
     \BRitem[control]{Revisor}{Sin asignar.}

 \BRsection[control]{Atributos}

    \BRitem[admin]{Clase}{\bcCondition}%
    %\BRitem[admin]{Clase}{\bcIntegridad}%
    %\BRitem[admin]{Clase}{\bcAutorizacion}%
    %\BRitem[admin]{Clase}{\bcDerivacion}%

    \BRitem[admin]{Tipo}{\btEnabler}%
    %\BRitem[admin]{Tipo}{\btTimer}%
    %\BRitem[admin]{Tipo}{\btExecutive}%

    \BRitem[admin]{Nivel}{\blControlling}
    %\BRitem[admin]{Nivel}{\blInfluencing}

    \BRitem{Descripción}{%
        El archivo seleccionado para la representación visual de los niveles debe
        ser una imagen con las extensiones {\it``png''} o {\it''jgp}, además el
        nombre del archivo que será subido no debe tener el nombre {\it``icon.png''}
        ya que posiblemente colisionaría con el \refElem{Plugin.icon} del plugin.
        % debido a que se el directorio donde se guardará será el directorio
        % para almacenar las imágenes del plugin.
    }

    \BRitem{Ejemplo positivo}{\hfill\par%
        \begin{itemize}
        \item El archivo seleccionado para ser la imagen de los niveles tiene
              como nombre {\it``logotipo''} con la extensión {\it png}.

        \item El archivo seleccionado para ser la imagen de los niveles tiene
              como nombre {\it``nivel''} con la extensión {\it jpg}
        \end{itemize}
    }

    \BRitem{Ejemplo negativo}{\hfill\par%
        \begin{itemize}
        \item El archivo seleccionado para ser la imagen de los niveles tiene
              como nombre {\it``icon''} con la extensión {\it png}.

        \item El archivo seleccionado para ser la imagen de los niveles tiene
              como nombre {\it``documento''} con la extensión {\it doc}.
        \end{itemize}
    }%

 \end{BusinessRule}
 % Restricciones sobre de imagen del nivel.
    
\begin{BusinessRule}[%
Autor/Ricard Naranjo Polit,%
Version/0.1,%
Estado/revision]%
%
{BR-C02}{Un usuario no puede desafiar a otro con el que tenga un desafío pendiente}
 % El archivo de instalación debe ser un archivo ZIP, el cual debe contener
 % exactamente un directorio que coincida con el nombre del plugin.
     \BRitem[control]{Revisor}{Sin asignar.}

 \BRsection[control]{Atributos}

    \BRitem[admin]{Clase}{\bcCondition}%
    %\BRitem[admin]{Clase}{\bcIntegridad}%
    %\BRitem[admin]{Clase}{\bcAutorizacion}%
    %\BRitem[admin]{Clase}{\bcDerivacion}%

    \BRitem[admin]{Tipo}{\btEnabler}%
    %\BRitem[admin]{Tipo}{\btTimer}%
    %\BRitem[admin]{Tipo}{\btExecutive}%

    \BRitem[admin]{Nivel}{\blControlling}
    %\BRitem[admin]{Nivel}{\blInfluencing}

    \BRitem{Descripción}{%
        Cuando un usuario desafía a un \refElem{aEstudiante} no lo podrá volver a desafiar hasta que el desafiante
        y desafiado terminen hayan completado la competencia.
        % debido a que se el directorio donde se guardará será el directorio
        % para almacenar las imágenes del plugin.
    }

    \BRitem{Ejemplo positivo}{\hfill\par%
        \begin{itemize}
        \item El usuario desafía a un estudiante, los dos terminan la competencia y el usuario vuelve a desafiar al mismo estudiante.

        \end{itemize}
    }

    \BRitem{Ejemplo negativo}{\hfill\par%
        \begin{itemize}
          \item El usuario desafía a un estudiante y no alguno de los dos no termina la competencia,
          el usuario no puede volver a desafiar al mismo estudiante.

        \end{itemize}
    }%

 \end{BusinessRule}
 % Permanencia del nivel de comperiencia.
    %
\begin{BusinessRule}[%
Autor/Daniel Isai Ortega Zúñiga,%
Version/0.1,%
Estado/revision]%
%
{BR-E03}{Tipos de Incremento}
    \BRitem[control]{Revisor}{Sin asignar.}

 \BRsection[control]{Atributos}
    
    \BRitem[admin]{Clase}{\bcCondition}%
    %\BRitem[admin]{Clase}{\bcIntegridad}%
    %\BRitem[admin]{Clase}{\bcAutorizacion}%
    %\BRitem[admin]{Clase}{\bcDerivacion}%
        
    \BRitem[admin]{Tipo}{\btEnabler}%
    %\BRitem[admin]{Tipo}{\btTimer}%
    %\BRitem[admin]{Tipo}{\btExecutive}%
        
    \BRitem[admin]{Nivel}{\blControlling}
    %\BRitem[admin]{Nivel}{\blInfluencing}
    
    \BRitem{Descripción}{%
    Cuando se modifiquen el \refElem{xp-scheme-settings} o la \refElem{levelXP} de las
    \refElem{xp-scheme-settings}
    }

    \BRitem{Ejemplo positivo}{\hfill\par%
        \begin{itemize}
        \item ...
        \end{itemize}
    }

    \BRitem{Ejemplo negativo}{\hfill\par%
        \begin{itemize}
        \item ...
        \end{itemize}
    }% 
    
 \end{BusinessRule}
 % Tipos de incremento
    %\begin{BusinessRule}[%
Autor/Daniel Isai Ortega Zúñiga,%
Version/0.1,%
Estado/revision]%
%
{BR-E04}{Calculo de experiencia del nivel con incremento porcentual}
    \BRitem[control]{Revisor}{Sin asignar.}

 \BRsection[control]{Atributos}
    
    \BRitem[admin]{Clase}{\bcCondition}%
    %\BRitem[admin]{Clase}{\bcIntegridad}%
    %\BRitem[admin]{Clase}{\bcAutorizacion}%
    %\BRitem[admin]{Clase}{\bcDerivacion}%
        
    \BRitem[admin]{Tipo}{\btEnabler}%
    %\BRitem[admin]{Tipo}{\btTimer}%
    %\BRitem[admin]{Tipo}{\btExecutive}%
        
    \BRitem[admin]{Nivel}{\blControlling}
    %\BRitem[admin]{Nivel}{\blInfluencing}
    
    \BRitem{Descripción}{%
        El calculo para obtener la experiencia del nivel $i$ uando el tipo de
        incremento es porcentual está dado por la siguiente fórmula: Sea {\it exp()}
        la función que optiene la experiencia de un nivel en específico, sea tambien
        $i$ el nivel del cual se calcula la experiencia, sea $inc$ el factor de
        incremento de nivel a nivel, y finalmente sea $round()$ una función de
        redondeo a números enteros, entonces:

            $$ exp(i) = round( exp(1) * (inc)^{(i-1)})$$
    }

%   \BRitem{Sentencia}{%
%       Si $fecha$ 
%   }%

    \BRitem{Ejemplo positivo}{\hfill\par%
        \begin{itemize}
        \item La experiencia requerida para superar el nivel 1 es de 2000 puntos y el
              factor de incremento entre los niveles es 1.1, entonces la experiencia
              requerida para pasar el nivel 5 es de 2928 puntos.
        \end{itemize}
    }

    \BRitem{Ejemplo negativo}{\hfill\par%
        \begin{itemize}
        \item La experinecia requerida para superar el nivel 1 es de 2000 puntos y el
              factor de incremento entre los niveles es 1.1, entonces la experiencia
              requerida para pasar el nivel 5 es de 2300 puntos.
        \end{itemize}
    }% 
    
\end{BusinessRule}
 % Incremento porcentual
    %\begin{BusinessRule}[%
Autor/Daniel Isai Ortega Zúñiga,%
Version/0.1,%
Estado/revision]%
%
{BR-E05}{Cálculo de experiencia del nivel con incremento linea} % Cuando están iniciados
    \BRitem[control]{Revisor}{Sin asignar.}

 \BRsection[control]{Atributos}
    
    \BRitem[admin]{Clase}{\bcCondition}%
    %\BRitem[admin]{Clase}{\bcIntegridad}%
    %\BRitem[admin]{Clase}{\bcAutorizacion}%
    %\BRitem[admin]{Clase}{\bcDerivacion}%
        
    \BRitem[admin]{Tipo}{\btEnabler}%
    %\BRitem[admin]{Tipo}{\btTimer}%
    %\BRitem[admin]{Tipo}{\btExecutive}%
        
    \BRitem[admin]{Nivel}{\blControlling}
    %\BRitem[admin]{Nivel}{\blInfluencing}
    
    \BRitem{Descripción}{%
    }

%   \BRitem{Sentencia}{%
%       Si $fecha$ 
%   }%

    \BRitem{Ejemplo positivo}{\hfill\par%
        \begin{itemize}
        \item ...
        \end{itemize}
    }

    \BRitem{Ejemplo negativo}{\hfill\par%
        \begin{itemize}
        \item ...
        \end{itemize}
    }% 
    
\end{BusinessRule}
 % Incremento lineal
    %\begin{BusinessRule}[%
Autor/Daniel Isai Ortega Zúñiga,%
Version/0.1,%
Estado/revision]%
%
{BR-E05}{Cálculo de experiencia del nivel con incremento linea} % Cuando están iniciados
    \BRitem[control]{Revisor}{Sin asignar.}

 \BRsection[control]{Atributos}
    
    \BRitem[admin]{Clase}{\bcCondition}%
    %\BRitem[admin]{Clase}{\bcIntegridad}%
    %\BRitem[admin]{Clase}{\bcAutorizacion}%
    %\BRitem[admin]{Clase}{\bcDerivacion}%
        
    \BRitem[admin]{Tipo}{\btEnabler}%
    %\BRitem[admin]{Tipo}{\btTimer}%
    %\BRitem[admin]{Tipo}{\btExecutive}%
        
    \BRitem[admin]{Nivel}{\blControlling}
    %\BRitem[admin]{Nivel}{\blInfluencing}
    
    \BRitem{Descripción}{%
    }

%   \BRitem{Sentencia}{%
%       Si $fecha$ 
%   }%

    \BRitem{Ejemplo positivo}{\hfill\par%
        \begin{itemize}
        \item ...
        \end{itemize}
    }

    \BRitem{Ejemplo negativo}{\hfill\par%
        \begin{itemize}
        \item ...
        \end{itemize}
    }% 
    
\end{BusinessRule}
 % Eliminación de cursos gamificados
    %\begin{BusinessRule}[%
Autor/El Despistado,%
Version/0.1,%
Estado/edicion]%
%
{BR-E07}{Valores iniciales de experiencia}

     \BRitem[control]{Revisada por}{Pendiente.}

 \BRsection[control]{Atributos}
    % Clases: \bcCondition, \bcIntegridad, \bcAutorization o \bcDerivation
    % Tipos: \btEnabler, \btTimer o \btExecutive
    % Niveles: \blControlling o \blInfluencing.
    
    \BRitem[admin]{Clase}{\bcIntegridad}%
        
    \BRitem[admin]{Tipo}{\btTimer}%
        
    \BRitem[admin]{Nivel}{\blControlling}
    
    \BRitem{Descripción}{%
        Cuando un \refElem{xp-user} es creado este debe de empezar a ganar puntos
        de experiencia a partir del \refElem{xp-user.level} uno, tenido cero puntos 
        de experiencia en la \refElem{xp-user.levelxp} y \refElem{xp-user.xp}. Ningun
        usuario puede comenzar con valores distintos a los indicados anteriomente.
    }

%   \BRitem{Sentencia}{%
%       Si $fecha$ 
%   }%

    \BRitem{Ejemplo positivo}{\hfill\par%
        \begin{itemize}
        \item ...
        \end{itemize}
    }

    \BRitem{Ejemplo negativo}{\hfill\par%
        \begin{itemize}
        \item ...
        \end{itemize}
    }
    
 \end{BusinessRule}
 % Valores iniciales de comperiencia
    %\begin{BusinessRule}[%
Autor/Daniel Isai Ortega Zúñiga,%
Version/0.1,%
Estado/edicion]%
%
{BR-E08}{Valores iniciales de experiencia de un curso}

     \BRitem[control]{Revisada por}{Pendiente.}

 \BRsection[control]{Atributos}
    % Clases: \bcCondition, \bcIntegridad, \bcAutorization o \bcDerivation
    % Tipos: \btEnabler, \btTimer o \btExecutive
    % Niveles: \blControlling o \blInfluencing.

    \BRitem[admin]{Clase}{\bcIntegridad}%

    \BRitem[admin]{Tipo}{\btTimer}%

    \BRitem[admin]{Nivel}{\blControlling}

    \BRitem{Descripción}{%
        Cuando un \refElem{xp-course} es creado la \refElem[experiencia total del curso]%
        {xp-scheme-settings.courseXP} de ser dividida uniformemente entre las
        \refElem[secciones del curso gamificado]{xp-course-section}. Si la división del
        total de experiencia entre el número de secciones genera un residuo entonces este
        se deberá agregan a la última sección del curso.
    }

%   \BRitem{Sentencia}{%
%       Si $fecha$
%   }%

    \BRitem{Ejemplo positivo}{\hfill\par%
        \begin{itemize}
        \item ...
        \end{itemize}
    }

    \BRitem{Ejemplo negativo}{\hfill\par%
        \begin{itemize}
        \item ...
        \end{itemize}
    }

 \end{BusinessRule}
 % Valores iniciales de experiencia del curso

    % INPUT: Cursos Igualitarios.
    % INPUT: Otorgar experiencia
    % INPUT: Administración de experiencia en el curso

\clearpage
\subsubsection{Casos de uso} % ============================================================

 En este apartado se especifican todos los casos de usos contemplados para el módulo de
competencia, para cada caso de uso se especifica su tabla de atributos la cual indica que casos
 de prueba deberán ejecutarse correctamente para corroborar la completitud del caso de uso.

\subsubsection*{Diagrama de casos de uso}

 En la figura \ref{comp:usecases} se detalla el diagrama de casos de uso correspondiente al módulo
 de competencia. Los casos de uso de moodle (en color blanco) son modelados como casos de uso
 abstractos, mientras que los casos de uso del módulo de competencia son diferenciados por el
 color azul, en total el desarrollo de este módulo consiste en 17 casos de uso principales.

    \addfigure{0.6}{modulos/comp/UseCases}{comp:usecases}{%
        Diagrama de casos de uso del módulo de competencia}

 \noindent
 Debido a que los plugins a desarrollar son elementos opcionales para Moodle, solo se puede
 acceder a los casos de uso del módulo de competencia a través de puntos de extensión de los
 casos de uso de moodle. Por otra parte los casos de uso que serán documentados en esta sección
 serán los del módulo de competencia debido a que Moodle proporciona en su página oficial, guías
 e instructivos como documentación de las funcionalidades que brinda.

    % MODULO DE EXPERIENCIA


% \ucstEnEdicion     Al terminar una revisión/aprobación con observaciones
%                    y al inicio del CU.
%
% \ucstEnRevision    Al terminar la edición del CU (version += 0.1).
% \ucstEnAprobacion  Al pasar la revision sin observaciones.
% \ucstAprobado      Al ser aprobado por el usuario (version += 1.0)

\begin{UseCase}[%
Autor/Ricardo Naranjo,%
Version/0.1,%
Estado/\ucstEnRevision]%
%
{CU-C01}{Crear instancia (Competencia uno contra uno)}{%
%
 Permite al \refElem{aProfesor} y al \refElem{aAdministrador} crear una nueva instancia de la actividad competencia uno contra uno en su curso.
 La conclusión de la trayectoria principal de esta caso de uso es una precondición para que
 algunos casos de uso del módulo de competencia puedan ejecutarse.\\%
 Este caso de uso es una extensión del caso de uso {\it Listar actividades disponibles} que es propio de moodle.}

	\UCitem[control]{Revisor}{ Sin asignar }
	\UCitem[control]{Último cambio}{ 13/NOV/19 }

 \UCsection{Atributos}

    \UCitem{Actor(es)}{%
        \refElem{aProfesor},
        \refElem{aAdministrador}
    }

	\UCitems{Propósito}{%
        \Titem Permitir al \refElem{aProfesor} y al \refElem{aAdministrador} incluir en su curso una nueva instancia de la actividad de competencia uno contra uno.

        \Titem Permitir al \refElem{aEstudiante}, \refElem{aProfesor} y \refElem{aAdministrador} con acceso al curso utilizar la instancia de la actividad de competencia uno contra uno creada por el \refElem{aProfesor} o \refElem{aAdministrador}.
	}

	\UCitem{Entradas}{\imprimeUC{entrada}}

	\UCitems{Origen}{%
        \Titem Mouse
        \Titem Teclado
	}

	\UCitem{Salidas}{\imprimeUC{salida}}

	\UCitem{Destino}{%
		\refElem{IU-M08}
	}

	\UCitems{Precondiciones}{%
        \Titem El plugin de competencia uno contra uno debe estar instalado en moodle.
        % Realizar el caso de uso "listar actividades disponibles?"
        % \Titem Si se trata de una actualización de un plugin la versión de este debe
               % cumplir con la regla \refElem{BR-M02}.
	}

	\UCitems{Postcondiciones}{%
        \Titem La nueva instancia de la actividad debe mostrarse en la pantalla \refElem{IU-M08}.%

	}

	\UCitem{Reglas de negocio}{\imprimeUC{regla}}

	\UCitems{Errores}{%
        \Titem \UCerr{Err1}{%
            No se ingresó un campo requerido en el formulario de creación de la actividad,}{% CAUSA
            no se puede crear la nueva instancia de la actividad}% EFECTO
	}

	% \UCitem{Viene de}{% Indicar si el Caso de uso es primario o se extiende de otro. La mayoría se
					  % extienden de Login.
		% EJEMPLO: \refIdElem{PY-CU1} o Caso de uso primario.
	% 	\TODO Especificar.
	% }

 \UCsection[design]{Datos de Diseño}

	\UCitems[design]{Casos de Prueba}{%
        \Titem \refElem{CPC-C01}
        \Titem \refElem{CPC-C01a}
	}

 \UCsection[admin]{Datos de Administración de Requerimiento}

	\UCitem[admin]{Observaciones}{}

\end{UseCase}

\subsubsection{Trayectorias del caso de uso}

\begin{UCtrayectoria}%
%
    \Actor Selecciona la actividad Gamedle - Competencia uno contra uno en la pantalla \refElem{IU-M08a}.
    \Sistema Muestra la descripción de la actividad Gamedle - Competencia uno contra uno en la pantalla.

    \Actor Presiona el botón {\bf Agregar} en la pantalla. \refTray{A}
    \Sistema Redirige a la pantalla \refElem{IU-C01}.
    \label{CU-C01-muestra-pantalla}

    \Actor Ingresa los datos correspondientes en el formulario.

    \Actor Presiona el botón {\bf Guardar cambios y regresar al curso}.\refTray{B} \refTray{C}

    \Sistema Valida que los campos ingresados sean válidos. \refTray{D} \refErr{Err1}

    \Sistema Establece los valores ingresados para la nueva instancia \refElem{comp-1v1-gmcompvs} (
      \entrada{comp-1v1-gmcompvs.name},
      \entrada{comp-1v1-gmcompvs.mdl-question-categories-id},
      \entrada{comp-1v1-gmcompvs.apuestas-activas},
      \entrada{comp-1v1-gmcompvs.completionnumwon}), especificadas en el modelo de información.

    \Sistema Redirige a la pantalla \refElem{IU-M08} y muestra la nueva instancia creada en el curso.

\end{UCtrayectoria}

\begin{UCtrayectoriaA}[Fin del caso de uso]%
  {A}{El \refElem{aProfesor} o \refElem{aAdministrador} desea cancelar la creación de la nueva instancia después que se le muestra la descripción de la actividad}

  \Actor Presiona el botón {\bf cancelar} en la pantalla \refElem{IU-M08a}.
  \Sistema Cierra la pantalla \refElem{IU-M08a} y redirige a la pantalla \refElem{IU-M08}.

\end{UCtrayectoriaA}

\begin{UCtrayectoriaA}[Fin del caso de uso]{B}{El \refElem{aProfesor} o \refElem{aAdministrador} desea ver la nueva instancia de la actividad}

    \Actor Presiona el botón {\bf Guardar cambios y mostrar} de la pantalla \refElem{IU-C01}.

    \Sistema Valida que los campos ingresados sean válidos. \refTray{D} \refErr{Err1}

    \Sistema Establece los valores ingresados para la nueva instancia \refElem{comp-1v1-gmcompvs} (
      \refElem{comp-1v1-gmcompvs.name},
      \refElem{comp-1v1-gmcompvs.mdl-question-categories-id},
      \refElem{comp-1v1-gmcompvs.apuestas-activas},
      \refElem{comp-1v1-gmcompvs.completionnumwon}), especificadas en el modelo de información.

    \Sistema Redirige a la pantalla \refElem{IU-C02}.

\end{UCtrayectoriaA}

\begin{UCtrayectoriaA}[Fin del caso de uso]%
  {C}{El \refElem{aProfesor} desea cancelar la creación de la nueva instancia después de mostrar el formulario de creación}

  \Actor Presiona el botón {\bf cancelar} en la pantalla \refElem{IU-C01}.
  \Sistema Redirige a la pantalla \refElem{IU-C01}.

\end{UCtrayectoriaA}

\begin{UCtrayectoriaA}{D}{Algún dato ingresado por el \refElem{aProfesor} o \refElem{aAdministrador} es inválido}

  \Sistema Muestra un mensaje de error "-Usted debe poner un valor aquí", en los campos de la pantalla \refElem{IU-C01} que sean requeridos.
  \Sistema Regresa al paso \ref{CU-C01-muestra-pantalla}

\end{UCtrayectoriaA}
   % Instalar plugin del esquema de comperiencia

% \ucstEnEdicion     Al terminar una revisión/aprobación con observaciones
%                    y al inicio del CU.
%
% \ucstEnRevision    Al terminar la edición del CU (version += 0.1).
% \ucstEnAprobacion  Al pasar la revision sin observaciones.
% \ucstAprobado      Al ser aprobado por el usuario (version += 1.0)

\begin{UseCase}[%
Autor/Ricardo Naranjo,%
Version/0.1,%
Estado/\ucstEnRevision]%
%
{CU-C02}{Actualizar instancia (Competencia uno contra uno)}{%
%
 Permite al \refElem{aProfesor} y al \refElem{aAdministrador} actualizar una instancia de la actividad competencia uno contra uno en su curso.
 Este caso de uso es una extensión del caso de uso {\it Ver curso} que es propio de moodle.}

	\UCitem[control]{Revisor}{ Sin asignar }
	\UCitem[control]{Último cambio}{ 13/NOV/19 }

 \UCsection{Atributos}

    \UCitem{Actor(es)}{%
        \refElem{aProfesor},
        \refElem{aAdministrador}
    }

	\UCitems{Propósito}{%
        \Titem Permitir al \refElem{aProfesor} y al \refElem{aAdministrador} actualizar una instancia de la actividad de competencia uno contra uno.

        \Titem Permitir al \refElem{aEstudiante}, \refElem{aProfesor} y \refElem{aAdministrador} con acceso al curso utilizar la instancia actualizada de la actividad de competencia uno contra uno creada por el \refElem{aProfesor} o \refElem{aAdministrador}.
	}

	\UCitem{Entradas}{\imprimeUC{entrada}}

	\UCitems{Origen}{%
        \Titem Mouse
        \Titem Teclado
	}

	\UCitem{Salidas}{\imprimeUC{salida}}

	\UCitem{Destino}{%
		\refElem{IU-M08}
	}

	\UCitems{Precondiciones}{%
        \Titem El plugin de competencia uno contra uno debe estar instalado en moodle.
        \Titem La instancia de la actividad de competencia uno contra uno debe estar creada.
        % Realizar el caso de uso "listar actividades disponibles?"
        % \Titem Si se trata de una actualización de un plugin la versión de este debe
               % cumplir con la regla \refElem{BR-M02}.
	}

	\UCitems{Postcondiciones}{%
        \Titem La instancia actualizada de la actividad debe mostrarse en la pantalla \refElem{IU-M08}.%

	}

	\UCitem{Reglas de negocio}{\imprimeUC{regla}}

	\UCitems{Errores}{%
        \Titem \UCerr{Err1}{%
            No se ingresó un campo requerido en el formulario de creación de la actividad,}{% CAUSA
            no se puede actualizar la instancia de la actividad}% EFECTO
	}

	% \UCitem{Viene de}{% Indicar si el Caso de uso es primario o se extiende de otro. La mayoría se
					  % extienden de Login.
		% EJEMPLO: \refIdElem{PY-CU1} o Caso de uso primario.
	% 	\TODO Especificar.
	% }

 \UCsection[design]{Datos de Diseño}

	\UCitems[design]{Casos de Prueba}{%
        \Titem \refElem{CPC-C01}
	}

 \UCsection[admin]{Datos de Administración de Requerimiento}

	\UCitem[admin]{Observaciones}{}

\end{UseCase}

\subsubsection{Trayectorias del caso de uso}

\begin{UCtrayectoria}%
%

    \Actor Activa la edición del curso en la pantalla \refElem{IU-M08}.
    \Sistema Redirige a la pantalla de edición del curso \refElem{IU-M08aa}.
    \Actor Presiona el botón {\bf Editar} de la instancia que desea actualizar.
    \Sistema Despliega el menú \refElem{IU-M08b}.
    \Actor Presiona el botón {\bf Editar ajustes} del menú desplegable \refElem{IU-M08b}.

    \Sistema Redirige a la pantalla \refElem{IU-C06} y carga los valores de la instancia \refElem{comp-1v1-gmcompvs} (
      \salida{comp-1v1-gmcompvs.name},
      \salida{comp-1v1-gmcompvs.mdl-question-categories-id},
      \salida{comp-1v1-gmcompvs.apuestas-activas},
      \salida{comp-1v1-gmcompvs.completionnumwon}).

    \label{CU-C02-muestra-pantalla}

    \Actor Actualiza los datos correspondientes en el formulario.

    \Actor Presiona el botón {\bf Guardar cambios y regresar al curso}.\refTray{A} \refTray{B}

    \Sistema Valida que los campos ingresados sean válidos. \refTray{C} \refErr{Err1}

    \Sistema Actualiza los valores ingresados para la instancia \refElem{comp-1v1-gmcompvs} (
      \entrada{comp-1v1-gmcompvs.name},
      \entrada{comp-1v1-gmcompvs.mdl-question-categories-id},
      \entrada{comp-1v1-gmcompvs.apuestas-activas},
      \entrada{comp-1v1-gmcompvs.completionnumwon}), especificadas en el modelo de información.

    \Sistema Redirige a la pantalla \refElem{IU-M08} y muestra la instancia actualizada en el curso.

\end{UCtrayectoria}

\begin{UCtrayectoriaA}[Fin del caso de uso]{A}{El \refElem{aProfesor} o \refElem{aAdministrador} desea ver la instancia actualizada de la actividad}

    \Actor Presiona el botón {\bf Guardar cambios y mostrar} de la pantalla \refElem{IU-C06}.

    \Sistema Valida que los campos ingresados sean válidos. \refTray{C} \refErr{Err1}

    \Sistema Actualiza los valores ingresados para la instancia \refElem{comp-1v1-gmcompvs} (
      \refElem{comp-1v1-gmcompvs.name},
      \refElem{comp-1v1-gmcompvs.mdl-question-categories-id},
      \refElem{comp-1v1-gmcompvs.apuestas-activas},
      \refElem{comp-1v1-gmcompvs.completionnumwon}), especificadas en el modelo de información.

    \Sistema Redirige a la pantalla \refElem{IU-C02}.

\end{UCtrayectoriaA}

\begin{UCtrayectoriaA}[Fin del caso de uso]%
  {B}{El \refElem{aProfesor} o \refElem{aAdministrador} desea cancelar la actualización de la instancia después de mostrar el formulario de actualización}

  \Actor Presiona el botón {\bf cancelar} en la pantalla \refElem{IU-C06}.
  \Sistema Redirige a la pantalla \refElem{IU-C01}.

\end{UCtrayectoriaA}

\begin{UCtrayectoriaA}{C}{Algún dato ingresado por el \refElem{aProfesor} o \refElem{aAdministrador} es inválido}

  \Sistema Muestra un mensaje de error "-Usted debe poner un valor aquí", en los campos de la pantalla \refElem{IU-C06} que sean requeridos.
  \Sistema Regresa al paso \ref{CU-C02-muestra-pantalla}

\end{UCtrayectoriaA}


% \ucstEnEdicion     Al terminar una revisión/aprobación con observaciones
%                    y al inicio del CU.
%
% \ucstEnRevision    Al terminar la edición del CU (version += 0.1).
% \ucstEnAprobacion  Al pasar la revision sin observaciones.
% \ucstAprobado      Al ser aprobado por el usuario (version += 1.0)

\begin{UseCase}[%
Autor/Ricardo Naranjo,%
Version/0.1,%
Estado/\ucstEnRevision]%
%
{CU-C03}{Eliminar instancia (Competencia uno contra uno)}{%
%
 Permite al \refElem{aProfesor} y al \refElem{aAdministrador} eliminar una instancia de la actividad competencia uno contra uno en su curso.
 Este caso de uso es una extensión del caso de uso {\it Ver curso} que es propio de moodle.}

	\UCitem[control]{Revisor}{ Sin asignar }
	\UCitem[control]{Último cambio}{ 13/NOV/19 }

 \UCsection{Atributos}

    \UCitem{Actor(es)}{%
        \refElem{aProfesor},
        \refElem{aAdministrador}
    }

	\UCitems{Propósito}{%
        \Titem Permitir al \refElem{aProfesor} y al \refElem{aAdministrador} eliminar una instancia de la actividad de competencia uno contra uno.
	}

	\UCitem{Entradas}{\imprimeUC{entrada}}

	\UCitems{Origen}{%
        \Titem Mouse
	}

	\UCitem{Salidas}{\imprimeUC{salida}}

	\UCitem{Destino}{%
		\refElem{IU-M08}
	}

	\UCitems{Precondiciones}{%
        \Titem El plugin de competencia uno contra uno debe estar instalado en moodle.
        \Titem La instancia de la actividad de competencia uno contra uno debe estar creada.
        % Realizar el caso de uso "listar actividades disponibles?"
        % \Titem Si se trata de una actualización de un plugin la versión de este debe
               % cumplir con la regla \refElem{BR-M02}.
	}

	\UCitems{Postcondiciones}{%
        \Titem La instancia de la actividad eliminada no debe mostrarse en la pantalla \refElem{IU-M08}.%

	}

	\UCitem{Reglas de negocio}{\imprimeUC{regla}}

	\UCitems{Errores}{%
	}

	% \UCitem{Viene de}{% Indicar si el Caso de uso es primario o se extiende de otro. La mayoría se
					  % extienden de Login.
		% EJEMPLO: \refIdElem{PY-CU1} o Caso de uso primario.
	% 	\TODO Especificar.
	% }

 \UCsection[design]{Datos de Diseño}

	\UCitems[design]{Casos de Prueba}{%
        \Titem \refElem{CPC-C03}
	}

 \UCsection[admin]{Datos de Administración de Requerimiento}

	\UCitem[admin]{Observaciones}{}

\end{UseCase}

\subsubsection{Trayectorias del caso de uso}

\begin{UCtrayectoria}%
%

    \Actor Activa la edición del curso en la pantalla \refElem{IU-M08}.

    \Sistema Redirige a la pantalla de edición del curso \refElem{IU-M08aa}.

    \Actor Presiona el botón {\bf Editar} de la instancia que desea eliminar.

    \Sistema Despliega el menú \refElem{IU-M08b}.

    \Actor Presiona el botón {\bf Eliminar} del menú desplegable \refElem{IU-M08b}.

    \Sistema Despliega mensaje de confirmación de eliminación. \refElem{IU-M08c}

    \Actor Presiona el botón {\bf Si}. \refTray{A}

    \Sistema Redirige a la pantalla \refElem{IU-M08} y elimina la instancia y los valores de la instancia \refElem{comp-1v1-gmcompvs}, así como los datos que dependen de la instancia en las siguientes entidades: \refElem{comp-1v1-gmdl-partida}, \refElem{comp-1v1-gmdl-participacion} y \refElem{comp-1v1-gmdl-partida}.

\end{UCtrayectoria}

\begin{UCtrayectoriaA}[Fin del caso de uso]{A}{El \refElem{aProfesor} o \refElem{aAdministrador} desea cancelar la eliminación después de mostrar el mensaje de confirmación}

  \Actor Presiona el botón {\bf No} en la mensaje de confirmación \refElem{IU-M08c}.
  \Sistema Redirige a la pantalla \refElem{IU-M08}.

\end{UCtrayectoriaA}


% \ucstEnEdicion     Al terminar una revisión/aprobación con observaciones
%                    y al inicio del CU.
%
% \ucstEnRevision    Al terminar la edición del CU (version += 0.1).
% \ucstEnAprobacion  Al pasar la revision sin observaciones.
% \ucstAprobado      Al ser aprobado por el usuario (version += 1.0)

\begin{UseCase}[%
Autor/Ricardo Naranjo,%
Version/0.1,%
Estado/\ucstEnRevision]%
%
{CU-C04}{Crear instancia (Competencia uno contra sistema)}{%
%
 Permite al \refElem{aProfesor} y al \refElem{aAdministrador} crear una nueva instancia de la actividad competencia uno contra sistema en su curso.
 La conclusión de la trayectoria principal de esta caso de uso es una precondición para que
 algunos casos de uso del módulo de competencia puedan ejecutarse.\\%
 Este caso de uso es una extensión del caso de uso {\it Listar actividades disponibles} que es propio de moodle.}

	\UCitem[control]{Revisor}{ Sin asignar }
	\UCitem[control]{Último cambio}{ 13/NOV/19 }

 \UCsection{Atributos}

    \UCitem{Actor(es)}{%
        \refElem{aProfesor},
        \refElem{aAdministrador}
    }

	\UCitems{Propósito}{%
        \Titem Permitir al \refElem{aProfesor} y al \refElem{aAdministrador} incluir en su curso una nueva instancia de la actividad de competencia uno contra sistema.

        \Titem Permitir al \refElem{aEstudiante}, \refElem{aProfesor} y \refElem{aAdministrador} con acceso al curso utilizar la instancia de la actividad de competencia uno contra sistema creada por el \refElem{aProfesor} o \refElem{aAdministrador}.
	}

	\UCitem{Entradas}{\imprimeUC{entrada}}

	\UCitems{Origen}{%
        \Titem Mouse
        \Titem Teclado
	}

	\UCitem{Salidas}{\imprimeUC{salida}}

	\UCitem{Destino}{%
		\refElem{IU-M07}
	}

	\UCitems{Precondiciones}{%
        \Titem El plugin de competencia 1 contra sistema debe estar instalado en moodle.
        % Realizar el caso de uso "listar actividades disponibles?"
        % \Titem Si se trata de una actualización de un plugin la versión de este debe
               % cumplir con la regla \refElem{BR-M02}.
	}

	\UCitems{Postcondiciones}{%
        \Titem La nueva instancia de la actividad debe mostrarse en la pantalla \refElem{IU-M07}.%

	}

	\UCitem{Reglas de negocio}{\imprimeUC{regla}}

	\UCitems{Errores}{%
        \Titem \UCerr{Err1}{%
            No se ingresó un campo requerido en el formulario de creación de la actividad,}{% CAUSA
            no se puede crear la nueva instancia de la actividad}% EFECTO
	}

	% \UCitem{Viene de}{% Indicar si el Caso de uso es primario o se extiende de otro. La mayoría se
					  % extienden de Login.
		% EJEMPLO: \refIdElem{PY-CU1} o Caso de uso primario.
	% 	\TODO Especificar.
	% }

 \UCsection[design]{Datos de Diseño}

	\UCitems[design]{Casos de Prueba}{%
        \Titem \refElem{CPC-C01}
	}

 \UCsection[admin]{Datos de Administración de Requerimiento}

	\UCitem[admin]{Observaciones}{}

\end{UseCase}

\subsubsection{Trayectorias del caso de uso}

\begin{UCtrayectoria}%
%
    \Actor Selecciona la actividad Gamedle - Competencia 1 contra sistema en la pantalla \refElem{IU-M07a}.
    \Sistema Muestra la descripción de la actividad Gamedle - Competencia 1 contra sistema en la pantalla.

    \Actor Presiona el botón {\bf Agregar} en la pantalla. \refTray{A}
    \Sistema Redirige a la pantalla \refElem{IU-C03}.
    \label{CU-C04-muestra-pantalla}

    \Actor Ingresa los datos correspondientes en el formulario.

    \Actor Presiona el botón {\bf Guardar cambios y regresar al curso}.\refTray{B} \refTray{C}

    \Sistema Valida que los campos ingresados sean válidos. \refTray{D} \refErr{Err1}

    \Sistema Establece los valores ingresados para la nueva instancia \refElem{comp-cpu-gmcompcpu} (
      \entrada{comp-cpu-gmcompcpu.name},
      \entrada{comp-cpu-gmcompcpu.mdl-question-categories-id},
      \entrada{comp-cpu-gmcompcpu.completioncpudiff}), especificadas en el modelo de información.

    \Sistema Redirige a la pantalla \refElem{IU-M07} y muestra la nueva instancia creada en el curso.

\end{UCtrayectoria}

\begin{UCtrayectoriaA}[Fin del caso de uso]%
  {A}{El \refElem{aProfesor} o \refElem{aAdministrador} desea cancelar la creación de la nueva instancia después que se le muestra la descripción de la actividad}

  \Actor Presiona el botón {\bf cancelar} en la pantalla \refElem{IU-M07a}.
  \Sistema Cierra la pantalla \refElem{IU-M07a} y redirige a la pantalla \refElem{IU-M07}.

\end{UCtrayectoriaA}

\begin{UCtrayectoriaA}[Fin del caso de uso]{B}{El \refElem{aProfesor} o \refElem{aAdministrador} desea ver la nueva instancia de la actividad}

    \Actor Presiona el botón {\bf Guardar cambios y mostrar} de la pantalla \refElem{IU-C03}.

    \Sistema Valida que los campos ingresados sean válidos. \refTray{D} \refErr{Err1}

    \Sistema Establece los valores ingresados para la nueva instancia \refElem{comp-cpu-gmcompcpu} (
      \refElem{comp-cpu-gmcompcpu.name},
      \refElem{comp-cpu-gmcompcpu.mdl-question-categories-id},
      \refElem{comp-cpu-gmcompcpu.completioncpudiff}), especificadas en el modelo de información.

    \Sistema Redirige a la pantalla \refElem{IU-C04}.

\end{UCtrayectoriaA}

\begin{UCtrayectoriaA}[Fin del caso de uso]%
  {C}{El \refElem{aProfesor} desea cancelar la creación de la nueva instancia después de mostrar el formulario de creación}

  \Actor Presiona el botón {\bf cancelar} en la pantalla \refElem{IU-C03}.
  \Sistema Redirige a la pantalla \refElem{IU-M07}.

\end{UCtrayectoriaA}

\begin{UCtrayectoriaA}{D}{Algún dato ingresado por el \refElem{aProfesor} o \refElem{aAdministrador} es inválido}

  \Sistema Muestra un mensaje de error "-Usted debe poner un valor aquí", en los campos de la pantalla \refElem{IU-C03} que sean requeridos.
  \Sistema Regresa al paso \ref{CU-C04-muestra-pantalla}

\end{UCtrayectoriaA}
   % Instalar plugin del esquema de comperiencia

% \ucstEnEdicion     Al terminar una revisión/aprobación con observaciones
%                    y al inicio del CU.
%
% \ucstEnRevision    Al terminar la edición del CU (version += 0.1).
% \ucstEnAprobacion  Al pasar la revision sin observaciones.
% \ucstAprobado      Al ser aprobado por el usuario (version += 1.0)

\begin{UseCase}[%
Autor/Ricardo Naranjo,%
Version/0.1,%
Estado/\ucstEnRevision]%
%
{CU-C05}{Actualizar instancia (Competencia uno contra sistema)}{%
%
 Permite al \refElem{aProfesor} y al \refElem{aAdministrador} actualizar una instancia de la actividad competencia uno contra sistema en su curso.
 Este caso de uso es una extensión del caso de uso {\it Ver curso} que es propio de moodle.}

	\UCitem[control]{Revisor}{ Sin asignar }
	\UCitem[control]{Último cambio}{ 13/NOV/19 }

 \UCsection{Atributos}

    \UCitem{Actor(es)}{%
        \refElem{aProfesor},
        \refElem{aAdministrador}
    }

	\UCitems{Propósito}{%
        \Titem Permitir al \refElem{aProfesor} y al \refElem{aAdministrador} actualizar una instancia de la actividad de competencia uno contra sistema.

        \Titem Permitir al \refElem{aEstudiante}, \refElem{aProfesor} y \refElem{aAdministrador} con acceso al curso utilizar la instancia actualizada de la actividad de competencia uno contra sistema creada por el \refElem{aProfesor} o \refElem{aAdministrador}.
	}

	\UCitem{Entradas}{\imprimeUC{entrada}}

	\UCitems{Origen}{%
        \Titem Mouse
        \Titem Teclado
	}

	\UCitem{Salidas}{\imprimeUC{salida}}

	\UCitem{Destino}{%
		\refElem{IU-M07}
	}

	\UCitems{Precondiciones}{%
        \Titem El plugin de competencia uno contra sistema debe estar instalado en moodle.
        \Titem La instancia de la actividad de competencia uno contra sistema debe estar creada.
        % Realizar el caso de uso "listar actividades disponibles?"
        % \Titem Si se trata de una actualización de un plugin la versión de este debe
               % cumplir con la regla \refElem{BR-M02}.
	}

	\UCitems{Postcondiciones}{%
        \Titem La instancia actualizada de la actividad debe mostrarse en la pantalla \refElem{IU-M07}.%

	}

	\UCitem{Reglas de negocio}{\imprimeUC{regla}}

	\UCitems{Errores}{%
        \Titem \UCerr{Err1}{%
            No se ingresó un campo requerido en el formulario de creación de la actividad,}{% CAUSA
            no se puede actualizar la instancia de la actividad}% EFECTO
	}

	% \UCitem{Viene de}{% Indicar si el Caso de uso es primario o se extiende de otro. La mayoría se
					  % extienden de Login.
		% EJEMPLO: \refIdElem{PY-CU1} o Caso de uso primario.
	% 	\TODO Especificar.
	% }

 \UCsection[design]{Datos de Diseño}

	\UCitems[design]{Casos de Prueba}{%
        \Titem \refElem{CPC-C01}
	}

 \UCsection[admin]{Datos de Administración de Requerimiento}

	\UCitem[admin]{Observaciones}{}

\end{UseCase}

\subsubsection{Trayectorias del caso de uso}

\begin{UCtrayectoria}%
%

    \Actor Activa la edición del curso en la pantalla \refElem{IU-M07}.
    \Sistema Redirige a la pantalla de edición del curso \refElem{IU-M07aa}.
    \Actor Presiona el botón {\bf Editar} de la instancia que desea actualizar.
    \Sistema Despliega el menú \refElem{IU-M07b}.
    \Actor Presiona el botón {\bf Editar ajustes} del menú desplegable \refElem{IU-M07b}.

    \Sistema Redirige a la pantalla \refElem{IU-C05} y carga los valores de la instancia \refElem{comp-cpu-gmcompcpu} (
      \salida{comp-cpu-gmcompcpu.name},
      \salida{comp-cpu-gmcompcpu.mdl-question-categories-id},
      \salida{comp-cpu-gmcompcpu.completioncpudiff}), especificadas en el modelo de información.

    \label{CU-C05-muestra-pantalla}

    \Actor Actualiza los datos correspondientes en el formulario.

    \Actor Presiona el botón {\bf Guardar cambios y regresar al curso}.\refTray{A} \refTray{B}

    \Sistema Valida que los campos ingresados sean válidos. \refTray{C} \refErr{Err1}

    \Sistema Actualiza los valores ingresados para la instancia \refElem{comp-cpu-gmcompcpu} (
      \entrada{comp-cpu-gmcompcpu.name},
      \entrada{comp-cpu-gmcompcpu.mdl-question-categories-id},
      \entrada{comp-cpu-gmcompcpu.completioncpudiff}), especificadas en el modelo de información.

    \Sistema Redirige a la pantalla \refElem{IU-M07} y muestra la instancia actualizada en el curso.

\end{UCtrayectoria}

\begin{UCtrayectoriaA}[Fin del caso de uso]{A}{El \refElem{aProfesor} o \refElem{aAdministrador} desea ver la instancia actualizada de la actividad}

    \Actor Presiona el botón {\bf Guardar cambios y mostrar} de la pantalla \refElem{IU-C05}.

    \Sistema Valida que los campos ingresados sean válidos. \refTray{C} \refErr{Err1}

    \Sistema Actualiza los valores ingresados para la instancia \refElem{comp-cpu-gmcompcpu} (
      \refElem{comp-cpu-gmcompcpu.name},
      \refElem{comp-cpu-gmcompcpu.mdl-question-categories-id},
      \refElem{comp-cpu-gmcompcpu.completioncpudiff}), especificadas en el modelo de información.

    \Sistema Redirige a la pantalla \refElem{IU-C02}.

\end{UCtrayectoriaA}

\begin{UCtrayectoriaA}[Fin del caso de uso]%
  {B}{El \refElem{aProfesor} o \refElem{aAdministrador} desea cancelar la actualización de la instancia después de mostrar el formulario de actualización}

  \Actor Presiona el botón {\bf cancelar} en la pantalla \refElem{IU-C05}.
  \Sistema Redirige a la pantalla \refElem{IU-C01}.

\end{UCtrayectoriaA}

\begin{UCtrayectoriaA}{C}{Algún dato ingresado por el \refElem{aProfesor} o \refElem{aAdministrador} es inválido}

  \Sistema Muestra un mensaje de error "-Usted debe poner un valor aquí", en los campos de la pantalla \refElem{IU-C05} que sean requeridos.
  \Sistema Regresa al paso \ref{CU-C05-muestra-pantalla}

\end{UCtrayectoriaA}


% \ucstEnEdicion     Al terminar una revisión/aprobación con observaciones
%                    y al inicio del CU.
%
% \ucstEnRevision    Al terminar la edición del CU (version += 0.1).
% \ucstEnAprobacion  Al pasar la revision sin observaciones.
% \ucstAprobado      Al ser aprobado por el usuario (version += 1.0)

\begin{UseCase}[%
Autor/Ricardo Naranjo,%
Version/0.1,%
Estado/\ucstEnRevision]%
%
{CU-C06}{Eliminar instancia (Competencia uno contra sistema)}{%
%
 Permite al \refElem{aProfesor} y al \refElem{aAdministrador} eliminar una instancia de la actividad competencia uno contra sistema en su curso.
 Este caso de uso es una extensión del caso de uso {\it Ver curso} que es propio de moodle.}

	\UCitem[control]{Revisor}{ Sin asignar }
	\UCitem[control]{Último cambio}{ 13/NOV/19 }

 \UCsection{Atributos}

    \UCitem{Actor(es)}{%
        \refElem{aProfesor},
        \refElem{aAdministrador}
    }

	\UCitems{Propósito}{%
        \Titem Permitir al \refElem{aProfesor} y al \refElem{aAdministrador} eliminar una instancia de la actividad de competencia uno contra sistema.
	}

	\UCitem{Entradas}{\imprimeUC{entrada}}

	\UCitems{Origen}{%
        \Titem Mouse
	}

	\UCitem{Salidas}{\imprimeUC{salida}}

	\UCitem{Destino}{%
		\refElem{IU-M08}
	}

	\UCitems{Precondiciones}{%
        \Titem El plugin de competencia uno contra sistema debe estar instalado en moodle.
        \Titem La instancia de la actividad de competencia uno contra sistema debe estar creada.
        % Realizar el caso de uso "listar actividades disponibles?"
        % \Titem Si se trata de una actualización de un plugin la versión de este debe
               % cumplir con la regla \refElem{BR-M02}.
	}

	\UCitems{Postcondiciones}{%
        \Titem La instancia de la actividad eliminada no debe mostrarse en la pantalla \refElem{IU-M08}.%

	}

	\UCitem{Reglas de negocio}{\imprimeUC{regla}}

	\UCitems{Errores}{%
	}

	% \UCitem{Viene de}{% Indicar si el Caso de uso es primario o se extiende de otro. La mayoría se
					  % extienden de Login.
		% EJEMPLO: \refIdElem{PY-CU1} o Caso de uso primario.
	% 	\TODO Especificar.
	% }

 \UCsection[design]{Datos de Diseño}

	\UCitems[design]{Casos de Prueba}{%
        \Titem \refElem{CPC-C06}
	}

 \UCsection[admin]{Datos de Administración de Requerimiento}

	\UCitem[admin]{Observaciones}{}

\end{UseCase}

\subsubsection{Trayectorias del caso de uso}

\begin{UCtrayectoria}%
%

    \Actor Activa la edición del curso en la pantalla \refElem{IU-M08}.

    \Sistema Redirige a la pantalla de edición del curso \refElem{IU-M08aa}.

    \Actor Presiona el botón {\bf Editar} de la instancia que desea eliminar.

    \Sistema Despliega el menú \refElem{IU-M08b}.

    \Actor Presiona el botón {\bf Eliminar} del menú desplegable \refElem{IU-M08b}.

    \Sistema Despliega mensaje de confirmación de eliminación. \refElem{IU-M08c}

    \Actor Presiona el botón {\bf Si}. \refTray{A}

    \Sistema Redirige a la pantalla \refElem{IU-M08} y elimina la instancia y los valores de la instancia \refElem{comp-cpu-gmcompcpu}, así como los datos que dependen de la instancia en las siguientes entidades: \refElem{comp-cpu-gmdl-intento} y \refElem{comp-cpu-gmdl-respuesta-cpu}.

\end{UCtrayectoria}

\begin{UCtrayectoriaA}[Fin del caso de uso]{A}{El \refElem{aProfesor} o \refElem{aAdministrador} desea cancelar la eliminación después de mostrar el mensaje de confirmación}

  \Actor Presiona el botón {\bf No} en la mensaje de confirmación \refElem{IU-M08c}.
  \Sistema Redirige a la pantalla \refElem{IU-M08}.

\end{UCtrayectoriaA}


% \ucstEnEdicion     Al terminar una revisión/aprobación con observaciones
%                    y al inicio del CU.
%
% \ucstEnRevision    Al terminar la edición del CU (version += 0.1).
% \ucstEnAprobacion  Al pasar la revision sin observaciones.
% \ucstAprobado      Al ser aprobado por el usuario (version += 1.0)

%\addfigure[(adaptado de {\it For The Win} \cite{ForTheWin})]%
%    {.4}{investigacion/images/forthewin}{fig:ForTheWin}%
%    {Jerarquía de elementos de juego segun For The Win}

\begin{UseCase}[%
Autor/Ricardo Naranjo,%
Version/0.1,%
Estado/\ucstEnRevision]%
%
{CU-C07}{Ver estado de instancia de actividad (Competencia uno contra uno)}{%
%
 Permite al \refElem{aEstudiante}, \refElem{aProfesor} y al \refElem{aAdministrador} ver el estado actual de una instancia de la actividad competencia uno contra uno en el curso.
 Este caso de uso es una extensión del caso de uso {\it Ver curso} que es propio de moodle.}

	\UCitem[control]{Revisor}{ Sin asignar }
	\UCitem[control]{Último cambio}{ 13/NOV/19 }

 \UCsection{Atributos}

    \UCitem{Actor(es)}{%
        \refElem{aEstudiante},
        \refElem{aProfesor},
        \refElem{aAdministrador}
    }

	\UCitems{Propósito}{%
        \Titem Permitir al \refElem{aEstudiante}, \refElem{aProfesor} y al \refElem{aAdministrador} ver el estado actual de una instancia de la actividad de competencia uno contra uno.
	}

	\UCitem{Entradas}{\imprimeUC{entrada}}

	\UCitems{Origen}{%
        \Titem Mouse
	}

	\UCitem{Salidas}{\imprimeUC{salida}}

	\UCitem{Destino}{%
		\refElem{IU-C02}
	}

	\UCitems{Precondiciones}{%
        \Titem El plugin de competencia uno contra uno debe estar instalado en moodle.
        \Titem La instancia de la actividad de competencia uno contra uno debe estar creada.
        % Realizar el caso de uso "listar actividades disponibles?"
        % \Titem Si se trata de una actualización de un plugin la versión de este debe
               % cumplir con la regla \refElem{BR-M02}.
	}

	\UCitems{Postcondiciones}{%
        \Titem La pantalla principal de la instancia de la actividad de competencia uno contra uno \refElem{IU-C02} debe mostrar los datos pertinentes al usuario que realizó el caso de uso.%

	}

	\UCitem{Reglas de negocio}{\imprimeUC{regla}}

	\UCitems{Errores}{%
	}

	% \UCitem{Viene de}{% Indicar si el Caso de uso es primario o se extiende de otro. La mayoría se
					  % extienden de Login.
		% EJEMPLO: \refIdElem{PY-CU1} o Caso de uso primario.
	% 	\TODO Especificar.
	% }

 \UCsection[design]{Datos de Diseño}

	\UCitems[design]{Casos de Prueba}{%
        \Titem \refElem{CPC-C07}
	}

 \UCsection[admin]{Datos de Administración de Requerimiento}

	\UCitem[admin]{Observaciones}{}

\end{UseCase}

\subsubsection{Trayectorias del caso de uso}

\begin{UCtrayectoria}%
%

    \Actor Presiona el nombre de la instancia a la que quiere acceder en la pantalla \refElem{IU-M07}.

    \Sistema Valida que las partidas en curso cumplan con la \regla{BR-C01} y las partidas que no la cumplen son terminadas.

    \Sistema Verifica que el actor esté dado de alto como usuario gamificado. \refTray{B}.

    \Sistema Valida que la bandera \refElem{comp-1v1-gmcompvs.apuestas-activas} esté activa. \refTray{A}.
    \label{CU-C07-mostrar informacion}
    \Sistema Redirige a la pantalla principal de la instancia \refElem{IU-C02}.

    \Sistema Muestra el \salida{número de victorias}.

    \Sistema Muestra en el área {\bf Compañeros del curso} un bloque compañero por cada \refElem{aEstudiante} que esté inscrito en el curso y que cumpla con la \regla{BR-C02}, dicho bloque se compone de la \salida{imagen de perfil compañero}, \salida{nombre compañero}, campo {\bf Monedas a apostar} y el botón {\bf Desafiar}.

    \Sistema Muestra en el área {\bf Desafíos pendientes} un bloque desafiante por cada \refElem{aEstudiante}, \refElem{aProfesor} o \refElem{aAdministrador} que haya desafiado al \refElem{aEstudiante} que está realizando el caso de uso, dicho bloque se compone de la \salida{imagen de perfil desafiante}, \salida{nombre desafiante}, campo {\bf Monedas a apostar} y el botón {\bf Aceptar desafío}.

\end{UCtrayectoria}

\begin{UCtrayectoriaA}[Fin del caso de uso]{A}{La bandera \refElem{comp-1v1-gmcompvs.apuestas-activas} está inactiva}

  \Sistema Redirige a la pantalla principal de la instancia \refElem{IU-C02a}.

  \Sistema Muestra el número de victorias.

  \Sistema Muestra en el área {\bf Compañeros del curso} un bloque por cada \refElem{aEstudiante} que esté inscrito en el curso y que cumpla con la \regla{BR-C02}, dicho bloque se compone de la imagen de perfil compañero, nombre compañero y el botón {\bf Desafiar}.

  \Sistema Muestra en el área {\bf Desafíos pendientes} un bloque desafiante por cada \refElem{aEstudiante}, \refElem{aProfesor} o \refElem{aAdministrador} que haya desafiado al \refElem{aEstudiante} que está realizando el caso de uso, dicho bloque se compone de la imagen de perfil desafiante, nombre desafiante y el botón {\bf Aceptar desafío}.

\end{UCtrayectoriaA}


\begin{UCtrayectoriaA}{B}{El actor no está dado de alta como usuario gamificado (\refElem{xp-user})}

  \Sistema Registra al actor en la entidad (\refElem{xp-user}).
    \item Se regresa al paso \ref{CU-C07-mostrar informacion} de la trayectoria principal.


\end{UCtrayectoriaA}

\subsubsection{Puntos de extensión}

\UCExtensionPoint{Ver historial de las partidas}{%

    El \refElem{aAdministrador}, \refElem{aProfesor} o \refElem{aEstudiante} desea ver su historial de las partidas de competencia uno contra uno.
%
    }{Al final de la trayectoria principal del caso de uso.
%
    }{\refElem{CU-C08}}


\UCExtensionPoint{Ver tabla de posiciones}{%

    El \refElem{aAdministrador}, \refElem{aProfesor} o \refElem{aEstudiante} desea ver la tabla de posiciones de una instancia de competencia uno contra uno.
%
    }{Al final de la trayectoria principal del caso de uso.
%
    }{\refElem{CU-C09}}

  \UCExtensionPoint{Desafiar a un estudiante con apuestas}{%

      El \refElem{aAdministrador}, \refElem{aProfesor} o \refElem{aEstudiante} desafiar a un estudiante y apostar una cantidad de monedas.
  %
      }{Al final de la trayectoria principal del caso de uso.
  %
      }{\refElem{CU-C10}}


\UCExtensionPoint{Desafiar a un estudiante sin apuestas}{%

    El \refElem{aAdministrador}, \refElem{aProfesor} o \refElem{aEstudiante} desafiar a un estudiante.
%
    }{Al final de la \refTray{A}.
%
    }{\refElem{CU-C11}}


% \ucstEnEdicion     Al terminar una revisión/aprobación con observaciones
%                    y al inicio del CU.
%
% \ucstEnRevision    Al terminar la edición del CU (version += 0.1).
% \ucstEnAprobacion  Al pasar la revision sin observaciones.
% \ucstAprobado      Al ser aprobado por el usuario (version += 1.0)

%\addfigure[(adaptado de {\it For The Win} \cite{ForTheWin})]%
%    {.4}{investigacion/images/forthewin}{fig:ForTheWin}%
%    {Jerarquía de elementos de juego segun For The Win}

\begin{UseCase}[%
Autor/Ricardo Naranjo,%
Version/0.1,%
Estado/\ucstEnRevision]%
%
{CU-C08}{Ver historial de sus partidas (Competencia uno contra uno)}{%
%
 Permite al \refElem{aEstudiante}, \refElem{aProfesor} y al \refElem{aAdministrador} ver su historial de una instancia de la actividad competencia uno contra uno en el curso.
 Este caso de uso es una extensión del caso de uso \refElem{CU-C07}.}

	\UCitem[control]{Revisor}{ Sin asignar }
	\UCitem[control]{Último cambio}{ 13/NOV/19 }

 \UCsection{Atributos}

    \UCitem{Actor(es)}{%
        \refElem{aEstudiante},
        \refElem{aProfesor},
        \refElem{aAdministrador}
    }

	\UCitems{Propósito}{%
        \Titem Permitir al \refElem{aEstudiante}, \refElem{aProfesor} y al \refElem{aAdministrador} ver su historial de una instancia de la actividad de competencia uno contra uno.
	}

	\UCitem{Entradas}{\imprimeUC{entrada}}

	\UCitems{Origen}{%
        \Titem Mouse
	}

	\UCitem{Salidas}{\imprimeUC{salida}}

	\UCitem{Destino}{%
		\refElem{IU-C25}
	}

	\UCitems{Precondiciones}{%
        \Titem El plugin de competencia uno contra uno debe estar instalado en moodle.
        \Titem La instancia de la actividad de competencia uno contra uno debe estar creada.
        % Realizar el caso de uso "listar actividades disponibles?"
        % \Titem Si se trata de una actualización de un plugin la versión de este debe
               % cumplir con la regla \refElem{BR-M02}.
	}

	\UCitems{Postcondiciones}{%
        \Titem La pantalla de historial de la instancia de la actividad de competencia uno contra uno \refElem{IU-C07} debe mostrar los datos pertinentes al usuario que realizó el caso de uso.%

	}

	\UCitem{Reglas de negocio}{\imprimeUC{regla}}

	\UCitems{Errores}{%
	}

	% \UCitem{Viene de}{% Indicar si el Caso de uso es primario o se extiende de otro. La mayoría se
					  % extienden de Login.
		% EJEMPLO: \refIdElem{PY-CU1} o Caso de uso primario.
	% 	\TODO Especificar.
	% }

 \UCsection[design]{Datos de Diseño}

	\UCitems[design]{Casos de Prueba}{%
        \Titem \refElem{CPC-C08}
	}

 \UCsection[admin]{Datos de Administración de Requerimiento}

	\UCitem[admin]{Observaciones}{}

\end{UseCase}

\subsubsection{Trayectorias del caso de uso}

\begin{UCtrayectoria}%
%

    \Actor Presiona el botón {\bf Historial} de la pantalla \refElem{IU-C02}.

    \Sistema Redirige a la pantalla historial de la instancia \refElem{IU-C07}.

    \Sistema Muestra en la sección {\bf Resumen de los estados de los desafíos} los siguientes números: \salida{Victorias}, \salida{Empates}, \salida{Derrotas}, \salida{En curso}, \salida{Retirada}.

    \Sistema Muestra en la sección inferior un bloque por cada partida iniciada, en el bloque se muestra: \salida{Imagen de perfil de contrincante}, \salida{Nombre de contrincante}, {\bf Puntos del actor} \salida{comp-vs-gmdl-participacion.puntuacion}, {\bf Monedas apostadas} \salida{comp-vs-gmdl-apuesta.monedas-plata}, {\bf Puntos contrincante}, {\bf Monedas apostadas contrincante}, \salida{Estado de desafío}.

\end{UCtrayectoria}


% \ucstEnEdicion     Al terminar una revisión/aprobación con observaciones
%                    y al inicio del CU.
%
% \ucstEnRevision    Al terminar la edición del CU (version += 0.1).
% \ucstEnAprobacion  Al pasar la revision sin observaciones.
% \ucstAprobado      Al ser aprobado por el usuario (version += 1.0)

%\addfigure[(adaptado de {\it For The Win} \cite{ForTheWin})]%
%    {.4}{investigacion/images/forthewin}{fig:ForTheWin}%
%    {Jerarquía de elementos de juego segun For The Win}

\begin{UseCase}[%
Autor/Ricardo Naranjo,%
Version/0.1,%
Estado/\ucstEnRevision]%
%
{CU-C09}{Ver tabla de posiciones (Competencia uno contra uno)}{%
%
 Permite al \refElem{aEstudiante}, \refElem{aProfesor} y al \refElem{aAdministrador} ver la tabla de posiciones de una instancia de la actividad competencia uno contra uno en el curso.
 Este caso de uso es una extensión del caso de uso \refElem{CU-C07}.}

	\UCitem[control]{Revisor}{ Sin asignar }
	\UCitem[control]{Último cambio}{ 13/NOV/19 }

 \UCsection{Atributos}

    \UCitem{Actor(es)}{%
        \refElem{aEstudiante},
        \refElem{aProfesor},
        \refElem{aAdministrador}
    }

	\UCitems{Propósito}{%
        \Titem Permitir al \refElem{aEstudiante}, \refElem{aProfesor} y al \refElem{aAdministrador} ver la tabla de posiciones de una instancia de la actividad de competencia uno contra uno.
	}

	\UCitem{Entradas}{\imprimeUC{entrada}}

	\UCitems{Origen}{%
        \Titem Mouse
	}

	\UCitem{Salidas}{\imprimeUC{salida}}

	\UCitem{Destino}{%
		\refElem{IU-C08}
	}

	\UCitems{Precondiciones}{%
        \Titem El plugin de competencia uno contra uno debe estar instalado en moodle.
        \Titem La instancia de la actividad de competencia uno contra uno debe estar creada.
        % Realizar el caso de uso "listar actividades disponibles?"
        % \Titem Si se trata de una actualización de un plugin la versión de este debe
               % cumplir con la regla \refElem{BR-M02}.
	}

	\UCitems{Postcondiciones}{%
        \Titem Se muestra la pantalla de tabla de posiciones de la instancia de la actividad de competencia uno contra uno \refElem{IU-C08} de acuerdo a su número de victorias.%

	}

	\UCitem{Reglas de negocio}{\imprimeUC{regla}}

	\UCitems{Errores}{%
	}

	% \UCitem{Viene de}{% Indicar si el Caso de uso es primario o se extiende de otro. La mayoría se
					  % extienden de Login.
		% EJEMPLO: \refIdElem{PY-CU1} o Caso de uso primario.
	% 	\TODO Especificar.
	% }

 \UCsection[design]{Datos de Diseño}

	\UCitems[design]{Casos de Prueba}{%
        \Titem \refElem{CPC-C09}
	}

 \UCsection[admin]{Datos de Administración de Requerimiento}

	\UCitem[admin]{Observaciones}{}

\end{UseCase}

\subsubsection{Trayectorias del caso de uso}

\begin{UCtrayectoria}%
%

    \Actor Presiona el botón {\bf Tabla posiciones} de la pantalla \refElem{IU-C02}.

    \Sistema Redirige a la pantalla de la instancia \refElem{IU-C08}.

    \Sistema Muestra un bloque por cada usuario con al menos una partida terminada, en el bloque se muestra: \salida{posición del usuario} \salida{Imagen de perfil de usuario}, \salida{Nombre de usuario}, \salida{Número de victorias del usuario}.

\end{UCtrayectoria}


% \ucstEnEdicion     Al terminar una revisión/aprobación con observaciones
%                    y al inicio del CU.
%
% \ucstEnRevision    Al terminar la edición del CU (version += 0.1).
% \ucstEnAprobacion  Al pasar la revision sin observaciones.
% \ucstAprobado      Al ser aprobado por el usuario (version += 1.0)

%\addfigure[(adaptado de {\it For The Win} \cite{ForTheWin})]%
%    {.4}{investigacion/images/forthewin}{fig:ForTheWin}%
%    {Jerarquía de elementos de juego segun For The Win}

\begin{UseCase}[%
Autor/Ricardo Naranjo,%
Version/0.1,%
Estado/\ucstEnRevision]%
%
{CU-C10}{Desafiar a un estudiante apostando (Competencia uno contra uno)}{%
%
 Permite al \refElem{aEstudiante}, \refElem{aProfesor} y al \refElem{aAdministrador} desafiar a un estudiante que se encuentra inscrito en el curso y a su vez se otorga la posibilidad de apostar en la partida.
 Este caso de uso es una extensión del caso de uso \refElem{CU-C07}.}

	\UCitem[control]{Revisor}{ Sin asignar }
	\UCitem[control]{Último cambio}{ 13/NOV/19 }

 \UCsection{Atributos}

    \UCitem{Actor(es)}{%
        \refElem{aEstudiante},
        \refElem{aProfesor},
        \refElem{aAdministrador}
    }

	\UCitems{Propósito}{%
        \Titem Permitir al \refElem{aEstudiante}, \refElem{aProfesor} y al \refElem{aAdministrador} desafiar a un estudiante que forma parte de un curso y apostar la cantidad de monedas decidida por el usuario.
        \Titem Permitir al \refElem{aEstudiante}, \refElem{aProfesor} y al \refElem{aAdministrador} terminar un desafío.
	}

	\UCitem{Entradas}{\imprimeUC{entrada}}

	\UCitems{Origen}{%
        \Titem Mouse
        \Titem Teclado
	}

	\UCitem{Salidas}{\imprimeUC{salida}}

	\UCitem{Destino}{%
		\refElem{IU-C10}
	}

	\UCitems{Precondiciones}{%
        \Titem El plugin de competencia uno contra uno debe estar instalado en moodle.
        \Titem La instancia de la actividad de competencia uno contra uno debe estar creada.
        % Realizar el caso de uso "listar actividades disponibles?"
        % \Titem Si se trata de una actualización de un plugin la versión de este debe
               % cumplir con la regla \refElem{BR-M02}.
	}

	\UCitems{Postcondiciones}{%
        \Titem Se muestra en la pantalla de historial del usuario \refElem{IU-C07} la partida que inició.%

	}

	\UCitem{Reglas de negocio}{\imprimeUC{regla}}

	\UCitems{Errores}{%
	}

	% \UCitem{Viene de}{% Indicar si el Caso de uso es primario o se extiende de otro. La mayoría se
					  % extienden de Login.
		% EJEMPLO: \refIdElem{PY-CU1} o Caso de uso primario.
	% 	\TODO Especificar.
	% }

 \UCsection[design]{Datos de Diseño}

	\UCitems[design]{Casos de Prueba}{%
        \Titem \refElem{CPC-C10}
	}

 \UCsection[admin]{Datos de Administración de Requerimiento}

	\UCitem[admin]{Observaciones}{}

\end{UseCase}

\subsubsection{Trayectorias del caso de uso}

\begin{UCtrayectoria}%
%

    \Actor Ingresa la cantidad de monedas que desea apostar en el campo "Monedas a apostar" en el bloque del estudiante que quiere desafiar en la pantalla \refElem{IU-C02}.

    \Actor Presiona el botón {\bf Desafiar}.

    \Sistema Inicia la partida y establece los valores correspondientes \refElem{comp-1v1-gmdl-partida}( \entrada{comp-1v1-gmdl-partida.gmdl-comp-vs-id} ), una entrada por cada usuario de la partida \refElem{comp-1v1-gmdl-participacion} ( \entrada{comp-1v1-gmdl-participacion.gmdl-usuario-id}, \entrada{comp-1v1-gmdl-participacion.gmdl-partida-id}, \entrada{comp-1v1-gmdl-participacion.fecha-inicio}, \refElem{comp-1v1-gmdl-participacion.puntuacion} ) y agrega la apuesta \refElem{comp-1v1-gmdl-apuesta}( \entrada{comp-1v1-gmdl-apuesta.gmdl-participacion-id}, \entrada{comp-1v1-gmdl-apuesta.monedas-plata}, \entrada{comp-1v1-gmdl-apuesta.activa} ).

    \Sistema Redirige a la pantalla del cuestionario del desafío \refElem{IU-C09}.

    \Actor Contesta las preguntas mostradas. \refTray{A}
    \label{CU-C10-contesta-cuestionario}
    \Actor Presiona el botón {\bf Terminar}.

    \Sistema Evalúa las respuestas ingresadas y calcula su puntuación.

    \Sistema Establece los valores correspondientes de la participación \refElem{comp-1v1-gmdl-participacion} (
    \entrada{comp-1v1-gmdl-participacion.fecha-fin},
    \entrada{comp-1v1-gmdl-participacion.puntuacion}).

    \Sistema Valida que el estudiante desafiado no haya terminado de contestar su cuestionario. \refTray{B}

    \Sistema Redirige a la pantalla \refElem{IU-C10}.

\end{UCtrayectoria}

\begin{UCtrayectoriaA}[Fin del caso de uso]%
  {A}{El actor desea salir del cuestionario sin terminarlo}

  \Actor Presiona cualquier botón, excepto el botón de {\bf Terminar}, que haga que abandone la pantalla.
  \Sistema Muestra mensaje de alerta \refElem{IU-C13}.
  \Actor Presiona el botón {\bf Abandonar}. \refTray{D}
  \Sistema Establece el valor correspondiente de la participación \refElem{comp-1v1-gmdl-participacion} (
  \refElem{comp-1v1-gmdl-participacion.fecha-fin}).
  \Sistema Redirige a la pantalla elegida por el actor.

\end{UCtrayectoriaA}

\begin{UCtrayectoriaA}[Fin del caso de uso]%
  {B}{El \refElem{aEstudiante} desafiado ya había terminado de contestar el cuestionario}

  \Sistema Calcula y otorga puntaje extra de acuerdo al tiempo tardado en contestar el cuestionario.
  \Sistema Comprueba que el actor haya obtenido un mayor puntaje que el \refElem{aEstudiante} desafiado. \refTray{C}
  \Sistema Lanza el evento para dar las monedas que le corresponden al actor.
  \Sistema Lanza el evento para dar la experiencia que le corresponde al actor.
  \Sistema Redirige a la pantalla \refElem{IU-C11}.
  \Sistema Muestra el puntaje de cada participante \salida{comp-1v1-gmdl-participacion.puntuacion}.

\end{UCtrayectoriaA}

\begin{UCtrayectoriaA}[Fin del caso de uso]%
  {C}{El \refElem{aEstudiante} desafiado obtuvo un mayor puntaje que el actor}

  \Sistema Lanza el evento para dar las monedas que le corresponden al \refElem{aEstudiante} desafiado.
  \Sistema Lanza el evento para dar la experiencia que le corresponde al \refElem{aEstudiante} desafiado.
  \Sistema Redirige a la pantalla \refElem{IU-C12}.
  \Sistema Muestra el puntaje de cada participante \refElem{comp-1v1-gmdl-participacion.puntuacion}.

\end{UCtrayectoriaA}

\begin{UCtrayectoriaA}{D}{El actor no quiere abandonar el cuestionario}

  \Actor Presiona el botón {\bf Cancelar}.
  \Sistema Cierra el mensaje de alerta.
  \Sistema Regresa al paso \ref{CU-C10-contesta-cuestionario}

\end{UCtrayectoriaA}


% \ucstEnEdicion     Al terminar una revisión/aprobación con observaciones
%                    y al inicio del CU.
%
% \ucstEnRevision    Al terminar la edición del CU (version += 0.1).
% \ucstEnAprobacion  Al pasar la revision sin observaciones.
% \ucstAprobado      Al ser aprobado por el usuario (version += 1.0)

%\addfigure[(adaptado de {\it For The Win} \cite{ForTheWin})]%
%    {.4}{investigacion/images/forthewin}{fig:ForTheWin}%
%    {Jerarquía de elementos de juego segun For The Win}

\begin{UseCase}[%
Autor/Ricardo Naranjo,%
Version/0.1,%
Estado/\ucstEnRevision]%
%
{CU-C11}{Desafiar a un estudiante sin apostar (Competencia uno contra uno)}{%
%
 Permite al \refElem{aEstudiante}, \refElem{aProfesor} y al \refElem{aAdministrador} desafiar a un estudiante que se encuentra inscrito en el curso.
 Este caso de uso es una extensión del caso de uso \refElem{CU-C07} en la trayectoria alternativa A.}

	\UCitem[control]{Revisor}{ Sin asignar }
	\UCitem[control]{Último cambio}{ 13/NOV/19 }

 \UCsection{Atributos}

    \UCitem{Actor(es)}{%
        \refElem{aEstudiante},
        \refElem{aProfesor},
        \refElem{aAdministrador}
    }

	\UCitems{Propósito}{%
        \Titem Permitir al \refElem{aEstudiante}, \refElem{aProfesor} y al \refElem{aAdministrador} desafiar a un estudiante que forma parte de un curso.
        \Titem Permitir al \refElem{aEstudiante}, \refElem{aProfesor} y al \refElem{aAdministrador} terminar un desafío.
	}

	\UCitem{Entradas}{\imprimeUC{entrada}}

	\UCitems{Origen}{%
        \Titem Mouse
        \Titem Teclado
	}

	\UCitem{Salidas}{\imprimeUC{salida}}

	\UCitem{Destino}{%
		\refElem{IU-C10}
	}

	\UCitems{Precondiciones}{%
        \Titem El plugin de competencia uno contra uno debe estar instalado en moodle.
        \Titem La instancia de la actividad de competencia uno contra uno debe estar creada.
        % Realizar el caso de uso "listar actividades disponibles?"
        % \Titem Si se trata de una actualización de un plugin la versión de este debe
               % cumplir con la regla \refElem{BR-M02}.
	}

	\UCitems{Postcondiciones}{%
        \Titem Se muestra en la pantalla de historial del usuario \refElem{IU-C07} la partida que inició.%

	}

	\UCitem{Reglas de negocio}{\imprimeUC{regla}}

	\UCitems{Errores}{%
	}

	% \UCitem{Viene de}{% Indicar si el Caso de uso es primario o se extiende de otro. La mayoría se
					  % extienden de Login.
		% EJEMPLO: \refIdElem{PY-CU1} o Caso de uso primario.
	% 	\TODO Especificar.
	% }

 \UCsection[design]{Datos de Diseño}

	\UCitems[design]{Casos de Prueba}{%
        \Titem \refElem{CPC-C11}
	}

 \UCsection[admin]{Datos de Administración de Requerimiento}

	\UCitem[admin]{Observaciones}{}

\end{UseCase}

\subsubsection{Trayectorias del caso de uso}

\begin{UCtrayectoria}%
%

    \Actor Presiona el botón {\bf Desafiar} en el bloque del estudiante que quiere desafiar en la pantalla \refElem{IU-C02a}.

    \Sistema Inicia la partida y establece los valores correspondientes \refElem{comp-1v1-gmdl-partida}( \entrada{comp-1v1-gmdl-partida.gmdl-comp-vs-id} ), y una entrada por cada usuario de la partida \refElem{comp-1v1-gmdl-participacion} ( \entrada{comp-1v1-gmdl-participacion.gmdl-usuario-id}, \entrada{comp-1v1-gmdl-participacion.gmdl-partida-id}, \entrada{comp-1v1-gmdl-participacion.fecha-inicio}, \refElem{comp-1v1-gmdl-participacion.puntuacion} ).

    \Sistema Redirige a la pantalla del cuestionario del desafío \refElem{IU-C09}.

    \Actor Contesta las preguntas mostradas. \refTray{A}
    \label{CU-C10-contesta-cuestionario}
    \Actor Presiona el botón {\bf Terminar}.

    \Sistema Evalúa las respuestas ingresadas y calcula su puntuación.

    \Sistema Establece los valores correspondientes de la participación \refElem{comp-1v1-gmdl-participacion} (
    \entrada{comp-1v1-gmdl-participacion.fecha-fin},
    \entrada{comp-1v1-gmdl-participacion.puntuacion}).

    \Sistema Valida que el \refElem{aEstudiante} desafiado no haya terminado de contestar su cuestionario. \refTray{B}

    \Sistema Redirige a la pantalla \refElem{IU-C10}.

\end{UCtrayectoria}

\begin{UCtrayectoriaA}[Fin del caso de uso]%
  {A}{El actor desea salir del cuestionario sin terminarlo}

  \Actor Presiona cualquier botón, excepto el botón de {\bf Terminar}, que haga que abandone la pantalla.
  \Sistema Muestra mensaje de alerta \refElem{IU-C13}.
  \Actor Presiona el botón {\bf Abandonar}. \refTray{D}
  \Sistema Establece el valor correspondiente de la participación \refElem{comp-1v1-gmdl-participacion} (
  \refElem{comp-1v1-gmdl-participacion.fecha-fin}).
  \Sistema Redirige a la pantalla elegida por el actor.

\end{UCtrayectoriaA}

\begin{UCtrayectoriaA}[Fin del caso de uso]%
  {B}{El \refElem{aEstudiante} desafiado ya había terminado de contestar el cuestionario}

  \Sistema Calcula y otorga puntaje extra de acuerdo al tiempo tardado en contestar el cuestionario.
  \Sistema Comprueba que el actor haya obtenido un mayor puntaje que el \refElem{aEstudiante} desafiado. \refTray{C}
  \Sistema Lanza el evento para dar las monedas que le corresponden al actor.
  \Sistema Lanza el evento para dar la experiencia que le corresponde al actor.
  \Sistema Redirige a la pantalla \refElem{IU-C11}.
  \Sistema Muestra el puntaje de cada participante \salida{comp-1v1-gmdl-participacion.puntuacion}.

\end{UCtrayectoriaA}

\begin{UCtrayectoriaA}[Fin del caso de uso]%
  {C}{El \refElem{aEstudiante} desafiado obtuvo un mayor puntaje que el actor}

  \Sistema Lanza el evento para dar las monedas que le corresponde al \refElem{aEstudiante} desafiado.
  \Sistema Lanza el evento para dar la experiencia que le corresponde al \refElem{aEstudiante} desafiado.
  \Sistema Redirige a la pantalla \refElem{IU-C12}.
  \Sistema Muestra el puntaje de cada participante \refElem{comp-1v1-gmdl-participacion.puntuacion}.

\end{UCtrayectoriaA}

\begin{UCtrayectoriaA}{D}{El actor no quiere abandonar el cuestionario}

  \Actor Presiona el botón {\bf Cancelar}.
  \Sistema Cierra el mensaje de alerta.
  \Sistema Regresa al paso \ref{CU-C10-contesta-cuestionario}

\end{UCtrayectoriaA}


% \ucstEnEdicion     Al terminar una revisión/aprobación con observaciones
%                    y al inicio del CU.
%
% \ucstEnRevision    Al terminar la edición del CU (version += 0.1).
% \ucstEnAprobacion  Al pasar la revision sin observaciones.
% \ucstAprobado      Al ser aprobado por el usuario (version += 1.0)

%\addfigure[(adaptado de {\it For The Win} \cite{ForTheWin})]%
%    {.4}{investigacion/images/forthewin}{fig:ForTheWin}%
%    {Jerarquía de elementos de juego segun For The Win}

\begin{UseCase}[%
Autor/Ricardo Naranjo,%
Version/0.1,%
Estado/\ucstEnRevision]%
%
{CU-C12}{Ver estado de instancia de actividad (Competencia uno contra sistema)}{%
%
 Permite al \refElem{aEstudiante}, \refElem{aProfesor} y al \refElem{aAdministrador} ver el estado actual de una instancia de la actividad competencia uno contra sistema en el curso.
 Este caso de uso es una extensión del caso de uso {\it Ver curso} que es propio de moodle.}

	\UCitem[control]{Revisor}{ Sin asignar }
	\UCitem[control]{Último cambio}{ 13/NOV/19 }

 \UCsection{Atributos}

    \UCitem{Actor(es)}{%
        \refElem{aEstudiante},
        \refElem{aProfesor},
        \refElem{aAdministrador}
    }

	\UCitems{Propósito}{%
        \Titem Permitir al \refElem{aEstudiante}, \refElem{aProfesor} y al \refElem{aAdministrador} ver el estado actual de una instancia de la actividad de competencia uno contra sistema.
	}

	\UCitem{Entradas}{\imprimeUC{entrada}}

	\UCitems{Origen}{%
        \Titem Mouse
	}

	\UCitem{Salidas}{\imprimeUC{salida} \begin{itemize}
    \item {\bf Icono de dificultad vencida}\IUCpuvencida
    \item {\bf Icono de dificultad no vencida}\IUCpunovencida
  \end{itemize} }

	\UCitem{Destino}{%
		\refElem{IU-C04}
	}

	\UCitems{Precondiciones}{%
        \Titem El plugin de competencia uno contra sistema debe estar instalado en moodle.
        \Titem La instancia de la actividad de competencia uno contra sistema debe estar creada.
        % Realizar el caso de uso "listar actividades disponibles?"
        % \Titem Si se trata de una actualización de un plugin la versión de este debe
               % cumplir con la regla \refElem{BR-M02}.
	}

	\UCitems{Postcondiciones}{%
        \Titem La pantalla principal de la instancia de la actividad de competencia uno contra sistema \refElem{IU-C04} debe mostrar los datos pertinentes al usuario que realizó el caso de uso.%

	}

	\UCitem{Reglas de negocio}{\imprimeUC{regla}}

	\UCitems{Errores}{%
	}

	% \UCitem{Viene de}{% Indicar si el Caso de uso es primario o se extiende de otro. La mayoría se
					  % extienden de Login.
		% EJEMPLO: \refIdElem{PY-CU1} o Caso de uso primario.
	% 	\TODO Especificar.
	% }

 \UCsection[design]{Datos de Diseño}

	\UCitems[design]{Casos de Prueba}{%
        \Titem \refElem{CPC-C12}
	}

 \UCsection[admin]{Datos de Administración de Requerimiento}

	\UCitem[admin]{Observaciones}{}

\end{UseCase}

\subsubsection{Trayectorias del caso de uso}

\begin{UCtrayectoria}%
%

    \Actor Presiona el nombre de la instancia a la que quiere acceder en la pantalla \refElem{IU-M07}.

    \Sistema Redirige a la pantalla principal de la instancia \refElem{IU-C04}.

    \Sistema Muestra las dificultades del sistema vencido por medio del icono \IUCpuvencida.

    \Sistema Muestra las dificultades del sistema que no han sido vencidas por medio del icono \IUCpunovencida.

    \Sistema Muestra en la sección {\bf Desafiar computadora} las dificultades que se pueden desafiar \salida{comp-cpu-gmdl-dificultad-cpu}

\end{UCtrayectoria}

\subsubsection{Puntos de extensión}

\UCExtensionPoint{Ver historial de sus partidas}{%

    El \refElem{aAdministrador}, \refElem{aProfesor} o \refElem{aEstudiante} desea ver su historial de las partidas de competencia uno contra sistema.
%
    }{Al final de la trayectoria principal del caso de uso.
%
    }{\refElem{CU-C15}}


\UCExtensionPoint{Ver tabla de puntuaciones}{%

    El \refElem{aAdministrador}, \refElem{aProfesor} o \refElem{aEstudiante} desea ver la tabla de puntuaciones de una instancia de competencia uno contra sistema.
%
    }{Al final de la trayectoria principal del caso de uso.
%
    }{\refElem{CU-C14}}

  \UCExtensionPoint{Desafiar al sistema}{%

      El \refElem{aAdministrador}, \refElem{aProfesor} o \refElem{aEstudiante} desea desafiar al sistema en alguna de sus dificultades.
  %
      }{Al final de la trayectoria principal del caso de uso.
  %
      }{\refElem{CU-C13}}


% \ucstEnEdicion     Al terminar una revisión/aprobación con observaciones
%                    y al inicio del CU.
%
% \ucstEnRevision    Al terminar la edición del CU (version += 0.1).
% \ucstEnAprobacion  Al pasar la revision sin observaciones.
% \ucstAprobado      Al ser aprobado por el usuario (version += 1.0)

%\addfigure[(adaptado de {\it For The Win} \cite{ForTheWin})]%
%    {.4}{investigacion/images/forthewin}{fig:ForTheWin}%
%    {Jerarquía de elementos de juego segun For The Win}

\begin{UseCase}[%
Autor/Ricardo Naranjo,%
Version/0.1,%
Estado/\ucstEnRevision]%
%
{CU-C13}{Desafiar al sistema (Competencia uno contra sistema)}{%
%
 Permite al \refElem{aEstudiante}, \refElem{aProfesor} y al \refElem{aAdministrador} desafiar al sistema.
 Este caso de uso es una extensión del caso de uso \refElem{CU-C12}.}

	\UCitem[control]{Revisor}{ Sin asignar }
	\UCitem[control]{Último cambio}{ 13/NOV/19 }

 \UCsection{Atributos}

    \UCitem{Actor(es)}{%
        \refElem{aEstudiante},
        \refElem{aProfesor},
        \refElem{aAdministrador}
    }

	\UCitems{Propósito}{%
        \Titem Permitir al \refElem{aEstudiante}, \refElem{aProfesor} y al \refElem{aAdministrador} desafiar al sistema.
        \Titem Permitir al \refElem{aEstudiante}, \refElem{aProfesor} y al \refElem{aAdministrador} terminar un desafío.
	}

	\UCitem{Entradas}{\imprimeUC{entrada}}

	\UCitems{Origen}{%
        \Titem Mouse
        \Titem Teclado
	}

	\UCitem{Salidas}{\imprimeUC{salida}}

	\UCitem{Destino}{%
		\refElem{IU-C10}
	}

	\UCitems{Precondiciones}{%
        \Titem El plugin de competencia uno contra sistema debe estar instalado en moodle.
        \Titem La instancia de la actividad de competencia uno contra sistema debe estar creada.
        % Realizar el caso de uso "listar actividades disponibles?"
        % \Titem Si se trata de una actualización de un plugin la versión de este debe
               % cumplir con la regla \refElem{BR-M02}.
	}

	\UCitems{Postcondiciones}{%
        \Titem Se muestra en la pantalla de historial del usuario \refElem{IU-C14} el desafío nuevo.%
	}

	\UCitem{Reglas de negocio}{\imprimeUC{regla}}

	\UCitems{Errores}{%
	}

	% \UCitem{Viene de}{% Indicar si el Caso de uso es primario o se extiende de otro. La mayoría se
					  % extienden de Login.
		% EJEMPLO: \refIdElem{PY-CU1} o Caso de uso primario.
	% 	\TODO Especificar.
	% }

 \UCsection[design]{Datos de Diseño}

	\UCitems[design]{Casos de Prueba}{%
        \Titem \refElem{CPC-C13}
	}

 \UCsection[admin]{Datos de Administración de Requerimiento}

	\UCitem[admin]{Observaciones}{}

\end{UseCase}

\subsubsection{Trayectorias del caso de uso}

\begin{UCtrayectoria}%
%

    \Actor Selecciona la dificultad deseada en el campo {\bf Seleccione una dificultad} de la pantalla \refElem{IU-C04}.

    \Actor Presiona el botón {\bf Empezar}.

    \Sistema Inicia la partida y establece los valores correspondientes \refElem{comp-cpu-gmdl-intento}( \entrada{comp-cpu-gmdl-intento.gmdl-dificultad-cpu-id}, \entrada{comp-cpu-gmdl-intento.gmdl-comp-cpu-id}, \entrada{comp-cpu-gmdl-intento.gmdl-usuario-id}, \entrada{comp-cpu-gmdl-intento.fecha-inicio} ).

    \Sistema Redirige a la pantalla del cuestionario del desafío \refElem{IU-C09}.

    \Actor Contesta las preguntas mostradas.

    \Actor Presiona el botón {\bf Terminar}.

    \Sistema Evalúa las respuestas ingresadas y calcula su puntuación.

    \Sistema Contesta el cuestionario y calcula su puntuación.

    \Sistema Establece los valores correspondientes del intento \refElem{comp-cpu-gmdl-intento} (
    \entrada{comp-cpu-gmdl-intento.fecha-fin},
    \entrada{comp-cpu-gmdl-intento.puntuacion-cpu},
    \entrada{comp-cpu-gmdl-intento.puntuacion-usuario}).

    \Sistema Valida que el actor haya sacado más o igual puntuación que el sistema. \refTray{A}

    \Sistema Lanza el evento para dar las monedas que le corresponden al actor.
    \Sistema Lanza el evento para dar la experiencia que le corresponde al actor.

    \Sistema Redirige a la pantalla \refElem{IU-C11}.
    \Sistema Muestra el puntaje del actor y del sistema \salida{comp-cpu-gmdl-intento.puntuacion-usuario}, \salida{comp-cpu-gmdl-intento.puntuacion-cpu}.

\end{UCtrayectoria}

\begin{UCtrayectoriaA}[Fin del caso de uso]%
  {A}{El sistema obtuvo un mayor puntaje que el actor}

  \Sistema Redirige a la pantalla \refElem{IU-C12}.
  \Sistema Muestra el puntaje del actor y del sistema \refElem{comp-cpu-gmdl-intento.puntuacion-usuario}, \refElem{comp-cpu-gmdl-intento.puntuacion-cpu}.

\end{UCtrayectoriaA}


% \ucstEnEdicion     Al terminar una revisión/aprobación con observaciones
%                    y al inicio del CU.
%
% \ucstEnRevision    Al terminar la edición del CU (version += 0.1).
% \ucstEnAprobacion  Al pasar la revision sin observaciones.
% \ucstAprobado      Al ser aprobado por el usuario (version += 1.0)

%\addfigure[(adaptado de {\it For The Win} \cite{ForTheWin})]%
%    {.4}{investigacion/images/forthewin}{fig:ForTheWin}%
%    {Jerarquía de elementos de juego segun For The Win}

\begin{UseCase}[%
Autor/Ricardo Naranjo,%
Version/0.1,%
Estado/\ucstEnRevision]%
%
{CU-C14}{Ver tabla de puntuaciones (Competencia uno contra sistema)}{%
%
 Permite al \refElem{aEstudiante}, \refElem{aProfesor} y al \refElem{aAdministrador} ver la tabla de puntuaciones de una instancia de la actividad competencia uno contra sistema en el curso.
 Este caso de uso es una extensión del caso de uso \refElem{CU-C12}.}

	\UCitem[control]{Revisor}{ Sin asignar }
	\UCitem[control]{Último cambio}{ 13/NOV/19 }

 \UCsection{Atributos}

    \UCitem{Actor(es)}{%
        \refElem{aEstudiante},
        \refElem{aProfesor},
        \refElem{aAdministrador}
    }

	\UCitems{Propósito}{%
        \Titem Permitir al \refElem{aEstudiante}, \refElem{aProfesor} y al \refElem{aAdministrador} ver la tabla de puntuaciones de una instancia de la actividad de competencia uno contra sistema.
	}

	\UCitem{Entradas}{\imprimeUC{entrada}}

	\UCitems{Origen}{%
        \Titem Mouse
	}

	\UCitem{Salidas}{\imprimeUC{salida}}

	\UCitem{Destino}{%
		\refElem{IU-C15}
	}

	\UCitems{Precondiciones}{%
        \Titem El plugin de competencia uno contra sistema debe estar instalado en moodle.
        \Titem La instancia de la actividad de competencia uno contra sistema debe estar creada.
        % Realizar el caso de uso "listar actividades disponibles?"
        % \Titem Si se trata de una actualización de un plugin la versión de este debe
               % cumplir con la regla \refElem{BR-M02}.
	}

	\UCitems{Postcondiciones}{%
        \Titem Se muestra la pantalla de tabla de puntuaciones de la instancia de la actividad de competencia uno contra sistema \refElem{IU-C15} de acuerdo a la dificultad seleccionada.%

	}

	\UCitem{Reglas de negocio}{\imprimeUC{regla}}

	\UCitems{Errores}{%
	}

	% \UCitem{Viene de}{% Indicar si el Caso de uso es primario o se extiende de otro. La mayoría se
					  % extienden de Login.
		% EJEMPLO: \refIdElem{PY-CU1} o Caso de uso primario.
	% 	\TODO Especificar.
	% }

 \UCsection[design]{Datos de Diseño}

	\UCitems[design]{Casos de Prueba}{%
        \Titem \refElem{CPC-C14}
	}

 \UCsection[admin]{Datos de Administración de Requerimiento}

	\UCitem[admin]{Observaciones}{}

\end{UseCase}

\subsubsection{Trayectorias del caso de uso}

\begin{UCtrayectoria}%
%

    \Actor Presiona el botón {\bf Puntuaciones} de la pantalla \refElem{IU-C04}.

    \Sistema Redirige a la pantalla de la instancia \refElem{IU-C15}.

    \Sistema Muestra un campo {\bf Seleccione una dificultad} que está por defecto en fácil. \refTray{A}

    \Sistema Muestra el \salida{comp-cpu-gmdl-dificultad-cpu.nombre} seleccionada en el campo {\bf Seleccione una dificultad}
    \label{CU-C14-muestra-informacion}

    \Sistema Muestra una tabla que presenta una fila, que contiene \salida{Posición del usuario}, \salida{Imagen de perfil}, \salida{Nombre de usuario}, \salida{comp-cpu-gmdl-intento.puntuacion-usuario} , por cada primer intento de cada usuario en esta instancia de la competencia en la dificultad seleccionada.

    \Sistema Muestra una tabla que presenta una fila, que contiene {\bf Posición del usuario}, {\bf Imagen de perfil}, {\bf Nombre de usuario}, \refElem{comp-cpu-gmdl-intento.puntuacion-usuario} , por cada mejor intento de cada usuario en esta instancia de la competencia en la dificultad seleccionada.

\end{UCtrayectoria}

\begin{UCtrayectoriaA}{A}{El actor desea seleccionar otra dificultad a ver}

  \Actor Selecciona en el campo {\bf Seleccione una dificultad} la dificultad deseada.
  \Sistema Regresa al paso \ref{CU-C14-muestra-informacion}

\end{UCtrayectoriaA}


% \ucstEnEdicion     Al terminar una revisión/aprobación con observaciones
%                    y al inicio del CU.
%
% \ucstEnRevision    Al terminar la edición del CU (version += 0.1).
% \ucstEnAprobacion  Al pasar la revision sin observaciones.
% \ucstAprobado      Al ser aprobado por el usuario (version += 1.0)

%\addfigure[(adaptado de {\it For The Win} \cite{ForTheWin})]%
%    {.4}{investigacion/images/forthewin}{fig:ForTheWin}%
%    {Jerarquía de elementos de juego segun For The Win}

\begin{UseCase}[%
Autor/Ricardo Naranjo,%
Version/0.1,%
Estado/\ucstEnRevision]%
%
{CU-C15}{Ver historial de sus partidas (Competencia uno contra sistema)}{%
%
 Permite al \refElem{aEstudiante}, \refElem{aProfesor} y al \refElem{aAdministrador} ver su historial de una instancia de la actividad competencia uno contra sistema en el curso.
 Este caso de uso es una extensión del caso de uso \refElem{CU-C12}.}

	\UCitem[control]{Revisor}{ Sin asignar }
	\UCitem[control]{Último cambio}{ 13/NOV/19 }

 \UCsection{Atributos}

    \UCitem{Actor(es)}{%
        \refElem{aEstudiante},
        \refElem{aProfesor},
        \refElem{aAdministrador}
    }

	\UCitems{Propósito}{%
        \Titem Permitir al \refElem{aEstudiante}, \refElem{aProfesor} y al \refElem{aAdministrador} ver su historial de una instancia de la actividad de competencia uno contra sistema.
	}

	\UCitem{Entradas}{\imprimeUC{entrada}}

	\UCitems{Origen}{%
        \Titem Mouse
	}

	\UCitem{Salidas}{\imprimeUC{salida}\begin{itemize}
    \item {\bf Icono de dificultad vencida}\IUCpuvencida
    \item {\bf Icono de dificultad no vencida}\IUCpunovencida
  \end{itemize}}

	\UCitem{Destino}{%
		\refElem{IU-C14}
	}

	\UCitems{Precondiciones}{%
        \Titem El plugin de competencia uno contra sistema debe estar instalado en moodle.
        \Titem La instancia de la actividad de competencia uno contra sistema debe estar creada.
        % Realizar el caso de uso "listar actividades disponibles?"
        % \Titem Si se trata de una actualización de un plugin la versión de este debe
               % cumplir con la regla \refElem{BR-M02}.
	}

	\UCitems{Postcondiciones}{%
        \Titem La pantalla de historial de la instancia de la actividad de competencia uno contra sistema \refElem{IU-C07} debe mostrar los datos pertinentes al actor.%

	}

	\UCitem{Reglas de negocio}{\imprimeUC{regla}}

	\UCitems{Errores}{%
	}

	% \UCitem{Viene de}{% Indicar si el Caso de uso es primario o se extiende de otro. La mayoría se
					  % extienden de Login.
		% EJEMPLO: \refIdElem{PY-CU1} o Caso de uso primario.
	% 	\TODO Especificar.
	% }

 \UCsection[design]{Datos de Diseño}

	\UCitems[design]{Casos de Prueba}{%
        \Titem \refElem{CPC-C15}
	}

 \UCsection[admin]{Datos de Administración de Requerimiento}

	\UCitem[admin]{Observaciones}{}

\end{UseCase}

\subsubsection{Trayectorias del caso de uso}

\begin{UCtrayectoria}%
%

    \Actor Presiona el botón {\bf Historial} de la pantalla \refElem{IU-C04}.

    \Sistema Redirige a la pantalla historial de la instancia \refElem{IU-C14}.

    \Sistema Muestra en la tabla {\bf Intentos realizados}: Una fila por cada intento del actor que contiene: \salida{comp-cpu-gmdl-dificultad-cpu.nombre}, \salida{comp-cpu-gmdl-intento.puntuacion-usuario}, \salida{comp-cpu-gmdl-intento.puntuacion-cpu}, el icono \IUCpuvencida si la puntuación del usuario en ese intento es mayor o igual que la puntuación del sistema. \refTray{A}

\end{UCtrayectoria}

\begin{UCtrayectoriaA}[Fin del caso de uso]{A}{La puntuación del usuario en el intento fue menor que la puntuación del sistema}

  \Sistema Muestra en la tabla {\bf Intentos realizados}: Una fila por cada intento del actor que contiene: \refElem{comp-cpu-gmdl-dificultad-cpu.nombre}, \refElem{comp-cpu-gmdl-intento.puntuacion-usuario}, \refElem{comp-cpu-gmdl-intento.puntuacion-cpu}, el icono \IUCpunovencida.

\end{UCtrayectoriaA}


% \ucstEnEdicion     Al terminar una revisión/aprobación con observaciones
%                    y al inicio del CU.
%
% \ucstEnRevision    Al terminar la edición del CU (version += 0.1).
% \ucstEnAprobacion  Al pasar la revision sin observaciones.
% \ucstAprobado      Al ser aprobado por el usuario (version += 1.0)

%\addfigure[(adaptado de {\it For The Win} \cite{ForTheWin})]%
%    {.4}{investigacion/images/forthewin}{fig:ForTheWin}%
%    {Jerarquía de elementos de juego segun For The Win}

\begin{UseCase}[%
Autor/Ricardo Naranjo,%
Version/0.1,%
Estado/\ucstEnRevision]%
%
{CU-C16}{Aceptar un desafío apostando (Competencia uno contra uno)}{%
%
 Permite al \refElem{aEstudiante} aceptar un desafío y a su vez se otorga la posibilidad de apostar en la partida.
 Este caso de uso es una extensión del caso de uso \refElem{CU-C07}.}

	\UCitem[control]{Revisor}{ Sin asignar }
	\UCitem[control]{Último cambio}{ 13/NOV/19 }

 \UCsection{Atributos}

    \UCitem{Actor(es)}{%
        \refElem{aEstudiante}
    }

	\UCitems{Propósito}{%
        \Titem Permitir al \refElem{aEstudiante} aceptar un desafío y apostar la cantidad de monedas decidida por el usuario.
        \Titem Permitir al \refElem{aEstudiante} terminar un desafío.
	}

	\UCitem{Entradas}{\imprimeUC{entrada}}

	\UCitems{Origen}{%
        \Titem Mouse
        \Titem Teclado
	}

	\UCitem{Salidas}{\imprimeUC{salida}}

	\UCitem{Destino}{%
		\refElem{IU-C10}
	}

	\UCitems{Precondiciones}{%
        \Titem El plugin de competencia uno contra uno debe estar instalado en moodle.
        \Titem La instancia de la actividad de competencia uno contra uno debe estar creada.
        % Realizar el caso de uso "listar actividades disponibles?"
        % \Titem Si se trata de una actualización de un plugin la versión de este debe
               % cumplir con la regla \refElem{BR-M02}.
	}

	\UCitems{Postcondiciones}{%
        \Titem Se muestra en la pantalla de historial del usuario \refElem{IU-C07} la nueva partida.%

	}

	\UCitem{Reglas de negocio}{\imprimeUC{regla}}

	\UCitems{Errores}{%
	}

	% \UCitem{Viene de}{% Indicar si el Caso de uso es primario o se extiende de otro. La mayoría se
					  % extienden de Login.
		% EJEMPLO: \refIdElem{PY-CU1} o Caso de uso primario.
	% 	\TODO Especificar.
	% }

 \UCsection[design]{Datos de Diseño}

	\UCitems[design]{Casos de Prueba}{%
        \Titem \refElem{CPC-C16}
	}

 \UCsection[admin]{Datos de Administración de Requerimiento}

	\UCitem[admin]{Observaciones}{}

\end{UseCase}

\subsubsection{Trayectorias del caso de uso}

\begin{UCtrayectoria}%
%

    \Actor Ingresa la cantidad de monedas que desea apostar en el campo "Monedas a apostar" en el bloque del usuario del que quiere aceptar el desafío en la pantalla \refElem{IU-C02} en la sección "Desafíos pendientes".

    \Actor Presiona el botón {\bf Aceptar desafío}.

    \Sistema Inicia la partida y establece los valores correspondientes en \refElem{comp-1v1-gmdl-participacion} ( \entrada{comp-1v1-gmdl-participacion.fecha-inicio} ) y agrega la apuesta \refElem{comp-1v1-gmdl-apuesta}( \entrada{comp-1v1-gmdl-apuesta.gmdl-participacion-id}, \entrada{comp-1v1-gmdl-apuesta.monedas-plata}, \entrada{comp-1v1-gmdl-apuesta.activa} ).

    \Sistema Redirige a la pantalla del cuestionario del desafío \refElem{IU-C09}.

    \Actor Contesta las preguntas mostradas. \refTray{A}
    \label{CU-C10-contesta-cuestionario}
    \Actor Presiona el botón {\bf Terminar}.

    \Sistema Evalúa las respuestas ingresadas y calcula su puntuación.

    \Sistema Establece los valores correspondientes de la participación \refElem{comp-1v1-gmdl-participacion} (
    \entrada{comp-1v1-gmdl-participacion.fecha-fin},
    \entrada{comp-1v1-gmdl-participacion.puntuacion}).

    \Sistema Valida que el usuario que lo desafió haya terminado de contestar su cuestionario. \refTray{B}

    \Sistema Calcula y otorga puntaje extra de acuerdo al tiempo tardado en contestar el cuestionario.
    \Sistema Comprueba que el actor haya obtenido un mayor puntaje que el usuario desafiante. \refTray{C}
    \Sistema Lanza el evento para dar las monedas que le corresponden al actor.
    \Sistema Lanza el evento para dar la experiencia que le corresponde al actor.
    \Sistema Redirige a la pantalla \refElem{IU-C11}.
    \Sistema Muestra el puntaje de cada participante \salida{comp-1v1-gmdl-participacion.puntuacion}.

\end{UCtrayectoria}

\begin{UCtrayectoriaA}[Fin del caso de uso]%
  {A}{El actor desea salir del cuestionario sin terminarlo}

  \Actor Presiona cualquier botón, excepto el botón de {\bf Terminar}, que haga que abandone la pantalla.
  \Sistema Muestra mensaje de alerta \refElem{IU-C13}.
  \Actor Presiona el botón {\bf Abandonar}. \refTray{D}
  \Sistema Establece el valor correspondiente de la participación \refElem{comp-1v1-gmdl-participacion} (
  \refElem{comp-1v1-gmdl-participacion.fecha-fin}).
  \Sistema Redirige a la pantalla elegida por el actor.

\end{UCtrayectoriaA}

\begin{UCtrayectoriaA}[Fin del caso de uso]%
  {B}{El usuario que lo desafió no ha terminado de contestar su cuestionario}

  \Sistema Redirige a la pantalla \refElem{IU-C10}

\end{UCtrayectoriaA}

\begin{UCtrayectoriaA}[Fin del caso de uso]%
  {C}{El usuario desafiante obtuvo un mayor puntaje que el actor}

  \Sistema Lanza el evento para dar las monedas que le corresponden al usuario desafiante.
  \Sistema Lanza el evento para dar la experiencia que le corresponde al usuario desafiante.
  \Sistema Redirige a la pantalla \refElem{IU-C12}.
  \Sistema Muestra el puntaje de cada participante \refElem{comp-1v1-gmdl-participacion.puntuacion}.

\end{UCtrayectoriaA}

\begin{UCtrayectoriaA}{D}{El actor no quiere abandonar el cuestionario}

  \Actor Presiona el botón {\bf Cancelar}.
  \Sistema Cierra el mensaje de alerta.
  \Sistema Regresa al paso \ref{CU-C10-contesta-cuestionario}

\end{UCtrayectoriaA}


% \ucstEnEdicion     Al terminar una revisión/aprobación con observaciones
%                    y al inicio del CU.
%
% \ucstEnRevision    Al terminar la edición del CU (version += 0.1).
% \ucstEnAprobacion  Al pasar la revision sin observaciones.
% \ucstAprobado      Al ser aprobado por el usuario (version += 1.0)

%\addfigure[(adaptado de {\it For The Win} \cite{ForTheWin})]%
%    {.4}{investigacion/images/forthewin}{fig:ForTheWin}%
%    {Jerarquía de elementos de juego segun For The Win}

\begin{UseCase}[%
Autor/Ricardo Naranjo,%
Version/0.1,%
Estado/\ucstEnRevision]%
%
{CU-C17}{Aceptar un desafío sin apostar (Competencia uno contra uno)}{%
%
 Permite al \refElem{aEstudiante} aceptar un desafío.
 Este caso de uso es una extensión del caso de uso \refElem{CU-C07} en la trayectoria alternativa A.}

	\UCitem[control]{Revisor}{ Sin asignar }
	\UCitem[control]{Último cambio}{ 13/NOV/19 }

 \UCsection{Atributos}

    \UCitem{Actor(es)}{%
        \refElem{aEstudiante}
    }

	\UCitems{Propósito}{%
        \Titem Permitir al \refElem{aEstudiante} aceptar un desafío.
        \Titem Permitir al \refElem{aEstudiante} terminar un desafío.
	}

	\UCitem{Entradas}{\imprimeUC{entrada}}

	\UCitems{Origen}{%
        \Titem Mouse
        \Titem Teclado
	}

	\UCitem{Salidas}{\imprimeUC{salida}}

	\UCitem{Destino}{%
		\refElem{IU-C11}
	}

	\UCitems{Precondiciones}{%
        \Titem El plugin de competencia uno contra uno debe estar instalado en moodle.
        \Titem La instancia de la actividad de competencia uno contra uno debe estar creada.
        % Realizar el caso de uso "listar actividades disponibles?"
        % \Titem Si se trata de una actualización de un plugin la versión de este debe
               % cumplir con la regla \refElem{BR-M02}.
	}

	\UCitems{Postcondiciones}{%
        \Titem Se muestra en la pantalla de historial del usuario \refElem{IU-C07} la nueva partida.%

	}

	\UCitem{Reglas de negocio}{\imprimeUC{regla}}

	\UCitems{Errores}{%
	}

	% \UCitem{Viene de}{% Indicar si el Caso de uso es primario o se extiende de otro. La mayoría se
					  % extienden de Login.
		% EJEMPLO: \refIdElem{PY-CU1} o Caso de uso primario.
	% 	\TODO Especificar.
	% }

 \UCsection[design]{Datos de Diseño}

	\UCitems[design]{Casos de Prueba}{%
        \Titem \refElem{CPC-C17}
	}

 \UCsection[admin]{Datos de Administración de Requerimiento}

	\UCitem[admin]{Observaciones}{}

\end{UseCase}

\subsubsection{Trayectorias del caso de uso}

\begin{UCtrayectoria}%
%

    \Actor Presiona el botón {\bf Aceptar desafío} en el bloque del usuario del que quiere aceptar el desafío en la pantalla \refElem{IU-C02a} en la sección "Desafíos pendientes"..

    \Sistema Inicia la partida y establece los valores correspondientes en \refElem{comp-1v1-gmdl-participacion} ( \entrada{comp-1v1-gmdl-participacion.fecha-inicio} ).

    \Sistema Redirige a la pantalla del cuestionario del desafío \refElem{IU-C09}.

    \Actor Contesta las preguntas mostradas. \refTray{A}
    \label{CU-C10-contesta-cuestionario}
    \Actor Presiona el botón {\bf Terminar}.

    \Sistema Evalúa las respuestas ingresadas y calcula su puntuación.

    \Sistema Establece los valores correspondientes de la participación \refElem{comp-1v1-gmdl-participacion} (
    \entrada{comp-1v1-gmdl-participacion.fecha-fin},
    \entrada{comp-1v1-gmdl-participacion.puntuacion}).

    \Sistema Valida que el usuario que lo desafió haya terminado de contestar su cuestionario. \refTray{B}

    \Sistema Calcula y otorga puntaje extra de acuerdo al tiempo tardado en contestar el cuestionario.
    \Sistema Comprueba que el actor haya obtenido un mayor puntaje que el usuario desafiante. \refTray{C}
    \Sistema Lanza el evento para dar las monedas que le corresponden al actor.
    \Sistema Lanza el evento para dar la experiencia que le corresponde al actor.
    \Sistema Redirige a la pantalla \refElem{IU-C11}.
    \Sistema Muestra el puntaje de cada participante \salida{comp-1v1-gmdl-participacion.puntuacion}.

\end{UCtrayectoria}

\begin{UCtrayectoriaA}[Fin del caso de uso]%
  {A}{El actor desea salir del cuestionario sin terminarlo}

  \Actor Presiona cualquier botón, excepto el botón de {\bf Terminar}, que haga que abandone la pantalla.
  \Sistema Muestra mensaje de alerta \refElem{IU-C13}.
  \Actor Presiona el botón {\bf Abandonar}. \refTray{D}
  \Sistema Establece el valor correspondiente de la participación \refElem{comp-1v1-gmdl-participacion} (
  \refElem{comp-1v1-gmdl-participacion.fecha-fin}).
  \Sistema Redirige a la pantalla elegida por el actor.

\end{UCtrayectoriaA}

\begin{UCtrayectoriaA}[Fin del caso de uso]%
  {B}{El usuario que lo desafió no ha terminado de contestar su cuestionario}

  \Sistema Redirige a la pantalla \refElem{IU-C10}

\end{UCtrayectoriaA}

\begin{UCtrayectoriaA}[Fin del caso de uso]%
  {C}{El usuario desafiante obtuvo un mayor puntaje que el actor}

  \Sistema Lanza el evento para dar las monedas que le corresponden al usuario desafiante.
  \Sistema Lanza el evento para dar la experiencia que le corresponde al usuario desafiante.
  \Sistema Redirige a la pantalla \refElem{IU-C12}.
  \Sistema Muestra el puntaje de cada participante \refElem{comp-1v1-gmdl-participacion.puntuacion}.

\end{UCtrayectoriaA}

\begin{UCtrayectoriaA}{D}{El actor no quiere abandonar el cuestionario}

  \Actor Presiona el botón {\bf Cancelar}.
  \Sistema Cierra el mensaje de alerta.
  \Sistema Regresa al paso \ref{CU-C10-contesta-cuestionario}

\end{UCtrayectoriaA}

    %
% \ucstEnEdicion     Al terminar una revisión/aprobación con observaciones
%                    y al inicio del CU.
%
% \ucstEnRevision    Al terminar la edición del CU (version += 0.1).
% \ucstEnAprobacion  Al pasar la revision sin observaciones.
% \ucstAprobado      Al ser aprobado por el usuario (version += 1.0)

\begin{UseCase}[%
Autor/Ricardo Naranjo,%
Version/0.1,%
Estado/\ucstEnRevision]%
%
{CU-C02}{Actualizar instancia (Competencia uno contra uno)}{%
%
 Permite al \refElem{aProfesor} y al \refElem{aAdministrador} actualizar una instancia de la actividad competencia uno contra uno en su curso.
 Este caso de uso es una extensión del caso de uso {\it Ver curso} que es propio de moodle.}

	\UCitem[control]{Revisor}{ Sin asignar }
	\UCitem[control]{Último cambio}{ 13/NOV/19 }

 \UCsection{Atributos}

    \UCitem{Actor(es)}{%
        \refElem{aProfesor},
        \refElem{aAdministrador}
    }

	\UCitems{Propósito}{%
        \Titem Permitir al \refElem{aProfesor} y al \refElem{aAdministrador} actualizar una instancia de la actividad de competencia uno contra uno.

        \Titem Permitir al \refElem{aEstudiante}, \refElem{aProfesor} y \refElem{aAdministrador} con acceso al curso utilizar la instancia actualizada de la actividad de competencia uno contra uno creada por el \refElem{aProfesor} o \refElem{aAdministrador}.
	}

	\UCitem{Entradas}{\imprimeUC{entrada}}

	\UCitems{Origen}{%
        \Titem Mouse
        \Titem Teclado
	}

	\UCitem{Salidas}{\imprimeUC{salida}}

	\UCitem{Destino}{%
		\refElem{IU-M08}
	}

	\UCitems{Precondiciones}{%
        \Titem El plugin de competencia uno contra uno debe estar instalado en moodle.
        \Titem La instancia de la actividad de competencia uno contra uno debe estar creada.
        % Realizar el caso de uso "listar actividades disponibles?"
        % \Titem Si se trata de una actualización de un plugin la versión de este debe
               % cumplir con la regla \refElem{BR-M02}.
	}

	\UCitems{Postcondiciones}{%
        \Titem La instancia actualizada de la actividad debe mostrarse en la pantalla \refElem{IU-M08}.%

	}

	\UCitem{Reglas de negocio}{\imprimeUC{regla}}

	\UCitems{Errores}{%
        \Titem \UCerr{Err1}{%
            No se ingresó un campo requerido en el formulario de creación de la actividad,}{% CAUSA
            no se puede actualizar la instancia de la actividad}% EFECTO
	}

	% \UCitem{Viene de}{% Indicar si el Caso de uso es primario o se extiende de otro. La mayoría se
					  % extienden de Login.
		% EJEMPLO: \refIdElem{PY-CU1} o Caso de uso primario.
	% 	\TODO Especificar.
	% }

 \UCsection[design]{Datos de Diseño}

	\UCitems[design]{Casos de Prueba}{%
        \Titem \refElem{CPC-C01}
	}

 \UCsection[admin]{Datos de Administración de Requerimiento}

	\UCitem[admin]{Observaciones}{}

\end{UseCase}

\subsubsection{Trayectorias del caso de uso}

\begin{UCtrayectoria}%
%

    \Actor Activa la edición del curso en la pantalla \refElem{IU-M08}.
    \Sistema Redirige a la pantalla de edición del curso \refElem{IU-M08aa}.
    \Actor Presiona el botón {\bf Editar} de la instancia que desea actualizar.
    \Sistema Despliega el menú \refElem{IU-M08b}.
    \Actor Presiona el botón {\bf Editar ajustes} del menú desplegable \refElem{IU-M08b}.

    \Sistema Redirige a la pantalla \refElem{IU-C06} y carga los valores de la instancia \refElem{comp-1v1-gmcompvs} (
      \salida{comp-1v1-gmcompvs.name},
      \salida{comp-1v1-gmcompvs.mdl-question-categories-id},
      \salida{comp-1v1-gmcompvs.apuestas-activas},
      \salida{comp-1v1-gmcompvs.completionnumwon}).

    \label{CU-C02-muestra-pantalla}

    \Actor Actualiza los datos correspondientes en el formulario.

    \Actor Presiona el botón {\bf Guardar cambios y regresar al curso}.\refTray{A} \refTray{B}

    \Sistema Valida que los campos ingresados sean válidos. \refTray{C} \refErr{Err1}

    \Sistema Actualiza los valores ingresados para la instancia \refElem{comp-1v1-gmcompvs} (
      \entrada{comp-1v1-gmcompvs.name},
      \entrada{comp-1v1-gmcompvs.mdl-question-categories-id},
      \entrada{comp-1v1-gmcompvs.apuestas-activas},
      \entrada{comp-1v1-gmcompvs.completionnumwon}), especificadas en el modelo de información.

    \Sistema Redirige a la pantalla \refElem{IU-M08} y muestra la instancia actualizada en el curso.

\end{UCtrayectoria}

\begin{UCtrayectoriaA}[Fin del caso de uso]{A}{El \refElem{aProfesor} o \refElem{aAdministrador} desea ver la instancia actualizada de la actividad}

    \Actor Presiona el botón {\bf Guardar cambios y mostrar} de la pantalla \refElem{IU-C06}.

    \Sistema Valida que los campos ingresados sean válidos. \refTray{C} \refErr{Err1}

    \Sistema Actualiza los valores ingresados para la instancia \refElem{comp-1v1-gmcompvs} (
      \refElem{comp-1v1-gmcompvs.name},
      \refElem{comp-1v1-gmcompvs.mdl-question-categories-id},
      \refElem{comp-1v1-gmcompvs.apuestas-activas},
      \refElem{comp-1v1-gmcompvs.completionnumwon}), especificadas en el modelo de información.

    \Sistema Redirige a la pantalla \refElem{IU-C02}.

\end{UCtrayectoriaA}

\begin{UCtrayectoriaA}[Fin del caso de uso]%
  {B}{El \refElem{aProfesor} o \refElem{aAdministrador} desea cancelar la actualización de la instancia después de mostrar el formulario de actualización}

  \Actor Presiona el botón {\bf cancelar} en la pantalla \refElem{IU-C06}.
  \Sistema Redirige a la pantalla \refElem{IU-C01}.

\end{UCtrayectoriaA}

\begin{UCtrayectoriaA}{C}{Algún dato ingresado por el \refElem{aProfesor} o \refElem{aAdministrador} es inválido}

  \Sistema Muestra un mensaje de error "-Usted debe poner un valor aquí", en los campos de la pantalla \refElem{IU-C06} que sean requeridos.
  \Sistema Regresa al paso \ref{CU-C02-muestra-pantalla}

\end{UCtrayectoriaA}
   % Configuraciones generales
    %\input{modulos/comp/CU/CU-C02-1} % Configurar visualización de niveles
    %\input{modulos/comp/CU/CU-C02-2} % Configurar esquema de comperiencia
    %\input{modulos/comp/CU/CU-C02-3} % Configurar Cventos de comperiencia
    %
% \ucstEnEdicion     Al terminar una revisión/aprobación con observaciones
%                    y al inicio del CU.
%
% \ucstEnRevision    Al terminar la edición del CU (version += 0.1).
% \ucstEnAprobacion  Al pasar la revision sin observaciones.
% \ucstAprobado      Al ser aprobado por el usuario (version += 1.0)

\begin{UseCase}[%
Autor/Ricardo Naranjo,%
Version/0.1,%
Estado/\ucstEnRevision]%
%
{CU-C03}{Eliminar instancia (Competencia uno contra uno)}{%
%
 Permite al \refElem{aProfesor} y al \refElem{aAdministrador} eliminar una instancia de la actividad competencia uno contra uno en su curso.
 Este caso de uso es una extensión del caso de uso {\it Ver curso} que es propio de moodle.}

	\UCitem[control]{Revisor}{ Sin asignar }
	\UCitem[control]{Último cambio}{ 13/NOV/19 }

 \UCsection{Atributos}

    \UCitem{Actor(es)}{%
        \refElem{aProfesor},
        \refElem{aAdministrador}
    }

	\UCitems{Propósito}{%
        \Titem Permitir al \refElem{aProfesor} y al \refElem{aAdministrador} eliminar una instancia de la actividad de competencia uno contra uno.
	}

	\UCitem{Entradas}{\imprimeUC{entrada}}

	\UCitems{Origen}{%
        \Titem Mouse
	}

	\UCitem{Salidas}{\imprimeUC{salida}}

	\UCitem{Destino}{%
		\refElem{IU-M08}
	}

	\UCitems{Precondiciones}{%
        \Titem El plugin de competencia uno contra uno debe estar instalado en moodle.
        \Titem La instancia de la actividad de competencia uno contra uno debe estar creada.
        % Realizar el caso de uso "listar actividades disponibles?"
        % \Titem Si se trata de una actualización de un plugin la versión de este debe
               % cumplir con la regla \refElem{BR-M02}.
	}

	\UCitems{Postcondiciones}{%
        \Titem La instancia de la actividad eliminada no debe mostrarse en la pantalla \refElem{IU-M08}.%

	}

	\UCitem{Reglas de negocio}{\imprimeUC{regla}}

	\UCitems{Errores}{%
	}

	% \UCitem{Viene de}{% Indicar si el Caso de uso es primario o se extiende de otro. La mayoría se
					  % extienden de Login.
		% EJEMPLO: \refIdElem{PY-CU1} o Caso de uso primario.
	% 	\TODO Especificar.
	% }

 \UCsection[design]{Datos de Diseño}

	\UCitems[design]{Casos de Prueba}{%
        \Titem \refElem{CPC-C03}
	}

 \UCsection[admin]{Datos de Administración de Requerimiento}

	\UCitem[admin]{Observaciones}{}

\end{UseCase}

\subsubsection{Trayectorias del caso de uso}

\begin{UCtrayectoria}%
%

    \Actor Activa la edición del curso en la pantalla \refElem{IU-M08}.

    \Sistema Redirige a la pantalla de edición del curso \refElem{IU-M08aa}.

    \Actor Presiona el botón {\bf Editar} de la instancia que desea eliminar.

    \Sistema Despliega el menú \refElem{IU-M08b}.

    \Actor Presiona el botón {\bf Eliminar} del menú desplegable \refElem{IU-M08b}.

    \Sistema Despliega mensaje de confirmación de eliminación. \refElem{IU-M08c}

    \Actor Presiona el botón {\bf Si}. \refTray{A}

    \Sistema Redirige a la pantalla \refElem{IU-M08} y elimina la instancia y los valores de la instancia \refElem{comp-1v1-gmcompvs}, así como los datos que dependen de la instancia en las siguientes entidades: \refElem{comp-1v1-gmdl-partida}, \refElem{comp-1v1-gmdl-participacion} y \refElem{comp-1v1-gmdl-partida}.

\end{UCtrayectoria}

\begin{UCtrayectoriaA}[Fin del caso de uso]{A}{El \refElem{aProfesor} o \refElem{aAdministrador} desea cancelar la eliminación después de mostrar el mensaje de confirmación}

  \Actor Presiona el botón {\bf No} en la mensaje de confirmación \refElem{IU-M08c}.
  \Sistema Redirige a la pantalla \refElem{IU-M08}.

\end{UCtrayectoriaA}
   % Desinstalar plugin del esquema de comperiencia
    %
% \ucstEnEdicion     Al terminar una revisión/aprobación con observaciones
%                    y al inicio del CU.
%
% \ucstEnRevision    Al terminar la edición del CU (version += 0.1).
% \ucstEnAprobacion  Al pasar la revision sin observaciones.
% \ucstAprobado      Al ser aprobado por el usuario (version += 1.0)

\begin{UseCase}[%
Autor/Ricardo Naranjo,%
Version/0.1,%
Estado/\ucstEnRevision]%
%
{CU-C04}{Crear instancia (Competencia uno contra sistema)}{%
%
 Permite al \refElem{aProfesor} y al \refElem{aAdministrador} crear una nueva instancia de la actividad competencia uno contra sistema en su curso.
 La conclusión de la trayectoria principal de esta caso de uso es una precondición para que
 algunos casos de uso del módulo de competencia puedan ejecutarse.\\%
 Este caso de uso es una extensión del caso de uso {\it Listar actividades disponibles} que es propio de moodle.}

	\UCitem[control]{Revisor}{ Sin asignar }
	\UCitem[control]{Último cambio}{ 13/NOV/19 }

 \UCsection{Atributos}

    \UCitem{Actor(es)}{%
        \refElem{aProfesor},
        \refElem{aAdministrador}
    }

	\UCitems{Propósito}{%
        \Titem Permitir al \refElem{aProfesor} y al \refElem{aAdministrador} incluir en su curso una nueva instancia de la actividad de competencia uno contra sistema.

        \Titem Permitir al \refElem{aEstudiante}, \refElem{aProfesor} y \refElem{aAdministrador} con acceso al curso utilizar la instancia de la actividad de competencia uno contra sistema creada por el \refElem{aProfesor} o \refElem{aAdministrador}.
	}

	\UCitem{Entradas}{\imprimeUC{entrada}}

	\UCitems{Origen}{%
        \Titem Mouse
        \Titem Teclado
	}

	\UCitem{Salidas}{\imprimeUC{salida}}

	\UCitem{Destino}{%
		\refElem{IU-M07}
	}

	\UCitems{Precondiciones}{%
        \Titem El plugin de competencia 1 contra sistema debe estar instalado en moodle.
        % Realizar el caso de uso "listar actividades disponibles?"
        % \Titem Si se trata de una actualización de un plugin la versión de este debe
               % cumplir con la regla \refElem{BR-M02}.
	}

	\UCitems{Postcondiciones}{%
        \Titem La nueva instancia de la actividad debe mostrarse en la pantalla \refElem{IU-M07}.%

	}

	\UCitem{Reglas de negocio}{\imprimeUC{regla}}

	\UCitems{Errores}{%
        \Titem \UCerr{Err1}{%
            No se ingresó un campo requerido en el formulario de creación de la actividad,}{% CAUSA
            no se puede crear la nueva instancia de la actividad}% EFECTO
	}

	% \UCitem{Viene de}{% Indicar si el Caso de uso es primario o se extiende de otro. La mayoría se
					  % extienden de Login.
		% EJEMPLO: \refIdElem{PY-CU1} o Caso de uso primario.
	% 	\TODO Especificar.
	% }

 \UCsection[design]{Datos de Diseño}

	\UCitems[design]{Casos de Prueba}{%
        \Titem \refElem{CPC-C01}
	}

 \UCsection[admin]{Datos de Administración de Requerimiento}

	\UCitem[admin]{Observaciones}{}

\end{UseCase}

\subsubsection{Trayectorias del caso de uso}

\begin{UCtrayectoria}%
%
    \Actor Selecciona la actividad Gamedle - Competencia 1 contra sistema en la pantalla \refElem{IU-M07a}.
    \Sistema Muestra la descripción de la actividad Gamedle - Competencia 1 contra sistema en la pantalla.

    \Actor Presiona el botón {\bf Agregar} en la pantalla. \refTray{A}
    \Sistema Redirige a la pantalla \refElem{IU-C03}.
    \label{CU-C04-muestra-pantalla}

    \Actor Ingresa los datos correspondientes en el formulario.

    \Actor Presiona el botón {\bf Guardar cambios y regresar al curso}.\refTray{B} \refTray{C}

    \Sistema Valida que los campos ingresados sean válidos. \refTray{D} \refErr{Err1}

    \Sistema Establece los valores ingresados para la nueva instancia \refElem{comp-cpu-gmcompcpu} (
      \entrada{comp-cpu-gmcompcpu.name},
      \entrada{comp-cpu-gmcompcpu.mdl-question-categories-id},
      \entrada{comp-cpu-gmcompcpu.completioncpudiff}), especificadas en el modelo de información.

    \Sistema Redirige a la pantalla \refElem{IU-M07} y muestra la nueva instancia creada en el curso.

\end{UCtrayectoria}

\begin{UCtrayectoriaA}[Fin del caso de uso]%
  {A}{El \refElem{aProfesor} o \refElem{aAdministrador} desea cancelar la creación de la nueva instancia después que se le muestra la descripción de la actividad}

  \Actor Presiona el botón {\bf cancelar} en la pantalla \refElem{IU-M07a}.
  \Sistema Cierra la pantalla \refElem{IU-M07a} y redirige a la pantalla \refElem{IU-M07}.

\end{UCtrayectoriaA}

\begin{UCtrayectoriaA}[Fin del caso de uso]{B}{El \refElem{aProfesor} o \refElem{aAdministrador} desea ver la nueva instancia de la actividad}

    \Actor Presiona el botón {\bf Guardar cambios y mostrar} de la pantalla \refElem{IU-C03}.

    \Sistema Valida que los campos ingresados sean válidos. \refTray{D} \refErr{Err1}

    \Sistema Establece los valores ingresados para la nueva instancia \refElem{comp-cpu-gmcompcpu} (
      \refElem{comp-cpu-gmcompcpu.name},
      \refElem{comp-cpu-gmcompcpu.mdl-question-categories-id},
      \refElem{comp-cpu-gmcompcpu.completioncpudiff}), especificadas en el modelo de información.

    \Sistema Redirige a la pantalla \refElem{IU-C04}.

\end{UCtrayectoriaA}

\begin{UCtrayectoriaA}[Fin del caso de uso]%
  {C}{El \refElem{aProfesor} desea cancelar la creación de la nueva instancia después de mostrar el formulario de creación}

  \Actor Presiona el botón {\bf cancelar} en la pantalla \refElem{IU-C03}.
  \Sistema Redirige a la pantalla \refElem{IU-M07}.

\end{UCtrayectoriaA}

\begin{UCtrayectoriaA}{D}{Algún dato ingresado por el \refElem{aProfesor} o \refElem{aAdministrador} es inválido}

  \Sistema Muestra un mensaje de error "-Usted debe poner un valor aquí", en los campos de la pantalla \refElem{IU-C03} que sean requeridos.
  \Sistema Regresa al paso \ref{CU-C04-muestra-pantalla}

\end{UCtrayectoriaA}
   % Crear un curso gamificado
    %
% \ucstEnEdicion     Al terminar una revisión/aprobación con observaciones
%                    y al inicio del CU.
%
% \ucstEnRevision    Al terminar la edición del CU (version += 0.1).
% \ucstEnAprobacion  Al pasar la revision sin observaciones.
% \ucstAprobado      Al ser aprobado por el usuario (version += 1.0)

\begin{UseCase}[%
Autor/Ricardo Naranjo,%
Version/0.1,%
Estado/\ucstEnRevision]%
%
{CU-C05}{Actualizar instancia (Competencia uno contra sistema)}{%
%
 Permite al \refElem{aProfesor} y al \refElem{aAdministrador} actualizar una instancia de la actividad competencia uno contra sistema en su curso.
 Este caso de uso es una extensión del caso de uso {\it Ver curso} que es propio de moodle.}

	\UCitem[control]{Revisor}{ Sin asignar }
	\UCitem[control]{Último cambio}{ 13/NOV/19 }

 \UCsection{Atributos}

    \UCitem{Actor(es)}{%
        \refElem{aProfesor},
        \refElem{aAdministrador}
    }

	\UCitems{Propósito}{%
        \Titem Permitir al \refElem{aProfesor} y al \refElem{aAdministrador} actualizar una instancia de la actividad de competencia uno contra sistema.

        \Titem Permitir al \refElem{aEstudiante}, \refElem{aProfesor} y \refElem{aAdministrador} con acceso al curso utilizar la instancia actualizada de la actividad de competencia uno contra sistema creada por el \refElem{aProfesor} o \refElem{aAdministrador}.
	}

	\UCitem{Entradas}{\imprimeUC{entrada}}

	\UCitems{Origen}{%
        \Titem Mouse
        \Titem Teclado
	}

	\UCitem{Salidas}{\imprimeUC{salida}}

	\UCitem{Destino}{%
		\refElem{IU-M07}
	}

	\UCitems{Precondiciones}{%
        \Titem El plugin de competencia uno contra sistema debe estar instalado en moodle.
        \Titem La instancia de la actividad de competencia uno contra sistema debe estar creada.
        % Realizar el caso de uso "listar actividades disponibles?"
        % \Titem Si se trata de una actualización de un plugin la versión de este debe
               % cumplir con la regla \refElem{BR-M02}.
	}

	\UCitems{Postcondiciones}{%
        \Titem La instancia actualizada de la actividad debe mostrarse en la pantalla \refElem{IU-M07}.%

	}

	\UCitem{Reglas de negocio}{\imprimeUC{regla}}

	\UCitems{Errores}{%
        \Titem \UCerr{Err1}{%
            No se ingresó un campo requerido en el formulario de creación de la actividad,}{% CAUSA
            no se puede actualizar la instancia de la actividad}% EFECTO
	}

	% \UCitem{Viene de}{% Indicar si el Caso de uso es primario o se extiende de otro. La mayoría se
					  % extienden de Login.
		% EJEMPLO: \refIdElem{PY-CU1} o Caso de uso primario.
	% 	\TODO Especificar.
	% }

 \UCsection[design]{Datos de Diseño}

	\UCitems[design]{Casos de Prueba}{%
        \Titem \refElem{CPC-C01}
	}

 \UCsection[admin]{Datos de Administración de Requerimiento}

	\UCitem[admin]{Observaciones}{}

\end{UseCase}

\subsubsection{Trayectorias del caso de uso}

\begin{UCtrayectoria}%
%

    \Actor Activa la edición del curso en la pantalla \refElem{IU-M07}.
    \Sistema Redirige a la pantalla de edición del curso \refElem{IU-M07aa}.
    \Actor Presiona el botón {\bf Editar} de la instancia que desea actualizar.
    \Sistema Despliega el menú \refElem{IU-M07b}.
    \Actor Presiona el botón {\bf Editar ajustes} del menú desplegable \refElem{IU-M07b}.

    \Sistema Redirige a la pantalla \refElem{IU-C05} y carga los valores de la instancia \refElem{comp-cpu-gmcompcpu} (
      \salida{comp-cpu-gmcompcpu.name},
      \salida{comp-cpu-gmcompcpu.mdl-question-categories-id},
      \salida{comp-cpu-gmcompcpu.completioncpudiff}), especificadas en el modelo de información.

    \label{CU-C05-muestra-pantalla}

    \Actor Actualiza los datos correspondientes en el formulario.

    \Actor Presiona el botón {\bf Guardar cambios y regresar al curso}.\refTray{A} \refTray{B}

    \Sistema Valida que los campos ingresados sean válidos. \refTray{C} \refErr{Err1}

    \Sistema Actualiza los valores ingresados para la instancia \refElem{comp-cpu-gmcompcpu} (
      \entrada{comp-cpu-gmcompcpu.name},
      \entrada{comp-cpu-gmcompcpu.mdl-question-categories-id},
      \entrada{comp-cpu-gmcompcpu.completioncpudiff}), especificadas en el modelo de información.

    \Sistema Redirige a la pantalla \refElem{IU-M07} y muestra la instancia actualizada en el curso.

\end{UCtrayectoria}

\begin{UCtrayectoriaA}[Fin del caso de uso]{A}{El \refElem{aProfesor} o \refElem{aAdministrador} desea ver la instancia actualizada de la actividad}

    \Actor Presiona el botón {\bf Guardar cambios y mostrar} de la pantalla \refElem{IU-C05}.

    \Sistema Valida que los campos ingresados sean válidos. \refTray{C} \refErr{Err1}

    \Sistema Actualiza los valores ingresados para la instancia \refElem{comp-cpu-gmcompcpu} (
      \refElem{comp-cpu-gmcompcpu.name},
      \refElem{comp-cpu-gmcompcpu.mdl-question-categories-id},
      \refElem{comp-cpu-gmcompcpu.completioncpudiff}), especificadas en el modelo de información.

    \Sistema Redirige a la pantalla \refElem{IU-C02}.

\end{UCtrayectoriaA}

\begin{UCtrayectoriaA}[Fin del caso de uso]%
  {B}{El \refElem{aProfesor} o \refElem{aAdministrador} desea cancelar la actualización de la instancia después de mostrar el formulario de actualización}

  \Actor Presiona el botón {\bf cancelar} en la pantalla \refElem{IU-C05}.
  \Sistema Redirige a la pantalla \refElem{IU-C01}.

\end{UCtrayectoriaA}

\begin{UCtrayectoriaA}{C}{Algún dato ingresado por el \refElem{aProfesor} o \refElem{aAdministrador} es inválido}

  \Sistema Muestra un mensaje de error "-Usted debe poner un valor aquí", en los campos de la pantalla \refElem{IU-C05} que sean requeridos.
  \Sistema Regresa al paso \ref{CU-C05-muestra-pantalla}

\end{UCtrayectoriaA}
   % Convertir un curso ordinario a uno gamificado
    %
% \ucstEnEdicion     Al terminar una revisión/aprobación con observaciones
%                    y al inicio del CU.
%
% \ucstEnRevision    Al terminar la edición del CU (version += 0.1).
% \ucstEnAprobacion  Al pasar la revision sin observaciones.
% \ucstAprobado      Al ser aprobado por el usuario (version += 1.0)

\begin{UseCase}[%
Autor/Ricardo Naranjo,%
Version/0.1,%
Estado/\ucstEnRevision]%
%
{CU-C06}{Eliminar instancia (Competencia uno contra sistema)}{%
%
 Permite al \refElem{aProfesor} y al \refElem{aAdministrador} eliminar una instancia de la actividad competencia uno contra sistema en su curso.
 Este caso de uso es una extensión del caso de uso {\it Ver curso} que es propio de moodle.}

	\UCitem[control]{Revisor}{ Sin asignar }
	\UCitem[control]{Último cambio}{ 13/NOV/19 }

 \UCsection{Atributos}

    \UCitem{Actor(es)}{%
        \refElem{aProfesor},
        \refElem{aAdministrador}
    }

	\UCitems{Propósito}{%
        \Titem Permitir al \refElem{aProfesor} y al \refElem{aAdministrador} eliminar una instancia de la actividad de competencia uno contra sistema.
	}

	\UCitem{Entradas}{\imprimeUC{entrada}}

	\UCitems{Origen}{%
        \Titem Mouse
	}

	\UCitem{Salidas}{\imprimeUC{salida}}

	\UCitem{Destino}{%
		\refElem{IU-M08}
	}

	\UCitems{Precondiciones}{%
        \Titem El plugin de competencia uno contra sistema debe estar instalado en moodle.
        \Titem La instancia de la actividad de competencia uno contra sistema debe estar creada.
        % Realizar el caso de uso "listar actividades disponibles?"
        % \Titem Si se trata de una actualización de un plugin la versión de este debe
               % cumplir con la regla \refElem{BR-M02}.
	}

	\UCitems{Postcondiciones}{%
        \Titem La instancia de la actividad eliminada no debe mostrarse en la pantalla \refElem{IU-M08}.%

	}

	\UCitem{Reglas de negocio}{\imprimeUC{regla}}

	\UCitems{Errores}{%
	}

	% \UCitem{Viene de}{% Indicar si el Caso de uso es primario o se extiende de otro. La mayoría se
					  % extienden de Login.
		% EJEMPLO: \refIdElem{PY-CU1} o Caso de uso primario.
	% 	\TODO Especificar.
	% }

 \UCsection[design]{Datos de Diseño}

	\UCitems[design]{Casos de Prueba}{%
        \Titem \refElem{CPC-C06}
	}

 \UCsection[admin]{Datos de Administración de Requerimiento}

	\UCitem[admin]{Observaciones}{}

\end{UseCase}

\subsubsection{Trayectorias del caso de uso}

\begin{UCtrayectoria}%
%

    \Actor Activa la edición del curso en la pantalla \refElem{IU-M08}.

    \Sistema Redirige a la pantalla de edición del curso \refElem{IU-M08aa}.

    \Actor Presiona el botón {\bf Editar} de la instancia que desea eliminar.

    \Sistema Despliega el menú \refElem{IU-M08b}.

    \Actor Presiona el botón {\bf Eliminar} del menú desplegable \refElem{IU-M08b}.

    \Sistema Despliega mensaje de confirmación de eliminación. \refElem{IU-M08c}

    \Actor Presiona el botón {\bf Si}. \refTray{A}

    \Sistema Redirige a la pantalla \refElem{IU-M08} y elimina la instancia y los valores de la instancia \refElem{comp-cpu-gmcompcpu}, así como los datos que dependen de la instancia en las siguientes entidades: \refElem{comp-cpu-gmdl-intento} y \refElem{comp-cpu-gmdl-respuesta-cpu}.

\end{UCtrayectoria}

\begin{UCtrayectoriaA}[Fin del caso de uso]{A}{El \refElem{aProfesor} o \refElem{aAdministrador} desea cancelar la eliminación después de mostrar el mensaje de confirmación}

  \Actor Presiona el botón {\bf No} en la mensaje de confirmación \refElem{IU-M08c}.
  \Sistema Redirige a la pantalla \refElem{IU-M08}.

\end{UCtrayectoriaA}
   % Convertir Curso gamificado -> curso ordinario
    %
% \ucstEnEdicion     Al terminar una revisión/aprobación con observaciones
%                    y al inicio del CU.
%
% \ucstEnRevision    Al terminar la edición del CU (version += 0.1).
% \ucstEnAprobacion  Al pasar la revision sin observaciones.
% \ucstAprobado      Al ser aprobado por el usuario (version += 1.0)

%\addfigure[(adaptado de {\it For The Win} \cite{ForTheWin})]%
%    {.4}{investigacion/images/forthewin}{fig:ForTheWin}%
%    {Jerarquía de elementos de juego segun For The Win}

\begin{UseCase}[%
Autor/Ricardo Naranjo,%
Version/0.1,%
Estado/\ucstEnRevision]%
%
{CU-C10}{Desafiar a un estudiante apostando (Competencia uno contra uno)}{%
%
 Permite al \refElem{aEstudiante}, \refElem{aProfesor} y al \refElem{aAdministrador} desafiar a un estudiante que se encuentra inscrito en el curso y a su vez se otorga la posibilidad de apostar en la partida.
 Este caso de uso es una extensión del caso de uso \refElem{CU-C07}.}

	\UCitem[control]{Revisor}{ Sin asignar }
	\UCitem[control]{Último cambio}{ 13/NOV/19 }

 \UCsection{Atributos}

    \UCitem{Actor(es)}{%
        \refElem{aEstudiante},
        \refElem{aProfesor},
        \refElem{aAdministrador}
    }

	\UCitems{Propósito}{%
        \Titem Permitir al \refElem{aEstudiante}, \refElem{aProfesor} y al \refElem{aAdministrador} desafiar a un estudiante que forma parte de un curso y apostar la cantidad de monedas decidida por el usuario.
        \Titem Permitir al \refElem{aEstudiante}, \refElem{aProfesor} y al \refElem{aAdministrador} terminar un desafío.
	}

	\UCitem{Entradas}{\imprimeUC{entrada}}

	\UCitems{Origen}{%
        \Titem Mouse
        \Titem Teclado
	}

	\UCitem{Salidas}{\imprimeUC{salida}}

	\UCitem{Destino}{%
		\refElem{IU-C10}
	}

	\UCitems{Precondiciones}{%
        \Titem El plugin de competencia uno contra uno debe estar instalado en moodle.
        \Titem La instancia de la actividad de competencia uno contra uno debe estar creada.
        % Realizar el caso de uso "listar actividades disponibles?"
        % \Titem Si se trata de una actualización de un plugin la versión de este debe
               % cumplir con la regla \refElem{BR-M02}.
	}

	\UCitems{Postcondiciones}{%
        \Titem Se muestra en la pantalla de historial del usuario \refElem{IU-C07} la partida que inició.%

	}

	\UCitem{Reglas de negocio}{\imprimeUC{regla}}

	\UCitems{Errores}{%
	}

	% \UCitem{Viene de}{% Indicar si el Caso de uso es primario o se extiende de otro. La mayoría se
					  % extienden de Login.
		% EJEMPLO: \refIdElem{PY-CU1} o Caso de uso primario.
	% 	\TODO Especificar.
	% }

 \UCsection[design]{Datos de Diseño}

	\UCitems[design]{Casos de Prueba}{%
        \Titem \refElem{CPC-C10}
	}

 \UCsection[admin]{Datos de Administración de Requerimiento}

	\UCitem[admin]{Observaciones}{}

\end{UseCase}

\subsubsection{Trayectorias del caso de uso}

\begin{UCtrayectoria}%
%

    \Actor Ingresa la cantidad de monedas que desea apostar en el campo "Monedas a apostar" en el bloque del estudiante que quiere desafiar en la pantalla \refElem{IU-C02}.

    \Actor Presiona el botón {\bf Desafiar}.

    \Sistema Inicia la partida y establece los valores correspondientes \refElem{comp-1v1-gmdl-partida}( \entrada{comp-1v1-gmdl-partida.gmdl-comp-vs-id} ), una entrada por cada usuario de la partida \refElem{comp-1v1-gmdl-participacion} ( \entrada{comp-1v1-gmdl-participacion.gmdl-usuario-id}, \entrada{comp-1v1-gmdl-participacion.gmdl-partida-id}, \entrada{comp-1v1-gmdl-participacion.fecha-inicio}, \refElem{comp-1v1-gmdl-participacion.puntuacion} ) y agrega la apuesta \refElem{comp-1v1-gmdl-apuesta}( \entrada{comp-1v1-gmdl-apuesta.gmdl-participacion-id}, \entrada{comp-1v1-gmdl-apuesta.monedas-plata}, \entrada{comp-1v1-gmdl-apuesta.activa} ).

    \Sistema Redirige a la pantalla del cuestionario del desafío \refElem{IU-C09}.

    \Actor Contesta las preguntas mostradas. \refTray{A}
    \label{CU-C10-contesta-cuestionario}
    \Actor Presiona el botón {\bf Terminar}.

    \Sistema Evalúa las respuestas ingresadas y calcula su puntuación.

    \Sistema Establece los valores correspondientes de la participación \refElem{comp-1v1-gmdl-participacion} (
    \entrada{comp-1v1-gmdl-participacion.fecha-fin},
    \entrada{comp-1v1-gmdl-participacion.puntuacion}).

    \Sistema Valida que el estudiante desafiado no haya terminado de contestar su cuestionario. \refTray{B}

    \Sistema Redirige a la pantalla \refElem{IU-C10}.

\end{UCtrayectoria}

\begin{UCtrayectoriaA}[Fin del caso de uso]%
  {A}{El actor desea salir del cuestionario sin terminarlo}

  \Actor Presiona cualquier botón, excepto el botón de {\bf Terminar}, que haga que abandone la pantalla.
  \Sistema Muestra mensaje de alerta \refElem{IU-C13}.
  \Actor Presiona el botón {\bf Abandonar}. \refTray{D}
  \Sistema Establece el valor correspondiente de la participación \refElem{comp-1v1-gmdl-participacion} (
  \refElem{comp-1v1-gmdl-participacion.fecha-fin}).
  \Sistema Redirige a la pantalla elegida por el actor.

\end{UCtrayectoriaA}

\begin{UCtrayectoriaA}[Fin del caso de uso]%
  {B}{El \refElem{aEstudiante} desafiado ya había terminado de contestar el cuestionario}

  \Sistema Calcula y otorga puntaje extra de acuerdo al tiempo tardado en contestar el cuestionario.
  \Sistema Comprueba que el actor haya obtenido un mayor puntaje que el \refElem{aEstudiante} desafiado. \refTray{C}
  \Sistema Lanza el evento para dar las monedas que le corresponden al actor.
  \Sistema Lanza el evento para dar la experiencia que le corresponde al actor.
  \Sistema Redirige a la pantalla \refElem{IU-C11}.
  \Sistema Muestra el puntaje de cada participante \salida{comp-1v1-gmdl-participacion.puntuacion}.

\end{UCtrayectoriaA}

\begin{UCtrayectoriaA}[Fin del caso de uso]%
  {C}{El \refElem{aEstudiante} desafiado obtuvo un mayor puntaje que el actor}

  \Sistema Lanza el evento para dar las monedas que le corresponden al \refElem{aEstudiante} desafiado.
  \Sistema Lanza el evento para dar la experiencia que le corresponde al \refElem{aEstudiante} desafiado.
  \Sistema Redirige a la pantalla \refElem{IU-C12}.
  \Sistema Muestra el puntaje de cada participante \refElem{comp-1v1-gmdl-participacion.puntuacion}.

\end{UCtrayectoriaA}

\begin{UCtrayectoriaA}{D}{El actor no quiere abandonar el cuestionario}

  \Actor Presiona el botón {\bf Cancelar}.
  \Sistema Cierra el mensaje de alerta.
  \Sistema Regresa al paso \ref{CU-C10-contesta-cuestionario}

\end{UCtrayectoriaA}
   % Climinar un curso gamificado
    %
% \ucstEnEdicion     Al terminar una revisión/aprobación con observaciones
%                    y al inicio del CU.
%
% \ucstEnRevision    Al terminar la edición del CU (version += 0.1).
% \ucstEnAprobacion  Al pasar la revision sin observaciones.
% \ucstAprobado      Al ser aprobado por el usuario (version += 1.0)

%\addfigure[(adaptado de {\it For The Win} \cite{ForTheWin})]%
%    {.4}{investigacion/images/forthewin}{fig:ForTheWin}%
%    {Jerarquía de elementos de juego segun For The Win}

\begin{UseCase}[%
Autor/Ricardo Naranjo,%
Version/0.1,%
Estado/\ucstEnRevision]%
%
{CU-C12}{Ver estado de instancia de actividad (Competencia uno contra sistema)}{%
%
 Permite al \refElem{aEstudiante}, \refElem{aProfesor} y al \refElem{aAdministrador} ver el estado actual de una instancia de la actividad competencia uno contra sistema en el curso.
 Este caso de uso es una extensión del caso de uso {\it Ver curso} que es propio de moodle.}

	\UCitem[control]{Revisor}{ Sin asignar }
	\UCitem[control]{Último cambio}{ 13/NOV/19 }

 \UCsection{Atributos}

    \UCitem{Actor(es)}{%
        \refElem{aEstudiante},
        \refElem{aProfesor},
        \refElem{aAdministrador}
    }

	\UCitems{Propósito}{%
        \Titem Permitir al \refElem{aEstudiante}, \refElem{aProfesor} y al \refElem{aAdministrador} ver el estado actual de una instancia de la actividad de competencia uno contra sistema.
	}

	\UCitem{Entradas}{\imprimeUC{entrada}}

	\UCitems{Origen}{%
        \Titem Mouse
	}

	\UCitem{Salidas}{\imprimeUC{salida} \begin{itemize}
    \item {\bf Icono de dificultad vencida}\IUCpuvencida
    \item {\bf Icono de dificultad no vencida}\IUCpunovencida
  \end{itemize} }

	\UCitem{Destino}{%
		\refElem{IU-C04}
	}

	\UCitems{Precondiciones}{%
        \Titem El plugin de competencia uno contra sistema debe estar instalado en moodle.
        \Titem La instancia de la actividad de competencia uno contra sistema debe estar creada.
        % Realizar el caso de uso "listar actividades disponibles?"
        % \Titem Si se trata de una actualización de un plugin la versión de este debe
               % cumplir con la regla \refElem{BR-M02}.
	}

	\UCitems{Postcondiciones}{%
        \Titem La pantalla principal de la instancia de la actividad de competencia uno contra sistema \refElem{IU-C04} debe mostrar los datos pertinentes al usuario que realizó el caso de uso.%

	}

	\UCitem{Reglas de negocio}{\imprimeUC{regla}}

	\UCitems{Errores}{%
	}

	% \UCitem{Viene de}{% Indicar si el Caso de uso es primario o se extiende de otro. La mayoría se
					  % extienden de Login.
		% EJEMPLO: \refIdElem{PY-CU1} o Caso de uso primario.
	% 	\TODO Especificar.
	% }

 \UCsection[design]{Datos de Diseño}

	\UCitems[design]{Casos de Prueba}{%
        \Titem \refElem{CPC-C12}
	}

 \UCsection[admin]{Datos de Administración de Requerimiento}

	\UCitem[admin]{Observaciones}{}

\end{UseCase}

\subsubsection{Trayectorias del caso de uso}

\begin{UCtrayectoria}%
%

    \Actor Presiona el nombre de la instancia a la que quiere acceder en la pantalla \refElem{IU-M07}.

    \Sistema Redirige a la pantalla principal de la instancia \refElem{IU-C04}.

    \Sistema Muestra las dificultades del sistema vencido por medio del icono \IUCpuvencida.

    \Sistema Muestra las dificultades del sistema que no han sido vencidas por medio del icono \IUCpunovencida.

    \Sistema Muestra en la sección {\bf Desafiar computadora} las dificultades que se pueden desafiar \salida{comp-cpu-gmdl-dificultad-cpu}

\end{UCtrayectoria}

\subsubsection{Puntos de extensión}

\UCExtensionPoint{Ver historial de sus partidas}{%

    El \refElem{aAdministrador}, \refElem{aProfesor} o \refElem{aEstudiante} desea ver su historial de las partidas de competencia uno contra sistema.
%
    }{Al final de la trayectoria principal del caso de uso.
%
    }{\refElem{CU-C15}}


\UCExtensionPoint{Ver tabla de puntuaciones}{%

    El \refElem{aAdministrador}, \refElem{aProfesor} o \refElem{aEstudiante} desea ver la tabla de puntuaciones de una instancia de competencia uno contra sistema.
%
    }{Al final de la trayectoria principal del caso de uso.
%
    }{\refElem{CU-C14}}

  \UCExtensionPoint{Desafiar al sistema}{%

      El \refElem{aAdministrador}, \refElem{aProfesor} o \refElem{aEstudiante} desea desafiar al sistema en alguna de sus dificultades.
  %
      }{Al final de la trayectoria principal del caso de uso.
  %
      }{\refElem{CU-C13}}
   % Crear cuenta de un usuario gamificado
    %
% \ucstEnEdicion     Al terminar una revisión/aprobación con observaciones
%                    y al inicio del CU.
%
% \ucstEnRevision    Al terminar la edición del CU (version += 0.1).
% \ucstEnAprobacion  Al pasar la revision sin observaciones.
% \ucstAprobado      Al ser aprobado por el usuario (version += 1.0)

%\addfigure[(adaptado de {\it For The Win} \cite{ForTheWin})]%
%    {.4}{investigacion/images/forthewin}{fig:ForTheWin}%
%    {Jerarquía de elementos de juego segun For The Win}

\begin{UseCase}[%
Autor/Ricardo Naranjo,%
Version/0.1,%
Estado/\ucstEnRevision]%
%
{CU-C13}{Desafiar al sistema (Competencia uno contra sistema)}{%
%
 Permite al \refElem{aEstudiante}, \refElem{aProfesor} y al \refElem{aAdministrador} desafiar al sistema.
 Este caso de uso es una extensión del caso de uso \refElem{CU-C12}.}

	\UCitem[control]{Revisor}{ Sin asignar }
	\UCitem[control]{Último cambio}{ 13/NOV/19 }

 \UCsection{Atributos}

    \UCitem{Actor(es)}{%
        \refElem{aEstudiante},
        \refElem{aProfesor},
        \refElem{aAdministrador}
    }

	\UCitems{Propósito}{%
        \Titem Permitir al \refElem{aEstudiante}, \refElem{aProfesor} y al \refElem{aAdministrador} desafiar al sistema.
        \Titem Permitir al \refElem{aEstudiante}, \refElem{aProfesor} y al \refElem{aAdministrador} terminar un desafío.
	}

	\UCitem{Entradas}{\imprimeUC{entrada}}

	\UCitems{Origen}{%
        \Titem Mouse
        \Titem Teclado
	}

	\UCitem{Salidas}{\imprimeUC{salida}}

	\UCitem{Destino}{%
		\refElem{IU-C10}
	}

	\UCitems{Precondiciones}{%
        \Titem El plugin de competencia uno contra sistema debe estar instalado en moodle.
        \Titem La instancia de la actividad de competencia uno contra sistema debe estar creada.
        % Realizar el caso de uso "listar actividades disponibles?"
        % \Titem Si se trata de una actualización de un plugin la versión de este debe
               % cumplir con la regla \refElem{BR-M02}.
	}

	\UCitems{Postcondiciones}{%
        \Titem Se muestra en la pantalla de historial del usuario \refElem{IU-C14} el desafío nuevo.%
	}

	\UCitem{Reglas de negocio}{\imprimeUC{regla}}

	\UCitems{Errores}{%
	}

	% \UCitem{Viene de}{% Indicar si el Caso de uso es primario o se extiende de otro. La mayoría se
					  % extienden de Login.
		% EJEMPLO: \refIdElem{PY-CU1} o Caso de uso primario.
	% 	\TODO Especificar.
	% }

 \UCsection[design]{Datos de Diseño}

	\UCitems[design]{Casos de Prueba}{%
        \Titem \refElem{CPC-C13}
	}

 \UCsection[admin]{Datos de Administración de Requerimiento}

	\UCitem[admin]{Observaciones}{}

\end{UseCase}

\subsubsection{Trayectorias del caso de uso}

\begin{UCtrayectoria}%
%

    \Actor Selecciona la dificultad deseada en el campo {\bf Seleccione una dificultad} de la pantalla \refElem{IU-C04}.

    \Actor Presiona el botón {\bf Empezar}.

    \Sistema Inicia la partida y establece los valores correspondientes \refElem{comp-cpu-gmdl-intento}( \entrada{comp-cpu-gmdl-intento.gmdl-dificultad-cpu-id}, \entrada{comp-cpu-gmdl-intento.gmdl-comp-cpu-id}, \entrada{comp-cpu-gmdl-intento.gmdl-usuario-id}, \entrada{comp-cpu-gmdl-intento.fecha-inicio} ).

    \Sistema Redirige a la pantalla del cuestionario del desafío \refElem{IU-C09}.

    \Actor Contesta las preguntas mostradas.

    \Actor Presiona el botón {\bf Terminar}.

    \Sistema Evalúa las respuestas ingresadas y calcula su puntuación.

    \Sistema Contesta el cuestionario y calcula su puntuación.

    \Sistema Establece los valores correspondientes del intento \refElem{comp-cpu-gmdl-intento} (
    \entrada{comp-cpu-gmdl-intento.fecha-fin},
    \entrada{comp-cpu-gmdl-intento.puntuacion-cpu},
    \entrada{comp-cpu-gmdl-intento.puntuacion-usuario}).

    \Sistema Valida que el actor haya sacado más o igual puntuación que el sistema. \refTray{A}

    \Sistema Lanza el evento para dar las monedas que le corresponden al actor.
    \Sistema Lanza el evento para dar la experiencia que le corresponde al actor.

    \Sistema Redirige a la pantalla \refElem{IU-C11}.
    \Sistema Muestra el puntaje del actor y del sistema \salida{comp-cpu-gmdl-intento.puntuacion-usuario}, \salida{comp-cpu-gmdl-intento.puntuacion-cpu}.

\end{UCtrayectoria}

\begin{UCtrayectoriaA}[Fin del caso de uso]%
  {A}{El sistema obtuvo un mayor puntaje que el actor}

  \Sistema Redirige a la pantalla \refElem{IU-C12}.
  \Sistema Muestra el puntaje del actor y del sistema \refElem{comp-cpu-gmdl-intento.puntuacion-usuario}, \refElem{comp-cpu-gmdl-intento.puntuacion-cpu}.

\end{UCtrayectoriaA}
   % Eliminar cuenta de un usuario gamificado

    %
% \ucstEnEdicion     Al terminar una revisión/aprobación con observaciones 
%                    y al inicio del CU.
%
% \ucstEnRevision    Al terminar la edición del CU (version += 0.1).
% \ucstEnAprobacion  Al pasar la revision sin observaciones.
% \ucstAprobado      Al ser aprobado por el usuario (version += 1.0)

\begin{UseCase}[%
Autor/Daniel Ortega,%
Version/0.1,%
Estado/\ucstEnEdicion]%
%
{CU-E02-1}{Habilitar/Deshabilitar el módulo de experiencia}{%
%
 Permite al \refElem{aActor} .}

	\UCitem[control]{Revisor}{ Sin asignar }
	\UCitem[control]{Último cambio}{ \today }

 \UCsection{Atributos}

    \UCitem{Actor(es)}{%
        \refElem{aActor}
    }

	\UCitem{Propósito}{%
        ...
	}
	
	\UCitem{Entradas}{\imprimeUC{entrada}}

	\UCitems{Origen}{%
        \Titem Mouse
        \Titem Teclado
	}

	\UCitem{Salidas}{\imprimeUC{salida}}

	\UCitems{Destino}{%
		\Titem \refElem{IU-M02a}
	}
	
	\UCitems{Precondiciones}{%
        \Titem ...
	}

	\UCitem{Postcondiciones}{%
        Ninguna
	}

	\UCitem{Reglas de negocio}{%
		Ninguna
	}

	\UCitems{Errores}{%
        \Titem \UCerr{Err1}{%
        % CAUSA
            ...,}{%
        % EFECTO
            ...}
	}

	% \UCitem{Viene de}{% Indicar si el Caso de uso es primario o se extiende de otro. La mayoría se 
					  % extienden de Login.
		% EJEMPLO: \refIdElem{PY-CU1} o Caso de uso primario.
	% 	\TODO Especificar.
	% }	

 \UCsection[design]{Datos de Diseño}

	\UCitems[design]{Casos de Prueba}{%
        \Titem \refElem{CPC-E0Y}
        \Titem \refElem{CPI-E0Y}
	}

 \UCsection[admin]{Datos de Administración de Requerimiento}

	\UCitem[admin]{Observaciones}{%
        Ninguna
	}

\end{UseCase}

\clearpage
\subsubsection{Trayectorias del caso de uso}

\begin{UCtrayectoria}%
%
 \Actor Presiona el botón \IUMenu en la esquina superior izquierda de la pantalla \refElem{IU-M01}
        para abrir el menu de navegación.

 \Actor Selecciona la opción {\it \IUAdminSitio Administración del sitio}

 \Sistema Carga la pantalla \refElem{IU-M02}

\end{UCtrayectoria}


\subsubsection{Puntos de extensión}

\UCExtensionPoint{Nombre del punto de extensión}{%

    El \refElem{aAdministrador} desea/requiere/necesita ....%
%
    }{En el paso \ref{CU-ET-1x} de la trayectoria principal  ...%
%
    }{\refElem{CU-E2-T}}

 % Configuraciones generales
    %
% \ucstEnEdicion     Al terminar una revisión/aprobación con observaciones
%                    y al inicio del CU.
%
% \ucstEnRevision    Al terminar la edición del CU (version += 0.1).
% \ucstEnAprobacion  Al pasar la revision sin observaciones.
% \ucstAprobado      Al ser aprobado por el usuario (version += 1.0)

\begin{UseCase}[%
Autor/Daniel Ortega,%
Version/0.1,%
Estado/\ucstEnEdicion]%
%
{CU-E02-2}{Configurar sistema de experiencia}{%
%
 Permite al \refElem{aAdministrador} establecer y modificar las cantidades de puntos
 de experiencia que brindan los cursos en la plataforma y la forma en que aumenta
 la cantidad de experiencia requerida para pasar de un nivel al siguiente. Los cambios
 sobre estas configuraciones pueden afectar el nivel en que se encuentran los usuarios
 y cambiar la cantidad de experiencia que los cursos brindan.}

	\UCitem[control]{Revisor}{ Sin asignar }
	\UCitem[control]{Último cambio}{ \today }

 \UCsection{Atributos}

    \UCitem{Actor(es)}{%
        \refElem{aAdministrador}
    }

	\UCitems{Propósito}{%
        \Titem Permitir al administrador configurar el sistema de experiencia.
        \Titem Establecer o modificar la cantidad de experiencia que brindan los
               cursos.
        \Titem Establecer la cantidad de experiencia requerida para pasar el primer
               nivel usada como calculo para los demás niveles.
        \Titem Cambiar la forma en cómo se incrementa la cantidad de experiencia
               requerida para avanzar de un nivel a otro.
	}

	\UCitem{Entradas}{\imprimeUC{entrada}}

	\UCitems{Origen}{%
        \Titem Mouse
        \Titem Teclado
        \Titem Sistema (para los datos de los \refElem[usuarios gamificados]{xp-user})
	}

	\UCitem{Salidas}{\imprimeUC{salida}}

	\UCitems{Destino}{%
		\Titem \refElem{IU-E04}
	}

	\UCitems{Precondiciones}{%
        \Titem Que los plugins del módulo de experiencia se encuentren instalados
        \Titem El módulo de experiencia debe estár habilitado en el caso de uso
               \refElem{CU-E02}.
	}

	\UCitem{Postcondiciones}{%
        Los nuevos valores de las \refElem{xp-scheme-settings} deber ser
        estár actualizados para todos los usuarios, además de persistirse en el
        sistema.
	}

	\UCitem{Reglas de negocio}{\imprimeUC{regla}}

	\UCitems{Errores}{%
        \Titem \UCerr{Err1}{%
        % CAUSA
            Los plugins del módulo de experiencia no se encuentran instalados,}{%
        % EFECTO
            no se presentan las opciones en el menú y por lo tanto no se puede
            acceder a las configuraciones}
	}

 \UCsection[design]{Datos de Diseño}

	\UCitems[design]{Casos de Prueba}{%
        \Titem \refElem{CPC-E02-2a}
        \Titem \refElem{CPC-E02-2b}
        \Titem \refElem{CPC-E02-2c}
        \Titem \refElem{CPI-E02-2}
	}

 \UCsection[admin]{Datos de Administración de Requerimiento}

	\UCitem[admin]{Observaciones}{%
        Los cambios en las configuraciones del sistema de experiencia podrían hacer
        que los usuarios aumenten de nivel si la cantidad de experiencia acumulada del
        nivel es mayor a la nueva cantidad de experiencia correspondiente a dicho nivel.
        Se recomienda que los cambios se realicen cuando la cantidad estudiantes que usen el
        sistema sea mínima.}

\end{UseCase}

\subsubsection{Trayectorias del caso de uso}

\begin{UCtrayectoria}%
%
  \includeUC{CU-M01} \refErr{Err1}

  \Actor Presiona la opción {\bf \refElem{tExpSettingsComportamiento}} en la categoría
         \refElem{tExpCategoria}. \refTray{A}
  \Sistema Obtiene el valor de si el módulo de experiencia está \refElem[activado]%
           {xp-general-settings.activated} o no. \refTray{B} \label{CU-E02-2-loading}
  \Sistema Obtiene los valores actuales de la configuración del sistema de experiencia:
           \salida{xp-scheme-settings.increment},
           \salida{xp-scheme-settings.incrementValue},
           \salida{xp-scheme-settings.levelXP} y
           \salida{xp-scheme-settings.courseXP}.
  \Sistema Carga la pantalla \refElem{IU-E04} estableciendo como valores por defecto
           las \refElem{xp-scheme-settings} obtenidas en el anterior paso.
  %\sistema muestra un mensaje informando al \refelem{aadministrador} de que si
  %         modifica la experiencia el nivel 1 o el tipo de incremento se alterarían
  %         la cantidad de experiencia requerida para subir de nivel.

  \Actor Especifica si el \entrada{xp-scheme-settings.increment} en la cantidad de
         experiencia de los niveles será {\it Lineal} o {\it Porcentual}.
  \Actor Ingresa el valor para el \entrada{xp-scheme-settings.incrementValue} con
         base en la regla \regla{BR-E03}.
  \Actor Ingresa los valores para la \entrada{xp-scheme-settings.levelXP} y la
         \entrada{xp-scheme-settings.courseXP}.
  \Actor Presiona la opción {\bf Guardar Cambios}. \refTray{C} \label{CU-E02-2-submit}

  \Sistema Valida que los valores ingresados por el usuario cumplan con las
           restricciones especificadas en el modelo de información.
  \Sistema Verifica que el \refElem{xp-scheme-settings.incrementValue} cumpla
           con la regla \refElem{BR-E03}. \refTray{D}
  \Sistema Actualiza los valores de las \refElem{xp-scheme-settings} con los
           ingresados por el usuario.

  % ACTUALIZACION DE NIVEL DE LOS USUARIOS
  \Sistema Obtiene la lista de los \refElem[usuarios gamificados]{xp-user} y por cada uno
           realiza las siguientes acciones.
  \Sistema Obtiene el \entrada{xp-user.level} y la cantidad de \entrada{xp-user.levelxp}.
  \Sistema Calcula la nueva cantidad de experiencia correspondiente al nivel del usuario con
           base en la regla \regla{BR-E04} o \regla{BR-E05} si el incremento es lineal o
           porcentual respectivamente.
  \Sistema - Si la cantidad de experiencia del usuario es menor a la experiencia correspondiente
           al nivel, entonces se procede al siguiente usuario. \refTray{E} \label{CU-E02-2-Usuarios}

  \Sistema Despliega la pantalla \refElem{IU-E04} con el mensaje de que los datos
           han sido actualizados exitosamente.

\end{UCtrayectoria}

\begin{UCtrayectoriaA}{A}{
El \refElem{aAdministrador} selecciona la categoría \refElem{tExpCategoria}}
  \Sistema Carga la pantalla \refElem{IU-E01}
  \Actor Regresa al paso \ref{CU-E02-2-loading}
\end{UCtrayectoriaA}

\begin{UCtrayectoriaA}{B}{
El módulo de experiencia no se encuentra activado}
  \Sistema Carga la pantalla \refElem{IU-E03a}.
  \Actor Presiona el botón {\bf Activar módulo de experiencia}
  \includeUC{CU-E02} a partir del paso \ref{CU-E02-ir-a-formulario},
                     para activar el módulo de experiencia.

  \Sistema Regresa al inicio de la trayectoria principal.

\end{UCtrayectoriaA}

\begin{UCtrayectoriaA}{C}{
El \refElem{aAdministrador} desea cancelar la modificación en el sistema de
experiencia}

  \Actor Presiona el botón {\bf Cancelar}.
  \Sistema Redirige a la pantalla \refElem{IU-M01}.
\end{UCtrayectoriaA}

\begin{UCtrayectoriaA}{D}{
Alguno de los valores ingresados por el usuario son incorrectos.}
  \Sistema Imprime los mensajes de error abajo de los campos con los valores
           incorrectos.
  \Actor Ingresa nuevamente los valores en los campos marcados como incorrectos.
  \Sistema Regresa al paso \ref{CU-E02-2-submit}.

\end{UCtrayectoriaA}

\begin{UCtrayectoriaA}{E}{%
La cantidad de \refElem{xp-user.levelxp} es mayor a la experiencia del nivel en el
que se encuentra}

  \Sistema Avanza al \refElem{xp-user} al siguiente nivel, usando los
           puntos de experiencia del mismo y establece el sobrante como
           la \refElem{xp-user.levelxp} correspondiente al nuevo nivel.
  \Sistema Repite el paso anterior hasta que la cantidad de experiencia
           del nivel del usuario sea menor que la del nivel.
  \Sistema Regresa al paso \ref{CU-E02-2-Usuarios}

\end{UCtrayectoriaA}
 % Visualización de niveles
    %
% \ucstEnEdicion     Al terminar una revisión/aprobación con observaciones
%                    y al inicio del CU.
%
% \ucstEnRevision    Al terminar la edición del CU (version += 0.1).
% \ucstEnAprobacion  Al pasar la revision sin observaciones.
% \ucstAprobado      Al ser aprobado por el usuario (version += 1.0)

\begin{UseCase}[%
Autor/Daniel Ortega,%
Version/0.1,%
Estado/\ucstEnEdicion]%
%
{CU-E02-3}{Configurar los eventos que entregan experiencia}{%
%
 Permite al \refElem{aAdministrador} elegir cuales eventos de los soportados
 brindarán experiencia y cuales no. Además de aquellos eventos que brindan
 experiencia puede especificar la cantidad de experiencia que estos entregan.}

	\UCitem[control]{Revisor}{ Sin asignar }
	\UCitem[control]{Último cambio}{ \today }

 \UCsection{Atributos}

    \UCitem{Actor(es)}{%
        \refElem{aAdministrador}
    }

	\UCitems{Propósito}{%
        \Titem Configurar que eventos proporcionan experiencia.
        \Titem Establecer la cantidad de experiencia de los eventos
               otorgarán.
        \Titem Deshabilitar eventos para que dejen de brindar
               experiencia.
	}

	\UCitem{Entradas}{\imprimeUC{entrada}}

	\UCitems{Origen}{%
        \Titem Mouse
        \Titem Teclado
	}

	\UCitem{Salidas}{\imprimeUC{salida}}

	\UCitems{Destino}{%
		\Titem \refElem{IU-E05}
	}

	\UCitems{Precondiciones}{%
        \Titem Los plugins correspondientes al módulo de experiencia deben
               de estar previamente instalados.
        \Titem El módulo de experiencia debe estar habilitado mediante el caso
               de uso \refElem{CU-E02}.
        \Titem Los eventos con experiencia deben estar habilitados mediante el caso
               de uso \refElem{CU-E02}.
	}

	\UCitem{Postcondiciones}{%
        Los nuevos valores para los eventos habilitados para dar experiencia deben
        de ser actualizados y persistirse en el sistema.
	}

	\UCitem{Reglas de negocio}{Ninguna}

	\UCitems{Errores}{%
        \Titem \UCerr{Err1}{%
        % CAUSA
            Los plugins del módulo de experiencia no se encuentran instalados,}{%
        % EFECTO
            no se presentan las opciones en el menú y por lo tanto no se acceder
            a las configuraciones}
	}

	% \UCitem{Viene de}{% Indicar si el Caso de uso es primario o se extiende de otro. La mayoría se
					  % extienden de Login.
		% EJEMPLO: \refIdElem{PY-CU1} o Caso de uso primario.
	% 	\TODO Especificar.
	% }

 \UCsection[design]{Datos de Diseño}

	\UCitems[design]{Casos de Prueba}{%
        \Titem \refElem{CPC-E02-3}
        \Titem \refElem{CPI-E02-3}
	}

 \UCsection[admin]{Datos de Administración de Requerimiento}

	\UCitem[admin]{Observaciones}{%
        Ninguna
	}

\end{UseCase}

\subsubsection{Trayectorias del caso de uso}

\begin{UCtrayectoria}%
%
  \includeUC{CU-M01} \refErr{Err1}

  \Actor Presiona la opción {\bf\refElem{tExpSettingsEventos}} en la categoría
         \refElem{tExpCategoria}. \refTray{A}

  \Sistema Obtiene el valor de si el módulo de experiencia está \refElem[activado]%
           {xp-general-settings.activated} o no. \refTray{B} \label{CU-E02-3-activated}

  \Sistema Obtiene el valor de si la funcionalidad de que los \refElem[eventos]%
           {xp-general-settings.events} otorgen experiencia está activada o no.
           \refTray{C} \label{CU-E02-3-events}

  \Sistema Obtiene los valores actuales de la configuración de los eventos con experiencia:
           \salida{xp-events-settings.competencecpuevent},
           \salida{xp-events-settings.competencecpuxp},
           \salida{xp-events-settings.competencevsevent},
           \salida{xp-events-settings.competencevsxp},
           \salida{xp-events-settings.preguntadiariaevento} y
           \salida{xp-events-settings.preguntadiariaxp}.

  \Sistema Carga la pantalla \refElem{IU-E05} estableciendo como valores por defecto
           las \refElem{xp-events-settings} obtenidas en el anterior paso.

  \Actor Habilita los eventos que desea que otorguen experiencia:
           \entrada{xp-events-settings.competencecpuevent},
           \entrada{xp-events-settings.competencevsevent} y
           \entrada{xp-events-settings.preguntadiariaevento}.

  \Sistema Habilita los campos para especificar la experiencia correspondientes a aquellos
           eventos que el usuario haya habilitado.

  \Actor Especifica los valores para la experiencia correspondientes a los eventos que haya
         habilitado, en los campos:
           \entrada{xp-events-settings.competencecpuxp},
           \entrada{xp-events-settings.competencevsxp} y
           \entrada{xp-events-settings.preguntadiariaxp}.

  \Actor Presiona el botón {\bf Guardar Cambios} \refTray{D} \label{CU-E02-3-submit}

  \Sistema Valida que los datos ingresados por el usuario cumplan con las restricciones
           de acuerdo con el modelo de información. \refTray{E}
  \Sistema Actualiza los datos de la configuración de eventos en el sistema
  \Sistema Muestra la pantalla \refElem{IU-E05} con el mensaje de que los datos han sido
           actualizados correctamente.

\end{UCtrayectoria}


\begin{UCtrayectoriaA}{A}{%
El \refElem{aAdministrador} selecciona la categoría \refElem{tExpCategoria}.
}
  \Sistema Carga a pantalla \refElem{IU-E01}.
  \Actor Regresa a paso \ref{CU-E02-3-activated}
\end{UCtrayectoriaA}

\begin{UCtrayectoriaA}{B}{%
El módulo de experiencia no se encuentra activado.
}
  \Sistema Carga la pantalla \refElem{IU-E03a}.
  \Actor Presiona el botón {\bf Activar módulo de experiencia}
  \includeUC{CU-E02} a partir del paso \ref{CU-E02-ir-a-formulario},
                     para activar el módulo de experiencia.

  \Sistema Regresa al inicio de la trayectoria principal.
\end{UCtrayectoriaA}

\begin{UCtrayectoriaA}{C}{%
La opción de que los eventos brinden experiencia no se encuentra activada.
}
  \Sistema Carga la pantalla \refElem{IU-E05a}.
  \Actor Presiona el botón {\bf Habilitar Eventos}
  \includeUC{CU-E02} a partir del paso \ref{CU-E02-ir-a-formulario},
                     para activar el módulo de experiencia.

  \Sistema Regresa al inicio de la trayectoria principal.
\end{UCtrayectoriaA}

\begin{UCtrayectoriaA}{D}{%
El \refElem{aAdministrador} desea cancelar la modificación de la configuración
de los eventos.
}
  \Actor Presiona el botón {\bf Cancelar}
  \Sistema Redirige a la pantalla \refElem{IU-M01}
\end{UCtrayectoriaA}

\begin{UCtrayectoriaA}{E}{%
Alguno de los campos ingresados por el \refElem{aAdministrador} son incorrectos.
}
  \Sistema Imprime los mensajes de error debajo de los con los valores incorrectos
  \Actor Ingresa nuevamente los valores en los campos marcados como incorrectos.
  \Sistema Regresa al paso \ref{CU-E02-3-submit}
\end{UCtrayectoriaA}
 % Esquema de experiencia
    %
\begin{UseCase}{CU-E2}{Crear curso con experiencia}{%
El profesor desea crear un nuevo curso en moodle con soporte para brindar puntos de experiencia a los alumnos, partiendo de la interfaz \IUref{moodle:nuevoCurso} llena los campos del curso, escoge el formato {\it Gamedle} y habilita la opción de experiencia, finalmente presiona el botón para crear el curso.}

	\UCrow{Versión}{\color{gray} 0.1 (Edición)}
    \UCrow{Autor}{\color{gray}	Daniel Ortega}
    \UCrow{Supervisa}{\color{gray}}
    \UCrow{Actor}{Profesor}
    \UCrow{Propósito}{Que el profesor pueda crear un curso que tenga soporte para brindar puntos de experiencia en las distintas secciones del curso.}
    \UCrow{Entradas}{
		\begin{Titemize}
		    \Titem{ Nombre completo y nombre corto del curso }
		    \Titem{ Datos generales y de descripción del curso }
		    \Titem{ Elección del formato del curso }
		    \Titem{ Numero de secciones }
		    \Titem{ Visibilidad de secciones ocultas }
		    \Titem{ Aspecto del curso }
		    \Titem{ Casilla de experiencia }
		    \Titem{ Conjunto de datos restantes }
		    \Titem{ Botón de confirmación \textit{Guardar y regresar} o \textit{Guardar cambios y mostrar} }
		\end{Titemize}
   	}
    \UCrow{Origen}{ Ratón para las acciones y elecciones, teclado para los campos de texto}
	\UCrow{Salidas}{ \IUref{moodle:} o \IUref{moodle:} } %\begin{Titemize}\Titem{Ninguna}\end{Titemize}
    \UCrow{Destino}{ Pantalla }
    \UCrow{Precondiciones}{
		\begin{Titemize}
	        \Titem{ Contar con los permisos necesarios para crear cursos }
	        \Titem{ Tener instalado el plugin ''Gamedle Level'' }
	        \Titem{ Tener instalado el plugin ''Gamedle Format'' }
	        \Titem{ Que el actor haya elegido el formato de curso {\it Gamedle} en el caso de uso {\it Crear curso} }
		\end{Titemize}
    }
    \UCrow{Postcondiciones}{%
        \begin{Titemize}
            \Titem{ Se crea un curso con soporte para brindar experiencia. }
            \Titem{ Las secciones del curso tienen experiencia predeterminada }
            \Titem{ El curso y las secciones se muestran de acuerdo a las configuraciones realizadas }
        \end{Titemize}
    }
	\UCrow{Errores}{ No se encuentra la opción ''Gamedle Format'' en los formatos del curso, debido a que los plugins no han sido instalados }
    \UCrow{Observaciones}{  }
\end{UseCase}
\clearpage

%\textbullet{Trayectorias}

\begin{UCtrayectoria}{Principal}
    \moodle Muestra la interfaz \IUref{moodle:nuevoCurso}. \UCnote{CU: Crear curso}
    \actor Especifica el ''nombre completo'' y ''nombre corto'' además de la ''descripción'' y los ''datos generales'' del curso.\\
    
    \setcounter{enumi}{0}
    \actor Selecciona el {\it formato del curso} {\bf Curso Gamedle}. \IUref{exp:format} \UCnote{CU: Crear curso con experiencia}
    \sistema Carga los nuevos datos para el formulario del formato: Curso Gamedle.
    \actor Selecciona el ''número de secciones'' que tendrá por defecto el curso.
    \actor Selecciona la ''visibilidad'' de forma colapsada u no visible de las secciones ocultas.
    \actor Especifica si el ''aspecto del curso'' es mostrar una sección por página o mostrar todas.
    \actor Habilita la ''casilla de experiencia''. \UCnote{\bf Trayectoria A}
    \actor Especifica el conjunto de datos restantes para la configuración del curso.
    \actor Presiona botón \fbox{Guardar y regresar} \UCnote{\bf Trayectoria B} \UCnote{\bf Trayectoria C} %\fbox{Guardar cambios}.
    \sistema Crea el curso y las secciones del mismo.
    \sistema Obtiene del esquema de experiencia la cantidad de experiencia para los cursos.
    \sistema Divide la cantidad de experiencia del curso entre las secciones creadas.
    \sistema Guarda los valores de experiencia que le corresponden a cada sección.
    \sistema Muestra la pantalla \IUref{moodle:}
        % en caso de que la división no sea entera, la última sección tendrá la cantidad para completar
    \item[- -] - - {\em Fin del caso de uso.}
\end{UCtrayectoria}

%\begin{UCtrayectoria}[Formato distinto a Gamedle.]{Alternativa A}
    %\actor Selecciona un formato de curso distinto a ''Gamedle Format''
    %\actor Especifica el conjunto de datos restantes para la configuración del curso.
    %\actor Presiona botón \fbox{Guardar y regresar} o \fbox{Guardar cambios y mostrar}.
    %\item[- -] - - {\em Fin del caso de uso}
%\end{UCtrayectoria}

\begin{UCtrayectoria}[Formato Gamedle sin experiencia]{Alternativa A}
    \actor Deshabilita la ''casilla de experiencia''
    \actor Especifica el conjunto de datos restantes para la configuración del curso.
    \item[- -] - - {\em Fin del caso de uso}
\end{UCtrayectoria}

\begin{UCtrayectoria}[Guardar cambios y mostrar]{Alternativa B}
    \actor Presiona botón \fbox{Guardar cambios y mostrar}.
    \sistema Crea el curso y las secciones del mismo.
    \sistema Obtiene del esquema de experiencia la cantidad de experiencia para los cursos.
    \sistema Divide la cantidad de experiencia del curso entre las secciones creadas.
    \sistema Guarda los valores de experiencia que le corresponden a cada sección.
    \sistema Muestra la pantalla \IUref{moodle:}
    \item[- -] - - {\em Fin del caso de uso}
\end{UCtrayectoria}

\begin{UCtrayectoria}[Cancelar]{Alternativa C}
    \actor Presiona botón \fbox{Cancelar}.
    \item[- -] - - {\em Fin del caso de uso}
\end{UCtrayectoria}

%\UserStory{Crear curso con experiencia}{Como {\bf administrador} me gustaría que la instalación de un ... por lo que ...}

\clearpage\clearpage % Crear curso con experiencia
    %\begin{UseCase}{CU-E9}{Recibir experiencia}{
Cuando un alumno conteste un ejercicio y suba un intento para revisión, el sistema le otorgará la experiencia correspondiente. Si recibe experiencia suficiente, el usuario subirá de nivel.
}
	\UCrow{Versión}{\color{gray} 0.2 (Revisado)}
    \UCrow{Autor}{\color{gray}	David Flores Casanova}
    \UCrow{Supervisa}{\color{gray}	Daniel Isaí Ortega Zúñiga}
    \UCrow{Actor}{Alumno} % Gerente, Instructor
    \UCrow{Propósito}{Otorgarle experiencia al actor por completar una actividad.}
    \UCrow{Entradas}{
        Selección en el botón \#2 ''Enviar todo y terminar'' de la interfaz \hyperref[IUM01]{IU-M01 Ver intento de examen} .\newline
        Selección en el botón \#1 ''Enviar todo y terminar'' de la interfaz \hyperref[IUM02]{IU-M02 Confirmación de envío de intento} .\newline
        Presión de la tecla ''Enter'' o la tecla ''espacio''.
   	}
    \UCrow{Origen}{Ratón y teclado de la computadora }
	\UCrow{Salidas}{
	    \begin{Titemize}
        \Titem{''\hyperref[table:METerminosExperiencia1]{Experiencia otorgada}''.}
        \Titem{''\hyperref[table:METerminosExperiencia1]{Experiencia actual}'' del actor.}
        \Titem{''\hyperref[table:METerminosExperiencia1]{Experiencia del nivel}'' del nivel actual del actor.}
        \Titem{Barra de progreso, mostrada según el \hyperref[table:METerminosExperiencia1]{porcentaje actual}.}
        \Titem{Nombre del nivel (Cadena de caracteres, longitud $\leq$ 60)}
        \Titem{Mensaje de felicitaciones del nivel (Cadena de caracteres, longitud $\leq$ 50)}
        \Titem{Imagen del nivel (Imagen, formato '.png')}
        \Titem{Descripción del nivel (Cadena de caracteres, longitud $\leq$ 200)}
	    \end{Titemize}
    }
    \UCrow{Destino}{Pantalla}
    \UCrow{Precondiciones}{
		\begin{CUTitemize}
	        \CUTitem{El actor está registrado como alumno del curso.}
            \CUTitem{El actor no tiene registrados intentos anteriores.}
			\CUTitem{La actividad del curso está creada.} 
            \CUTitem{El módulo de experiencia está habilitado.}
		\end{CUTitemize}
    }
    \UCrow{Postcondiciones}{
		\begin{CUTitemize}
	        \CUTitem{Al actor se le registra su nueva cantidad de experiencia.}
			\CUTitem{Se actualiza la cantidad de experiencia que ha recibido el actor en ese curso.}
		\end{CUTitemize}
    }
	\UCrow{Errores}{E1: El actor no está registrado como alumno del curso}
    \UCrow{Observaciones}{}
\end{UseCase}

%\textbullet{Trayectorias}

\begin{UCtrayectoria}{Principal}
    \actor se encuentra en la interfaz \hyperref[IUM01]{IU-M01 Ver intento de examen}.
    \actor selecciona el botón \#2 ''Enviar todo y terminar'' .
    \sistema muestra la ventana emergente \hyperref[IUM02]{IU-M02 Confirmación de envío de intento}.
    \actor selecciona el botón \#1 ''Enviar todo y terminar'' ({\it Trayectoria alternativa A})
    \sistema comprueba que el módulo de experiencia esté habilitado ({\it Trayectoria alternativa B}).
    \sistema comprueba que el actor que subió el intento es un alumno del curso ({\it Trayectoria alternativa C}).
    \sistema comprueba que el actor no tenga registrados intentos anteriores ({\it Trayectoria alternativa D}).
    \sistema calcula si la experiencia que se le dará al actor provoca que este ''suba de nivel'' ({\it Trayectoria alternativa E}).
    \sistema carga la interfaz \hyperref[IUM03]{IU-M03 Intento calificado}.
    \item[- -] - - {\em El caso de uso termina.}
\end{UCtrayectoria}

\begin{UCtrayectoria}{alternativa A}
    \item[- -] - - {\em El usuario presionó el botón \#2 \fbox{Cancelar} o  el botón \#3 \fbox{X}.}
    \sistema cierra la interfaz \hyperref[IUM02]{IU-M02 Confirmación de envío de intento}.
    \item[- -] - - {\em El caso de uso termina.}
\end{UCtrayectoria}

\begin{UCtrayectoria}{alternativa B}
    \item[- -] - - {\em El módulo de experiencia no está habilitado.}
    \item[- -] - - {\em El caso de uso termina.}
\end{UCtrayectoria}

\begin{UCtrayectoria}{alternativa C}
    \item[- -] - - {\em El actor no está registrado como alumno del curso.}
    \item[- -] - - {\em El caso de uso termina.}
\end{UCtrayectoria}

\begin{UCtrayectoria}{alternativa D}
    \item[- -] - - {\em El actor ya había hecho intentos anteriores.}
    \item[- -] - - {\em El caso de uso termina.}
\end{UCtrayectoria}

\begin{UCtrayectoria}{alternativa E}
    \item[- -] - - {\em La ''\hyperref[table:METerminosExperiencia1]{experiencia otorgada}'' que recibe el actor es suficiente para ''subir de nivel''.}
    %\item[- -] - - {\em Se extiende al CU-E02.}
    \sistema calcula la ''\hyperref[table:METerminosExperiencia2]{experiencia sobrante}''.
    \sistema incrementa el nivel actual del actor en una unidad.
    \sistema cambia el valor de la ''\hyperref[table:METerminosExperiencia1]{experiencia actual}'' por la de la ''\hyperref[table:METerminosExperiencia1]{experiencia sobrante}''
    \sistema guarda el nivel actual del actor y la ''\hyperref[table:METerminosExperiencia1]{experiencia actual}''.
    \sistema comprueba si el nivel actual del actor existe dentro de un rango de niveles ({\it Trayectoria alternativa F}). 
    \sistema carga la interfaz \hyperref[IUE02]{IU-E02 Subir de nivel} con la información por defecto de niveles.
    \actor presiona la tecla ''enter'' o ''espacio''.
    \sistema cierra la interfaz \hyperref[IUE02]{IU-E02 Subir de nivel}.
    \item[- -] - - {\em Se continua en el paso \#9 de la trayectoria principal.}
\end{UCtrayectoria}


\begin{UCtrayectoria}{alternativa F}
    \item[- -] - - {\em El nivel actual del actor está dentro de un rango de niveles.}
    \sistema carga la interfaz \hyperref[IUE02]{IU-E02 Subir de nivel}  con la información especificada en el rango de niveles.
    \item[- -] - - {\em Se continua en el paso \#7 de la trayectoria alternativa E.}
\end{UCtrayectoria}

\vfill\clearpage\clearpage % Recibir experiencia

% =========================================================
\clearpage
\subsection{Diseño}

\subsubsection{Interfaces del módulo de competencia}

    
<<<<<<< HEAD
\subsubsection{IU-M07 Pantalla principal del curso}

 Descripción ...

    \IUfig{1}{modulos/moodle/IU/p_principal_curso}{IU-M07}{Pantalla principal del curso}
=======
\subsubsection{IU-M07 Página Inicial del Sitio}

 Descripción ...

    \IUfig{1}{modulos/moodle/IU/PaginaInicial}{IU-M07}{Página inicial del Sitio}
>>>>>>> fd9af89a23d9b2fb18c947b44db69846e63131db

\subsubsection{Elementos Relevantes}

    \begin{itemize}
    \item {\bf Lorem ipsum}
        ...
    \end{itemize}

\subsubsection{Acciones relevantes}

    \begin{itemize}
    \item {\bf Lorem ipsum}
        ...
    \end{itemize}

\clearpage

    
\subsubsection{IU-M08aa Pantalla principal del curso con la edición activada}

 Descripción ...

    \IUfig{1}{modulos/moodle/IU/p_principal_curso_editar}{IU-M08aa}{ Pantalla principal del curso con la edición activada}

\subsubsection{Elementos Relevantes}

    \begin{itemize}
    \item {\bf Lorem ipsum}
        ...
    \end{itemize}

\subsubsection{Acciones relevantes}

    \begin{itemize}
    \item {\bf Lorem ipsum}
        ...
    \end{itemize}

\clearpage
  % Configuraciones
    
\subsubsection{IU-M07a Pantalla de listado de actividades disponibles}

 Descripción ...

    \IUfig{.5}{modulos/moodle/IU/p_listado_actividades}{IU-M07a}{Pantalla de listado de actividades disponibles}

\subsubsection{Elementos Relevantes}

    \begin{itemize}
    \item {\bf Lorem ipsum}
        ...
    \end{itemize}

\subsubsection{Acciones relevantes}

    \begin{itemize}
    \item {\bf Lorem ipsum}
        ...
    \end{itemize}

\clearpage

    
\subsubsection{IU-M08b Menú desplegable al presionar el botón Editar}

 Descripción ...

    \IUfig{.3}{modulos/moodle/IU/md_boton_editar_instancia}{IU-M08b}{Menú desplegable al presionar el botón Editar}

\subsubsection{Elementos Relevantes}

    \begin{itemize}
    \item {\bf Lorem ipsum}
        ...
    \end{itemize}

\subsubsection{Acciones relevantes}

    \begin{itemize}
    \item {\bf Lorem ipsum}
        ...
    \end{itemize}

\clearpage

    
\subsubsection{IU-M08c Mensaje de confirmación al presionar el botón Eliminar}

 Este mensaje de confirmación se muestra para reducir la cantidad de errores que ocurren cuando se presiona el botón equivocado en el menú \refElem{IU-M08b}.

    \IUfig{.3}{modulos/moodle/IU/m_confirmacion_eliminar_instancia}{IU-M08c}{Mensaje de confirmación al presionar el botón Eliminar}

\subsubsection{Elementos Relevantes}

    \begin{itemize}
    \item {\bf Si}
    \item {\bf No}

    \end{itemize}

\subsubsection{Acciones relevantes}

    \begin{itemize}
    \item {\bf Confirmar la eliminación de la instancia}
    \item {\bf Cancelar la eliminación de la instancia}

    \end{itemize}

\clearpage
  % Configuraciones
    
\subsubsection{IU-C01: Pantalla creación de nueva instancia de actividad de competencia uno contra uno}

 Descripción ...

    \IUfig{1}{modulos/comp/IU/p_creacion_competencia1vs1}{IU-C01}{%
        Pantalla de creación de nueva instancia de actividad de competencia uno contra uno}

\subsubsection{Elementos Relevantes}

    \begin{itemize}
    \item {\bf Lorem ipsum}
        ...
    \end{itemize}

\subsubsection{Acciones relevantes}

    \begin{itemize}
    \item {\bf Lorem ipsum}
        ...
    \end{itemize}

\clearpage
  % Configuraciones Generales
    
\subsubsection{IU-C02: Pantalla principal de actividad de competencia uno contra uno}

Esta pantalla es la principal de la competencia uno contra uno, en ella se puede ver el número de victorias que tiene el usuario actual y se le muestran los estudiantes a los que puede desafiar, también, si tiene desafíos pendientes se le muestran para que los pueda aceptar.

    \IUfig{1}{modulos/comp/IU/p_principal_competencia1vs1}{IU-C02}{%
        Pantalla principal de actividad de competencia uno contra uno}

\subsubsection{Elementos Relevantes}

    \begin{itemize}
    \item {\bf Número de victorias}
    \item {\bf Compañeros del curso}
    \item {\bf Desafíos pendientes}
    \item {\bf Tabla de posiciones}
    \item {\bf Historial}
    \item {\bf Desafiar}
    \item {\bf Aceptar desafío}
    \item {\bf Monedas a apostar}
    \end{itemize}

\subsubsection{Acciones relevantes}

    \begin{itemize}
    \item {\bf Desafiar a un estudiante}
    \item {\bf Aceptar un desafío de un usuario}
    \item {\bf Abrir la pestaña de tabla de posiciones}
    \item {\bf Abrir la pestaña de historial}
    \end{itemize}

\clearpage

    
\subsubsection{IU-C02a: Pantalla principal de actividad de competencia uno contra uno sin apuestas}

Esta pantalla es la principal de la competencia uno contra uno, en ella se puede ver el número de victorias que tiene el usuario actual y se le muestran los estudiantes a los que puede desafiar, también, si tiene desafíos pendientes se le muestran para que los pueda aceptar.

    \IUfig{1}{modulos/comp/IU/p_principal_competencia1vs1_sinapuesta}{IU-C02a}{%
        Pantalla principal de actividad de competencia uno contra uno sin apuestas}

\subsubsection{Elementos Relevantes}

    \begin{itemize}
    \item {\bf Número de victorias}
    \item {\bf Compañeros del curso}
    \item {\bf Desafíos pendientes}
    \item {\bf Tabla de posiciones}
    \item {\bf Historial}
    \item {\bf Desafiar}
    \item {\bf Aceptar desafío}
    \end{itemize}

\subsubsection{Acciones relevantes}

    \begin{itemize}
    \item {\bf Desafiar a un estudiante}
    \item {\bf Aceptar un desafío de un usuario}
    \item {\bf Abrir la pestaña de tabla de posiciones}
    \item {\bf Abrir la pestaña de historial}
    \end{itemize}

\clearpage
  % Configuraciones Visuales
    
\subsubsection{IU-C03: Pantalla de creación de nueva instancia de actividad de competencia uno contra sistema}

 Esta pantalla es el formulario que se le muestra al usuario cuando quiere crear una nueva instancia de la actividad uno contra sistema.

    \IUfig{1}{modulos/comp/IU/p_creacion_competencia1vscpu}{IU-C03}{%
        Pantalla de creación de nueva instancia de actividad de competencia uno contra sistema}

        \subsubsection{Elementos Relevantes}

            \begin{itemize}
            \item {\bf Nombre de actividad}
            \item {\bf Categoría de preguntas}
            \item {\bf Dificultad requerida}
            \item {\bf El estudiante debe de vencer al menos a la dificultad:}
            \item {\bf Guardar cambios y regresar al curso}
            \item {\bf Guardar cambios y mostrar}
            \item {\bf Cancelar}
            \end{itemize}

        \subsubsection{Acciones relevantes}

            \begin{itemize}
            \item {\bf Se puede crear una nueva instancia y mostrar su creación en la pantalla principal del curso}
            \item {\bf Se puede crear una nueva instancia y mostrar su pantalla principal}
            \item {\bf Se puede elegir como se finalizará la actividad}
            \end{itemize}

        \clearpage
  % Configuraciones Generales
    
\subsubsection{IU-C04: Pantalla principal de actividad de competencia uno contra sistema}

Esta pantalla es la principal de la competencia uno contra sistema, en ella se pueden ver las dificultades del sistema que han sido derrotadas, así como elegir el desafiar a una dificultad.

    \IUfig{1}{modulos/comp/IU/p_principal_competencia1vscpu}{IU-C04}{%
        Pantalla principal de actividad de competencia uno contra sistema}

        \subsubsection{Elementos Relevantes}

            \begin{itemize}
            \item {\bf Puntuación}
            \item {\bf Historial}
            \item {\bf Seleccione una dificultad}
            \item {\bf Empezar}
            \end{itemize}

        \subsubsection{Acciones relevantes}

            \begin{itemize}
            \item {\bf Desafiar a una dificultad}
            \item {\bf Abrir la pestaña de puntuaciones}
            \item {\bf Abrir la pestaña de historial}
            \end{itemize}

        \clearpage

    
\subsubsection{IU-C05: Pantalla Actualización de instancia de actividad de competencia uno contra sistema}

 Descripción ...

    \IUfig{1}{modulos/comp/IU/p_actualizacion_competencia1vscpu}{IU-C05}{%
        Pantalla de actualizacion instancia de actividad de competencia uno contra sistema}

\subsubsection{Elementos Relevantes}

    \begin{itemize}
    \item {\bf Lorem ipsum}
        ...
    \end{itemize}

\subsubsection{Acciones relevantes}

    \begin{itemize}
    \item {\bf Lorem ipsum}
        ...
    \end{itemize}

\clearpage

    
\subsubsection{IU-C06: Pantalla Actualización de instancia de actividad de competencia uno contra uno}

 Esta pantalla es el formulario que se le muestra al usuario cuando quiere actualizar una instancia de la actividad uno contra uno.

    \IUfig{1}{modulos/comp/IU/p_actualizacion_competencia1vs1}{IU-C06}{%
        Pantalla de actualización instancia de actividad de competencia uno contra uno}

        \subsubsection{Elementos Relevantes}

            \begin{itemize}
            \item {\bf Nombre de actividad}
            \item {\bf Categoría de preguntas}
            \item {\bf Apuestas}
            \item {\bf Competencias ganadas requeridas}
            \item {\bf Los estudiantes deben de haber ganado un mínimo de competencias de:}
            \item {\bf Guardar cambios y regresar al curso}
            \item {\bf Guardar cambios y mostrar}
            \item {\bf Cancelar}
            \end{itemize}

        \subsubsection{Acciones relevantes}

            \begin{itemize}
            \item {\bf Se puede actualizar una instancia y mostrar su creación en la pantalla principal del curso}
            \item {\bf Se puede actualizar una instancia y mostrar su pantalla principal}
            \item {\bf Se puede elegir como se finalizará la actividad}
            \end{itemize}

        \clearpage

    
\subsubsection{IU-C07: Pantalla de historial de actividad de competencia uno contra uno}

 En esta pantalla se le muestra un resumen del historial de las partidas del usuario actual. También se le muestra un desglose de cada partida, los resultados obtenidos en ella y su estado actual.

    \IUfig{1}{modulos/comp/IU/p_historial_competencia1vs1}{IU-C07}{%
        Pantalla de historial de actividad de competencia uno contra uno}

\subsubsection{Elementos Relevantes}

    \begin{itemize}
    \item {\bf Victorias}
    \item {\bf Empates}
    \item {\bf Derrotas}
    \item {\bf En curso}
    \item {\bf Retirada}

    \end{itemize}

\subsubsection{Acciones relevantes}

    \begin{itemize}
    \item {\bf Mostrar el historial del usuario actual}

    \end{itemize}

\clearpage

    
\subsubsection{IU-C08: Pantalla de tabla de posiciones de actividad de competencia uno contra uno}

 Esta pantalla muestra la tabla de posiciones de los usuarios que han realizado competencias, las posiciones son otorgadas de acuerdo el número de victorias del usuario.

    \IUfig{1}{modulos/comp/IU/p_tablap_competencia1vs1}{IU-C08}{%
        Pantalla de tabla de posiciones de actividad de competencia uno contra uno}

\subsubsection{Elementos Relevantes}

    \begin{itemize}
    \item {\bf Posición}
    \item {\bf Nombre}
    \item {\bf Número victorias}
    \end{itemize}

\subsubsection{Acciones relevantes}

    \begin{itemize}
    \item {\bf Ver tabla de posiciones}

    \end{itemize}

\clearpage

    
\subsubsection{IU-C09: Pantalla de cuestionario}

 Descripción ...

    \IUfig{1}{modulos/comp/IU/p_questionario}{IU-C09}{%
        Pantalla de cuestionario}

\subsubsection{Elementos Relevantes}

    \begin{itemize}
    \item {\bf Lorem ipsum}
        ...
    \end{itemize}

\subsubsection{Acciones relevantes}

    \begin{itemize}
    \item {\bf Lorem ipsum}
        ...
    \end{itemize}

\clearpage

    
\subsubsection{IU-C10: Pantalla partida en progreso}

 Descripción ...

    \IUfig{1}{modulos/comp/IU/p_esperar_respuesta_1v1}{IU-C10}{%
        Pantalla partida en progreso}

\subsubsection{Elementos Relevantes}

    \begin{itemize}
    \item {\bf Lorem ipsum}
        ...
    \end{itemize}

\subsubsection{Acciones relevantes}

    \begin{itemize}
    \item {\bf Lorem ipsum}
        ...
    \end{itemize}

\clearpage

    
\subsubsection{IU-C11: Pantalla partida ganada}

 Descripción ...

    \IUfig{1}{modulos/comp/IU/p_gano_1v1}{IU-C11}{%
        Pantalla partida ganada}

\subsubsection{Elementos Relevantes}

    \begin{itemize}
    \item {\bf Lorem ipsum}
        ...
    \end{itemize}

\subsubsection{Acciones relevantes}

    \begin{itemize}
    \item {\bf Lorem ipsum}
        ...
    \end{itemize}

\clearpage

    
\subsubsection{IU-C12: Pantalla partida perdida}

 Descripción ...

    \IUfig{1}{modulos/comp/IU/p_perdio_1v1}{IU-C12}{%
        Pantalla partida perdida}

\subsubsection{Elementos Relevantes}

    \begin{itemize}
    \item {\bf Lorem ipsum}
        ...
    \end{itemize}

\subsubsection{Acciones relevantes}

    \begin{itemize}
    \item {\bf Lorem ipsum}
        ...
    \end{itemize}

\clearpage

    
\subsubsection{IU-C13: Mensaje de alerta al abandonar el cuestionario}

Este mensaje de advertencia se muestra cuando el usuario quiere abandonar un cuestionario sin haberlo terminado.

    \IUfig{.3}{modulos/comp/IU/m_alerta_cuestionario}{IU-C13}{%
        Mensaje de alerta al abandonar el cuestionario}

\subsubsection{Elementos Relevantes}

    \begin{itemize}
    \item {\bf abandonar}
    \item {\bf Cancelar}
    \end{itemize}

\subsubsection{Acciones relevantes}

    \begin{itemize}
    \item {\bf Presionar abandonar} y salir del cuestionario
    \item {\bf Presionar cancelar} y mantenerse dentro del cuestionario
    \end{itemize}

\clearpage

    
\subsubsection{IU-C14: Pantalla de historial de actividad de competencia uno contra sistema}

 Descripción ...

    \IUfig{1}{modulos/comp/IU/p_historial_competenciacpu}{IU-C14}{%
        Pantalla de historial de actividad de competencia uno contra sistema}

\subsubsection{Elementos Relevantes}

    \begin{itemize}
    \item {\bf Lorem ipsum}
        ...
    \end{itemize}

\subsubsection{Acciones relevantes}

    \begin{itemize}
    \item {\bf Lorem ipsum}
        ...
    \end{itemize}

\clearpage

    
\subsubsection{IU-C15: Pantalla de tabla de puntuaciones de actividad de competencia uno contra sistema}

 Esta pantalla muestra las tablas de puntuaciones ''Primer intento'' y ''Mejor intento'' de los intentos de todos los usuarios que han utilizado la instancia de la competencia. También se filtra por dificultad del sistema para solo mostrar los intentos de una misma dificultad.

    \IUfig{1}{modulos/comp/IU/p_puntuacion}{IU-C15}{%
        Pantalla de tabla de puntuaciones de actividad de competencia uno contra sistema}

\subsubsection{Elementos Relevantes}

    \begin{itemize}
    \item {\bf Seleccione una dificultad}
    \item {\bf Primer intento}
    \begin{itemize}
      \item {\bf Posición}
      \item {\bf Nombre}
      \item {\bf Puntuación}
    \end{itemize}
    \item {\bf Mejor intento}
    \begin{itemize}
      \item {\bf Posición}
      \item {\bf Nombre}
      \item {\bf Puntuación}
    \end{itemize}
    \end{itemize}

\subsubsection{Acciones relevantes}

    \begin{itemize}
    \item {\bf Seleccionar la dificultad de la que se quiere saber los intentos realizados.}
    \item {\bf Mostrar todos los intentos de los usuarios que han utilizado la instancia}
    \end{itemize}

\clearpage
  % Configuraciones Visuales
    %
\subsubsection{IU-E03a Módulo de experiencia desactivado}

 Descripción ...

    \IUfig{1}{modulos/exp/IU/settingsExperienceDisabled}{IU-E03a}{%
        Módulo de experiencia desactivado}

\subsubsection{Elementos Relevantes}

    \begin{description}
    \bTerm{tSelectColor}{panel de colores} ...
    \end{description}

\subsubsection{Acciones relevantes}

    \begin{itemize}
    \item {\bf Lorem ipsum}
        ...
    \end{itemize}

\clearpage
 % Configuraciones Mod Exp desactivado
    %
\subsection{IU-E04: Configuraciones del sistema de experiencia}

 Descripción ...

    \IUfig{1}{modulos/exp/IU/settingsEsquema}{IU-E04}{%
        Configuraciones del sistema de experiencia}

\subsubsection{Elementos Relevantes}

    \begin{itemize}
    \item {\bf Lorem ipsum}
        ...
    \end{itemize}

\subsubsection{Acciones relevantes}

    \begin{itemize}
    \item {\bf Lorem ipsum}
        ...
    \end{itemize}

\clearpage
  % Configuraciones Esquema
    %
\subsection{IU-E05 Configuración de eventos con experiencia}

 Descripción ...

    \IUfig{1}{modulos/exp/IU/SettingsEventos}{IU-E05}%
        {Configuracion de eventos con experiencia}

\subsubsection{Elementos Relevantes}

    \begin{itemize}
    \item {\bf Lorem ipsum}
        ...
    \end{itemize}

\subsubsection{Acciones relevantes}

    \begin{itemize}
    \item {\bf Lorem ipsum}
        ...
    \end{itemize}

\clearpage
  % Configuraciones de Eventos
    %
\subsection{IU-E05a Eventos con experiencia desactivados}

 Descripción ...

    \IUfig{1}{modulos/exp/IU/SettingsEventsDisabled}{IU-E05a}%
    {Eventos con experiencia desactivados}

\subsubsection{Elementos Relevantes}

    \begin{itemize}
    \item {\bf Lorem ipsum}
        ...
    \end{itemize}

\subsubsection{Acciones relevantes}

    \begin{itemize}
    \item {\bf Lorem ipsum}
        ...
    \end{itemize}

\clearpage
 % Configuraciones Events desativados
    %
\subsection{IU-E06 Creación de un curso gamificado}

 Descripción ...

    \IUfig{1}{modulos/exp/IU/CursoCreate}{IU-E06}{Creación curso gamificado}

\subsubsection{Elementos Relevantes}

    \begin{itemize}
    \item {\bf Lorem ipsum}
        ...
    \end{itemize}

\subsubsection{Acciones relevantes}

    \begin{itemize}
    \item {\bf Lorem ipsum}
        ...
    \end{itemize}

\clearpage
  % Curso Gamificado
    %
\subsection{IU-E06a Curso Gamificado}

 Descripción ...

    \IUfig{1}{modulos/exp/IU/CursoView}{IU-E06a}{Curso Gamificado}

\subsubsection{Elementos Relevantes}

    \begin{itemize}
    \item {\bf Lorem ipsum}
        ...
    \end{itemize}

\subsubsection{Acciones relevantes}

    \begin{itemize}
    \item {\bf Lorem ipsum}
        ...
    \end{itemize}

\clearpage
  % Curso Gamificado
    %
\subsection{IU-E06b Curso gamificado en edición}

 Descripción ...

    \IUfig{1}{modulos/exp/IU/CursoEdit}{IU-E06b}{Curso gamificado en edición}

\subsubsection{Elementos Relevantes}

    \begin{itemize}
    \item {\bf Lorem ipsum}
        ...
    \end{itemize}

\subsubsection{Acciones relevantes}

    \begin{itemize}
    \item {\bf Lorem ipsum}
        ...
    \end{itemize}

\clearpage
 % Curso Gamificado (editing)

    %
\subsection{IU-E04: Configuraciones del sistema de experiencia}

 Descripción ...

    \IUfig{1}{modulos/exp/IU/settingsEsquema}{IU-E04}{%
        Configuraciones del sistema de experiencia}

\subsubsection{Elementos Relevantes}

    \begin{itemize}
    \item {\bf Lorem ipsum}
        ...
    \end{itemize}

\subsubsection{Acciones relevantes}

    \begin{itemize}
    \item {\bf Lorem ipsum}
        ...
    \end{itemize}

\clearpage
 % Configuraciones de Comportamiento
    %
\subsection{IU-E05 Configuración de eventos con experiencia}

 Descripción ...

    \IUfig{1}{modulos/exp/IU/SettingsEventos}{IU-E05}%
        {Configuracion de eventos con experiencia}

\subsubsection{Elementos Relevantes}

    \begin{itemize}
    \item {\bf Lorem ipsum}
        ...
    \end{itemize}

\subsubsection{Acciones relevantes}

    \begin{itemize}
    \item {\bf Lorem ipsum}
        ...
    \end{itemize}

\clearpage
 % Configuraciones de eventos

\subsubsection{Diseño de plugins} % TODO CHANGE FOR INPUTS
\subsubsection{Diagrama de componentes} % TODO CHANGE FOR INPUTS
\subsubsection{Diagrama de clases} % TODO CHANGE FOR INPUTS

\subsection{Pruebas}

    
    
\TestCase{CPC-C01}{Crear nuevas instancias de la actividad de competencia 1 contra 1}
   % Instalar plugin del esquema de comperiencia
    
\TestCase{CPC-C02}{Actualizar instancia de la actividad de competencia uno contra uno}

\begin{quote} %Contendrá la descripción del guión
	\textbf{ID:} C02.\\
    \textbf{Autor: } Ricardo Naranjo Polit\\
	\textbf{Alcance:}  \refElem{CU-C02}.\\
    \textbf{Preparación:}\\
    	%Contendrá todos los distintos datos que deberán estar preparados
      -Que se haya realizado el caso de prueba \refElem{CPC-C01}\\

\end{quote}

    \textbf{Entradas:}\\
    \begin{enumerate}
        \item \textbf{Nombre de actividad:} ''Competencia uno contra uno actualizada''
    \end{enumerate}
    \textbf{Pasos:}\\

    Trayectoria principal de \refElem{CU-C02}\\

    \textbf{Salida:}\\

     En la pantalla \refElem{IU-M08} la instancia de la competencia uno contra uno llamada ''Competencia uno contra uno actualizada''.

    
\TestCase{CPC-C03}{Eliminar instancia de la actividad de competencia uno contra uno}

\begin{quote} %Contendrá la descripción del guión
	\textbf{ID:} C03.\\
    \textbf{Autor: } Ricardo Naranjo Polit\\
	\textbf{Alcance:}  \refElem{CU-C03}.\\
    \textbf{Preparación:}\\
    	%Contendrá todos los distintos datos que deberán estar preparados
      -Que se haya realizado el caso de prueba \refElem{CPC-C01}\\

\end{quote}

    \textbf{Entradas:}\\
    Ninguna\\
    \textbf{Pasos:}\\

    Trayectoria principal de \refElem{CU-C03}\\

    \textbf{Salida:}\\

     En la pantalla \refElem{IU-M08} la instancia de la competencia uno contra uno llamada ''Competencia uno contra uno'' no debe mostrarse.

    
\TestCase{CPC-C04}{Crear nueva instancia de la actividad de competencia uno contra sistema}
   % Instalar plugin del esquema de comperiencia
    
\TestCase{CPC-C05}{Actualizar instancia de la actividad de competencia uno contra sistema}
\begin{quote} %Contendrá la descripción del guión
	\textbf{ID:} C05.\\
    \textbf{Autor: } Ricardo Naranjo Polit\\
	\textbf{Alcance:}  \refElem{CU-C05}.\\
    \textbf{Preparación:}\\
    	%Contendrá todos los distintos datos que deberán estar preparados
      -Que se haya realizado el caso de prueba \refElem{CPC-C04}\\

\end{quote}

    \textbf{Entradas:}\\
    \begin{enumerate}
        \item \textbf{Nombre de actividad:} ''Competencia uno contra sistema actualizada''
    \end{enumerate}
    \textbf{Pasos:}\\

    Trayectoria principal de \refElem{CU-C05}\\

    \textbf{Salida:}\\

     En la pantalla \refElem{IU-M08} la instancia de la competencia uno contra sistema llamada ''Competencia uno contra sistema actualizada''.

    
\TestCase{CPC-C06}{Eliminar instancia de la actividad de competencia uno contra sistema}

\begin{quote} %Contendrá la descripción del guión
	\textbf{ID:} C06.\\
    \textbf{Autor: } Ricardo Naranjo Polit\\
	\textbf{Alcance:}  \refElem{CU-C06}.\\
    \textbf{Preparación:}\\
    	%Contendrá todos los distintos datos que deberán estar preparados
      -Que se haya realizado el caso de prueba \refElem{CPC-C04}\\

\end{quote}

    \textbf{Entradas:}\\
    Ninguna\\
    \textbf{Pasos:}\\

    Trayectoria principal de \refElem{CU-C06}\\

    \textbf{Salida:}\\

     En la pantalla \refElem{IU-M08} la instancia de la competencia uno contra sistema llamada ''Competencia uno contra sistema'' no debe mostrarse.

    
\TestCase{CPC-C07}{Mostrar el estado de la competencia uno contra uno}

\begin{quote} %Contendrá la descripción del guión
	\textbf{ID:} C07.\\
    \textbf{Autor: } Ricardo Naranjo Polit\\
	\textbf{Alcance:}  \refElem{CU-C07}.\\
    \textbf{Preparación:}\\
    	%Contendrá todos los distintos datos que deberán estar preparados
      -Que se haya realizado el caso de prueba \refElem{CPC-C01}\\
      -Que el usuario actual pueda ingresar al curso en el que se encuentra la instancia
\end{quote}

    \textbf{Entradas:}\\
    Ninguna\\
    \textbf{Pasos:}\\

    Trayectoria principal de \refElem{CU-C07}\\

    \textbf{Salida:}\\

    Se debe mostrar la pantalla \refElem{IU-C02}.

    
\TestCase{CPC-C08}{Mostrar el historial de las partidas de competencia uno contra uno}

    
\TestCase{CPC-C09}{Mostrar la tabla de posiciones de competencia uno contra uno}

    
\TestCase{CPC-C10}{Desafiar a un estudiante apostando}

\begin{quote} %Contendrá la descripción del guión
	\textbf{ID:} C10.\\
    \textbf{Autor: } Ricardo Naranjo Polit\\
	\textbf{Alcance:}  \refElem{CU-C10}.\\
    \textbf{Preparación:}\\
    	%Contendrá todos los distintos datos que deberán estar preparados
      -Que se haya realizado el caso de prueba \refElem{CPC-C01}\\

      \begin{quote}
      -Que haya usuarios inscritos en el curso.\\
      \textbf{Usuario 1}:
        	\begin{itemize} %Contendrá todos los atributos de dicho ejemplo
                \item \textbf{Nombre\_de\_usuario:} ''usuario1''.
                \item \textbf{Nombre:} ''Usuario uno''
                \item \textbf{Apellidos:} ''Apellido usuario uno''
                \item \textbf{Dirección Email:} ''usuario1@test.com''

            \end{itemize}
      \textbf{Usuario 2}:
        	\begin{itemize} %Contendrá todos los atributos de dicho ejemplo
                \item \textbf{Nombre\_de\_usuario:} ''usuario2''.
                \item \textbf{Nombre:} ''Usuario dos''
                \item \textbf{Apellidos:} ''Apellido usuario dos''
                \item \textbf{Dirección Email:} ''usuario2@test.com''

            \end{itemize}
    \end{quote}


\end{quote}

    \textbf{Entradas:}\\
    \begin{enumerate}
        \item \textbf{Nombre de actividad:} ''Nueva competencia uno contra uno''
        \item \textbf{Categoría de preguntas:} ''nueva''
        \item \textbf{Apuestas:} ''Activada''
        \item \textbf{Seguimiento de finalización:} ''Mostrar la actividad como completada cuando se cumplan las condiciones''
        \item \textbf{Requerir ver:} Desactivado
        \item \textbf{Competencias ganadas requeridas:} Activado
        \item \textbf{Los estudiantes deben de haber ganado un mínimo de competencias de:} 1
    \end{enumerate}
    \textbf{Pasos:}\\

    Trayectoria principal de \refElem{CU-C10}\\

    \textbf{Salida:}\\

     En la pantalla \refElem{IU-M08} la nueva instancia de la competencia uno contra uno llamada ''Nueva competencia uno contra uno''.

    
\TestCase{CPC-C11}{Desafiar a un estudiante sin apostar}

    
\TestCase{CPC-C12}{Mostrar el estado de la competencia uno contra sistema}

    
\TestCase{CPC-C13}{Desafiar al sistema}

    
\TestCase{CPC-C14}{Mostrar la tabla de puntuaciones de competencia uno contra sistema}

    
\TestCase{CPC-C15}{Ver historial de puntuaciones de competencia uno contra sistema}

    
\TestCase{CPC-C16}{Aceptar un desafío en competencia uno contra uno apostando}

    
\TestCase{CPC-C17}{Aceptar un desafío en competencia uno contra uno sin apostar}

    %
\TestCase{CPC-E02}{Realizar configuraciones del módulo de experiencia}

    %
\TestCase{CPC-E02-1}{Realizar configuración de visualización de niveles}

    %
\TestCase{CPI-E02-1a}{Realizar configuraciones visuales con todos los datos erroneos}

    %
\TestCase{CPI-E02-1b}{Configuraciones visuales con formato y nombre de imagen inválidos}


    %
\TestCase{CPC-E02-2a}{Realizar configuraciones del sistema de experiencia}

    %
\TestCase{CPC-E02-2b}{Realizar configuraciones con cursos iniciados}

    %
\TestCase{CPC-E02-2c}{Realizar configuraciones del sistema de experiencia con estudiantes con experiencia establecida}

    %
\TestCase{CPI-E02-2}{Realizar configuraciones del sistema de experiencia con datos inválidos}


    %
\TestCase{CPC-E02-3}{Realizar configuraciones del sistema de experiencia con datos correctoss}

    %
\TestCase{CPI-E02-3}{Realizar configuraciones del eventos con datos inválidos}


    %
\TestCase{CPC-E03}{Desinstalar plugins del módulo de experiencia}


    %
\TestCase{CPC-E04}{Crear un curso gamificado}

    %
\TestCase{CPI-E04}{Crear un curso gamificado con la experiencia deshabilitada}


    %
\TestCase{CPC-E05}{Eliminar un curso gamificado sin estudiantes inscritos}

    %
\TestCase{CPC-E05a}{Eliminar un curso gamificado con alumnos inscritos}


    %
\TestCase{CPC-E12}{Crear un usuario gamificado (administrador)}

    %
\TestCase{CPC-E12a}{Crear un usuario gamificado mediante el auto-registro}


    %
\TestCase{CPC-E13}{Eliminar usuario gamificado}


\subsection{Funcionalidades de moodle}

En esta sección se abordan las funcionalidades propias de moodle que fueron utilizadas para el desarrollo
de los módulos de competencia, así como una descripción de su objetivo y problemas encontrados al utilizarlas.\\

  
\subsection{Entidades de moodle}

Debido a que moodle cuenta con más de 400 tablas en su versión 3.5, se opta
por mostrar 2 subconjuntos que muestren las tablas que se utilizan para el proyecto.\\

\noindent El primer subconjunto es aquel que explica la forma en que moodle implementa los cursos, 
secciones de curso, actividades, usuarios y roles (el cual se presenta en la figura \ref{fig:BD-ER-M1}), 
mientras que el segundo conjunto muestra como moodle maneja toda la 
estructura de las preguntas creadas por el profesor y respondidas por el usuario
(el cual se presenta en la figura \ref{fig:BD-ER-M2}).  



En lugar de describir y mostrar cada uno de los campos de cada una de las entidades de moodle que se contemplan,
lo que se quiere lograr con ambos esquemas (\ref{fig:BD-ER-M1}) y \ref{fig:BD-ER-M2}))
es expresar la idea general del comportamiento.

\clearpage
\addfigure{0.7}{analisis/diagrams/db_module_structure}{fig:BD-ER-M1}{Esquema de la base de datos de moodle 'Cursos'}


\noindent Utilizando la figura \ref{fig:BD-ER-M1}, se obtuvieron las siguientes reglas y caracteristicas que contiene moodle respecto a los usuarios en un curso y a la estructura de los cursos.
\begin{enumerate}
    \item Un usuario -{\it mdl\_user}- tiene un rol -{\it mdl\_role}- en un cierto contexto -{\it mdl\_context}-, cuyo  '{\it context\_level}' sea igual a cincuenta(50).
    \item Si el contexto '{\it context\_level}' es de 50, el atributo '{\it instance\_id}' hace referencia al atributo '{\it id}' de un curso -{\it mdl\_course}-.
    \item El curso -{\it mdl\_course}- tiene varias secciones -{\it mdl\_course\_sections}-.
    \item Cada seccion -{\it mdl\_course\_sections}- tiene varias actividades -{\it mdl\_course\_modules}- que pertenecen a un tipo de actividad -{\it mdl\_modules}-.
    \item Por cada registro en tipo de actividad -{\it mdl\_modules}-, se tiene una entidad que lleva el mismo nombre.
    \item El atributo '{\it instance\_id}' de una actividad  -{\it mdl\_course\_modules}- apunta a diferentes entidades. La entidad a la que apunta depende del nombre del tipo de actividad -{\it mdl\_modules}-.
    \item Un usuario -{\it mdl\_user}- se inscribió -{\it mdl\_user\_enrolments}- a un curso -{\it mdl\_course}-, por medio de un formato soportado de inscripción -{\it mdl\_enrol}-.
\end{enumerate}

\clearpage

 \addfigure{0.7}{analisis/diagrams/db_module_questions}{fig:BD-ER-M2}{Esquema de la base de datos de moodle 'Preguntas' }



\noindent Utilizando la figura \ref{fig:BD-ER-M2}, se obtuvieron las siguientes reglas y caracteristicas que contiene moodle respecto a las preguntas.
\begin{enumerate}
    \item Las preguntas -{\it mdl\_question}- tienen versiones -{\it mdl\_question\_attempts}-.
    \item Una pregunta -{\it mdl\_question}- pertenece a un banco de preguntas -{\it mdl\_question\_categories}-.
    \item La versión de una pregunta -{\it mdl\_question\_attempts}- es contestada -{\it mdl\_question\_usages}- en un determinado contexto -{\it mdl\_context}-.
    \item Un usuario -{\it mdl\_user}- responde una versión de una pregunta -{\it mdl\_question\_attempt\_stepts}-.
    \item El responder una versión de una pregunta -{\it mdl\_question\_attempt\_stepts}- conlleva pasos -{\it mdl\_question\_attempt\_stept\_data}-, los cuales son: cómo se muestra, si ya se terminó de responder y qué se respondió.
\end{enumerate}


 A continuación se presenta la especificación de las entidades del esquema de base
 de datos de moodle que son relevantes para el desarrollo de los módulos y submódulos
 de proyecto.

    \begin{cdtEntidad}{mdl-config-plugins}{Configuración de Plugin}{%
    Es una tabla del núcleo de moodle que almacena todas las configuraciones globales
    relacionadas a los plugins instalados, al iniciar moodle las configuraciones de los
    plugins instalados y habilitados se cargan en memoria.}

	    \brAttr{id}{Id}{tInt}{%
	        Es el dígito que representa el identificador único para una configuración
            específica de un plugin.\par

            \it Restricciones:
            \refElem{tPrimaryKey},
            \refElem{tAutoIncrement}.
        }

        \brAttr{plugin}{Plugin}{tVarchar}{%
            Cadena de caracteres del nombre identificador del plugin al cual pertenece
            la configuración.\par

            \it Restricciones:
            \refElem{tRequired},
            \refElem{tRange} (0,100),
            \refElem{tUniqueKey}
        }

        \brAttr{name}{Nombre}{tVarchar}{%
            Cadena de caracteres que representa el nombre de la configuración de un
            plugin en específico.\par

            \it Restricciones:
            \refElem{tUniqueKey},
            \refElem{tRange} (0,100),
            \refElem{tRequired}
        }

        \brAttr{value}{Valor}{tVarchar}{%
            Cadena que almacena el valor de una configuración perteneciente a alguno
            de los plugins instalados.\par

            \it Restricciones:
            \refElem{tRange} (0,4294967295),
            \refElem{tRequired}
        }
    \end{cdtEntidad}\schemeName{config\_plugins}

    \begin{cdtEntidad}{mdl-user}{Usuario de moodle}{%
    Es una tabla del núcleo de moodle que contiene toda la información que se
    almacena de los usuarios en la plataforma, independientemente del rol que
    estos contenga, esta relación contiene más de 53 atributos, sin embargo solo
    se detallan aquellos relevantes.}

	    \brAttr{id}{Id}{tInt}{%
	        Es el dígito que representa el identificador único para cada uno
            de los usuarios en moodle.\par

            \it Restricciones:
            \refElem{tPrimaryKey},
            \refElem{tAutoIncrement}.
        }
	    \brAttr{username}{nombre de usuario}{tVarchar}{%
	        .\par

            \it Restricciones:
            \refElem{tRequired},
            \refElem{tLength} 0-100
        }
	    \brAttr{password}{contraseña}{tVarchar}{%
	        .\par

            \it Restricciones:
            \refElem{tRequired},
            \refElem{tLength} 0-255.
        }
	    \brAttr{firstname}{nombre}{tVarchar}{%
	        .\par

            \it Restricciones:
            \refElem{tRequired},
            \refElem{tLength} 0-100
        }
	    \brAttr{lastname}{apellido}{tVarchar}{%
	        .\par

            \it Restricciones:
            \refElem{tRequired},
            \refElem{tLength} 0-100
        }
	    \brAttr{email}{correo}{tVarchar}{%
	        .\par

            \it Restricciones:
            \refElem{tRequired},
            \refElem{tLength} 0-100
        }
	    \brAttr{lastaccess}{último registro}{tInt}{%
	        .\par

            \it Restricciones:
            \refElem{tRequired},
            \refElem{tLength} 10
        }
	    \brAttr{city}{ciudad}{tVarchar}{%
	        .\par

            \it Restricciones:
            \refElem{tRequired},
            \refElem{tLength} 0-120
        }
	    \brAttr{country}{pais}{tVarchar}{%
	        .\par

            \it Restricciones:
            \refElem{tRequired},
            \refElem{tLength} 2
        }

    \end{cdtEntidad}\schemeName{user}

    \begin{cdtEntidad}{mdl-course}{Curso de moodle}{%
    Es una tabla del núcleo de moodle que contiene la información principal de cada 
    curso registrado en moodle. Esta entidad contiene 31 atributos, a continuación se
    detallan los atributos relevantes para la especificación de este proyecto.}

	    \brAttr{id}{Id}{tInt}{%
	        Es el dígito que representa al identificador único para cada uno
            de los cursos en moodle.\par

            \it Restricciones:
            \refElem{tPrimaryKey},
            \refElem{tAutoIncrement}.
        }

	    \brAttr{format}{formato}{tVarchar}{%
	        Es el dígito que representa al identificador único para cada uno
            de los cursos en moodle.\par

            \it Restricciones:
            \refElem{tRequired}.
            \refElem{tDefault} topics,
            \refElem{tLength} 0-21.
        }

	    \brAttr{fullname}{nombre completo}{tVarchar}{%
	        Es el nombre completo que se le asigna al curso.\par

            \it Restricciones:
            \refElem{tRequired}.
            \refElem{tLength} 0-21.
        }

	    \brAttr{shortname}{nombre corto}{tVarchar}{%
            Es el nombre corto que se le asigna al curso.\par

            \it Restricciones:
            \refElem{tRequired}.
            \refElem{tLength} 0-21.
        }

    \end{cdtEntidad}\schemeName{course}

    \begin{cdtEntidad}{mdl-course-sections}{Secciones del curso de moodle}{%
    }
	    \brAttr{id}{Id}{tInt}{%
	        Es el dígito que representa al identificador único para cada seccion
            de los cursos en moodle.\par

            \it Restricciones:
            \refElem{tPrimaryKey},
            \refElem{tAutoIncrement}.
        }
    \end{cdtEntidad}\schemeName{course\_sections}

    \begin{cdtEntidad}{mdl-course-format-options}{Opciones del formato del curso}{%
    }

	    \brAttr{id}{Id}{tInt}{%
	        Es el dígito que representa al identificador único para cada uno
            de los cursos en moodle.\par

            \it Restricciones:
            \refElem{tPrimaryKey},
            \refElem{tAutoIncrement}.
        }

	    \brAttr{courseid}{Id}{tInt}{%
	        Es el dígito que representa al identificador único para cada uno
            de los cursos en moodle.\par

            \it Restricciones:
            \refElem{tForeignKey},
            \refElem{tRequired}
        }

	    \brAttr{format}{formato}{tVarchar}{%
	        Es el dígito que representa al identificador único para cada uno
            de los cursos en moodle.\par

            \it Restricciones:
            \refElem{tRequired}.
            \refElem{tDefault} topics,
            \refElem{tLength} 0-21.
        }

	    \brAttr{name}{opcion}{tVarchar}{%
	        Es el dígito que representa al identificador único para cada uno
            de los cursos en moodle.\par

            \it Restricciones:
            \refElem{tPrimaryKey},
            \refElem{tLength} 0-100
        }

	    \brAttr{value}{valor}{tVarchar}{%
	        Es el dígito que representa al identificador único para cada uno
            de los cursos en moodle.\par

            \it Restricciones:
            \refElem{tRequired}
        }

    \end{cdtEntidad}\schemeName{course\_format\_options}

    \begin{cdtEntidad}{mdl-course-category}{Categoria de curso}{%
    .}
    \end{cdtEntidad}\schemeName{course\_category}

    \begin{cdtEntidad}{Plugin}{Plugin}{%
    La forma en que moodle obtiene información acerca de los plugins es analizando
    los archivos internos de cada uno, a pesar de que los plugins no forman parte
    del esquema de base de datos, si forman parte del modelo de información que
    utiliza Moodle.}

	    \brAttr{componente}{Componente}{tVarchar}{%
	        Cadena compuesta por el tipo de plugin y el nombre del mismo, que
            representa a la clase principal del plugin que contiene los métodos
            principales del plugin.\par

            \it Restricciones: Ninguna
        }

	    \brAttr{pluginname}{Nombre}{tVarchar}{%
	        Es el nombre del plugin obtenido de los archivos de
            internacionalización presentes en el plugin, el valor de esta cadena
            depende del lenguaje seleccionado en moodle.\par

            \it Restricciones: Ninguna
        }

	    \brAttr{fullpath}{Ruta absoluta}{tPath}{%
	        La ruta absoluta de un plugin denota la ubicación del plugin en el
            sistema de archivos, esta ruta está compuesta por la ruta absoluta
            de la instalación de moodle, la carpeta correspondiente al tipo del
            plugin y el nombre del plugin.\par

            \it Restricciones: Formato ``/path/to/moodle/plugintype/pluginname''
        }

	    \brAttr{path}{Ruta relativa}{tPath}{%
	        La ruta relativa denota la ubicación del plugin dentro de la carpeta 
            donde se encuentran los archivos de moodle, esta ruta está compuesta
            por la carpeta correspondiente al tipo del plugin y el nombre del
            plugin.\par

            \it Restricciones: Formato ``plugintype/pluginname''
        }

	    \brAttr{version}{Versión}{tVersion}{%
	        Numero entero de longitud de 10 dígitos que representa la versión del 
            plugin.\par

            \it Restricciones: Ninguna adicional al tipo de dato
        }

	    \brAttr{moodle}{Versión de Moodle}{tVersion}{%
	        Número entero de longitud de 10 dígitos que representa la versión de 
            moodle en la que se puede instalar el plugin.\par

            \it Restricciones: Ninguna adicional al tipo de dato
        }

        \brAttr{dependencies}{Dependencias}{tObject}{%
            Objeto que almacena un conjunto de claves con sus respectivos valores, 
            donde cada clave representa el nombre del componente del plugin y el valor 
            es la \refElem{Plugin.version} requerida del mismo.

            \it Restricciones: Ninguna
        }

        \brAttr{icon}{ícono}{tImage}{%
            Imagen para el ícono del plugin, debe estar contenida en el directorio
            {\it pix/} del plugin y tener como nombre {\it icon.png} o {\it icon.svg},
            moodle recomienda tener ambos archivos por si los navegadores no soportan
            algun tipo de archivo \cite{moodlePluginfiles}.\par 

            \it Restricciones: El nombre debe ser icon con extensiones png o svg
        }

    \end{cdtEntidad}


\begin{comment}
\section{Submódulos}


%\begin{comment} % TERMINOS Y EJEMPLOS DE EXPERIENCIAi
\subsection{Esquema de Experiencia}

 Es la especificación de los conceptos relacionados con los puntos de experiencia, cuales
 son los tipos de incremento y cómo se usan y cuales restricciones se aplican para la
 implementación de los puntos de experiencia, niveles y la acción subir de nivel.
 % y cuántos niveles hay.\\

\subsubsection{Conceptos de puntos de experiencia}
\noindent Como se especificó en el marco teórico, los puntos de experiencia son una unidad que representa la cantidad de actividades completadas por un usuario, sin embargo, se puede referir en diversos contextos a estos puntos. Es por ello que se utilizarán conceptos definidos por los cuadros  \hyperref[table:METerminosExperiencia1]{6.1} y  \hyperref[table:METerminosExperiencia2]{6.2} .\\

\noindent Para poder explicar mejor los conceptos que se tienen para los puntos de experiencia, se utilizará un ejemplo y cada concepto en los cuadros  \hyperref[table:METerminosExperiencia1]{6.1} y  \hyperref[table:METerminosExperiencia2]{6.2} , tendrá su relación con este ejemplo.\\

\noindent Se tiene un usuario ''\textit{U}'', y dicho usuario está actualmente en el nivel ''5'' con ''1000'' puntos de experiencia, y además los puntos de experiencia relacionados con los niveles son los siguientes:
\begin{itemize}
    \item Estando en el nivel 1 se necesitan 1000 puntos de experiencia para subir al nivel 2.
    \item Estando en el nivel 2 se necesitan 1250 puntos de experiencia para subir al nivel 3.
    \item Estando en el nivel 3 se necesitan 1500 puntos de experiencia para subir al nivel 4.
    \item Estando en el nivel 4 se necesitan 1750 puntos de experiencia para subir al nivel 5.
    \item Estando en el nivel 5 se necesitan 2000 puntos de experiencia para subir al nivel 6.
\end{itemize}

\noindent Agreguemos que existe una actividad ''\textit{A}'' que al ser completada otorga 1050 puntos de experiencia.\\

\noindent Con el ejemplo anterior, podemos proceder a definir nuestros conceptos y relacionarlos con el ejemplo para que quede todo más claro.

\begin{table}[h!]
    \label{table:METerminosExperiencia1}
    \centering
        \begin{tabular}{|m{0.2 \textwidth}|m{0.32 \textwidth}|m { 0.42\textwidth}|}\hline
        \textbf{Nombre} & \textbf{Definición} & \textbf{Ejemplo} \\\hline

        Experiencia actual  &
        Es la cantidad de puntos de experiencia que un usuario ha conseguido mientras tiene asociado un cierto nivel. &
        ''\textit{U}'' tiene una \textbf{experiencial actual} de 1000 puntos de experiencia. Y el nivel al cual está asociado es el nivel 5.
        \\\hline

        Experiencia del nivel &
        Es la cantidad de puntos de experiencia que se requieren para subir de un nivel ''\textit{A}'' a un nivel ''\textit{B}'', sin contemplar la \textbf{experiencia actual} del usuario.&
        Con nuestro ejemplo la \textbf{experiencia del nivel} 1, es 1000 mientras que la del nivel 5 es 2000. \\\hline

        Porcentaje actual &
        Es un número entero positivo con rango del 1 al 100, que se calcula de la forma $ \frac{experiencia\_actual}{experiencia\_del\_nivel} * 100 $ &
        ''\textit{U}'' actualmente está en el nivel 5, por lo tanto el cálculo sería: $ \frac{1000}{2000} * 100 = 50 $
        \\\hline

        Experiencia otorgada &
        Es la cantidad de puntos de experiencia que se otorgan al completar una actividad. &
        La actividad ''\textit{A}'' tiene una \textbf{experiencia otorgada} de 1050.
        \\\hline

        Experiencia acumulada &
        Es la cantidad de puntos de experiencia que un usuario ha conseguido a través de los niveles. &
        ''\textit{U}'' está actualmente en el nivel 5, esto quiere decir que ha pasado por los niveles 1, 2, 3 y 4. Cada uno de estos últimos tiene su \textbf{experiencia del nivel}, por lo tanto, nuestro usuario ''\textit{U}'' a conseguido $ 1000 + 1250 + 1500 + 1750 $ puntos para llegar al nivel 5. Además, ''\textit{U}'' tiene una \textbf{experiencia actual} de 1000 puntos.\newline

        Sumando todo lo anterior $ 1000 + 1250 + 1500 + 1750 + 1000  = 6500 $, por lo tanto el usuario ''\textit{U}'' tiene una \textbf{experiencia acumulada} de 6500 \\\hline

        \end{tabular}
    \caption{Conceptos referentes a los puntos de experiencia (parte 1).}
\end{table}
\clearpage
\begin{table}[h!]
    \label{table:METerminosExperiencia2}
    \centering
        \begin{tabular}{|m{0.2 \textwidth}|m{0.32 \textwidth}|m { 0.42\textwidth}|}\hline
        \textbf{Nombre} & \textbf{Definición} & \textbf{Ejemplo} \\\hline

        Experiencia necesaria &
        Es la cantidad de puntos de experiencia que un usuario necesita para que su \textbf{experiencia actual} alcance la \textbf{experiencia del nivel}. Expresado de forma matemática:
        \begin{center}
            experiencia del nivel\newline
            \underline{ - experiencia actual} \newline
            experiencia necesaria \newline
        \end{center}&

        Para  ''\textit{U}'' su \textbf{experiencia necesaria} sería.

        \begin{center}
              2000\newline
            \underline{ - 1000} \newline
            1000\newline
        \end{center}
        \\\hline

        Experiencia sobrante &
        Es la cantidad de puntos de experiencia que rebasan la \textbf{experiencia del nivel} cuando a un usuario recibe \textbf{experiencia otorgada}. Expresado de forma matemática:
        \begin{center}
            experiencia otorgada\newline
            + experiencia actual\newline
            \underline{ - experiencia del nivel} \newline
            experiencia sobrante \newline
        \end{center}&
        Si el estudiante ''U'' realiza la actividad ''A'', recibiría una \textbf{experiencia otorgada} de 1050 , sin embargo, su \textbf{experiencia actual} es de 1000. Por lo tanto, tendría una \textbf{experiencia sobrante} de 50 puntos de experiencia.

        \begin{center}
            1050\newline
            + 1000\newline
            \underline{ - 2000} \newline
            50 \newline
        \end{center}\\\hline
        \end{tabular}
    \caption{Conceptos referentes a los puntos de experiencia (parte 2).}
\end{table}

\noindent A lo largo de este capítulo se utilizaran los conceptos anteriores.\\
%\end{comment}

\subsubsection{Tipos de incremento}

 Un tipo de incremento define cómo se calcula la diferencia entre los valores de
 ''\hyperref[table:METerminosExperiencia1]{experiencia del nivel}'' de un nivel
 \textit{$n_i$} y el siguiente a él \textit{$n_(i+1)$}.\\

 \noindent
 Se da soporte a los siguientes tipos de incremento entre los niveles:

    \begin{quote}
    \begin{description}
        \item[Lineal] Establece que la diferencia es una cantidad fija.\\
        Sea $f(n_i)$ una función la cual indica la ''\hyperref[table:METerminosExperiencia1]{experiencia del nivel}'' de un nivel $n_i$. Y sea $e$ una constante que representa la diferencia de la ''\hyperref[table:METerminosExperiencia1]{experiencia del nivel}'' entre 2 niveles continuos. Entonces.
            $$\forall n_i \in Nivel\ | \left(\ f(n_{i+1}) - f(n_i)\ \right) = e$$

        \item[Porcentual] Establece que la diferencia está regida por la siguiente regla:\\
        Sea $n_i$ un nivel de experiencia,  $f(n_i)$ una función la cual indica la ''\hyperref[table:METerminosExperiencia1]{experiencia del nivel}'' de un nivel $n_i$ y  $c$ una constante tal que $1 \leq c \leq 2$ , entonces
            $$\forall n_i \in Nivel\ |\ (c)f(n_i) = f(n_{i+1}).$$
    \end{description}
    \end{quote}


%\begin{comment}
    \begin{quote}
    \begin{description}
        \item[Lineal] Establece que la diferencia es una cantidad fija.\\
        Sea $f(n_i)$ una función la cual indica la cantidad de experiencia requerida para subir al nivel $n_i$. y sea $e$ una constante que representa el incremento de la cantidad de experiencia requerida entre niveles. Entonces.
            $$\forall n_i \in Nivel\ | \left(\ f(n_{i+1}) - f(n_i)\ \right) = e$$

        \item[Porcentual] Establece que la diferencia entre cantidad necesaria de experiencia para subir del nivel $i$ al nivel $i+1$ está regido por la siguiente regla:\\
        Sea $n_i$ un nivel de experiencia y $n_{i+1}$ el siguiente nivel, y sea $c$ una constante tal que $1 \leq c \leq 2$ , entonces
            $$\forall n_i \in Nivel\ |\ (c)f(n_i) = f(n_{i+1}).$$
    \end{description}
    \end{quote}
%\end{comment}

\subsubsection{Esquema configurable}

 Se quiere que el administrador de la página pueda configurar:
 \begin{quote}
 \begin{itemize}
    \item{La ''\hyperref[table:METerminosExperiencia1]{experiencia del nivel}'' del nivel 1.}
    \item {El tipo de incremento.}
    \item {La cantidad de los puntos de experiencia en el incremento.}
    \item {La ''\hyperref[table:METerminosExperiencia1]{experiencia otorgada}'' que da resolver cualquier actividad.}
 \end{itemize}
 \end{quote}

\subsection{Submódulo de Niveles}

 Presenta a los estudiantes su progreso utilizando un sistema de niveles que se van alcanzado
 obteniendo puntos de experiencia. Al alcanzar un nuevo nivel la barra que muestra la
 cantidad de experiencia del nivel se actualizará.
 % y cada vez que se alcanza un nivel, los puntos de experiencia se regresan a cero.

    \begin{quote}
    \begin{description}
    \item[Objetivo] \hfill\\
        Mostrar a los estudiantes el nivel actual de experiencia que tienen y el avance que tienen de ese mismo nivel.
        %Mostrar el nivel actual que tienen los estudiantes, así como el avance que tienen en ese mismo nivel.

        %Proveer información al estudiante que indique la cantidad de tiempo y esfuerzo que le ha dedicado a la plataforma.

    \item[Principios a los que brinda soporte:] \hfill
        \begin{itemize}
            \item 2 \principioII
            \item 6 \principioVI
        \end{itemize}
    \end{description}
    \end{quote}

%\begin{comment}%

\subsection{Submódulo de Barra de Progreso}

Muestra al estudiante el progreso que lleva en un curso usando un valor de 0\% a 100\% dependiendo de los ejercicios que haya hecho del curso o del tiempo que haya transcurrido.

    \begin{quote}
    \begin{description}
        \item[Objetivo] \hfill\\
            Proveer información al estudiante que indique el tiempo y esfuerzo que le ha dedicado a un curso, así como el que le falta por dedicar.

        \item[Principios a los que brinda soporte:] \hfill
        \begin{itemize}
            \item 2 \principioII
        \end{itemize}
    \end{description}
    \end{quote}
%\end{comment}%

%\subsection{Comportamiento en Moodle} ESTO ES DISEÑO

%\subsubsection{Plugin}
%\subsubsection{Opciones para el administrador}

\clearpage
%\subsection{Reglas de Uso} % Reglas de Negocio | But there's no business

\section{Interfaces}

\subsection*{IU-E01 Bloque de experiencia}
\label{IUE01}

    Visualización en Moodle del bloque de experiencia.

%    \addfigure{1}{IU/IU_E01_Bloque_Experiencia}{fig:IUE01}{IU-E01: Bloque de experiencia.}

    \noindent {\bf Elementos:}
    \begin{quote}
    \begin{description}
    	\item[Nombre del bloque] Nombre para diferenciar los bloques en las interfaces de Moodle.
    	\item[Imagen del nivel] Imagen del nivel configurada por el administrador.
    	\item[Número del nivel] Número entero positivo que representa el nivel actual del usuario.
    	\item[Barra de progreso] Barra que se llena según el ''\hyperref[table:METerminosExperiencia1]{porcentaje actual}'' del usuario
    	\item[Experiencia actual del nivel] Número entero positivo que representa la ''\hyperref[table:METerminosExperiencia1]{experiencia actual}'' del usuario.
    	\item[Experiencia total del nivel] Número entero positivo que representa la ''\hyperref[table:METerminosExperiencia1]{experiencia del nivel}''.
    	\item[Experiencia acumulada] Número entero positivo que representa la cantidad de ''\hyperref[table:METerminosExperiencia1]{experiencia acumulada}''.
    \end{description}
    \end{quote}
	\clearpage

\subsection*{IU-E02 Subir de nivel}
\label{IUE02}

    Esta interfaz es una ventana emergente que sale siempre que el usuario tenga ''\hyperref[table:METerminosExperiencia1]{experiencia sobrante}'' al ejecutar el \hyperref[CU-E01]{CU-E01 Recibir experiencia}.

%    \addfigure{1}{IU/IU_E02_PopUp_SubirNivel}{fig:IUE02}{IU-E02: Ventana emergente que aparece al subir de nivel.}

    \noindent {\bf Elementos:}
    \begin{quote}
    \begin{description}
    	\item[Mensaje] Mensaje de felicitaciones configurado por el administrador.
    	\item[Imagen del nivel] Imagen del nivel configurada por el administrador.
    	\item[Número del nivel] Número entero positivo que representa el nivel actual del usuario.
    	\item[Nombre] Nombre que reciben los niveles, que es configurado por el administrador.
    	\item[Descripción] Descripción asociada a los niveles, que es configurada por el administrador.
    \end{description}
    \end{quote}
	\clearpage

\subsection*{IU-E03 Configuración del esquema de experiencia}
\label{IUE03}
    Esta interfaz es definida por el archivo \textbf{settings.php}, sin embargo, está es generada por Moodle.\\
    En esta interfaz el administrador puede modificar los aspectos visuales que tienen las interfaces \hyperref[IUE01]{IU-E01 Bloque de experiencia} y \hyperref[IUE02]{IU-E02 Subir de nivel} , y configurar el tipo de incremento, cuanto incremento hay por nivel, la ''\hyperref[table:METerminosExperiencia1]{experiencia del nivel}'' del nivel 1 y la ''\hyperref[table:METerminosExperiencia1]{experiencia otorgada}'' que darán todas las actividades.\\
    Esta interfaz también activa y desactiva el funcionamiento del módulo de experiencia.\\

    %\addfigure{1}{IU/IU_E03_config_parte1}{fig:IUE03_1}{Interfaz donde se  configura el esquema de experiencia parte 1.}


    \noindent {\bf Elementos importantes:}
    \begin{quote}
    \begin{description}
    	\item[Opción \#1] Permite activar y desactivar el módulo de experiencia.
    	\item[Opción \#2] Permite seleccionar entre los 2 tipos de incremento 'Lineal' y 'Porcentual'.
    	\item[Opción \#3] Permite asignar cuanta experiencia habrá de diferencia entre la ''\hyperref[table:METerminosExperiencia1]{experiencia del nivel}'' del nivel \textit{A} y la ''\hyperref[table:METerminosExperiencia1]{experiencia del nivel}'' del nivel \textit{B}.
    	\item[Opción \#4] Permite asignar la ''\hyperref[table:METerminosExperiencia1]{experiencia del nivel}'' 1.
    	\item[Opción \#5] Permite asignar la ''\hyperref[table:METerminosExperiencia1]{experiencia otorgada}'' que entregarán todas las actividades.
    \end{description}
    \end{quote}

\clearpage
    %\addfigure{1}{IU/IU_E03_config_parte2}{fig:IUE03_2}{Interfaz donde se  configura el esquema de experiencia parte 2.}


    \noindent {\bf Elementos importantes:}
    \begin{quote}
    \begin{description}
    	\item[Opción \#6] Permite asignar un nombre por defecto a los niveles.
    	\item[Opción \#7] Permite asignar un mensaje de felicitaciones por defecto al subir de nivel.
    	\item[Opción \#8] Permite asignar una descripción por defecto al subir de nivel.
    	\item[Opción \#9] Permite asignar el color por defecto del número de nivel.
    \end{description}
    \end{quote}


\clearpage

    %\addfigure{1}{IU/IU_E03_config_parte3}{fig:IUE03_3}{Interfaz donde se  configura el esquema de experiencia parte 3.}

    \noindent {\bf Elementos importantes:}
    \begin{quote}
    \begin{description}
    	\item[Opción \#10] Permite asignar el color por defecto de la barra de progreso.
    	\item[Opción \#11] Permite asignar una imagen por defecto a los niveles.
    	\item[Botón \#1 'Guardar cambios'] Con este botón el administrador puede guardar los cambios que haya hecho en las opciones de la interfaz.
    \end{description}
    \end{quote}

\clearpage

\subsection*{IU-M01 Ver intento de examen}
\label{IUM01}

    Esta interfaz es de Moodle, sin embargo, es utilizada para el caso de uso \hyperref[CU-E01]{CU-E01 Recibir experiencia}.\\

    \noindent En esta interfaz los estudiantes pueden ver un resumen de su intento para resolver el examen, así como poder reanudar el intento para terminar o corregir, o para enviar las respuestas para su revisión.

    %\addfigure{1}{IU/IU_M01_VerIntento}{fig:IUM01}{IU-M01: Ver intento.}

    \noindent {\bf Elementos importantes:}
    \begin{quote}
    \begin{description}
    	\item[Botón \# 1 'Regresar al intento'] Con este botón el usuario puede seguir contestando el examen.
    	\item[Botón \# 2 'Enviar todo y terminar'] Con este botón el usuario indica que terminó de responder el examen y que quiere enviar las respuestas para su revisión.
    \end{description}
    \end{quote}
	\clearpage

\subsection*{IU-M02 Confirmación de envío de intento}
\label{IUM02}

    Esta interfaz es de Moodle, sin embargo, es utilizada para el caso de uso  \hyperref[CU-E01]{CU-E01 Recibir experiencia}.\\

    \noindent Esta interfaz es una ventana emergente que permite al usuario pensárselo una segunda vez antes de subir sus respuestas para ser calificadas. Esto inclusive por si le da por error.\\

    %\addfigure{1}{IU/IU_M02_PopUp_Confirmacion}{fig:IUM02}{IU-M02: Confirmación de envío de intento.}

    \noindent {\bf Elementos importantes:}
    \begin{quote}
    \begin{description}
    	\item[Botón \#1 'Enviar todo y terminar'] Con este botón el usuario puede reafirmar que quiere mandar sus respuestas para ser calificadas.
    	\item[Botón \#2 'Cancelar'] Con este botón el usuario puede cancelar el subir sus respuestas.
    	\item[Botón \#3 'X'] Este botón tiene el mismo efecto que el Botón \# 2 'Cancelar'.
    \end{description}
    \end{quote}
	\clearpage

\subsection*{IU-M03 Intento calificado}
\label{IUM03}

    Esta interfaz es de Moodle, sin embargo, es utilizada para el caso de uso   \hyperref[CU-E01]{CU-E01 Recibir experiencia}.\\

    \noindent Si un usuario envía su intento para ser calificado
    %y dicho intento puede ser calificado por el sistema,
    se muestra esta interfaz con la calificación de su intento. Algo importante es que la interfaz  \textbf{IU-E01 Bloque de experiencia} está visible, haciendo posible mostrar la actualización de los datos del usuario.\\

    %\addfigure{1}{IU/IU_M03_FinalizarIntento}{fig:IUM03}{IU-M03: Intento calificado.}


	\clearpage

\subsection*{IU-M04 Sección de plugins}
\label{IUM04}

    Esta interfaz es de Moodle, sin embargo, es utilizada para el caso de uso \hyperref[CU-E02]{CU-E02 Configuración esquema de experiencia} .\\

    \noindent Esta interfaz contiene la lista de todos los plugins que tienen configuraciones globales, donde cada uno de sus elementos es un enlace a la página de configuración respectiva a cada plugin. Además en su primera sección tiene las opciones para ver, manejar e instalar plugins.\\

%    \addfigure{1}{IU/IU_M04_SeccionPlugins}{fig:IUM04}{IU-M04: Sección donde están todos los plugins con configuraciones globales.}
	\clearpage


\subsection*{Moodle: IU-M05 Crear Curso}

 El objetivo de esta pantalla (Figura \ref{moodle:nuevoCurso}) es permitirle al profesor crear un nuevo curso, especificando los datos generales del curso, la descripción, apariencia, tamaño de los archivos, el seguimiento de finalización, los grupos, renombre de roles y las marcas vinculadas al curso.

%    \addfigureB{1}{IU/mCrearCurso}{moodle:nuevoCurso}{Moodle IU-M05 Crear curso}

    {\bf Elementos}
    \begin{itemize}
        \item \bf{Datos Generales} es un formulario que contiene el nombre completo y corto del curso (obligatorios), su identificador, categoría, visibilidad, así como las fechas de inicio y término del curso.
        \item \bf{Descripción}
        \item \bf{Formato de curso}
        \item \bf{Apariencia}
        \item \bf{Archivos y subidas}
        \item \bf{Seguimiento de finalización}
        \item \bf{Grupos}
        \item \bf{Renombrar rol}
        \item \bf{Marcas}
    \end{itemize}


\chapter{Diseño}

\subsection{Diagrama de Clases}

    En la figura \ref{fig:classesXP} se muestra el diagrama de clases, los archivos {\it lib, events, settings, version} y los {\it módulos AMD} son representados mediante el uso de clases. Para facilitar la lectura del diagrama se representa a moodle como un paquete completo, el cual lee los distintos archivos y clases que requiere el plugin para funcionar.

%    \addfigure{1}{diagrams/classesExp}{fig:classesXP}{Diagrama de clases del Módulo de Experiencia}
\clearpage

\subsection{Diagrama de componentes}

    En la figura \ref{fig:bloques1} se muestra el diagrama de componentes del Módulo de experiencia que contiene como interactúa el Módulo con la plataforma Moodle.

%    \addfigure{1}{diagrams/bloques1}{fig:bloques1}{Diagrama de componentes del Módulo de Experiencia}

\clearpage
\subsection{Diagramas de Secuencia}
\subsection*{DS-E2: Crear curso con experiencia}

    Para diseñar la forma en que se ejecuta el caso de uso CU-E2, se tomó en consideración el flujo normal de eventos emitidos cuando se crea un curso en moodle. Los eventos emitidos en orden cronológico son {\it course\_created}, {\it course\_section\_created} y {\it enrol\_instance\_created}.\\

    \noindent En la figura \ref{ds:e2} se detalla la interacción entre el core de moodle, los eventos emitidos, y las clases del plugin {\bf Format Gamedle}.

\chapter{Pruebas}
\end{comment}
