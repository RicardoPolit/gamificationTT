
\subsection{Análisis}

 Este apartado contiene el análisis requerido para la elaboración de módulo de competencia,
 contiene la especificación del alcance de este módulo, la descripción de las funcionalidades
 a desarrollar, la reglas de negocio que rigen el comportamiento del módulo, y por último la
 especificación de los casos de uso a los que brinda soporte.

%\subsubsection{Submódulo de competencia 1 contra 1}
%\subsubsection{Funcionalidades}

\subsubsection{Reglas de negocio} %========================================================

 En esta sección se especifican todas las reglas de negocio relevantes para el módulo de
 experiencia. Las reglas de negocio que establece moodle son diferenciadas por tener la letra {\it M}
 antecediendo al número consecutivo en su identificador.

    %
\subsection{Entidades de moodle}

Debido a que moodle cuenta con más de 400 tablas en su versión 3.5, se opta
por mostrar 2 subconjuntos que muestren las tablas que se utilizan para el proyecto.\\

\noindent El primer subconjunto es aquel que explica la forma en que moodle implementa los cursos, 
secciones de curso, actividades, usuarios y roles (el cual se presenta en la figura \ref{fig:BD-ER-M1}), 
mientras que el segundo conjunto muestra como moodle maneja toda la 
estructura de las preguntas creadas por el profesor y respondidas por el usuario
(el cual se presenta en la figura \ref{fig:BD-ER-M2}).  



En lugar de describir y mostrar cada uno de los campos de cada una de las entidades de moodle que se contemplan,
lo que se quiere lograr con ambos esquemas (\ref{fig:BD-ER-M1}) y \ref{fig:BD-ER-M2}))
es expresar la idea general del comportamiento.

\clearpage
\addfigure{0.7}{analisis/diagrams/db_module_structure}{fig:BD-ER-M1}{Esquema de la base de datos de moodle 'Cursos'}


\noindent Utilizando la figura \ref{fig:BD-ER-M1}, se obtuvieron las siguientes reglas y caracteristicas que contiene moodle respecto a los usuarios en un curso y a la estructura de los cursos.
\begin{enumerate}
    \item Un usuario -{\it mdl\_user}- tiene un rol -{\it mdl\_role}- en un cierto contexto -{\it mdl\_context}-, cuyo  '{\it context\_level}' sea igual a cincuenta(50).
    \item Si el contexto '{\it context\_level}' es de 50, el atributo '{\it instance\_id}' hace referencia al atributo '{\it id}' de un curso -{\it mdl\_course}-.
    \item El curso -{\it mdl\_course}- tiene varias secciones -{\it mdl\_course\_sections}-.
    \item Cada seccion -{\it mdl\_course\_sections}- tiene varias actividades -{\it mdl\_course\_modules}- que pertenecen a un tipo de actividad -{\it mdl\_modules}-.
    \item Por cada registro en tipo de actividad -{\it mdl\_modules}-, se tiene una entidad que lleva el mismo nombre.
    \item El atributo '{\it instance\_id}' de una actividad  -{\it mdl\_course\_modules}- apunta a diferentes entidades. La entidad a la que apunta depende del nombre del tipo de actividad -{\it mdl\_modules}-.
    \item Un usuario -{\it mdl\_user}- se inscribió -{\it mdl\_user\_enrolments}- a un curso -{\it mdl\_course}-, por medio de un formato soportado de inscripción -{\it mdl\_enrol}-.
\end{enumerate}

\clearpage

 \addfigure{0.7}{analisis/diagrams/db_module_questions}{fig:BD-ER-M2}{Esquema de la base de datos de moodle 'Preguntas' }



\noindent Utilizando la figura \ref{fig:BD-ER-M2}, se obtuvieron las siguientes reglas y caracteristicas que contiene moodle respecto a las preguntas.
\begin{enumerate}
    \item Las preguntas -{\it mdl\_question}- tienen versiones -{\it mdl\_question\_attempts}-.
    \item Una pregunta -{\it mdl\_question}- pertenece a un banco de preguntas -{\it mdl\_question\_categories}-.
    \item La versión de una pregunta -{\it mdl\_question\_attempts}- es contestada -{\it mdl\_question\_usages}- en un determinado contexto -{\it mdl\_context}-.
    \item Un usuario -{\it mdl\_user}- responde una versión de una pregunta -{\it mdl\_question\_attempt\_stepts}-.
    \item El responder una versión de una pregunta -{\it mdl\_question\_attempt\_stepts}- conlleva pasos -{\it mdl\_question\_attempt\_stept\_data}-, los cuales son: cómo se muestra, si ya se terminó de responder y qué se respondió.
\end{enumerate}


 A continuación se presenta la especificación de las entidades del esquema de base
 de datos de moodle que son relevantes para el desarrollo de los módulos y submódulos
 de proyecto.

    \begin{cdtEntidad}{mdl-config-plugins}{Configuración de Plugin}{%
    Es una tabla del núcleo de moodle que almacena todas las configuraciones globales
    relacionadas a los plugins instalados, al iniciar moodle las configuraciones de los
    plugins instalados y habilitados se cargan en memoria.}

	    \brAttr{id}{Id}{tInt}{%
	        Es el dígito que representa el identificador único para una configuración
            específica de un plugin.\par

            \it Restricciones:
            \refElem{tPrimaryKey},
            \refElem{tAutoIncrement}.
        }

        \brAttr{plugin}{Plugin}{tVarchar}{%
            Cadena de caracteres del nombre identificador del plugin al cual pertenece
            la configuración.\par

            \it Restricciones:
            \refElem{tRequired},
            \refElem{tRange} (0,100),
            \refElem{tUniqueKey}
        }

        \brAttr{name}{Nombre}{tVarchar}{%
            Cadena de caracteres que representa el nombre de la configuración de un
            plugin en específico.\par

            \it Restricciones:
            \refElem{tUniqueKey},
            \refElem{tRange} (0,100),
            \refElem{tRequired}
        }

        \brAttr{value}{Valor}{tVarchar}{%
            Cadena que almacena el valor de una configuración perteneciente a alguno
            de los plugins instalados.\par

            \it Restricciones:
            \refElem{tRange} (0,4294967295),
            \refElem{tRequired}
        }
    \end{cdtEntidad}\schemeName{config\_plugins}

    \begin{cdtEntidad}{mdl-user}{Usuario de moodle}{%
    Es una tabla del núcleo de moodle que contiene toda la información que se
    almacena de los usuarios en la plataforma, independientemente del rol que
    estos contenga, esta relación contiene más de 53 atributos, sin embargo solo
    se detallan aquellos relevantes.}

	    \brAttr{id}{Id}{tInt}{%
	        Es el dígito que representa el identificador único para cada uno
            de los usuarios en moodle.\par

            \it Restricciones:
            \refElem{tPrimaryKey},
            \refElem{tAutoIncrement}.
        }
	    \brAttr{username}{nombre de usuario}{tVarchar}{%
	        .\par

            \it Restricciones:
            \refElem{tRequired},
            \refElem{tLength} 0-100
        }
	    \brAttr{password}{contraseña}{tVarchar}{%
	        .\par

            \it Restricciones:
            \refElem{tRequired},
            \refElem{tLength} 0-255.
        }
	    \brAttr{firstname}{nombre}{tVarchar}{%
	        .\par

            \it Restricciones:
            \refElem{tRequired},
            \refElem{tLength} 0-100
        }
	    \brAttr{lastname}{apellido}{tVarchar}{%
	        .\par

            \it Restricciones:
            \refElem{tRequired},
            \refElem{tLength} 0-100
        }
	    \brAttr{email}{correo}{tVarchar}{%
	        .\par

            \it Restricciones:
            \refElem{tRequired},
            \refElem{tLength} 0-100
        }
	    \brAttr{lastaccess}{último registro}{tInt}{%
	        .\par

            \it Restricciones:
            \refElem{tRequired},
            \refElem{tLength} 10
        }
	    \brAttr{city}{ciudad}{tVarchar}{%
	        .\par

            \it Restricciones:
            \refElem{tRequired},
            \refElem{tLength} 0-120
        }
	    \brAttr{country}{pais}{tVarchar}{%
	        .\par

            \it Restricciones:
            \refElem{tRequired},
            \refElem{tLength} 2
        }

    \end{cdtEntidad}\schemeName{user}

    \begin{cdtEntidad}{mdl-course}{Curso de moodle}{%
    Es una tabla del núcleo de moodle que contiene la información principal de cada 
    curso registrado en moodle. Esta entidad contiene 31 atributos, a continuación se
    detallan los atributos relevantes para la especificación de este proyecto.}

	    \brAttr{id}{Id}{tInt}{%
	        Es el dígito que representa al identificador único para cada uno
            de los cursos en moodle.\par

            \it Restricciones:
            \refElem{tPrimaryKey},
            \refElem{tAutoIncrement}.
        }

	    \brAttr{format}{formato}{tVarchar}{%
	        Es el dígito que representa al identificador único para cada uno
            de los cursos en moodle.\par

            \it Restricciones:
            \refElem{tRequired}.
            \refElem{tDefault} topics,
            \refElem{tLength} 0-21.
        }

	    \brAttr{fullname}{nombre completo}{tVarchar}{%
	        Es el nombre completo que se le asigna al curso.\par

            \it Restricciones:
            \refElem{tRequired}.
            \refElem{tLength} 0-21.
        }

	    \brAttr{shortname}{nombre corto}{tVarchar}{%
            Es el nombre corto que se le asigna al curso.\par

            \it Restricciones:
            \refElem{tRequired}.
            \refElem{tLength} 0-21.
        }

    \end{cdtEntidad}\schemeName{course}

    \begin{cdtEntidad}{mdl-course-sections}{Secciones del curso de moodle}{%
    }
	    \brAttr{id}{Id}{tInt}{%
	        Es el dígito que representa al identificador único para cada seccion
            de los cursos en moodle.\par

            \it Restricciones:
            \refElem{tPrimaryKey},
            \refElem{tAutoIncrement}.
        }
    \end{cdtEntidad}\schemeName{course\_sections}

    \begin{cdtEntidad}{mdl-course-format-options}{Opciones del formato del curso}{%
    }

	    \brAttr{id}{Id}{tInt}{%
	        Es el dígito que representa al identificador único para cada uno
            de los cursos en moodle.\par

            \it Restricciones:
            \refElem{tPrimaryKey},
            \refElem{tAutoIncrement}.
        }

	    \brAttr{courseid}{Id}{tInt}{%
	        Es el dígito que representa al identificador único para cada uno
            de los cursos en moodle.\par

            \it Restricciones:
            \refElem{tForeignKey},
            \refElem{tRequired}
        }

	    \brAttr{format}{formato}{tVarchar}{%
	        Es el dígito que representa al identificador único para cada uno
            de los cursos en moodle.\par

            \it Restricciones:
            \refElem{tRequired}.
            \refElem{tDefault} topics,
            \refElem{tLength} 0-21.
        }

	    \brAttr{name}{opcion}{tVarchar}{%
	        Es el dígito que representa al identificador único para cada uno
            de los cursos en moodle.\par

            \it Restricciones:
            \refElem{tPrimaryKey},
            \refElem{tLength} 0-100
        }

	    \brAttr{value}{valor}{tVarchar}{%
	        Es el dígito que representa al identificador único para cada uno
            de los cursos en moodle.\par

            \it Restricciones:
            \refElem{tRequired}
        }

    \end{cdtEntidad}\schemeName{course\_format\_options}

    \begin{cdtEntidad}{mdl-course-category}{Categoria de curso}{%
    .}
    \end{cdtEntidad}\schemeName{course\_category}

    \begin{cdtEntidad}{Plugin}{Plugin}{%
    La forma en que moodle obtiene información acerca de los plugins es analizando
    los archivos internos de cada uno, a pesar de que los plugins no forman parte
    del esquema de base de datos, si forman parte del modelo de información que
    utiliza Moodle.}

	    \brAttr{componente}{Componente}{tVarchar}{%
	        Cadena compuesta por el tipo de plugin y el nombre del mismo, que
            representa a la clase principal del plugin que contiene los métodos
            principales del plugin.\par

            \it Restricciones: Ninguna
        }

	    \brAttr{pluginname}{Nombre}{tVarchar}{%
	        Es el nombre del plugin obtenido de los archivos de
            internacionalización presentes en el plugin, el valor de esta cadena
            depende del lenguaje seleccionado en moodle.\par

            \it Restricciones: Ninguna
        }

	    \brAttr{fullpath}{Ruta absoluta}{tPath}{%
	        La ruta absoluta de un plugin denota la ubicación del plugin en el
            sistema de archivos, esta ruta está compuesta por la ruta absoluta
            de la instalación de moodle, la carpeta correspondiente al tipo del
            plugin y el nombre del plugin.\par

            \it Restricciones: Formato ``/path/to/moodle/plugintype/pluginname''
        }

	    \brAttr{path}{Ruta relativa}{tPath}{%
	        La ruta relativa denota la ubicación del plugin dentro de la carpeta 
            donde se encuentran los archivos de moodle, esta ruta está compuesta
            por la carpeta correspondiente al tipo del plugin y el nombre del
            plugin.\par

            \it Restricciones: Formato ``plugintype/pluginname''
        }

	    \brAttr{version}{Versión}{tVersion}{%
	        Numero entero de longitud de 10 dígitos que representa la versión del 
            plugin.\par

            \it Restricciones: Ninguna adicional al tipo de dato
        }

	    \brAttr{moodle}{Versión de Moodle}{tVersion}{%
	        Número entero de longitud de 10 dígitos que representa la versión de 
            moodle en la que se puede instalar el plugin.\par

            \it Restricciones: Ninguna adicional al tipo de dato
        }

        \brAttr{dependencies}{Dependencias}{tObject}{%
            Objeto que almacena un conjunto de claves con sus respectivos valores, 
            donde cada clave representa el nombre del componente del plugin y el valor 
            es la \refElem{Plugin.version} requerida del mismo.

            \it Restricciones: Ninguna
        }

        \brAttr{icon}{ícono}{tImage}{%
            Imagen para el ícono del plugin, debe estar contenida en el directorio
            {\it pix/} del plugin y tener como nombre {\it icon.png} o {\it icon.svg},
            moodle recomienda tener ambos archivos por si los navegadores no soportan
            algun tipo de archivo \cite{moodlePluginfiles}.\par 

            \it Restricciones: El nombre debe ser icon con extensiones png o svg
        }

    \end{cdtEntidad}
 % Archivo de plugin
    
\begin{BusinessRule}[%
Autor/Ricard Naranjo Polit,%
Version/0.1,%
Estado/revision]%
%
{BR-C01}{Restricciones del tiempo que tiene }
 % El archivo de instalación debe ser un archivo ZIP, el cual debe contener
 % exactamente un directorio que coincida con el nombre del plugin.
     \BRitem[control]{Revisor}{Sin asignar.}

 \BRsection[control]{Atributos}

    \BRitem[admin]{Clase}{\bcCondition}%
    %\BRitem[admin]{Clase}{\bcIntegridad}%
    %\BRitem[admin]{Clase}{\bcAutorizacion}%
    %\BRitem[admin]{Clase}{\bcDerivacion}%

    \BRitem[admin]{Tipo}{\btEnabler}%
    %\BRitem[admin]{Tipo}{\btTimer}%
    %\BRitem[admin]{Tipo}{\btExecutive}%

    \BRitem[admin]{Nivel}{\blControlling}
    %\BRitem[admin]{Nivel}{\blInfluencing}

    \BRitem{Descripción}{%
        El archivo seleccionado para la representación visual de los niveles debe
        ser una imagen con las extensiones {\it``png''} o {\it''jgp}, además el
        nombre del archivo que será subido no debe tener el nombre {\it``icon.png''}
        ya que posiblemente colisionaría con el \refElem{Plugin.icon} del plugin.
        % debido a que se el directorio donde se guardará será el directorio
        % para almacenar las imágenes del plugin.
    }

    \BRitem{Ejemplo positivo}{\hfill\par%
        \begin{itemize}
        \item El archivo seleccionado para ser la imagen de los niveles tiene
              como nombre {\it``logotipo''} con la extensión {\it png}.

        \item El archivo seleccionado para ser la imagen de los niveles tiene
              como nombre {\it``nivel''} con la extensión {\it jpg}
        \end{itemize}
    }

    \BRitem{Ejemplo negativo}{\hfill\par%
        \begin{itemize}
        \item El archivo seleccionado para ser la imagen de los niveles tiene
              como nombre {\it``icon''} con la extensión {\it png}.

        \item El archivo seleccionado para ser la imagen de los niveles tiene
              como nombre {\it``documento''} con la extensión {\it doc}.
        \end{itemize}
    }%

 \end{BusinessRule}
 % Restricciones sobre de imagen del nivel.
    
\begin{BusinessRule}[%
Autor/Ricard Naranjo Polit,%
Version/0.1,%
Estado/revision]%
%
{BR-C02}{Un usuario no puede desafiar a otro con el que tenga un desafío pendiente}
 % El archivo de instalación debe ser un archivo ZIP, el cual debe contener
 % exactamente un directorio que coincida con el nombre del plugin.
     \BRitem[control]{Revisor}{Sin asignar.}

 \BRsection[control]{Atributos}

    \BRitem[admin]{Clase}{\bcCondition}%
    %\BRitem[admin]{Clase}{\bcIntegridad}%
    %\BRitem[admin]{Clase}{\bcAutorizacion}%
    %\BRitem[admin]{Clase}{\bcDerivacion}%

    \BRitem[admin]{Tipo}{\btEnabler}%
    %\BRitem[admin]{Tipo}{\btTimer}%
    %\BRitem[admin]{Tipo}{\btExecutive}%

    \BRitem[admin]{Nivel}{\blControlling}
    %\BRitem[admin]{Nivel}{\blInfluencing}

    \BRitem{Descripción}{%
        Cuando un usuario desafía a un \refElem{aEstudiante} no lo podrá volver a desafiar hasta que el desafiante
        y desafiado terminen hayan completado la competencia.
        % debido a que se el directorio donde se guardará será el directorio
        % para almacenar las imágenes del plugin.
    }

    \BRitem{Ejemplo positivo}{\hfill\par%
        \begin{itemize}
        \item El usuario desafía a un estudiante, los dos terminan la competencia y el usuario vuelve a desafiar al mismo estudiante.

        \end{itemize}
    }

    \BRitem{Ejemplo negativo}{\hfill\par%
        \begin{itemize}
          \item El usuario desafía a un estudiante y no alguno de los dos no termina la competencia,
          el usuario no puede volver a desafiar al mismo estudiante.

        \end{itemize}
    }%

 \end{BusinessRule}
 % Permanencia del nivel de comperiencia.
    %
\begin{BusinessRule}[%
Autor/Daniel Isai Ortega Zúñiga,%
Version/0.1,%
Estado/revision]%
%
{BR-E03}{Tipos de Incremento}
    \BRitem[control]{Revisor}{Sin asignar.}

 \BRsection[control]{Atributos}
    
    \BRitem[admin]{Clase}{\bcCondition}%
    %\BRitem[admin]{Clase}{\bcIntegridad}%
    %\BRitem[admin]{Clase}{\bcAutorizacion}%
    %\BRitem[admin]{Clase}{\bcDerivacion}%
        
    \BRitem[admin]{Tipo}{\btEnabler}%
    %\BRitem[admin]{Tipo}{\btTimer}%
    %\BRitem[admin]{Tipo}{\btExecutive}%
        
    \BRitem[admin]{Nivel}{\blControlling}
    %\BRitem[admin]{Nivel}{\blInfluencing}
    
    \BRitem{Descripción}{%
    Cuando se modifiquen el \refElem{xp-scheme-settings} o la \refElem{levelXP} de las
    \refElem{xp-scheme-settings}
    }

    \BRitem{Ejemplo positivo}{\hfill\par%
        \begin{itemize}
        \item ...
        \end{itemize}
    }

    \BRitem{Ejemplo negativo}{\hfill\par%
        \begin{itemize}
        \item ...
        \end{itemize}
    }% 
    
 \end{BusinessRule}
 % Tipos de incremento
    %\begin{BusinessRule}[%
Autor/Daniel Isai Ortega Zúñiga,%
Version/0.1,%
Estado/revision]%
%
{BR-E04}{Calculo de experiencia del nivel con incremento porcentual}
    \BRitem[control]{Revisor}{Sin asignar.}

 \BRsection[control]{Atributos}
    
    \BRitem[admin]{Clase}{\bcCondition}%
    %\BRitem[admin]{Clase}{\bcIntegridad}%
    %\BRitem[admin]{Clase}{\bcAutorizacion}%
    %\BRitem[admin]{Clase}{\bcDerivacion}%
        
    \BRitem[admin]{Tipo}{\btEnabler}%
    %\BRitem[admin]{Tipo}{\btTimer}%
    %\BRitem[admin]{Tipo}{\btExecutive}%
        
    \BRitem[admin]{Nivel}{\blControlling}
    %\BRitem[admin]{Nivel}{\blInfluencing}
    
    \BRitem{Descripción}{%
        El calculo para obtener la experiencia del nivel $i$ uando el tipo de
        incremento es porcentual está dado por la siguiente fórmula: Sea {\it exp()}
        la función que optiene la experiencia de un nivel en específico, sea tambien
        $i$ el nivel del cual se calcula la experiencia, sea $inc$ el factor de
        incremento de nivel a nivel, y finalmente sea $round()$ una función de
        redondeo a números enteros, entonces:

            $$ exp(i) = round( exp(1) * (inc)^{(i-1)})$$
    }

%   \BRitem{Sentencia}{%
%       Si $fecha$ 
%   }%

    \BRitem{Ejemplo positivo}{\hfill\par%
        \begin{itemize}
        \item La experiencia requerida para superar el nivel 1 es de 2000 puntos y el
              factor de incremento entre los niveles es 1.1, entonces la experiencia
              requerida para pasar el nivel 5 es de 2928 puntos.
        \end{itemize}
    }

    \BRitem{Ejemplo negativo}{\hfill\par%
        \begin{itemize}
        \item La experinecia requerida para superar el nivel 1 es de 2000 puntos y el
              factor de incremento entre los niveles es 1.1, entonces la experiencia
              requerida para pasar el nivel 5 es de 2300 puntos.
        \end{itemize}
    }% 
    
\end{BusinessRule}
 % Incremento porcentual
    %\begin{BusinessRule}[%
Autor/Daniel Isai Ortega Zúñiga,%
Version/0.1,%
Estado/revision]%
%
{BR-E05}{Cálculo de experiencia del nivel con incremento linea} % Cuando están iniciados
    \BRitem[control]{Revisor}{Sin asignar.}

 \BRsection[control]{Atributos}
    
    \BRitem[admin]{Clase}{\bcCondition}%
    %\BRitem[admin]{Clase}{\bcIntegridad}%
    %\BRitem[admin]{Clase}{\bcAutorizacion}%
    %\BRitem[admin]{Clase}{\bcDerivacion}%
        
    \BRitem[admin]{Tipo}{\btEnabler}%
    %\BRitem[admin]{Tipo}{\btTimer}%
    %\BRitem[admin]{Tipo}{\btExecutive}%
        
    \BRitem[admin]{Nivel}{\blControlling}
    %\BRitem[admin]{Nivel}{\blInfluencing}
    
    \BRitem{Descripción}{%
    }

%   \BRitem{Sentencia}{%
%       Si $fecha$ 
%   }%

    \BRitem{Ejemplo positivo}{\hfill\par%
        \begin{itemize}
        \item ...
        \end{itemize}
    }

    \BRitem{Ejemplo negativo}{\hfill\par%
        \begin{itemize}
        \item ...
        \end{itemize}
    }% 
    
\end{BusinessRule}
 % Incremento lineal
    %\begin{BusinessRule}[%
Autor/Daniel Isai Ortega Zúñiga,%
Version/0.1,%
Estado/revision]%
%
{BR-E05}{Cálculo de experiencia del nivel con incremento linea} % Cuando están iniciados
    \BRitem[control]{Revisor}{Sin asignar.}

 \BRsection[control]{Atributos}
    
    \BRitem[admin]{Clase}{\bcCondition}%
    %\BRitem[admin]{Clase}{\bcIntegridad}%
    %\BRitem[admin]{Clase}{\bcAutorizacion}%
    %\BRitem[admin]{Clase}{\bcDerivacion}%
        
    \BRitem[admin]{Tipo}{\btEnabler}%
    %\BRitem[admin]{Tipo}{\btTimer}%
    %\BRitem[admin]{Tipo}{\btExecutive}%
        
    \BRitem[admin]{Nivel}{\blControlling}
    %\BRitem[admin]{Nivel}{\blInfluencing}
    
    \BRitem{Descripción}{%
    }

%   \BRitem{Sentencia}{%
%       Si $fecha$ 
%   }%

    \BRitem{Ejemplo positivo}{\hfill\par%
        \begin{itemize}
        \item ...
        \end{itemize}
    }

    \BRitem{Ejemplo negativo}{\hfill\par%
        \begin{itemize}
        \item ...
        \end{itemize}
    }% 
    
\end{BusinessRule}
 % Eliminación de cursos gamificados
    %\begin{BusinessRule}[%
Autor/El Despistado,%
Version/0.1,%
Estado/edicion]%
%
{BR-E07}{Valores iniciales de experiencia}

     \BRitem[control]{Revisada por}{Pendiente.}

 \BRsection[control]{Atributos}
    % Clases: \bcCondition, \bcIntegridad, \bcAutorization o \bcDerivation
    % Tipos: \btEnabler, \btTimer o \btExecutive
    % Niveles: \blControlling o \blInfluencing.
    
    \BRitem[admin]{Clase}{\bcIntegridad}%
        
    \BRitem[admin]{Tipo}{\btTimer}%
        
    \BRitem[admin]{Nivel}{\blControlling}
    
    \BRitem{Descripción}{%
        Cuando un \refElem{xp-user} es creado este debe de empezar a ganar puntos
        de experiencia a partir del \refElem{xp-user.level} uno, tenido cero puntos 
        de experiencia en la \refElem{xp-user.levelxp} y \refElem{xp-user.xp}. Ningun
        usuario puede comenzar con valores distintos a los indicados anteriomente.
    }

%   \BRitem{Sentencia}{%
%       Si $fecha$ 
%   }%

    \BRitem{Ejemplo positivo}{\hfill\par%
        \begin{itemize}
        \item ...
        \end{itemize}
    }

    \BRitem{Ejemplo negativo}{\hfill\par%
        \begin{itemize}
        \item ...
        \end{itemize}
    }
    
 \end{BusinessRule}
 % Valores iniciales de comperiencia
    %\begin{BusinessRule}[%
Autor/Daniel Isai Ortega Zúñiga,%
Version/0.1,%
Estado/edicion]%
%
{BR-E08}{Valores iniciales de experiencia de un curso}

     \BRitem[control]{Revisada por}{Pendiente.}

 \BRsection[control]{Atributos}
    % Clases: \bcCondition, \bcIntegridad, \bcAutorization o \bcDerivation
    % Tipos: \btEnabler, \btTimer o \btExecutive
    % Niveles: \blControlling o \blInfluencing.

    \BRitem[admin]{Clase}{\bcIntegridad}%

    \BRitem[admin]{Tipo}{\btTimer}%

    \BRitem[admin]{Nivel}{\blControlling}

    \BRitem{Descripción}{%
        Cuando un \refElem{xp-course} es creado la \refElem[experiencia total del curso]%
        {xp-scheme-settings.courseXP} de ser dividida uniformemente entre las
        \refElem[secciones del curso gamificado]{xp-course-section}. Si la división del
        total de experiencia entre el número de secciones genera un residuo entonces este
        se deberá agregan a la última sección del curso.
    }

%   \BRitem{Sentencia}{%
%       Si $fecha$
%   }%

    \BRitem{Ejemplo positivo}{\hfill\par%
        \begin{itemize}
        \item ...
        \end{itemize}
    }

    \BRitem{Ejemplo negativo}{\hfill\par%
        \begin{itemize}
        \item ...
        \end{itemize}
    }

 \end{BusinessRule}
 % Valores iniciales de experiencia del curso

    % INPUT: Cursos Igualitarios.
    % INPUT: Otorgar experiencia
    % INPUT: Administración de experiencia en el curso

\clearpage
\subsubsection{Casos de uso} % ============================================================

 En este apartado se especifican todos los casos de usos contemplados para el módulo de
competencia, para cada caso de uso se especifica su tabla de atributos la cual indica que casos
 de prueba deberán ejecutarse correctamente para corroborar la completitud del caso de uso.

\subsubsection*{Diagrama de casos de uso}

 En la figura \ref{comp:usecases} se detalla el diagrama de casos de uso correspondiente al módulo
 de competencia. Los casos de uso de moodle (en color blanco) son modelados como casos de uso
 abstractos, mientras que los casos de uso del módulo de competencia son diferenciados por el
 color azul, en total el desarrollo de este módulo consiste en 17 casos de uso principales.

    \addfigure{0.6}{modulos/comp/diagrams/UseCases}{comp:usecases}{%
        Diagrama de casos de uso del módulo de competencia}

 \noindent
 Debido a que los plugins a desarrollar son elementos opcionales para Moodle, solo se puede
 acceder a los casos de uso del módulo de competencia a través de puntos de extensión de los
 casos de uso de moodle. Por otra parte los casos de uso que serán documentados en esta sección
 serán los del módulo de competencia debido a que Moodle proporciona en su página oficial, guías
 e instructivos como documentación de las funcionalidades que brinda.

    % MODULO DE EXPERIENCIA


% \ucstEnEdicion     Al terminar una revisión/aprobación con observaciones
%                    y al inicio del CU.
%
% \ucstEnRevision    Al terminar la edición del CU (version += 0.1).
% \ucstEnAprobacion  Al pasar la revision sin observaciones.
% \ucstAprobado      Al ser aprobado por el usuario (version += 1.0)

\begin{UseCase}[%
Autor/Ricardo Naranjo,%
Version/0.1,%
Estado/\ucstEnRevision]%
%
{CU-C01}{Crear instancia (Competencia uno contra uno)}{%
%
 Permite al \refElem{aProfesor} y al \refElem{aAdministrador} crear una nueva instancia de la actividad competencia uno contra uno en su curso.
 La conclusión de la trayectoria principal de esta caso de uso es una precondición para que
 algunos casos de uso del módulo de competencia puedan ejecutarse.\\%
 Este caso de uso es una extensión del caso de uso {\it Listar actividades disponibles} que es propio de moodle.}

	\UCitem[control]{Revisor}{ Sin asignar }
	\UCitem[control]{Último cambio}{ 13/NOV/19 }

 \UCsection{Atributos}

    \UCitem{Actor(es)}{%
        \refElem{aProfesor},
        \refElem{aAdministrador}
    }

	\UCitems{Propósito}{%
        \Titem Permitir al \refElem{aProfesor} y al \refElem{aAdministrador} incluir en su curso una nueva instancia de la actividad de competencia uno contra uno.

        \Titem Permitir al \refElem{aEstudiante}, \refElem{aProfesor} y \refElem{aAdministrador} con acceso al curso utilizar la instancia de la actividad de competencia uno contra uno creada por el \refElem{aProfesor} o \refElem{aAdministrador}.
	}

	\UCitem{Entradas}{\imprimeUC{entrada}}

	\UCitems{Origen}{%
        \Titem Mouse
        \Titem Teclado
	}

	\UCitem{Salidas}{\imprimeUC{salida}}

	\UCitem{Destino}{%
		\refElem{IU-M08}
	}

	\UCitems{Precondiciones}{%
        \Titem El plugin de competencia uno contra uno debe estar instalado en moodle.
        % Realizar el caso de uso "listar actividades disponibles?"
        % \Titem Si se trata de una actualización de un plugin la versión de este debe
               % cumplir con la regla \refElem{BR-M02}.
	}

	\UCitems{Postcondiciones}{%
        \Titem La nueva instancia de la actividad debe mostrarse en la pantalla \refElem{IU-M08}.%

	}

	\UCitem{Reglas de negocio}{\imprimeUC{regla}}

	\UCitems{Errores}{%
        \Titem \UCerr{Err1}{%
            No se ingresó un campo requerido en el formulario de creación de la actividad,}{% CAUSA
            no se puede crear la nueva instancia de la actividad}% EFECTO
	}

	% \UCitem{Viene de}{% Indicar si el Caso de uso es primario o se extiende de otro. La mayoría se
					  % extienden de Login.
		% EJEMPLO: \refIdElem{PY-CU1} o Caso de uso primario.
	% 	\TODO Especificar.
	% }

 \UCsection[design]{Datos de Diseño}

	\UCitems[design]{Casos de Prueba}{%
        \Titem \refElem{CPC-C01}
        \Titem \refElem{CPC-C01a}
	}

 \UCsection[admin]{Datos de Administración de Requerimiento}

	\UCitem[admin]{Observaciones}{}

\end{UseCase}

\subsubsection{Trayectorias del caso de uso}

\begin{UCtrayectoria}%
%
    \Actor Selecciona la actividad Gamedle - Competencia uno contra uno en la pantalla \refElem{IU-M08a}.
    \Sistema Muestra la descripción de la actividad Gamedle - Competencia uno contra uno en la pantalla.

    \Actor Presiona el botón {\bf Agregar} en la pantalla. \refTray{A}
    \Sistema Redirige a la pantalla \refElem{IU-C01}.
    \label{CU-C01-muestra-pantalla}

    \Actor Ingresa los datos correspondientes en el formulario.

    \Actor Presiona el botón {\bf Guardar cambios y regresar al curso}.\refTray{B} \refTray{C}

    \Sistema Valida que los campos ingresados sean válidos. \refTray{D} \refErr{Err1}

    \Sistema Establece los valores ingresados para la nueva instancia \refElem{comp-1v1-gmcompvs} (
      \entrada{comp-1v1-gmcompvs.name},
      \entrada{comp-1v1-gmcompvs.mdl-question-categories-id},
      \entrada{comp-1v1-gmcompvs.apuestas-activas},
      \entrada{comp-1v1-gmcompvs.completionnumwon}), especificadas en el modelo de información.

    \Sistema Redirige a la pantalla \refElem{IU-M08} y muestra la nueva instancia creada en el curso.

\end{UCtrayectoria}

\begin{UCtrayectoriaA}[Fin del caso de uso]%
  {A}{El \refElem{aProfesor} o \refElem{aAdministrador} desea cancelar la creación de la nueva instancia después que se le muestra la descripción de la actividad}

  \Actor Presiona el botón {\bf cancelar} en la pantalla \refElem{IU-M08a}.
  \Sistema Cierra la pantalla \refElem{IU-M08a} y redirige a la pantalla \refElem{IU-M08}.

\end{UCtrayectoriaA}

\begin{UCtrayectoriaA}[Fin del caso de uso]{B}{El \refElem{aProfesor} o \refElem{aAdministrador} desea ver la nueva instancia de la actividad}

    \Actor Presiona el botón {\bf Guardar cambios y mostrar} de la pantalla \refElem{IU-C01}.

    \Sistema Valida que los campos ingresados sean válidos. \refTray{D} \refErr{Err1}

    \Sistema Establece los valores ingresados para la nueva instancia \refElem{comp-1v1-gmcompvs} (
      \refElem{comp-1v1-gmcompvs.name},
      \refElem{comp-1v1-gmcompvs.mdl-question-categories-id},
      \refElem{comp-1v1-gmcompvs.apuestas-activas},
      \refElem{comp-1v1-gmcompvs.completionnumwon}), especificadas en el modelo de información.

    \Sistema Redirige a la pantalla \refElem{IU-C02}.

\end{UCtrayectoriaA}

\begin{UCtrayectoriaA}[Fin del caso de uso]%
  {C}{El \refElem{aProfesor} desea cancelar la creación de la nueva instancia después de mostrar el formulario de creación}

  \Actor Presiona el botón {\bf cancelar} en la pantalla \refElem{IU-C01}.
  \Sistema Redirige a la pantalla \refElem{IU-C01}.

\end{UCtrayectoriaA}

\begin{UCtrayectoriaA}{D}{Algún dato ingresado por el \refElem{aProfesor} o \refElem{aAdministrador} es inválido}

  \Sistema Muestra un mensaje de error "-Usted debe poner un valor aquí", en los campos de la pantalla \refElem{IU-C01} que sean requeridos.
  \Sistema Regresa al paso \ref{CU-C01-muestra-pantalla}

\end{UCtrayectoriaA}
   % Instalar plugin del esquema de comperiencia

% \ucstEnEdicion     Al terminar una revisión/aprobación con observaciones
%                    y al inicio del CU.
%
% \ucstEnRevision    Al terminar la edición del CU (version += 0.1).
% \ucstEnAprobacion  Al pasar la revision sin observaciones.
% \ucstAprobado      Al ser aprobado por el usuario (version += 1.0)

\begin{UseCase}[%
Autor/Ricardo Naranjo,%
Version/0.1,%
Estado/\ucstEnRevision]%
%
{CU-C02}{Actualizar instancia (Competencia uno contra uno)}{%
%
 Permite al \refElem{aProfesor} y al \refElem{aAdministrador} actualizar una instancia de la actividad competencia uno contra uno en su curso.
 Este caso de uso es una extensión del caso de uso {\it Ver curso} que es propio de moodle.}

	\UCitem[control]{Revisor}{ Sin asignar }
	\UCitem[control]{Último cambio}{ 13/NOV/19 }

 \UCsection{Atributos}

    \UCitem{Actor(es)}{%
        \refElem{aProfesor},
        \refElem{aAdministrador}
    }

	\UCitems{Propósito}{%
        \Titem Permitir al \refElem{aProfesor} y al \refElem{aAdministrador} actualizar una instancia de la actividad de competencia uno contra uno.

        \Titem Permitir al \refElem{aEstudiante}, \refElem{aProfesor} y \refElem{aAdministrador} con acceso al curso utilizar la instancia actualizada de la actividad de competencia uno contra uno creada por el \refElem{aProfesor} o \refElem{aAdministrador}.
	}

	\UCitem{Entradas}{\imprimeUC{entrada}}

	\UCitems{Origen}{%
        \Titem Mouse
        \Titem Teclado
	}

	\UCitem{Salidas}{\imprimeUC{salida}}

	\UCitem{Destino}{%
		\refElem{IU-M08}
	}

	\UCitems{Precondiciones}{%
        \Titem El plugin de competencia uno contra uno debe estar instalado en moodle.
        \Titem La instancia de la actividad de competencia uno contra uno debe estar creada.
        % Realizar el caso de uso "listar actividades disponibles?"
        % \Titem Si se trata de una actualización de un plugin la versión de este debe
               % cumplir con la regla \refElem{BR-M02}.
	}

	\UCitems{Postcondiciones}{%
        \Titem La instancia actualizada de la actividad debe mostrarse en la pantalla \refElem{IU-M08}.%

	}

	\UCitem{Reglas de negocio}{\imprimeUC{regla}}

	\UCitems{Errores}{%
        \Titem \UCerr{Err1}{%
            No se ingresó un campo requerido en el formulario de creación de la actividad,}{% CAUSA
            no se puede actualizar la instancia de la actividad}% EFECTO
	}

	% \UCitem{Viene de}{% Indicar si el Caso de uso es primario o se extiende de otro. La mayoría se
					  % extienden de Login.
		% EJEMPLO: \refIdElem{PY-CU1} o Caso de uso primario.
	% 	\TODO Especificar.
	% }

 \UCsection[design]{Datos de Diseño}

	\UCitems[design]{Casos de Prueba}{%
        \Titem \refElem{CPC-C01}
	}

 \UCsection[admin]{Datos de Administración de Requerimiento}

	\UCitem[admin]{Observaciones}{}

\end{UseCase}

\subsubsection{Trayectorias del caso de uso}

\begin{UCtrayectoria}%
%

    \Actor Activa la edición del curso en la pantalla \refElem{IU-M08}.
    \Sistema Redirige a la pantalla de edición del curso \refElem{IU-M08aa}.
    \Actor Presiona el botón {\bf Editar} de la instancia que desea actualizar.
    \Sistema Despliega el menú \refElem{IU-M08b}.
    \Actor Presiona el botón {\bf Editar ajustes} del menú desplegable \refElem{IU-M08b}.

    \Sistema Redirige a la pantalla \refElem{IU-C06} y carga los valores de la instancia \refElem{comp-1v1-gmcompvs} (
      \salida{comp-1v1-gmcompvs.name},
      \salida{comp-1v1-gmcompvs.mdl-question-categories-id},
      \salida{comp-1v1-gmcompvs.apuestas-activas},
      \salida{comp-1v1-gmcompvs.completionnumwon}).

    \label{CU-C02-muestra-pantalla}

    \Actor Actualiza los datos correspondientes en el formulario.

    \Actor Presiona el botón {\bf Guardar cambios y regresar al curso}.\refTray{A} \refTray{B}

    \Sistema Valida que los campos ingresados sean válidos. \refTray{C} \refErr{Err1}

    \Sistema Actualiza los valores ingresados para la instancia \refElem{comp-1v1-gmcompvs} (
      \entrada{comp-1v1-gmcompvs.name},
      \entrada{comp-1v1-gmcompvs.mdl-question-categories-id},
      \entrada{comp-1v1-gmcompvs.apuestas-activas},
      \entrada{comp-1v1-gmcompvs.completionnumwon}), especificadas en el modelo de información.

    \Sistema Redirige a la pantalla \refElem{IU-M08} y muestra la instancia actualizada en el curso.

\end{UCtrayectoria}

\begin{UCtrayectoriaA}[Fin del caso de uso]{A}{El \refElem{aProfesor} o \refElem{aAdministrador} desea ver la instancia actualizada de la actividad}

    \Actor Presiona el botón {\bf Guardar cambios y mostrar} de la pantalla \refElem{IU-C06}.

    \Sistema Valida que los campos ingresados sean válidos. \refTray{C} \refErr{Err1}

    \Sistema Actualiza los valores ingresados para la instancia \refElem{comp-1v1-gmcompvs} (
      \refElem{comp-1v1-gmcompvs.name},
      \refElem{comp-1v1-gmcompvs.mdl-question-categories-id},
      \refElem{comp-1v1-gmcompvs.apuestas-activas},
      \refElem{comp-1v1-gmcompvs.completionnumwon}), especificadas en el modelo de información.

    \Sistema Redirige a la pantalla \refElem{IU-C02}.

\end{UCtrayectoriaA}

\begin{UCtrayectoriaA}[Fin del caso de uso]%
  {B}{El \refElem{aProfesor} o \refElem{aAdministrador} desea cancelar la actualización de la instancia después de mostrar el formulario de actualización}

  \Actor Presiona el botón {\bf cancelar} en la pantalla \refElem{IU-C06}.
  \Sistema Redirige a la pantalla \refElem{IU-C01}.

\end{UCtrayectoriaA}

\begin{UCtrayectoriaA}{C}{Algún dato ingresado por el \refElem{aProfesor} o \refElem{aAdministrador} es inválido}

  \Sistema Muestra un mensaje de error "-Usted debe poner un valor aquí", en los campos de la pantalla \refElem{IU-C06} que sean requeridos.
  \Sistema Regresa al paso \ref{CU-C02-muestra-pantalla}

\end{UCtrayectoriaA}


% \ucstEnEdicion     Al terminar una revisión/aprobación con observaciones
%                    y al inicio del CU.
%
% \ucstEnRevision    Al terminar la edición del CU (version += 0.1).
% \ucstEnAprobacion  Al pasar la revision sin observaciones.
% \ucstAprobado      Al ser aprobado por el usuario (version += 1.0)

\begin{UseCase}[%
Autor/Ricardo Naranjo,%
Version/0.1,%
Estado/\ucstEnRevision]%
%
{CU-C03}{Eliminar instancia (Competencia uno contra uno)}{%
%
 Permite al \refElem{aProfesor} y al \refElem{aAdministrador} eliminar una instancia de la actividad competencia uno contra uno en su curso.
 Este caso de uso es una extensión del caso de uso {\it Ver curso} que es propio de moodle.}

	\UCitem[control]{Revisor}{ Sin asignar }
	\UCitem[control]{Último cambio}{ 13/NOV/19 }

 \UCsection{Atributos}

    \UCitem{Actor(es)}{%
        \refElem{aProfesor},
        \refElem{aAdministrador}
    }

	\UCitems{Propósito}{%
        \Titem Permitir al \refElem{aProfesor} y al \refElem{aAdministrador} eliminar una instancia de la actividad de competencia uno contra uno.
	}

	\UCitem{Entradas}{\imprimeUC{entrada}}

	\UCitems{Origen}{%
        \Titem Mouse
	}

	\UCitem{Salidas}{\imprimeUC{salida}}

	\UCitem{Destino}{%
		\refElem{IU-M08}
	}

	\UCitems{Precondiciones}{%
        \Titem El plugin de competencia uno contra uno debe estar instalado en moodle.
        \Titem La instancia de la actividad de competencia uno contra uno debe estar creada.
        % Realizar el caso de uso "listar actividades disponibles?"
        % \Titem Si se trata de una actualización de un plugin la versión de este debe
               % cumplir con la regla \refElem{BR-M02}.
	}

	\UCitems{Postcondiciones}{%
        \Titem La instancia de la actividad eliminada no debe mostrarse en la pantalla \refElem{IU-M08}.%

	}

	\UCitem{Reglas de negocio}{\imprimeUC{regla}}

	\UCitems{Errores}{%
	}

	% \UCitem{Viene de}{% Indicar si el Caso de uso es primario o se extiende de otro. La mayoría se
					  % extienden de Login.
		% EJEMPLO: \refIdElem{PY-CU1} o Caso de uso primario.
	% 	\TODO Especificar.
	% }

 \UCsection[design]{Datos de Diseño}

	\UCitems[design]{Casos de Prueba}{%
        \Titem \refElem{CPC-C03}
	}

 \UCsection[admin]{Datos de Administración de Requerimiento}

	\UCitem[admin]{Observaciones}{}

\end{UseCase}

\subsubsection{Trayectorias del caso de uso}

\begin{UCtrayectoria}%
%

    \Actor Activa la edición del curso en la pantalla \refElem{IU-M08}.

    \Sistema Redirige a la pantalla de edición del curso \refElem{IU-M08aa}.

    \Actor Presiona el botón {\bf Editar} de la instancia que desea eliminar.

    \Sistema Despliega el menú \refElem{IU-M08b}.

    \Actor Presiona el botón {\bf Eliminar} del menú desplegable \refElem{IU-M08b}.

    \Sistema Despliega mensaje de confirmación de eliminación. \refElem{IU-M08c}

    \Actor Presiona el botón {\bf Si}. \refTray{A}

    \Sistema Redirige a la pantalla \refElem{IU-M08} y elimina la instancia y los valores de la instancia \refElem{comp-1v1-gmcompvs}, así como los datos que dependen de la instancia en las siguientes entidades: \refElem{comp-1v1-gmdl-partida}, \refElem{comp-1v1-gmdl-participacion} y \refElem{comp-1v1-gmdl-partida}.

\end{UCtrayectoria}

\begin{UCtrayectoriaA}[Fin del caso de uso]{A}{El \refElem{aProfesor} o \refElem{aAdministrador} desea cancelar la eliminación después de mostrar el mensaje de confirmación}

  \Actor Presiona el botón {\bf No} en la mensaje de confirmación \refElem{IU-M08c}.
  \Sistema Redirige a la pantalla \refElem{IU-M08}.

\end{UCtrayectoriaA}


% \ucstEnEdicion     Al terminar una revisión/aprobación con observaciones
%                    y al inicio del CU.
%
% \ucstEnRevision    Al terminar la edición del CU (version += 0.1).
% \ucstEnAprobacion  Al pasar la revision sin observaciones.
% \ucstAprobado      Al ser aprobado por el usuario (version += 1.0)

\begin{UseCase}[%
Autor/Ricardo Naranjo,%
Version/0.1,%
Estado/\ucstEnRevision]%
%
{CU-C04}{Crear instancia (Competencia uno contra sistema)}{%
%
 Permite al \refElem{aProfesor} y al \refElem{aAdministrador} crear una nueva instancia de la actividad competencia uno contra sistema en su curso.
 La conclusión de la trayectoria principal de esta caso de uso es una precondición para que
 algunos casos de uso del módulo de competencia puedan ejecutarse.\\%
 Este caso de uso es una extensión del caso de uso {\it Listar actividades disponibles} que es propio de moodle.}

	\UCitem[control]{Revisor}{ Sin asignar }
	\UCitem[control]{Último cambio}{ 13/NOV/19 }

 \UCsection{Atributos}

    \UCitem{Actor(es)}{%
        \refElem{aProfesor},
        \refElem{aAdministrador}
    }

	\UCitems{Propósito}{%
        \Titem Permitir al \refElem{aProfesor} y al \refElem{aAdministrador} incluir en su curso una nueva instancia de la actividad de competencia uno contra sistema.

        \Titem Permitir al \refElem{aEstudiante}, \refElem{aProfesor} y \refElem{aAdministrador} con acceso al curso utilizar la instancia de la actividad de competencia uno contra sistema creada por el \refElem{aProfesor} o \refElem{aAdministrador}.
	}

	\UCitem{Entradas}{\imprimeUC{entrada}}

	\UCitems{Origen}{%
        \Titem Mouse
        \Titem Teclado
	}

	\UCitem{Salidas}{\imprimeUC{salida}}

	\UCitem{Destino}{%
		\refElem{IU-M07}
	}

	\UCitems{Precondiciones}{%
        \Titem El plugin de competencia 1 contra sistema debe estar instalado en moodle.
        % Realizar el caso de uso "listar actividades disponibles?"
        % \Titem Si se trata de una actualización de un plugin la versión de este debe
               % cumplir con la regla \refElem{BR-M02}.
	}

	\UCitems{Postcondiciones}{%
        \Titem La nueva instancia de la actividad debe mostrarse en la pantalla \refElem{IU-M07}.%

	}

	\UCitem{Reglas de negocio}{\imprimeUC{regla}}

	\UCitems{Errores}{%
        \Titem \UCerr{Err1}{%
            No se ingresó un campo requerido en el formulario de creación de la actividad,}{% CAUSA
            no se puede crear la nueva instancia de la actividad}% EFECTO
	}

	% \UCitem{Viene de}{% Indicar si el Caso de uso es primario o se extiende de otro. La mayoría se
					  % extienden de Login.
		% EJEMPLO: \refIdElem{PY-CU1} o Caso de uso primario.
	% 	\TODO Especificar.
	% }

 \UCsection[design]{Datos de Diseño}

	\UCitems[design]{Casos de Prueba}{%
        \Titem \refElem{CPC-C01}
	}

 \UCsection[admin]{Datos de Administración de Requerimiento}

	\UCitem[admin]{Observaciones}{}

\end{UseCase}

\subsubsection{Trayectorias del caso de uso}

\begin{UCtrayectoria}%
%
    \Actor Selecciona la actividad Gamedle - Competencia 1 contra sistema en la pantalla \refElem{IU-M07a}.
    \Sistema Muestra la descripción de la actividad Gamedle - Competencia 1 contra sistema en la pantalla.

    \Actor Presiona el botón {\bf Agregar} en la pantalla. \refTray{A}
    \Sistema Redirige a la pantalla \refElem{IU-C03}.
    \label{CU-C04-muestra-pantalla}

    \Actor Ingresa los datos correspondientes en el formulario.

    \Actor Presiona el botón {\bf Guardar cambios y regresar al curso}.\refTray{B} \refTray{C}

    \Sistema Valida que los campos ingresados sean válidos. \refTray{D} \refErr{Err1}

    \Sistema Establece los valores ingresados para la nueva instancia \refElem{comp-cpu-gmcompcpu} (
      \entrada{comp-cpu-gmcompcpu.name},
      \entrada{comp-cpu-gmcompcpu.mdl-question-categories-id},
      \entrada{comp-cpu-gmcompcpu.completioncpudiff}), especificadas en el modelo de información.

    \Sistema Redirige a la pantalla \refElem{IU-M07} y muestra la nueva instancia creada en el curso.

\end{UCtrayectoria}

\begin{UCtrayectoriaA}[Fin del caso de uso]%
  {A}{El \refElem{aProfesor} o \refElem{aAdministrador} desea cancelar la creación de la nueva instancia después que se le muestra la descripción de la actividad}

  \Actor Presiona el botón {\bf cancelar} en la pantalla \refElem{IU-M07a}.
  \Sistema Cierra la pantalla \refElem{IU-M07a} y redirige a la pantalla \refElem{IU-M07}.

\end{UCtrayectoriaA}

\begin{UCtrayectoriaA}[Fin del caso de uso]{B}{El \refElem{aProfesor} o \refElem{aAdministrador} desea ver la nueva instancia de la actividad}

    \Actor Presiona el botón {\bf Guardar cambios y mostrar} de la pantalla \refElem{IU-C03}.

    \Sistema Valida que los campos ingresados sean válidos. \refTray{D} \refErr{Err1}

    \Sistema Establece los valores ingresados para la nueva instancia \refElem{comp-cpu-gmcompcpu} (
      \refElem{comp-cpu-gmcompcpu.name},
      \refElem{comp-cpu-gmcompcpu.mdl-question-categories-id},
      \refElem{comp-cpu-gmcompcpu.completioncpudiff}), especificadas en el modelo de información.

    \Sistema Redirige a la pantalla \refElem{IU-C04}.

\end{UCtrayectoriaA}

\begin{UCtrayectoriaA}[Fin del caso de uso]%
  {C}{El \refElem{aProfesor} desea cancelar la creación de la nueva instancia después de mostrar el formulario de creación}

  \Actor Presiona el botón {\bf cancelar} en la pantalla \refElem{IU-C03}.
  \Sistema Redirige a la pantalla \refElem{IU-M07}.

\end{UCtrayectoriaA}

\begin{UCtrayectoriaA}{D}{Algún dato ingresado por el \refElem{aProfesor} o \refElem{aAdministrador} es inválido}

  \Sistema Muestra un mensaje de error "-Usted debe poner un valor aquí", en los campos de la pantalla \refElem{IU-C03} que sean requeridos.
  \Sistema Regresa al paso \ref{CU-C04-muestra-pantalla}

\end{UCtrayectoriaA}
   % Instalar plugin del esquema de comperiencia

% \ucstEnEdicion     Al terminar una revisión/aprobación con observaciones
%                    y al inicio del CU.
%
% \ucstEnRevision    Al terminar la edición del CU (version += 0.1).
% \ucstEnAprobacion  Al pasar la revision sin observaciones.
% \ucstAprobado      Al ser aprobado por el usuario (version += 1.0)

\begin{UseCase}[%
Autor/Ricardo Naranjo,%
Version/0.1,%
Estado/\ucstEnRevision]%
%
{CU-C05}{Actualizar instancia (Competencia uno contra sistema)}{%
%
 Permite al \refElem{aProfesor} y al \refElem{aAdministrador} actualizar una instancia de la actividad competencia uno contra sistema en su curso.
 Este caso de uso es una extensión del caso de uso {\it Ver curso} que es propio de moodle.}

	\UCitem[control]{Revisor}{ Sin asignar }
	\UCitem[control]{Último cambio}{ 13/NOV/19 }

 \UCsection{Atributos}

    \UCitem{Actor(es)}{%
        \refElem{aProfesor},
        \refElem{aAdministrador}
    }

	\UCitems{Propósito}{%
        \Titem Permitir al \refElem{aProfesor} y al \refElem{aAdministrador} actualizar una instancia de la actividad de competencia uno contra sistema.

        \Titem Permitir al \refElem{aEstudiante}, \refElem{aProfesor} y \refElem{aAdministrador} con acceso al curso utilizar la instancia actualizada de la actividad de competencia uno contra sistema creada por el \refElem{aProfesor} o \refElem{aAdministrador}.
	}

	\UCitem{Entradas}{\imprimeUC{entrada}}

	\UCitems{Origen}{%
        \Titem Mouse
        \Titem Teclado
	}

	\UCitem{Salidas}{\imprimeUC{salida}}

	\UCitem{Destino}{%
		\refElem{IU-M07}
	}

	\UCitems{Precondiciones}{%
        \Titem El plugin de competencia uno contra sistema debe estar instalado en moodle.
        \Titem La instancia de la actividad de competencia uno contra sistema debe estar creada.
        % Realizar el caso de uso "listar actividades disponibles?"
        % \Titem Si se trata de una actualización de un plugin la versión de este debe
               % cumplir con la regla \refElem{BR-M02}.
	}

	\UCitems{Postcondiciones}{%
        \Titem La instancia actualizada de la actividad debe mostrarse en la pantalla \refElem{IU-M07}.%

	}

	\UCitem{Reglas de negocio}{\imprimeUC{regla}}

	\UCitems{Errores}{%
        \Titem \UCerr{Err1}{%
            No se ingresó un campo requerido en el formulario de creación de la actividad,}{% CAUSA
            no se puede actualizar la instancia de la actividad}% EFECTO
	}

	% \UCitem{Viene de}{% Indicar si el Caso de uso es primario o se extiende de otro. La mayoría se
					  % extienden de Login.
		% EJEMPLO: \refIdElem{PY-CU1} o Caso de uso primario.
	% 	\TODO Especificar.
	% }

 \UCsection[design]{Datos de Diseño}

	\UCitems[design]{Casos de Prueba}{%
        \Titem \refElem{CPC-C01}
	}

 \UCsection[admin]{Datos de Administración de Requerimiento}

	\UCitem[admin]{Observaciones}{}

\end{UseCase}

\subsubsection{Trayectorias del caso de uso}

\begin{UCtrayectoria}%
%

    \Actor Activa la edición del curso en la pantalla \refElem{IU-M07}.
    \Sistema Redirige a la pantalla de edición del curso \refElem{IU-M07aa}.
    \Actor Presiona el botón {\bf Editar} de la instancia que desea actualizar.
    \Sistema Despliega el menú \refElem{IU-M07b}.
    \Actor Presiona el botón {\bf Editar ajustes} del menú desplegable \refElem{IU-M07b}.

    \Sistema Redirige a la pantalla \refElem{IU-C05} y carga los valores de la instancia \refElem{comp-cpu-gmcompcpu} (
      \salida{comp-cpu-gmcompcpu.name},
      \salida{comp-cpu-gmcompcpu.mdl-question-categories-id},
      \salida{comp-cpu-gmcompcpu.completioncpudiff}), especificadas en el modelo de información.

    \label{CU-C05-muestra-pantalla}

    \Actor Actualiza los datos correspondientes en el formulario.

    \Actor Presiona el botón {\bf Guardar cambios y regresar al curso}.\refTray{A} \refTray{B}

    \Sistema Valida que los campos ingresados sean válidos. \refTray{C} \refErr{Err1}

    \Sistema Actualiza los valores ingresados para la instancia \refElem{comp-cpu-gmcompcpu} (
      \entrada{comp-cpu-gmcompcpu.name},
      \entrada{comp-cpu-gmcompcpu.mdl-question-categories-id},
      \entrada{comp-cpu-gmcompcpu.completioncpudiff}), especificadas en el modelo de información.

    \Sistema Redirige a la pantalla \refElem{IU-M07} y muestra la instancia actualizada en el curso.

\end{UCtrayectoria}

\begin{UCtrayectoriaA}[Fin del caso de uso]{A}{El \refElem{aProfesor} o \refElem{aAdministrador} desea ver la instancia actualizada de la actividad}

    \Actor Presiona el botón {\bf Guardar cambios y mostrar} de la pantalla \refElem{IU-C05}.

    \Sistema Valida que los campos ingresados sean válidos. \refTray{C} \refErr{Err1}

    \Sistema Actualiza los valores ingresados para la instancia \refElem{comp-cpu-gmcompcpu} (
      \refElem{comp-cpu-gmcompcpu.name},
      \refElem{comp-cpu-gmcompcpu.mdl-question-categories-id},
      \refElem{comp-cpu-gmcompcpu.completioncpudiff}), especificadas en el modelo de información.

    \Sistema Redirige a la pantalla \refElem{IU-C02}.

\end{UCtrayectoriaA}

\begin{UCtrayectoriaA}[Fin del caso de uso]%
  {B}{El \refElem{aProfesor} o \refElem{aAdministrador} desea cancelar la actualización de la instancia después de mostrar el formulario de actualización}

  \Actor Presiona el botón {\bf cancelar} en la pantalla \refElem{IU-C05}.
  \Sistema Redirige a la pantalla \refElem{IU-C01}.

\end{UCtrayectoriaA}

\begin{UCtrayectoriaA}{C}{Algún dato ingresado por el \refElem{aProfesor} o \refElem{aAdministrador} es inválido}

  \Sistema Muestra un mensaje de error "-Usted debe poner un valor aquí", en los campos de la pantalla \refElem{IU-C05} que sean requeridos.
  \Sistema Regresa al paso \ref{CU-C05-muestra-pantalla}

\end{UCtrayectoriaA}


% \ucstEnEdicion     Al terminar una revisión/aprobación con observaciones
%                    y al inicio del CU.
%
% \ucstEnRevision    Al terminar la edición del CU (version += 0.1).
% \ucstEnAprobacion  Al pasar la revision sin observaciones.
% \ucstAprobado      Al ser aprobado por el usuario (version += 1.0)

\begin{UseCase}[%
Autor/Ricardo Naranjo,%
Version/0.1,%
Estado/\ucstEnRevision]%
%
{CU-C06}{Eliminar instancia (Competencia uno contra sistema)}{%
%
 Permite al \refElem{aProfesor} y al \refElem{aAdministrador} eliminar una instancia de la actividad competencia uno contra sistema en su curso.
 Este caso de uso es una extensión del caso de uso {\it Ver curso} que es propio de moodle.}

	\UCitem[control]{Revisor}{ Sin asignar }
	\UCitem[control]{Último cambio}{ 13/NOV/19 }

 \UCsection{Atributos}

    \UCitem{Actor(es)}{%
        \refElem{aProfesor},
        \refElem{aAdministrador}
    }

	\UCitems{Propósito}{%
        \Titem Permitir al \refElem{aProfesor} y al \refElem{aAdministrador} eliminar una instancia de la actividad de competencia uno contra sistema.
	}

	\UCitem{Entradas}{\imprimeUC{entrada}}

	\UCitems{Origen}{%
        \Titem Mouse
	}

	\UCitem{Salidas}{\imprimeUC{salida}}

	\UCitem{Destino}{%
		\refElem{IU-M08}
	}

	\UCitems{Precondiciones}{%
        \Titem El plugin de competencia uno contra sistema debe estar instalado en moodle.
        \Titem La instancia de la actividad de competencia uno contra sistema debe estar creada.
        % Realizar el caso de uso "listar actividades disponibles?"
        % \Titem Si se trata de una actualización de un plugin la versión de este debe
               % cumplir con la regla \refElem{BR-M02}.
	}

	\UCitems{Postcondiciones}{%
        \Titem La instancia de la actividad eliminada no debe mostrarse en la pantalla \refElem{IU-M08}.%

	}

	\UCitem{Reglas de negocio}{\imprimeUC{regla}}

	\UCitems{Errores}{%
	}

	% \UCitem{Viene de}{% Indicar si el Caso de uso es primario o se extiende de otro. La mayoría se
					  % extienden de Login.
		% EJEMPLO: \refIdElem{PY-CU1} o Caso de uso primario.
	% 	\TODO Especificar.
	% }

 \UCsection[design]{Datos de Diseño}

	\UCitems[design]{Casos de Prueba}{%
        \Titem \refElem{CPC-C06}
	}

 \UCsection[admin]{Datos de Administración de Requerimiento}

	\UCitem[admin]{Observaciones}{}

\end{UseCase}

\subsubsection{Trayectorias del caso de uso}

\begin{UCtrayectoria}%
%

    \Actor Activa la edición del curso en la pantalla \refElem{IU-M08}.

    \Sistema Redirige a la pantalla de edición del curso \refElem{IU-M08aa}.

    \Actor Presiona el botón {\bf Editar} de la instancia que desea eliminar.

    \Sistema Despliega el menú \refElem{IU-M08b}.

    \Actor Presiona el botón {\bf Eliminar} del menú desplegable \refElem{IU-M08b}.

    \Sistema Despliega mensaje de confirmación de eliminación. \refElem{IU-M08c}

    \Actor Presiona el botón {\bf Si}. \refTray{A}

    \Sistema Redirige a la pantalla \refElem{IU-M08} y elimina la instancia y los valores de la instancia \refElem{comp-cpu-gmcompcpu}, así como los datos que dependen de la instancia en las siguientes entidades: \refElem{comp-cpu-gmdl-intento} y \refElem{comp-cpu-gmdl-respuesta-cpu}.

\end{UCtrayectoria}

\begin{UCtrayectoriaA}[Fin del caso de uso]{A}{El \refElem{aProfesor} o \refElem{aAdministrador} desea cancelar la eliminación después de mostrar el mensaje de confirmación}

  \Actor Presiona el botón {\bf No} en la mensaje de confirmación \refElem{IU-M08c}.
  \Sistema Redirige a la pantalla \refElem{IU-M08}.

\end{UCtrayectoriaA}


% \ucstEnEdicion     Al terminar una revisión/aprobación con observaciones
%                    y al inicio del CU.
%
% \ucstEnRevision    Al terminar la edición del CU (version += 0.1).
% \ucstEnAprobacion  Al pasar la revision sin observaciones.
% \ucstAprobado      Al ser aprobado por el usuario (version += 1.0)

%\addfigure[(adaptado de {\it For The Win} \cite{ForTheWin})]%
%    {.4}{investigacion/images/forthewin}{fig:ForTheWin}%
%    {Jerarquía de elementos de juego segun For The Win}

\begin{UseCase}[%
Autor/Ricardo Naranjo,%
Version/0.1,%
Estado/\ucstEnRevision]%
%
{CU-C07}{Ver estado de instancia de actividad (Competencia uno contra uno)}{%
%
 Permite al \refElem{aEstudiante}, \refElem{aProfesor} y al \refElem{aAdministrador} ver el estado actual de una instancia de la actividad competencia uno contra uno en el curso.
 Este caso de uso es una extensión del caso de uso {\it Ver curso} que es propio de moodle.}

	\UCitem[control]{Revisor}{ Sin asignar }
	\UCitem[control]{Último cambio}{ 13/NOV/19 }

 \UCsection{Atributos}

    \UCitem{Actor(es)}{%
        \refElem{aEstudiante},
        \refElem{aProfesor},
        \refElem{aAdministrador}
    }

	\UCitems{Propósito}{%
        \Titem Permitir al \refElem{aEstudiante}, \refElem{aProfesor} y al \refElem{aAdministrador} ver el estado actual de una instancia de la actividad de competencia uno contra uno.
	}

	\UCitem{Entradas}{\imprimeUC{entrada}}

	\UCitems{Origen}{%
        \Titem Mouse
	}

	\UCitem{Salidas}{\imprimeUC{salida}}

	\UCitem{Destino}{%
		\refElem{IU-C02}
	}

	\UCitems{Precondiciones}{%
        \Titem El plugin de competencia uno contra uno debe estar instalado en moodle.
        \Titem La instancia de la actividad de competencia uno contra uno debe estar creada.
        % Realizar el caso de uso "listar actividades disponibles?"
        % \Titem Si se trata de una actualización de un plugin la versión de este debe
               % cumplir con la regla \refElem{BR-M02}.
	}

	\UCitems{Postcondiciones}{%
        \Titem La pantalla principal de la instancia de la actividad de competencia uno contra uno \refElem{IU-C02} debe mostrar los datos pertinentes al usuario que realizó el caso de uso.%

	}

	\UCitem{Reglas de negocio}{\imprimeUC{regla}}

	\UCitems{Errores}{%
	}

	% \UCitem{Viene de}{% Indicar si el Caso de uso es primario o se extiende de otro. La mayoría se
					  % extienden de Login.
		% EJEMPLO: \refIdElem{PY-CU1} o Caso de uso primario.
	% 	\TODO Especificar.
	% }

 \UCsection[design]{Datos de Diseño}

	\UCitems[design]{Casos de Prueba}{%
        \Titem \refElem{CPC-C07}
	}

 \UCsection[admin]{Datos de Administración de Requerimiento}

	\UCitem[admin]{Observaciones}{}

\end{UseCase}

\subsubsection{Trayectorias del caso de uso}

\begin{UCtrayectoria}%
%

    \Actor Presiona el nombre de la instancia a la que quiere acceder en la pantalla \refElem{IU-M07}.

    \Sistema Valida que las partidas en curso cumplan con la \regla{BR-C01} y las partidas que no la cumplen son terminadas.

    \Sistema Verifica que el actor esté dado de alto como usuario gamificado. \refTray{B}.

    \Sistema Valida que la bandera \refElem{comp-1v1-gmcompvs.apuestas-activas} esté activa. \refTray{A}.
    \label{CU-C07-mostrar informacion}
    \Sistema Redirige a la pantalla principal de la instancia \refElem{IU-C02}.

    \Sistema Muestra el \salida{número de victorias}.

    \Sistema Muestra en el área {\bf Compañeros del curso} un bloque compañero por cada \refElem{aEstudiante} que esté inscrito en el curso y que cumpla con la \regla{BR-C02}, dicho bloque se compone de la \salida{imagen de perfil compañero}, \salida{nombre compañero}, campo {\bf Monedas a apostar} y el botón {\bf Desafiar}.

    \Sistema Muestra en el área {\bf Desafíos pendientes} un bloque desafiante por cada \refElem{aEstudiante}, \refElem{aProfesor} o \refElem{aAdministrador} que haya desafiado al \refElem{aEstudiante} que está realizando el caso de uso, dicho bloque se compone de la \salida{imagen de perfil desafiante}, \salida{nombre desafiante}, campo {\bf Monedas a apostar} y el botón {\bf Aceptar desafío}.

\end{UCtrayectoria}

\begin{UCtrayectoriaA}[Fin del caso de uso]{A}{La bandera \refElem{comp-1v1-gmcompvs.apuestas-activas} está inactiva}

  \Sistema Redirige a la pantalla principal de la instancia \refElem{IU-C02a}.

  \Sistema Muestra el número de victorias.

  \Sistema Muestra en el área {\bf Compañeros del curso} un bloque por cada \refElem{aEstudiante} que esté inscrito en el curso y que cumpla con la \regla{BR-C02}, dicho bloque se compone de la imagen de perfil compañero, nombre compañero y el botón {\bf Desafiar}.

  \Sistema Muestra en el área {\bf Desafíos pendientes} un bloque desafiante por cada \refElem{aEstudiante}, \refElem{aProfesor} o \refElem{aAdministrador} que haya desafiado al \refElem{aEstudiante} que está realizando el caso de uso, dicho bloque se compone de la imagen de perfil desafiante, nombre desafiante y el botón {\bf Aceptar desafío}.

\end{UCtrayectoriaA}


\begin{UCtrayectoriaA}{B}{El actor no está dado de alta como usuario gamificado (\refElem{xp-user})}

  \Sistema Registra al actor en la entidad (\refElem{xp-user}).
    \item Se regresa al paso \ref{CU-C07-mostrar informacion} de la trayectoria principal.


\end{UCtrayectoriaA}

\subsubsection{Puntos de extensión}

\UCExtensionPoint{Ver historial de las partidas}{%

    El \refElem{aAdministrador}, \refElem{aProfesor} o \refElem{aEstudiante} desea ver su historial de las partidas de competencia uno contra uno.
%
    }{Al final de la trayectoria principal del caso de uso.
%
    }{\refElem{CU-C08}}


\UCExtensionPoint{Ver tabla de posiciones}{%

    El \refElem{aAdministrador}, \refElem{aProfesor} o \refElem{aEstudiante} desea ver la tabla de posiciones de una instancia de competencia uno contra uno.
%
    }{Al final de la trayectoria principal del caso de uso.
%
    }{\refElem{CU-C09}}

  \UCExtensionPoint{Desafiar a un estudiante con apuestas}{%

      El \refElem{aAdministrador}, \refElem{aProfesor} o \refElem{aEstudiante} desafiar a un estudiante y apostar una cantidad de monedas.
  %
      }{Al final de la trayectoria principal del caso de uso.
  %
      }{\refElem{CU-C10}}


\UCExtensionPoint{Desafiar a un estudiante sin apuestas}{%

    El \refElem{aAdministrador}, \refElem{aProfesor} o \refElem{aEstudiante} desafiar a un estudiante.
%
    }{Al final de la \refTray{A}.
%
    }{\refElem{CU-C11}}


% \ucstEnEdicion     Al terminar una revisión/aprobación con observaciones
%                    y al inicio del CU.
%
% \ucstEnRevision    Al terminar la edición del CU (version += 0.1).
% \ucstEnAprobacion  Al pasar la revision sin observaciones.
% \ucstAprobado      Al ser aprobado por el usuario (version += 1.0)

%\addfigure[(adaptado de {\it For The Win} \cite{ForTheWin})]%
%    {.4}{investigacion/images/forthewin}{fig:ForTheWin}%
%    {Jerarquía de elementos de juego segun For The Win}

\begin{UseCase}[%
Autor/Ricardo Naranjo,%
Version/0.1,%
Estado/\ucstEnRevision]%
%
{CU-C08}{Ver historial de sus partidas (Competencia uno contra uno)}{%
%
 Permite al \refElem{aEstudiante}, \refElem{aProfesor} y al \refElem{aAdministrador} ver su historial de una instancia de la actividad competencia uno contra uno en el curso.
 Este caso de uso es una extensión del caso de uso \refElem{CU-C07}.}

	\UCitem[control]{Revisor}{ Sin asignar }
	\UCitem[control]{Último cambio}{ 13/NOV/19 }

 \UCsection{Atributos}

    \UCitem{Actor(es)}{%
        \refElem{aEstudiante},
        \refElem{aProfesor},
        \refElem{aAdministrador}
    }

	\UCitems{Propósito}{%
        \Titem Permitir al \refElem{aEstudiante}, \refElem{aProfesor} y al \refElem{aAdministrador} ver su historial de una instancia de la actividad de competencia uno contra uno.
	}

	\UCitem{Entradas}{\imprimeUC{entrada}}

	\UCitems{Origen}{%
        \Titem Mouse
	}

	\UCitem{Salidas}{\imprimeUC{salida}}

	\UCitem{Destino}{%
		\refElem{IU-C25}
	}

	\UCitems{Precondiciones}{%
        \Titem El plugin de competencia uno contra uno debe estar instalado en moodle.
        \Titem La instancia de la actividad de competencia uno contra uno debe estar creada.
        % Realizar el caso de uso "listar actividades disponibles?"
        % \Titem Si se trata de una actualización de un plugin la versión de este debe
               % cumplir con la regla \refElem{BR-M02}.
	}

	\UCitems{Postcondiciones}{%
        \Titem La pantalla de historial de la instancia de la actividad de competencia uno contra uno \refElem{IU-C07} debe mostrar los datos pertinentes al usuario que realizó el caso de uso.%

	}

	\UCitem{Reglas de negocio}{\imprimeUC{regla}}

	\UCitems{Errores}{%
	}

	% \UCitem{Viene de}{% Indicar si el Caso de uso es primario o se extiende de otro. La mayoría se
					  % extienden de Login.
		% EJEMPLO: \refIdElem{PY-CU1} o Caso de uso primario.
	% 	\TODO Especificar.
	% }

 \UCsection[design]{Datos de Diseño}

	\UCitems[design]{Casos de Prueba}{%
        \Titem \refElem{CPC-C08}
	}

 \UCsection[admin]{Datos de Administración de Requerimiento}

	\UCitem[admin]{Observaciones}{}

\end{UseCase}

\subsubsection{Trayectorias del caso de uso}

\begin{UCtrayectoria}%
%

    \Actor Presiona el botón {\bf Historial} de la pantalla \refElem{IU-C02}.

    \Sistema Redirige a la pantalla historial de la instancia \refElem{IU-C07}.

    \Sistema Muestra en la sección {\bf Resumen de los estados de los desafíos} los siguientes números: \salida{Victorias}, \salida{Empates}, \salida{Derrotas}, \salida{En curso}, \salida{Retirada}.

    \Sistema Muestra en la sección inferior un bloque por cada partida iniciada, en el bloque se muestra: \salida{Imagen de perfil de contrincante}, \salida{Nombre de contrincante}, {\bf Puntos del actor} \salida{comp-vs-gmdl-participacion.puntuacion}, {\bf Monedas apostadas} \salida{comp-vs-gmdl-apuesta.monedas-plata}, {\bf Puntos contrincante}, {\bf Monedas apostadas contrincante}, \salida{Estado de desafío}.

\end{UCtrayectoria}


% \ucstEnEdicion     Al terminar una revisión/aprobación con observaciones
%                    y al inicio del CU.
%
% \ucstEnRevision    Al terminar la edición del CU (version += 0.1).
% \ucstEnAprobacion  Al pasar la revision sin observaciones.
% \ucstAprobado      Al ser aprobado por el usuario (version += 1.0)

%\addfigure[(adaptado de {\it For The Win} \cite{ForTheWin})]%
%    {.4}{investigacion/images/forthewin}{fig:ForTheWin}%
%    {Jerarquía de elementos de juego segun For The Win}

\begin{UseCase}[%
Autor/Ricardo Naranjo,%
Version/0.1,%
Estado/\ucstEnRevision]%
%
{CU-C09}{Ver tabla de posiciones (Competencia uno contra uno)}{%
%
 Permite al \refElem{aEstudiante}, \refElem{aProfesor} y al \refElem{aAdministrador} ver la tabla de posiciones de una instancia de la actividad competencia uno contra uno en el curso.
 Este caso de uso es una extensión del caso de uso \refElem{CU-C07}.}

	\UCitem[control]{Revisor}{ Sin asignar }
	\UCitem[control]{Último cambio}{ 13/NOV/19 }

 \UCsection{Atributos}

    \UCitem{Actor(es)}{%
        \refElem{aEstudiante},
        \refElem{aProfesor},
        \refElem{aAdministrador}
    }

	\UCitems{Propósito}{%
        \Titem Permitir al \refElem{aEstudiante}, \refElem{aProfesor} y al \refElem{aAdministrador} ver la tabla de posiciones de una instancia de la actividad de competencia uno contra uno.
	}

	\UCitem{Entradas}{\imprimeUC{entrada}}

	\UCitems{Origen}{%
        \Titem Mouse
	}

	\UCitem{Salidas}{\imprimeUC{salida}}

	\UCitem{Destino}{%
		\refElem{IU-C08}
	}

	\UCitems{Precondiciones}{%
        \Titem El plugin de competencia uno contra uno debe estar instalado en moodle.
        \Titem La instancia de la actividad de competencia uno contra uno debe estar creada.
        % Realizar el caso de uso "listar actividades disponibles?"
        % \Titem Si se trata de una actualización de un plugin la versión de este debe
               % cumplir con la regla \refElem{BR-M02}.
	}

	\UCitems{Postcondiciones}{%
        \Titem Se muestra la pantalla de tabla de posiciones de la instancia de la actividad de competencia uno contra uno \refElem{IU-C08} de acuerdo a su número de victorias.%

	}

	\UCitem{Reglas de negocio}{\imprimeUC{regla}}

	\UCitems{Errores}{%
	}

	% \UCitem{Viene de}{% Indicar si el Caso de uso es primario o se extiende de otro. La mayoría se
					  % extienden de Login.
		% EJEMPLO: \refIdElem{PY-CU1} o Caso de uso primario.
	% 	\TODO Especificar.
	% }

 \UCsection[design]{Datos de Diseño}

	\UCitems[design]{Casos de Prueba}{%
        \Titem \refElem{CPC-C09}
	}

 \UCsection[admin]{Datos de Administración de Requerimiento}

	\UCitem[admin]{Observaciones}{}

\end{UseCase}

\subsubsection{Trayectorias del caso de uso}

\begin{UCtrayectoria}%
%

    \Actor Presiona el botón {\bf Tabla posiciones} de la pantalla \refElem{IU-C02}.

    \Sistema Redirige a la pantalla de la instancia \refElem{IU-C08}.

    \Sistema Muestra un bloque por cada usuario con al menos una partida terminada, en el bloque se muestra: \salida{posición del usuario} \salida{Imagen de perfil de usuario}, \salida{Nombre de usuario}, \salida{Número de victorias del usuario}.

\end{UCtrayectoria}


% \ucstEnEdicion     Al terminar una revisión/aprobación con observaciones
%                    y al inicio del CU.
%
% \ucstEnRevision    Al terminar la edición del CU (version += 0.1).
% \ucstEnAprobacion  Al pasar la revision sin observaciones.
% \ucstAprobado      Al ser aprobado por el usuario (version += 1.0)

%\addfigure[(adaptado de {\it For The Win} \cite{ForTheWin})]%
%    {.4}{investigacion/images/forthewin}{fig:ForTheWin}%
%    {Jerarquía de elementos de juego segun For The Win}

\begin{UseCase}[%
Autor/Ricardo Naranjo,%
Version/0.1,%
Estado/\ucstEnRevision]%
%
{CU-C10}{Desafiar a un estudiante apostando (Competencia uno contra uno)}{%
%
 Permite al \refElem{aEstudiante}, \refElem{aProfesor} y al \refElem{aAdministrador} desafiar a un estudiante que se encuentra inscrito en el curso y a su vez se otorga la posibilidad de apostar en la partida.
 Este caso de uso es una extensión del caso de uso \refElem{CU-C07}.}

	\UCitem[control]{Revisor}{ Sin asignar }
	\UCitem[control]{Último cambio}{ 13/NOV/19 }

 \UCsection{Atributos}

    \UCitem{Actor(es)}{%
        \refElem{aEstudiante},
        \refElem{aProfesor},
        \refElem{aAdministrador}
    }

	\UCitems{Propósito}{%
        \Titem Permitir al \refElem{aEstudiante}, \refElem{aProfesor} y al \refElem{aAdministrador} desafiar a un estudiante que forma parte de un curso y apostar la cantidad de monedas decidida por el usuario.
        \Titem Permitir al \refElem{aEstudiante}, \refElem{aProfesor} y al \refElem{aAdministrador} terminar un desafío.
	}

	\UCitem{Entradas}{\imprimeUC{entrada}}

	\UCitems{Origen}{%
        \Titem Mouse
        \Titem Teclado
	}

	\UCitem{Salidas}{\imprimeUC{salida}}

	\UCitem{Destino}{%
		\refElem{IU-C10}
	}

	\UCitems{Precondiciones}{%
        \Titem El plugin de competencia uno contra uno debe estar instalado en moodle.
        \Titem La instancia de la actividad de competencia uno contra uno debe estar creada.
        % Realizar el caso de uso "listar actividades disponibles?"
        % \Titem Si se trata de una actualización de un plugin la versión de este debe
               % cumplir con la regla \refElem{BR-M02}.
	}

	\UCitems{Postcondiciones}{%
        \Titem Se muestra en la pantalla de historial del usuario \refElem{IU-C07} la partida que inició.%

	}

	\UCitem{Reglas de negocio}{\imprimeUC{regla}}

	\UCitems{Errores}{%
	}

	% \UCitem{Viene de}{% Indicar si el Caso de uso es primario o se extiende de otro. La mayoría se
					  % extienden de Login.
		% EJEMPLO: \refIdElem{PY-CU1} o Caso de uso primario.
	% 	\TODO Especificar.
	% }

 \UCsection[design]{Datos de Diseño}

	\UCitems[design]{Casos de Prueba}{%
        \Titem \refElem{CPC-C10}
	}

 \UCsection[admin]{Datos de Administración de Requerimiento}

	\UCitem[admin]{Observaciones}{}

\end{UseCase}

\subsubsection{Trayectorias del caso de uso}

\begin{UCtrayectoria}%
%

    \Actor Ingresa la cantidad de monedas que desea apostar en el campo "Monedas a apostar" en el bloque del estudiante que quiere desafiar en la pantalla \refElem{IU-C02}.

    \Actor Presiona el botón {\bf Desafiar}.

    \Sistema Inicia la partida y establece los valores correspondientes \refElem{comp-1v1-gmdl-partida}( \entrada{comp-1v1-gmdl-partida.gmdl-comp-vs-id} ), una entrada por cada usuario de la partida \refElem{comp-1v1-gmdl-participacion} ( \entrada{comp-1v1-gmdl-participacion.gmdl-usuario-id}, \entrada{comp-1v1-gmdl-participacion.gmdl-partida-id}, \entrada{comp-1v1-gmdl-participacion.fecha-inicio}, \refElem{comp-1v1-gmdl-participacion.puntuacion} ) y agrega la apuesta \refElem{comp-1v1-gmdl-apuesta}( \entrada{comp-1v1-gmdl-apuesta.gmdl-participacion-id}, \entrada{comp-1v1-gmdl-apuesta.monedas-plata}, \entrada{comp-1v1-gmdl-apuesta.activa} ).

    \Sistema Redirige a la pantalla del cuestionario del desafío \refElem{IU-C09}.

    \Actor Contesta las preguntas mostradas. \refTray{A}
    \label{CU-C10-contesta-cuestionario}
    \Actor Presiona el botón {\bf Terminar}.

    \Sistema Evalúa las respuestas ingresadas y calcula su puntuación.

    \Sistema Establece los valores correspondientes de la participación \refElem{comp-1v1-gmdl-participacion} (
    \entrada{comp-1v1-gmdl-participacion.fecha-fin},
    \entrada{comp-1v1-gmdl-participacion.puntuacion}).

    \Sistema Valida que el estudiante desafiado no haya terminado de contestar su cuestionario. \refTray{B}

    \Sistema Redirige a la pantalla \refElem{IU-C10}.

\end{UCtrayectoria}

\begin{UCtrayectoriaA}[Fin del caso de uso]%
  {A}{El actor desea salir del cuestionario sin terminarlo}

  \Actor Presiona cualquier botón, excepto el botón de {\bf Terminar}, que haga que abandone la pantalla.
  \Sistema Muestra mensaje de alerta \refElem{IU-C13}.
  \Actor Presiona el botón {\bf Abandonar}. \refTray{D}
  \Sistema Establece el valor correspondiente de la participación \refElem{comp-1v1-gmdl-participacion} (
  \refElem{comp-1v1-gmdl-participacion.fecha-fin}).
  \Sistema Redirige a la pantalla elegida por el actor.

\end{UCtrayectoriaA}

\begin{UCtrayectoriaA}[Fin del caso de uso]%
  {B}{El \refElem{aEstudiante} desafiado ya había terminado de contestar el cuestionario}

  \Sistema Calcula y otorga puntaje extra de acuerdo al tiempo tardado en contestar el cuestionario.
  \Sistema Comprueba que el actor haya obtenido un mayor puntaje que el \refElem{aEstudiante} desafiado. \refTray{C}
  \Sistema Lanza el evento para dar las monedas que le corresponden al actor.
  \Sistema Lanza el evento para dar la experiencia que le corresponde al actor.
  \Sistema Redirige a la pantalla \refElem{IU-C11}.
  \Sistema Muestra el puntaje de cada participante \salida{comp-1v1-gmdl-participacion.puntuacion}.

\end{UCtrayectoriaA}

\begin{UCtrayectoriaA}[Fin del caso de uso]%
  {C}{El \refElem{aEstudiante} desafiado obtuvo un mayor puntaje que el actor}

  \Sistema Lanza el evento para dar las monedas que le corresponden al \refElem{aEstudiante} desafiado.
  \Sistema Lanza el evento para dar la experiencia que le corresponde al \refElem{aEstudiante} desafiado.
  \Sistema Redirige a la pantalla \refElem{IU-C12}.
  \Sistema Muestra el puntaje de cada participante \refElem{comp-1v1-gmdl-participacion.puntuacion}.

\end{UCtrayectoriaA}

\begin{UCtrayectoriaA}{D}{El actor no quiere abandonar el cuestionario}

  \Actor Presiona el botón {\bf Cancelar}.
  \Sistema Cierra el mensaje de alerta.
  \Sistema Regresa al paso \ref{CU-C10-contesta-cuestionario}

\end{UCtrayectoriaA}


% \ucstEnEdicion     Al terminar una revisión/aprobación con observaciones
%                    y al inicio del CU.
%
% \ucstEnRevision    Al terminar la edición del CU (version += 0.1).
% \ucstEnAprobacion  Al pasar la revision sin observaciones.
% \ucstAprobado      Al ser aprobado por el usuario (version += 1.0)

%\addfigure[(adaptado de {\it For The Win} \cite{ForTheWin})]%
%    {.4}{investigacion/images/forthewin}{fig:ForTheWin}%
%    {Jerarquía de elementos de juego segun For The Win}

\begin{UseCase}[%
Autor/Ricardo Naranjo,%
Version/0.1,%
Estado/\ucstEnRevision]%
%
{CU-C11}{Desafiar a un estudiante sin apostar (Competencia uno contra uno)}{%
%
 Permite al \refElem{aEstudiante}, \refElem{aProfesor} y al \refElem{aAdministrador} desafiar a un estudiante que se encuentra inscrito en el curso.
 Este caso de uso es una extensión del caso de uso \refElem{CU-C07} en la trayectoria alternativa A.}

	\UCitem[control]{Revisor}{ Sin asignar }
	\UCitem[control]{Último cambio}{ 13/NOV/19 }

 \UCsection{Atributos}

    \UCitem{Actor(es)}{%
        \refElem{aEstudiante},
        \refElem{aProfesor},
        \refElem{aAdministrador}
    }

	\UCitems{Propósito}{%
        \Titem Permitir al \refElem{aEstudiante}, \refElem{aProfesor} y al \refElem{aAdministrador} desafiar a un estudiante que forma parte de un curso.
        \Titem Permitir al \refElem{aEstudiante}, \refElem{aProfesor} y al \refElem{aAdministrador} terminar un desafío.
	}

	\UCitem{Entradas}{\imprimeUC{entrada}}

	\UCitems{Origen}{%
        \Titem Mouse
        \Titem Teclado
	}

	\UCitem{Salidas}{\imprimeUC{salida}}

	\UCitem{Destino}{%
		\refElem{IU-C10}
	}

	\UCitems{Precondiciones}{%
        \Titem El plugin de competencia uno contra uno debe estar instalado en moodle.
        \Titem La instancia de la actividad de competencia uno contra uno debe estar creada.
        % Realizar el caso de uso "listar actividades disponibles?"
        % \Titem Si se trata de una actualización de un plugin la versión de este debe
               % cumplir con la regla \refElem{BR-M02}.
	}

	\UCitems{Postcondiciones}{%
        \Titem Se muestra en la pantalla de historial del usuario \refElem{IU-C07} la partida que inició.%

	}

	\UCitem{Reglas de negocio}{\imprimeUC{regla}}

	\UCitems{Errores}{%
	}

	% \UCitem{Viene de}{% Indicar si el Caso de uso es primario o se extiende de otro. La mayoría se
					  % extienden de Login.
		% EJEMPLO: \refIdElem{PY-CU1} o Caso de uso primario.
	% 	\TODO Especificar.
	% }

 \UCsection[design]{Datos de Diseño}

	\UCitems[design]{Casos de Prueba}{%
        \Titem \refElem{CPC-C11}
	}

 \UCsection[admin]{Datos de Administración de Requerimiento}

	\UCitem[admin]{Observaciones}{}

\end{UseCase}

\subsubsection{Trayectorias del caso de uso}

\begin{UCtrayectoria}%
%

    \Actor Presiona el botón {\bf Desafiar} en el bloque del estudiante que quiere desafiar en la pantalla \refElem{IU-C02a}.

    \Sistema Inicia la partida y establece los valores correspondientes \refElem{comp-1v1-gmdl-partida}( \entrada{comp-1v1-gmdl-partida.gmdl-comp-vs-id} ), y una entrada por cada usuario de la partida \refElem{comp-1v1-gmdl-participacion} ( \entrada{comp-1v1-gmdl-participacion.gmdl-usuario-id}, \entrada{comp-1v1-gmdl-participacion.gmdl-partida-id}, \entrada{comp-1v1-gmdl-participacion.fecha-inicio}, \refElem{comp-1v1-gmdl-participacion.puntuacion} ).

    \Sistema Redirige a la pantalla del cuestionario del desafío \refElem{IU-C09}.

    \Actor Contesta las preguntas mostradas. \refTray{A}
    \label{CU-C10-contesta-cuestionario}
    \Actor Presiona el botón {\bf Terminar}.

    \Sistema Evalúa las respuestas ingresadas y calcula su puntuación.

    \Sistema Establece los valores correspondientes de la participación \refElem{comp-1v1-gmdl-participacion} (
    \entrada{comp-1v1-gmdl-participacion.fecha-fin},
    \entrada{comp-1v1-gmdl-participacion.puntuacion}).

    \Sistema Valida que el \refElem{aEstudiante} desafiado no haya terminado de contestar su cuestionario. \refTray{B}

    \Sistema Redirige a la pantalla \refElem{IU-C10}.

\end{UCtrayectoria}

\begin{UCtrayectoriaA}[Fin del caso de uso]%
  {A}{El actor desea salir del cuestionario sin terminarlo}

  \Actor Presiona cualquier botón, excepto el botón de {\bf Terminar}, que haga que abandone la pantalla.
  \Sistema Muestra mensaje de alerta \refElem{IU-C13}.
  \Actor Presiona el botón {\bf Abandonar}. \refTray{D}
  \Sistema Establece el valor correspondiente de la participación \refElem{comp-1v1-gmdl-participacion} (
  \refElem{comp-1v1-gmdl-participacion.fecha-fin}).
  \Sistema Redirige a la pantalla elegida por el actor.

\end{UCtrayectoriaA}

\begin{UCtrayectoriaA}[Fin del caso de uso]%
  {B}{El \refElem{aEstudiante} desafiado ya había terminado de contestar el cuestionario}

  \Sistema Calcula y otorga puntaje extra de acuerdo al tiempo tardado en contestar el cuestionario.
  \Sistema Comprueba que el actor haya obtenido un mayor puntaje que el \refElem{aEstudiante} desafiado. \refTray{C}
  \Sistema Lanza el evento para dar las monedas que le corresponden al actor.
  \Sistema Lanza el evento para dar la experiencia que le corresponde al actor.
  \Sistema Redirige a la pantalla \refElem{IU-C11}.
  \Sistema Muestra el puntaje de cada participante \salida{comp-1v1-gmdl-participacion.puntuacion}.

\end{UCtrayectoriaA}

\begin{UCtrayectoriaA}[Fin del caso de uso]%
  {C}{El \refElem{aEstudiante} desafiado obtuvo un mayor puntaje que el actor}

  \Sistema Lanza el evento para dar las monedas que le corresponde al \refElem{aEstudiante} desafiado.
  \Sistema Lanza el evento para dar la experiencia que le corresponde al \refElem{aEstudiante} desafiado.
  \Sistema Redirige a la pantalla \refElem{IU-C12}.
  \Sistema Muestra el puntaje de cada participante \refElem{comp-1v1-gmdl-participacion.puntuacion}.

\end{UCtrayectoriaA}

\begin{UCtrayectoriaA}{D}{El actor no quiere abandonar el cuestionario}

  \Actor Presiona el botón {\bf Cancelar}.
  \Sistema Cierra el mensaje de alerta.
  \Sistema Regresa al paso \ref{CU-C10-contesta-cuestionario}

\end{UCtrayectoriaA}


% \ucstEnEdicion     Al terminar una revisión/aprobación con observaciones
%                    y al inicio del CU.
%
% \ucstEnRevision    Al terminar la edición del CU (version += 0.1).
% \ucstEnAprobacion  Al pasar la revision sin observaciones.
% \ucstAprobado      Al ser aprobado por el usuario (version += 1.0)

%\addfigure[(adaptado de {\it For The Win} \cite{ForTheWin})]%
%    {.4}{investigacion/images/forthewin}{fig:ForTheWin}%
%    {Jerarquía de elementos de juego segun For The Win}

\begin{UseCase}[%
Autor/Ricardo Naranjo,%
Version/0.1,%
Estado/\ucstEnRevision]%
%
{CU-C12}{Ver estado de instancia de actividad (Competencia uno contra sistema)}{%
%
 Permite al \refElem{aEstudiante}, \refElem{aProfesor} y al \refElem{aAdministrador} ver el estado actual de una instancia de la actividad competencia uno contra sistema en el curso.
 Este caso de uso es una extensión del caso de uso {\it Ver curso} que es propio de moodle.}

	\UCitem[control]{Revisor}{ Sin asignar }
	\UCitem[control]{Último cambio}{ 13/NOV/19 }

 \UCsection{Atributos}

    \UCitem{Actor(es)}{%
        \refElem{aEstudiante},
        \refElem{aProfesor},
        \refElem{aAdministrador}
    }

	\UCitems{Propósito}{%
        \Titem Permitir al \refElem{aEstudiante}, \refElem{aProfesor} y al \refElem{aAdministrador} ver el estado actual de una instancia de la actividad de competencia uno contra sistema.
	}

	\UCitem{Entradas}{\imprimeUC{entrada}}

	\UCitems{Origen}{%
        \Titem Mouse
	}

	\UCitem{Salidas}{\imprimeUC{salida} \begin{itemize}
    \item {\bf Icono de dificultad vencida}\IUCpuvencida
    \item {\bf Icono de dificultad no vencida}\IUCpunovencida
  \end{itemize} }

	\UCitem{Destino}{%
		\refElem{IU-C04}
	}

	\UCitems{Precondiciones}{%
        \Titem El plugin de competencia uno contra sistema debe estar instalado en moodle.
        \Titem La instancia de la actividad de competencia uno contra sistema debe estar creada.
        % Realizar el caso de uso "listar actividades disponibles?"
        % \Titem Si se trata de una actualización de un plugin la versión de este debe
               % cumplir con la regla \refElem{BR-M02}.
	}

	\UCitems{Postcondiciones}{%
        \Titem La pantalla principal de la instancia de la actividad de competencia uno contra sistema \refElem{IU-C04} debe mostrar los datos pertinentes al usuario que realizó el caso de uso.%

	}

	\UCitem{Reglas de negocio}{\imprimeUC{regla}}

	\UCitems{Errores}{%
	}

	% \UCitem{Viene de}{% Indicar si el Caso de uso es primario o se extiende de otro. La mayoría se
					  % extienden de Login.
		% EJEMPLO: \refIdElem{PY-CU1} o Caso de uso primario.
	% 	\TODO Especificar.
	% }

 \UCsection[design]{Datos de Diseño}

	\UCitems[design]{Casos de Prueba}{%
        \Titem \refElem{CPC-C12}
	}

 \UCsection[admin]{Datos de Administración de Requerimiento}

	\UCitem[admin]{Observaciones}{}

\end{UseCase}

\subsubsection{Trayectorias del caso de uso}

\begin{UCtrayectoria}%
%

    \Actor Presiona el nombre de la instancia a la que quiere acceder en la pantalla \refElem{IU-M07}.

    \Sistema Redirige a la pantalla principal de la instancia \refElem{IU-C04}.

    \Sistema Muestra las dificultades del sistema vencido por medio del icono \IUCpuvencida.

    \Sistema Muestra las dificultades del sistema que no han sido vencidas por medio del icono \IUCpunovencida.

    \Sistema Muestra en la sección {\bf Desafiar computadora} las dificultades que se pueden desafiar \salida{comp-cpu-gmdl-dificultad-cpu}

\end{UCtrayectoria}

\subsubsection{Puntos de extensión}

\UCExtensionPoint{Ver historial de sus partidas}{%

    El \refElem{aAdministrador}, \refElem{aProfesor} o \refElem{aEstudiante} desea ver su historial de las partidas de competencia uno contra sistema.
%
    }{Al final de la trayectoria principal del caso de uso.
%
    }{\refElem{CU-C15}}


\UCExtensionPoint{Ver tabla de puntuaciones}{%

    El \refElem{aAdministrador}, \refElem{aProfesor} o \refElem{aEstudiante} desea ver la tabla de puntuaciones de una instancia de competencia uno contra sistema.
%
    }{Al final de la trayectoria principal del caso de uso.
%
    }{\refElem{CU-C14}}

  \UCExtensionPoint{Desafiar al sistema}{%

      El \refElem{aAdministrador}, \refElem{aProfesor} o \refElem{aEstudiante} desea desafiar al sistema en alguna de sus dificultades.
  %
      }{Al final de la trayectoria principal del caso de uso.
  %
      }{\refElem{CU-C13}}


% \ucstEnEdicion     Al terminar una revisión/aprobación con observaciones
%                    y al inicio del CU.
%
% \ucstEnRevision    Al terminar la edición del CU (version += 0.1).
% \ucstEnAprobacion  Al pasar la revision sin observaciones.
% \ucstAprobado      Al ser aprobado por el usuario (version += 1.0)

%\addfigure[(adaptado de {\it For The Win} \cite{ForTheWin})]%
%    {.4}{investigacion/images/forthewin}{fig:ForTheWin}%
%    {Jerarquía de elementos de juego segun For The Win}

\begin{UseCase}[%
Autor/Ricardo Naranjo,%
Version/0.1,%
Estado/\ucstEnRevision]%
%
{CU-C13}{Desafiar al sistema (Competencia uno contra sistema)}{%
%
 Permite al \refElem{aEstudiante}, \refElem{aProfesor} y al \refElem{aAdministrador} desafiar al sistema.
 Este caso de uso es una extensión del caso de uso \refElem{CU-C12}.}

	\UCitem[control]{Revisor}{ Sin asignar }
	\UCitem[control]{Último cambio}{ 13/NOV/19 }

 \UCsection{Atributos}

    \UCitem{Actor(es)}{%
        \refElem{aEstudiante},
        \refElem{aProfesor},
        \refElem{aAdministrador}
    }

	\UCitems{Propósito}{%
        \Titem Permitir al \refElem{aEstudiante}, \refElem{aProfesor} y al \refElem{aAdministrador} desafiar al sistema.
        \Titem Permitir al \refElem{aEstudiante}, \refElem{aProfesor} y al \refElem{aAdministrador} terminar un desafío.
	}

	\UCitem{Entradas}{\imprimeUC{entrada}}

	\UCitems{Origen}{%
        \Titem Mouse
        \Titem Teclado
	}

	\UCitem{Salidas}{\imprimeUC{salida}}

	\UCitem{Destino}{%
		\refElem{IU-C10}
	}

	\UCitems{Precondiciones}{%
        \Titem El plugin de competencia uno contra sistema debe estar instalado en moodle.
        \Titem La instancia de la actividad de competencia uno contra sistema debe estar creada.
        % Realizar el caso de uso "listar actividades disponibles?"
        % \Titem Si se trata de una actualización de un plugin la versión de este debe
               % cumplir con la regla \refElem{BR-M02}.
	}

	\UCitems{Postcondiciones}{%
        \Titem Se muestra en la pantalla de historial del usuario \refElem{IU-C14} el desafío nuevo.%
	}

	\UCitem{Reglas de negocio}{\imprimeUC{regla}}

	\UCitems{Errores}{%
	}

	% \UCitem{Viene de}{% Indicar si el Caso de uso es primario o se extiende de otro. La mayoría se
					  % extienden de Login.
		% EJEMPLO: \refIdElem{PY-CU1} o Caso de uso primario.
	% 	\TODO Especificar.
	% }

 \UCsection[design]{Datos de Diseño}

	\UCitems[design]{Casos de Prueba}{%
        \Titem \refElem{CPC-C13}
	}

 \UCsection[admin]{Datos de Administración de Requerimiento}

	\UCitem[admin]{Observaciones}{}

\end{UseCase}

\subsubsection{Trayectorias del caso de uso}

\begin{UCtrayectoria}%
%

    \Actor Selecciona la dificultad deseada en el campo {\bf Seleccione una dificultad} de la pantalla \refElem{IU-C04}.

    \Actor Presiona el botón {\bf Empezar}.

    \Sistema Inicia la partida y establece los valores correspondientes \refElem{comp-cpu-gmdl-intento}( \entrada{comp-cpu-gmdl-intento.gmdl-dificultad-cpu-id}, \entrada{comp-cpu-gmdl-intento.gmdl-comp-cpu-id}, \entrada{comp-cpu-gmdl-intento.gmdl-usuario-id}, \entrada{comp-cpu-gmdl-intento.fecha-inicio} ).

    \Sistema Redirige a la pantalla del cuestionario del desafío \refElem{IU-C09}.

    \Actor Contesta las preguntas mostradas.

    \Actor Presiona el botón {\bf Terminar}.

    \Sistema Evalúa las respuestas ingresadas y calcula su puntuación.

    \Sistema Contesta el cuestionario y calcula su puntuación.

    \Sistema Establece los valores correspondientes del intento \refElem{comp-cpu-gmdl-intento} (
    \entrada{comp-cpu-gmdl-intento.fecha-fin},
    \entrada{comp-cpu-gmdl-intento.puntuacion-cpu},
    \entrada{comp-cpu-gmdl-intento.puntuacion-usuario}).

    \Sistema Valida que el actor haya sacado más o igual puntuación que el sistema. \refTray{A}

    \Sistema Lanza el evento para dar las monedas que le corresponden al actor.
    \Sistema Lanza el evento para dar la experiencia que le corresponde al actor.

    \Sistema Redirige a la pantalla \refElem{IU-C11}.
    \Sistema Muestra el puntaje del actor y del sistema \salida{comp-cpu-gmdl-intento.puntuacion-usuario}, \salida{comp-cpu-gmdl-intento.puntuacion-cpu}.

\end{UCtrayectoria}

\begin{UCtrayectoriaA}[Fin del caso de uso]%
  {A}{El sistema obtuvo un mayor puntaje que el actor}

  \Sistema Redirige a la pantalla \refElem{IU-C12}.
  \Sistema Muestra el puntaje del actor y del sistema \refElem{comp-cpu-gmdl-intento.puntuacion-usuario}, \refElem{comp-cpu-gmdl-intento.puntuacion-cpu}.

\end{UCtrayectoriaA}


% \ucstEnEdicion     Al terminar una revisión/aprobación con observaciones
%                    y al inicio del CU.
%
% \ucstEnRevision    Al terminar la edición del CU (version += 0.1).
% \ucstEnAprobacion  Al pasar la revision sin observaciones.
% \ucstAprobado      Al ser aprobado por el usuario (version += 1.0)

%\addfigure[(adaptado de {\it For The Win} \cite{ForTheWin})]%
%    {.4}{investigacion/images/forthewin}{fig:ForTheWin}%
%    {Jerarquía de elementos de juego segun For The Win}

\begin{UseCase}[%
Autor/Ricardo Naranjo,%
Version/0.1,%
Estado/\ucstEnRevision]%
%
{CU-C14}{Ver tabla de puntuaciones (Competencia uno contra sistema)}{%
%
 Permite al \refElem{aEstudiante}, \refElem{aProfesor} y al \refElem{aAdministrador} ver la tabla de puntuaciones de una instancia de la actividad competencia uno contra sistema en el curso.
 Este caso de uso es una extensión del caso de uso \refElem{CU-C12}.}

	\UCitem[control]{Revisor}{ Sin asignar }
	\UCitem[control]{Último cambio}{ 13/NOV/19 }

 \UCsection{Atributos}

    \UCitem{Actor(es)}{%
        \refElem{aEstudiante},
        \refElem{aProfesor},
        \refElem{aAdministrador}
    }

	\UCitems{Propósito}{%
        \Titem Permitir al \refElem{aEstudiante}, \refElem{aProfesor} y al \refElem{aAdministrador} ver la tabla de puntuaciones de una instancia de la actividad de competencia uno contra sistema.
	}

	\UCitem{Entradas}{\imprimeUC{entrada}}

	\UCitems{Origen}{%
        \Titem Mouse
	}

	\UCitem{Salidas}{\imprimeUC{salida}}

	\UCitem{Destino}{%
		\refElem{IU-C15}
	}

	\UCitems{Precondiciones}{%
        \Titem El plugin de competencia uno contra sistema debe estar instalado en moodle.
        \Titem La instancia de la actividad de competencia uno contra sistema debe estar creada.
        % Realizar el caso de uso "listar actividades disponibles?"
        % \Titem Si se trata de una actualización de un plugin la versión de este debe
               % cumplir con la regla \refElem{BR-M02}.
	}

	\UCitems{Postcondiciones}{%
        \Titem Se muestra la pantalla de tabla de puntuaciones de la instancia de la actividad de competencia uno contra sistema \refElem{IU-C15} de acuerdo a la dificultad seleccionada.%

	}

	\UCitem{Reglas de negocio}{\imprimeUC{regla}}

	\UCitems{Errores}{%
	}

	% \UCitem{Viene de}{% Indicar si el Caso de uso es primario o se extiende de otro. La mayoría se
					  % extienden de Login.
		% EJEMPLO: \refIdElem{PY-CU1} o Caso de uso primario.
	% 	\TODO Especificar.
	% }

 \UCsection[design]{Datos de Diseño}

	\UCitems[design]{Casos de Prueba}{%
        \Titem \refElem{CPC-C14}
	}

 \UCsection[admin]{Datos de Administración de Requerimiento}

	\UCitem[admin]{Observaciones}{}

\end{UseCase}

\subsubsection{Trayectorias del caso de uso}

\begin{UCtrayectoria}%
%

    \Actor Presiona el botón {\bf Puntuaciones} de la pantalla \refElem{IU-C04}.

    \Sistema Redirige a la pantalla de la instancia \refElem{IU-C15}.

    \Sistema Muestra un campo {\bf Seleccione una dificultad} que está por defecto en fácil. \refTray{A}

    \Sistema Muestra el \salida{comp-cpu-gmdl-dificultad-cpu.nombre} seleccionada en el campo {\bf Seleccione una dificultad}
    \label{CU-C14-muestra-informacion}

    \Sistema Muestra una tabla que presenta una fila, que contiene \salida{Posición del usuario}, \salida{Imagen de perfil}, \salida{Nombre de usuario}, \salida{comp-cpu-gmdl-intento.puntuacion-usuario} , por cada primer intento de cada usuario en esta instancia de la competencia en la dificultad seleccionada.

    \Sistema Muestra una tabla que presenta una fila, que contiene {\bf Posición del usuario}, {\bf Imagen de perfil}, {\bf Nombre de usuario}, \refElem{comp-cpu-gmdl-intento.puntuacion-usuario} , por cada mejor intento de cada usuario en esta instancia de la competencia en la dificultad seleccionada.

\end{UCtrayectoria}

\begin{UCtrayectoriaA}{A}{El actor desea seleccionar otra dificultad a ver}

  \Actor Selecciona en el campo {\bf Seleccione una dificultad} la dificultad deseada.
  \Sistema Regresa al paso \ref{CU-C14-muestra-informacion}

\end{UCtrayectoriaA}


% \ucstEnEdicion     Al terminar una revisión/aprobación con observaciones
%                    y al inicio del CU.
%
% \ucstEnRevision    Al terminar la edición del CU (version += 0.1).
% \ucstEnAprobacion  Al pasar la revision sin observaciones.
% \ucstAprobado      Al ser aprobado por el usuario (version += 1.0)

%\addfigure[(adaptado de {\it For The Win} \cite{ForTheWin})]%
%    {.4}{investigacion/images/forthewin}{fig:ForTheWin}%
%    {Jerarquía de elementos de juego segun For The Win}

\begin{UseCase}[%
Autor/Ricardo Naranjo,%
Version/0.1,%
Estado/\ucstEnRevision]%
%
{CU-C15}{Ver historial de sus partidas (Competencia uno contra sistema)}{%
%
 Permite al \refElem{aEstudiante}, \refElem{aProfesor} y al \refElem{aAdministrador} ver su historial de una instancia de la actividad competencia uno contra sistema en el curso.
 Este caso de uso es una extensión del caso de uso \refElem{CU-C12}.}

	\UCitem[control]{Revisor}{ Sin asignar }
	\UCitem[control]{Último cambio}{ 13/NOV/19 }

 \UCsection{Atributos}

    \UCitem{Actor(es)}{%
        \refElem{aEstudiante},
        \refElem{aProfesor},
        \refElem{aAdministrador}
    }

	\UCitems{Propósito}{%
        \Titem Permitir al \refElem{aEstudiante}, \refElem{aProfesor} y al \refElem{aAdministrador} ver su historial de una instancia de la actividad de competencia uno contra sistema.
	}

	\UCitem{Entradas}{\imprimeUC{entrada}}

	\UCitems{Origen}{%
        \Titem Mouse
	}

	\UCitem{Salidas}{\imprimeUC{salida}\begin{itemize}
    \item {\bf Icono de dificultad vencida}\IUCpuvencida
    \item {\bf Icono de dificultad no vencida}\IUCpunovencida
  \end{itemize}}

	\UCitem{Destino}{%
		\refElem{IU-C14}
	}

	\UCitems{Precondiciones}{%
        \Titem El plugin de competencia uno contra sistema debe estar instalado en moodle.
        \Titem La instancia de la actividad de competencia uno contra sistema debe estar creada.
        % Realizar el caso de uso "listar actividades disponibles?"
        % \Titem Si se trata de una actualización de un plugin la versión de este debe
               % cumplir con la regla \refElem{BR-M02}.
	}

	\UCitems{Postcondiciones}{%
        \Titem La pantalla de historial de la instancia de la actividad de competencia uno contra sistema \refElem{IU-C07} debe mostrar los datos pertinentes al actor.%

	}

	\UCitem{Reglas de negocio}{\imprimeUC{regla}}

	\UCitems{Errores}{%
	}

	% \UCitem{Viene de}{% Indicar si el Caso de uso es primario o se extiende de otro. La mayoría se
					  % extienden de Login.
		% EJEMPLO: \refIdElem{PY-CU1} o Caso de uso primario.
	% 	\TODO Especificar.
	% }

 \UCsection[design]{Datos de Diseño}

	\UCitems[design]{Casos de Prueba}{%
        \Titem \refElem{CPC-C15}
	}

 \UCsection[admin]{Datos de Administración de Requerimiento}

	\UCitem[admin]{Observaciones}{}

\end{UseCase}

\subsubsection{Trayectorias del caso de uso}

\begin{UCtrayectoria}%
%

    \Actor Presiona el botón {\bf Historial} de la pantalla \refElem{IU-C04}.

    \Sistema Redirige a la pantalla historial de la instancia \refElem{IU-C14}.

    \Sistema Muestra en la tabla {\bf Intentos realizados}: Una fila por cada intento del actor que contiene: \salida{comp-cpu-gmdl-dificultad-cpu.nombre}, \salida{comp-cpu-gmdl-intento.puntuacion-usuario}, \salida{comp-cpu-gmdl-intento.puntuacion-cpu}, el icono \IUCpuvencida si la puntuación del usuario en ese intento es mayor o igual que la puntuación del sistema. \refTray{A}

\end{UCtrayectoria}

\begin{UCtrayectoriaA}[Fin del caso de uso]{A}{La puntuación del usuario en el intento fue menor que la puntuación del sistema}

  \Sistema Muestra en la tabla {\bf Intentos realizados}: Una fila por cada intento del actor que contiene: \refElem{comp-cpu-gmdl-dificultad-cpu.nombre}, \refElem{comp-cpu-gmdl-intento.puntuacion-usuario}, \refElem{comp-cpu-gmdl-intento.puntuacion-cpu}, el icono \IUCpunovencida.

\end{UCtrayectoriaA}


% \ucstEnEdicion     Al terminar una revisión/aprobación con observaciones
%                    y al inicio del CU.
%
% \ucstEnRevision    Al terminar la edición del CU (version += 0.1).
% \ucstEnAprobacion  Al pasar la revision sin observaciones.
% \ucstAprobado      Al ser aprobado por el usuario (version += 1.0)

%\addfigure[(adaptado de {\it For The Win} \cite{ForTheWin})]%
%    {.4}{investigacion/images/forthewin}{fig:ForTheWin}%
%    {Jerarquía de elementos de juego segun For The Win}

\begin{UseCase}[%
Autor/Ricardo Naranjo,%
Version/0.1,%
Estado/\ucstEnRevision]%
%
{CU-C16}{Aceptar un desafío apostando (Competencia uno contra uno)}{%
%
 Permite al \refElem{aEstudiante} aceptar un desafío y a su vez se otorga la posibilidad de apostar en la partida.
 Este caso de uso es una extensión del caso de uso \refElem{CU-C07}.}

	\UCitem[control]{Revisor}{ Sin asignar }
	\UCitem[control]{Último cambio}{ 13/NOV/19 }

 \UCsection{Atributos}

    \UCitem{Actor(es)}{%
        \refElem{aEstudiante}
    }

	\UCitems{Propósito}{%
        \Titem Permitir al \refElem{aEstudiante} aceptar un desafío y apostar la cantidad de monedas decidida por el usuario.
        \Titem Permitir al \refElem{aEstudiante} terminar un desafío.
	}

	\UCitem{Entradas}{\imprimeUC{entrada}}

	\UCitems{Origen}{%
        \Titem Mouse
        \Titem Teclado
	}

	\UCitem{Salidas}{\imprimeUC{salida}}

	\UCitem{Destino}{%
		\refElem{IU-C10}
	}

	\UCitems{Precondiciones}{%
        \Titem El plugin de competencia uno contra uno debe estar instalado en moodle.
        \Titem La instancia de la actividad de competencia uno contra uno debe estar creada.
        % Realizar el caso de uso "listar actividades disponibles?"
        % \Titem Si se trata de una actualización de un plugin la versión de este debe
               % cumplir con la regla \refElem{BR-M02}.
	}

	\UCitems{Postcondiciones}{%
        \Titem Se muestra en la pantalla de historial del usuario \refElem{IU-C07} la nueva partida.%

	}

	\UCitem{Reglas de negocio}{\imprimeUC{regla}}

	\UCitems{Errores}{%
	}

	% \UCitem{Viene de}{% Indicar si el Caso de uso es primario o se extiende de otro. La mayoría se
					  % extienden de Login.
		% EJEMPLO: \refIdElem{PY-CU1} o Caso de uso primario.
	% 	\TODO Especificar.
	% }

 \UCsection[design]{Datos de Diseño}

	\UCitems[design]{Casos de Prueba}{%
        \Titem \refElem{CPC-C16}
	}

 \UCsection[admin]{Datos de Administración de Requerimiento}

	\UCitem[admin]{Observaciones}{}

\end{UseCase}

\subsubsection{Trayectorias del caso de uso}

\begin{UCtrayectoria}%
%

    \Actor Ingresa la cantidad de monedas que desea apostar en el campo "Monedas a apostar" en el bloque del usuario del que quiere aceptar el desafío en la pantalla \refElem{IU-C02} en la sección "Desafíos pendientes".

    \Actor Presiona el botón {\bf Aceptar desafío}.

    \Sistema Inicia la partida y establece los valores correspondientes en \refElem{comp-1v1-gmdl-participacion} ( \entrada{comp-1v1-gmdl-participacion.fecha-inicio} ) y agrega la apuesta \refElem{comp-1v1-gmdl-apuesta}( \entrada{comp-1v1-gmdl-apuesta.gmdl-participacion-id}, \entrada{comp-1v1-gmdl-apuesta.monedas-plata}, \entrada{comp-1v1-gmdl-apuesta.activa} ).

    \Sistema Redirige a la pantalla del cuestionario del desafío \refElem{IU-C09}.

    \Actor Contesta las preguntas mostradas. \refTray{A}
    \label{CU-C10-contesta-cuestionario}
    \Actor Presiona el botón {\bf Terminar}.

    \Sistema Evalúa las respuestas ingresadas y calcula su puntuación.

    \Sistema Establece los valores correspondientes de la participación \refElem{comp-1v1-gmdl-participacion} (
    \entrada{comp-1v1-gmdl-participacion.fecha-fin},
    \entrada{comp-1v1-gmdl-participacion.puntuacion}).

    \Sistema Valida que el usuario que lo desafió haya terminado de contestar su cuestionario. \refTray{B}

    \Sistema Calcula y otorga puntaje extra de acuerdo al tiempo tardado en contestar el cuestionario.
    \Sistema Comprueba que el actor haya obtenido un mayor puntaje que el usuario desafiante. \refTray{C}
    \Sistema Lanza el evento para dar las monedas que le corresponden al actor.
    \Sistema Lanza el evento para dar la experiencia que le corresponde al actor.
    \Sistema Redirige a la pantalla \refElem{IU-C11}.
    \Sistema Muestra el puntaje de cada participante \salida{comp-1v1-gmdl-participacion.puntuacion}.

\end{UCtrayectoria}

\begin{UCtrayectoriaA}[Fin del caso de uso]%
  {A}{El actor desea salir del cuestionario sin terminarlo}

  \Actor Presiona cualquier botón, excepto el botón de {\bf Terminar}, que haga que abandone la pantalla.
  \Sistema Muestra mensaje de alerta \refElem{IU-C13}.
  \Actor Presiona el botón {\bf Abandonar}. \refTray{D}
  \Sistema Establece el valor correspondiente de la participación \refElem{comp-1v1-gmdl-participacion} (
  \refElem{comp-1v1-gmdl-participacion.fecha-fin}).
  \Sistema Redirige a la pantalla elegida por el actor.

\end{UCtrayectoriaA}

\begin{UCtrayectoriaA}[Fin del caso de uso]%
  {B}{El usuario que lo desafió no ha terminado de contestar su cuestionario}

  \Sistema Redirige a la pantalla \refElem{IU-C10}

\end{UCtrayectoriaA}

\begin{UCtrayectoriaA}[Fin del caso de uso]%
  {C}{El usuario desafiante obtuvo un mayor puntaje que el actor}

  \Sistema Lanza el evento para dar las monedas que le corresponden al usuario desafiante.
  \Sistema Lanza el evento para dar la experiencia que le corresponde al usuario desafiante.
  \Sistema Redirige a la pantalla \refElem{IU-C12}.
  \Sistema Muestra el puntaje de cada participante \refElem{comp-1v1-gmdl-participacion.puntuacion}.

\end{UCtrayectoriaA}

\begin{UCtrayectoriaA}{D}{El actor no quiere abandonar el cuestionario}

  \Actor Presiona el botón {\bf Cancelar}.
  \Sistema Cierra el mensaje de alerta.
  \Sistema Regresa al paso \ref{CU-C10-contesta-cuestionario}

\end{UCtrayectoriaA}


% \ucstEnEdicion     Al terminar una revisión/aprobación con observaciones
%                    y al inicio del CU.
%
% \ucstEnRevision    Al terminar la edición del CU (version += 0.1).
% \ucstEnAprobacion  Al pasar la revision sin observaciones.
% \ucstAprobado      Al ser aprobado por el usuario (version += 1.0)

%\addfigure[(adaptado de {\it For The Win} \cite{ForTheWin})]%
%    {.4}{investigacion/images/forthewin}{fig:ForTheWin}%
%    {Jerarquía de elementos de juego segun For The Win}

\begin{UseCase}[%
Autor/Ricardo Naranjo,%
Version/0.1,%
Estado/\ucstEnRevision]%
%
{CU-C17}{Aceptar un desafío sin apostar (Competencia uno contra uno)}{%
%
 Permite al \refElem{aEstudiante} aceptar un desafío.
 Este caso de uso es una extensión del caso de uso \refElem{CU-C07} en la trayectoria alternativa A.}

	\UCitem[control]{Revisor}{ Sin asignar }
	\UCitem[control]{Último cambio}{ 13/NOV/19 }

 \UCsection{Atributos}

    \UCitem{Actor(es)}{%
        \refElem{aEstudiante}
    }

	\UCitems{Propósito}{%
        \Titem Permitir al \refElem{aEstudiante} aceptar un desafío.
        \Titem Permitir al \refElem{aEstudiante} terminar un desafío.
	}

	\UCitem{Entradas}{\imprimeUC{entrada}}

	\UCitems{Origen}{%
        \Titem Mouse
        \Titem Teclado
	}

	\UCitem{Salidas}{\imprimeUC{salida}}

	\UCitem{Destino}{%
		\refElem{IU-C11}
	}

	\UCitems{Precondiciones}{%
        \Titem El plugin de competencia uno contra uno debe estar instalado en moodle.
        \Titem La instancia de la actividad de competencia uno contra uno debe estar creada.
        % Realizar el caso de uso "listar actividades disponibles?"
        % \Titem Si se trata de una actualización de un plugin la versión de este debe
               % cumplir con la regla \refElem{BR-M02}.
	}

	\UCitems{Postcondiciones}{%
        \Titem Se muestra en la pantalla de historial del usuario \refElem{IU-C07} la nueva partida.%

	}

	\UCitem{Reglas de negocio}{\imprimeUC{regla}}

	\UCitems{Errores}{%
	}

	% \UCitem{Viene de}{% Indicar si el Caso de uso es primario o se extiende de otro. La mayoría se
					  % extienden de Login.
		% EJEMPLO: \refIdElem{PY-CU1} o Caso de uso primario.
	% 	\TODO Especificar.
	% }

 \UCsection[design]{Datos de Diseño}

	\UCitems[design]{Casos de Prueba}{%
        \Titem \refElem{CPC-C17}
	}

 \UCsection[admin]{Datos de Administración de Requerimiento}

	\UCitem[admin]{Observaciones}{}

\end{UseCase}

\subsubsection{Trayectorias del caso de uso}

\begin{UCtrayectoria}%
%

    \Actor Presiona el botón {\bf Aceptar desafío} en el bloque del usuario del que quiere aceptar el desafío en la pantalla \refElem{IU-C02a} en la sección "Desafíos pendientes"..

    \Sistema Inicia la partida y establece los valores correspondientes en \refElem{comp-1v1-gmdl-participacion} ( \entrada{comp-1v1-gmdl-participacion.fecha-inicio} ).

    \Sistema Redirige a la pantalla del cuestionario del desafío \refElem{IU-C09}.

    \Actor Contesta las preguntas mostradas. \refTray{A}
    \label{CU-C10-contesta-cuestionario}
    \Actor Presiona el botón {\bf Terminar}.

    \Sistema Evalúa las respuestas ingresadas y calcula su puntuación.

    \Sistema Establece los valores correspondientes de la participación \refElem{comp-1v1-gmdl-participacion} (
    \entrada{comp-1v1-gmdl-participacion.fecha-fin},
    \entrada{comp-1v1-gmdl-participacion.puntuacion}).

    \Sistema Valida que el usuario que lo desafió haya terminado de contestar su cuestionario. \refTray{B}

    \Sistema Calcula y otorga puntaje extra de acuerdo al tiempo tardado en contestar el cuestionario.
    \Sistema Comprueba que el actor haya obtenido un mayor puntaje que el usuario desafiante. \refTray{C}
    \Sistema Lanza el evento para dar las monedas que le corresponden al actor.
    \Sistema Lanza el evento para dar la experiencia que le corresponde al actor.
    \Sistema Redirige a la pantalla \refElem{IU-C11}.
    \Sistema Muestra el puntaje de cada participante \salida{comp-1v1-gmdl-participacion.puntuacion}.

\end{UCtrayectoria}

\begin{UCtrayectoriaA}[Fin del caso de uso]%
  {A}{El actor desea salir del cuestionario sin terminarlo}

  \Actor Presiona cualquier botón, excepto el botón de {\bf Terminar}, que haga que abandone la pantalla.
  \Sistema Muestra mensaje de alerta \refElem{IU-C13}.
  \Actor Presiona el botón {\bf Abandonar}. \refTray{D}
  \Sistema Establece el valor correspondiente de la participación \refElem{comp-1v1-gmdl-participacion} (
  \refElem{comp-1v1-gmdl-participacion.fecha-fin}).
  \Sistema Redirige a la pantalla elegida por el actor.

\end{UCtrayectoriaA}

\begin{UCtrayectoriaA}[Fin del caso de uso]%
  {B}{El usuario que lo desafió no ha terminado de contestar su cuestionario}

  \Sistema Redirige a la pantalla \refElem{IU-C10}

\end{UCtrayectoriaA}

\begin{UCtrayectoriaA}[Fin del caso de uso]%
  {C}{El usuario desafiante obtuvo un mayor puntaje que el actor}

  \Sistema Lanza el evento para dar las monedas que le corresponden al usuario desafiante.
  \Sistema Lanza el evento para dar la experiencia que le corresponde al usuario desafiante.
  \Sistema Redirige a la pantalla \refElem{IU-C12}.
  \Sistema Muestra el puntaje de cada participante \refElem{comp-1v1-gmdl-participacion.puntuacion}.

\end{UCtrayectoriaA}

\begin{UCtrayectoriaA}{D}{El actor no quiere abandonar el cuestionario}

  \Actor Presiona el botón {\bf Cancelar}.
  \Sistema Cierra el mensaje de alerta.
  \Sistema Regresa al paso \ref{CU-C10-contesta-cuestionario}

\end{UCtrayectoriaA}


% =========================================================
\clearpage
\subsection{Diseño}

\subsubsection{Interfaces del módulo de competencia}

    
<<<<<<< HEAD
\subsubsection{IU-M07 Pantalla principal del curso}

 Descripción ...

    \IUfig{1}{modulos/moodle/IU/p_principal_curso}{IU-M07}{Pantalla principal del curso}
=======
\subsubsection{IU-M07 Página Inicial del Sitio}

 Descripción ...

    \IUfig{1}{modulos/moodle/IU/PaginaInicial}{IU-M07}{Página inicial del Sitio}
>>>>>>> fd9af89a23d9b2fb18c947b44db69846e63131db

\subsubsection{Elementos Relevantes}

    \begin{itemize}
    \item {\bf Lorem ipsum}
        ...
    \end{itemize}

\subsubsection{Acciones relevantes}

    \begin{itemize}
    \item {\bf Lorem ipsum}
        ...
    \end{itemize}

\clearpage

    
\subsubsection{IU-M08aa Pantalla principal del curso con la edición activada}

 Descripción ...

    \IUfig{1}{modulos/moodle/IU/p_principal_curso_editar}{IU-M08aa}{ Pantalla principal del curso con la edición activada}

\subsubsection{Elementos Relevantes}

    \begin{itemize}
    \item {\bf Lorem ipsum}
        ...
    \end{itemize}

\subsubsection{Acciones relevantes}

    \begin{itemize}
    \item {\bf Lorem ipsum}
        ...
    \end{itemize}

\clearpage
  % Configuraciones
    
\subsubsection{IU-M07a Pantalla de listado de actividades disponibles}

 Descripción ...

    \IUfig{.5}{modulos/moodle/IU/p_listado_actividades}{IU-M07a}{Pantalla de listado de actividades disponibles}

\subsubsection{Elementos Relevantes}

    \begin{itemize}
    \item {\bf Lorem ipsum}
        ...
    \end{itemize}

\subsubsection{Acciones relevantes}

    \begin{itemize}
    \item {\bf Lorem ipsum}
        ...
    \end{itemize}

\clearpage

    
\subsubsection{IU-M08b Menú desplegable al presionar el botón Editar}

 Descripción ...

    \IUfig{.3}{modulos/moodle/IU/md_boton_editar_instancia}{IU-M08b}{Menú desplegable al presionar el botón Editar}

\subsubsection{Elementos Relevantes}

    \begin{itemize}
    \item {\bf Lorem ipsum}
        ...
    \end{itemize}

\subsubsection{Acciones relevantes}

    \begin{itemize}
    \item {\bf Lorem ipsum}
        ...
    \end{itemize}

\clearpage

    
\subsubsection{IU-M08c Mensaje de confirmación al presionar el botón Eliminar}

 Este mensaje de confirmación se muestra para reducir la cantidad de errores que ocurren cuando se presiona el botón equivocado en el menú \refElem{IU-M08b}.

    \IUfig{.3}{modulos/moodle/IU/m_confirmacion_eliminar_instancia}{IU-M08c}{Mensaje de confirmación al presionar el botón Eliminar}

\subsubsection{Elementos Relevantes}

    \begin{itemize}
    \item {\bf Si}
    \item {\bf No}

    \end{itemize}

\subsubsection{Acciones relevantes}

    \begin{itemize}
    \item {\bf Confirmar la eliminación de la instancia}
    \item {\bf Cancelar la eliminación de la instancia}

    \end{itemize}

\clearpage
  % Configuraciones
    
\subsubsection{IU-C01: Pantalla creación de nueva instancia de actividad de competencia uno contra uno}

 Descripción ...

    \IUfig{1}{modulos/comp/IU/p_creacion_competencia1vs1}{IU-C01}{%
        Pantalla de creación de nueva instancia de actividad de competencia uno contra uno}

\subsubsection{Elementos Relevantes}

    \begin{itemize}
    \item {\bf Lorem ipsum}
        ...
    \end{itemize}

\subsubsection{Acciones relevantes}

    \begin{itemize}
    \item {\bf Lorem ipsum}
        ...
    \end{itemize}

\clearpage
  % Configuraciones Generales
    
\subsubsection{IU-C02: Pantalla principal de actividad de competencia uno contra uno}

Esta pantalla es la principal de la competencia uno contra uno, en ella se puede ver el número de victorias que tiene el usuario actual y se le muestran los estudiantes a los que puede desafiar, también, si tiene desafíos pendientes se le muestran para que los pueda aceptar.

    \IUfig{1}{modulos/comp/IU/p_principal_competencia1vs1}{IU-C02}{%
        Pantalla principal de actividad de competencia uno contra uno}

\subsubsection{Elementos Relevantes}

    \begin{itemize}
    \item {\bf Número de victorias}
    \item {\bf Compañeros del curso}
    \item {\bf Desafíos pendientes}
    \item {\bf Tabla de posiciones}
    \item {\bf Historial}
    \item {\bf Desafiar}
    \item {\bf Aceptar desafío}
    \item {\bf Monedas a apostar}
    \end{itemize}

\subsubsection{Acciones relevantes}

    \begin{itemize}
    \item {\bf Desafiar a un estudiante}
    \item {\bf Aceptar un desafío de un usuario}
    \item {\bf Abrir la pestaña de tabla de posiciones}
    \item {\bf Abrir la pestaña de historial}
    \end{itemize}

\clearpage

    
\subsubsection{IU-C02a: Pantalla principal de actividad de competencia uno contra uno sin apuestas}

Esta pantalla es la principal de la competencia uno contra uno, en ella se puede ver el número de victorias que tiene el usuario actual y se le muestran los estudiantes a los que puede desafiar, también, si tiene desafíos pendientes se le muestran para que los pueda aceptar.

    \IUfig{1}{modulos/comp/IU/p_principal_competencia1vs1_sinapuesta}{IU-C02a}{%
        Pantalla principal de actividad de competencia uno contra uno sin apuestas}

\subsubsection{Elementos Relevantes}

    \begin{itemize}
    \item {\bf Número de victorias}
    \item {\bf Compañeros del curso}
    \item {\bf Desafíos pendientes}
    \item {\bf Tabla de posiciones}
    \item {\bf Historial}
    \item {\bf Desafiar}
    \item {\bf Aceptar desafío}
    \end{itemize}

\subsubsection{Acciones relevantes}

    \begin{itemize}
    \item {\bf Desafiar a un estudiante}
    \item {\bf Aceptar un desafío de un usuario}
    \item {\bf Abrir la pestaña de tabla de posiciones}
    \item {\bf Abrir la pestaña de historial}
    \end{itemize}

\clearpage
  % Configuraciones Visuales
    
\subsubsection{IU-C03: Pantalla de creación de nueva instancia de actividad de competencia uno contra sistema}

 Esta pantalla es el formulario que se le muestra al usuario cuando quiere crear una nueva instancia de la actividad uno contra sistema.

    \IUfig{1}{modulos/comp/IU/p_creacion_competencia1vscpu}{IU-C03}{%
        Pantalla de creación de nueva instancia de actividad de competencia uno contra sistema}

        \subsubsection{Elementos Relevantes}

            \begin{itemize}
            \item {\bf Nombre de actividad}
            \item {\bf Categoría de preguntas}
            \item {\bf Dificultad requerida}
            \item {\bf El estudiante debe de vencer al menos a la dificultad:}
            \item {\bf Guardar cambios y regresar al curso}
            \item {\bf Guardar cambios y mostrar}
            \item {\bf Cancelar}
            \end{itemize}

        \subsubsection{Acciones relevantes}

            \begin{itemize}
            \item {\bf Se puede crear una nueva instancia y mostrar su creación en la pantalla principal del curso}
            \item {\bf Se puede crear una nueva instancia y mostrar su pantalla principal}
            \item {\bf Se puede elegir como se finalizará la actividad}
            \end{itemize}

        \clearpage
  % Configuraciones Generales
    
\subsubsection{IU-C04: Pantalla principal de actividad de competencia uno contra sistema}

Esta pantalla es la principal de la competencia uno contra sistema, en ella se pueden ver las dificultades del sistema que han sido derrotadas, así como elegir el desafiar a una dificultad.

    \IUfig{1}{modulos/comp/IU/p_principal_competencia1vscpu}{IU-C04}{%
        Pantalla principal de actividad de competencia uno contra sistema}

        \subsubsection{Elementos Relevantes}

            \begin{itemize}
            \item {\bf Puntuación}
            \item {\bf Historial}
            \item {\bf Seleccione una dificultad}
            \item {\bf Empezar}
            \end{itemize}

        \subsubsection{Acciones relevantes}

            \begin{itemize}
            \item {\bf Desafiar a una dificultad}
            \item {\bf Abrir la pestaña de puntuaciones}
            \item {\bf Abrir la pestaña de historial}
            \end{itemize}

        \clearpage

    
\subsubsection{IU-C05: Pantalla Actualización de instancia de actividad de competencia uno contra sistema}

 Descripción ...

    \IUfig{1}{modulos/comp/IU/p_actualizacion_competencia1vscpu}{IU-C05}{%
        Pantalla de actualizacion instancia de actividad de competencia uno contra sistema}

\subsubsection{Elementos Relevantes}

    \begin{itemize}
    \item {\bf Lorem ipsum}
        ...
    \end{itemize}

\subsubsection{Acciones relevantes}

    \begin{itemize}
    \item {\bf Lorem ipsum}
        ...
    \end{itemize}

\clearpage

    
\subsubsection{IU-C06: Pantalla Actualización de instancia de actividad de competencia uno contra uno}

 Esta pantalla es el formulario que se le muestra al usuario cuando quiere actualizar una instancia de la actividad uno contra uno.

    \IUfig{1}{modulos/comp/IU/p_actualizacion_competencia1vs1}{IU-C06}{%
        Pantalla de actualización instancia de actividad de competencia uno contra uno}

        \subsubsection{Elementos Relevantes}

            \begin{itemize}
            \item {\bf Nombre de actividad}
            \item {\bf Categoría de preguntas}
            \item {\bf Apuestas}
            \item {\bf Competencias ganadas requeridas}
            \item {\bf Los estudiantes deben de haber ganado un mínimo de competencias de:}
            \item {\bf Guardar cambios y regresar al curso}
            \item {\bf Guardar cambios y mostrar}
            \item {\bf Cancelar}
            \end{itemize}

        \subsubsection{Acciones relevantes}

            \begin{itemize}
            \item {\bf Se puede actualizar una instancia y mostrar su creación en la pantalla principal del curso}
            \item {\bf Se puede actualizar una instancia y mostrar su pantalla principal}
            \item {\bf Se puede elegir como se finalizará la actividad}
            \end{itemize}

        \clearpage

    
\subsubsection{IU-C07: Pantalla de historial de actividad de competencia uno contra uno}

 En esta pantalla se le muestra un resumen del historial de las partidas del usuario actual. También se le muestra un desglose de cada partida, los resultados obtenidos en ella y su estado actual.

    \IUfig{1}{modulos/comp/IU/p_historial_competencia1vs1}{IU-C07}{%
        Pantalla de historial de actividad de competencia uno contra uno}

\subsubsection{Elementos Relevantes}

    \begin{itemize}
    \item {\bf Victorias}
    \item {\bf Empates}
    \item {\bf Derrotas}
    \item {\bf En curso}
    \item {\bf Retirada}

    \end{itemize}

\subsubsection{Acciones relevantes}

    \begin{itemize}
    \item {\bf Mostrar el historial del usuario actual}

    \end{itemize}

\clearpage

    
\subsubsection{IU-C08: Pantalla de tabla de posiciones de actividad de competencia uno contra uno}

 Esta pantalla muestra la tabla de posiciones de los usuarios que han realizado competencias, las posiciones son otorgadas de acuerdo el número de victorias del usuario.

    \IUfig{1}{modulos/comp/IU/p_tablap_competencia1vs1}{IU-C08}{%
        Pantalla de tabla de posiciones de actividad de competencia uno contra uno}

\subsubsection{Elementos Relevantes}

    \begin{itemize}
    \item {\bf Posición}
    \item {\bf Nombre}
    \item {\bf Número victorias}
    \end{itemize}

\subsubsection{Acciones relevantes}

    \begin{itemize}
    \item {\bf Ver tabla de posiciones}

    \end{itemize}

\clearpage

    
\subsubsection{IU-C09: Pantalla de cuestionario}

 Descripción ...

    \IUfig{1}{modulos/comp/IU/p_questionario}{IU-C09}{%
        Pantalla de cuestionario}

\subsubsection{Elementos Relevantes}

    \begin{itemize}
    \item {\bf Lorem ipsum}
        ...
    \end{itemize}

\subsubsection{Acciones relevantes}

    \begin{itemize}
    \item {\bf Lorem ipsum}
        ...
    \end{itemize}

\clearpage

    
\subsubsection{IU-C10: Pantalla partida en progreso}

 Descripción ...

    \IUfig{1}{modulos/comp/IU/p_esperar_respuesta_1v1}{IU-C10}{%
        Pantalla partida en progreso}

\subsubsection{Elementos Relevantes}

    \begin{itemize}
    \item {\bf Lorem ipsum}
        ...
    \end{itemize}

\subsubsection{Acciones relevantes}

    \begin{itemize}
    \item {\bf Lorem ipsum}
        ...
    \end{itemize}

\clearpage

    
\subsubsection{IU-C11: Pantalla partida ganada}

 Descripción ...

    \IUfig{1}{modulos/comp/IU/p_gano_1v1}{IU-C11}{%
        Pantalla partida ganada}

\subsubsection{Elementos Relevantes}

    \begin{itemize}
    \item {\bf Lorem ipsum}
        ...
    \end{itemize}

\subsubsection{Acciones relevantes}

    \begin{itemize}
    \item {\bf Lorem ipsum}
        ...
    \end{itemize}

\clearpage

    
\subsubsection{IU-C12: Pantalla partida perdida}

 Descripción ...

    \IUfig{1}{modulos/comp/IU/p_perdio_1v1}{IU-C12}{%
        Pantalla partida perdida}

\subsubsection{Elementos Relevantes}

    \begin{itemize}
    \item {\bf Lorem ipsum}
        ...
    \end{itemize}

\subsubsection{Acciones relevantes}

    \begin{itemize}
    \item {\bf Lorem ipsum}
        ...
    \end{itemize}

\clearpage

    
\subsubsection{IU-C13: Mensaje de alerta al abandonar el cuestionario}

Este mensaje de advertencia se muestra cuando el usuario quiere abandonar un cuestionario sin haberlo terminado.

    \IUfig{.3}{modulos/comp/IU/m_alerta_cuestionario}{IU-C13}{%
        Mensaje de alerta al abandonar el cuestionario}

\subsubsection{Elementos Relevantes}

    \begin{itemize}
    \item {\bf abandonar}
    \item {\bf Cancelar}
    \end{itemize}

\subsubsection{Acciones relevantes}

    \begin{itemize}
    \item {\bf Presionar abandonar} y salir del cuestionario
    \item {\bf Presionar cancelar} y mantenerse dentro del cuestionario
    \end{itemize}

\clearpage

    
\subsubsection{IU-C14: Pantalla de historial de actividad de competencia uno contra sistema}

 Descripción ...

    \IUfig{1}{modulos/comp/IU/p_historial_competenciacpu}{IU-C14}{%
        Pantalla de historial de actividad de competencia uno contra sistema}

\subsubsection{Elementos Relevantes}

    \begin{itemize}
    \item {\bf Lorem ipsum}
        ...
    \end{itemize}

\subsubsection{Acciones relevantes}

    \begin{itemize}
    \item {\bf Lorem ipsum}
        ...
    \end{itemize}

\clearpage

    
\subsubsection{IU-C15: Pantalla de tabla de puntuaciones de actividad de competencia uno contra sistema}

 Esta pantalla muestra las tablas de puntuaciones ''Primer intento'' y ''Mejor intento'' de los intentos de todos los usuarios que han utilizado la instancia de la competencia. También se filtra por dificultad del sistema para solo mostrar los intentos de una misma dificultad.

    \IUfig{1}{modulos/comp/IU/p_puntuacion}{IU-C15}{%
        Pantalla de tabla de puntuaciones de actividad de competencia uno contra sistema}

\subsubsection{Elementos Relevantes}

    \begin{itemize}
    \item {\bf Seleccione una dificultad}
    \item {\bf Primer intento}
    \begin{itemize}
      \item {\bf Posición}
      \item {\bf Nombre}
      \item {\bf Puntuación}
    \end{itemize}
    \item {\bf Mejor intento}
    \begin{itemize}
      \item {\bf Posición}
      \item {\bf Nombre}
      \item {\bf Puntuación}
    \end{itemize}
    \end{itemize}

\subsubsection{Acciones relevantes}

    \begin{itemize}
    \item {\bf Seleccionar la dificultad de la que se quiere saber los intentos realizados.}
    \item {\bf Mostrar todos los intentos de los usuarios que han utilizado la instancia}
    \end{itemize}

\clearpage



% =========================================================
    

\subsubsection{Diseño de complementos}



A continuación se presenta cómo los submódulos de competencia
se implmenetan en moodle.\\


\noindent Resumiendo el módulo de competencia tiene 2 actividades establecidas, llamadas; 
competencia uno contra uno y competencia uno contra sistema. 
Ambas actividades deben aparecer dentro de la lista de actividades de moodle. Para ello 
moodle cuenta con un tipo de complemento que se denomina \textbf{'mod'}, este tipo de complemento al ser instalado 
en una plataforma de moodle, crea una nueva opción a la lista de actividades.\\

\noindent Tomando en consideración lo anterior y que existe el complemento gamedlemaster, se presenta en la figura \ref{fig:diseno-comp-comp}
los complementos contemplados y las dependencias entre los mismos.


    \addfigure{1}{modulos/comp/diagrams/diseno_complementos}{fig:diseno-comp-comp}{Implementación del modulo de competencia}


Cada complemento en la figura \ref{fig:diseno-comp-comp} está representado con una cadena que sigue el formato 'tipo\_de\_complemento:nombre\_de\_complemento'. Los tipos de complemento son;
\begin{itemize}
    \item \textbf{mod} - Este complemento permite crear una actividad que aparece en la lista de actividades a agregar a un curso.
    \item \textbf{local} -  Este complemento moodle lo iterpreta como un comdín, el cual puede ser usado para múltiples propósitos relacionados con la gestión de la información.
    \item \textbf{block} - Este complemento permite desplegar un cuadro en la mayoría de las páginas de moodle, el cuál puede representar valores 
\end{itemize}

La función de cada uno de los complementos presentados en la figura \ref{fig:diseno-comp-comp} son:


\begin{itemize}
    \item \textbf{gmcompcpu} Definir la competencia uno contra sistema.
    \item \textbf{gmcompvs} Definir la competencia uno contra uno.
    \item \textbf{gmcs} Entregar las monedas por ganar cada una de las competencias anteriores.
\end{itemize}

El complemento de tipo  \textbf{'mod'} tiene un requerimiento en su nombre, el cual es; 'El nombre del complemento a instalar debe ser igual a un nombre
de una de las tablas en la base de datos'. Debido a que moodle no soporta nombres de complementos que contengan quiones bajos, el
nombre de la tabla ya no puede llevarlos.






%\subsubsection{Diagrama de componentes} % TODO CHANGE FOR INPUTS
%\subsubsection{Diagrama de clases} % TODO CHANGE FOR INPUTS
\clearpage
\subsubsection{Comportamiento del sistema de la competencia uno contra sistema}
    

A continuación se presenta de manera general el algoritmo que sigue el
sistema para responder un cuestionario.

\noindent Al diseñar el algoritmo se tuvo en cuenta que no respondiera únicamente 
de manera correcta o incorrecta determinado por una probabilidad.
Se diseño para que cada nivel de dificultad del sistema hiciera las mismas acciones, 
pero con más oportunidades o más probabilidades de contestar correctamente. 
Es por ello que se concideran 4 factores:

\begin{enumerate}
    \item Saber cuántas respuestas se deben elegir para responer correctamente la pregunta.
    \item El número de intentos para responder la pregunta correctamente.
    \item La probabilidad de que una respuesta sea elegida ante las otras respuestas.
    \item La posibilidad de reducir el número de respuestas que se tienen.
\end{enumerate}

El flujo general que no depende de la dificultad del sistema está representado en la figura \ref{fig:algoritmo-cpu-1},
en dicha figura se presentan cuadros azules cuya especificación se encuentran en las figuras:


\begin{enumerate}
    \item 'Calcular intentos [i]' está representado en la figura \ref{fig:algoritmo-cpu-2}.
    \item 'Calcular [c]' está representado en la figura \ref{fig:algoritmo-cpu-3} .
    \item 'Ponderar las respuestas' está representado en la figura \ref{fig:algoritmo-cpu-4} .
    \item 'Intentar descartar la respuesta [r] como una opción' está representado en la figura \ref{fig:algoritmo-cpu-5} .
\end{enumerate}

\clearpage
    \addfigure{0.8}{modulos/comp/diagrams/algoritmo_parte_1}{fig:algoritmo-cpu-1}{Diagrama de flujo del algoritmo, General}

\clearpage
    \addfigure{0.8}{modulos/comp/diagrams/algoritmo_parte_2}{fig:algoritmo-cpu-2}{Diagrama de flujo del algoritmo, 'Obtener inentos'}
    

\clearpage
    El algoritmo debe tomar en cuenta los casos en que las preguntas tienen más de una respuesta correcta 
    y los casos en que la respuesta correcta a una pregunta es una combinación de 2 o más respuestas.\\
    \addfigure{0.8}{modulos/comp/diagrams/algoritmo_parte_3}{fig:algoritmo-cpu-3}{Diagrama de flujo del algoritmo, 'Respuestas a elegir'}
\clearpage
    Debido a que el sistema elige una pregunta de manera al azar, 
    dependiendo de la dificultad se hace más probable que eliga una respuesta correcta o que se eliga una respuesta incorrecta.
    \addfigure{0.8}{modulos/comp/diagrams/algoritmo_parte_4}{fig:algoritmo-cpu-4}{Diagrama de flujo del algoritmo, 'Ponderación de respuestas'}
\clearpage
    Para aprovechar los intentos que tiene un sistema,
    se hace que cada nivel de dificultad pueda reducir las respuestas posibles una vez haya seleccionado una respuesta.
    \addfigure{0.8}{modulos/comp/diagrams/algoritmo_parte_5}{fig:algoritmo-cpu-5}{Diagrama de flujo del algoritmo, 'Reducir opciones'}
\clearpage


\noindent Para corroborar el funcionamiento del algoritmo y que cada dificultad tenga más probabilidad de obtener una mejor calificación se hicieron pruebas. 
Dichas pruebas consistían en eligir una dificultad y que el sistema contestara 10,000 veces el mismo cuestionario y con ello obtener 
el promedio de su calificación en ese questionario \ref{table:resultados-calificaciones-algoritmo-sistema}, 
el promedio de obtención de cada una de las calificaciones  \ref{table:resultados-calificacion-promedio-algoritmo-sistema} 
y la desviación estándar que se tiene \ref{table:resultados-desviacion-algoritmo-sistema} . 



\begin{table}[h!]
    \centering
    \begin{tabular}{|c|c|c|c|c|} \hline
        Dificultad del sistema & Fácil &     Normal &    Difícil &   Imposible \\\hline
        Caliifcación de 0 &  0.37\% &    0.02\% &   0.0\%  &  0.0\%   \\\hline
        Caliifcación de 1 &  0.91\% &    0.05\% &   0.0\%  &  0.0\%\\\hline
        Caliifcación de 2 &  3.73\% &    0.27\% &   0.0\%  &  0.0\%\\\hline
        Caliifcación de 3 &  5.3\% &    0.68\% &   0.0\%  &  0.0\%\\\hline
        Caliifcación de 4 &  13.5\% &    2.95\% &   0.1\%  &  0.0\%\\\hline
        Caliifcación de 5 &  13.92\% &    4.4\% &   0.04\%  &  0.01\%\\\hline
        Caliifcación de 6 &  22.5\% &    13.44\% &   1.3\%  &  0.23\%\\\hline
        Caliifcación de 7 &  13.57\% &    12.67\% &   1.41\%  &  0.21\%\\\hline
        Caliifcación de 8 &  16.72\% &    29.25\% &   16.55\%  &  6.44\%\\\hline
        Caliifcación de 9 &  5\% &    13.68\% &   8.62\%  &  3.66\%\\\hline
        Caliifcación de 10 &  4.48\% &    22.59\% &   71.98\%  &  89.45\%\\\hline
    \end{tabular}
    \caption{Tabla de resultados- Prueba algoritmo del sistema 'Porcentaje de obtención de calificación'}
    \label{table:resultados-calificaciones-algoritmo-sistema}
\end{table}


\begin{table}[h!]
    \centering
    \begin{tabular}{|c|c|c|c|c|} \hline
        Dificultad del sistema &                 Fácil &     Normal &    Difícil &   Imposible \\\hline
        Promedio de calificación &  6.0142 &    7.8878 &    9.4805 &    9.8186 \\ \hline
    \end{tabular}
    \caption{Tabla de resultados- Prueba algoritmo del sistema 'Calificación promedio'}
    \label{table:resultados-calificacion-promedio-algoritmo-sistema}
\end{table}


\begin{table}[h!]
    \centering
    \begin{tabular}{|c|c|c|c|c|} \hline
        Dificultad del sistema &                 Fácil &     Normal &    Difícil &   Imposible \\\hline
        Porcentaje de confianza &  2.0210884097436 &    3.2341593158042 &    4.7286951952952 &    9.8186 \\\hline
    \end{tabular}
    \caption{Tabla de resultados- Prueba algoritmo del sistema 'Calificación promedio'}
    \label{table:resultados-desviacion-algoritmo-sistema}
\end{table}




\subsection{Pruebas}


A continuación se enlistan los casos de prueba identificados y probados
correspondientes a cada uno de los casos de uso especificados. Los casos
de prueba listados a continuación son de dos tipos, los correctos e
incorrectos identificados por los prefijos CPC y CPI respectivamente.

\subsubsection{\refElem{CU-E01}}

    \begin{itemize}
    \TestCase{CPC-E01}{Instalar plugins del módulo de experiencia}
    \end{itemize}

\subsubsection{\refElem{CU-E02}}

    \begin{itemize}
    \TestCase{CPC-E02}{Realizar configuraciones del módulo de experiencia}
    \end{itemize}

\subsubsection{\refElem{CU-E02-1}}

    \begin{itemize}
    \TestCase{CPC-E02-1}{Realizar configuración de visualización de niveles}
    \TestCase{CPI-E02-1a}{Realizar configuraciones visuales con todos los datos
                          erróneos}
    \TestCase{CPI-E02-1b}{Configuraciones visuales con formato y nombre de imagen
                          inválidos}
    \end{itemize}

\subsubsection{\refElem{CU-E02-2}}

    \begin{itemize}
    \TestCase{CPC-E02-2a}{Realizar configuraciones del sistema de experiencia}
    \TestCase{CPC-E02-2b}{Realizar configuraciones con cursos iniciados}
    \TestCase{CPC-E02-2c}{Realizar configuraciones del sistema de experiencia con
                          alumnos con experiencia establecida}

    \TestCase{CPI-E02-2}{Realizar configuraciones del sistema de experiencia con
                         datos inválidos}
    \end{itemize}

\subsubsection{\refElem{CU-E02-3}}

    \begin{itemize}
    \TestCase{CPC-E02-3}{Realizar configuraciones del sistema de experiencia con datos
                         correctos}

    \TestCase{CPI-E02-3}{Realizar configuraciones del eventos con datos inválidos}
    \end{itemize}

\subsubsection{\refElem{CU-E03}}

    \begin{itemize}
    \TestCase{CPC-E03}{Desinstalar plugins del módulo de experiencia}
    \end{itemize}

\subsubsection{\refElem{CU-E04}}

    \begin{itemize}
    \TestCase{CPC-E04}{Crear un curso gamificado}
    \TestCase{CPI-E04}{Crear un curso gamificado con la experiencia deshabilitada}
    \end{itemize}

\subsubsection{\refElem{CU-E05}}

    \begin{itemize}
    \TestCase{CPC-E05}{Eliminar un curso gamificado sin alumnos inscritos}
    \TestCase{CPC-E05a}{Eliminar un curso gamificado con alumnos inscritos}
    \end{itemize}

\subsubsection{\refElem{CU-E06}}

    \begin{itemize}
    \TestCase{CPC-E06}{Eliminar el soporte para experiencia en un curso sin alumnos}
    \TestCase{CPC-E06a}{Eliminar un curso gamificado con alumnos inscritos}
    \end{itemize}

\subsubsection{\refElem{CU-E07}}

    \begin{itemize}
    \TestCase{CPC-E07}{Administrar experiencia en un curso recientemente creado}
    \TestCase{CPC-E07a}{Administrar experiencia en un curso con secciones completadas %
                       por alumnos}
    \TestCase{CPI-E07}{Administrar experiencia de un curso estableciendo valores
                       de experiencia incorrectos}
    \end{itemize}

\subsubsection{\refElem{CU-E08}}

    \begin{itemize}
    \TestCase{CPC-E08}{Crear una sección con soporte para experiencia en un curso
                       gamificado}
    \end{itemize}

\subsubsection{\refElem{CU-E12}}

    \begin{itemize}
    \TestCase{CPC-E12}{Crear un usuario gamificado (administrador)}
    \TestCase{CPC-E12a}{Crear un usuario gamificado mediante el auto-registro}
    \end{itemize}

\subsubsection{\refElem{CU-E13}}

    \begin{itemize}
    \TestCase{CPC-E13}{Eliminar usuario gamificado}
    \end{itemize}


    %
\TestCase{CPC-C01}{Crear nuevas instancias de la actividad de competencia 1 contra 1}

    %
\TestCase{CPC-C02}{Actualizar instancia de la actividad de competencia uno contra uno}

\begin{quote} %Contendrá la descripción del guión
	\textbf{ID:} C02.\\
    \textbf{Autor: } Ricardo Naranjo Polit\\
	\textbf{Alcance:}  \refElem{CU-C02}.\\
    \textbf{Preparación:}\\
    	%Contendrá todos los distintos datos que deberán estar preparados
      -Que se haya realizado el caso de prueba \refElem{CPC-C01}\\

\end{quote}

    \textbf{Entradas:}\\
    \begin{enumerate}
        \item \textbf{Nombre de actividad:} ''Competencia uno contra uno actualizada''
    \end{enumerate}
    \textbf{Pasos:}\\

    Trayectoria principal de \refElem{CU-C02}\\

    \textbf{Salida:}\\

     En la pantalla \refElem{IU-M08} la instancia de la competencia uno contra uno llamada ''Competencia uno contra uno actualizada''.

    %
\TestCase{CPC-C03}{Eliminar instancia de la actividad de competencia uno contra uno}

\begin{quote} %Contendrá la descripción del guión
	\textbf{ID:} C03.\\
    \textbf{Autor: } Ricardo Naranjo Polit\\
	\textbf{Alcance:}  \refElem{CU-C03}.\\
    \textbf{Preparación:}\\
    	%Contendrá todos los distintos datos que deberán estar preparados
      -Que se haya realizado el caso de prueba \refElem{CPC-C01}\\

\end{quote}

    \textbf{Entradas:}\\
    Ninguna\\
    \textbf{Pasos:}\\

    Trayectoria principal de \refElem{CU-C03}\\

    \textbf{Salida:}\\

     En la pantalla \refElem{IU-M08} la instancia de la competencia uno contra uno llamada ''Competencia uno contra uno'' no debe mostrarse.

    %
\TestCase{CPC-C04}{Crear nueva instancia de la actividad de competencia uno contra sistema}

    %
\TestCase{CPC-C05}{Actualizar instancia de la actividad de competencia uno contra sistema}
\begin{quote} %Contendrá la descripción del guión
	\textbf{ID:} C05.\\
    \textbf{Autor: } Ricardo Naranjo Polit\\
	\textbf{Alcance:}  \refElem{CU-C05}.\\
    \textbf{Preparación:}\\
    	%Contendrá todos los distintos datos que deberán estar preparados
      -Que se haya realizado el caso de prueba \refElem{CPC-C04}\\

\end{quote}

    \textbf{Entradas:}\\
    \begin{enumerate}
        \item \textbf{Nombre de actividad:} ''Competencia uno contra sistema actualizada''
    \end{enumerate}
    \textbf{Pasos:}\\

    Trayectoria principal de \refElem{CU-C05}\\

    \textbf{Salida:}\\

     En la pantalla \refElem{IU-M08} la instancia de la competencia uno contra sistema llamada ''Competencia uno contra sistema actualizada''.

    %
\TestCase{CPC-C06}{Eliminar instancia de la actividad de competencia uno contra sistema}

\begin{quote} %Contendrá la descripción del guión
	\textbf{ID:} C06.\\
    \textbf{Autor: } Ricardo Naranjo Polit\\
	\textbf{Alcance:}  \refElem{CU-C06}.\\
    \textbf{Preparación:}\\
    	%Contendrá todos los distintos datos que deberán estar preparados
      -Que se haya realizado el caso de prueba \refElem{CPC-C04}\\

\end{quote}

    \textbf{Entradas:}\\
    Ninguna\\
    \textbf{Pasos:}\\

    Trayectoria principal de \refElem{CU-C06}\\

    \textbf{Salida:}\\

     En la pantalla \refElem{IU-M08} la instancia de la competencia uno contra sistema llamada ''Competencia uno contra sistema'' no debe mostrarse.

    %
\TestCase{CPC-C07}{Mostrar el estado de la competencia uno contra uno}

\begin{quote} %Contendrá la descripción del guión
	\textbf{ID:} C07.\\
    \textbf{Autor: } Ricardo Naranjo Polit\\
	\textbf{Alcance:}  \refElem{CU-C07}.\\
    \textbf{Preparación:}\\
    	%Contendrá todos los distintos datos que deberán estar preparados
      -Que se haya realizado el caso de prueba \refElem{CPC-C01}\\
      -Que el usuario actual pueda ingresar al curso en el que se encuentra la instancia
\end{quote}

    \textbf{Entradas:}\\
    Ninguna\\
    \textbf{Pasos:}\\

    Trayectoria principal de \refElem{CU-C07}\\

    \textbf{Salida:}\\

    Se debe mostrar la pantalla \refElem{IU-C02}.

    %
\TestCase{CPC-C08}{Mostrar el historial de las partidas de competencia uno contra uno}

    %
\TestCase{CPC-C09}{Mostrar la tabla de posiciones de competencia uno contra uno}

    %
\TestCase{CPC-C10}{Desafiar a un estudiante apostando}

\begin{quote} %Contendrá la descripción del guión
	\textbf{ID:} C10.\\
    \textbf{Autor: } Ricardo Naranjo Polit\\
	\textbf{Alcance:}  \refElem{CU-C10}.\\
    \textbf{Preparación:}\\
    	%Contendrá todos los distintos datos que deberán estar preparados
      -Que se haya realizado el caso de prueba \refElem{CPC-C01}\\

      \begin{quote}
      -Que haya usuarios inscritos en el curso.\\
      \textbf{Usuario 1}:
        	\begin{itemize} %Contendrá todos los atributos de dicho ejemplo
                \item \textbf{Nombre\_de\_usuario:} ''usuario1''.
                \item \textbf{Nombre:} ''Usuario uno''
                \item \textbf{Apellidos:} ''Apellido usuario uno''
                \item \textbf{Dirección Email:} ''usuario1@test.com''

            \end{itemize}
      \textbf{Usuario 2}:
        	\begin{itemize} %Contendrá todos los atributos de dicho ejemplo
                \item \textbf{Nombre\_de\_usuario:} ''usuario2''.
                \item \textbf{Nombre:} ''Usuario dos''
                \item \textbf{Apellidos:} ''Apellido usuario dos''
                \item \textbf{Dirección Email:} ''usuario2@test.com''

            \end{itemize}
    \end{quote}


\end{quote}

    \textbf{Entradas:}\\
    \begin{enumerate}
        \item \textbf{Nombre de actividad:} ''Nueva competencia uno contra uno''
        \item \textbf{Categoría de preguntas:} ''nueva''
        \item \textbf{Apuestas:} ''Activada''
        \item \textbf{Seguimiento de finalización:} ''Mostrar la actividad como completada cuando se cumplan las condiciones''
        \item \textbf{Requerir ver:} Desactivado
        \item \textbf{Competencias ganadas requeridas:} Activado
        \item \textbf{Los estudiantes deben de haber ganado un mínimo de competencias de:} 1
    \end{enumerate}
    \textbf{Pasos:}\\

    Trayectoria principal de \refElem{CU-C10}\\

    \textbf{Salida:}\\

     En la pantalla \refElem{IU-M08} la nueva instancia de la competencia uno contra uno llamada ''Nueva competencia uno contra uno''.

    %
\TestCase{CPC-C11}{Desafiar a un estudiante sin apostar}

    %
\TestCase{CPC-C12}{Mostrar el estado de la competencia uno contra sistema}

    %
\TestCase{CPC-C13}{Desafiar al sistema}

    %
\TestCase{CPC-C14}{Mostrar la tabla de puntuaciones de competencia uno contra sistema}

    %
\TestCase{CPC-C15}{Ver historial de puntuaciones de competencia uno contra sistema}

    %
\TestCase{CPC-C16}{Aceptar un desafío en competencia uno contra uno apostando}

    %
\TestCase{CPC-C17}{Aceptar un desafío en competencia uno contra uno sin apostar}


\subsection{Funcionalidades de moodle}

En esta sección se abordan las funcionalidades propias de moodle que fueron utilizadas para el desarrollo
de los módulos de competencia, así como una descripción de su objetivo y problemas encontrados al utilizarlas.\\

  
\subsection{Entidades de moodle}

Debido a que moodle cuenta con más de 400 tablas en su versión 3.5, se opta
por mostrar 2 subconjuntos que muestren las tablas que se utilizan para el proyecto.\\

\noindent El primer subconjunto es aquel que explica la forma en que moodle implementa los cursos, 
secciones de curso, actividades, usuarios y roles (el cual se presenta en la figura \ref{fig:BD-ER-M1}), 
mientras que el segundo conjunto muestra como moodle maneja toda la 
estructura de las preguntas creadas por el profesor y respondidas por el usuario
(el cual se presenta en la figura \ref{fig:BD-ER-M2}).  



En lugar de describir y mostrar cada uno de los campos de cada una de las entidades de moodle que se contemplan,
lo que se quiere lograr con ambos esquemas (\ref{fig:BD-ER-M1}) y \ref{fig:BD-ER-M2}))
es expresar la idea general del comportamiento.

\clearpage
\addfigure{0.7}{analisis/diagrams/db_module_structure}{fig:BD-ER-M1}{Esquema de la base de datos de moodle 'Cursos'}


\noindent Utilizando la figura \ref{fig:BD-ER-M1}, se obtuvieron las siguientes reglas y caracteristicas que contiene moodle respecto a los usuarios en un curso y a la estructura de los cursos.
\begin{enumerate}
    \item Un usuario -{\it mdl\_user}- tiene un rol -{\it mdl\_role}- en un cierto contexto -{\it mdl\_context}-, cuyo  '{\it context\_level}' sea igual a cincuenta(50).
    \item Si el contexto '{\it context\_level}' es de 50, el atributo '{\it instance\_id}' hace referencia al atributo '{\it id}' de un curso -{\it mdl\_course}-.
    \item El curso -{\it mdl\_course}- tiene varias secciones -{\it mdl\_course\_sections}-.
    \item Cada seccion -{\it mdl\_course\_sections}- tiene varias actividades -{\it mdl\_course\_modules}- que pertenecen a un tipo de actividad -{\it mdl\_modules}-.
    \item Por cada registro en tipo de actividad -{\it mdl\_modules}-, se tiene una entidad que lleva el mismo nombre.
    \item El atributo '{\it instance\_id}' de una actividad  -{\it mdl\_course\_modules}- apunta a diferentes entidades. La entidad a la que apunta depende del nombre del tipo de actividad -{\it mdl\_modules}-.
    \item Un usuario -{\it mdl\_user}- se inscribió -{\it mdl\_user\_enrolments}- a un curso -{\it mdl\_course}-, por medio de un formato soportado de inscripción -{\it mdl\_enrol}-.
\end{enumerate}

\clearpage

 \addfigure{0.7}{analisis/diagrams/db_module_questions}{fig:BD-ER-M2}{Esquema de la base de datos de moodle 'Preguntas' }



\noindent Utilizando la figura \ref{fig:BD-ER-M2}, se obtuvieron las siguientes reglas y caracteristicas que contiene moodle respecto a las preguntas.
\begin{enumerate}
    \item Las preguntas -{\it mdl\_question}- tienen versiones -{\it mdl\_question\_attempts}-.
    \item Una pregunta -{\it mdl\_question}- pertenece a un banco de preguntas -{\it mdl\_question\_categories}-.
    \item La versión de una pregunta -{\it mdl\_question\_attempts}- es contestada -{\it mdl\_question\_usages}- en un determinado contexto -{\it mdl\_context}-.
    \item Un usuario -{\it mdl\_user}- responde una versión de una pregunta -{\it mdl\_question\_attempt\_stepts}-.
    \item El responder una versión de una pregunta -{\it mdl\_question\_attempt\_stepts}- conlleva pasos -{\it mdl\_question\_attempt\_stept\_data}-, los cuales son: cómo se muestra, si ya se terminó de responder y qué se respondió.
\end{enumerate}


 A continuación se presenta la especificación de las entidades del esquema de base
 de datos de moodle que son relevantes para el desarrollo de los módulos y submódulos
 de proyecto.

    \begin{cdtEntidad}{mdl-config-plugins}{Configuración de Plugin}{%
    Es una tabla del núcleo de moodle que almacena todas las configuraciones globales
    relacionadas a los plugins instalados, al iniciar moodle las configuraciones de los
    plugins instalados y habilitados se cargan en memoria.}

	    \brAttr{id}{Id}{tInt}{%
	        Es el dígito que representa el identificador único para una configuración
            específica de un plugin.\par

            \it Restricciones:
            \refElem{tPrimaryKey},
            \refElem{tAutoIncrement}.
        }

        \brAttr{plugin}{Plugin}{tVarchar}{%
            Cadena de caracteres del nombre identificador del plugin al cual pertenece
            la configuración.\par

            \it Restricciones:
            \refElem{tRequired},
            \refElem{tRange} (0,100),
            \refElem{tUniqueKey}
        }

        \brAttr{name}{Nombre}{tVarchar}{%
            Cadena de caracteres que representa el nombre de la configuración de un
            plugin en específico.\par

            \it Restricciones:
            \refElem{tUniqueKey},
            \refElem{tRange} (0,100),
            \refElem{tRequired}
        }

        \brAttr{value}{Valor}{tVarchar}{%
            Cadena que almacena el valor de una configuración perteneciente a alguno
            de los plugins instalados.\par

            \it Restricciones:
            \refElem{tRange} (0,4294967295),
            \refElem{tRequired}
        }
    \end{cdtEntidad}\schemeName{config\_plugins}

    \begin{cdtEntidad}{mdl-user}{Usuario de moodle}{%
    Es una tabla del núcleo de moodle que contiene toda la información que se
    almacena de los usuarios en la plataforma, independientemente del rol que
    estos contenga, esta relación contiene más de 53 atributos, sin embargo solo
    se detallan aquellos relevantes.}

	    \brAttr{id}{Id}{tInt}{%
	        Es el dígito que representa el identificador único para cada uno
            de los usuarios en moodle.\par

            \it Restricciones:
            \refElem{tPrimaryKey},
            \refElem{tAutoIncrement}.
        }
	    \brAttr{username}{nombre de usuario}{tVarchar}{%
	        .\par

            \it Restricciones:
            \refElem{tRequired},
            \refElem{tLength} 0-100
        }
	    \brAttr{password}{contraseña}{tVarchar}{%
	        .\par

            \it Restricciones:
            \refElem{tRequired},
            \refElem{tLength} 0-255.
        }
	    \brAttr{firstname}{nombre}{tVarchar}{%
	        .\par

            \it Restricciones:
            \refElem{tRequired},
            \refElem{tLength} 0-100
        }
	    \brAttr{lastname}{apellido}{tVarchar}{%
	        .\par

            \it Restricciones:
            \refElem{tRequired},
            \refElem{tLength} 0-100
        }
	    \brAttr{email}{correo}{tVarchar}{%
	        .\par

            \it Restricciones:
            \refElem{tRequired},
            \refElem{tLength} 0-100
        }
	    \brAttr{lastaccess}{último registro}{tInt}{%
	        .\par

            \it Restricciones:
            \refElem{tRequired},
            \refElem{tLength} 10
        }
	    \brAttr{city}{ciudad}{tVarchar}{%
	        .\par

            \it Restricciones:
            \refElem{tRequired},
            \refElem{tLength} 0-120
        }
	    \brAttr{country}{pais}{tVarchar}{%
	        .\par

            \it Restricciones:
            \refElem{tRequired},
            \refElem{tLength} 2
        }

    \end{cdtEntidad}\schemeName{user}

    \begin{cdtEntidad}{mdl-course}{Curso de moodle}{%
    Es una tabla del núcleo de moodle que contiene la información principal de cada 
    curso registrado en moodle. Esta entidad contiene 31 atributos, a continuación se
    detallan los atributos relevantes para la especificación de este proyecto.}

	    \brAttr{id}{Id}{tInt}{%
	        Es el dígito que representa al identificador único para cada uno
            de los cursos en moodle.\par

            \it Restricciones:
            \refElem{tPrimaryKey},
            \refElem{tAutoIncrement}.
        }

	    \brAttr{format}{formato}{tVarchar}{%
	        Es el dígito que representa al identificador único para cada uno
            de los cursos en moodle.\par

            \it Restricciones:
            \refElem{tRequired}.
            \refElem{tDefault} topics,
            \refElem{tLength} 0-21.
        }

	    \brAttr{fullname}{nombre completo}{tVarchar}{%
	        Es el nombre completo que se le asigna al curso.\par

            \it Restricciones:
            \refElem{tRequired}.
            \refElem{tLength} 0-21.
        }

	    \brAttr{shortname}{nombre corto}{tVarchar}{%
            Es el nombre corto que se le asigna al curso.\par

            \it Restricciones:
            \refElem{tRequired}.
            \refElem{tLength} 0-21.
        }

    \end{cdtEntidad}\schemeName{course}

    \begin{cdtEntidad}{mdl-course-sections}{Secciones del curso de moodle}{%
    }
	    \brAttr{id}{Id}{tInt}{%
	        Es el dígito que representa al identificador único para cada seccion
            de los cursos en moodle.\par

            \it Restricciones:
            \refElem{tPrimaryKey},
            \refElem{tAutoIncrement}.
        }
    \end{cdtEntidad}\schemeName{course\_sections}

    \begin{cdtEntidad}{mdl-course-format-options}{Opciones del formato del curso}{%
    }

	    \brAttr{id}{Id}{tInt}{%
	        Es el dígito que representa al identificador único para cada uno
            de los cursos en moodle.\par

            \it Restricciones:
            \refElem{tPrimaryKey},
            \refElem{tAutoIncrement}.
        }

	    \brAttr{courseid}{Id}{tInt}{%
	        Es el dígito que representa al identificador único para cada uno
            de los cursos en moodle.\par

            \it Restricciones:
            \refElem{tForeignKey},
            \refElem{tRequired}
        }

	    \brAttr{format}{formato}{tVarchar}{%
	        Es el dígito que representa al identificador único para cada uno
            de los cursos en moodle.\par

            \it Restricciones:
            \refElem{tRequired}.
            \refElem{tDefault} topics,
            \refElem{tLength} 0-21.
        }

	    \brAttr{name}{opcion}{tVarchar}{%
	        Es el dígito que representa al identificador único para cada uno
            de los cursos en moodle.\par

            \it Restricciones:
            \refElem{tPrimaryKey},
            \refElem{tLength} 0-100
        }

	    \brAttr{value}{valor}{tVarchar}{%
	        Es el dígito que representa al identificador único para cada uno
            de los cursos en moodle.\par

            \it Restricciones:
            \refElem{tRequired}
        }

    \end{cdtEntidad}\schemeName{course\_format\_options}

    \begin{cdtEntidad}{mdl-course-category}{Categoria de curso}{%
    .}
    \end{cdtEntidad}\schemeName{course\_category}

    \begin{cdtEntidad}{Plugin}{Plugin}{%
    La forma en que moodle obtiene información acerca de los plugins es analizando
    los archivos internos de cada uno, a pesar de que los plugins no forman parte
    del esquema de base de datos, si forman parte del modelo de información que
    utiliza Moodle.}

	    \brAttr{componente}{Componente}{tVarchar}{%
	        Cadena compuesta por el tipo de plugin y el nombre del mismo, que
            representa a la clase principal del plugin que contiene los métodos
            principales del plugin.\par

            \it Restricciones: Ninguna
        }

	    \brAttr{pluginname}{Nombre}{tVarchar}{%
	        Es el nombre del plugin obtenido de los archivos de
            internacionalización presentes en el plugin, el valor de esta cadena
            depende del lenguaje seleccionado en moodle.\par

            \it Restricciones: Ninguna
        }

	    \brAttr{fullpath}{Ruta absoluta}{tPath}{%
	        La ruta absoluta de un plugin denota la ubicación del plugin en el
            sistema de archivos, esta ruta está compuesta por la ruta absoluta
            de la instalación de moodle, la carpeta correspondiente al tipo del
            plugin y el nombre del plugin.\par

            \it Restricciones: Formato ``/path/to/moodle/plugintype/pluginname''
        }

	    \brAttr{path}{Ruta relativa}{tPath}{%
	        La ruta relativa denota la ubicación del plugin dentro de la carpeta 
            donde se encuentran los archivos de moodle, esta ruta está compuesta
            por la carpeta correspondiente al tipo del plugin y el nombre del
            plugin.\par

            \it Restricciones: Formato ``plugintype/pluginname''
        }

	    \brAttr{version}{Versión}{tVersion}{%
	        Numero entero de longitud de 10 dígitos que representa la versión del 
            plugin.\par

            \it Restricciones: Ninguna adicional al tipo de dato
        }

	    \brAttr{moodle}{Versión de Moodle}{tVersion}{%
	        Número entero de longitud de 10 dígitos que representa la versión de 
            moodle en la que se puede instalar el plugin.\par

            \it Restricciones: Ninguna adicional al tipo de dato
        }

        \brAttr{dependencies}{Dependencias}{tObject}{%
            Objeto que almacena un conjunto de claves con sus respectivos valores, 
            donde cada clave representa el nombre del componente del plugin y el valor 
            es la \refElem{Plugin.version} requerida del mismo.

            \it Restricciones: Ninguna
        }

        \brAttr{icon}{ícono}{tImage}{%
            Imagen para el ícono del plugin, debe estar contenida en el directorio
            {\it pix/} del plugin y tener como nombre {\it icon.png} o {\it icon.svg},
            moodle recomienda tener ambos archivos por si los navegadores no soportan
            algun tipo de archivo \cite{moodlePluginfiles}.\par 

            \it Restricciones: El nombre debe ser icon con extensiones png o svg
        }

    \end{cdtEntidad}

