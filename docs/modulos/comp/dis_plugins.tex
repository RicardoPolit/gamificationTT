

\subsubsection{Diseño de complementos}



A continuación se presenta cómo los submódulos de competencia
se implementan en moodle.\\


\noindent Resumiendo el módulo de competencia tiene 2 actividades establecidas, llamadas;
competencia uno contra uno y competencia uno contra sistema.
Ambas actividades deben aparecer dentro de la lista de actividades de moodle. Para ello
moodle cuenta con un tipo de complemento que se denomina \textbf{'mod'}, este tipo de complemento al ser instalado
en una plataforma de moodle, crea una nueva opción a la lista de actividades.\\

\noindent Tomando en consideración lo anterior y que existe el complemento gamedlemaster, se presenta en la figura \ref{fig:diseno-comp-comp}
los complementos contemplados y las dependencias entre los mismos.


    \addfigure{1}{modulos/comp/diagrams/diseno_complementos}{fig:diseno-comp-comp}{Implementación del modulo de competencia}


Cada complemento en la figura \ref{fig:diseno-comp-comp} está representado con una cadena que sigue el formato 'tipo\_de\_complemento:nombre\_de\_complemento'. Los tipos de complemento son;
\begin{itemize}
    \item \textbf{mod} - Este complemento permite crear una actividad que aparece en la lista de actividades a agregar a un curso.
    \item \textbf{local} -  Este complemento puede ser usado para múltiples propósitos relacionados con la gestión de la información.
    \item \textbf{block} - Este complemento permite desplegar una sección en la mayoría de las páginas de moodle, la cuál puede representar código html.
\end{itemize}

La función de cada uno de los complementos presentados en la figura \ref{fig:diseno-comp-comp} son:


\begin{itemize}
    \item \textbf{gamedlemaster} Definir la base de datos y los eventos a manejar.
    \item \textbf{gmcompcpu} Definir la competencia uno contra sistema.
    \item \textbf{gmcompvs} Definir la competencia uno contra uno.
    \item \textbf{gmcs} Entregar las monedas por ganar cada una de las competencias anteriores.
\end{itemize}

El complemento de tipo  \textbf{'mod'} tiene un requerimiento en su nombre, el cual es; 'El nombre del complemento a instalar debe ser igual a un nombre
de una de las tablas en la base de datos'. Debido a esto y que moodle no soporta nombres de complementos que contengan guiones bajos, el
nombre de la tabla ya no puede llevarlos.\\
