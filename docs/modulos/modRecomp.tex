

\chapter{Módulo de Recompensa}
\label{mod:recompensa}

    Este módulo contiene a las herramientas de logros, que premian acciones favorables para el usuario; la herramienta de antilogros, diseñada para notificar al alumno cuando descuide su desempeño en la plataforma; y la herramienta de marcadores, que muestra la lista de los alumnos con un mejor aprovechamiento.

\section{Submódulo de Logros}

    Cada vez que un alumno cumpla con las condiciones de un logro, este se desbloqueará para dicho alumno. Cuando se desbloquea un logro se le otorga al alumno monedas de plata, experiencia y si el alumno va a la sección de logros -ya sea del club o de la plataforma- verá dicho logro como completado.
    
    \begin{quote}
    \begin{description}
    \item[Objetivo] \hfill\\
        Proveer metas al alumno que el decida si quiere alcanzar o no.
    
    \item[Principios a los que dá soporte:] \hfill
        \begin{itemize}
            \item 2 \principioII
            \item 6 \principioVI
        \end{itemize}
    \end{description}
    \end{quote}
    
\section{Submódulo de Advertencias}

    La advertencia tiene el mismo comportamiento que un logro, sin embargo, esta representa una aspecto negativo. La herramienta le recordará al alumno que tenga cuidado de no cumplir con las condiciones de la advertencia. Si el alumno llega a cumplir con las condiciones, tendrá desbloqueada la advertencia en la sección de logros.
    
    \begin{quote}
    \begin{description}    
    \item[Objetivo] \hfill\\
        Que el alumno evite cumplir con las condiciones de la advertencia, en otras palabras, que el alumno no tenga la advertencia desbloqueada.
    
    \item[Principios a los que da soporte:] \hfill
        \begin{itemize}
            \item 8 \principioVIII
        \end{itemize}
    \end{description}
    \end{quote}
    
\section{Submódulo de Marcadores}

    Tener varias tablas de líderes de los ''n'' mejores alumnos de toda la plataforma según ciertos criterios como la experiencia, los cursos tomados, las competencias ganadas, etc. Los marcadores también contaran con la posición actual del alumno en ese listado junto a los 2 lugares superiores así como los 2 lugares inferiores, haciendo un total de 15 nombres a mostrar en el listado. 
    
    \begin{quote}
    \begin{description}
    \item[Objetivo] \hfill\\
        Permitirle al usuario el compararse con los mejores en la plataforma así como a las personas que están cerca de él.
    
    \item[Principios a los que da soporte:] \hfill
        \begin{itemize}
            \item 2 \principioII
            \item 5 \principioV
        \end{itemize}
    \end{description}
    \end{quote}
 
\clearpage