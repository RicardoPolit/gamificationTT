
\subsection{IU-M02 Pantalla principal}

 La página de portada, o página principal mostrada en la figura \ref{IU-M02}, es la
 página inicial que ve alguien que llega a un sitio Moodle antes o después de entrar al sitio.
 Típicamente un estudiante verá los cursos, algunos bloques de información, mostrados en un tema.
 En la Barra de navegación y en el menú de navegación (esquina superior izquierda).\\

 \noindent 
 La combinación de las políticas del sitio, autenticación del usuario y configuraciones de la
 portada determinan quién puede llegar a la portada, los elementos que pueden ver y acciones
 que pueden realizar \cite{MoodlePortada}.
    % https://docs.moodle.org/all/es/Portada

    \IUfig{1}{modulos/IUMoodle/Dashboard.png}{IU-M02}{Pantalla Principal de Moodle}

\subsubsection{Elementos relevantes}

    \begin{itemize}
    \item
    {\bf Menú Superior}
        Como su nombre lo indica se encuentra en la parte superior, este elemento se
        encuentra en la mayoría de las pantallas de moodle.

    \item
    {\bf Menú de Navegación}
        Cuando esta visible se encuentra en la parte izquierda de la parte izquierda
        de la mayoría de las pantallas de moodle. Se puede ocultar o mostrar con la
        acción \IUMenu[].

    \item
    {\bf Contenido}
        Tiene todos los demás elementos que conforman el contenido de la pantalla.

    \end{itemize}

\subsubsection{Acciones relevantes}

    \begin{itemize}
    
    \item
    {\bf \IUMenu (Desplegar el menú)}
        Si el menú está oculto, cuando el usuario presione el botón \IUMenu el menú de
        navegación se desplegará.

    \item {\bf \IUMenu (Ocultar Menu)}
        Si el menú está visible, cuando el usuario presione el botón \IUMenu el menú de
        navegación se ocultará.

    \item {\bf \IUAdminSitio Administración del sitio }
        Cuando el menú está visible, el botón de administración del sitio nos permitirá
        navegar a la pantalla \refElem{IU-M03}

    \end{itemize}
