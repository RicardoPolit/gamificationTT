
\subsection{IU-M02: Instalador de plugin}

 La página del instalador de plugins permite al \refElem{aAdministrador} instalar nuevos plugins al 
 moodle que administra de una forma sencilla y sin tener que manipular los archivos en el servidor
 donde se tenga moodle instalado, para ello cada \refElem[plugin]{Plugin} a instalar debe estar
 compresos en un archivo {\it ZIP} cumpliendo con la regla \refElem{BR-M1}.

 % TODO: BR-M1: Restricciones el archivo de instalación.
 % El archivo de instalación debe ser un archivo ZIP, el cual debe contener exactamente un
 % directorio que coincida con el nombre del plugin.

    \IUfig{1}{modulos/moodleIU/InstallPlugin.png}{IU-M02}{Instalador de plugin}

\subsubsection{Elementos relevantes}

    \begin{itemize}
    \item {\bf Selector de archivos.}
        Permite elegir un archivo y prepararlo para subirlo a moodle
        y realizar las acciones correspondientes.
    \end{itemize}

\subsubsection{Acciones relevantes}

    \begin{itemize}
    \item {\bf Selección de un archivo}
        Permite seleccionar un \refElem{Plugin} compreso en un archivo {\it ZIP} para
        ser instalado en moodle.

    \item {\bf Instalar plugin desde un archivo ZIP}
        Confirma el envió del formulario que contiene principalmente al archivo compreso con 
        el plugin que será instalado. Redirige a la pantalla \refElem{IU-M02a}.
    \end{itemize}

\clearpage
