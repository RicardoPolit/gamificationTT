
\subsection{Análisis}

 Este apartado contiene el análisis requerido para la elaboración de módulo de seguimiento,
 contiene la especificación del alcance de este módulo, la descripción de las funcionalidades
 a desarrollar, la reglas de negocio que rigen el comportamiento del módulo, y por último la
 especificación de los casos de uso a los que brinda soporte.

\subsubsection{Reglas de negocio} %========================================================

 En esta sección se especifican todas las reglas de negocio relevantes para el módulo de
 experiencia. Las reglas de negocio que establece moodle son diferenciadas por tener la letra {\it M}
 antecediendo al número consecutivo en su identificador.

\clearpage
\subsubsection{Casos de uso} % ============================================================

 En este apartado se especifican todos los casos de usos contemplados para el módulo de
seguimiento, para cada caso de uso se especifica su tabla de atributos la cual indica que casos
 de prueba deberán ejecutarse correctamente para corroborar la completitud del caso de uso.

\subsubsection*{Diagrama de casos de uso}

 En la figura \ref{seguimiento:usecases} se detalla el diagrama de casos de uso correspondiente al módulo
 de seguimiento. Los casos de uso de moodle (en color blanco) son modelados como casos de uso
 abstractos, mientras que los casos de uso del módulo de seguimiento son diferenciados por el
 color azul, en total el desarrollo de este módulo consiste en 17 casos de uso principales.

    \addfigure{0.6}{modulos/seguimiento/diagrams/UseCases}{seguimiento:usecases}{%
        Diagrama de casos de uso del módulo de seguimiento}

 \noindent
 Debido a que los plugins a desarrollar son elementos opcionales para Moodle, solo se puede
 acceder a los casos de uso del módulo de seguimiento a través de puntos de extensión de los
 casos de uso de moodle. Por otra parte los casos de uso que serán documentados en esta sección
 serán los del módulo de seguimiento debido a que Moodle proporciona en su página oficial, guías
 e instructivos como documentación de las funcionalidades que brinda.

    % MODULO DE EXPERIENCIA


% \ucstEnEdicion     Al terminar una revisión/aprobación con observaciones
%                    y al inicio del CU.
%
% \ucstEnRevision    Al terminar la edición del CU (version += 0.1).
% \ucstEnAprobacion  Al pasar la revision sin observaciones.
% \ucstAprobado      Al ser aprobado por el usuario (version += 1.0)

%\addfigure[(adaptado de {\it For The Win} \cite{ForTheWin})]%
%    {.4}{investigacion/images/forthewin}{fig:ForTheWin}%
%    {Jerarquía de elementos de juego segun For The Win}

\begin{UseCase}[%
Autor/David Flores Casanova,%
Version/0.1,%
Estado/\ucstEnRevision]%
%
{CU-S01}{Ver estado de actividad}{%
%
 El actor puede entrar a la actividad y ver el progreso que lleva.
 Este caso de uso es una extensión del caso de uso {\it Ver curso} que es propio de moodle.}

	\UCitem[control]{Revisor}{ Sin asignar }
	\UCitem[control]{Último cambio}{ 13/NOV/19 }

 \UCsection{Atributos}

    \UCitem{Actor(es)}{%
        \refElem{aEstudiante},
        \refElem{aProfesor},
        \refElem{aAdministrador}
    }

	\UCitems{Propósito}{%
        \Titem El actor desea ver el progreso que lleva en la actividad, así como ver, si ya respondió la pregunta diaria.
	}

	\UCitem{Entradas}{\imprimeUC{entrada}}

	\UCitems{Origen}{%
        \Titem Mouse
	}

	\UCitem{Salidas}{\imprimeUC{salida}}

	\UCitem{Destino}{%
		\refElem{IU-S01a}
	}

	\UCitems{Precondiciones}{%
        \Titem El plugin de preguntas diarias debe estar instalado en moodle.
        \Titem La instancia de la actividad de preguntas diarias debe estar creada.
	}

	\UCitems{Postcondiciones}{%
        \Titem Se muestra el estado de la pregunta diaria para el actor.

	}

	\UCitem{Reglas de negocio}{\imprimeUC{regla}}

	\UCitems{Errores}{%
	}

 \UCsection[design]{Datos de Diseño}

	\UCitems[design]{Casos de Prueba}{%
        \Titem \refElem{CPC-S01-1}
        \Titem \refElem{CPC-S01-2}
        \Titem \refElem{CPC-S01-3}
	}

 \UCsection[admin]{Datos de Administración de Requerimiento}

	\UCitem[admin]{Observaciones}{}

\end{UseCase}

\subsubsection{Trayectorias del caso de uso}

\begin{UCtrayectoria}%
%
 	\Actor Presiona el nombre de la instancia de la actividad a la que quiere acceder en la pantalla \refElem{IU-M07}.
    \Sistema Verifica que el actor esté dado de alto como actor gamificado. \refTray{A}
    \label{CU-S01-cracion-usuario}
	\Sistema Verifica que el actor no haya respondido la pregunta diaria. \refTray{B}
    \Sistema Redirige a la pantalla principal de la instancia \refElem{IU-S01a}.
	\Sistema Obtiene el número de preguntas que el usuario ha respondido.
    \label{CU-S01-mostrar-informacion}
	y la cantidad de preguntas que le faltan por responder y genera la sección de la barra de progreso y la muestra en pantalla.

\end{UCtrayectoria}

\begin{UCtrayectoriaA}{A}{El actor no está dado de alta como usuario gamificado (\refElem{xp-user})}

  \Sistema Registra al actor en la entidad (\refElem{xp-user}).
    \item Se regresa al paso \ref{CU-S01-cracion-usuario} de la trayectoria principal.


\end{UCtrayectoriaA}


\begin{UCtrayectoriaA}{B}{El actor ya respondió la pregunta diaria (\refElem{xp-user})}

    \Sistema Redirige a la pantalla principal de la instancia \refElem{IU-S01b}.
    \item Se regresa al paso \ref{CU-S01-mostrar-informacion} de la trayectoria principal.


\end{UCtrayectoriaA}



\UCExtensionPoint{Ver tabla de posiciones}{%

    El actor desea ver la tabla de posiciones de una instancia de preguntas diarias.
%
    }{Al final de la trayectoria principal del caso de uso.}{\refElem{CU-S02}}

\UCExtensionPoint{Responder pregunta diaria}{%

    El actor desea responder la pregunta del día.
%
    }{Al final de la trayectoria principal del caso de uso.}{\refElem{CU-S03}}


% \ucstEnEdicion     Al terminar una revisión/aprobación con observaciones
%                    y al inicio del CU.
%
% \ucstEnRevision    Al terminar la edición del CU (version += 0.1).
% \ucstEnAprobacion  Al pasar la revision sin observaciones.
% \ucstAprobado      Al ser aprobado por el usuario (version += 1.0)

%\addfigure[(adaptado de {\it For The Win} \cite{ForTheWin})]%
%    {.4}{investigacion/images/forthewin}{fig:ForTheWin}%
%    {Jerarquía de elementos de juego segun For The Win}

\begin{UseCase}[%
Autor/David Flores Casanova,%
Version/0.1,%
Estado/\ucstEnRevision]%
%
{CU-S02}{Ver tabla de posiciones (Preguntas diarias)}{%
%
 Permite al actor ver la tabla de posiciones de una instancia de la actividad preguntas diarias.
 Este caso de uso es una extensión del caso de uso \refElem{CU-S01}.}

	\UCitem[control]{Revisor}{ Sin asignar }
	\UCitem[control]{Último cambio}{ 13/NOV/19 }

 \UCsection{Atributos}

    \UCitem{Actor(es)}{%
        \refElem{aEstudiante},
        \refElem{aProfesor},
        \refElem{aAdministrador}
    }

	\UCitems{Propósito}{%
        \Titem Ver la posición de cada usuario que participa en la instancia de la actividad.
	}

	\UCitem{Entradas}{\imprimeUC{entrada}}

	\UCitems{Origen}{%
        \Titem Mouse
	}

	\UCitem{Salidas}{\imprimeUC{salida}
	}

	\UCitem{Destino}{%
		\refElem{IU-S02}
	}

	\UCitems{Precondiciones}{%
        \Titem El complemento de preguntas diarias esté instalado en moodle.
        \Titem La instancia de la actividad de preguntas diarias debe estar creada.
        % Realizar el caso de uso "listar actividades disponibles?"
        % \Titem Si se trata de una actualización de un plugin la versión de este debe
               % cumplir con la regla \refElem{BR-M02}.
	}

	\UCitems{Postcondiciones}{%
        \Titem Se muestra la pantalla de tabla de posiciones perteneciente a la instancia de la actividad preguntas diarias.%

	}

	\UCitem{Reglas de negocio}{\imprimeUC{regla}}

	\UCitems{Errores}{%
	}

 \UCsection[design]{Datos de Diseño}

	\UCitems[design]{Casos de Prueba}{%
	}

 \UCsection[admin]{Datos de Administración de Requerimiento}

	\UCitem[admin]{Observaciones}{}

\end{UseCase}

\subsubsection{Trayectorias del caso de uso}

\begin{UCtrayectoria}%
%
    \Actor Presiona el botón {\bf Tabla posiciones} de la pantalla \refElem{IU-S01a} o \refElem{IU-S01b}.
    \Sistema Redirige a la pantalla \refElem{IU-S02}.
    \Sistema Carga el estado de cada usuario que ha participado en la actividad y los ordene de acuerdo a cuántas preguntas diarias han contestado de mayor a menor.

\end{UCtrayectoria}
   

% \ucstEnEdicion     Al terminar una revisión/aprobación con observaciones
%                    y al inicio del CU.
%
% \ucstEnRevision    Al terminar la edición del CU (version += 0.1).
% \ucstEnAprobacion  Al pasar la revision sin observaciones.
% \ucstAprobado      Al ser aprobado por el usuario (version += 1.0)

%\addfigure[(adaptado de {\it For The Win} \cite{ForTheWin})]%
%    {.4}{investigacion/images/forthewin}{fig:ForTheWin}%
%    {Jerarquía de elementos de juego segun For The Win}

\begin{UseCase}[%
Autor/David Flores Casanova,%
Version/0.1,%
Estado/\ucstEnRevision]%
%
{CU-S03}{Responder pregunta diaria}{%
%
 El actor responde una pregunta diaria escogida al azara del banco de repguntas si aún tiene disponibles.
 Este caso de uso es una extensión del caso de uso \refElem{CU-S01}.}

	\UCitem[control]{Revisor}{ Sin asignar }
	\UCitem[control]{Último cambio}{ 13/NOV/19 }

 \UCsection{Atributos}

    \UCitem{Actor(es)}{%
        \refElem{aEstudiante},
        \refElem{aProfesor},
        \refElem{aAdministrador}
    }

	\UCitems{Propósito}{%
        \Titem Avanzar en la actividad de preguntas diarias.
	}

	\UCitem{Entradas}{\imprimeUC{entrada}}

	\UCitems{Origen}{%
        \Titem Mouse
	}

	\UCitem{Salidas}{\imprimeUC{salida}
        Pregunta a responder generada por moodle}

	\UCitem{Destino}{%
		\refElem{IU-S02}
	}

	\UCitems{Precondiciones}{%
        \Titem El complemento de preguntas diarias esté instalado en moodle.
        \Titem La instancia de la actividad de preguntas diarias debe estar creada.
        \Titem Que el actor no haya respondido la pregunta diaria del día.
        % Realizar el caso de uso "listar actividades disponibles?"
        % \Titem Si se trata de una actualización de un plugin la versión de este debe
               % cumplir con la regla \refElem{BR-M02}.
	}

	\UCitems{Postcondiciones}{%
        \Titem Se agrega la pregunta a las preguntas ya respondidas por el actor.%

	}

	\UCitem{Reglas de negocio}{\imprimeUC{regla}}

	\UCitems{Errores}{%
	}

 \UCsection[design]{Datos de Diseño}

	\UCitems[design]{Casos de Prueba}{%
	}

 \UCsection[admin]{Datos de Administración de Requerimiento}

	\UCitem[admin]{Observaciones}{}

\end{UseCase}

\subsubsection{Trayectorias del caso de uso}

\begin{UCtrayectoria}%
%
    \Actor Presiona el botón {\bf Contestar} de la pantalla \refElem{IU-S01a}.
    \Sistema Obtiene una pregunta de manera aleatoria de las preguntas que aún no ha respondido el actor. \refTray{A}
    \Sistema Redirige a la pantalla \refElem{IU-S03}.
    \Sistema Guarda que se intenta responder una pregunta diaria.
    \Actor Responde la pregunta.
    \Actor Presiona el botón {\bf Terminar}.
    \Sistema Califica el intento. \refTray{B}
    \Sistema Redirige a la pantalla \refElem{IU-S04}.
    \Sistema Otorga experiencia al actor por haber contestado correctamente la pregunta diaria.
    \Sistema Otorga monedas al actor por haber contestado correctamente la pregunta diaria.
    

\end{UCtrayectoria}



\begin{UCtrayectoriaA}[Fin del caso de uso]{A}{El actor ya ha contestado todas las preguntas disponibles}

  \Sistema Muestra el mensaje \refElem{MSG-S01}.

\end{UCtrayectoriaA}


\begin{UCtrayectoriaA}[Fin del caso de uso]{B}{El actor no contestó correctamente la pregunta}

    \Sistema Redirige a la pantalla \refElem{IU-S04}.

\end{UCtrayectoriaA}   

% \ucstEnEdicion     Al terminar una revisión/aprobación con observaciones
%                    y al inicio del CU.
%
% \ucstEnRevision    Al terminar la edición del CU (version += 0.1).
% \ucstEnAprobacion  Al pasar la revision sin observaciones.
% \ucstAprobado      Al ser aprobado por el usuario (version += 1.0)

\begin{UseCase}[%
Autor/David Flores Casanova,%
Version/0.1,%
Estado/\ucstEnRevision]%
%
{CU-S04}{Eliminar instancia (Preguntas diarias)}{%
%
 Permite al actor eliminar una instancia de la actividad de preguntas diarias en el curso.
 Este caso de uso es una extensión del caso de uso {\it Ver curso} que es propio de moodle.}

	\UCitem[control]{Revisor}{ Sin asignar }
	\UCitem[control]{Último cambio}{ 13/NOV/19 }

 \UCsection{Atributos}

    \UCitem{Actor(es)}{%
        \refElem{aProfesor},
        \refElem{aAdministrador}
    }

	\UCitems{Propósito}{%
        \Titem Permitir quitar la actividad de su curso.
	}

	\UCitem{Entradas}{\imprimeUC{entrada}}

	\UCitems{Origen}{%
        \Titem Mouse
	}

	\UCitem{Salidas}{\imprimeUC{salida}}

	\UCitem{Destino}{%
		\refElem{IU-M08}
	}

	\UCitems{Precondiciones}{%
        \Titem El complemento de preguntas diarias debe estar instalado en moodle.
        \Titem La instancia de la actividad de preguntads diarias debe estar creada.
	}

	\UCitems{Postcondiciones}{%
        \Titem La instancia de la actividad eliminada no debe mostrarse en la pantalla \refElem{IU-M08}.%

	}

	\UCitem{Reglas de negocio}{\imprimeUC{regla}}

	\UCitems{Errores}{%
	}

 \UCsection[design]{Datos de Diseño}

	\UCitems[design]{Casos de Prueba}{%
	}

 \UCsection[admin]{Datos de Administración de Requerimiento}

	\UCitem[admin]{Observaciones}{}

\end{UseCase}

\subsubsection{Trayectorias del caso de uso}

\begin{UCtrayectoria}%
%

    \Actor Activa la edición del curso en la pantalla \refElem{IU-M08}.
    \Sistema Redirige a la pantalla de edición del curso \refElem{IU-M08aa}.
    \Actor Presiona el botón {\bf Editar} de la instancia que desea eliminar.
    \Sistema Despliega el menú \refElem{IU-M08b}.
    \Actor Presiona el botón {\bf Eliminar} del menú desplegable \refElem{IU-M08b}.
    \Sistema Despliega mensaje de confirmación de eliminación. \refElem{IU-M08c}
    \Actor Presiona el botón {\bf Sí}. \refTray{A}
    \Sistema  Elimina la instancia y los valores de la instancia \refElem{seg-gmpregdiarias}, y los valores que dependen de ella \refElem{seg-gmdl-intento-diario}.
    \Sistema Redirige a la pantalla \refElem{IU-M08}

\end{UCtrayectoria}

\begin{UCtrayectoriaA}[Fin del caso de uso]{A}{El \refElem{aProfesor} o \refElem{aAdministrador} desea cancelar la eliminación después de mostrar el mensaje de confirmación}

  \Actor Presiona el botón {\bf No} en la mensaje de confirmación \refElem{IU-M08c}.
  \Sistema Redirige a la pantalla \refElem{IU-M08}.

\end{UCtrayectoriaA}
   

% \ucstEnEdicion     Al terminar una revisión/aprobación con observaciones
%                    y al inicio del CU.
%
% \ucstEnRevision    Al terminar la edición del CU (version += 0.1).
% \ucstEnAprobacion  Al pasar la revision sin observaciones.
% \ucstAprobado      Al ser aprobado por el usuario (version += 1.0)

\begin{UseCase}[%
Autor/David Flores Casanova,%
Version/0.1,%
Estado/\ucstEnRevision]%
%
{CU-S05}{Crear instancia (Preguntas diarias)}{%
%
 Permite al actor crear una nueva instancia de la actividad competencia uno contra uno en su curso.
 La conclusión de la trayectoria principal de esta caso de uso es una precondición para que
 algunos casos de uso del módulo de competencia puedan ejecutarse.\\%
 Este caso de uso es una extensión del caso de uso {\it Listar actividades disponibles} que es propio de moodle.}

	\UCitem[control]{Revisor}{ Sin asignar }
	\UCitem[control]{Último cambio}{ 13/NOV/19 }

 \UCsection{Atributos}

    \UCitem{Actor(es)}{%
        \refElem{aProfesor},
        \refElem{aAdministrador}
    }

	\UCitems{Propósito}{%
        \Titem Permitir al actor incluir en su curso una nueva instancia de la actividad de competencia uno contra uno.
	}

	\UCitem{Entradas}{\imprimeUC{entrada}}

	\UCitems{Origen}{%
        \Titem Mouse
        \Titem Teclado
	}

	\UCitem{Salidas}{\imprimeUC{salida}}

	\UCitem{Destino}{%
		\refElem{IU-M08}
	}

	\UCitems{Precondiciones}{%
        \Titem El complemento de preguntas diarias debe estar instalado en moodle.
	}

	\UCitems{Postcondiciones}{%
        \Titem La nueva instancia de la actividad debe mostrarse en la pantalla \refElem{IU-M08}.%

	}

	\UCitem{Reglas de negocio}{\imprimeUC{regla}}

	\UCitems{Errores}{%
        \Titem \UCerr{Err1}{%
            No se ingresó un campo requerido en el formulario de creación de la actividad,}{% CAUSA
            no se puede crear la nueva instancia de la actividad}% EFECTO
	}


 \UCsection[design]{Datos de Diseño}

	\UCitems[design]{Casos de Prueba}{%
        \Titem \refElem{CPC-C01}
	}

 \UCsection[admin]{Datos de Administración de Requerimiento}

	\UCitem[admin]{Observaciones}{}

\end{UseCase}

\subsubsection{Trayectorias del caso de uso}

\begin{UCtrayectoria}%
%
    \Actor Selecciona la actividad Gamedle - Preguntas diarias en la pantalla \refElem{IU-M08a}.
    \Sistema Muestra la descripción de la actividad Gamedle - Competencia uno contra uno en la pantalla.
    \Actor Presiona el botón {\bf Agregar} en la pantalla. \refTray{A}
    \Sistema Redirige a la pantalla \refElem{IU-S05}.
    \label{CU-C01-muestra-pantalla}
    \Actor Ingresa los datos correspondientes en el formulario.
    \Actor Presiona el botón {\bf Guardar cambios y regresar al curso}.\refTray{B} \refTray{C}
    \Sistema Valida que los campos ingresados sean válidos. \refTray{D} \refErr{Err1}
    \Sistema Establece los valores ingresados para la nueva instancia \refElem{seg-gmpregdiarias}, especificadas en el modelo de información.
    \Sistema Redirige a la pantalla \refElem{IU-M08} y muestra la nueva instancia creada en el curso.

\end{UCtrayectoria}

\begin{UCtrayectoriaA}[Fin del caso de uso]%
  {A}{El \refElem{aProfesor} o \refElem{aAdministrador} desea cancelar la creación de la nueva instancia después que se le muestra la descripción de la actividad}

  \Actor Presiona el botón {\bf cancelar} en la pantalla \refElem{IU-M08a}.
  \Sistema Cierra la pantalla \refElem{IU-M08a} y redirige a la pantalla \refElem{IU-M08}.

\end{UCtrayectoriaA}

\begin{UCtrayectoriaA}[Fin del caso de uso]{B}{El \refElem{aProfesor} o \refElem{aAdministrador} desea ver la nueva instancia de la actividad}

    \Actor Presiona el botón {\bf Guardar cambios y mostrar} de la pantalla \refElem{IU-S05}.
    \Sistema Valida que los campos ingresados sean válidos. \refTray{D} \refErr{Err1}
    \Sistema Establece los valores ingresados para la nueva instancia \refElem{seg-gmpregdiarias}, especificadas en el modelo de información.
    \Sistema Redirige a la pantalla \refElem{IU-S01}.

\end{UCtrayectoriaA}

\begin{UCtrayectoriaA}[Fin del caso de uso]%
  {C}{El \refElem{aProfesor} desea cancelar la creación de la nueva instancia después de mostrar el formulario de creación}

  \Actor Presiona el botón {\bf cancelar} en la pantalla \refElem{IU-S05}.
  \Sistema Redirige a la pantalla \refElem{IU-S05}.

\end{UCtrayectoriaA}

\begin{UCtrayectoriaA}{D}{Algún dato ingresado por el \refElem{aProfesor} o \refElem{aAdministrador} es inválido}

  \Sistema Muestra un mensaje de error "-Usted debe poner un valor aquí", en los campos de la pantalla \refElem{IU-S05} que sean requeridos.
  \Sistema Regresa al paso \ref{CU-C01-muestra-pantalla}

\end{UCtrayectoriaA}
   

% =========================================================
\clearpage
\subsection{Diseño}

\subsubsection{Interfaces del módulo de seguimiento}

    
\subsubsection{IU-S01 Pantalla principal de actividad preguntas diarias}

 Esta pantalla es la pantalla inicial que se le muestra al usuario una vez que entra a la actividad.
 Esta interfaz tiene 2 versiones dependiendo de si el usuario respondió el día actual la pregunta diaria.

    \IUfig{1}{modulos/seguimiento/IU/inicio_no_contestado}{IU-S01a}{%
        Pantalla principal de actividad - pregunta diaria no contestada}

    \IUfig{1}{modulos/seguimiento/IU/inicio_contestado}{IU-S01b}{%
        Pantalla principal de actividad - pregunta diaria contestada}

\subsubsection{Elementos Relevantes}

    \begin{itemize}
    \item {\bf Barra de progreso}
        Barra que muestra cuántas preguntas del banco de preguntas ya ha respondido el usuario, y cuántas le faltan por responder.
    \end{itemize}

\subsubsection{Acciones relevantes}

    \begin{itemize}
    \item {\bf Tabla de posiciones}
        Botón que redirige a la pantalla \refElem{IU-S02}
    \end{itemize}

\clearpage

    
\subsubsection{IU-S02  Pantalla de la tabla de posiciones de una actividad de preguntas diarias}

 En esta pantalla se muestra una tabla de posiciones que muestra el estado de cada usuario que participa en la actividad.
 El estado de cada usuario depende de las preguntas que este ya haya respondido.

    \IUfig{1}{modulos/seguimiento/IU/tabla_posiciones}{IU-S02}{%
        Tabla de posiciones de la actividad preguntas diarias}

\subsubsection{Elementos Relevantes}

    \begin{itemize}
    \item {\bf Tabla de posiciones}
        Tabla que muestra la posición actual de cada usuario, junto a su nombre y la barra de progreso que lleva en la instancia de la actividad.
    \end{itemize}

\subsubsection{Acciones relevantes}

    \begin{itemize}
    \item {\bf Inicio}
        Botón que redirige a la pantalla \refElem{IU-S01a} o \refElem{IU-S01b}
    \end{itemize}

\clearpage

    
\subsubsection{IU-S03  Contestar pregunta diaria}

 En esta pantalla se muestra la pregunta diaria que deberá responder el usuario.

    \IUfig{1}{modulos/seguimiento/IU/pregunta}{IU-S03}{%
        Contestar pregunta diaria}

\subsubsection{Elementos Relevantes}

    \begin{itemize}
    \item {\bf Pregunta a responder}
        Este pregunta es obtenida del banco de preguntas indicado por el maestro. Dicha pregunta es guardada y desplegada por moodle.
    \end{itemize}

\subsubsection{Acciones relevantes}

    \begin{itemize}
    \item {\bf Terminar}
        Botón que indica que el intento del usuario terminó y le redirige a la pantalla \refElem{IU-S04}.
    \end{itemize}

\clearpage

    
\subsubsection{IU-S04  Resultado de la pregunta diaria}

 En este pantalla se muestra el resultado de cómo le fue al usuario al responder la pregunta diaria.

    \IUfig{1}{modulos/seguimiento/IU/resultado}{IU-S04}{%
        Resultado de la pregunta diaria}

\subsubsection{Elementos Relevantes}

    \begin{itemize}
    \item {\bf Mensaje}
        Este mensaje cambia dependiendo la
        calificación que sacó el usuario al responder la pregunta. De 5 o menos, le dice que tiene que practicar más y de 6 o más le dice que contestó correctamente la pregunta.
    \end{itemize}

\subsubsection{Acciones relevantes}

    \begin{itemize}
    \item {\bf Volver}
        Botón que redirige a la pantalla \refElem{IU-S01b}.
    \end{itemize}

\clearpage

    
\subsubsection{IU-S05: Creación de instancia actividad (Preguntas diarias)}

    Esta pantalla se utiliza para crear una instancia de la actividad (Preguntas diarias).

    \IUfig{1}{modulos/seguimiento/IU/creacion_seguimiento}{IU-S05}{%
        Creación de instancia actividad (Preguntas diarias)}

\subsubsection{Elementos Relevantes}

    \begin{itemize}
    \item {\bf Nombre}
        Nombre que aparecerá en el curso que identifica a la instancia.
    \item {\bf Descripción}
        Descripción de la instancia de la actividad.

    \item {\bf Categoría de preguntas }
        Banco de preguntas de donde se sacarán las preguntas diarias.
    \end{itemize}

\subsubsection{Acciones relevantes}

    \begin{itemize}
    \item {\bf Guardar cambios y continuar editando}
        Botón que crea la instancia de la actividad y te permite seguir editando.
    \item {\bf Guardar cambios y mostrar}
        Botón que crea la instancia de la actividad y te redirige a la pantalla \refElem{IU-S01a}.
    \end{itemize}

\clearpage





\subsubsection{Diseño de complementos}



A continuación se presenta cómo los submódulos de competencia
se implmenetan en moodle.\\


\noindent Resumiendo el módulo de competencia tiene 2 actividades establecidas, llamadas; 
competencia uno contra uno y competencia uno contra sistema. 
Ambas actividades deben aparecer dentro de la lista de actividades de moodle. Para ello 
moodle cuenta con un tipo de complemento que se denomina \textbf{'mod'}, este tipo de complemento al ser instalado 
en una plataforma de moodle, crea una nueva opción a la lista de actividades.\\

\noindent Tomando en consideración lo anterior y que existe el complemento gamedlemaster, se presenta en la figura \ref{fig:diseno-comp-comp}
los complementos contemplados y las dependencias entre los mismos.


    \addfigure{1}{modulos/comp/diagrams/diseno_complementos}{fig:diseno-comp-comp}{Implementación del modulo de competencia}


Cada complemento en la figura \ref{fig:diseno-comp-comp} está representado con una cadena que sigue el formato 'tipo\_de\_complemento:nombre\_de\_complemento'. Los tipos de complemento son;
\begin{itemize}
    \item \textbf{mod} - Este complemento permite crear una actividad que aparece en la lista de actividades a agregar a un curso.
    \item \textbf{local} -  Este complemento moodle lo iterpreta como un comdín, el cual puede ser usado para múltiples propósitos relacionados con la gestión de la información.
    \item \textbf{block} - Este complemento permite desplegar un cuadro en la mayoría de las páginas de moodle, el cuál puede representar valores 
\end{itemize}

La función de cada uno de los complementos presentados en la figura \ref{fig:diseno-comp-comp} son:


\begin{itemize}
    \item \textbf{gmcompcpu} Definir la competencia uno contra sistema.
    \item \textbf{gmcompvs} Definir la competencia uno contra uno.
    \item \textbf{gmcs} Entregar las monedas por ganar cada una de las competencias anteriores.
\end{itemize}

El complemento de tipo  \textbf{'mod'} tiene un requerimiento en su nombre, el cual es; 'El nombre del complemento a instalar debe ser igual a un nombre
de una de las tablas en la base de datos'. Debido a que moodle no soporta nombres de complementos que contengan quiones bajos, el
nombre de la tabla ya no puede llevarlos.






\subsection{Pruebas}
     %
\TestCase{CPC-E13}{Eliminar usuario gamificado}




A continuación se enlistan los casos de prueba identificados y probados
correspondientes a cada uno de los casos de uso especificados. Los casos
de prueba listados a continuación son de dos tipos, los correctos e
incorrectos identificados por los prefijos CPC y CPI respectivamente.

\subsubsection{\refElem{CU-E01}}

    \begin{itemize}
    \TestCase{CPC-E01}{Instalar plugins del módulo de experiencia}
    \end{itemize}

\subsubsection{\refElem{CU-E02}}

    \begin{itemize}
    \TestCase{CPC-E02}{Realizar configuraciones del módulo de experiencia}
    \end{itemize}

\subsubsection{\refElem{CU-E02-1}}

    \begin{itemize}
    \TestCase{CPC-E02-1}{Realizar configuración de visualización de niveles}
    \TestCase{CPI-E02-1a}{Realizar configuraciones visuales con todos los datos
                          erróneos}
    \TestCase{CPI-E02-1b}{Configuraciones visuales con formato y nombre de imagen
                          inválidos}
    \end{itemize}

\subsubsection{\refElem{CU-E02-2}}

    \begin{itemize}
    \TestCase{CPC-E02-2a}{Realizar configuraciones del sistema de experiencia}
    \TestCase{CPC-E02-2b}{Realizar configuraciones con cursos iniciados}
    \TestCase{CPC-E02-2c}{Realizar configuraciones del sistema de experiencia con
                          alumnos con experiencia establecida}

    \TestCase{CPI-E02-2}{Realizar configuraciones del sistema de experiencia con
                         datos inválidos}
    \end{itemize}

\subsubsection{\refElem{CU-E02-3}}

    \begin{itemize}
    \TestCase{CPC-E02-3}{Realizar configuraciones del sistema de experiencia con datos
                         correctos}

    \TestCase{CPI-E02-3}{Realizar configuraciones del eventos con datos inválidos}
    \end{itemize}

\subsubsection{\refElem{CU-E03}}

    \begin{itemize}
    \TestCase{CPC-E03}{Desinstalar plugins del módulo de experiencia}
    \end{itemize}

\subsubsection{\refElem{CU-E04}}

    \begin{itemize}
    \TestCase{CPC-E04}{Crear un curso gamificado}
    \TestCase{CPI-E04}{Crear un curso gamificado con la experiencia deshabilitada}
    \end{itemize}

\subsubsection{\refElem{CU-E05}}

    \begin{itemize}
    \TestCase{CPC-E05}{Eliminar un curso gamificado sin alumnos inscritos}
    \TestCase{CPC-E05a}{Eliminar un curso gamificado con alumnos inscritos}
    \end{itemize}

\subsubsection{\refElem{CU-E06}}

    \begin{itemize}
    \TestCase{CPC-E06}{Eliminar el soporte para experiencia en un curso sin alumnos}
    \TestCase{CPC-E06a}{Eliminar un curso gamificado con alumnos inscritos}
    \end{itemize}

\subsubsection{\refElem{CU-E07}}

    \begin{itemize}
    \TestCase{CPC-E07}{Administrar experiencia en un curso recientemente creado}
    \TestCase{CPC-E07a}{Administrar experiencia en un curso con secciones completadas %
                       por alumnos}
    \TestCase{CPI-E07}{Administrar experiencia de un curso estableciendo valores
                       de experiencia incorrectos}
    \end{itemize}

\subsubsection{\refElem{CU-E08}}

    \begin{itemize}
    \TestCase{CPC-E08}{Crear una sección con soporte para experiencia en un curso
                       gamificado}
    \end{itemize}

\subsubsection{\refElem{CU-E12}}

    \begin{itemize}
    \TestCase{CPC-E12}{Crear un usuario gamificado (administrador)}
    \TestCase{CPC-E12a}{Crear un usuario gamificado mediante el auto-registro}
    \end{itemize}

\subsubsection{\refElem{CU-E13}}

    \begin{itemize}
    \TestCase{CPC-E13}{Eliminar usuario gamificado}
    \end{itemize}
