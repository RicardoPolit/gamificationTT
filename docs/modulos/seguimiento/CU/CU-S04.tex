
% \ucstEnEdicion     Al terminar una revisión/aprobación con observaciones
%                    y al inicio del CU.
%
% \ucstEnRevision    Al terminar la edición del CU (version += 0.1).
% \ucstEnAprobacion  Al pasar la revision sin observaciones.
% \ucstAprobado      Al ser aprobado por el usuario (version += 1.0)

\begin{UseCase}[%
Autor/David Flores Casanova,%
Version/0.1,%
Estado/\ucstEnRevision]%
%
{CU-S04}{Eliminar instancia (Preguntas diarias)}{%
%
 Permite al actor eliminar una instancia de la actividad de preguntas diarias en el curso.
 Este caso de uso es una extensión del caso de uso {\it Ver curso} que es propio de moodle.}

	\UCitem[control]{Revisor}{ Sin asignar }
	\UCitem[control]{Último cambio}{ 13/NOV/19 }

 \UCsection{Atributos}

    \UCitem{Actor(es)}{%
        \refElem{aProfesor},
        \refElem{aAdministrador}
    }

	\UCitems{Propósito}{%
        \Titem Permitir quitar la actividad de su curso.
	}

	\UCitem{Entradas}{\imprimeUC{entrada}}

	\UCitems{Origen}{%
        \Titem Mouse
	}

	\UCitem{Salidas}{\imprimeUC{salida}}

	\UCitem{Destino}{%
		\refElem{IU-M08}
	}

	\UCitems{Precondiciones}{%
        \Titem El complemento de preguntas diarias debe estar instalado en moodle.
        \Titem La instancia de la actividad de preguntas diarias debe estar creada.
	}

	\UCitems{Postcondiciones}{%
        \Titem La instancia de la actividad eliminada no debe mostrarse en la pantalla \refElem{IU-M08}.%

	}

	\UCitem{Reglas de negocio}{\imprimeUC{regla}}

	\UCitems{Errores}{%
	}

 \UCsection[design]{Datos de Diseño}

	\UCitems[design]{Casos de Prueba}{%
	}

 \UCsection[admin]{Datos de Administración de Requerimiento}

	\UCitem[admin]{Observaciones}{}

\end{UseCase}

\subsubsection{Trayectorias del caso de uso}

\begin{UCtrayectoria}%
%

    \Actor Activa la edición del curso en la pantalla \refElem{IU-M08}.
    \Sistema Redirige a la pantalla de edición del curso \refElem{IU-M08aa}.
    \Actor Presiona el botón {\bf Editar} de la instancia que desea eliminar.
    \Sistema Despliega el menú \refElem{IU-M08b}.
    \Actor Presiona el botón {\bf Eliminar} del menú desplegable \refElem{IU-M08b}.
    \Sistema Despliega mensaje de confirmación de eliminación. \refElem{IU-M08c}
    \Actor Presiona el botón {\bf Sí}. \refTray{A}
    \Sistema  Elimina la instancia y los valores de la instancia \refElem{seg-gmpregdiarias}, y los valores que dependen de ella \refElem{seg-gmdl-intento-diario}.
    \Sistema Redirige a la pantalla \refElem{IU-M08}

\end{UCtrayectoria}

\begin{UCtrayectoriaA}[Fin del caso de uso]{A}{El \refElem{aProfesor} o \refElem{aAdministrador} desea cancelar la eliminación después de mostrar el mensaje de confirmación}

  \Actor Presiona el botón {\bf No} en la mensaje de confirmación \refElem{IU-M08c}.
  \Sistema Redirige a la pantalla \refElem{IU-M08}.

\end{UCtrayectoriaA}
