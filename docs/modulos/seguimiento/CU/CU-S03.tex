
% \ucstEnEdicion     Al terminar una revisión/aprobación con observaciones
%                    y al inicio del CU.
%
% \ucstEnRevision    Al terminar la edición del CU (version += 0.1).
% \ucstEnAprobacion  Al pasar la revision sin observaciones.
% \ucstAprobado      Al ser aprobado por el usuario (version += 1.0)

%\addfigure[(adaptado de {\it For The Win} \cite{ForTheWin})]%
%    {.4}{investigacion/images/forthewin}{fig:ForTheWin}%
%    {Jerarquía de elementos de juego segun For The Win}

\begin{UseCase}[%
Autor/David Flores Casanova,%
Version/0.1,%
Estado/\ucstEnRevision]%
%
{CU-S03}{Responder pregunta diaria}{%
%
 El actor responde una pregunta diaria escogida al azara del banco de preguntas si aún tiene disponibles.
 Este caso de uso es una extensión del caso de uso \refElem{CU-S01}.}

	\UCitem[control]{Revisor}{ Sin asignar }
	\UCitem[control]{Último cambio}{ 13/NOV/19 }

 \UCsection{Atributos}

    \UCitem{Actor(es)}{%
        \refElem{aEstudiante},
        \refElem{aProfesor},
        \refElem{aAdministrador}
    }

	\UCitems{Propósito}{%
        \Titem Avanzar en la actividad de preguntas diarias.
	}

	\UCitem{Entradas}{\imprimeUC{entrada}}

	\UCitems{Origen}{%
        \Titem Mouse
	}

	\UCitem{Salidas}{\imprimeUC{salida}
        Pregunta a responder generada por moodle}

	\UCitem{Destino}{%
		\refElem{IU-S02}
	}

	\UCitems{Precondiciones}{%
        \Titem El complemento de preguntas diarias esté instalado en moodle.
        \Titem La instancia de la actividad de preguntas diarias debe estar creada.
        \Titem Que el actor no haya respondido la pregunta diaria del día.
        % Realizar el caso de uso "listar actividades disponibles?"
        % \Titem Si se trata de una actualización de un plugin la versión de este debe
               % cumplir con la regla \refElem{BR-M02}.
	}

	\UCitems{Postcondiciones}{%
        \Titem Se agrega la pregunta a las preguntas ya respondidas por el actor.%

	}

	\UCitem{Reglas de negocio}{\imprimeUC{regla}}

	\UCitems{Errores}{%
	}

 \UCsection[design]{Datos de Diseño}

	\UCitems[design]{Casos de Prueba}{%
        \Titem \refElem{CPC-S03-1}
        \Titem \refElem{CPC-S03-2}
        \Titem \refElem{CPC-S03-3}
        \Titem \refElem{CPI-S03-4}
	}

 \UCsection[admin]{Datos de Administración de Requerimiento}

	\UCitem[admin]{Observaciones}{}

\end{UseCase}

\subsubsection{Trayectorias del caso de uso}

\begin{UCtrayectoria}%
%
    \Actor Presiona el botón {\bf Contestar} de la pantalla \refElem{IU-S01a}.
    \Sistema Obtiene una pregunta de manera aleatoria de las preguntas que aún no ha respondido el actor. \refTray{A}
    \Sistema Redirige a la pantalla \refElem{IU-S03}.
    \Sistema Guarda que se intenta responder una pregunta diaria.
    \Actor Responde la pregunta.
    \Actor Presiona el botón {\bf Terminar}.
    \Sistema Califica el intento. \refTray{B}
    \Sistema Redirige a la pantalla \refElem{IU-S04}.
    \Sistema Otorga experiencia al actor por haber contestado correctamente la pregunta diaria.
    \Sistema Otorga monedas al actor por haber contestado correctamente la pregunta diaria.


\end{UCtrayectoria}



\begin{UCtrayectoriaA}[Fin del caso de uso]{A}{El actor ya ha contestado todas las preguntas disponibles}

  \Sistema Muestra el mensaje 'Has terminado de responder todas las preguntas.'.

\end{UCtrayectoriaA}


\begin{UCtrayectoriaA}[Fin del caso de uso]{B}{El actor no contestó correctamente la pregunta}

    \Sistema Redirige a la pantalla \refElem{IU-S04}.

\end{UCtrayectoriaA}
