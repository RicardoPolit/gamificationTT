
% \ucstEnEdicion     Al terminar una revisión/aprobación con observaciones
%                    y al inicio del CU.
%
% \ucstEnRevision    Al terminar la edición del CU (version += 0.1).
% \ucstEnAprobacion  Al pasar la revision sin observaciones.
% \ucstAprobado      Al ser aprobado por el usuario (version += 1.0)

%\addfigure[(adaptado de {\it For The Win} \cite{ForTheWin})]%
%    {.4}{investigacion/images/forthewin}{fig:ForTheWin}%
%    {Jerarquía de elementos de juego segun For The Win}

\begin{UseCase}[%
Autor/David Flores Casanova,%
Version/0.1,%
Estado/\ucstEnRevision]%
%
{CU-S02}{Ver tabla de posiciones (Preguntas diarias)}{%
%
 Permite al actor ver la tabla de posiciones de una instancia de la actividadpreguntas diarias.
 Este caso de uso es una extensión del caso de uso \refElem{CU-S01}.}

	\UCitem[control]{Revisor}{ Sin asignar }
	\UCitem[control]{Último cambio}{ 13/NOV/19 }

 \UCsection{Atributos}

    \UCitem{Actor(es)}{%
        \refElem{aEstudiante},
        \refElem{aProfesor},
        \refElem{aAdministrador}
    }

	\UCitems{Propósito}{%
        \Titem Ver la posición de cada usuario que participa en la instancia de la actividad.
	}

	\UCitem{Entradas}{\imprimeUC{entrada}}

	\UCitems{Origen}{%
        \Titem Mouse
	}

	\UCitem{Salidas}{\imprimeUC{salida}
	}

	\UCitem{Destino}{%
		\refElem{IU-S02}
	}

	\UCitems{Precondiciones}{%
        \Titem El complemento de preguntas diarias esté instalado en moodle.
        \Titem La instancia de la actividad de preguntas diarias debe estar creada.
        % Realizar el caso de uso "listar actividades disponibles?"
        % \Titem Si se trata de una actualización de un plugin la versión de este debe
               % cumplir con la regla \refElem{BR-M02}.
	}

	\UCitems{Postcondiciones}{%
        \Titem Se muestra la pantalla de tabla de posiciones perteneciente a la instancia de la actividad preguntas diarias.%

	}

	\UCitem{Reglas de negocio}{\imprimeUC{regla}}

	\UCitems{Errores}{%
	}

 \UCsection[design]{Datos de Diseño}

	\UCitems[design]{Casos de Prueba}{%
	}

 \UCsection[admin]{Datos de Administración de Requerimiento}

	\UCitem[admin]{Observaciones}{}

\end{UseCase}

\subsubsection{Trayectorias del caso de uso}

\begin{UCtrayectoria}%
%
    \Actor Presiona el botón {\bf Tabla posiciones} de la pantalla \refElem{IU-S01a} o \refElem{IU-S01b}.
    \Sistema Redirige a la pantalla \refElem{IU-S02}.
    \Sistema Carga el estado de cada usuario que ha participado en la actividad y los ordene de acuerdo a cuántas preguntas diarias han contestado de mayor a menor.

\end{UCtrayectoria}
