
% \ucstEnEdicion     Al terminar una revisión/aprobación con observaciones
%                    y al inicio del CU.
%
% \ucstEnRevision    Al terminar la edición del CU (version += 0.1).
% \ucstEnAprobacion  Al pasar la revision sin observaciones.
% \ucstAprobado      Al ser aprobado por el usuario (version += 1.0)

%\addfigure[(adaptado de {\it For The Win} \cite{ForTheWin})]%
%    {.4}{investigacion/images/forthewin}{fig:ForTheWin}%
%    {Jerarquía de elementos de juego segun For The Win}

\begin{UseCase}[%
Autor/David Flores Casanova,%
Version/0.1,%
Estado/\ucstEnRevision]%
%
{CU-S01}{Ver estado de actividad}{%
%
 El actor puede entrar a la actividad y ver el progreso que lleva.
 Este caso de uso es una extensión del caso de uso {\it Ver curso} que es propio de moodle.}

	\UCitem[control]{Revisor}{ Sin asignar }
	\UCitem[control]{Último cambio}{ 13/NOV/19 }

 \UCsection{Atributos}

    \UCitem{Actor(es)}{%
        \refElem{aEstudiante},
        \refElem{aProfesor},
        \refElem{aAdministrador}
    }

	\UCitems{Propósito}{%
        \Titem El actor desea ver el progreso que lleva en la actividad, así como ver, si ya respondió la pregunta diaria.
	}

	\UCitem{Entradas}{\imprimeUC{entrada}}

	\UCitems{Origen}{%
        \Titem Mouse
	}

	\UCitem{Salidas}{\imprimeUC{salida}}

	\UCitem{Destino}{%
		\refElem{IU-S01a}
	}

	\UCitems{Precondiciones}{%
        \Titem El plugin de preguntas diarias debe estar instalado en moodle.
        \Titem La instancia de la actividad de preguntas diarias debe estar creada.
	}

	\UCitems{Postcondiciones}{%
        \Titem Se muestra el estado de la pregunta diaria para el actor.

	}

	\UCitem{Reglas de negocio}{\imprimeUC{regla}}

	\UCitems{Errores}{%
	}

 \UCsection[design]{Datos de Diseño}

	\UCitems[design]{Casos de Prueba}{%
	}

 \UCsection[admin]{Datos de Administración de Requerimiento}

	\UCitem[admin]{Observaciones}{}

\end{UseCase}

\subsubsection{Trayectorias del caso de uso}

\begin{UCtrayectoria}%
%
 	\Actor Presiona el nombre de la instancia de la actividad a la que quiere acceder en la pantalla \refElem{IU-M07}.
    \Sistema Verifica que el actor esté dado de alto como actor gamificado. \refTray{A}
    \label{CU-S01-cracion-usuario}
	\Sistema Verifica que el actor no haya respondido la pregunta diaria. \refTray{B}
    \Sistema Redirige a la pantalla principal de la instancia \refElem{IU-S01a}.
	\Sistema Obtiene el número de preguntas que el usuario ha respondido.
    \label{CU-S01-mostrar-informacion}
	y la cantidad de preguntas que le faltan por responder y genera la sección de la barra de progreso y la muestra en pantalla.

\end{UCtrayectoria}

\begin{UCtrayectoriaA}{A}{El actor no está dado de alta como usuario gamificado (\refElem{xp-user})}

  \Sistema Registra al actor en la entidad (\refElem{xp-user}).
    \item Se regresa al paso \ref{CU-S01-cracion-usuario} de la trayectoria principal.


\end{UCtrayectoriaA}


\begin{UCtrayectoriaA}{B}{El actor ya respondió la pregunta diaria (\refElem{xp-user})}

    \Sistema Redirige a la pantalla principal de la instancia \refElem{IU-S01b}.
    \item Se regresa al paso \ref{CU-S01-mostrar-informacion} de la trayectoria principal.


\end{UCtrayectoriaA}



\UCExtensionPoint{Ver tabla de posiciones}{%

    El actor desea ver la tabla de posiciones de una instancia de preguntas diarias.
%
    }{Al final de la trayectoria principal del caso de uso.}{\refElem{CU-S02}}

\UCExtensionPoint{Responder pregunta diaria}{%

    El actor desea responder la pregunta del día.
%
    }{Al final de la trayectoria principal del caso de uso.}{\refElem{CU-S03}}
