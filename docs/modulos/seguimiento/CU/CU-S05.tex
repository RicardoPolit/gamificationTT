
% \ucstEnEdicion     Al terminar una revisión/aprobación con observaciones
%                    y al inicio del CU.
%
% \ucstEnRevision    Al terminar la edición del CU (version += 0.1).
% \ucstEnAprobacion  Al pasar la revision sin observaciones.
% \ucstAprobado      Al ser aprobado por el usuario (version += 1.0)

\begin{UseCase}[%
Autor/David Flores Casanova,%
Version/0.1,%
Estado/\ucstEnRevision]%
%
{CU-S05}{Crear instancia (Preguntas diarias)}{%
%
 Permite al actor crear una nueva instancia de la actividad competencia uno contra uno en su curso.
 La conclusión de la trayectoria principal de esta caso de uso es una precondición para que
 algunos casos de uso del módulo de competencia puedan ejecutarse.\\%
 Este caso de uso es una extensión del caso de uso {\it Listar actividades disponibles} que es propio de moodle.}

	\UCitem[control]{Revisor}{ Sin asignar }
	\UCitem[control]{Último cambio}{ 13/NOV/19 }

 \UCsection{Atributos}

    \UCitem{Actor(es)}{%
        \refElem{aProfesor},
        \refElem{aAdministrador}
    }

	\UCitems{Propósito}{%
        \Titem Permitir al actor incluir en su curso una nueva instancia de la actividad de competencia uno contra uno.
	}

	\UCitem{Entradas}{\imprimeUC{entrada}}

	\UCitems{Origen}{%
        \Titem Mouse
        \Titem Teclado
	}

	\UCitem{Salidas}{\imprimeUC{salida}}

	\UCitem{Destino}{%
		\refElem{IU-M08}
	}

	\UCitems{Precondiciones}{%
        \Titem El complemento de preguntas diarias debe estar instalado en moodle.
	}

	\UCitems{Postcondiciones}{%
        \Titem La nueva instancia de la actividad debe mostrarse en la pantalla \refElem{IU-M08}.%

	}

	\UCitem{Reglas de negocio}{\imprimeUC{regla}}

	\UCitems{Errores}{%
        \Titem \UCerr{Err1}{%
            No se ingresó un campo requerido en el formulario de creación de la actividad,}{% CAUSA
            no se puede crear la nueva instancia de la actividad}% EFECTO
	}


 \UCsection[design]{Datos de Diseño}

	\UCitems[design]{Casos de Prueba}{%
        \Titem \refElem{CPC-S05-1}
        \Titem \refElem{CPI-S05-2}
	}

 \UCsection[admin]{Datos de Administración de Requerimiento}

	\UCitem[admin]{Observaciones}{}

\end{UseCase}

\subsubsection{Trayectorias del caso de uso}

\begin{UCtrayectoria}%
%
    \Actor Selecciona la actividad Gamedle - Preguntas diarias en la pantalla \refElem{IU-M08a}.
    \Sistema Muestra la descripción de la actividad Gamedle - Competencia uno contra uno en la pantalla.
    \Actor Presiona el botón {\bf Agregar} en la pantalla. \refTray{A}
    \Sistema Redirige a la pantalla \refElem{IU-S05}.
    \label{CU-C01-muestra-pantalla}
    \Actor Ingresa los datos correspondientes en el formulario.
    \Actor Presiona el botón {\bf Guardar cambios y regresar al curso}.\refTray{B} \refTray{C}
    \Sistema Valida que los campos ingresados sean válidos. \refTray{D} \refErr{Err1}
    \Sistema Establece los valores ingresados para la nueva instancia \refElem{seg-gmpregdiarias}, especificadas en el modelo de información.
    \Sistema Redirige a la pantalla \refElem{IU-M08} y muestra la nueva instancia creada en el curso.

\end{UCtrayectoria}

\begin{UCtrayectoriaA}[Fin del caso de uso]%
  {A}{El \refElem{aProfesor} o \refElem{aAdministrador} desea cancelar la creación de la nueva instancia después que se le muestra la descripción de la actividad}

  \Actor Presiona el botón {\bf cancelar} en la pantalla \refElem{IU-M08a}.
  \Sistema Cierra la pantalla \refElem{IU-M08a} y redirige a la pantalla \refElem{IU-M08}.

\end{UCtrayectoriaA}

\begin{UCtrayectoriaA}[Fin del caso de uso]{B}{El \refElem{aProfesor} o \refElem{aAdministrador} desea ver la nueva instancia de la actividad}

    \Actor Presiona el botón {\bf Guardar cambios y mostrar} de la pantalla \refElem{IU-S05}.
    \Sistema Valida que los campos ingresados sean válidos. \refTray{D} \refErr{Err1}
    \Sistema Establece los valores ingresados para la nueva instancia \refElem{seg-gmpregdiarias}, especificadas en el modelo de información.
    \Sistema Redirige a la pantalla \refElem{IU-S01}.

\end{UCtrayectoriaA}

\begin{UCtrayectoriaA}[Fin del caso de uso]%
  {C}{El \refElem{aProfesor} desea cancelar la creación de la nueva instancia después de mostrar el formulario de creación}

  \Actor Presiona el botón {\bf cancelar} en la pantalla \refElem{IU-S05}.
  \Sistema Redirige a la pantalla \refElem{IU-S05}.

\end{UCtrayectoriaA}

\begin{UCtrayectoriaA}{D}{Algún dato ingresado por el \refElem{aProfesor} o \refElem{aAdministrador} es inválido}

  \Sistema Muestra un mensaje de error "-Usted debe poner un valor aquí", en los campos de la pantalla \refElem{IU-S05} que sean requeridos.
  \Sistema Regresa al paso \ref{CU-C01-muestra-pantalla}

\end{UCtrayectoriaA}
