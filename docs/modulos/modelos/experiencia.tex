
\section{Relaciones del módulo de experiencia}

 A continuación se presenta la especificación de las relaciones del esquema de base
 de datos de moodle que son relevantes para el desarrollo del módulo de experiencia.

    \begin{cdtEntidad}{xp-general-settings}{Configuraciones Generales}{
    A pesar de que las configuraciones son almacenadas en la entidad 
    \refElem{mdl-config-plugins}. Se decidió representar cómo una entidad diferente 
    el conjunto de valores para las configuraciones generales del módulo de 
    experiencia.}

        \brAttr{activated}{activado}{tBoolean}{%
            Valor que indica si el módulo de experiencia está activado o no.\par

            \it Restricciones:
            \refElem{tRequired}.
            \refElem{tDefault}$:verdadero$
        }

        \brAttr{events}{eventos activados}{tBoolean}{%
            Valor que indica si se entregará experiencia a los eventos establecidos
            en la entidad \refElem{xp-events-settings}.\par

            \it Restricciones:
            \refElem{tRequired}.
            \refElem{tDefault}$:verdadero$
        }
    \end{cdtEntidad}

    \begin{cdtEntidad}{xp-visual-settings}{Configuraciones Visuales}{
    De la misma forma que las configuraciones generales, las configuraciones visuales,
    los datos correspondientes a las configuraciones visuales son representados 
    mediante esta entidad.}

        \brAttr{title}{Título de niveles}{tVarchar}{%
            Cadena que contiene el título que tienen por defecto todos los niveles.\par

            \it Restricciones:
            \refElem{tLength}$0-30$,
            \refElem{tDefault} ``Gamedle Levels'',
            \refElem{tRequired}
        }

        \brAttr{description}{Descripción}{tVarchar}{%
            Cadena que contiene la descripción qe tiene por defecto todos los niveles.
            \par

            \it Restricciones:
            \refElem{tRequired},
            \refElem{tDefault} ``New reached level''
        }

        \brAttr{message}{Mensaje}{tVarchar}{%
            Cadena que contiene el mensaje de felicitaciones que tienen por defecto 
            todos los niveles.\par

            \it Restricciones:
            \refElem{tRequired}
            \refElem{tDefault} ``CONGRATULATIONS''
        }

        \brAttr{colorLvl}{Color de número de nivel}{tColor}{%
            Valor que contiene el código de color con el cual se coloreará el número 
            del nivel.\par

            \it Restricciones:
            \refElem{tRequired},
            \refElem{tDefault} \#0B619F
        }

        \brAttr{colorBar}{Color de la barra de progreso}{tColor}{%
            Valor que contiene el código de color con el cual se pintará el avance en 
            la barra de progreso del nivel.\par

            \it Restricciones:
            \refElem{tRequired},
            \refElem{tDefault} \#0B619F
        }

        \brAttr{image}{Imagen}{tImage}{%
            Imagen que se desplegará en los niveles cómo el escudo por defecto. \par

            \it Restricciones:
            \refElem{tRequired}, \refElem{tDefault} image.jpg (incluida en los
            archivos del plugin)
        }
    \end{cdtEntidad}

    \begin{cdtEntidad}{xp-scheme-settings}{Configuraciones de Comportamiento}{
    De la misma forma que las configuraciones generales, los datos correspondientes a 
    las configuraciones visuales son representados mediante esta entidad.}

        \brAttr{increment}{tipo de incremento}{tInt}{%
            Valor que indica que tipo de incremento es usado para los niveles de 
            experiencia, El valor 0 indica un incremento lineal y el valor 1 indica 
            un incremento porcentual.\par

            \it Restricciones:
            \refElem{tRequired},
            \refElem{tRange} [0,1],
            \refElem{tDefault}$:1$
        }

        \brAttr{incrementValue}{valor de incremento}{tDouble}{%
            Valor que indica el factor o valor del incremento para realizar el 
            calculo de la experiencia requerida para subir de nivel.\par

            \it Restricciones:
            \refElem{tRequired},\par
            Si el \refElem{xp-scheme-settings.increment} es 0, \refElem{tNatural}.
            En caso contrario \refElem{tRange} (1,2],
            \refElem{tDefault}$:1.3$
        }

        \brAttr{levelXP}{experiencia del nivel 1}{tInt}{%
            Valor que indica la cantidad de experiencia que tendrá el primer nivel 
            para ser completado.\par

            \it Restricciones:
            \refElem{tRequired},
            \refElem{tNatural},
            \refElem{tDefault}$:1000$
        }

        \brAttr{courseXP}{experiencia por curso}{tInt}{%
            Valor que indica la cantidad de experiencia que otorgan los cursos al ser
            completados.\par

            \it Restricciones:
            \refElem{tRequired},
            \refElem{tDefault}$:1500$
        }

    \end{cdtEntidad}

    \begin{cdtEntidad}{xp-events-settings}{Configuraciones de Eventos}{%
    De la misma forma que las configuraciones anteriores, los datos correspondientes 
    a las configuraciones de eventos son representados mediante esta entidad.}

        \brAttr{competence}{competencia ganada}{tBoolean}{%
            Valor que indica la se brindarán puntos de experiencia al evento que 
            es emitido cuando una competencia es ganada.\par

            \it Restricciones:
            \refElem{tRequired},
            \refElem{tDefault}$:true$
        }

        \brAttr{competenceXP}{experiencia de competencia}{tInt}{%
            Valor que indica la cantidad de experiencia que otorgará cuando el
            evento de \refElem{xp-events-settings.competence} sea emitido.\par

            \it Restricciones:
            \refElem{tRequired},
            \refElem{tDefault}$:1500$
        }

    \end{cdtEntidad}

    \begin{cdtEntidad}{xp-user}{Usuario gamificado}{%
    Esta entidad es una especialización de la entidad \refElem{mdl-user} la cual 
    permite agregar atributos cómo la cantidad de experiencia obtenida y el nivel 
    actual del usuario.}

        \brAttr{id}{id}{tInt}{%
            Es el número que representa el identificador único para cada usuario
            gamificado en la plataforma.\par

            \it Restricciones:
            \refElem{tPrimaryKey},
            \refElem{tAutoIncrement}
        }

        \brAttr{mdl-user}{mdl-usuario}{tInt}{%
            Es el número que permite conocer a que usuario de moodle pertecene la 
            información de un usuario gamificado.\par

            \it Restricciones:
            \refElem{tForeignKey},
            \refElem{tRequired},
            \refElem{tUniqueKey}
        }

        \brAttr{level}{nivel}{tNatural}{%
            Representa el nivel actual de experiencia que tiene un usuario gamificado 
            en moodle. \par

            \it Restricciones:
            \refElem{tDefault} 1
        }

        \brAttr{levelxp}{experiencia del nivel}{tNatural}{%
            Representa el nivel actual de experiencia que tiene un usuario gamificado 
            en moodle.\par

            \it Restricciones:
            \refElem{tDefault} 0
        }

        \brAttr{xp}{experiencia}{tInt}{%
            Representa los puntos de experiencia que tiene un usuario en moodle \par

            \it Restricciones:
            \refElem{tPositive},
            \refElem{tDefault} 0
        }

    \end{cdtEntidad}

    \begin{cdtEntidad}{xp-course}{Curso gamificado}{%
    Un curso gamificado es un \refElem{mdl-course} el cual tiene el
    \refElem{mdl-course.format} de curso {\it ``gamedle''}, a pesar de que las
    opciones de este formato son almacenadas en la tabla 
    \refElem{mdl-course-format-options}, se decidió incluir esta entidad en el
    dmodelo de información ya que reune las características que pertenecen a un curso
    gamificado.}

        \brAttr{id}{Id}{tInt}{%
            . \par

            \it Restricciones:
            \refElem{tPrimaryKey},
            \refElem{tAutoIncrement}.
        }

        \brAttr{format}{formato}{tVarchar}{%
            Para que un curso pueda ser manejado como un curso gamificado este 
            atributo debe de tener como valor ``Gamedle''. \par

            \it Restricciones:
        }

        \brAttr{xpEnabled}{experiencia habilitada}{tBoolean}{%
            Para que un curso pueda ser manejado como un curso gamificado este 
            atributo debe de tener como valor ``Gamedle''. \par

            \it Restricciones:
        }

        \brAttr{sections}{número de secciones}{tInt}{%
        }

        \brAttr{hiddensections}{secciones ocultas}{tVarchar}{%
        }

        \brAttr{coursedisplay}{aspecto del curso}{tVarchar}{%
        }
    \end{cdtEntidad}

    \begin{cdtEntidad}{xp-course-section}{Sección de curso gamificado}{%
    Esta entidad }
        \brAttr{id}{Id}{tInt}{%
            . \par

            \it Restricciones:
            \refElem{tPrimaryKey},
            \refElem{tAutoIncrement}.
        }
    \end{cdtEntidad}

    \begin{cdtEntidad}{xp-section-reward}{Recompensa de sección}{%
    Esta entidad }
    \end{cdtEntidad}
