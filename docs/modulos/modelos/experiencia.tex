\clearpage
\subsection{Entidades del módulo de experiencia}


 A continuación se presentan las entidades del módulo de seguimiento,
 las cuales están representadas en la figura  \ref{fig:BD-ER-XP},
 el cual muestra en color amarillo las entidades propias del módulo,
 en azul las entidades que se utilizan de moodle  y en blanco las entidades generales.

    \addfigure{0.8}{analisis/diagrams/db_exp}{fig:BD-ER-XP}{Esquema de la base de datos del módulo 'Experiencia'}

    \begin{cdtEntidad}{xp-general-settings}{Configuraciones generales}{
    A pesar de que las configuraciones son almacenadas en la entidad
    \refElem{mdl-config-plugins}. Se decidió representar cómo una entidad diferente
    el conjunto de valores para las configuraciones generales del módulo de
    experiencia.}

        \brAttr{activated}{activado}{tBoolean}{%
            Valor que indica si el módulo de experiencia está activado o no.\par

            \it Restricciones:
            \refElem{tRequired}.
            \refElem{tDefault} $:verdadero$
        }

        \brAttr{events}{eventos activados}{tBoolean}{%
            Valor que indica si se entregará experiencia a los eventos establecidos
            en la entidad \refElem{xp-events-settings}.\par

            \it Restricciones:
            \refElem{tRequired}.
            \refElem{tDefault} $:verdadero$
        }
    \end{cdtEntidad}

    \begin{cdtEntidad}{xp-visual-settings}{Configuraciones Visuales}{
    De la misma forma que las configuraciones generales, las configuraciones visuales,
    los datos correspondientes a las configuraciones visuales son representados
    mediante esta entidad.}

        \brAttr{title}{Título de niveles}{tVarchar}{%
            Cadena que contiene el título que tienen por defecto todos los niveles.\par

            \it Restricciones:
            \refElem{tLength} $0-30$,
            \refElem{tDefault} ``Gamedle Levels'',
            \refElem{tRequired}
        }

        \brAttr{description}{Descripción}{tVarchar}{%
            Cadena que contiene la descripción que tiene por defecto todos los niveles.
            \par

            \it Restricciones:
            \refElem{tRequired},
            \refElem{tDefault} ``New reached level''
        }

        \brAttr{message}{Mensaje}{tVarchar}{%
            Cadena que contiene el mensaje de felicitaciones que tienen por defecto
            todos los niveles.\par

            \it Restricciones:
            \refElem{tRequired}
            \refElem{tDefault} ``CONGRATULATIONS''
        }

        \brAttr{colorLvl}{Color de número de nivel}{tColor}{%
            Valor que contiene el código de color con el cual se coloreará el número
            del nivel.\par

            \it Restricciones:
            \refElem{tRequired},
            \refElem{tDefault} \#0B619F
        }

        \brAttr{colorBar}{Color de la barra de progreso}{tColor}{%
            Valor que contiene el código de color con el cual se pintará el avance en
            la barra de progreso del nivel.\par

            \it Restricciones:
            \refElem{tRequired},
            \refElem{tDefault} \#0B619F
        }

        \brAttr{image}{Imagen}{tImage}{%
            Imagen que se desplegará en los niveles cómo el escudo por defecto. \par

            \it Restricciones:
            \refElem{tRequired}, \refElem{tDefault} image.jpg (incluida en los
            archivos del plugin)
        }
    \end{cdtEntidad}

    \begin{cdtEntidad}{xp-scheme-settings}{Configuraciones de Comportamiento}{
    De la misma forma que las configuraciones generales, los datos correspondientes a
    las configuraciones visuales son representados mediante esta entidad.}

        \brAttr{increment}{tipo de incremento}{tInt}{%
            Valor que indica que tipo de incremento es usado para los niveles de
            experiencia, El valor 0 indica un incremento lineal y el valor 1 indica
            un incremento porcentual.\par

            \it Restricciones:
            \refElem{tRequired},
            \refElem{tRange} [0,1],
            \refElem{tDefault}$:1$
        }

        \brAttr{incrementValue}{valor de incremento}{tDouble}{%
            Valor que indica el factor o valor del incremento para realizar el
            calculo de la experiencia requerida para subir de nivel.\par

            \it Restricciones:
            \refElem{tRequired},\par
            Si el \refElem{xp-scheme-settings.increment} es 0, \refElem{tNatural}.
            En caso contrario \refElem{tRange} (1,2],
            \refElem{tDefault}$:1.3$
        }

        \brAttr{levelXP}{experiencia del nivel 1}{tInt}{%
            Valor que indica la cantidad de experiencia que tendrá el primer nivel
            para ser completado.\par

            \it Restricciones:
            \refElem{tRequired},
            \refElem{tNatural},
            \refElem{tDefault}$:1000$
        }

        \brAttr{courseXP}{experiencia por curso}{tInt}{%
            Valor que indica la cantidad de experiencia que otorgan los cursos al ser
            completados.\par

            \it Restricciones:
            \refElem{tRequired},
            \refElem{tDefault}$:1500$
        }

    \end{cdtEntidad}

    \begin{cdtEntidad}{xp-events-settings}{Configuraciones de Eventos}{%
    De la misma forma que las configuraciones anteriores, los datos correspondientes
    a las configuraciones de eventos son representados mediante esta entidad.}

        \brAttr{competencecpuevent}{competencia contra sistema ganada}{tBoolean}{%
            Valor que indica si se brindarán puntos de experiencia al evento que
            es emitido cuando una competencia contra el sistema es ganada.\par

            \it Restricciones:
            \refElem{tRequired},
            \refElem{tDefault}$: true$
        }

        \brAttr{competencecpuxp}{experiencia de competencia contra sistema}{tInt}{%
            Valor que indica la cantidad de experiencia que otorgará cuando el
            evento de \refElem{xp-events-settings.competencecpuevent} sea emitido.\par

            \it Restricciones:
            \refElem{tRequired},
            \refElem{tDefault}$: 1000$
        }

        \brAttr{competencevsevent}{competencia ganada}{tBoolean}{%
            Valor que indica si se brindarán puntos de experiencia al evento que
            es emitido cuando una competencia es ganada.\par

            \it Restricciones:
            \refElem{tRequired},
            \refElem{tDefault}$: true$
        }

        \brAttr{competencevsxp}{experiencia de competencia}{tInt}{%
            Valor que indica la cantidad de experiencia que otorgará cuando el
            evento de \refElem{xp-events-settings.competencevsevent} sea emitido.\par

            \it Restricciones:
            \refElem{tRequired},
            \refElem{tDefault}$: 1000$
        }

        \brAttr{preguntadiariaevento}{pregunta diaria}{tBoolean}{%
            Valor que indica si se brindarán puntos de experiencia al evento que
            es emitido cuando una pregunta diaria es respondida.\par

            \it Restricciones:
            \refElem{tRequired},
            \refElem{tDefault}$: true$
        }

        \brAttr{preguntadiariaxp}{experiencia de pregunta diaria}{tInt}{%
            Valor que indica la cantidad de experiencia que otorgará cuando el
            evento de \refElem{xp-events-settings.preguntadiariaevento} sea emitido.\par

            \it Restricciones:
            \refElem{tRequired},
            \refElem{tDefault}$: 1000$
        }
    \end{cdtEntidad}

    \begin{cdtEntidad}{xp-course}{Curso gamificado}{%
    Un curso gamificado es un \refElem{mdl-course} el cual tiene el
    \refElem{mdl-course.format} de curso {\it ``gamedle''}, a pesar de que las
    opciones de este formato son almacenadas en la tabla
    \refElem{mdl-course-format-options}, se decidió incluir esta entidad en el
    modelo de información ya que reune las características que pertenecen a un curso
    gamificado.}

        \brAttr{id}{Id}{tInt}{%
            . \par

            \it Restricciones:
            \refElem{tPrimaryKey},
            \refElem{tAutoIncrement}.
        }

        \brAttr{format}{formato}{tVarchar}{%
            Para que un curso pueda ser manejado como un curso gamificado este
            atributo debe de tener como valor ``Gamedle''. \par

            \it Restricciones:
        }

        \brAttr{xpEnabled}{experiencia habilitada}{tBoolean}{%
            Para que un curso pueda ser manejado como un curso gamificado este
            atributo debe de tener como valor ``Gamedle''. \par

            \it Restricciones:
        }

        \brAttr{sections}{número de secciones}{tInt}{%
        }

        \brAttr{hiddensections}{secciones ocultas}{tVarchar}{%
        }

        \brAttr{coursedisplay}{aspecto del curso}{tVarchar}{%
        }
    \end{cdtEntidad}

    \begin{cdtEntidad}{xp-course-section}{Sección de curso gamificado}{%
    Esta entidad es la extensión de la sección de moodle, la cual agrega configuraciones de experiencia. }
    \brAttr{id}{id}{tInt}{%
        Es el número que representa el identificador único.\par

        \it Restricciones:
        \refElem{tPrimaryKey},
        \refElem{tAutoIncrement}
    }
        \brAttr{xp}{experiencia seccion}{tInt}{%
            Valor de experiencia que se le dará al usuario una vez complete las actividades de la sección. \par

            \it Restricciones:
            \refElem{tRequired}.
        }
    \end{cdtEntidad}

    \begin{cdtEntidad}{xp-section-reward}{Recompensa de sección}{%
    Esta entidad es el registro de qué recompensas a recibido el usuario al completar una sección.}
    \brAttr{id}{id}{tInt}{%
        Es el número que representa el identificador único.\par

        \it Restricciones:
        \refElem{tPrimaryKey},
        \refElem{tAutoIncrement}
    }
        \brAttr{gmdl-id-usuario}{Usuario}{tInt}{%
            Identificador del usuario al cual se le otorgaron las recompensas. \par

            \it Restricciones:
            \refElem{tForeignKey}.
            \refElem{tRequired}.
        }
        \brAttr{gmdl-id-seccion-curso}{Seccion curso}{tInt}{%
            Identificador de la sección gamificada de la cual se otorgaron las recompensas. \par

            \it Restricciones:
            \refElem{tForeignKey}.
            \refElem{tRequired}.
        }
    \end{cdtEntidad}
