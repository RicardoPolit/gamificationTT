\clearpage
\subsection{Entidades del módulo de competencia}

 En esta sección se presentan las entidades del módulo de competencia,
 las cuales están divididas en 2 submódulos;
 el submódulo de competencia uno contra uno y el submódulo de competencia uno contra sistema.\\

    \subsubsection{Entidades del submódulo competencia uno contra uno}

        El esquema que se utiliza para este submódulo, se presenta en la figura
        \ref{fig:BD-ER-C-1v1}, el cual muestra en color rojo las entidades propias del submódulo,
        en azul las entidades que se utilizan de moodle  y en blanco las entidades generales.

        \addfigure{0.8}{analisis/diagrams/db_comp_1v1}{fig:BD-ER-C-1v1}{Esquema de la base de datos del submódulo 'Competencia 1 contra 1'}


        \begin{cdtEntidad}{comp-1v1-gmcompvs}{Actividad competencia uno contra uno}{
        Esta relación es requerida por moodle para poder guardar los registros de cada instancia de la actividad creadas a lo largo de los cursos. Los requisitos que debe cumplir esta relación son:\\
            1.- El nombre de la tabla debe ser igual al nombre del plugin.\\
            2.- Se deben contar con los atributos \{id, course, name, intro, introformat, timemodified.\}}

            \brAttr{id}{id}{tInt}{%
                Es el número que representa el identificador único.\par

                \it Restricciones:
                \refElem{tPrimaryKey},
                \refElem{tAutoIncrement}
            }

            \brAttr{course}{Curso}{tInt}{%
                Identificador del curso donde se creó la instancia.\par

                \it Restricciones:
                \refElem{tForeignKey},
                \refElem{tRequired}.
            }

            \brAttr{name}{Nombre}{tVarchar}{%
                El nombre que lleva la instancia de la actividad.\par

                \it Restricciones:
                \refElem{tLength} $MAX(255)$,
                \refElem{tRequired}.
            }

            \brAttr{intro}{Introducción}{tText}{%
                La introducción de la instancia de la actividad.\par

                \it Restricciones:
                \refElem{tRequired}.
            }
            \brAttr{introformat}{Formato de la introducción}{tInt}{%
                Entero que especifica en qué formato ha sido escrita la introducción del curso.\par

                \it Restricciones:
                \refElem{tPositive},
                \refElem{tRequired}.
            }


            \brAttr{timemodified}{Tiempo de la última modificación}{tTime}{%
                Fecha en la que se actualizó la instancia de la actividad.
                \it Restricciones:
                \refElem{tRequired}.

            }
            \brAttr{mdl-question-categories-id}{banco de preguntas}{tInt}{%
                Identificador del banco de preguntas del cual se obtendrán las preguntas a mostrar en la instancia de la actividad.\par

                \it Restricciones:
                \refElem{tForeignKey},
                \refElem{tRequired}.
            }

            \brAttr{completionnumwon}{Victorias para finalizar}{tInt}{%
                Número de victorias necesarias para que un usuario complete la instancia de la actividad.\par

                \it Restricciones:
                \refElem{tPositive},
                \refElem{tRequired}.
            }
            \brAttr{apuestas-activas}{Apuestas activas}{tBoolean}{%
                Indicador para saber si en esa instancia de la actividad se pueden llevar acabo apuestas al competir o no.\par

                \it Restricciones:
                \refElem{tRequired}.
            }

        \end{cdtEntidad}
        \schemeName{gmcompvs}


        \begin{cdtEntidad}{comp-1v1-gmdl-partida}{Partida}{
            Esta relación contiene las partidas que se llevan acabo por cada instancia.}

            \brAttr{id}{id}{tInt}{%
                Es el número que representa el identificador único.\par

                \it Restricciones:
                \refElem{tPrimaryKey},
                \refElem{tAutoIncrement}
            }

            \brAttr{gmdl-comp-vs-id}{Instancia de la actividad}{tInt}{%
                Identificador de la instancia en la que se lleva acabo la partida.\par

                \it Restricciones:
                \refElem{tForeignKey},
                \refElem{tRequired}.
            }

        \end{cdtEntidad}
        \schemeName{gmdl\_partida}


        \begin{cdtEntidad}{comp-1v1-gmdl-participacion}{Participación}{
            Esta relación contiene los usuarios que participaron en una partida.}

            \brAttr{id}{id}{tInt}{%
                Es el número que representa el identificador único.\par

                \it Restricciones:
                \refElem{tPrimaryKey},
                \refElem{tAutoIncrement}
            }

            \brAttr{gmdl-usuario-id}{Usuario gamificado}{tInt}{%
                Identificador del usuario que realizó la participación.

                \it Restricciones:
                \refElem{tForeignKey},
                \refElem{tRequired}.
            }

            \brAttr{gmdl-partida-id}{Partida}{tInt}{%
                Identificador de la partida en la que se participó.

                \it Restricciones:
                \refElem{tForeignKey},
                \refElem{tRequired}.
            }

            \brAttr{fecha-inicio}{Fecha inicio}{tTime}{%
                Fecha en la que se empezó la participación.

            }

            \brAttr{fecha-fin}{Fecha fin}{tTime}{%
                Fecha en la que se terminó la participación.

            }

            \brAttr{puntuacion}{Puntuación}{tInt}{%
                Puntos obtenidos en la participación.

                \it Restricciones:
                \refElem{tDefault} $: 0$,
                \refElem{tRequired}.
            }
        \end{cdtEntidad}
        \schemeName{gmdl\_participacion}

        \begin{cdtEntidad}{comp-1v1-gmdl-apuesta}{Apuesta}{
            Esta relación contiene los usuarios que participaron en una partida.}

            \brAttr{id}{id}{tInt}{%
                Es el número que representa el identificador único.\par

                \it Restricciones:
                \refElem{tPrimaryKey},
                \refElem{tAutoIncrement}
            }

            \brAttr{gmdl-participacion-id}{Participación}{tInt}{%
                Identificador de la partida dónde se apostó.

                \it Restricciones:
                \refElem{tUniqueKey},
                \refElem{tForeignKey},
                \refElem{tRequired}.
            }

            \brAttr{monedas-plata}{Monedas de plata}{tInt}{%
                Número de monedas de plata apostadas.

                \it Restricciones:
                \refElem{tDefault} $: 0$ ,
                \refElem{tRequired}.
            }


            \brAttr{activa}{Activa}{tBoolean}{%
                Indica si la apuesta sigue activa o no.

                \it Restricciones:
                \refElem{tDefault} $: Falso$,
                \refElem{tRequired}.
            }
        \end{cdtEntidad}
        \schemeName{gmdl\_apuesta}


\clearpage
    \subsubsection{Entidades del submódulo competencia uno contra sistema}

        El esquema que se utiliza para este submódulo, se presenta en la figura
        \ref{fig:BD-ER-C-CPU}, el cual muestra en color rojo las entidades propias del submódulo,
        en azul las entidades que se utilizan de moodle  y en blanco las entidades generales.

        \addfigure{0.8}{analisis/diagrams/db_comp_cpu}{fig:BD-ER-C-CPU}{Esquema de la base de datos del submódulo 'Competencia 1 contra sistema'}


        \begin{cdtEntidad}{comp-cpu-gmcompcpu}{Actividad competencia uno contra sistema}{
        Esta relación es requerida por moodle para poder guardar los registros de cada instancia de la actividad creadas a lo largo de los cursos. Los requisitos que debe cumplir esta relación son:\\
            1.- El nombre de la tabla debe contener el mismo nombre que el nombre del plugin.\\
            2.- Se deben contar con los atributos \{id, course, name, intro, introformat, timemodified\}}

            \brAttr{id}{id}{tInt}{%
                Es el número que representa el identificador único.\par

                \it Restricciones:
                \refElem{tPrimaryKey},
                \refElem{tAutoIncrement}
            }

            \brAttr{course}{Curso}{tInt}{%
                Identificador del curso donde se creó la instancia.\par

                \it Restricciones:
                \refElem{tForeignKey},
                \refElem{tRequired}.
            }

            \brAttr{name}{Nombre}{tVarchar}{%
                El nombre que lleva la instancia de la actividad.\par

                \it Restricciones:
                \refElem{tLength} $1-255$,
                \refElem{tRequired}.
            }

            \brAttr{intro}{Introducción}{tText}{%
                La introducción de la instancia de la actividad.\par

                \it Restricciones:
                \refElem{tRequired}.
            }
            \brAttr{introformat}{Formato de la introducción}{tInt}{%
                Entero que especifica en qué formato ha sido escrita la introducción del curso.\par

                \it Restricciones:
                \refElem{tPositive},
                \refElem{tRequired}.
            }

            \brAttr{timemodified}{Tiempo de la última modificación}{tTime}{%
                Fecha en la que se actualizó la instancia de la actividad.
                \it Restricciones:
                \refElem{tRequired}.

            }
            \brAttr{mdl-question-categories-id}{banco de preguntas}{tInt}{%
                Identificador del banco de preguntas del cual se obtendrán las preguntas a mostrar en la instancia de la actividad.\par

                \it Restricciones:
                \refElem{tForeignKey},
                \refElem{tRequired}.
            }

            \brAttr{completioncpudiff}{Dificultad mínima}{tInt}{%
                Identificador del nivel de dificultad que debe tener el sistema al que venció el usuario para poder completar la actividad.\par

                \it Restricciones:
                \refElem{tPositive},
                \refElem{tRequired}.
            }

        \end{cdtEntidad}
        \schemeName{gmcompcpu}

        \begin{cdtEntidad}{comp-cpu-dificultad-cpu}{Dificultad del sistema}{
            Esta relación contiene un catálogo de dificultades.}

            \brAttr{id}{id}{tInt}{%
                Es el número que representa el identificador único.\par

                \it Restricciones:
                \refElem{tPrimaryKey},
                \refElem{tAutoIncrement}
            }

            \brAttr{nombre}{Nombre}{tVarchar}{%
                Nombre de la dificultad que puede tener el sistema.

                \it Restricciones:
                \refElem{tRequired}.
            }

        \end{cdtEntidad}
        \schemeName{gmdl\_dificultad\_cpu}


        \begin{cdtEntidad}{comp-cpu-gmdl-intento}{Intento}{
            Esta relación contiene los intentos de un usuario de vencer a un sistema de cierto nivel de dificultad.}

            \brAttr{id}{id}{tInt}{%
                Es el número que representa el identificador único.\par

                \it Restricciones:
                \refElem{tPrimaryKey},
                \refElem{tAutoIncrement}
            }

            \brAttr{gmdl-usuario-id}{Usuario gamificado}{tInt}{%
                Identificador del usuario que realizó el intento.

                \it Restricciones:
                \refElem{tForeignKey},
                \refElem{tRequired}.
            }

            \brAttr{gmdl-dificultad-cpu-id}{Dificultad}{tInt}{%
                Identificador del nivel de dificultad del sistema al que se intentó derrotar.

                \it Restricciones:
                \refElem{tForeignKey},
                \refElem{tRequired}.
            }

            \brAttr{puntiacion-usuario}{Puntos usuario}{tInt}{%
                Puntuación que logró el usuario en ese intento.

            }
            \brAttr{puntiacion-cpu}{Puntos sistema}{tInt}{%
                Puntuación que logró el sistema en ese intento.

            }

            \brAttr{fecha-inicio}{Fecha inicio}{tTime}{%
                Fecha en la que se empezó el intento.
                \it Restricciones:
                \refElem{tRequired}.

            }

            \brAttr{fecha-fin}{Fecha fin}{tTime}{%
                Fecha en la que se terminó el intento.

            }

        \end{cdtEntidad}
        \schemeName{gmdl\_intento}


        \begin{cdtEntidad}{comp-cpu-gmdl-respuesta-cpu}{Respuesta sistema}{
            Esta relación contiene las respuesta que eligió responder el sistema de una cierta pregunta.}

            \brAttr{id}{id}{tInt}{%
                Es el número que representa el identificador único.\par

                \it Restricciones:
                \refElem{tPrimaryKey},
                \refElem{tAutoIncrement}
            }

            \brAttr{gmdl-intento-id}{Intento}{tInt}{%
                Identificador del intento donde se respondió la pregunta.

                \it Restricciones:
                \refElem{tForeignKey},
                \refElem{tRequired}.
            }

            \brAttr{mdl-question-id}{Pregunta}{tInt}{%
                Identificador de la pregunta que respondió el sistema.

                \it Restricciones:
                \refElem{tForeignKey},
                \refElem{tRequired}.
            }

            \brAttr{mdl-question-answers}{Respuesta}{tInt}{%
                Identificador de la respuesta que se eligió.\par
                \it Restricciones:
                \refElem{tForeignKey},
                \refElem{tRequired}.

            }
        \end{cdtEntidad}
        \schemeName{gmdl\_respuesta\-CPU}
