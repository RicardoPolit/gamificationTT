
\section{Relaciones de moodle}

 A continuación se presenta la especificación de las relaciones del esquema de base
 de datos de moodle que son relevantes para el desarrollo de los módulos y submódulos
 de proyecto.

    \begin{cdtEntidad}{mdl-config-plugins}{Configuración de Plugin}{%
    Es una tabla del núcleo de moodle que almacena todas las configuraciones globales
    relacionadas a los plugins instalados, al iniciar moodle las configuraciones de los
    plugins instalados y habilitados se cargan en memoria.}

	    \brAttr{id}{Id}{tInt}{%
	        Es el dígito que representa el identificador único para una configuración
            específica de un plugin.\par

            \it Restricciones:
            \refElem{tPrimaryKey},
            \refElem{tAutoIncrement}.
        }

        \brAttr{plugin}{Plugin}{tVarchar}{%
            Cadena de caracteres del nombre identificador del plugin al cual pertenece
            la configuración.\par

            \it Restricciones:
            \refElem{tRequired},
            \refElem{tRange} (0,100),
            \refElem{tUniqueKey}
        }

        \brAttr{name}{Nombre}{tVarchar}{%
            Cadena de caracteres que representa el nombre de la configuración de un
            plugin en específico.\par

            \it Restricciones:
            \refElem{tUniqueKey},
            \refElem{tRange} (0,100),
            \refElem{tRequired}
        }

        \brAttr{value}{Valor}{tVarchar}{%
            Cadena que almacena el valor de una configuración perteneciente a alguno
            de los plugins instalados.\par

            \it Restricciones:
            \refElem{tRange} (0,4294967295),
            \refElem{tRequired}
        }
	%\cdtEntityRelSection
	%\brRel{\brRelComposition}{Ejemplo}{Escribe el enunciado que denota la composición.}
	%\brRel{\brRelAgregation}{Ejemplo}{Tambien puede ser con agregaciones.}
	%\brRel{\brRelGeneralization}{Ejemplo}{De la misma forma con Generalizaciones.}
    \end{cdtEntidad}\schemeName{config\_plugins}

    \begin{cdtEntidad}{mdl-user}{Usuario de moodle}{%
    Es una tabla del núcleo de moodle que contiene toda la información que se
    almacena de los usuarios en la plataforma, independientemente del rol que
    estos contenga, esta relación contiene más de 53 atributos, sin embargo solo
    se detallan aquellos relevantes.}

	    \brAttr{id}{Id}{tInt}{%
	        Es el dígito que representa el identificador único para cada uno
            de los usuarios en moodle.\par

            \it Restricciones:
            \refElem{tPrimaryKey},
            \refElem{tAutoIncrement}.
        }

    \end{cdtEntidad}\schemeName{user}

    \begin{cdtEntidad}{Plugin}{Plugin}{%
    La forma en que moodle obtiene información acerca de los plugins es analizando
    los archivos internos de cada uno, a pesar de que los plugins no forman parte
    del esquema de base de datos, si forman parte del modelo de información que
    utiliza Moodle.}
    \end{cdtEntidad}
