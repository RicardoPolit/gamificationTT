
\clearpage
\subsection{Entidades del módulo de seguimiento}

 En esta sección se presentan las entidades del módulo de seguimiento,
 las cuales están representadas en la figura  \ref{fig:BD_ER_S_PD},
 el cual muestra en color amarillo las entidades propias del módulo,
 en azul las entidades que se utilizan de moodle  y en blanco las entidades generales.

        \addfigure{0.7}{analisis/diagrams/db_seguimiento}{fig:BD_ER_S_PD}{Esquema de la base de datos del módulo 'Seguimiento'}

        \begin{cdtEntidad}{seg-gmpregdiarias}{Actividad preguntas diarias}{

        Esta entidad es requerida por moodle para poder guardar los registros de cada instancia de la actividad creadas a lo largo de los cursos. Los requisitos que debe cumplir esta entidad son:\\
            1.- El nombre de la tabla debe contener el mismo nombre que el nombre del plugin.\\
            2.- Se deben contar con los atributos \{id, course, name, intro, introformat, timemodified.\}
            }

            \brAttr{id}{id}{tInt}{%
                Es el número que representa el identificador único.\par

                \it Restricciones:
                \refElem{tPrimaryKey},
                \refElem{tAutoIncrement}
            }

            \brAttr{course}{Curso}{tInt}{%
                Identificador del curso donde se creó la instancia.\par

                \it Restricciones:
                \refElem{tForeignKey}
                \refElem{tRequired}.
            }

            \brAttr{name}{Nombre}{tVarchar}{%
                El nombre que lleva la instancia de la actividad.\par

                \it Restricciones:
                \refElem{tLength} $1-255$,
                \refElem{tRequired}.
            }

            \brAttr{intro}{Introducción}{tText}{%
                La introducción de la instancia de la actividad.\par

                \it Restricciones:
                \refElem{tRequired}.
            }
            \brAttr{introformat}{Formato de la introducción}{tInt}{%
                Entero que especifica en qué formato ha sido escrita la introducción del curso.\par

                \it Restricciones:
                \refElem{tNatural}
                \refElem{tRequired}.
            }


            \brAttr{timemodified}{Tiempo de la última modificación}{tTime}{%
                Fecha en la que se actualizó la instancia de la actividad.
                \it Restricciones:
                \refElem{tRequired}.

            }
            \brAttr{mdl-question-categories-id}{banco de preguntas}{tInt}{%
                Identificador del banco de preguntas del cual se obtendrán las preguntas a mostrar en la instancia de la actividad.\par

                \it Restricciones:
                \refElem{tForeignKey},
                \refElem{tRequired}.
            }

        \end{cdtEntidad}
        \schemeName{gmpregdiarias}


        \begin{cdtEntidad}{seg-gmdl-intento-diario}{Intento diario}{
            Esta entidad contiene los intentos diarios que se llevan a cabo por cada instancia.}

            \brAttr{id}{id}{tInt}{%
                Es el número que representa el identificador único.\par

                \it Restricciones:
                \refElem{tPrimaryKey},
                \refElem{tAutoIncrement}
            }

            \brAttr{gmdl-usuario-id}{Usuario gamificado}{tInt}{%
                Identificador del usuario que respondió el intento diario.

                \it Restricciones:
                \refElem{tForeignKey}
                \refElem{tRequired}.
            }
            \brAttr{gmdl-preg-diarias-id}{Instancia de la actividad}{tInt}{%
                Identificador de la instancia en la que se llevó acabo el intento.\par

                \it Restricciones:
                \refElem{tForeignKey}
                \refElem{tRequired}.
            }

            \brAttr{fecha}{Fecha}{tTime}{%
                Fecha en la que se realizó el intento.

                \it Restricciones:
                \refElem{tRequired}.
            }

            \brAttr{calificacion}{Calificacion}{tDouble}{%
                Calificación que obtuvo el usuario al responder la pregunta diaria.

                \it Restricciones:
                \refElem{tRange} (-1.0 a 1.0).
                \refElem{tRequired}.
            }

            \brAttr{mdl-question-id}{Pregunta}{tInt}{%
                Identificador de la pregunta que se respondió en el intento.

                \it Restricciones:
                \refElem{tForeignKey}
                \refElem{tRequired}.
            }

        \end{cdtEntidad}
        \schemeName{gmdl\_intento\_diario}
