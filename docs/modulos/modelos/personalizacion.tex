
\clearpage
\subsection{Entidades del módulo de personalización}

 En esta sección se presentan las entidades del módulo de personalización,
 las cuales están representadas en la figura  \ref{fig:BD-ER-P},
 el cual muestra en color verde las entidades propias del módulo,
 en azul las entidades que se utilizan de moodle  y en blanco las entidades generales.

        \addfigure{0.8}{analisis/diagrams/db_tienda}{fig:BD-ER-P}{Esquema de la base de datos del módulo 'Personalización'}

        \begin{cdtEntidad}{tienda-gmdl-tipo-objeto}{Tipo objeto}{
            Esta entidad guarda los tipos de objetos disponibles.
            }

            \brAttr{id}{id}{tInt}{%
                Es el número que representa el identificador único.\par

                \it Restricciones:
                \refElem{tPrimaryKey},
                \refElem{tAutoIncrement}
            }

            \brAttr{nombre}{Nombre}{tVarchar}{%
                Nombre del tipo de objeto.\par

                \it Restricciones:
                \refElem{tLength} $1-50$,
                \refElem{tRequired}.
            }

        \end{cdtEntidad}
        \schemeName{gmdl\_tipo\_objeto}


        \begin{cdtEntidad}{tienda-gmdl-rareza-objeto}{Rareza objeto}{
            Esta entidad guarda la especificación de las rarezas que puede tener un objeto.
            }

            \brAttr{id}{id}{tInt}{%
                Es el número que representa el identificador único.\par

                \it Restricciones:
                \refElem{tPrimaryKey},
                \refElem{tAutoIncrement}
            }

            \brAttr{nombre}{Nombre}{tVarchar}{%
                Nombre del tipo de objeto.\par

                \it Restricciones:
                \refElem{tLength} $1-50$,
                \refElem{tRequired}.
            }

            \brAttr{costo-adquisicion}{Costo adquisición}{tInt}{%
                El precio del objeto en monedas de plata del sistema.\par

                \it Restricciones:
                \refElem{tRequired}.
            }

            \brAttr{probabilidad-obtencion}{Probabilidad de obtención}{tInt}{%
                La probabilidad de obtener el objeto cuando se recibe un premio al azar.\par

                \it Restricciones:
                \refElem{tRange} (0 - 100)\%,
                \refElem{tRequired}.
            }

        \end{cdtEntidad}
        \schemeName{gmdl\_rareza\_objeto}


        \begin{cdtEntidad}{tienda-gmdl-objeto}{Objeto}{
            Esta entidad guarda los elementos de personalización a los que denominamos objetos.
            }

            \brAttr{id}{id}{tInt}{%
                Es el número que representa el identificador único.\par

                \it Restricciones:
                \refElem{tPrimaryKey},
                \refElem{tAutoIncrement}
            }

            \brAttr{nombre}{Nombre}{tVarchar}{%
                Nombre del objeto.\par

                \it Restricciones:
                \refElem{tLength} $1-50$,
                \refElem{tRequired}.
            }

            \brAttr{gmdl-tipo-objeto-id}{Tipo objeto}{tVarchar}{%
                Identificador del tipo al que pertenece el objeto.\par

                \it Restricciones:
                \refElem{tForeignKey},
                \refElem{tRequired}.
            }
            \brAttr{gmdl-rareza-objeto-id}{Rareza objeto}{tVarchar}{%
                Identificador de la rareza a la que pertenece el objeto.\par

                \it Restricciones:
                \refElem{tForeignKey},
                \refElem{tRequired}.
            }
            \brAttr{valor}{Valor}{tText}{%
                Campo de texto que guarda  la forma de desplegar el objeto. Esto puede ser la ruta del archivo, la ruta del directorio o el nombre de las clases de CSS.\par

                \it Restricciones:
                \refElem{tRequired}.
            }
        \end{cdtEntidad}
        \schemeName{gmdl\_objeto}




        \begin{cdtEntidad}{tienda-gmdl-objeto-desbloqueado}{Objeto desbloqueado}{
            Esta entidad guarda los objetos desbloqueados por usuarios y si un dicho objeto está seleccionado o no.
            }

            \brAttr{id}{id}{tInt}{%
                Es el número que representa el identificador único.\par

                \it Restricciones:
                \refElem{tPrimaryKey},
                \refElem{tAutoIncrement}
            }

            \brAttr{gmdl-usuario-id}{Tipo objeto}{tInt}{%
                Identificador del usuario que desbloqueó el objeto.\par

                \it Restricciones:
                \refElem{tForeignKey},
                \refElem{tRequired}.
            }
            \brAttr{gmdl-objeto-id}{Rareza objeto}{tInt}{%
                Identificador del objeto que fue desbloqueado.\par

                \it Restricciones:
                \refElem{tForeignKey},
                \refElem{tRequired}.
            }
            \brAttr{elegido}{Elegido}{tBoolean}{%
                Bandera que indica si el objeto desbloqueado está seleccionado.\par

                \it Restricciones:
                \refElem{tRequired}.
            }
        \end{cdtEntidad}
        \schemeName{gmdl\_objeto\_desbloqueado}
