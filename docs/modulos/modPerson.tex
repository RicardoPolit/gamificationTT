\chapter{Módulo de Personalización}
\label{mod:personalizacion}

    Este módulo define la forma en que se podrá personalizar el perfil de los usuarios estudiantes mediante la ''Herramienta de Personalización'', la ''Definición de items'' de personalización y, la forma en que el profesor podrá añadir una temática a un curso mediante la ''Herramienta de Narrativa''.

\section{Definición de Items}

    Los items de personalización son aquellos que permiten a un estudiante configurar su perfil a su gusto. Se tiene pensado diseñar

    %Para estos items se han diseñado cuatro categorías descritas en el cuadro \ref{tbl:categoriasItemsP}. El color se utiliza para representar la



    %los de personalización y las Loot Boxes las cuales permiten adquirir los ''items''  de personalización.

    %Cada item tiene asignado; un valor de adquisición representado con monedas de oro y una rareza de  las siguientes 4:\\

    %\addtable{|l|l|}{tbl:categoriasItemsP}{
        %{\bf Categoría} & {\bf Identificador} \\\hline
        %Común      & Color blanco  \\\hline
        %Raro       & Color azul    \\\hline
        %Épico      & Color morado  \\\hline
        %Legendario & Color naranja \\\hline
     %}{Categorías de los items de personalización}

\section{Submódulo de Narrativa}

    %Permite al profesor agregar contenido multimedia (Videos, imágenes, audios y texto plano) al curso entre los temas y ejercicios del curso para poder presentar una historia, mundo o idea.

    Le permite al profesor establecer una temática a un curso añadiendo una narrativa, además el profesor puede añadir un alias a la monedas o experiencia para que los estudiantes, en lugar de recibir moneda/experiencia reciban, por ejemplo, diamantes/gemas.

    \begin{quote}
    \begin{description}
    \item[Objetivo] \hfill\\
        Permitirle al profesor crear una historia, un mundo o una idea dentro de su curso que le permita convertir el mismo en más que solo ejercicios listados.

    \item[Principios a los que da Soporte:] \hfill
        \begin{itemize}
            \item 1 \principioI{}
        \end{itemize}
    \end{description}
    \end{quote}



\section{Submódulo de Personalización}

    Permitirle al usuario modificar aspectos como la imagen de perfil, una vez obtenida la funcionalidad para cambiarla; emojis, que son comprados o obtenidos mediante loot boxes y se pueden incluir en el nombre de usuario; el tema de perfil, que permita cambiar la imagen (o el color) de fondo del perfil; y fondo del nombre de usuario, una vez obtenidos distintos fondos de perfil.

    %\begin{itemize}
        %\item Selección de apodos - Es visible a los demás usuarios
        %\item Temas de "plataforma" (Contiene la parte del color y el patron) - Solo visible para el usuario
    %\end{itemize}

    \begin{quote}
    \begin{description}
    \item[Objetivo] \hfill\\
        Permitirle al usuario poder modificar aspectos estéticos de su perfil y su vista de la plataforma con los ''items'' que ha obtenido.

    \item[Principios a los que da soporte:] \hfill
        \begin{itemize}
            \item 4 \principioIV
            \item 6 \principioVI
        \end{itemize}
    \end{description}
    \end{quote}

\clearpage