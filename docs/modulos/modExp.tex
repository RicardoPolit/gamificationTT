
\chapter{Análisis}

 Este apartado contiene el análisis requerido para la elaboración de módulo de experiencia,
 contiene la especificación del alcance de este módulo, la descripción de las funcionalidades
 a desarrollar, la reglas de negocio que rigen el comportamiento del módulo, y por último la
 especificación de los casos de uso a los que brinda soporte.

\section{Esquema de experiencia}

 El esquema de experiencia le proporciona al \refElem{aAdministrador} y a los \refElem[Profesores]%
 {aProfesor} un mecanismo mediante el cual pueden configurar la forma en que se obtienen los puntos
 de experiencia, la cantidad a otorgar, el número de puntos de cada nivel y finalmente la
 visualización del nivel y de los puntos de cada usuario.\\

 \noindent
 Las configuraciones fueron organizadas en dos grupos: las {\it configuraciones a
 nivel plataforma} las cuales definen valores de forma global, y las {\it configuraciones a nivel curso}
 las cuales definen valores para un curso en específico. A continuación se describen cada uno de los
 grupos.\\

    \begin{cdtEntidad}{ConfigXP}{Configuración de Experiencia}{}%
    %Esta entidad es únicamente una representación de las configuraciones correspondientes al módulo de
    %experiencia y no forma parte del modelo de datos ya que las configuraciones de todos los plugins son
    %almacenados en la entidad \refElem{Mdl-configuration}.}

        \brAttr{activated}{activado}{tBoolean}{%
            Valor que indica si el módulo de experiencia está activado o no.\par

            \it Restricciones:
            \refElem{tRequerido}.
            \refElem{tDefault}$:verdadero$
        }

        \brAttr{xpEvents}{eventosExp}{tBoolean}{%
            Valor que indica si se brinda experiencia a la lista de eventos \refElem{EventsXP}.\par

            \it Restricciones:
            \refElem{tRequerido}.
            \refElem{tDefault}$:verdadero$
        }

        \brAttr{increment}{tipo de incremento}{tInt}{%
            Valor que indica que tipo de incremento es usado para los niveles de experiencia,
            El valor 0 indica un incremento lineal y el valor 1 indica un incremento porcentual.\par

            \it Restricciones:
            \refElem{tRequerido},
            \refElem{tRango} [0,1],
            \refElem{tDefault}$:1$
        }

        \brAttr{incrementValue}{valor de incremento}{tDouble}{%
            Valor que indica el factor o valor del incremento para realizar el calculo de la
            experiencia requerida para subir de nivel.\par

            \it Restricciones:
            \refElem{tRequerido},\par
            Si el \refElem{ConfigXP.increment} es 0, \refElem{tNatural}.
            En caso contrario \refElem{tRango} (1,2],
            \refElem{tDefault}$:1.3$
        }

        \brAttr{xpLvl}{experiencia del nivel 1}{tInt}{%
            Valor que indica la cantidad de experiencia que tendrá el primer nivel para ser
            completado.\par

            \it Restricciones:
            \refElem{tRequerido},
            \refElem{tNatural},
            \refElem{tDefault}$:1000$
        }

        \brAttr{xpCourse}{experiencia por curso}{tInt}{%
            Valor que indica la cantidad de experiencia que otorgan los cursos al ser
            completados.\par

            \it Restricciones:
            \refElem{tRequerido},
            \refElem{tDefault}$:1500$
        }

    \end{cdtEntidad}
    \begin{cdtEntidad}{ConfigXPVisual}{Configuración de los Niveles}{}

        \brAttr{title}{Título de niveles}{tVarchar}{%
            Cadena que contiene el título que tienen por defecto todos los niveles.\par

            \it Restricciones:
            \refElem{tLength}$0-10$,
            \refElem{tRequerido}
        }

        \brAttr{description}{Descripción}{tVarchar}{%
            Cadena que contiene la descripción qe tiene por defecto todos los niveles.\par

            \it Restricciones:
            \refElem{tRequerido},
            \refElem{tDefault} ``Casual Levels''
        }

        \brAttr{message}{Mensaje}{tVarchar}{%
            Cadena que contiene el mensaje de felicitaciones que tienen por defecto todos
            los niveles.\par

            \it Restricciones:
            \refElem{tRequerido}
            \refElem{tDefault} ``CONGRATULATIONS''
        }

        \brAttr{colorLvl}{Color de número de nivel}{tColor}{%
            Valor que contiene el código de color con el cual se coloreará el número de nivel.\par

            \it Restricciones:
            \refElem{tRequerido},
            \refElem{tDefault} \#0B619F
        }

        \brAttr{colorBar}{Color de la barra de progreso}{tColor}{%
            Valor que contiene el código de color con el cual se pintará el avance en la barra
            de progreso del nivel.\par

            \it Restricciones:
            \refElem{tRequerido},
            \refElem{tDefault} \#0B619F
        }

        \brAttr{image}{Imagen}{tImage}{%
            Imagen que se desplegará en los niveles cómo el escudo por defecto.\par

            \it Restricciones:
            \refElem{tRequerido}
        }

    \end{cdtEntidad}


\section{Submódulo de niveles}

\section{Funcionalidades}

\section{Reglas de negocio}
\subsection{BR-201} % INPUT: Cursos Igualitarios
\subsection{...}    % INPUT: Incremento lineal en niveles
                    % INPUT: Incremento porcentual en niveles
                    % INPUT: Otorgar experiencia
                    % INPUT: Administración de experiencia en el curso

\section{Casos de uso}

 En este apartado se especifican todos los casos de usos contemplados para el módulo de
 experiencia, para cada caso de uso se especifica su tabla de atributos la cual indica que casos
 de prueba deberán ejecutarse correctamente para corrobarar la completitud del caso de uso.

\subsection{Diagrama de casos de uso}

 En la figura \ref{exp:usecases} se detalla el diagrama de casos de uso correspondiente al módulo
 de experiencia. Los casos de uso de moodle (en color blanco) son modelados como casos de uso
 abstractos, mientras que los casos de uso del módulo de experiencia son diferenciados por el
 color azul, en total el desarrollo de este módulo consiste en 13 casos de uso principales.

    \addfigure{0.9}{modulos/UseCases}{exp:usecases}{%
        Diagrama de casos de uso del módulo de experiencia}

 \noindent
 Debido a que los plugins a desarrollar son elementos opcionales para Moodle, solo se puede
 acceder a los casos de uso del módulo de experiencia a través de puntos de extensión de los
 casos de uso de moodle. Por otra parte los casos de uso que serán documentados en esta sección
 serán los del módulo de experiencia debido a que Moodle proporciona en su página oficial, guías
 e instructivos como documentación de las funcionalidades que brinda.

    
% \ucstEnEdicion     Al terminar una revisión/aprobación con observaciones 
%                    y al inicio del CU.
%
% \ucstEnRevision    Al terminar la edición del CU (version += 0.1).
% \ucstEnAprobacion  Al pasar la revision sin observaciones.
% \ucstAprobado      Al ser aprobado por el usuario (version += 1.0)

\begin{UseCase}[%
Autor/Daniel Ortega,%
Version/0.1,%
Estado/\ucstEnEdicion]%
%
{CU-E02}{Realizar configuraciones del módulo de experiencia}{%
%
 Permite al \refElem{aAdministrador} acceder a las configuraciones del esquema de
 experiencia para consultar y cambiar los aspectos generales del módulo de experiencia, los cuales consisten en habilitar/deshabilitar el esquema de experiencia además de
 habilitar/deshabilitar que los eventos proporcionen experiencia.}

	\UCitem[control]{Revisor}{ Sin asignar }
	\UCitem[control]{Último cambio}{ \today }

 \UCsection{Atributos}

    \UCitem{Actor(es)}{%
        \refElem{aAdministrador}
    }

	\UCitems{Propósito}{%
        \Titem Habilitar/deshabilitar el módulo de experiencia.
        \Titem Habilitar/deshabilitar que los eventos proporcionen experiencia.
	}
	
%% BEGIN-BLOQUE PARA AGREGAR UNA REVISION ------------------------------------->
%% Copiar y descomentar este bloque por cada revision que se realice
%	\UCsection[control]{% Indicar la versión objeto de la revisión.
%		Revisión de la Versión \TODO X.X
%	}
%	\UCitem[control]{Revisó}{% Coloque el nombre de quien realizó la revisión
%		\TODO Especificar
%	}
%	\UCitem[control]{Fecha}{% Coloque la fecha de la revisión
%		% EJEMPLO: 21 de Septiembre de 2019.
%		\TODO Especificar
%	}
%	\UCitem[control]{Resultado}{% Las opciones son: 
%								% Pendiente: se pasa a EnEdicion y se agregan las observaciones
%								% Aprobado: Se pasa a EnAprobacion.
%		\TODO Especificar
%	}
%	\UCitems[control]{Observaciones}{
%		% Agregar las observaciones resultado de la revision o la palabra ``Ninguna''
%		\Titem \TODO Agregar observaciones en cada viñeta, usar el comando \TODO %\TOCHK \DONE.
%	}
%% <------------------------------------------ END-BLOQUE PARA AGREGAR UNA REVISION
	
	\UCitems{Entradas}{\imprimeUC{entrada}}

	\UCitem{Origen}{%
        Mouse
	}

	\UCitems{Salidas}{\imprimeUC{salida}}

	\UCitem{Destino}{%
		\refElem{IU-M01}
	}
	
	\UCitem{Precondiciones}{%
        Que los plugins del módulo de experiencia se encuentren instalados
	}

	\UCitem{Postcondiciones}{%
        Los nuevos valores de las \refElem{xp-general-settings} deben de ser
        almacenados en el sistema.
	}

	\UCitem{Reglas de negocio}{%
		Ninguna
	}

	\UCitems{Errores}{%
        \Titem \UCerr{Err1}{%
        % CAUSA
            Los plugins del módulo de experiencia no se encuentran instalados,}{%
        % EFECTO
            No se presenta en el menu de opciones las opciones para modificar %
            el esquema de experiencia.}
	}

	% \UCitem{Viene de}{% Indicar si el Caso de uso es primario o se extiende de otro. La mayoría se 
					  % extienden de Login.
		% EJEMPLO: \refIdElem{PY-CU1} o Caso de uso primario.
	% 	\TODO Especificar.
	% }	

 \UCsection[design]{Datos de Diseño}

	\UCitems[design]{Casos de Prueba}{%
        \Titem \refElem{CPC-E02}
	}

 \UCsection[admin]{Datos de Administración de Requerimiento}

	\UCitem[admin]{Observaciones}{%
        Ninguna
	}

\end{UseCase}

\clearpage
\subsubsection{Trayectorias del caso de uso}

\begin{UCtrayectoria}%
%
  \includeUC{CU-M01} \refErr{Err1}

  \Actor Presiona la opción {\bf \refElem{tExpSettingsGeneral}} en la categoría
         \refElem{tExpCategoria}. \refTray{A} \label{CU-E02-ir-a-formulario}
  \Sistema Obtiene el valor establecido de los \salida[eventos activados]%
           {xp-general-settings.events} que determina si los eventos brindan
           experiencia o no. \label{CU-E03-formulario}
  \Sistema Obiene el valor \salida{xp-general-settings.activated} que define si el
           módulo de experiencia está activado.
  \Sistema Carga la pantalla \refElem{IU-E02} estableciendo los valores por defecto del
           formulario con los valores obtenidos.

  \Actor Establece el valor \entrada{xp-general-settings.activated} para definir si
         el sistema de experiencia estará habilitado o deshabilitado. \refTray{B}
  \Actor Establece el valor \entrada[eventos activados]{xp-general-settings.events}
         para definir si se les brindará soporte a los eventos dependiendo si estos
         están activados o no.
  \Actor Presiona el botón {\bf Guardar cambios}.
  \Sistema Obtiene el valor \refElem{xp-general-settings.activated} y de
           \refElem[eventos activados]{xp-general-settings.events} y los valores
           establecidos por el \refElem{aActor}.
  \Sistema Carga la pantalla \refElem{IU-M01}.

\end{UCtrayectoria}

\begin{UCtrayectoriaA}{A}{%
El \refElem{aAdministrador} selecciona la categoría \refElem{tExpCategoria}}

    \Sistema Carga la pantalla \refElem{IU-E01}.
    \Sistema Regresa al paso \ref{CU-E02-ir-a-formulario}.
\end{UCtrayectoriaA}

 % Configurar esquema de experiencia
    
% \ucstEnEdicion     Al terminar una revisión/aprobación con observaciones 
%                    y al inicio del CU.
%
% \ucstEnRevision    Al terminar la edición del CU (version += 0.1).
% \ucstEnAprobacion  Al pasar la revision sin observaciones.
% \ucstAprobado      Al ser aprobado por el usuario (version += 1.0)

\begin{UseCase}[%
Autor/Daniel Ortega,%
Version/0.1,%
Estado/\ucstEnEdicion]%
%
{CU-E02-1}{Habilitar/Deshabilitar el módulo de experiencia}{%
%
 Permite al \refElem{aActor} .}

	\UCitem[control]{Revisor}{ Sin asignar }
	\UCitem[control]{Último cambio}{ \today }

 \UCsection{Atributos}

    \UCitem{Actor(es)}{%
        \refElem{aActor}
    }

	\UCitem{Propósito}{%
        ...
	}
	
	\UCitem{Entradas}{\imprimeUC{entrada}}

	\UCitems{Origen}{%
        \Titem Mouse
        \Titem Teclado
	}

	\UCitem{Salidas}{\imprimeUC{salida}}

	\UCitems{Destino}{%
		\Titem \refElem{IU-M02a}
	}
	
	\UCitems{Precondiciones}{%
        \Titem ...
	}

	\UCitem{Postcondiciones}{%
        Ninguna
	}

	\UCitem{Reglas de negocio}{%
		Ninguna
	}

	\UCitems{Errores}{%
        \Titem \UCerr{Err1}{%
        % CAUSA
            ...,}{%
        % EFECTO
            ...}
	}

	% \UCitem{Viene de}{% Indicar si el Caso de uso es primario o se extiende de otro. La mayoría se 
					  % extienden de Login.
		% EJEMPLO: \refIdElem{PY-CU1} o Caso de uso primario.
	% 	\TODO Especificar.
	% }	

 \UCsection[design]{Datos de Diseño}

	\UCitems[design]{Casos de Prueba}{%
        \Titem \refElem{CPC-E0Y}
        \Titem \refElem{CPI-E0Y}
	}

 \UCsection[admin]{Datos de Administración de Requerimiento}

	\UCitem[admin]{Observaciones}{%
        Ninguna
	}

\end{UseCase}

\clearpage
\subsubsection{Trayectorias del caso de uso}

\begin{UCtrayectoria}%
%
 \Actor Presiona el botón \IUMenu en la esquina superior izquierda de la pantalla \refElem{IU-M01}
        para abrir el menu de navegación.

 \Actor Selecciona la opción {\it \IUAdminSitio Administración del sitio}

 \Sistema Carga la pantalla \refElem{IU-M02}

\end{UCtrayectoria}


\subsubsection{Puntos de extensión}

\UCExtensionPoint{Nombre del punto de extensión}{%

    El \refElem{aAdministrador} desea/requiere/necesita ....%
%
    }{En el paso \ref{CU-ET-1x} de la trayectoria principal  ...%
%
    }{\refElem{CU-E2-T}}

 % Configuraciones generales
    
% \ucstEnEdicion     Al terminar una revisión/aprobación con observaciones
%                    y al inicio del CU.
%
% \ucstEnRevision    Al terminar la edición del CU (version += 0.1).
% \ucstEnAprobacion  Al pasar la revision sin observaciones.
% \ucstAprobado      Al ser aprobado por el usuario (version += 1.0)

\begin{UseCase}[%
Autor/Daniel Ortega,%
Version/0.1,%
Estado/\ucstEnEdicion]%
%
{CU-E02-2}{Configurar sistema de experiencia}{%
%
 Permite al \refElem{aAdministrador} establecer y modificar las cantidades de puntos
 de experiencia que brindan los cursos en la plataforma y la forma en que aumenta
 la cantidad de experiencia requerida para pasar de un nivel al siguiente. Los cambios
 sobre estas configuraciones pueden afectar el nivel en que se encuentran los usuarios
 y cambiar la cantidad de experiencia que los cursos brindan.}

	\UCitem[control]{Revisor}{ Sin asignar }
	\UCitem[control]{Último cambio}{ \today }

 \UCsection{Atributos}

    \UCitem{Actor(es)}{%
        \refElem{aAdministrador}
    }

	\UCitems{Propósito}{%
        \Titem Permitir al administrador configurar el sistema de experiencia.
        \Titem Establecer o modificar la cantidad de experiencia que brindan los
               cursos.
        \Titem Establecer la cantidad de experiencia requerida para pasar el primer
               nivel usada como calculo para los demás niveles.
        \Titem Cambiar la forma en cómo se incrementa la cantidad de experiencia
               requerida para avanzar de un nivel a otro.
	}

	\UCitem{Entradas}{\imprimeUC{entrada}}

	\UCitems{Origen}{%
        \Titem Mouse
        \Titem Teclado
        \Titem Sistema (para los datos de los \refElem[usuarios gamificados]{xp-user})
	}

	\UCitem{Salidas}{\imprimeUC{salida}}

	\UCitems{Destino}{%
		\Titem \refElem{IU-E04}
	}

	\UCitems{Precondiciones}{%
        \Titem Que los plugins del módulo de experiencia se encuentren instalados
        \Titem El módulo de experiencia debe estár habilitado en el caso de uso
               \refElem{CU-E02}.
	}

	\UCitem{Postcondiciones}{%
        Los nuevos valores de las \refElem{xp-scheme-settings} deber ser
        estár actualizados para todos los usuarios, además de persistirse en el
        sistema.
	}

	\UCitem{Reglas de negocio}{\imprimeUC{regla}}

	\UCitems{Errores}{%
        \Titem \UCerr{Err1}{%
        % CAUSA
            Los plugins del módulo de experiencia no se encuentran instalados,}{%
        % EFECTO
            no se presentan las opciones en el menú y por lo tanto no se puede
            acceder a las configuraciones}
	}

 \UCsection[design]{Datos de Diseño}

	\UCitems[design]{Casos de Prueba}{%
        \Titem \refElem{CPC-E02-2a}
        \Titem \refElem{CPC-E02-2b}
        \Titem \refElem{CPC-E02-2c}
        \Titem \refElem{CPI-E02-2}
	}

 \UCsection[admin]{Datos de Administración de Requerimiento}

	\UCitem[admin]{Observaciones}{%
        Los cambios en las configuraciones del sistema de experiencia podrían hacer
        que los usuarios aumenten de nivel si la cantidad de experiencia acumulada del
        nivel es mayor a la nueva cantidad de experiencia correspondiente a dicho nivel.
        Se recomienda que los cambios se realicen cuando la cantidad estudiantes que usen el
        sistema sea mínima.}

\end{UseCase}

\subsubsection{Trayectorias del caso de uso}

\begin{UCtrayectoria}%
%
  \includeUC{CU-M01} \refErr{Err1}

  \Actor Presiona la opción {\bf \refElem{tExpSettingsComportamiento}} en la categoría
         \refElem{tExpCategoria}. \refTray{A}
  \Sistema Obtiene el valor de si el módulo de experiencia está \refElem[activado]%
           {xp-general-settings.activated} o no. \refTray{B} \label{CU-E02-2-loading}
  \Sistema Obtiene los valores actuales de la configuración del sistema de experiencia:
           \salida{xp-scheme-settings.increment},
           \salida{xp-scheme-settings.incrementValue},
           \salida{xp-scheme-settings.levelXP} y
           \salida{xp-scheme-settings.courseXP}.
  \Sistema Carga la pantalla \refElem{IU-E04} estableciendo como valores por defecto
           las \refElem{xp-scheme-settings} obtenidas en el anterior paso.
  %\sistema muestra un mensaje informando al \refelem{aadministrador} de que si
  %         modifica la experiencia el nivel 1 o el tipo de incremento se alterarían
  %         la cantidad de experiencia requerida para subir de nivel.

  \Actor Especifica si el \entrada{xp-scheme-settings.increment} en la cantidad de
         experiencia de los niveles será {\it Lineal} o {\it Porcentual}.
  \Actor Ingresa el valor para el \entrada{xp-scheme-settings.incrementValue} con
         base en la regla \regla{BR-E03}.
  \Actor Ingresa los valores para la \entrada{xp-scheme-settings.levelXP} y la
         \entrada{xp-scheme-settings.courseXP}.
  \Actor Presiona la opción {\bf Guardar Cambios}. \refTray{C} \label{CU-E02-2-submit}

  \Sistema Valida que los valores ingresados por el usuario cumplan con las
           restricciones especificadas en el modelo de información.
  \Sistema Verifica que el \refElem{xp-scheme-settings.incrementValue} cumpla
           con la regla \refElem{BR-E03}. \refTray{D}
  \Sistema Actualiza los valores de las \refElem{xp-scheme-settings} con los
           ingresados por el usuario.

  % ACTUALIZACION DE NIVEL DE LOS USUARIOS
  \Sistema Obtiene la lista de los \refElem[usuarios gamificados]{xp-user} y por cada uno
           realiza las siguientes acciones.
  \Sistema Obtiene el \entrada{xp-user.level} y la cantidad de \entrada{xp-user.levelxp}.
  \Sistema Calcula la nueva cantidad de experiencia correspondiente al nivel del usuario con
           base en la regla \regla{BR-E04} o \regla{BR-E05} si el incremento es lineal o
           porcentual respectivamente.
  \Sistema - Si la cantidad de experiencia del usuario es menor a la experiencia correspondiente
           al nivel, entonces se procede al siguiente usuario. \refTray{E} \label{CU-E02-2-Usuarios}

  \Sistema Despliega la pantalla \refElem{IU-E04} con el mensaje de que los datos
           han sido actualizados exitosamente.

\end{UCtrayectoria}

\begin{UCtrayectoriaA}{A}{
El \refElem{aAdministrador} selecciona la categoría \refElem{tExpCategoria}}
  \Sistema Carga la pantalla \refElem{IU-E01}
  \Actor Regresa al paso \ref{CU-E02-2-loading}
\end{UCtrayectoriaA}

\begin{UCtrayectoriaA}{B}{
El módulo de experiencia no se encuentra activado}
  \Sistema Carga la pantalla \refElem{IU-E03a}.
  \Actor Presiona el botón {\bf Activar módulo de experiencia}
  \includeUC{CU-E02} a partir del paso \ref{CU-E02-ir-a-formulario},
                     para activar el módulo de experiencia.

  \Sistema Regresa al inicio de la trayectoria principal.

\end{UCtrayectoriaA}

\begin{UCtrayectoriaA}{C}{
El \refElem{aAdministrador} desea cancelar la modificación en el sistema de
experiencia}

  \Actor Presiona el botón {\bf Cancelar}.
  \Sistema Redirige a la pantalla \refElem{IU-M01}.
\end{UCtrayectoriaA}

\begin{UCtrayectoriaA}{D}{
Alguno de los valores ingresados por el usuario son incorrectos.}
  \Sistema Imprime los mensajes de error abajo de los campos con los valores
           incorrectos.
  \Actor Ingresa nuevamente los valores en los campos marcados como incorrectos.
  \Sistema Regresa al paso \ref{CU-E02-2-submit}.

\end{UCtrayectoriaA}

\begin{UCtrayectoriaA}{E}{%
La cantidad de \refElem{xp-user.levelxp} es mayor a la experiencia del nivel en el
que se encuentra}

  \Sistema Avanza al \refElem{xp-user} al siguiente nivel, usando los
           puntos de experiencia del mismo y establece el sobrante como
           la \refElem{xp-user.levelxp} correspondiente al nuevo nivel.
  \Sistema Repite el paso anterior hasta que la cantidad de experiencia
           del nivel del usuario sea menor que la del nivel.
  \Sistema Regresa al paso \ref{CU-E02-2-Usuarios}

\end{UCtrayectoriaA}
 % Visualización de niveles
    
% \ucstEnEdicion     Al terminar una revisión/aprobación con observaciones
%                    y al inicio del CU.
%
% \ucstEnRevision    Al terminar la edición del CU (version += 0.1).
% \ucstEnAprobacion  Al pasar la revision sin observaciones.
% \ucstAprobado      Al ser aprobado por el usuario (version += 1.0)

\begin{UseCase}[%
Autor/Daniel Ortega,%
Version/0.1,%
Estado/\ucstEnEdicion]%
%
{CU-E02-3}{Configurar los eventos que entregan experiencia}{%
%
 Permite al \refElem{aAdministrador} elegir cuales eventos de los soportados
 brindarán experiencia y cuales no. Además de aquellos eventos que brindan
 experiencia puede especificar la cantidad de experiencia que estos entregan.}

	\UCitem[control]{Revisor}{ Sin asignar }
	\UCitem[control]{Último cambio}{ \today }

 \UCsection{Atributos}

    \UCitem{Actor(es)}{%
        \refElem{aAdministrador}
    }

	\UCitems{Propósito}{%
        \Titem Configurar que eventos proporcionan experiencia.
        \Titem Establecer la cantidad de experiencia de los eventos
               otorgarán.
        \Titem Deshabilitar eventos para que dejen de brindar
               experiencia.
	}

	\UCitem{Entradas}{\imprimeUC{entrada}}

	\UCitems{Origen}{%
        \Titem Mouse
        \Titem Teclado
	}

	\UCitem{Salidas}{\imprimeUC{salida}}

	\UCitems{Destino}{%
		\Titem \refElem{IU-E05}
	}

	\UCitems{Precondiciones}{%
        \Titem Los plugins correspondientes al módulo de experiencia deben
               de estar previamente instalados.
        \Titem El módulo de experiencia debe estar habilitado mediante el caso
               de uso \refElem{CU-E02}.
        \Titem Los eventos con experiencia deben estar habilitados mediante el caso
               de uso \refElem{CU-E02}.
	}

	\UCitem{Postcondiciones}{%
        Los nuevos valores para los eventos habilitados para dar experiencia deben
        de ser actualizados y persistirse en el sistema.
	}

	\UCitem{Reglas de negocio}{Ninguna}

	\UCitems{Errores}{%
        \Titem \UCerr{Err1}{%
        % CAUSA
            Los plugins del módulo de experiencia no se encuentran instalados,}{%
        % EFECTO
            no se presentan las opciones en el menú y por lo tanto no se acceder
            a las configuraciones}
	}

	% \UCitem{Viene de}{% Indicar si el Caso de uso es primario o se extiende de otro. La mayoría se
					  % extienden de Login.
		% EJEMPLO: \refIdElem{PY-CU1} o Caso de uso primario.
	% 	\TODO Especificar.
	% }

 \UCsection[design]{Datos de Diseño}

	\UCitems[design]{Casos de Prueba}{%
        \Titem \refElem{CPC-E02-3}
        \Titem \refElem{CPI-E02-3}
	}

 \UCsection[admin]{Datos de Administración de Requerimiento}

	\UCitem[admin]{Observaciones}{%
        Ninguna
	}

\end{UseCase}

\subsubsection{Trayectorias del caso de uso}

\begin{UCtrayectoria}%
%
  \includeUC{CU-M01} \refErr{Err1}

  \Actor Presiona la opción {\bf\refElem{tExpSettingsEventos}} en la categoría
         \refElem{tExpCategoria}. \refTray{A}

  \Sistema Obtiene el valor de si el módulo de experiencia está \refElem[activado]%
           {xp-general-settings.activated} o no. \refTray{B} \label{CU-E02-3-activated}

  \Sistema Obtiene el valor de si la funcionalidad de que los \refElem[eventos]%
           {xp-general-settings.events} otorgen experiencia está activada o no.
           \refTray{C} \label{CU-E02-3-events}

  \Sistema Obtiene los valores actuales de la configuración de los eventos con experiencia:
           \salida{xp-events-settings.competencecpuevent},
           \salida{xp-events-settings.competencecpuxp},
           \salida{xp-events-settings.competencevsevent},
           \salida{xp-events-settings.competencevsxp},
           \salida{xp-events-settings.preguntadiariaevento} y
           \salida{xp-events-settings.preguntadiariaxp}.

  \Sistema Carga la pantalla \refElem{IU-E05} estableciendo como valores por defecto
           las \refElem{xp-events-settings} obtenidas en el anterior paso.

  \Actor Habilita los eventos que desea que otorguen experiencia:
           \entrada{xp-events-settings.competencecpuevent},
           \entrada{xp-events-settings.competencevsevent} y
           \entrada{xp-events-settings.preguntadiariaevento}.

  \Sistema Habilita los campos para especificar la experiencia correspondientes a aquellos
           eventos que el usuario haya habilitado.

  \Actor Especifica los valores para la experiencia correspondientes a los eventos que haya
         habilitado, en los campos:
           \entrada{xp-events-settings.competencecpuxp},
           \entrada{xp-events-settings.competencevsxp} y
           \entrada{xp-events-settings.preguntadiariaxp}.

  \Actor Presiona el botón {\bf Guardar Cambios} \refTray{D} \label{CU-E02-3-submit}

  \Sistema Valida que los datos ingresados por el usuario cumplan con las restricciones
           de acuerdo con el modelo de información. \refTray{E}
  \Sistema Actualiza los datos de la configuración de eventos en el sistema
  \Sistema Muestra la pantalla \refElem{IU-E05} con el mensaje de que los datos han sido
           actualizados correctamente.

\end{UCtrayectoria}


\begin{UCtrayectoriaA}{A}{%
El \refElem{aAdministrador} selecciona la categoría \refElem{tExpCategoria}.
}
  \Sistema Carga a pantalla \refElem{IU-E01}.
  \Actor Regresa a paso \ref{CU-E02-3-activated}
\end{UCtrayectoriaA}

\begin{UCtrayectoriaA}{B}{%
El módulo de experiencia no se encuentra activado.
}
  \Sistema Carga la pantalla \refElem{IU-E03a}.
  \Actor Presiona el botón {\bf Activar módulo de experiencia}
  \includeUC{CU-E02} a partir del paso \ref{CU-E02-ir-a-formulario},
                     para activar el módulo de experiencia.

  \Sistema Regresa al inicio de la trayectoria principal.
\end{UCtrayectoriaA}

\begin{UCtrayectoriaA}{C}{%
La opción de que los eventos brinden experiencia no se encuentra activada.
}
  \Sistema Carga la pantalla \refElem{IU-E05a}.
  \Actor Presiona el botón {\bf Habilitar Eventos}
  \includeUC{CU-E02} a partir del paso \ref{CU-E02-ir-a-formulario},
                     para activar el módulo de experiencia.

  \Sistema Regresa al inicio de la trayectoria principal.
\end{UCtrayectoriaA}

\begin{UCtrayectoriaA}{D}{%
El \refElem{aAdministrador} desea cancelar la modificación de la configuración
de los eventos.
}
  \Actor Presiona el botón {\bf Cancelar}
  \Sistema Redirige a la pantalla \refElem{IU-M01}
\end{UCtrayectoriaA}

\begin{UCtrayectoriaA}{E}{%
Alguno de los campos ingresados por el \refElem{aAdministrador} son incorrectos.
}
  \Sistema Imprime los mensajes de error debajo de los con los valores incorrectos
  \Actor Ingresa nuevamente los valores en los campos marcados como incorrectos.
  \Sistema Regresa al paso \ref{CU-E02-3-submit}
\end{UCtrayectoriaA}
 % Esquema de experiencia
    %
\begin{UseCase}{CU-E2}{Crear curso con experiencia}{%
El profesor desea crear un nuevo curso en moodle con soporte para brindar puntos de experiencia a los alumnos, partiendo de la interfaz \IUref{moodle:nuevoCurso} llena los campos del curso, escoge el formato {\it Gamedle} y habilita la opción de experiencia, finalmente presiona el botón para crear el curso.}

	\UCrow{Versión}{\color{gray} 0.1 (Edición)}
    \UCrow{Autor}{\color{gray}	Daniel Ortega}
    \UCrow{Supervisa}{\color{gray}}
    \UCrow{Actor}{Profesor}
    \UCrow{Propósito}{Que el profesor pueda crear un curso que tenga soporte para brindar puntos de experiencia en las distintas secciones del curso.}
    \UCrow{Entradas}{
		\begin{Titemize}
		    \Titem{ Nombre completo y nombre corto del curso }
		    \Titem{ Datos generales y de descripción del curso }
		    \Titem{ Elección del formato del curso }
		    \Titem{ Numero de secciones }
		    \Titem{ Visibilidad de secciones ocultas }
		    \Titem{ Aspecto del curso }
		    \Titem{ Casilla de experiencia }
		    \Titem{ Conjunto de datos restantes }
		    \Titem{ Botón de confirmación \textit{Guardar y regresar} o \textit{Guardar cambios y mostrar} }
		\end{Titemize}
   	}
    \UCrow{Origen}{ Ratón para las acciones y elecciones, teclado para los campos de texto}
	\UCrow{Salidas}{ \IUref{moodle:} o \IUref{moodle:} } %\begin{Titemize}\Titem{Ninguna}\end{Titemize}
    \UCrow{Destino}{ Pantalla }
    \UCrow{Precondiciones}{
		\begin{Titemize}
	        \Titem{ Contar con los permisos necesarios para crear cursos }
	        \Titem{ Tener instalado el plugin ''Gamedle Level'' }
	        \Titem{ Tener instalado el plugin ''Gamedle Format'' }
	        \Titem{ Que el actor haya elegido el formato de curso {\it Gamedle} en el caso de uso {\it Crear curso} }
		\end{Titemize}
    }
    \UCrow{Postcondiciones}{%
        \begin{Titemize}
            \Titem{ Se crea un curso con soporte para brindar experiencia. }
            \Titem{ Las secciones del curso tienen experiencia predeterminada }
            \Titem{ El curso y las secciones se muestran de acuerdo a las configuraciones realizadas }
        \end{Titemize}
    }
	\UCrow{Errores}{ No se encuentra la opción ''Gamedle Format'' en los formatos del curso, debido a que los plugins no han sido instalados }
    \UCrow{Observaciones}{  }
\end{UseCase}
\clearpage

%\textbullet{Trayectorias}

\begin{UCtrayectoria}{Principal}
    \moodle Muestra la interfaz \IUref{moodle:nuevoCurso}. \UCnote{CU: Crear curso}
    \actor Especifica el ''nombre completo'' y ''nombre corto'' además de la ''descripción'' y los ''datos generales'' del curso.\\
    
    \setcounter{enumi}{0}
    \actor Selecciona el {\it formato del curso} {\bf Curso Gamedle}. \IUref{exp:format} \UCnote{CU: Crear curso con experiencia}
    \sistema Carga los nuevos datos para el formulario del formato: Curso Gamedle.
    \actor Selecciona el ''número de secciones'' que tendrá por defecto el curso.
    \actor Selecciona la ''visibilidad'' de forma colapsada u no visible de las secciones ocultas.
    \actor Especifica si el ''aspecto del curso'' es mostrar una sección por página o mostrar todas.
    \actor Habilita la ''casilla de experiencia''. \UCnote{\bf Trayectoria A}
    \actor Especifica el conjunto de datos restantes para la configuración del curso.
    \actor Presiona botón \fbox{Guardar y regresar} \UCnote{\bf Trayectoria B} \UCnote{\bf Trayectoria C} %\fbox{Guardar cambios}.
    \sistema Crea el curso y las secciones del mismo.
    \sistema Obtiene del esquema de experiencia la cantidad de experiencia para los cursos.
    \sistema Divide la cantidad de experiencia del curso entre las secciones creadas.
    \sistema Guarda los valores de experiencia que le corresponden a cada sección.
    \sistema Muestra la pantalla \IUref{moodle:}
        % en caso de que la división no sea entera, la última sección tendrá la cantidad para completar
    \item[- -] - - {\em Fin del caso de uso.}
\end{UCtrayectoria}

%\begin{UCtrayectoria}[Formato distinto a Gamedle.]{Alternativa A}
    %\actor Selecciona un formato de curso distinto a ''Gamedle Format''
    %\actor Especifica el conjunto de datos restantes para la configuración del curso.
    %\actor Presiona botón \fbox{Guardar y regresar} o \fbox{Guardar cambios y mostrar}.
    %\item[- -] - - {\em Fin del caso de uso}
%\end{UCtrayectoria}

\begin{UCtrayectoria}[Formato Gamedle sin experiencia]{Alternativa A}
    \actor Deshabilita la ''casilla de experiencia''
    \actor Especifica el conjunto de datos restantes para la configuración del curso.
    \item[- -] - - {\em Fin del caso de uso}
\end{UCtrayectoria}

\begin{UCtrayectoria}[Guardar cambios y mostrar]{Alternativa B}
    \actor Presiona botón \fbox{Guardar cambios y mostrar}.
    \sistema Crea el curso y las secciones del mismo.
    \sistema Obtiene del esquema de experiencia la cantidad de experiencia para los cursos.
    \sistema Divide la cantidad de experiencia del curso entre las secciones creadas.
    \sistema Guarda los valores de experiencia que le corresponden a cada sección.
    \sistema Muestra la pantalla \IUref{moodle:}
    \item[- -] - - {\em Fin del caso de uso}
\end{UCtrayectoria}

\begin{UCtrayectoria}[Cancelar]{Alternativa C}
    \actor Presiona botón \fbox{Cancelar}.
    \item[- -] - - {\em Fin del caso de uso}
\end{UCtrayectoria}

%\UserStory{Crear curso con experiencia}{Como {\bf administrador} me gustaría que la instalación de un ... por lo que ...}

\clearpage\clearpage % Crear curso con experiencia
    %\begin{UseCase}{CU-E9}{Recibir experiencia}{
Cuando un alumno conteste un ejercicio y suba un intento para revisión, el sistema le otorgará la experiencia correspondiente. Si recibe experiencia suficiente, el usuario subirá de nivel.
}
	\UCrow{Versión}{\color{gray} 0.2 (Revisado)}
    \UCrow{Autor}{\color{gray}	David Flores Casanova}
    \UCrow{Supervisa}{\color{gray}	Daniel Isaí Ortega Zúñiga}
    \UCrow{Actor}{Alumno} % Gerente, Instructor
    \UCrow{Propósito}{Otorgarle experiencia al actor por completar una actividad.}
    \UCrow{Entradas}{
        Selección en el botón \#2 ''Enviar todo y terminar'' de la interfaz \hyperref[IUM01]{IU-M01 Ver intento de examen} .\newline
        Selección en el botón \#1 ''Enviar todo y terminar'' de la interfaz \hyperref[IUM02]{IU-M02 Confirmación de envío de intento} .\newline
        Presión de la tecla ''Enter'' o la tecla ''espacio''.
   	}
    \UCrow{Origen}{Ratón y teclado de la computadora }
	\UCrow{Salidas}{
	    \begin{Titemize}
        \Titem{''\hyperref[table:METerminosExperiencia1]{Experiencia otorgada}''.}
        \Titem{''\hyperref[table:METerminosExperiencia1]{Experiencia actual}'' del actor.}
        \Titem{''\hyperref[table:METerminosExperiencia1]{Experiencia del nivel}'' del nivel actual del actor.}
        \Titem{Barra de progreso, mostrada según el \hyperref[table:METerminosExperiencia1]{porcentaje actual}.}
        \Titem{Nombre del nivel (Cadena de caracteres, longitud $\leq$ 60)}
        \Titem{Mensaje de felicitaciones del nivel (Cadena de caracteres, longitud $\leq$ 50)}
        \Titem{Imagen del nivel (Imagen, formato '.png')}
        \Titem{Descripción del nivel (Cadena de caracteres, longitud $\leq$ 200)}
	    \end{Titemize}
    }
    \UCrow{Destino}{Pantalla}
    \UCrow{Precondiciones}{
		\begin{CUTitemize}
	        \CUTitem{El actor está registrado como alumno del curso.}
            \CUTitem{El actor no tiene registrados intentos anteriores.}
			\CUTitem{La actividad del curso está creada.} 
            \CUTitem{El módulo de experiencia está habilitado.}
		\end{CUTitemize}
    }
    \UCrow{Postcondiciones}{
		\begin{CUTitemize}
	        \CUTitem{Al actor se le registra su nueva cantidad de experiencia.}
			\CUTitem{Se actualiza la cantidad de experiencia que ha recibido el actor en ese curso.}
		\end{CUTitemize}
    }
	\UCrow{Errores}{E1: El actor no está registrado como alumno del curso}
    \UCrow{Observaciones}{}
\end{UseCase}

%\textbullet{Trayectorias}

\begin{UCtrayectoria}{Principal}
    \actor se encuentra en la interfaz \hyperref[IUM01]{IU-M01 Ver intento de examen}.
    \actor selecciona el botón \#2 ''Enviar todo y terminar'' .
    \sistema muestra la ventana emergente \hyperref[IUM02]{IU-M02 Confirmación de envío de intento}.
    \actor selecciona el botón \#1 ''Enviar todo y terminar'' ({\it Trayectoria alternativa A})
    \sistema comprueba que el módulo de experiencia esté habilitado ({\it Trayectoria alternativa B}).
    \sistema comprueba que el actor que subió el intento es un alumno del curso ({\it Trayectoria alternativa C}).
    \sistema comprueba que el actor no tenga registrados intentos anteriores ({\it Trayectoria alternativa D}).
    \sistema calcula si la experiencia que se le dará al actor provoca que este ''suba de nivel'' ({\it Trayectoria alternativa E}).
    \sistema carga la interfaz \hyperref[IUM03]{IU-M03 Intento calificado}.
    \item[- -] - - {\em El caso de uso termina.}
\end{UCtrayectoria}

\begin{UCtrayectoria}{alternativa A}
    \item[- -] - - {\em El usuario presionó el botón \#2 \fbox{Cancelar} o  el botón \#3 \fbox{X}.}
    \sistema cierra la interfaz \hyperref[IUM02]{IU-M02 Confirmación de envío de intento}.
    \item[- -] - - {\em El caso de uso termina.}
\end{UCtrayectoria}

\begin{UCtrayectoria}{alternativa B}
    \item[- -] - - {\em El módulo de experiencia no está habilitado.}
    \item[- -] - - {\em El caso de uso termina.}
\end{UCtrayectoria}

\begin{UCtrayectoria}{alternativa C}
    \item[- -] - - {\em El actor no está registrado como alumno del curso.}
    \item[- -] - - {\em El caso de uso termina.}
\end{UCtrayectoria}

\begin{UCtrayectoria}{alternativa D}
    \item[- -] - - {\em El actor ya había hecho intentos anteriores.}
    \item[- -] - - {\em El caso de uso termina.}
\end{UCtrayectoria}

\begin{UCtrayectoria}{alternativa E}
    \item[- -] - - {\em La ''\hyperref[table:METerminosExperiencia1]{experiencia otorgada}'' que recibe el actor es suficiente para ''subir de nivel''.}
    %\item[- -] - - {\em Se extiende al CU-E02.}
    \sistema calcula la ''\hyperref[table:METerminosExperiencia2]{experiencia sobrante}''.
    \sistema incrementa el nivel actual del actor en una unidad.
    \sistema cambia el valor de la ''\hyperref[table:METerminosExperiencia1]{experiencia actual}'' por la de la ''\hyperref[table:METerminosExperiencia1]{experiencia sobrante}''
    \sistema guarda el nivel actual del actor y la ''\hyperref[table:METerminosExperiencia1]{experiencia actual}''.
    \sistema comprueba si el nivel actual del actor existe dentro de un rango de niveles ({\it Trayectoria alternativa F}). 
    \sistema carga la interfaz \hyperref[IUE02]{IU-E02 Subir de nivel} con la información por defecto de niveles.
    \actor presiona la tecla ''enter'' o ''espacio''.
    \sistema cierra la interfaz \hyperref[IUE02]{IU-E02 Subir de nivel}.
    \item[- -] - - {\em Se continua en el paso \#9 de la trayectoria principal.}
\end{UCtrayectoria}


\begin{UCtrayectoria}{alternativa F}
    \item[- -] - - {\em El nivel actual del actor está dentro de un rango de niveles.}
    \sistema carga la interfaz \hyperref[IUE02]{IU-E02 Subir de nivel}  con la información especificada en el rango de niveles.
    \item[- -] - - {\em Se continua en el paso \#7 de la trayectoria alternativa E.}
\end{UCtrayectoria}

\vfill\clearpage\clearpage % Recibir experiencia

% =========================================================
\clearpage
\section{Interfaces del módulo de Experiencia}

    
\subsubsection{IU-M01: Administración del sitio}

 La página de administración del sitio permite al \refElem{aAdministrador} acceder a todas las
 opciones para administrar la apariencia, seguridad, usuarios, permisos, cursos, plugins y demás
 funcionalidades que brinda moodle. La amplia cantidad de configuraciones están agrupadas en nueve
 categorías principales: {\it administración del sitio, usuarios, grupos, calificaciones, plugins,
 apariencia, servidor, reportes y desarrollo}.

    \IUfig{1}{modulos/moodle/IU/AdministracionSitio.png}{IU-M01}{Administración del sitio}

\subsubsection{Elementos relevantes}

    \begin{itemize}
    \item {\bf Pestañas}
        Permiten acceder al conjunto de herramientas y configuraciones que brindan
        cada una de las categorías principales para la administración del sitio.
    \end{itemize}

\subsubsection{Acciones relevantes}

    \begin{itemize}
    \item {\bf Plugins (pestaña)}
        Permite administrar los plugins así como acceder a las configuraciones particulares de
        cada plugin, redirige a la pantalla \refElem{IU-M01a}.
    \end{itemize}

\clearpage
 % Pantalla principal
    
\subsection{IU-M02 Pantalla principal}

 La página de portada, o página principal mostrada en la figura \ref{IU-M02}, es la
 página inicial que ve alguien que llega a un sitio Moodle antes o después de entrar al sitio.
 Típicamente un estudiante verá los cursos, algunos bloques de información, mostrados en un tema.
 En la Barra de navegación y en el menú de navegación (esquina superior izquierda).\\

 \noindent 
 La combinación de las políticas del sitio, autenticación del usuario y configuraciones de la
 portada determinan quién puede llegar a la portada, los elementos que pueden ver y acciones
 que pueden realizar \cite{MoodlePortada}.
    % https://docs.moodle.org/all/es/Portada

    \IUfig{1}{modulos/IUMoodle/Dashboard.png}{IU-M02}{Pantalla Principal de Moodle}

\subsubsection{Elementos relevantes}

    \begin{itemize}
    \item
    {\bf Menú Superior}
        Como su nombre lo indica se encuentra en la parte superior, este elemento se
        encuentra en la mayoría de las pantallas de moodle.

    \item
    {\bf Menú de Navegación}
        Cuando esta visible se encuentra en la parte izquierda de la parte izquierda
        de la mayoría de las pantallas de moodle. Se puede ocultar o mostrar con la
        acción \IUMenu[].

    \item
    {\bf Contenido}
        Tiene todos los demás elementos que conforman el contenido de la pantalla.

    \end{itemize}

\subsubsection{Acciones relevantes}

    \begin{itemize}
    
    \item
    {\bf \IUMenu (Desplegar el menú)}
        Si el menú está oculto, cuando el usuario presione el botón \IUMenu el menú de
        navegación se desplegará.

    \item {\bf \IUMenu (Ocultar Menu)}
        Si el menú está visible, cuando el usuario presione el botón \IUMenu el menú de
        navegación se ocultará.

    \item {\bf \IUAdminSitio Administración del sitio }
        Cuando el menú está visible, el botón de administración del sitio nos permitirá
        navegar a la pantalla \refElem{IU-M03}

    \end{itemize}
 % Administración del Sitio
    %\input{modulos/IUMoodle/IU-M02-1} % Configuraciones generales
    %\input{modulos/IUMoodle/IU-M02-2} % Visualización de niveles
    %\input{modulos/IUMoodle/IU-M02-3} % Esquema de experiencia

\chapter{Diseño}
\section{Diseño de plugins}
\section{Diagrama de componentes}
\section{Diagrama de clases}

\chapter{Pruebas}

\TestCase{CPC-E02}{Configurar esquema de experiencia caso correcto}
\TestCase{CPI-E02}{Configurar esquema de experiencia sin plugins}

\begin{comment}
\section{Submódulos}


%\begin{comment} % TERMINOS Y EJEMPLOS DE EXPERIENCIAi
\subsection{Esquema de Experiencia}

 Es la especificación de los conceptos relacionados con los puntos de experiencia, cuales
 son los tipos de incremento y cómo se usan y cuales restricciones se aplican para la
 implementación de los puntos de experiencia, niveles y la acción subir de nivel.
 % y cuántos niveles hay.\\

\subsubsection{Conceptos de puntos de experiencia}
\noindent Como se especificó en el marco teórico, los puntos de experiencia son una unidad que representa la cantidad de actividades completadas por un usuario, sin embargo, se puede referir en diversos contextos a estos puntos. Es por ello que se utilizarán conceptos definidos por los cuadros  \hyperref[table:METerminosExperiencia1]{6.1} y  \hyperref[table:METerminosExperiencia2]{6.2} .\\

\noindent Para poder explicar mejor los conceptos que se tienen para los puntos de experiencia, se utilizará un ejemplo y cada concepto en los cuadros  \hyperref[table:METerminosExperiencia1]{6.1} y  \hyperref[table:METerminosExperiencia2]{6.2} , tendrá su relación con este ejemplo.\\

\noindent Se tiene un usuario ''\textit{U}'', y dicho usuario está actualmente en el nivel ''5'' con ''1000'' puntos de experiencia, y además los puntos de experiencia relacionados con los niveles son los siguientes:
\begin{itemize}
    \item Estando en el nivel 1 se necesitan 1000 puntos de experiencia para subir al nivel 2.
    \item Estando en el nivel 2 se necesitan 1250 puntos de experiencia para subir al nivel 3.
    \item Estando en el nivel 3 se necesitan 1500 puntos de experiencia para subir al nivel 4.
    \item Estando en el nivel 4 se necesitan 1750 puntos de experiencia para subir al nivel 5.
    \item Estando en el nivel 5 se necesitan 2000 puntos de experiencia para subir al nivel 6.
\end{itemize}

\noindent Agreguemos que existe una actividad ''\textit{A}'' que al ser completada otorga 1050 puntos de experiencia.\\

\noindent Con el ejemplo anterior, podemos proceder a definir nuestros conceptos y relacionarlos con el ejemplo para que quede todo más claro.

\begin{table}[h!]
    \label{table:METerminosExperiencia1}
    \centering
        \begin{tabular}{|m{0.2 \textwidth}|m{0.32 \textwidth}|m { 0.42\textwidth}|}\hline
        \textbf{Nombre} & \textbf{Definición} & \textbf{Ejemplo} \\\hline

        Experiencia actual  &
        Es la cantidad de puntos de experiencia que un usuario ha conseguido mientras tiene asociado un cierto nivel. &
        ''\textit{U}'' tiene una \textbf{experiencial actual} de 1000 puntos de experiencia. Y el nivel al cual está asociado es el nivel 5.
        \\\hline

        Experiencia del nivel &
        Es la cantidad de puntos de experiencia que se requieren para subir de un nivel ''\textit{A}'' a un nivel ''\textit{B}'', sin contemplar la \textbf{experiencia actual} del usuario.&
        Con nuestro ejemplo la \textbf{experiencia del nivel} 1, es 1000 mientras que la del nivel 5 es 2000. \\\hline

        Porcentaje actual &
        Es un número entero positivo con rango del 1 al 100, que se calcula de la forma $ \frac{experiencia\_actual}{experiencia\_del\_nivel} * 100 $ &
        ''\textit{U}'' actualmente está en el nivel 5, por lo tanto el cálculo sería: $ \frac{1000}{2000} * 100 = 50 $
        \\\hline

        Experiencia otorgada &
        Es la cantidad de puntos de experiencia que se otorgan al completar una actividad. &
        La actividad ''\textit{A}'' tiene una \textbf{experiencia otorgada} de 1050.
        \\\hline

        Experiencia acumulada &
        Es la cantidad de puntos de experiencia que un usuario ha conseguido a través de los niveles. &
        ''\textit{U}'' está actualmente en el nivel 5, esto quiere decir que ha pasado por los niveles 1, 2, 3 y 4. Cada uno de estos últimos tiene su \textbf{experiencia del nivel}, por lo tanto, nuestro usuario ''\textit{U}'' a conseguido $ 1000 + 1250 + 1500 + 1750 $ puntos para llegar al nivel 5. Además, ''\textit{U}'' tiene una \textbf{experiencia actual} de 1000 puntos.\newline

        Sumando todo lo anterior $ 1000 + 1250 + 1500 + 1750 + 1000  = 6500 $, por lo tanto el usuario ''\textit{U}'' tiene una \textbf{experiencia acumulada} de 6500 \\\hline

        \end{tabular}
    \caption{Conceptos referentes a los puntos de experiencia (parte 1).}
\end{table}
\clearpage
\begin{table}[h!]
    \label{table:METerminosExperiencia2}
    \centering
        \begin{tabular}{|m{0.2 \textwidth}|m{0.32 \textwidth}|m { 0.42\textwidth}|}\hline
        \textbf{Nombre} & \textbf{Definición} & \textbf{Ejemplo} \\\hline

        Experiencia necesaria &
        Es la cantidad de puntos de experiencia que un usuario necesita para que su \textbf{experiencia actual} alcance la \textbf{experiencia del nivel}. Expresado de forma matemática:
        \begin{center}
            experiencia del nivel\newline
            \underline{ - experiencia actual} \newline
            experiencia necesaria \newline
        \end{center}&

        Para  ''\textit{U}'' su \textbf{experiencia necesaria} sería.

        \begin{center}
              2000\newline
            \underline{ - 1000} \newline
            1000\newline
        \end{center}
        \\\hline

        Experiencia sobrante &
        Es la cantidad de puntos de experiencia que rebasan la \textbf{experiencia del nivel} cuando a un usuario recibe \textbf{experiencia otorgada}. Expresado de forma matemática:
        \begin{center}
            experiencia otorgada\newline
            + experiencia actual\newline
            \underline{ - experiencia del nivel} \newline
            experiencia sobrante \newline
        \end{center}&
        Si el alumno ''U'' realiza la actividad ''A'', recibiría una \textbf{experiencia otorgada} de 1050 , sin embargo, su \textbf{experiencia actual} es de 1000. Por lo tanto, tendría una \textbf{experiencia sobrante} de 50 puntos de experiencia.

        \begin{center}
            1050\newline
            + 1000\newline
            \underline{ - 2000} \newline
            50 \newline
        \end{center}\\\hline
        \end{tabular}
    \caption{Conceptos referentes a los puntos de experiencia (parte 2).}
\end{table}

\noindent A lo largo de este capítulo se utilizaran los conceptos anteriores.\\
%\end{comment}

\subsubsection{Tipos de incremento}

 Un tipo de incremento define cómo se calcula la diferencia entre los valores de
 ''\hyperref[table:METerminosExperiencia1]{experiencia del nivel}'' de un nivel
 \textit{$n_i$} y el siguiente a él \textit{$n_(i+1)$}.\\

 \noindent
 Se da soporte a los siguientes tipos de incremento entre los niveles:

    \begin{quote}
    \begin{description}
        \item[Lineal] Establece que la diferencia es una cantidad fija.\\
        Sea $f(n_i)$ una función la cual indica la ''\hyperref[table:METerminosExperiencia1]{experiencia del nivel}'' de un nivel $n_i$. Y sea $e$ una constante que representa la diferencia de la ''\hyperref[table:METerminosExperiencia1]{experiencia del nivel}'' entre 2 niveles continuos. Entonces.
            $$\forall n_i \in Nivel\ | \left(\ f(n_{i+1}) - f(n_i)\ \right) = e$$

        \item[Porcentual] Establece que la diferencia está regida por la siguiente regla:\\
        Sea $n_i$ un nivel de experiencia,  $f(n_i)$ una función la cual indica la ''\hyperref[table:METerminosExperiencia1]{experiencia del nivel}'' de un nivel $n_i$ y  $c$ una constante tal que $1 \leq c \leq 2$ , entonces
            $$\forall n_i \in Nivel\ |\ (c)f(n_i) = f(n_{i+1}).$$
    \end{description}
    \end{quote}


%\begin{comment}
    \begin{quote}
    \begin{description}
        \item[Lineal] Establece que la diferencia es una cantidad fija.\\
        Sea $f(n_i)$ una función la cual indica la cantidad de experiencia requerida para subir al nivel $n_i$. y sea $e$ una constante que representa el incremento de la cantidad de experiencia requerida entre niveles. Entonces.
            $$\forall n_i \in Nivel\ | \left(\ f(n_{i+1}) - f(n_i)\ \right) = e$$

        \item[Porcentual] Establece que la diferencia entre cantidad necesaria de experiencia para subir del nivel $i$ al nivel $i+1$ está regido por la siguiente regla:\\
        Sea $n_i$ un nivel de experiencia y $n_{i+1}$ el siguiente nivel, y sea $c$ una constante tal que $1 \leq c \leq 2$ , entonces
            $$\forall n_i \in Nivel\ |\ (c)f(n_i) = f(n_{i+1}).$$
    \end{description}
    \end{quote}
%\end{comment}

\subsubsection{Esquema configurable}

 Se quiere que el administrador de la página pueda configurar:
 \begin{quote}
 \begin{itemize}
    \item{La ''\hyperref[table:METerminosExperiencia1]{experiencia del nivel}'' del nivel 1.}
    \item {El tipo de incremento.}
    \item {La cantidad de los puntos de experiencia en el incremento.}
    \item {La ''\hyperref[table:METerminosExperiencia1]{experiencia otorgada}'' que da resolver cualquier actividad.}
 \end{itemize}
 \end{quote}

\subsection{Submódulo de Niveles}

 Presenta a los alumnos su progreso utilizando un sistema de niveles que se van alcanzado
 obteniendo puntos de experiencia. Al alcanzar un nuevo nivel la barra que muestra la
 cantidad de experiencia del nivel se actualizará.
 % y cada vez que se alcanza un nivel, los puntos de experiencia se regresan a cero.

    \begin{quote}
    \begin{description}
    \item[Objetivo] \hfill\\
        Mostrar a los alumnos el nivel actual de experiencia que tienen y el avance que tienen de ese mismo nivel.
        %Mostrar el nivel actual que tienen los alumnos, así como el avance que tienen en ese mismo nivel.

        %Proveer información al alumno que indique la cantidad de tiempo y esfuerzo que le ha dedicado a la plataforma.

    \item[Principios a los que brinda soporte:] \hfill
        \begin{itemize}
            \item 2 \principioII
            \item 6 \principioVI
        \end{itemize}
    \end{description}
    \end{quote}

%\begin{comment}

\subsection{Submódulo de Barra de Progreso}

Muestra al alumno el progreso que lleva en un curso usando un valor de 0\% a 100\% dependiendo de los ejercicios que haya hecho del curso o del tiempo que haya transcurrido.

    \begin{quote}
    \begin{description}
        \item[Objetivo] \hfill\\
            Proveer información al alumno que indique el tiempo y esfuerzo que le ha dedicado a un curso, así como el que le falta por dedicar.

        \item[Principios a los que brinda soporte:] \hfill
        \begin{itemize}
            \item 2 \principioII
        \end{itemize}
    \end{description}
    \end{quote}
%\end{comment}

%\subsection{Comportamiento en Moodle} ESTO ES DISEÑO

%\subsubsection{Plugin}
%\subsubsection{Opciones para el administrador}

\clearpage
%\subsection{Reglas de Uso} % Reglas de Negocio | But there's no business

\section{Interfaces}

\subsection*{IU-E01 Bloque de experiencia}
\label{IUE01}

    Visualización en Moodle del bloque de experiencia.

%    \addfigure{1}{IU/IU_E01_Bloque_Experiencia}{fig:IUE01}{IU-E01: Bloque de experiencia.}

    \noindent {\bf Elementos:}
    \begin{quote}
    \begin{description}
    	\item[Nombre del bloque] Nombre para diferenciar los bloques en las interfaces de Moodle.
    	\item[Imagen del nivel] Imagen del nivel configurada por el administrador.
    	\item[Número del nivel] Número entero positivo que representa el nivel actual del usuario.
    	\item[Barra de progreso] Barra que se llena según el ''\hyperref[table:METerminosExperiencia1]{porcentaje actual}'' del usuario
    	\item[Experiencia actual del nivel] Número entero positivo que representa la ''\hyperref[table:METerminosExperiencia1]{experiencia actual}'' del usuario.
    	\item[Experiencia total del nivel] Número entero positivo que representa la ''\hyperref[table:METerminosExperiencia1]{experiencia del nivel}''.
    	\item[Experiencia acumulada] Número entero positivo que representa la cantidad de ''\hyperref[table:METerminosExperiencia1]{experiencia acumulada}''.
    \end{description}
    \end{quote}
	\clearpage

\subsection*{IU-E02 Subir de nivel}
\label{IUE02}

    Esta interfaz es una ventana emergente que sale siempre que el usuario tenga ''\hyperref[table:METerminosExperiencia1]{experiencia sobrante}'' al ejecutar el \hyperref[CU-E01]{CU-E01 Recibir experiencia}.

%    \addfigure{1}{IU/IU_E02_PopUp_SubirNivel}{fig:IUE02}{IU-E02: Ventana emergente que aparece al subir de nivel.}

    \noindent {\bf Elementos:}
    \begin{quote}
    \begin{description}
    	\item[Mensaje] Mensaje de felicitaciones configurado por el administrador.
    	\item[Imagen del nivel] Imagen del nivel configurada por el administrador.
    	\item[Número del nivel] Número entero positivo que representa el nivel actual del usuario.
    	\item[Nombre] Nombre que reciben los niveles, que es configurado por el administrador.
    	\item[Descripción] Descripción asociada a los niveles, que es configurada por el administrador.
    \end{description}
    \end{quote}
	\clearpage

\subsection*{IU-E03 Configuración del esquema de experiencia}
\label{IUE03}
    Esta interfaz es definida por el archivo \textbf{settings.php}, sin embargo, está es generada por Moodle.\\
    En esta interfaz el administrador puede modificar los aspectos visuales que tienen las interfaces \hyperref[IUE01]{IU-E01 Bloque de experiencia} y \hyperref[IUE02]{IU-E02 Subir de nivel} , y configurar el tipo de incremento, cuanto incremento hay por nivel, la ''\hyperref[table:METerminosExperiencia1]{experiencia del nivel}'' del nivel 1 y la ''\hyperref[table:METerminosExperiencia1]{experiencia otorgada}'' que darán todas las actividades.\\
    Esta interfaz también activa y desactiva el funcionamiento del módulo de experiencia.\\

    %\addfigure{1}{IU/IU_E03_config_parte1}{fig:IUE03_1}{Interfaz donde se  configura el esquema de experiencia parte 1.}


    \noindent {\bf Elementos importantes:}
    \begin{quote}
    \begin{description}
    	\item[Opción \#1] Permite activar y desactivar el módulo de experiencia.
    	\item[Opción \#2] Permite seleccionar entre los 2 tipos de incremento 'Lineal' y 'Porcentual'.
    	\item[Opción \#3] Permite asignar cuanta experiencia habrá de diferencia entre la ''\hyperref[table:METerminosExperiencia1]{experiencia del nivel}'' del nivel \textit{A} y la ''\hyperref[table:METerminosExperiencia1]{experiencia del nivel}'' del nivel \textit{B}.
    	\item[Opción \#4] Permite asignar la ''\hyperref[table:METerminosExperiencia1]{experiencia del nivel}'' 1.
    	\item[Opción \#5] Permite asignar la ''\hyperref[table:METerminosExperiencia1]{experiencia otorgada}'' que entregarán todas las actividades.
    \end{description}
    \end{quote}

\clearpage
    %\addfigure{1}{IU/IU_E03_config_parte2}{fig:IUE03_2}{Interfaz donde se  configura el esquema de experiencia parte 2.}


    \noindent {\bf Elementos importantes:}
    \begin{quote}
    \begin{description}
    	\item[Opción \#6] Permite asignar un nombre por defecto a los niveles.
    	\item[Opción \#7] Permite asignar un mensaje de felicitaciones por defecto al subir de nivel.
    	\item[Opción \#8] Permite asignar una descripción por defecto al subir de nivel.
    	\item[Opción \#9] Permite asignar el color por defecto del número de nivel.
    \end{description}
    \end{quote}


\clearpage

    %\addfigure{1}{IU/IU_E03_config_parte3}{fig:IUE03_3}{Interfaz donde se  configura el esquema de experiencia parte 3.}

    \noindent {\bf Elementos importantes:}
    \begin{quote}
    \begin{description}
    	\item[Opción \#10] Permite asignar el color por defecto de la barra de progreso.
    	\item[Opción \#11] Permite asignar una imagen por defecto a los niveles.
    	\item[Botón \#1 'Guardar cambios'] Con este botón el administrador puede guardar los cambios que haya hecho en las opciones de la interfaz.
    \end{description}
    \end{quote}

\clearpage

\subsection*{IU-M01 Ver intento de examen}
\label{IUM01}

    Esta interfaz es de Moodle, sin embargo, es utilizada para el caso de uso \hyperref[CU-E01]{CU-E01 Recibir experiencia}.\\

    \noindent En esta interfaz los alumnos pueden ver un resumen de su intento para resolver el examen, así como poder reanudar el intento para terminar o corregir, o para enviar las respuestas para su revisión.

    %\addfigure{1}{IU/IU_M01_VerIntento}{fig:IUM01}{IU-M01: Ver intento.}

    \noindent {\bf Elementos importantes:}
    \begin{quote}
    \begin{description}
    	\item[Botón \# 1 'Regresar al intento'] Con este botón el usuario puede seguir contestando el examen.
    	\item[Botón \# 2 'Enviar todo y terminar'] Con este botón el usuario indica que terminó de responder el examen y que quiere enviar las respuestas para su revisión.
    \end{description}
    \end{quote}
	\clearpage

\subsection*{IU-M02 Confirmación de envío de intento}
\label{IUM02}

    Esta interfaz es de Moodle, sin embargo, es utilizada para el caso de uso  \hyperref[CU-E01]{CU-E01 Recibir experiencia}.\\

    \noindent Esta interfaz es una ventana emergente que permite al usuario pensárselo una segunda vez antes de subir sus respuestas para ser calificadas. Esto inclusive por si le da por error.\\

    %\addfigure{1}{IU/IU_M02_PopUp_Confirmacion}{fig:IUM02}{IU-M02: Confirmación de envío de intento.}

    \noindent {\bf Elementos importantes:}
    \begin{quote}
    \begin{description}
    	\item[Botón \#1 'Enviar todo y terminar'] Con este botón el usuario puede reafirmar que quiere mandar sus respuestas para ser calificadas.
    	\item[Botón \#2 'Cancelar'] Con este botón el usuario puede cancelar el subir sus respuestas.
    	\item[Botón \#3 'X'] Este botón tiene el mismo efecto que el Botón \# 2 'Cancelar'.
    \end{description}
    \end{quote}
	\clearpage

\subsection*{IU-M03 Intento calificado}
\label{IUM03}

    Esta interfaz es de Moodle, sin embargo, es utilizada para el caso de uso   \hyperref[CU-E01]{CU-E01 Recibir experiencia}.\\

    \noindent Si un usuario envía su intento para ser calificado
    %y dicho intento puede ser calificado por el sistema,
    se muestra esta interfaz con la calificación de su intento. Algo importante es que la interfaz  \textbf{IU-E01 Bloque de experiencia} está visible, haciendo posible mostrar la actualización de los datos del usuario.\\

    %\addfigure{1}{IU/IU_M03_FinalizarIntento}{fig:IUM03}{IU-M03: Intento calificado.}


	\clearpage

\subsection*{IU-M04 Sección de plugins}
\label{IUM04}

    Esta interfaz es de Moodle, sin embargo, es utilizada para el caso de uso \hyperref[CU-E02]{CU-E02 Configuración esquema de experiencia} .\\

    \noindent Esta interfaz contiene la lista de todos los plugins que tienen configuraciones globales, donde cada uno de sus elementos es un enlace a la página de configuración respectiva a cada plugin. Además en su primera sección tiene las opciones para ver, manejar e instalar plugins.\\

%    \addfigure{1}{IU/IU_M04_SeccionPlugins}{fig:IUM04}{IU-M04: Sección donde están todos los plugins con configuraciones globales.}
	\clearpage


\subsection*{Moodle: IU-M05 Crear Curso}

 El objetivo de esta pantalla (Figura \ref{moodle:nuevoCurso}) es permitirle al profesor crear un nuevo curso, especificando los datos generales del curso, la descripción, apariencia, tamaño de los archivos, el seguimiento de finalización, los grupos, renombre de roles y las marcas vinculadas al curso.

%    \addfigureB{1}{IU/mCrearCurso}{moodle:nuevoCurso}{Moodle IU-M05 Crear curso}

    {\bf Elementos}
    \begin{itemize}
        \item \bf{Datos Generales} es un formulario que contiene el nombre completo y corto del curso (obligatorios), su identificador, categoría, visibilidad, así como las fechas de inicio y término del curso.
        \item \bf{Descripción}
        \item \bf{Formato de curso}
        \item \bf{Apariencia}
        \item \bf{Archivos y subidas}
        \item \bf{Seguimiento de finalización}
        \item \bf{Grupos}
        \item \bf{Renombrar rol}
        \item \bf{Marcas}
    \end{itemize}


\chapter{Diseño}

\subsection{Diagrama de Clases}

    En la figura \ref{fig:classesXP} se muestra el diagrama de clases, los archivos {\it lib, events, settings, version} y los {\it módulos AMD} son representados mediante el uso de clases. Para facilitar la lectura del diagrama se representa a móodle como un paquete completo, el cual lee los distintos archivos y clases que requiere el plugin para funcionar.

%    \addfigure{1}{diagrams/classesExp}{fig:classesXP}{Diagrama de clases del Módulo de Experiencia}
\clearpage

\subsection{Diagrama de componentes}

    En la figura \ref{fig:bloques1} se muestra el diagrama de componentes del Módulo de experiencia que contiene como interactúa el Módulo con la plataforma Moodle.

%    \addfigure{1}{diagrams/bloques1}{fig:bloques1}{Diagrama de componentes del Módulo de Experiencia}

\clearpage
\subsection{Diagramas de Secuencia}
\subsection*{DS-E2: Crear curso con experiencia}

    Para diseñar la forma en que se ejecuta el caso de uso CU-E2, se tomó en consideración el flujo normal de eventos emitidos cuando se crea un curso en moodle. Los eventos emitidos en orden cronológico son {\it course\_created}, {\it course\_section\_created} y {\it enrol\_instance\_created}.\\

    \noindent En la figura \ref{ds:e2} se detalla la interacción entre el core de moodle, los eventos emitidos, y las clases del plugin {\bf Format Gamedle}.

\chapter{Pruebas}
\end{comment}
