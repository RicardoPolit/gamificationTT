\begin{UseCase}{CU-E9}{Recibir experiencia}{
Cuando un alumno conteste un ejercicio y suba un intento para revisión, el sistema le otorgará la experiencia correspondiente. Si recibe experiencia suficiente, el usuario subirá de nivel.
}
	\UCrow{Versión}{\color{gray} 0.2 (Revisado)}
    \UCrow{Autor}{\color{gray}	David Flores Casanova}
    \UCrow{Supervisa}{\color{gray}	Daniel Isaí Ortega Zúñiga}
    \UCrow{Actor}{Alumno} % Gerente, Instructor
    \UCrow{Propósito}{Otorgarle experiencia al actor por completar una actividad.}
    \UCrow{Entradas}{
        Selección en el botón \#2 ''Enviar todo y terminar'' de la interfaz \hyperref[IUM01]{IU-M01 Ver intento de examen} .\newline
        Selección en el botón \#1 ''Enviar todo y terminar'' de la interfaz \hyperref[IUM02]{IU-M02 Confirmación de envío de intento} .\newline
        Presión de la tecla ''Enter'' o la tecla ''espacio''.
   	}
    \UCrow{Origen}{Ratón y teclado de la computadora }
	\UCrow{Salidas}{
	    \begin{Titemize}
        \Titem{''\hyperref[table:METerminosExperiencia1]{Experiencia otorgada}''.}
        \Titem{''\hyperref[table:METerminosExperiencia1]{Experiencia actual}'' del actor.}
        \Titem{''\hyperref[table:METerminosExperiencia1]{Experiencia del nivel}'' del nivel actual del actor.}
        \Titem{Barra de progreso, mostrada según el \hyperref[table:METerminosExperiencia1]{porcentaje actual}.}
        \Titem{Nombre del nivel (Cadena de caracteres, longitud $\leq$ 60)}
        \Titem{Mensaje de felicitaciones del nivel (Cadena de caracteres, longitud $\leq$ 50)}
        \Titem{Imagen del nivel (Imagen, formato '.png')}
        \Titem{Descripción del nivel (Cadena de caracteres, longitud $\leq$ 200)}
	    \end{Titemize}
    }
    \UCrow{Destino}{Pantalla}
    \UCrow{Precondiciones}{
		\begin{CUTitemize}
	        \CUTitem{El actor está registrado como alumno del curso.}
            \CUTitem{El actor no tiene registrados intentos anteriores.}
			\CUTitem{La actividad del curso está creada.} 
            \CUTitem{El módulo de experiencia está habilitado.}
		\end{CUTitemize}
    }
    \UCrow{Postcondiciones}{
		\begin{CUTitemize}
	        \CUTitem{Al actor se le registra su nueva cantidad de experiencia.}
			\CUTitem{Se actualiza la cantidad de experiencia que ha recibido el actor en ese curso.}
		\end{CUTitemize}
    }
	\UCrow{Errores}{E1: El actor no está registrado como alumno del curso}
    \UCrow{Observaciones}{}
\end{UseCase}

%\textbullet{Trayectorias}

\begin{UCtrayectoria}{Principal}
    \actor se encuentra en la interfaz \hyperref[IUM01]{IU-M01 Ver intento de examen}.
    \actor selecciona el botón \#2 ''Enviar todo y terminar'' .
    \sistema muestra la ventana emergente \hyperref[IUM02]{IU-M02 Confirmación de envío de intento}.
    \actor selecciona el botón \#1 ''Enviar todo y terminar'' ({\it Trayectoria alternativa A})
    \sistema comprueba que el módulo de experiencia esté habilitado ({\it Trayectoria alternativa B}).
    \sistema comprueba que el actor que subió el intento es un alumno del curso ({\it Trayectoria alternativa C}).
    \sistema comprueba que el actor no tenga registrados intentos anteriores ({\it Trayectoria alternativa D}).
    \sistema calcula si la experiencia que se le dará al actor provoca que este ''suba de nivel'' ({\it Trayectoria alternativa E}).
    \sistema carga la interfaz \hyperref[IUM03]{IU-M03 Intento calificado}.
    \item[- -] - - {\em El caso de uso termina.}
\end{UCtrayectoria}

\begin{UCtrayectoria}{alternativa A}
    \item[- -] - - {\em El usuario presionó el botón \#2 \fbox{Cancelar} o  el botón \#3 \fbox{X}.}
    \sistema cierra la interfaz \hyperref[IUM02]{IU-M02 Confirmación de envío de intento}.
    \item[- -] - - {\em El caso de uso termina.}
\end{UCtrayectoria}

\begin{UCtrayectoria}{alternativa B}
    \item[- -] - - {\em El módulo de experiencia no está habilitado.}
    \item[- -] - - {\em El caso de uso termina.}
\end{UCtrayectoria}

\begin{UCtrayectoria}{alternativa C}
    \item[- -] - - {\em El actor no está registrado como alumno del curso.}
    \item[- -] - - {\em El caso de uso termina.}
\end{UCtrayectoria}

\begin{UCtrayectoria}{alternativa D}
    \item[- -] - - {\em El actor ya había hecho intentos anteriores.}
    \item[- -] - - {\em El caso de uso termina.}
\end{UCtrayectoria}

\begin{UCtrayectoria}{alternativa E}
    \item[- -] - - {\em La ''\hyperref[table:METerminosExperiencia1]{experiencia otorgada}'' que recibe el actor es suficiente para ''subir de nivel''.}
    %\item[- -] - - {\em Se extiende al CU-E02.}
    \sistema calcula la ''\hyperref[table:METerminosExperiencia2]{experiencia sobrante}''.
    \sistema incrementa el nivel actual del actor en una unidad.
    \sistema cambia el valor de la ''\hyperref[table:METerminosExperiencia1]{experiencia actual}'' por la de la ''\hyperref[table:METerminosExperiencia1]{experiencia sobrante}''
    \sistema guarda el nivel actual del actor y la ''\hyperref[table:METerminosExperiencia1]{experiencia actual}''.
    \sistema comprueba si el nivel actual del actor existe dentro de un rango de niveles ({\it Trayectoria alternativa F}). 
    \sistema carga la interfaz \hyperref[IUE02]{IU-E02 Subir de nivel} con la información por defecto de niveles.
    \actor presiona la tecla ''enter'' o ''espacio''.
    \sistema cierra la interfaz \hyperref[IUE02]{IU-E02 Subir de nivel}.
    \item[- -] - - {\em Se continua en el paso \#9 de la trayectoria principal.}
\end{UCtrayectoria}


\begin{UCtrayectoria}{alternativa F}
    \item[- -] - - {\em El nivel actual del actor está dentro de un rango de niveles.}
    \sistema carga la interfaz \hyperref[IUE02]{IU-E02 Subir de nivel}  con la información especificada en el rango de niveles.
    \item[- -] - - {\em Se continua en el paso \#7 de la trayectoria alternativa E.}
\end{UCtrayectoria}

\vfill\clearpage