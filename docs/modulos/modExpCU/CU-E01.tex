
% \ucstEnEdicion     Al terminar una revisión/aprobación con observaciones 
%                    y al inicio del CU.
%
% \ucstEnRevision    Al terminar la edición del CU (version += 0.1).
% \ucstEnAprobacion  Al pasar la revision sin observaciones.
% \ucstAprobado      Al ser aprobado por el usuario (version += 1.0)

\begin{UseCase}[%
Autor/Daniel Ortega,%
Version/0.1,%
Estado/\ucstEnEdicion]%
%
{CU-E01}{Instalar/actualizar plugin del módulo de experiencia}{%
%
 Permite al \refElem{aAdministrador} extender las funcionalidades que brinda moodle añadiendo el
 sistema de experiencia el cual brinda al \refElem[administrador]{aAdministrador} y los
 \refElem[profesores]{aProfesor} la opción para configurar la cantidad y la forma en que se otogará
 la experiencia a los \refElem[alumnos]{aAlumno}.}

	\UCitem[control]{Revisor}{ Sin asignar }
	\UCitem[control]{Último cambio}{ \today }

 \UCsection{Atributos}

    \UCitem{Actor(es)}{%
        \refElem{aAdministrador}
    }

	\UCitem{Propósito}{%
        Permitir al administrador agregar las distintas funcionalidades que brinda el
        módulo de experiencia.
	}
	
	\UCitem{Entradas}{\imprimeUC{entrada}}

	\UCitems{Origen}{%
        \Titem Mouse
	}

	\UCitem{Salidas}{\imprimeUC{salida}}

	\UCitems{Destino}{%
		\Titem \refElem{IU-M02a} Pantalla de configuración
	}
	
	\UCitems{Precondiciones}{%
        \Titem Para que moodle detecte que se desea instalar o actualizar un plugin,
               es requerido que los archivos del plugin se encuentren en la carpeta
               correcta.
        \Titem Si se trata de una actualización de un plugin la versión del plugin
               debe ser mayor a la indicada por la regla \refElem{BR-M1}.
	}

	\UCitem{Postcondiciones}{%
        La actualización de las \refElem{tExpSettingsGeneral} del módulo de experiencia
        deben persistirse en el sistema.
	}

	\UCitem{Reglas de negocio}{%
		\imprimeUC{regla}
	}

	\UCitems{Errores}{%
        \Titem \UCerr{Err1}{%
        % CAUSA
            La versión del plugin no cumple con la regla \refElem{BR-M1},}{%
        % EFECTO
            Se cancela la instalación y/o actualización}
	}

	% \UCitem{Viene de}{% Indicar si el Caso de uso es primario o se extiende de otro. La mayoría se 
					  % extienden de Login.
		% EJEMPLO: \refIdElem{PY-CU1} o Caso de uso primario.
	% 	\TODO Especificar.
	% }	

 \UCsection[design]{Datos de Diseño}

	\UCitems[design]{Casos de Prueba}{%
        \Titem \refElem{CPC-E01} Instalando plugin
        \Titem \refElem{CPC-E01a} Actualizando plugin
        \Titem \refElem{CPI-E01b} Instalando plugin con versión inválida
        \Titem \refElem{CPI-E01b} Actualizando plugin con versión inválida
	}

 \UCsection[admin]{Datos de Administración de Requerimiento}

	\UCitem[admin]{Observaciones}{%
        El mecanismo que nosotros recomendamos para que el administrador instale plugins
        es incluyendo los plugins en la carpeta correspondiente al plugin en el directorio
        de moodle, por ejemplo para instalar un {\bf bloque} es requerido poner la carpeta
        del plugin dentro de la carpeta {\bf blocks}.
	}

\end{UseCase}

\clearpage
\subsubsection{Trayectorias del caso de uso}

\begin{UCtrayectoria}%
%
 \Actor Realiza la preinstalación del plugin incluyendo la carpeta del archivo en la ubicación
        correspondiente.

 \Moodle Detecta que existe almenos un nuevo plugin listo para ser instalado.

 \Sistema Redirige a la pantalla \refElem{IU-M01}
\end{UCtrayectoria}


\subsubsection{Puntos de extensión}

\UCExtensionPoint{Nombre del punto de extensión}{%

    El \refElem{aAdministrador} desea/requiere/necesita ....%
%
    }{En el paso \ref{CU-ET-1x} de la trayectoria principal  ...%
%
    }{\refElem{CU-E2-T}}

