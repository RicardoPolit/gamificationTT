
% \ucstEnEdicion     Al terminar una revisión/aprobación con observaciones 
%                    y al inicio del CU.
%
% \ucstEnRevision    Al terminar la edición del CU (version += 0.1).
% \ucstEnAprobacion  Al pasar la revision sin observaciones.
% \ucstAprobado      Al ser aprobado por el usuario (version += 1.0)

\begin{UseCase}[%
Autor/Daniel Ortega,%
Version/0.1,%
Estado/\ucstEnEdicion]%
%
{CU-E01}{Instalar plugin del módulo de experiencia}{% TODO; Deberia se Instalar/Actualizar ???
%
 Permite al \refElem{aAdministrador} incluir todas las funcionalidades que brinda el módulo de
 experiencia al moodle que administra mediante la instalación de los plugins correspondientes.
 La conclusión de la trayectoria principal de esta caso de uso es una precondición para que los
 demás casos de uso puedan ejecutarse.}

	\UCitem[control]{Revisor}{ Sin asignar }
	\UCitem[control]{Último cambio}{ \today }

 \UCsection{Atributos}

    \UCitem{Actor(es)}{%
        \refElem{aAdministrador}
    }

	\UCitems{Propósito}{%
        \Titem Permitir al administrador incluir todas las funcionalidades que brinda el módulo de
        experiencia al moodle que administra.

        \Titem Permitir a los usuarios de moodle ver su progreso en la plataforma mediante puntos
        de experiencia.
	}
	
	\UCitem{Entradas}{\imprimeUC{entrada}}

	\UCitems{Origen}{%
        \Titem Mouse
	}

	\UCitem{Salidas}{\imprimeUC{salida}}

	\UCitems{Destino}{%
		\Titem \refElem{IU-M01}
	}
	
	\UCitems{Precondiciones}{%
        \Titem La caperta comprimida que contiene los archivos del plugin
        \Titem El plugin debe cumplir con la regla \refElem{BR-M01} para poder ser
               instalado.
        % \Titem Si se trata de una actualización de un plugin la versión de este debe
               % cumplir con la regla \refElem{BR-M02}.
	}

	\UCitems{Postcondiciones}{%
        \Titem El plugin debe permanecer instalado en moodle.%
        \Titem La actualización de las \refElem{xp-general-settings} del módulo de experiencia
               deben persistirse en el sistema.
        \Titem Los \refElem[usuarios]{mdl-user} registrados en moodle deberán tener la
               asociada la información de un \refElem{xp-user}.
	}

	\UCitem{Reglas de negocio}{\imprimeUC{regla}}

	\UCitems{Errores}{%
        \Titem \UCerr{Err1}{%
        % CAUSA
            ...}{%
        % EFECTO
            ...}
	}

	% \UCitem{Viene de}{% Indicar si el Caso de uso es primario o se extiende de otro. La mayoría se 
					  % extienden de Login.
		% EJEMPLO: \refIdElem{PY-CU1} o Caso de uso primario.
	% 	\TODO Especificar.
	% }	

 \UCsection[design]{Datos de Diseño}

	\UCitems[design]{Casos de Prueba}{%
        \Titem \refElem{CPC-E01} (importante)
        \Titem \refElem{CPI-E01b} 
	}

 \UCsection[admin]{Datos de Administración de Requerimiento}

	\UCitem[admin]{Observaciones}{%
	}

\end{UseCase}

\clearpage
\subsubsection{Trayectorias del caso de uso}

\begin{UCtrayectoria}%
%
 \Actor Realiza la preinstalación del plugin incluyendo la carpeta del archivo en la ubicación
        correspondiente.

 \Moodle Detecta que existe almenos un nuevo plugin listo para ser instalado.

 \Sistema Redirige a la pantalla \refElem{IU-M01}
\end{UCtrayectoria}


\subsubsection{Puntos de extensión}

\UCExtensionPoint{Nombre del punto de extensión}{%

    El \refElem{aAdministrador} desea/requiere/necesita ....%
%
    }{En el paso \ref{CU-ET-1x} de la trayectoria principal  ...%
%
    }{\refElem{CU-E2-T}}

