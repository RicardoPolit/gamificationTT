
% \ucstEnEdicion     Al terminar una revisión/aprobación con observaciones 
%                    y al inicio del CU.
%
% \ucstEnRevision    Al terminar la edición del CU (version += 0.1).
% \ucstEnAprobacion  Al pasar la revision sin observaciones.
% \ucstAprobado      Al ser aprobado por el usuario (version += 1.0)

\begin{UseCase}[%
Autor/Daniel Ortega,%
Version/0.1,%
Estado/\ucstEnEdicion]%
%
{CU-E02-1}{Configurar visualización de niveles}{%
%
 Permite al \refElem{aAdministrador} establecer y modificar los colores, textos e
 imagenes que se muestran en la visualización del nivel actual y en las ventanas
 emergentes de un nuevo nivel alcanzado. Cuando un administrador actualize los valores
 de la configuración los usuarios serán capaces de ver los cambios al renderizarse la
 siguiente página.}

	\UCitem[control]{Revisor}{ Sin asignar }
	\UCitem[control]{Último cambio}{ \today }

 \UCsection{Atributos}

    \UCitem{Actor(es)}{%
        \refElem{aAdministrador}
    }

	\UCitem{Propósito}{%
      \Titem Cambiar las \refElem{xp-visual-settings} del módulo de experiencia.
      \Titem Cambiar el color de la barra de progreso del nivel actual que ve cada
             usuario.
      \Titem Establecer la imagen que se presenta en cada nivel de la plataforma.
      \Titem Establecer el color que tendrá el número
      \Titem Cambiar los mensajes y titulos que se muestran en el popup cada vez que
             se avanza de nivel.
	}
	
	\UCitem{Entradas}{\imprimeUC{entrada}}

	\UCitems{Origen}{%
        \Titem Mouse
        \Titem Teclado
	}

	\UCitem{Salidas}{\imprimeUC{salida}}

	\UCitems{Destino}{%
		\Titem \refElem{IU-M01}
	}
	
	\UCitems{Precondiciones}{%
        Que los plugins del módulo de experiencia se encuentren instalados,
	}

	\UCitem{Postcondiciones}{%
        Los nuevos valores de las \refElem{xp-visual-settings} deben ser actualizados
        para todos los usuarios y además deben persistirse en el sistema.
	}

	\UCitem{Reglas de negocio}{\imprimeUC{regla}}

	\UCitems{Errores}{%
        \Titem \UCerr{Err1}{%
        % CAUSA
            ...,}{%
        % EFECTO
            ...}
	}

 \UCsection[design]{Datos de Diseño}

	\UCitems[design]{Casos de Prueba}{%
        \Titem \refElem{CPC-E02-1}
        \Titem \refElem{CPI-E02-1a}
        \Titem \refElem{CPI-E02-1b}
	}

 \UCsection[admin]{Datos de Administración de Requerimiento}

	\UCitem[admin]{Observaciones}{%
        Ninguna
	}

\end{UseCase}

\subsubsection{Trayectorias del caso de uso}

\begin{UCtrayectoria}%
  \includeUC{CU-M01} \refErr{Err1}
  \Actor Presiona la opción {\bf \refElem{tExpSettingsVisual}} en la categoría
         \refElem{tExpCategoria}.
  \Sistema Obtiene el valor de si el módulo de experiencia está \refElem[activado]%
           {xp-general-settings.activated} o no. \refTray{A}
  \Sistema Obtiene los valores actuales de la configuración:
           \salida{xp-visual-settings.title},
           \salida{xp-visual-settings.description},
           \salida{xp-visual-settings.message},
           \salida{xp-visual-settings.colorLvl},
           \salida{xp-visual-settings.colorBar} e
           \salida{xp-visual-settings.image}.
  \Sistema Vincula las opciones 
  \Sistema Carga la pantalla \refElem{IU-E03} estableciendo cómo valor por defecto
           de los elementos del formulario las \refElem{tExpSettingsVisual}.

  \Actor 

\end{UCtrayectoria}

\begin{UCtrayectoriaA}{A}{El módulo de experiencia no se encuentra activado}
\end{UCtrayectoriaA}

\subsubsection{Puntos de extensión}

\UCExtensionPoint{Nombre del punto de extensión}{%

    El \refElem{aAdministrador} desea/requiere/necesita ....%
%
    }{En el paso \ref{CU-ET-1x} de la trayectoria principal  ...%
%
    }{\refElem{CU-E2-T}}

