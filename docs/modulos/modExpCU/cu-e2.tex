
% \ucstEnEdicion     Al terminar una revisión/aprobación con observaciones 
%                    y al inicio del CU.
%
% \ucstEnRevision    Al terminar la edición del CU (version += 0.1).
% \ucstEnAprobacion  Al pasar la revision sin observaciones.
% \ucstAprobado      Al ser aprobado por el usuario (version += 1.0)

\begin{UseCase}[%
Autor/Daniel Ortega,%
Version/0.0,%
Estado/\ucstEnEdicion]%
%
{CU-E2}{Configurar esquema de experiencia}{%
%
 Permite al \refElem{aAdministrador} configurar el esquema de experiencia, especificando
 la forma en que se obtienen los puntos de experiencia, la cantidad de puntos a otorgar, 
 el número de puntos de cada nivel y finalmente la visualización del nivel y de los puntos
 de cada usuario.}

	\UCitem[control]{Revisor}{ \TODO Especificar }
	\UCitem[control]{Último cambio}{ \today }

 \UCsection{Atributos}

    \UCitem{Actor(es)}{%
        \refElem{aAdministrador}
    }

	\UCitems{Propósito}{%
        \Titem Acceder a las menús de configuración del esquema de experiencia.
		% \Titem Establecer la cantidad de experiencia que brindarán los cursos.
        % \Titem Establecer la forma en que incrementa la cantidad de experiencia de cada nivel con respecto al anterior.
        % \Titem Configurar la visualización de la pantalla emergente al subir de nivel
        % \Titem Configurar la visualización del bloque que muestra el nivel actual del usuario.
        % \Titem Establecer visualización agrupando niveles
	}
	
%% BEGIN-BLOQUE PARA AGREGAR UNA REVISION ------------------------------------->
%% Copiar y descomentar este bloque por cada revision que se realice
%	\UCsection[control]{% Indicar la versión objeto de la revisión.
%		Revisión de la Versión \TODO X.X
%	}
%	\UCitem[control]{Revisó}{% Coloque el nombre de quien realizó la revisión
%		\TODO Especificar
%	}
%	\UCitem[control]{Fecha}{% Coloque la fecha de la revisión
%		% EJEMPLO: 21 de Septiembre de 2019.
%		\TODO Especificar
%	}
%	\UCitem[control]{Resultado}{% Las opciones son: 
%								% Pendiente: se pasa a EnEdicion y se agregan las observaciones
%								% Aprobado: Se pasa a EnAprobacion.
%		\TODO Especificar
%	}
%	\UCitems[control]{Observaciones}{
%		% Agregar las observaciones resultado de la revision o la palabra ``Ninguna''
%		\Titem \TODO Agregar observaciones en cada viñeta, usar el comando \TODO %\TOCHK \DONE.
%	}
%% <------------------------------------------ END-BLOQUE PARA AGREGAR UNA REVISION
	
	%----------------------------------------------
	\UCitems{Entradas}{%
		\imprimeUC{entrada}
	}

	\UCitems{Origen}{%
		\Titem Teclado
        \Titem Mouse
	}

	\UCitems{Salida}{%
		\imprimeUC{salida}
	}

	\UCitems{Destino}{%
		\Titem \refElem{IU-M01}
	}
	
	\UCitems{Precondiciones}{%
        % TODO: ¿Se deben especificar aunque eso sea diseño?
        \Titem Que los plugins del módulo de experiencia se encuentren instalados
	}

	\UCitem{Postcondiciones}{%
        Ninguna
	}

	\UCitem{Reglas de Negocio}{%
		Ninguna
	}

	\UCitems{Errores}{%
        \Titem \UCerr{1}{%
        % CAUSA
            Los plugins del módulo de experiencia no se encuentran instalados,}{%
        % EFECTO
            No se presenta en el menu de opciones las opciones para modificar %
            el esquema de experiencia.}
	}

	% \UCitem{Viene de}{% Indicar si el Caso de uso es primario o se extiende de otro. La mayoría se 
					  % extienden de Login.
		% EJEMPLO: \refIdElem{PY-CU1} o Caso de uso primario.
	% 	\TODO Especificar.
	% }	

 \UCsection[design]{Datos de Diseño}

	\UCitems[design]{Casos de Prueba}{%
        % CASOS CORRECTOS
        \Titem CP-E2-1: Con los plugins requeridos instalados
        % CASOS INCORRECTOS
        \Titem CP-E2-2: Con los plugins requeridos no instalados
	}

 \UCsection[admin]{Datos de Administración de Requerimiento}

	\UCitem[admin]{Observaciones}{%
        Ninguna
	}

\end{UseCase}

\newcommand{\Sistema}{\UCpaso[\UCsist]}
\newcommand{\Actor}{\UCpaso[\UCactor]}

\begin{UCtrayectoria}
    \Sistema ... \label{CU-E2-Menu}
    \Actor ...  \refTray{B} \ref{CU-E2-Menu}
\end{UCtrayectoria}

\begin{comment}
% \end{comment}

%\textbullet{Trayectorias}

\begin{UCtrayectoria}{Principal}
    \actor se encuentra en la interfaz \hyperref[IUM04]{IU-M04 Sección de plugins}.
    \actor Busca la sección de ''Bloques'' .
    \actor selecciona el componente ''Gamedle level''.
    \sistema carga la interfaz \hyperref[IUE03]{IU-E03 Configuración del esquema de experiencia}.
    \actor modifica las opciones que desea ({\it Trayectoria alternativa A}).
    \actor le da al botón \#1 ''Guardar cambios''  ({\it Trayectoria alternativa B}).
    \item[- -] - - {\em El caso de uso termina.}
    
\end{UCtrayectoria}

\begin{UCtrayectoria}{alternativa A}
    \item[- -] - - {\em El actor no quiere cambiar ninguna configuración.}
    \actor selecciona alguna opción en el menú de Moodle o en el directorio de la página
    \sistema carga la interfaz correspondiente a la selección del actor.
    \item[- -] - - {\em El caso de uso termina.}
\end{UCtrayectoria}


\begin{UCtrayectoria}{alternativa B}
    \item[- -] - - {\em Alguna de las entradas no es válida.}
    \sistema despliega el mensaje MS1.
    \sistema despliega el mensaje MS2 en cada una de las entradas que ha encontrado como inválidas.
    
    \item[- -] - - {\em se continúa en el paso \#5 de la trayectoria principal.}
\end{UCtrayectoria}
\end{comment}


\vfill\clearpage
