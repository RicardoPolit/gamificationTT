
% \ucstEnEdicion     Al terminar una revisión/aprobación con observaciones 
%                    y al inicio del CU.
%
% \ucstEnRevision    Al terminar la edición del CU (version += 0.1).
% \ucstEnAprobacion  Al pasar la revision sin observaciones.
% \ucstAprobado      Al ser aprobado por el usuario (version += 1.0)

\begin{UseCase}[%
Autor/Daniel Ortega,%
Version/0.0,%
Estado/\ucstEnEdicion]%
%
{CU-E2}{Configurar esquema de experiencia}{%
%
 Permite al \refElem{aAdministrador} configurar el esquema de experiencia, especificando
 la forma en que se obtienen los puntos de experiencia, la cantidad de puntos a otorgar, 
 el número de puntos de cada nivel y finalmente la visualización del nivel y de los puntos
 de cada usuario.}

	\UCitem[control]{Revisor}{ \TODO Especificar }
	\UCitem[control]{Último cambio}{ \today }

 \UCsection{Atributos}

    \UCitem{Actor(es)}{%
        \refElem{aAdministrador}
    }

	\UCitem{Propósito}{%
        Acceder a las menús de configuración del esquema de experiencia.
		% \Titem Establecer la cantidad de experiencia que brindarán los cursos.
        % \Titem Establecer la forma en que incrementa la cantidad de experiencia de cada nivel con respecto al anterior.
        % \Titem Configurar la visualización de la pantalla emergente al subir de nivel
        % \Titem Configurar la visualización del bloque que muestra el nivel actual del usuario.
        % \Titem Establecer visualización agrupando niveles
	}
	
%% BEGIN-BLOQUE PARA AGREGAR UNA REVISION ------------------------------------->
%% Copiar y descomentar este bloque por cada revision que se realice
%	\UCsection[control]{% Indicar la versión objeto de la revisión.
%		Revisión de la Versión \TODO X.X
%	}
%	\UCitem[control]{Revisó}{% Coloque el nombre de quien realizó la revisión
%		\TODO Especificar
%	}
%	\UCitem[control]{Fecha}{% Coloque la fecha de la revisión
%		% EJEMPLO: 21 de Septiembre de 2019.
%		\TODO Especificar
%	}
%	\UCitem[control]{Resultado}{% Las opciones son: 
%								% Pendiente: se pasa a EnEdicion y se agregan las observaciones
%								% Aprobado: Se pasa a EnAprobacion.
%		\TODO Especificar
%	}
%	\UCitems[control]{Observaciones}{
%		% Agregar las observaciones resultado de la revision o la palabra ``Ninguna''
%		\Titem \TODO Agregar observaciones en cada viñeta, usar el comando \TODO %\TOCHK \DONE.
%	}
%% <------------------------------------------ END-BLOQUE PARA AGREGAR UNA REVISION
	
	\UCitem{Entradas}{%
		%\imprimeUC{entrada}
        Ninguna
	}

	\UCitems{Origen}{%
        \Titem Mouse
	}

	\UCitems{Salidas}{%
		\imprimeUC{salida}
	}

	\UCitems{Destino}{%
		\Titem \refElem{IU-M02a}
	}
	
	\UCitems{Precondiciones}{%
        % TODO: ¿Se deben especificar aunque eso sea diseño?
        \Titem Que los plugins del módulo de experiencia se encuentren instalados
	}

	\UCitem{Postcondiciones}{%
        Ninguna
	}

	\UCitem{Reglas de negocio}{%
		Ninguna
	}

	\UCitems{Errores}{%
        \Titem \UCerr{Err1}{%
        % CAUSA
            Los plugins del módulo de experiencia no se encuentran instalados,}{%
        % EFECTO
            No se presenta en el menu de opciones las opciones para modificar %
            el esquema de experiencia.}
	}

	% \UCitem{Viene de}{% Indicar si el Caso de uso es primario o se extiende de otro. La mayoría se 
					  % extienden de Login.
		% EJEMPLO: \refIdElem{PY-CU1} o Caso de uso primario.
	% 	\TODO Especificar.
	% }	

 \UCsection[design]{Datos de Diseño}

	\UCitems[design]{Casos de Prueba}{%
        % CASOS CORRECTOS
        \Titem \refElem{CP-E2-A}: Con los plugins requeridos instalados
        % CASOS INCORRECTOS
        \Titem \refElem{CP-E2-B}: Con los plugins requeridos no instalados
	}

 \UCsection[admin]{Datos de Administración de Requerimiento}

	\UCitem[admin]{Observaciones}{%
        Ninguna
	}

\end{UseCase}

\clearpage
\subsubsection{Trayectorias del caso de uso}

\begin{UCtrayectoria}%
%
 \Actor Presiona el botón \IUMenu en la esquina superior izquierda de la pantalla \refElem{IU-M01}
        para abrir el menu de navegación.

 \Actor Selecciona la opción {\it \IUAdminSitio Administración del sitio}

 \Sistema Carga la pantalla \refElem{IU-M02}

 \Actor Selecciona ver las opciones para administrar plugins presionando la pestaña
        {\it Plugins}

 \Sistema Obtiene las categorias de las opciones que se muestran en la pantalla.

 \Sistema En la categoría {\it Bloques} agrega la subcategoría \salida{tExpSettingsGeneral} con
          los enlaces a las configuraciones \salida{tExpSettingsVisual} y 
          \salida{tExpSettingsComportamiento} del módulo de experiencia. \refErr{Err1}

 \Sistema Carga la pantalla \refElem{IU-M02a}.

    % \Sistema ... \label{CU-E2-Menu}
    % \Actor ...  \refTray{B} \ref{CU-E2-Menu}
\end{UCtrayectoria}


\subsubsection{Puntos de extensión}

\UCExtensionPoint{Configuraciones generales}{%

    El \refElem{aAdministrador} desea cambiar las configuraciones generales del módulo de 
    experiencia, las cuales incluyen habilitar/deshabilitar el módulo de experiencia o
    la entrega de recompensa a otros eventos de Gamedle.%
%
    }{En el paso \ref{CU-E2-1x} de la trayectoria principal%
%
    }{\refElem{CU-E2-1}}

\UCExtensionPoint{Configuraciones Visuales}{%

    El \refElem{aAdministrador} desea cambiar las configuraciones generales del módulo de 
    experiencia, las cuales incluyen habilitar/deshabilitar el módulo de experiencia o
    la entrega de recompensa a otros eventos de Gamedle.%
%
    }{En el paso \ref{CU-E2-1x} de la trayectoria principal%
%
    }{\refElem{CU-E2-1}}

\UCExtensionPoint{Configuraciones de Comportamiento}{%

    El \refElem{aAdministrador} desea cambiar las configuraciones generales del módulo de 
    experiencia, las cuales incluyen habilitar/deshabilitar el módulo de experiencia o
    la entrega de recompensa a otros eventos de Gamedle.%
%
    }{En el paso \ref{CU-E2-1x} de la trayectoria principal%
%
    }{\refElem{CU-E2-1}}

