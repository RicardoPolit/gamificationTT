
% \ucstEnEdicion     Al terminar una revisión/aprobación con observaciones 
%                    y al inicio del CU.
%
% \ucstEnRevision    Al terminar la edición del CU (version += 0.1).
% \ucstEnAprobacion  Al pasar la revision sin observaciones.
% \ucstAprobado      Al ser aprobado por el usuario (version += 1.0)

\begin{comment}
\begin{UseCase}{CU-E1}{Configurar esquema de experiencia}{%
    Permite al \refElem{aAdministrador} configurar el esquema de experiencia, especificando
    la forma en que se obtienen los puntos de experiencia, la cantidad de puntos a otorgar, 
    el número de puntos de cada nivel y finalmente la visualización del nivel y de los puntos
    de cada usuario.}

    \UCitem{Propósito}{Modificar esquema de experiencia.}
    \UCitem{Entradas}{
        Selección en el componente ''Gamedle Level'' de la interfaz \hyperref[IUM04]{IU-M04 Sección de plugins}.\newline
        El actor puede modificar cualquiera y cuantas quiera de las opciones de la interfaz \hyperref[IUE03]{IU-E03 Configuración del esquema de experiencia}, dichas opciones se enlistan a continuación:\par\vspace{1.5em}
		\begin{Titemize}
		    \Titem{Selección en el parámetro ''Componente activado''.}
	        \Titem{Selección en el tipo de incremento.}
	        \Titem{Selección en el color del nivel.}
	        \Titem{Selección en el color de la barra de progreso.}
            \Titem{La cantidad del incremento (Número real $ \geq $ 0).}
            \Titem{La '' \hyperref[table:METerminosExperiencia1]{experiencia del nivel} '' del nivel 1 (Número entero $ > $ 0).}
            \Titem{La '' \hyperref[table:METerminosExperiencia1]{experiencia otorgada} '' de todas las actividades (Número entero $ > $ 0).}
	        \Titem{Nombre del nivel (Cadena de caracteres, longitud $\leq$ 60)}
	        \Titem{Mensaje de felicitaciones del nivel (Cadena de caracteres, longitud $\leq$ 50)}
	        \Titem{Imagen del nivel (Imagen, formato '.png')}
	        \Titem{Descripción del nivel (Cadena de caracteres, longitud $\leq$ 200)}
		\end{Titemize}
   	}

    \UCitem{Origen}{Ratón y teclado de la computadora}
	\UCitem{Salidas}{ 
	    El sistema carga las configuraciones actuales o las asignadas por defecto de las siguientes opciones:
		\begin{CUTitemize}
		    \CUTitem{Selección en el parámetro ''Componente activado''.}
	        \CUTitem{Selección en el tipo de incremento.}
	        \CUTitem{Selección en el color del nivel.}
	        \CUTitem{Selección en el color de la barra de progreso.}
            \CUTitem{La cantidad del incremento (Número real  $ > $ 0).}
            \CUTitem{La ''\hyperref[table:METerminosExperiencia1]{experiencia del nivel}'' del nivel 1 (Número entero  $ > $ 0).}
            \CUTitem{La '' \hyperref[table:METerminosExperiencia1]{experiencia otorgada} '' de todas las actividades (Número entero $ > $ 0).}
	        \CUTitem{Nombre del nivel (Cadena de caracteres, longitud $\leq$ 60)}
	        \CUTitem{Mensaje de felicitaciones del nivel (Cadena de caracteres, longitud $\leq$ 50)}
	        \CUTitem{Imagen del nivel (Imagen, formato '.png')}
	        \CUTitem{Descripción del nivel (Cadena de caracteres, longitud $\leq$ 200)}
		\end{CUTitemize}
		Mensajes de error:
		\begin{CUTitemize}
		    \CUTitem{MS1: Algunos ajustes no se han cambiado debido a un error.}
		    \CUTitem{MS2: Este valor no es válido.}
		\end{CUTitemize}
		
    }

    \UCitems{Destino}{
        \Titem Pantalla
    }

    \UCitem{Precondiciones}{
		\begin{CUTitemize}
	        \CUTitem{Tener instalado el componente ''Gamedle level''.}
            \CUTitem{Que el actor esté ingresado en una cuenta de administrador.}
		\end{CUTitemize}
    }

    \UCitem{Postcondiciones}{
		\begin{CUTitemize}
			\CUTitem{Se actualiza la configuración global del componente ''Gamedle Level''.}
		\end{CUTitemize}
    }

	\UCitem{Errores}{E1: Formato de entradas no válido.}
    \UCitem{Observaciones}{Las opciones al  tienen un valor por defecto, es por eso que no es necesario que el actor modifique ninguna de las entradas respecto a este punto, debido a que las configuraciones por defecto se cargan en la pantalla. }
\end{UseCase}
\end{comment}


\nopagebreak
\begin{UseCase}[%
Autor/Daniel Ortega,%
Version/0.0,%
Estado/\ucstEnEdicion]%
%
{CU-E2}{Configurar esquema de experiencia}{%
    %
 Permite al \refElem{aAdministrador} configurar el esquema de experiencia, especificando
 la forma en que se obtienen los puntos de experiencia, la cantidad de puntos a otorgar, 
 el número de puntos de cada nivel y finalmente la visualización del nivel y de los puntos
 de cada usuario.}

	\UCitem[control]{Revisor}{ \TODO Especificar }
	\UCitem[control]{Último cambio}{ \today }

 \UCsection{Atributos}

    \UCitem{Actor(es)}{
        \refElem{aAdministrador}
    }

	\UCitems{Propósito}{%
		\Titem Establecer la cantidad de experiencia que brindarán los cursos.
        \Titem Establecer la forma en que incrementa la cantidad de experiencia de cada nivel con respecto al anterior.
        \Titem Configurar la visualización de la pantalla emergente al subir de nivel
        \Titem Configurar la visualización del bloque que muestra el nivel actual del usuario.
        % \Titem Establecer visualización agrupando niveles
	}
	
%% BEGIN-BLOQUE PARA AGREGAR UNA REVISION ------------------------------------->
%% Copiar y descomentar este bloque por cada revision que se realice
%	\UCsection[control]{% Indicar la versión objeto de la revisión.
%		Revisión de la Versión \TODO X.X
%	}
%	\UCitem[control]{Revisó}{% Coloque el nombre de quien realizó la revisión
%		\TODO Especificar
%	}
%	\UCitem[control]{Fecha}{% Coloque la fecha de la revisión
%		% EJEMPLO: 21 de Septiembre de 2019.
%		\TODO Especificar
%	}
%	\UCitem[control]{Resultado}{% Las opciones son: 
%								% Pendiente: se pasa a EnEdicion y se agregan las observaciones
%								% Aprobado: Se pasa a EnAprobacion.
%		\TODO Especificar
%	}
%	\UCitems[control]{Observaciones}{
%		% Agregar las observaciones resultado de la revision o la palabra ``Ninguna''
%		\Titem \TODO Agregar observaciones en cada viñeta, usar el comando \TODO %\TOCHK \DONE.
%	}
%% <------------------------------------------ END-BLOQUE PARA AGREGAR UNA REVISION
	
	%----------------------------------------------
	\UCitems{Entradas}{%
		\imprimeUC{entrada}
	}

	\UCitems{Origen}{%
		\Titem Teclado
        \Titem Mouse
	}

	\UCitems{Salida}{%
		\imprimeUC{salida}
	}

	\UCitems{Destino}{% Indique por donde se darán las salidas de sistema: 
					  %  pantalla, impresora, repositorio, hacia un servidor, un archivo, etc.
		\Titem \TODO Especificar en viñetas.
	}
	
	\UCitems{Precondiciones}{% Indicar las precondiciones que son:
							% ?`Que cosas debieron haber ocurrido para que funcione el CU?
							% ?`Que se requiere previamente para poder ejecutar el CU?
							% Marque cada precondición como:
							%  - Restrictiva: El sistema debe controlar su cumplimiento.
							%  - Contemplada: El sistema debe ofrecer información para que 
							%                 el usuario la haga baler en el proceso.
							% Agregue la referencia a la Regla de negocio o requerimiento
							%   que sustenta la precondición en caso de que aplique.
		% EJEMPLOS: 
		%  - Restrictiva: Debe haber una solicitud en el sistema con estado pendiente.
		%  - Restrictiva: No pueden haber mas de doce productos en una sola venta (ver RN-34).
		%  - Contemplada: El alumno debe mostrar su credencial vigente al momento 
		%                 de realizar la inscripción.
		\Titem\textbf{Restrictiva}: \TODO Especificar.
	}
	\UCitems{Postcondiciones}{% Indicar las postcondiciones:
							  % - ?`Que cambia en el sistema al terminar el CU?
							  % - ?`Que efectos colaterales sucederán?
							  % - ?`Que funciones se habilitan o deshabilitan?
							  % - ?`Como se puede verificar que el CU terminó correctamente?
							  % Indicar en las postcondiciones si es: 
							  % - Efecto directo: implicación directa de la operación.
							  % - Efecto colateral: implicación que se refleja en otros CU o entidades.
							  % - Condición de término: Se puede verificar directamente al 
							  %                         termino del CU.
		% EJEMPLOS:
		%  - Efecto directo: El conductor podrá conducir el vehículo asignado.
		%  - Efecto colateral: El auto asignado al conductor cambiará a estado ``asignado''.
		%  - Efecto colateral: El auto asignado al conductor ya no podrá asignarse a otro conductor 
		%                      a menos que se elimine la asignación.
		%  - Condición de término: En la pantalla de consulta del conductores aparecerá la 
		%                      placa del vehículo asignado.
		\Titem \textbf{Efecto directo}: \TODO Especificar
	}
	\UCitems{Reglas de Negocio}{% Listar las reglas de negocio que aparecen referenciadas
	                            % en la trayectoria de este Caso de Uso.
		\Titem \refIdElem{BR-N0XX}
	}
	\UCitems{Errores}{% Liste los errores que pueden ocurrir en el CU, los errores
	                 % deberán estar marcados en la trayectoria. De cada error se debe especificar:
	                 % - Número de error
	                 % - Descripción de la causa del error
	                 % - La reacción del sistema, incluyendo:
	                 %    - mensaje al usuario
	                 %    - paso de la trayectoria en el que debe continuar.
	    % EJEMPLO: Uno: El actor no proporcionó un dato obligatorio el sistema indica la omisión con el mensaje ``favor de proporcionar los datos faltantes'' y marcando los campos faltantes. Después continúa en el paso 3 de la trayectoria principal.
		\Titem \UCerr{NUMERO}{CAUSA DEL ERROR,}{REACCION DEL SISTEMA}
	}
    \UCitem[admin]{Auditable}{% Indique si la ejecución de la operación deberá ser registrada 
                              % en una bitácora y los datos que se deben guardar:
    						  % - Quien realizó la operación
						      % - A que hora.
						      % - En que consistió la operación.
						      % de no ser auditable poner la palabra ``no''.
		% EJEMPLO: Sí, se debe llevar el registro en la bitácora del usuario, hora, fecha y operación realizada indicando el nombre del cliente afectado y la cantidad.
    	\Titem \TODO Especificar.
    }                 
	\UCitems{Datos sensibles}{% Indicar si alguno de los datos que aparecen en el CU deben 
	                          % tener un tratamiento especial de Aviso de privacidad o Ley 
	                          % de transparencia. Indicar cual y el tratamiento que se le debe dar.
	                          % En caso de no haber colocar la palabra ``Ninguno''.
	    % EJEMPLO: - Mostrar aviso de privacidad para: nombre, domicilio y fecha de nacimiento.
	    %          - Mostrar como dato reservado: Monto de la cotización y nombre del proveedor.
		\Titem \TODO Especificar.
	}
	\UCitem{Viene de}{% Indicar si el Caso de uso es primario o se extiende de otro. La mayoría se 
					  % extienden de Login.
		% EJEMPLO: \refIdElem{PY-CU1} o Caso de uso primario.
		\TODO Especificar.
	}	
	%----------------------------------------------
	\UCsection[design]{Datos de Diseño}
	\UCitems[design]{Disparador}{% Especificar que evento dispara el caso de uso suelen ser de tipo:
	                            % - Evento: No programado, solicitado por el cliente.
	                            % - Periódico: Cada fin de mes, semana, diario, etc.
	                            % - Compensación: Se usa para corregir un error en algún proceso.
	                            % - Mensaje: se recibe una solicitud, oficio, correo, etc.
	                            % - Señal: El usuario detecta que se requiere ejecutar el CU.
	                            % Especificar:
	                            % - Evento: descripción del evento que motiva al usuario.
	                            % - Frecuencia: cada cuanto se ejecuta el CU, usar:
	                            %    - ``Dato duro'' cuando se conoce exactamente cuantas veces ocurre.
	                            %    - ``En promedio X con mas menos Y'' cuando se tienen datos 
	                            %      estadísticos que respaldan el dato.
	                            %    - ``Aproximadamente XX'' cuando es la opinion del usuario.
	                            %    - Indeterminado, cuando es un proceso nuevo o no hay un 
	                            %      usuario con experiencia al cual preguntar.
		% EJEMPLOS: 
		% - Evento: Se debe realizar una vez antes de que termine el mes. Frecuencia: mensual. Usuarios: uno por departamento, 23 en total.
		% - Evento: Se recibe un oficio de solicitud de Trámite de licencia. Frecuencia: En promedio 25 y no mas de 50 diarios . Usuarios: Dos, a contraturno.
		% - Evento: Cada que el jefe lo solicita. Frecuencia: Aproximadamente 5 por semana. Usuarios: uno.
		% - Evento: Cada que hay una solicitud sin atender. Frecuencia: no hay datos para estimar. Usuarios: Aproximadamente 5.
		\Titem \textbf{Evento}: \TODO Especificar
		\Titem \textbf{Frecuencia}: \TODO Especificar
		\Titem \textbf{Usuarios}: \TODO Especificar
	}
	\UCitems[design]{Casos de Prueba}{% Especificar todos los casos de prueba que pueda identificar.
	                                  % TE QUIEROOOOO 
	                                  % Considere:
	                                  % - Casos correctos.
	                                  % - Validación de datos.
	                                  % - Valores a la frontera.
	                                  % - Validar precondiciones.
	                                  % - Validar postcondiciones.
	                                  % - Validar Reglas de negocio.
	                                  % - Verificar efectos colaterales.
		% EJEMPLOS:
		% - CPS-1: Proporcionando todos los datos correctos para un cliente Persona moral
		% - CPS-2: Proporcionando todos los datos correctos para un cliente Persona física
		% - CPS-3: Dejando vacío uno por uno cada uno de los datos obligatorios.
		% - CPS-4: Especificar una fecha de nacimiento posterior a la del día de hoy.
		% - CPS-5: Registrar la venta especificando un precio con valor negativo.
		% - CPS-6: Registrar de una venta sin productos.
		% - CPS-7: Realizar una venta de un producto sin existencias.
		\Titem \TODO Especificar
	}
	%----------------------------------------------    
	\UCsection[admin]{Datos de Administración de Requerimiento}
	\UCitem[admin]{Prioridad}{% Indicar la prioridad del CU como está especificada en el 
						      % Product Backlog, los valores son: Muy Alta, Alta, Media, Baja, Muy Baja
		\TODO Especificar
	}
	\UCitems[admin]{Referencia Documental}{% Indicar los documentos que sirvieron de base para escribir el CU.
		% EJEMPLO: - Reglamento interno Artículo 45 y 25 del capítulo 5.
		%          - Minuta TR-PL-45.
		%          - Manual de procedimientos proceso PR-098.
		\Titem \TODO Especificar.
	}
    \UCitems[admin]{Impedimentos}{% UN BUEN CASO DE USO DEBE DECIR EN ESTE APARTADO ``Ninguno''.
                                  % Use este apartado cuando se vea obligado a entregar a 
                                  % revisión o aprobación un CU al que aun le faltan cosas.
                                  % Los impedimentos son:
                                  % - Datos faltantes solicitados y que no han sido entregados 
                                  % - Reglamentos no proporcionados.
                                  % - Incapacidad de llegar a un acuerdo
                                  % - Requerimientos constantemente cambiantes.
                                  % - Dependencia a factores que no se pueden controlar ni definir.
                                  % - Inaccesibilidad al usuario o a alguien que valide.
                                  % - Contraindicaciones entre requerimientos.
		% EJEMPLO: 
		% - Este Caso de Uso no contempla la nueva legislación de seguridad.
		% - Este caso de uso está basado en el proceso anterior por desconocimiento del nuevo proceso.
		% - Este caso de uso solo funcionará lara pólizas personales por que las empresariales 
		%   nunca fueron proporcionadas.
		% - No se ha podido acordar la forma de pago por parte del comité, falta definir esa parte.
		% - Finanzas y Contabilidad no han podido ponerse de acuerdo en relación al catálogo a utilizar.
		\Titem \TODO Especificar
	}
	\UCitems[admin]{Suposiciones}{% UN BUEN CASO DE USO DEBE DECIR EN ESTE APARTADO ``Ninguno''.
	                             % Todas las desiciones que no han podido ser verificadas se 
	                             % deben especificar aquí.
		% EJEMPLO: Se considera que el usuario debió haber sido capacitado por la empresa.
		\Titem \TODO Especificar
	}
	\UCitems[admin]{Observaciones}{% Use este apartado para agregar cualquier información que 
	                               % sea relevante y que no corresponda a ningún otro apartado.
	                               % Si no hay se debe poner ``Ninguna''.
		% EJEMPLO: En este caso de uso el usuario revisa la información minuciosamente por mas de 30 minutos, por lo que se debe considerar no cerrar la sesión e implementar el autoguardado.
		\Titem \TODO Especificar.
	}
\end{UseCase}

\begin{comment}
% \end{comment}

%\textbullet{Trayectorias}

\begin{UCtrayectoria}{Principal}
    \actor se encuentra en la interfaz \hyperref[IUM04]{IU-M04 Sección de plugins}.
    \actor Busca la sección de ''Bloques'' .
    \actor selecciona el componente ''Gamedle level''.
    \sistema carga la interfaz \hyperref[IUE03]{IU-E03 Configuración del esquema de experiencia}.
    \actor modifica las opciones que desea ({\it Trayectoria alternativa A}).
    \actor le da al botón \#1 ''Guardar cambios''  ({\it Trayectoria alternativa B}).
    \item[- -] - - {\em El caso de uso termina.}
    
\end{UCtrayectoria}

\begin{UCtrayectoria}{alternativa A}
    \item[- -] - - {\em El actor no quiere cambiar ninguna configuración.}
    \actor selecciona alguna opción en el menú de Moodle o en el directorio de la página
    \sistema carga la interfaz correspondiente a la selección del actor.
    \item[- -] - - {\em El caso de uso termina.}
\end{UCtrayectoria}


\begin{UCtrayectoria}{alternativa B}
    \item[- -] - - {\em Alguna de las entradas no es válida.}
    \sistema despliega el mensaje MS1.
    \sistema despliega el mensaje MS2 en cada una de las entradas que ha encontrado como inválidas.
    
    \item[- -] - - {\em se continúa en el paso \#5 de la trayectoria principal.}
\end{UCtrayectoria}
\end{comment}


\vfill\clearpage
