

\subsubsection{Diseño de complementos}



A continuación se presenta cómo el módulo financiero
se implmenetan en moodle.\\


\noindent Resumiendo el módulo financiero, se  tiene una pantalla donde se pueden comprar
objetos, la cual es la misma donde se edita el perfil gamificado.
Esta pantalla tiene diferente comporamiento dependiendo que complemento está activo.


\noindent Tomando en consideración lo anterior y que existe el complemento gamedlemaster, se presenta en la figura \ref{fig:diseno-comp-finan}
los complementos contemplados y las dependencias entre los mismos.


    \addfigure{1}{modulos/finan/diagrams/diseno_complementos}{fig:diseno-comp-finan}{Implementación del modulo de competencia}


Cada complemento en la figura \ref{fig:diseno-comp-finan} está representado con una cadena que sigue el formato 'tipo\_de\_complemento:nombre\_de\_complemento'.
Los tipos de complemento son;
\begin{itemize}
    \item \textbf{local} -  Este complemento moodle lo iterpreta como un comdín, el cual puede ser usado para múltiples propósitos.
    \item \textbf{block} - Este complemento permite desplegar una sección en la mayoría de las páginas de moodle, la cuál puede representar código html.
\end{itemize}

La función de cada uno de los complementos presentados en la figura \ref{fig:diseno-comp-finan} son:


\begin{itemize}
    \item \textbf{gamedlemaster} Definir la base de datos y los eventos a manejar.
    \item \textbf{gmtienda} Permitirle al usuario comprar objetos para su perfil gamificado.
\end{itemize} 

