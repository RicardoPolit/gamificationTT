
% \ucstEnEdicion     Al terminar una revisión/aprobación con observaciones
%                    y al inicio del CU.
%
% \ucstEnRevision    Al terminar la edición del CU (version += 0.1).
% \ucstEnAprobacion  Al pasar la revision sin observaciones.
% \ucstAprobado      Al ser aprobado por el usuario (version += 1.0)

\begin{UseCase}[%
Autor/David Flores Casanova,%
Version/0.1,%
Estado/\ucstEnRevision]%
%
{CU-F03}{Comprar objeto}{%
%
Permite al usuario (Ya sea un \refElem{aProfesor}, un \refElem{aAdministrador} o un \refElem{aEstudiante})
 de moodle adquirir un objeto para la personalización de su perfil utilizando sus monedas disponibles.
 Este caso de uso es una extensión del caso de uso {\it \refElem{CU-P01}}.}

	\UCitem[control]{Revisor}{ Sin asignar }
	\UCitem[control]{Último cambio}{ 17/NOV/19 }

 \UCsection{Atributos}

    \UCitem{Actor(es)}{%
        \refElem{aProfesor},
        \refElem{aAdministrador},
        \refElem{aEstudiante}
    }

	\UCitems{Propósito}{%
        \Titem El usuario quiere saber el estado actual de su perfil, así como las monedas que tiene disponibles.
	}

	\UCitem{Entradas}{\imprimeUC{entrada}}

	\UCitems{Origen}{%
        \Titem Mouse
	}

	\UCitem{Salidas}{
        \imprimeUC{salida}
        \Titem ''¡Objeto [NombreDelObjeto] comprado!''%\refElem{MSG-F01}
        \Titem ''¡Oh no!, No tienes suficiente dinero para comprar el objeto [NombreDelObjeto]!''%\refElem{MSG-F02}
        \Titem ''¡Ya tenías desbloqueado el objeto [NombreDelObjeto]!''%\refElem{MSG-F03}

        }

	\UCitem{Destino}{%
		\refElem{IU-P01}
	}

	\UCitems{Precondiciones}{%
        \Titem El usuario debió de haber ejecutado el \refElem{CU-P01}.
        \Titem El usuario debe contener las monedas suficientes para comprar el objeto.
        \Titem El objeto que se quiere  comprar no debe estar ya adquirido.
	}

	\UCitems{Postcondiciones}{%
        \Titem El usuario tendrá desbloqueado el objeto para usarlo.
        \Titem El usuario se le restarán las \refElem{xp-user.monedas-plata} del objeto que adquirió.
	}

	\UCitem{Reglas de negocio}{\imprimeUC{regla}}

	\UCitems{Errores}{%
        \Titem El usuario no cuenta con las suficientes monedas para adquirir el objeto.
	}

 \UCsection[design]{Datos de Diseño}

	\UCitems[design]{Casos de Prueba}{%
	}

 \UCsection[admin]{Datos de Administración de Requerimiento}

	\UCitem[admin]{Observaciones}{}

\end{UseCase}

\subsubsection{Trayectorias del caso de uso}

\begin{UCtrayectoria}%
%

    \Actor Selecciona dando clic a la opción \textbf{Comprar} (Indicado por el ícono de las monedas \IUMonedas{}) del objeto que desea comprar en la pantalla \refElem{IU-P01}.
    \Sistema Comprueba que el actor cuenta con suficientes \refElem{xp-user.monedas-plata} para comprar el objeto. \refTray{A}
    \Sistema Comprueba que el actor no cuente con ese objeto todavía. \refTray{B}
    \Sistema Desbloquea el objeto seleccionado para el actor, guardándolo en \refElem{tienda-gmdl-objeto-desbloqueado}.
    \Sistema Le resta al actor las \refElem{xp-user.monedas-plata} dependiendo el \refElem{tienda-gmdl-rareza-objeto.costo-adquisicion} del objeto seleccionado.
    \Sistema Muestra el mensaje ''¡Objeto [NombreDelObjeto] comprado!''%\refElem{MSG-P01}

\end{UCtrayectoria}

\begin{UCtrayectoriaA}[Fin del caso de uso]%
  {A}{El actor no cuenta con las monedas necesarias }

  \Sistema Muestra el mensaje  ''¡Oh no!, No tienes suficiente dinero para comprar el objeto [NombreDelObjeto]!''%\refElem{MSG-F02}.

\end{UCtrayectoriaA}

\begin{UCtrayectoriaA}[Fin del caso de uso]%
{B}{El actor ya tiene comprado el objeto}

    \Sistema Muestra el mensaje ''¡Ya tenías desbloqueado el objeto [NombreDelObjeto]!''%\refElem{MSG-F03}

\end{UCtrayectoriaA}
