
% \ucstEnEdicion     Al terminar una revisión/aprobación con observaciones
%                    y al inicio del CU.
%
% \ucstEnRevision    Al terminar la edición del CU (version += 0.1).
% \ucstEnAprobacion  Al pasar la revision sin observaciones.
% \ucstAprobado      Al ser aprobado por el usuario (version += 1.0)

\begin{UseCase}[%
Autor/David Flores Casanova,%
Version/0.1,%
Estado/\ucstEnRevision]%
%
{CU-F02}{Instalar esquema financiero }{%
%
    El \refElem{aAdministrador} instala el complementto como lo especifica moodle y al instalarlo moodle le permite 
    configurar las opciones una vez instalado.
    Este caso de uso es una extensión del caso de uso {\it Instalar complemento.}}

	\UCitem[control]{Revisor}{ Sin asignar }
	\UCitem[control]{Último cambio}{ 17/NOV/19 }

 \UCsection{Atributos}

    \UCitem{Actor(es)}{%
        \refElem{aAdministrador}
    }

	\UCitems{Propósito}{%
        \Titem El actor desea utilizar las funciones que brina el módulo financiero .
	}

	\UCitem{Entradas}{\imprimeUC{entrada}}

	\UCitems{Origen}{%
        \Titem Mouse
	}

	\UCitem{Salidas}{
        \imprimeUC{salida}}

	\UCitem{Destino}{%
		\refElem{IU-F01}
	}

	\UCitems{Precondiciones}{%
        \Titem El complemento del módulo financiero no debe de estar instalado.
	}

	\UCitems{Postcondiciones}{%
        \Titem El módulo financiero ahora estará funcionando en la plataforma del actor.
        \Titem Se guardará la configuración del actor.
	}

	\UCitem{Reglas de negocio}{\imprimeUC{regla}}

	\UCitems{Errores}{%
	}

 \UCsection[design]{Datos de Diseño}

	\UCitems[design]{Casos de Prueba}{%
	}

 \UCsection[admin]{Datos de Administración de Requerimiento}

	\UCitem[admin]{Observaciones}{}

\end{UseCase}

\subsubsection{Trayectorias del caso de uso}

\begin{UCtrayectoria}%
%

    \Actor Modifica si desea que el módulo financiero esté activado o no en la plataforma, usando pantalla \refElem{IU-F02}.
    \Actor Presiona el botón \textbf{Guardar cambios}. 
    \label{CU-F01-guardar-cambios}
    \Sistema Redirige a la pantalla \refElem{IU-F01}.

\end{UCtrayectoria}
