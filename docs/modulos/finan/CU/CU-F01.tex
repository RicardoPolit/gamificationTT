
% \ucstEnEdicion     Al terminar una revisión/aprobación con observaciones
%                    y al inicio del CU.
%
% \ucstEnRevision    Al terminar la edición del CU (version += 0.1).
% \ucstEnAprobacion  Al pasar la revision sin observaciones.
% \ucstAprobado      Al ser aprobado por el usuario (version += 1.0)

\begin{UseCase}[%
Autor/David Flores Casanova,%
Version/0.1,%
Estado/\ucstEnRevision]%
%
{CU-F01}{Modificar esquema financiero }{%
%
    Permite al \refElem{aAdministrador} modificar la cantidad de monedas que otorga cada evento,
    así como, si desea que el esquema financiero esté activado o no.
    Este caso de uso es una extensión del caso de uso {\it Entrar a administración de complementos} que es propio de moodle }

	\UCitem[control]{Revisor}{ Sin asignar }
	\UCitem[control]{Último cambio}{ 17/NOV/19 }

 \UCsection{Atributos}

    \UCitem{Actor(es)}{%
        \refElem{aAdministrador}
    }

	\UCitems{Propósito}{%
        \Titem El actor desea cambiar la cantidad de monedas que da un evento o quiere activar  o desactivar el esquema financiero.
	}

	\UCitem{Entradas}{\imprimeUC{entrada}}

	\UCitems{Origen}{%
        \Titem Mouse
	}

	\UCitem{Salidas}{
        \imprimeUC{salida}}

	\UCitem{Destino}{%
		\refElem{IU-F01}
	}

	\UCitems{Precondiciones}{%
        \Titem El complemento del módulo financiero debe estar instalado.
	}

	\UCitems{Postcondiciones}{%
        \Titem Se guardará la nueva configuración del actor.
	}

	\UCitem{Reglas de negocio}{\imprimeUC{regla}}

	\UCitems{Errores}{%
        \Titem Alguno de los campos fueron ingresados de manera errónea.
	}

 \UCsection[design]{Datos de Diseño}

	\UCitems[design]{Casos de Prueba}{%
        \Titem \refElem{CPC-F01-1}
        \Titem \refElem{CPI-F01-2}
        \Titem \refElem{CPI-F01-3}
	}

 \UCsection[admin]{Datos de Administración de Requerimiento}

	\UCitem[admin]{Observaciones}{}

\end{UseCase}

\subsubsection{Trayectorias del caso de uso}

\begin{UCtrayectoria}%
%

    \Actor Selecciona dando clic a la opción \textbf{Gamedle: Módulo financiero} en la pantalla \refElem{IU-M11}. \refTray{A}
    \Sistema Redirige a la pantalla \refElem{IU-F01}.
    \Actor Modifica las opciones que desea. \refTray{B}
    \label{CU-F01-hacer-cambios}
    \Actor Presiona el botón \textbf{Guardar cambios}.
    \label{CU-F01-guardar-cambios}
    \Sistema Valida que los valores ingresados sean correctos. \refTray{C}
    \Sistema Guarda la nueva configuración.

\end{UCtrayectoria}

\begin{UCtrayectoriaA}[Fin del caso de uso]%
  {A}{El complemento del  módulo financiero no se encuentra instalado}

  \Actor No encuentra la opción en la pantalla  \textbf{Gamedle: Módulo financiero}, porque el sistema no la generó.

\end{UCtrayectoriaA}

\begin{UCtrayectoriaA}%
{B}{El actor desea modificar la configuración financiera}
    \Actor Selecciona dando clic a la opción \textbf{Gamedle: Módulo financiero} en la pantalla \refElem{IU-F01}.
    \Sistema Redirige a la pantalla \refElem{IU-F02}.
    \Actor Modifica las opciones que desea.
    \item Se regresa al paso \ref{CU-F01-guardar-cambios} de la trayectoria principal.

\end{UCtrayectoriaA}

\begin{UCtrayectoriaA}%
{B}{El actor desea modificar la configuración financiera}
    \Actor Selecciona dando clic a la opción \textbf{Gamedle: Módulo financiero} en la pantalla \refElem{IU-F01}.
    \Sistema Redirige a la pantalla \refElem{IU-F02}.
    \Actor Modifica las opciones que desea.
    \item Se regresa al paso \ref{CU-F01-guardar-cambios} de la trayectoria principal.

\end{UCtrayectoriaA}


\begin{UCtrayectoriaA}%
{C}{Se han ingresaron datos erróneos}
    \Sistema Muestra el mensaje \refElem{MSG-F04} en cada campo con valores erróneos.
    \item Se regresa al paso \ref{CU-F01-hacer-cambios}.

\end{UCtrayectoriaA}
