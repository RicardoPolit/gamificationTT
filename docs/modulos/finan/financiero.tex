
\subsection{Análisis}

 Este apartado contiene el análisis requerido para la elaboración de módulo financiero,
 contiene la especificación del alcance de este módulo, la descripción de las funcionalidades
 a desarrollar, la reglas de negocio que rigen el comportamiento del módulo, y por último la
 especificación de los casos de uso a los que brinda soporte.

%\subsubsection{Submódulo de competencia 1 contra 1}
%\subsubsection{Funcionalidades}

\subsubsection{Reglas de negocio} %========================================================

 En esta sección se especifican todas las reglas de negocio relevantes para el módulo de
 experiencia. Las reglas de negocio que establece moodle son diferenciadas por tener la letra {\it M}
 antecediendo al número consecutivo en su identificador.

    %
\subsection{Entidades de moodle}

Debido a que moodle cuenta con más de 400 tablas en su versión 3.5, se opta
por mostrar 2 subconjuntos que muestren las tablas que se utilizan para el proyecto.\\

\noindent El primer subconjunto es aquel que explica la forma en que moodle implementa los cursos, 
secciones de curso, actividades, usuarios y roles (el cual se presenta en la figura \ref{fig:BD-ER-M1}), 
mientras que el segundo conjunto muestra como moodle maneja toda la 
estructura de las preguntas creadas por el profesor y respondidas por el usuario
(el cual se presenta en la figura \ref{fig:BD-ER-M2}).  



En lugar de describir y mostrar cada uno de los campos de cada una de las entidades de moodle que se contemplan,
lo que se quiere lograr con ambos esquemas (\ref{fig:BD-ER-M1}) y \ref{fig:BD-ER-M2}))
es expresar la idea general del comportamiento.

\clearpage
\addfigure{0.7}{analisis/diagrams/db_module_structure}{fig:BD-ER-M1}{Esquema de la base de datos de moodle 'Cursos'}


\noindent Utilizando la figura \ref{fig:BD-ER-M1}, se obtuvieron las siguientes reglas y caracteristicas que contiene moodle respecto a los usuarios en un curso y a la estructura de los cursos.
\begin{enumerate}
    \item Un usuario -{\it mdl\_user}- tiene un rol -{\it mdl\_role}- en un cierto contexto -{\it mdl\_context}-, cuyo  '{\it context\_level}' sea igual a cincuenta(50).
    \item Si el contexto '{\it context\_level}' es de 50, el atributo '{\it instance\_id}' hace referencia al atributo '{\it id}' de un curso -{\it mdl\_course}-.
    \item El curso -{\it mdl\_course}- tiene varias secciones -{\it mdl\_course\_sections}-.
    \item Cada seccion -{\it mdl\_course\_sections}- tiene varias actividades -{\it mdl\_course\_modules}- que pertenecen a un tipo de actividad -{\it mdl\_modules}-.
    \item Por cada registro en tipo de actividad -{\it mdl\_modules}-, se tiene una entidad que lleva el mismo nombre.
    \item El atributo '{\it instance\_id}' de una actividad  -{\it mdl\_course\_modules}- apunta a diferentes entidades. La entidad a la que apunta depende del nombre del tipo de actividad -{\it mdl\_modules}-.
    \item Un usuario -{\it mdl\_user}- se inscribió -{\it mdl\_user\_enrolments}- a un curso -{\it mdl\_course}-, por medio de un formato soportado de inscripción -{\it mdl\_enrol}-.
\end{enumerate}

\clearpage

 \addfigure{0.7}{analisis/diagrams/db_module_questions}{fig:BD-ER-M2}{Esquema de la base de datos de moodle 'Preguntas' }



\noindent Utilizando la figura \ref{fig:BD-ER-M2}, se obtuvieron las siguientes reglas y caracteristicas que contiene moodle respecto a las preguntas.
\begin{enumerate}
    \item Las preguntas -{\it mdl\_question}- tienen versiones -{\it mdl\_question\_attempts}-.
    \item Una pregunta -{\it mdl\_question}- pertenece a un banco de preguntas -{\it mdl\_question\_categories}-.
    \item La versión de una pregunta -{\it mdl\_question\_attempts}- es contestada -{\it mdl\_question\_usages}- en un determinado contexto -{\it mdl\_context}-.
    \item Un usuario -{\it mdl\_user}- responde una versión de una pregunta -{\it mdl\_question\_attempt\_stepts}-.
    \item El responder una versión de una pregunta -{\it mdl\_question\_attempt\_stepts}- conlleva pasos -{\it mdl\_question\_attempt\_stept\_data}-, los cuales son: cómo se muestra, si ya se terminó de responder y qué se respondió.
\end{enumerate}


 A continuación se presenta la especificación de las entidades del esquema de base
 de datos de moodle que son relevantes para el desarrollo de los módulos y submódulos
 de proyecto.

    \begin{cdtEntidad}{mdl-config-plugins}{Configuración de Plugin}{%
    Es una tabla del núcleo de moodle que almacena todas las configuraciones globales
    relacionadas a los plugins instalados, al iniciar moodle las configuraciones de los
    plugins instalados y habilitados se cargan en memoria.}

	    \brAttr{id}{Id}{tInt}{%
	        Es el dígito que representa el identificador único para una configuración
            específica de un plugin.\par

            \it Restricciones:
            \refElem{tPrimaryKey},
            \refElem{tAutoIncrement}.
        }

        \brAttr{plugin}{Plugin}{tVarchar}{%
            Cadena de caracteres del nombre identificador del plugin al cual pertenece
            la configuración.\par

            \it Restricciones:
            \refElem{tRequired},
            \refElem{tRange} (0,100),
            \refElem{tUniqueKey}
        }

        \brAttr{name}{Nombre}{tVarchar}{%
            Cadena de caracteres que representa el nombre de la configuración de un
            plugin en específico.\par

            \it Restricciones:
            \refElem{tUniqueKey},
            \refElem{tRange} (0,100),
            \refElem{tRequired}
        }

        \brAttr{value}{Valor}{tVarchar}{%
            Cadena que almacena el valor de una configuración perteneciente a alguno
            de los plugins instalados.\par

            \it Restricciones:
            \refElem{tRange} (0,4294967295),
            \refElem{tRequired}
        }
    \end{cdtEntidad}\schemeName{config\_plugins}

    \begin{cdtEntidad}{mdl-user}{Usuario de moodle}{%
    Es una tabla del núcleo de moodle que contiene toda la información que se
    almacena de los usuarios en la plataforma, independientemente del rol que
    estos contenga, esta relación contiene más de 53 atributos, sin embargo solo
    se detallan aquellos relevantes.}

	    \brAttr{id}{Id}{tInt}{%
	        Es el dígito que representa el identificador único para cada uno
            de los usuarios en moodle.\par

            \it Restricciones:
            \refElem{tPrimaryKey},
            \refElem{tAutoIncrement}.
        }
	    \brAttr{username}{nombre de usuario}{tVarchar}{%
	        .\par

            \it Restricciones:
            \refElem{tRequired},
            \refElem{tLength} 0-100
        }
	    \brAttr{password}{contraseña}{tVarchar}{%
	        .\par

            \it Restricciones:
            \refElem{tRequired},
            \refElem{tLength} 0-255.
        }
	    \brAttr{firstname}{nombre}{tVarchar}{%
	        .\par

            \it Restricciones:
            \refElem{tRequired},
            \refElem{tLength} 0-100
        }
	    \brAttr{lastname}{apellido}{tVarchar}{%
	        .\par

            \it Restricciones:
            \refElem{tRequired},
            \refElem{tLength} 0-100
        }
	    \brAttr{email}{correo}{tVarchar}{%
	        .\par

            \it Restricciones:
            \refElem{tRequired},
            \refElem{tLength} 0-100
        }
	    \brAttr{lastaccess}{último registro}{tInt}{%
	        .\par

            \it Restricciones:
            \refElem{tRequired},
            \refElem{tLength} 10
        }
	    \brAttr{city}{ciudad}{tVarchar}{%
	        .\par

            \it Restricciones:
            \refElem{tRequired},
            \refElem{tLength} 0-120
        }
	    \brAttr{country}{pais}{tVarchar}{%
	        .\par

            \it Restricciones:
            \refElem{tRequired},
            \refElem{tLength} 2
        }

    \end{cdtEntidad}\schemeName{user}

    \begin{cdtEntidad}{mdl-course}{Curso de moodle}{%
    Es una tabla del núcleo de moodle que contiene la información principal de cada 
    curso registrado en moodle. Esta entidad contiene 31 atributos, a continuación se
    detallan los atributos relevantes para la especificación de este proyecto.}

	    \brAttr{id}{Id}{tInt}{%
	        Es el dígito que representa al identificador único para cada uno
            de los cursos en moodle.\par

            \it Restricciones:
            \refElem{tPrimaryKey},
            \refElem{tAutoIncrement}.
        }

	    \brAttr{format}{formato}{tVarchar}{%
	        Es el dígito que representa al identificador único para cada uno
            de los cursos en moodle.\par

            \it Restricciones:
            \refElem{tRequired}.
            \refElem{tDefault} topics,
            \refElem{tLength} 0-21.
        }

	    \brAttr{fullname}{nombre completo}{tVarchar}{%
	        Es el nombre completo que se le asigna al curso.\par

            \it Restricciones:
            \refElem{tRequired}.
            \refElem{tLength} 0-21.
        }

	    \brAttr{shortname}{nombre corto}{tVarchar}{%
            Es el nombre corto que se le asigna al curso.\par

            \it Restricciones:
            \refElem{tRequired}.
            \refElem{tLength} 0-21.
        }

    \end{cdtEntidad}\schemeName{course}

    \begin{cdtEntidad}{mdl-course-sections}{Secciones del curso de moodle}{%
    }
	    \brAttr{id}{Id}{tInt}{%
	        Es el dígito que representa al identificador único para cada seccion
            de los cursos en moodle.\par

            \it Restricciones:
            \refElem{tPrimaryKey},
            \refElem{tAutoIncrement}.
        }
    \end{cdtEntidad}\schemeName{course\_sections}

    \begin{cdtEntidad}{mdl-course-format-options}{Opciones del formato del curso}{%
    }

	    \brAttr{id}{Id}{tInt}{%
	        Es el dígito que representa al identificador único para cada uno
            de los cursos en moodle.\par

            \it Restricciones:
            \refElem{tPrimaryKey},
            \refElem{tAutoIncrement}.
        }

	    \brAttr{courseid}{Id}{tInt}{%
	        Es el dígito que representa al identificador único para cada uno
            de los cursos en moodle.\par

            \it Restricciones:
            \refElem{tForeignKey},
            \refElem{tRequired}
        }

	    \brAttr{format}{formato}{tVarchar}{%
	        Es el dígito que representa al identificador único para cada uno
            de los cursos en moodle.\par

            \it Restricciones:
            \refElem{tRequired}.
            \refElem{tDefault} topics,
            \refElem{tLength} 0-21.
        }

	    \brAttr{name}{opcion}{tVarchar}{%
	        Es el dígito que representa al identificador único para cada uno
            de los cursos en moodle.\par

            \it Restricciones:
            \refElem{tPrimaryKey},
            \refElem{tLength} 0-100
        }

	    \brAttr{value}{valor}{tVarchar}{%
	        Es el dígito que representa al identificador único para cada uno
            de los cursos en moodle.\par

            \it Restricciones:
            \refElem{tRequired}
        }

    \end{cdtEntidad}\schemeName{course\_format\_options}

    \begin{cdtEntidad}{mdl-course-category}{Categoria de curso}{%
    .}
    \end{cdtEntidad}\schemeName{course\_category}

    \begin{cdtEntidad}{Plugin}{Plugin}{%
    La forma en que moodle obtiene información acerca de los plugins es analizando
    los archivos internos de cada uno, a pesar de que los plugins no forman parte
    del esquema de base de datos, si forman parte del modelo de información que
    utiliza Moodle.}

	    \brAttr{componente}{Componente}{tVarchar}{%
	        Cadena compuesta por el tipo de plugin y el nombre del mismo, que
            representa a la clase principal del plugin que contiene los métodos
            principales del plugin.\par

            \it Restricciones: Ninguna
        }

	    \brAttr{pluginname}{Nombre}{tVarchar}{%
	        Es el nombre del plugin obtenido de los archivos de
            internacionalización presentes en el plugin, el valor de esta cadena
            depende del lenguaje seleccionado en moodle.\par

            \it Restricciones: Ninguna
        }

	    \brAttr{fullpath}{Ruta absoluta}{tPath}{%
	        La ruta absoluta de un plugin denota la ubicación del plugin en el
            sistema de archivos, esta ruta está compuesta por la ruta absoluta
            de la instalación de moodle, la carpeta correspondiente al tipo del
            plugin y el nombre del plugin.\par

            \it Restricciones: Formato ``/path/to/moodle/plugintype/pluginname''
        }

	    \brAttr{path}{Ruta relativa}{tPath}{%
	        La ruta relativa denota la ubicación del plugin dentro de la carpeta 
            donde se encuentran los archivos de moodle, esta ruta está compuesta
            por la carpeta correspondiente al tipo del plugin y el nombre del
            plugin.\par

            \it Restricciones: Formato ``plugintype/pluginname''
        }

	    \brAttr{version}{Versión}{tVersion}{%
	        Numero entero de longitud de 10 dígitos que representa la versión del 
            plugin.\par

            \it Restricciones: Ninguna adicional al tipo de dato
        }

	    \brAttr{moodle}{Versión de Moodle}{tVersion}{%
	        Número entero de longitud de 10 dígitos que representa la versión de 
            moodle en la que se puede instalar el plugin.\par

            \it Restricciones: Ninguna adicional al tipo de dato
        }

        \brAttr{dependencies}{Dependencias}{tObject}{%
            Objeto que almacena un conjunto de claves con sus respectivos valores, 
            donde cada clave representa el nombre del componente del plugin y el valor 
            es la \refElem{Plugin.version} requerida del mismo.

            \it Restricciones: Ninguna
        }

        \brAttr{icon}{ícono}{tImage}{%
            Imagen para el ícono del plugin, debe estar contenida en el directorio
            {\it pix/} del plugin y tener como nombre {\it icon.png} o {\it icon.svg},
            moodle recomienda tener ambos archivos por si los navegadores no soportan
            algun tipo de archivo \cite{moodlePluginfiles}.\par 

            \it Restricciones: El nombre debe ser icon con extensiones png o svg
        }

    \end{cdtEntidad}
 % Archivo de plugin
    %
\begin{BusinessRule}[%
Autor/Ricard Naranjo Polit,%
Version/0.1,%
Estado/revision]%
%
{BR-C01}{Restricciones del tiempo que tiene }
 % El archivo de instalación debe ser un archivo ZIP, el cual debe contener
 % exactamente un directorio que coincida con el nombre del plugin.
     \BRitem[control]{Revisor}{Sin asignar.}

 \BRsection[control]{Atributos}

    \BRitem[admin]{Clase}{\bcCondition}%
    %\BRitem[admin]{Clase}{\bcIntegridad}%
    %\BRitem[admin]{Clase}{\bcAutorizacion}%
    %\BRitem[admin]{Clase}{\bcDerivacion}%

    \BRitem[admin]{Tipo}{\btEnabler}%
    %\BRitem[admin]{Tipo}{\btTimer}%
    %\BRitem[admin]{Tipo}{\btExecutive}%

    \BRitem[admin]{Nivel}{\blControlling}
    %\BRitem[admin]{Nivel}{\blInfluencing}

    \BRitem{Descripción}{%
        El archivo seleccionado para la representación visual de los niveles debe
        ser una imagen con las extensiones {\it``png''} o {\it''jgp}, además el
        nombre del archivo que será subido no debe tener el nombre {\it``icon.png''}
        ya que posiblemente colisionaría con el \refElem{Plugin.icon} del plugin.
        % debido a que se el directorio donde se guardará será el directorio
        % para almacenar las imágenes del plugin.
    }

    \BRitem{Ejemplo positivo}{\hfill\par%
        \begin{itemize}
        \item El archivo seleccionado para ser la imagen de los niveles tiene
              como nombre {\it``logotipo''} con la extensión {\it png}.

        \item El archivo seleccionado para ser la imagen de los niveles tiene
              como nombre {\it``nivel''} con la extensión {\it jpg}
        \end{itemize}
    }

    \BRitem{Ejemplo negativo}{\hfill\par%
        \begin{itemize}
        \item El archivo seleccionado para ser la imagen de los niveles tiene
              como nombre {\it``icon''} con la extensión {\it png}.

        \item El archivo seleccionado para ser la imagen de los niveles tiene
              como nombre {\it``documento''} con la extensión {\it doc}.
        \end{itemize}
    }%

 \end{BusinessRule}
 % Restricciones sobre de imagen del nivel.
    %
\begin{BusinessRule}[%
Autor/Ricard Naranjo Polit,%
Version/0.1,%
Estado/revision]%
%
{BR-C02}{Un usuario no puede desafiar a otro con el que tenga un desafío pendiente}
 % El archivo de instalación debe ser un archivo ZIP, el cual debe contener
 % exactamente un directorio que coincida con el nombre del plugin.
     \BRitem[control]{Revisor}{Sin asignar.}

 \BRsection[control]{Atributos}

    \BRitem[admin]{Clase}{\bcCondition}%
    %\BRitem[admin]{Clase}{\bcIntegridad}%
    %\BRitem[admin]{Clase}{\bcAutorizacion}%
    %\BRitem[admin]{Clase}{\bcDerivacion}%

    \BRitem[admin]{Tipo}{\btEnabler}%
    %\BRitem[admin]{Tipo}{\btTimer}%
    %\BRitem[admin]{Tipo}{\btExecutive}%

    \BRitem[admin]{Nivel}{\blControlling}
    %\BRitem[admin]{Nivel}{\blInfluencing}

    \BRitem{Descripción}{%
        Cuando un usuario desafía a un \refElem{aEstudiante} no lo podrá volver a desafiar hasta que el desafiante
        y desafiado terminen hayan completado la competencia.
        % debido a que se el directorio donde se guardará será el directorio
        % para almacenar las imágenes del plugin.
    }

    \BRitem{Ejemplo positivo}{\hfill\par%
        \begin{itemize}
        \item El usuario desafía a un estudiante, los dos terminan la competencia y el usuario vuelve a desafiar al mismo estudiante.

        \end{itemize}
    }

    \BRitem{Ejemplo negativo}{\hfill\par%
        \begin{itemize}
          \item El usuario desafía a un estudiante y no alguno de los dos no termina la competencia,
          el usuario no puede volver a desafiar al mismo estudiante.

        \end{itemize}
    }%

 \end{BusinessRule}
 % Permanencia del nivel de comperiencia.
    %
\begin{BusinessRule}[%
Autor/Daniel Isai Ortega Zúñiga,%
Version/0.1,%
Estado/revision]%
%
{BR-E03}{Tipos de Incremento}
    \BRitem[control]{Revisor}{Sin asignar.}

 \BRsection[control]{Atributos}
    
    \BRitem[admin]{Clase}{\bcCondition}%
    %\BRitem[admin]{Clase}{\bcIntegridad}%
    %\BRitem[admin]{Clase}{\bcAutorizacion}%
    %\BRitem[admin]{Clase}{\bcDerivacion}%
        
    \BRitem[admin]{Tipo}{\btEnabler}%
    %\BRitem[admin]{Tipo}{\btTimer}%
    %\BRitem[admin]{Tipo}{\btExecutive}%
        
    \BRitem[admin]{Nivel}{\blControlling}
    %\BRitem[admin]{Nivel}{\blInfluencing}
    
    \BRitem{Descripción}{%
    Cuando se modifiquen el \refElem{xp-scheme-settings} o la \refElem{levelXP} de las
    \refElem{xp-scheme-settings}
    }

    \BRitem{Ejemplo positivo}{\hfill\par%
        \begin{itemize}
        \item ...
        \end{itemize}
    }

    \BRitem{Ejemplo negativo}{\hfill\par%
        \begin{itemize}
        \item ...
        \end{itemize}
    }% 
    
 \end{BusinessRule}
 % Tipos de incremento
    %\begin{BusinessRule}[%
Autor/Daniel Isai Ortega Zúñiga,%
Version/0.1,%
Estado/revision]%
%
{BR-E04}{Calculo de experiencia del nivel con incremento porcentual}
    \BRitem[control]{Revisor}{Sin asignar.}

 \BRsection[control]{Atributos}
    
    \BRitem[admin]{Clase}{\bcCondition}%
    %\BRitem[admin]{Clase}{\bcIntegridad}%
    %\BRitem[admin]{Clase}{\bcAutorizacion}%
    %\BRitem[admin]{Clase}{\bcDerivacion}%
        
    \BRitem[admin]{Tipo}{\btEnabler}%
    %\BRitem[admin]{Tipo}{\btTimer}%
    %\BRitem[admin]{Tipo}{\btExecutive}%
        
    \BRitem[admin]{Nivel}{\blControlling}
    %\BRitem[admin]{Nivel}{\blInfluencing}
    
    \BRitem{Descripción}{%
        El calculo para obtener la experiencia del nivel $i$ uando el tipo de
        incremento es porcentual está dado por la siguiente fórmula: Sea {\it exp()}
        la función que optiene la experiencia de un nivel en específico, sea tambien
        $i$ el nivel del cual se calcula la experiencia, sea $inc$ el factor de
        incremento de nivel a nivel, y finalmente sea $round()$ una función de
        redondeo a números enteros, entonces:

            $$ exp(i) = round( exp(1) * (inc)^{(i-1)})$$
    }

%   \BRitem{Sentencia}{%
%       Si $fecha$ 
%   }%

    \BRitem{Ejemplo positivo}{\hfill\par%
        \begin{itemize}
        \item La experiencia requerida para superar el nivel 1 es de 2000 puntos y el
              factor de incremento entre los niveles es 1.1, entonces la experiencia
              requerida para pasar el nivel 5 es de 2928 puntos.
        \end{itemize}
    }

    \BRitem{Ejemplo negativo}{\hfill\par%
        \begin{itemize}
        \item La experinecia requerida para superar el nivel 1 es de 2000 puntos y el
              factor de incremento entre los niveles es 1.1, entonces la experiencia
              requerida para pasar el nivel 5 es de 2300 puntos.
        \end{itemize}
    }% 
    
\end{BusinessRule}
 % Incremento porcentual
    %\begin{BusinessRule}[%
Autor/Daniel Isai Ortega Zúñiga,%
Version/0.1,%
Estado/revision]%
%
{BR-E05}{Cálculo de experiencia del nivel con incremento linea} % Cuando están iniciados
    \BRitem[control]{Revisor}{Sin asignar.}

 \BRsection[control]{Atributos}
    
    \BRitem[admin]{Clase}{\bcCondition}%
    %\BRitem[admin]{Clase}{\bcIntegridad}%
    %\BRitem[admin]{Clase}{\bcAutorizacion}%
    %\BRitem[admin]{Clase}{\bcDerivacion}%
        
    \BRitem[admin]{Tipo}{\btEnabler}%
    %\BRitem[admin]{Tipo}{\btTimer}%
    %\BRitem[admin]{Tipo}{\btExecutive}%
        
    \BRitem[admin]{Nivel}{\blControlling}
    %\BRitem[admin]{Nivel}{\blInfluencing}
    
    \BRitem{Descripción}{%
    }

%   \BRitem{Sentencia}{%
%       Si $fecha$ 
%   }%

    \BRitem{Ejemplo positivo}{\hfill\par%
        \begin{itemize}
        \item ...
        \end{itemize}
    }

    \BRitem{Ejemplo negativo}{\hfill\par%
        \begin{itemize}
        \item ...
        \end{itemize}
    }% 
    
\end{BusinessRule}
 % Incremento lineal
    %\begin{BusinessRule}[%
Autor/Daniel Isai Ortega Zúñiga,%
Version/0.1,%
Estado/revision]%
%
{BR-E05}{Cálculo de experiencia del nivel con incremento linea} % Cuando están iniciados
    \BRitem[control]{Revisor}{Sin asignar.}

 \BRsection[control]{Atributos}
    
    \BRitem[admin]{Clase}{\bcCondition}%
    %\BRitem[admin]{Clase}{\bcIntegridad}%
    %\BRitem[admin]{Clase}{\bcAutorizacion}%
    %\BRitem[admin]{Clase}{\bcDerivacion}%
        
    \BRitem[admin]{Tipo}{\btEnabler}%
    %\BRitem[admin]{Tipo}{\btTimer}%
    %\BRitem[admin]{Tipo}{\btExecutive}%
        
    \BRitem[admin]{Nivel}{\blControlling}
    %\BRitem[admin]{Nivel}{\blInfluencing}
    
    \BRitem{Descripción}{%
    }

%   \BRitem{Sentencia}{%
%       Si $fecha$ 
%   }%

    \BRitem{Ejemplo positivo}{\hfill\par%
        \begin{itemize}
        \item ...
        \end{itemize}
    }

    \BRitem{Ejemplo negativo}{\hfill\par%
        \begin{itemize}
        \item ...
        \end{itemize}
    }% 
    
\end{BusinessRule}
 % Eliminación de cursos gamificados
    %\begin{BusinessRule}[%
Autor/El Despistado,%
Version/0.1,%
Estado/edicion]%
%
{BR-E07}{Valores iniciales de experiencia}

     \BRitem[control]{Revisada por}{Pendiente.}

 \BRsection[control]{Atributos}
    % Clases: \bcCondition, \bcIntegridad, \bcAutorization o \bcDerivation
    % Tipos: \btEnabler, \btTimer o \btExecutive
    % Niveles: \blControlling o \blInfluencing.
    
    \BRitem[admin]{Clase}{\bcIntegridad}%
        
    \BRitem[admin]{Tipo}{\btTimer}%
        
    \BRitem[admin]{Nivel}{\blControlling}
    
    \BRitem{Descripción}{%
        Cuando un \refElem{xp-user} es creado este debe de empezar a ganar puntos
        de experiencia a partir del \refElem{xp-user.level} uno, tenido cero puntos 
        de experiencia en la \refElem{xp-user.levelxp} y \refElem{xp-user.xp}. Ningun
        usuario puede comenzar con valores distintos a los indicados anteriomente.
    }

%   \BRitem{Sentencia}{%
%       Si $fecha$ 
%   }%

    \BRitem{Ejemplo positivo}{\hfill\par%
        \begin{itemize}
        \item ...
        \end{itemize}
    }

    \BRitem{Ejemplo negativo}{\hfill\par%
        \begin{itemize}
        \item ...
        \end{itemize}
    }
    
 \end{BusinessRule}
 % Valores iniciales de comperiencia
    %\begin{BusinessRule}[%
Autor/Daniel Isai Ortega Zúñiga,%
Version/0.1,%
Estado/edicion]%
%
{BR-E08}{Valores iniciales de experiencia de un curso}

     \BRitem[control]{Revisada por}{Pendiente.}

 \BRsection[control]{Atributos}
    % Clases: \bcCondition, \bcIntegridad, \bcAutorization o \bcDerivation
    % Tipos: \btEnabler, \btTimer o \btExecutive
    % Niveles: \blControlling o \blInfluencing.

    \BRitem[admin]{Clase}{\bcIntegridad}%

    \BRitem[admin]{Tipo}{\btTimer}%

    \BRitem[admin]{Nivel}{\blControlling}

    \BRitem{Descripción}{%
        Cuando un \refElem{xp-course} es creado la \refElem[experiencia total del curso]%
        {xp-scheme-settings.courseXP} de ser dividida uniformemente entre las
        \refElem[secciones del curso gamificado]{xp-course-section}. Si la división del
        total de experiencia entre el número de secciones genera un residuo entonces este
        se deberá agregan a la última sección del curso.
    }

%   \BRitem{Sentencia}{%
%       Si $fecha$
%   }%

    \BRitem{Ejemplo positivo}{\hfill\par%
        \begin{itemize}
        \item ...
        \end{itemize}
    }

    \BRitem{Ejemplo negativo}{\hfill\par%
        \begin{itemize}
        \item ...
        \end{itemize}
    }

 \end{BusinessRule}
 % Valores iniciales de experiencia del curso

    % INPUT: Cursos Igualitarios.
    % INPUT: Otorgar experiencia
    % INPUT: Administración de experiencia en el curso

\clearpage
\subsubsection{Casos de uso} % ============================================================

 En este apartado se especifican todos los casos de usos contemplados para el financiero,
 para cada caso de uso se especifica su tabla de atributos la cual indica que casos
 de prueba deberán ejecutarse correctamente para corroborar la completitud del caso de uso.

\subsubsection*{Diagrama de casos de uso}

 En la figura \ref{financiero:usecases} se detalla el diagrama de casos de uso correspondiente al módulo
financiero. Los casos de uso de moodle (en color blanco) son modelados como casos de uso
 abstractos, mientras que los casos de uso del módulo de competencia son diferenciados por el
 color azul, en total el desarrollo de este módulo consiste en 3 casos de uso.

    \addfigure{0.6}{modulos/finan/diagrams/UseCases}{financiero:usecases}{%
        Diagrama de casos de uso del módulo financiero}

 \noindent
 Debido a que los plugins a desarrollar son elementos opcionales para Moodle, solo se puede
 acceder a los casos de uso del módulo financiero a través de puntos de extensión de los
 casos de uso de moodle. Por otra parte los casos de uso que serán documentados en esta sección
 serán los del módulo financiero, debido a que Moodle proporciona en su página oficial guías,
 instructivos y documentación de las funcionalidades que brinda.


%\begin{comment}



%\subsubsection{Mensajes}
%
%    \begin{mensaje2}{MSG-F01}{Compra exitosa}{Operación exitosa}
%        \item[Redacción:] ¡Objeto [NombreDelObjeto] comprado!
%        \item[Parámetros:]
%        \begin{Citemize}
%            \item NombreDelObjeto: \refElem(tienda-gmdl-objeto.nombre).
%        \end{Citemize}
%       \item[Ejemplo:] ¡Objeto compañero comprado!
%    \end{mensaje2}
%
%    \begin{mensaje2}{MSG-F02}{No se tienen suficientes monedas}{Operación exitosa}
%        \item[Redacción:] ¡Oh no!, No tienes suficiente dinero para comprar el objeto [NombreDelObjeto]!
%        \item[Parámetros:]
%        \begin{Citemize}
%            \item NombreDelObjeto: \refElem(tienda-gmdl-objeto.nombre).
%        \end{Citemize}
%       \item[Ejemplo:] ¡Objeto compañero comprado!
%    \end{mensaje2}
%
%    \begin{mensaje2}{MSG-F03}{Objeto ya adquirido}{Mensaje de advertencia}
%        \item[Redacción:] ¡Ya tenías desbloqueado el objeto [NombreDelObjeto]!
%        \item[Parámetros:]
%        \begin{Citemize}
%            \item NombreDelObjeto: \refElem(tienda-gmdl-objeto.nombre).
%        \end{Citemize}
%       \item[Ejemplo:] ¡Objeto compañero comprado!
%    \end{mensaje2}
%
%    \begin{mensaje2}{MSG-F04}{Campo erróneo, monedas}{Mensaje de error}
%        \item[Redacción:] Las monedas deben ser un número natural.
%    \end{mensaje2}

%\end{comment}

    % MODULO FINANCIERO


% \ucstEnEdicion     Al terminar una revisión/aprobación con observaciones
%                    y al inicio del CU.
%
% \ucstEnRevision    Al terminar la edición del CU (version += 0.1).
% \ucstEnAprobacion  Al pasar la revision sin observaciones.
% \ucstAprobado      Al ser aprobado por el usuario (version += 1.0)

\begin{UseCase}[%
Autor/David Flores Casanova,%
Version/0.1,%
Estado/\ucstEnRevision]%
%
{CU-F01}{Modificar esquema financiero }{%
%
    Permite al \refElem{aAdministrador} modificar la cantidad de monedas que otorga cada evento, 
    así como, si desea que el esquema financiero esté activado o no.
    Este caso de uso es una extensión del caso de uso {\it Entrar a administración de complementos} que es propio de moodle }

	\UCitem[control]{Revisor}{ Sin asignar }
	\UCitem[control]{Último cambio}{ 17/NOV/19 }

 \UCsection{Atributos}

    \UCitem{Actor(es)}{%
        \refElem{aAdministrador}
    }

	\UCitems{Propósito}{%
        \Titem El actor desea cambiar la cantidad de monedas que da un evento o quiere activar  o desactivar el esquema financiero.
	}

	\UCitem{Entradas}{\imprimeUC{entrada}}

	\UCitems{Origen}{%
        \Titem Mouse
	}

	\UCitem{Salidas}{
        \imprimeUC{salida}}

	\UCitem{Destino}{%
		\refElem{IU-F01}
	}

	\UCitems{Precondiciones}{%
        \Titem El complemento del módulo financiero debe estar instalado.
	}

	\UCitems{Postcondiciones}{%
        \Titem Se guardará la nueva configuración del actor.
	}

	\UCitem{Reglas de negocio}{\imprimeUC{regla}}

	\UCitems{Errores}{%
        \Titem Alguno de los campos fueron ingresados de manera erronea.
	}

 \UCsection[design]{Datos de Diseño}

	\UCitems[design]{Casos de Prueba}{%
	}

 \UCsection[admin]{Datos de Administración de Requerimiento}

	\UCitem[admin]{Observaciones}{}

\end{UseCase}

\subsubsection{Trayectorias del caso de uso}

\begin{UCtrayectoria}%
%

    \Actor Selecciona dando clic a la opción \textbf{Gamedle: Módulo financiero} en la pantalla \refElem{IU-M11}. \refTray{A}
    \Sistema Redirige a la pantalla \refElem{IU-F01}.
    \Actor Modifica las opciones que desea. \refTray{B}
    \label{CU-F01-hacer-cambios}
    \Actor Presiona el botón \textbf{Guardar cambios}. 
    \label{CU-F01-guardar-cambios}
    \Sistema Valida que los valores ingresados sean correctos. \refTray{C}
    \Sistema Guarda la nueva configuración.

\end{UCtrayectoria}

\begin{UCtrayectoriaA}[Fin del caso de uso]%
  {A}{El complemento del  módulo financiero no se encuentra instalado}

  \Actor No encuentra la opción en la pantalla  \textbf{Gamedle: Módulo financiero}, porque el sistema no la generó.

\end{UCtrayectoriaA}

\begin{UCtrayectoriaA}%
{B}{El actor desea modificar la configuración financiera}
    \Actor Selecciona dando clic a la opción \textbf{Gamedle: Módulo financiero} en la pantalla \refElem{IU-F01}.
    \Sistema Redirige a la pantalla \refElem{IU-F02}.
    \Actor Modifica las opciones que desea. 
    \item Se regresa al paso \ref{CU-F01-guardar-cambios} de la trayectoria principal.

\end{UCtrayectoriaA}

\begin{UCtrayectoriaA}%
{B}{El actor desea modificar la configuración financiera}
    \Actor Selecciona dando clic a la opción \textbf{Gamedle: Módulo financiero} en la pantalla \refElem{IU-F01}.
    \Sistema Redirige a la pantalla \refElem{IU-F02}.
    \Actor Modifica las opciones que desea. 
    \item Se regresa al paso \ref{CU-F01-guardar-cambios} de la trayectoria principal.

\end{UCtrayectoriaA}


\begin{UCtrayectoriaA}%
{C}{Se han ingresaron datos erróneos}
    \Sistema Muestra el mensaje \refElem{MSG-F04} en cada campo con valores erróneos.. 
    \item Se regresa al paso \ref{CU-F01-hacer-cambios}.

\end{UCtrayectoriaA}   % Modificar esquema financiero

% \ucstEnEdicion     Al terminar una revisión/aprobación con observaciones
%                    y al inicio del CU.
%
% \ucstEnRevision    Al terminar la edición del CU (version += 0.1).
% \ucstEnAprobacion  Al pasar la revision sin observaciones.
% \ucstAprobado      Al ser aprobado por el usuario (version += 1.0)

\begin{UseCase}[%
Autor/David Flores Casanova,%
Version/0.1,%
Estado/\ucstEnRevision]%
%
{CU-F02}{Instalar esquema financiero }{%
%
    El \refElem{aAdministrador} instala el complemento como lo especifica moodle y al instalarlo moodle le permite
    configurar las opciones una vez instalado.
    Este caso de uso es una extensión del caso de uso {\it Instalar complemento.}}

	\UCitem[control]{Revisor}{ Sin asignar }
	\UCitem[control]{Último cambio}{ 17/NOV/19 }

 \UCsection{Atributos}

    \UCitem{Actor(es)}{%
        \refElem{aAdministrador}
    }

	\UCitems{Propósito}{%
        \Titem El actor desea utilizar las funciones que brinda el módulo financiero .
	}

	\UCitem{Entradas}{\imprimeUC{entrada}}

	\UCitems{Origen}{%
        \Titem Mouse
	}

	\UCitem{Salidas}{
        \imprimeUC{salida}}

	\UCitem{Destino}{%
		\refElem{IU-F01}
	}

	\UCitems{Precondiciones}{%
        \Titem El complemento del módulo financiero no debe de estar instalado.
	}

	\UCitems{Postcondiciones}{%
        \Titem El módulo financiero ahora estará funcionando en la plataforma del actor.
        \Titem Se guardará la configuración del actor.
	}

	\UCitem{Reglas de negocio}{\imprimeUC{regla}}

	\UCitems{Errores}{%
	}

 \UCsection[design]{Datos de Diseño}

	\UCitems[design]{Casos de Prueba}{%
        \Titem \refElem{CPC-F02-1}
        \Titem \refElem{CPI-F02-2}
	}

 \UCsection[admin]{Datos de Administración de Requerimiento}

	\UCitem[admin]{Observaciones}{}

\end{UseCase}

\subsubsection{Trayectorias del caso de uso}

\begin{UCtrayectoria}%
%

    \Actor Modifica si desea que el módulo financiero esté activado o no en la plataforma, usando pantalla \refElem{IU-F02}.
    \Actor Presiona el botón \textbf{Guardar cambios}.
    \label{CU-F01-guardar-cambios}
    \Sistema Redirige a la pantalla \refElem{IU-F01}.

\end{UCtrayectoria}
   % Instalar esquema financiero

% \ucstEnEdicion     Al terminar una revisión/aprobación con observaciones
%                    y al inicio del CU.
%
% \ucstEnRevision    Al terminar la edición del CU (version += 0.1).
% \ucstEnAprobacion  Al pasar la revision sin observaciones.
% \ucstAprobado      Al ser aprobado por el usuario (version += 1.0)

\begin{UseCase}[%
Autor/David Flores Casanova,%
Version/0.1,%
Estado/\ucstEnRevision]%
%
{CU-F03}{Comprar objeto}{%
%
Permite al usuario (Ya sea un \refElem{aProfesor}, un \refElem{aAdministrador} o un \refElem{aEstudiante})
 de moodle adquirir un objeto para la personalización de su perfil utilizando sus monedas disponibles.
 Este caso de uso es una extensión del caso de uso {\it \refElem{CU-P01}}.}

	\UCitem[control]{Revisor}{ Sin asignar }
	\UCitem[control]{Último cambio}{ 17/NOV/19 }

 \UCsection{Atributos}

    \UCitem{Actor(es)}{%
        \refElem{aProfesor},
        \refElem{aAdministrador},
        \refElem{aEstudiante}
    }

	\UCitems{Propósito}{%
        \Titem El usuario quiere saber el estado actual de su perfil, así como las monedas que tiene disponibles.
	}

	\UCitem{Entradas}{\imprimeUC{entrada}}

	\UCitems{Origen}{%
        \Titem Mouse
	}

	\UCitem{Salidas}{
        \imprimeUC{salida}
        \Titem ''¡Objeto [NombreDelObjeto] comprado!''%\refElem{MSG-F01}
        \Titem ''¡Oh no!, No tienes suficiente dinero para comprar el objeto [NombreDelObjeto]!''%\refElem{MSG-F02}
        \Titem ''¡Ya tenías desbloqueado el objeto [NombreDelObjeto]!''%\refElem{MSG-F03}

        }

	\UCitem{Destino}{%
		\refElem{IU-P01}
	}

	\UCitems{Precondiciones}{%
        \Titem El usuario debió de haber ejecutado el \refElem{CU-P01}.
        \Titem El usuario debe contener las monedas suficientes para comprar el objeto.
        \Titem El objeto que se quiere  comprar no debe estar ya adquirido.
	}

	\UCitems{Postcondiciones}{%
        \Titem El usuario tendrá desbloqueado el objeto para usarlo.
        \Titem El usuario se le restarán las \refElem{xp-user.monedas-plata} del objeto que adquirió.
	}

	\UCitem{Reglas de negocio}{\imprimeUC{regla}}

	\UCitems{Errores}{%
        \Titem El usuario no cuenta con las suficientes monedas para adquirir el objeto.
	}

 \UCsection[design]{Datos de Diseño}

	\UCitems[design]{Casos de Prueba}{%
        \Titem \refElem{CPC-F03-1}
        \Titem \refElem{CPI-F03-2}
        \Titem \refElem{CPI-F03-3}
	}

 \UCsection[admin]{Datos de Administración de Requerimiento}

	\UCitem[admin]{Observaciones}{}

\end{UseCase}

\subsubsection{Trayectorias del caso de uso}

\begin{UCtrayectoria}%
%

    \Actor Selecciona dando clic a la opción \textbf{Comprar} (Indicado por el ícono de las monedas \IUMonedas{}) del objeto que desea comprar en la pantalla \refElem{IU-P01}.
    \Sistema Comprueba que el actor cuenta con suficientes \refElem{xp-user.monedas-plata} para comprar el objeto. \refTray{A}
    \Sistema Comprueba que el actor no cuente con ese objeto todavía. \refTray{B}
    \Sistema Desbloquea el objeto seleccionado para el actor, guardándolo en \refElem{tienda-gmdl-objeto-desbloqueado}.
    \Sistema Le resta al actor las \refElem{xp-user.monedas-plata} dependiendo el \refElem{tienda-gmdl-rareza-objeto.costo-adquisicion} del objeto seleccionado.
    \Sistema Muestra el mensaje ''¡Objeto [NombreDelObjeto] comprado!''%\refElem{MSG-P01}

\end{UCtrayectoria}

\begin{UCtrayectoriaA}[Fin del caso de uso]%
  {A}{El actor no cuenta con las monedas necesarias }

  \Sistema Muestra el mensaje  ''¡Oh no!, No tienes suficiente dinero para comprar el objeto [NombreDelObjeto]!''%\refElem{MSG-F02}.

\end{UCtrayectoriaA}

\begin{UCtrayectoriaA}[Fin del caso de uso]%
{B}{El actor ya tiene comprado el objeto}

    \Sistema Muestra el mensaje ''¡Ya tenías desbloqueado el objeto [NombreDelObjeto]!''%\refElem{MSG-F03}

\end{UCtrayectoriaA}
   % Comprar objeto

% =========================================================
\clearpage
\subsection{Diseño}

\subsubsection{Interfaces del módulo de competencia}

    \subsubsection{IU-M11 Administración de complementos}


    \IUfig{1}{modulos/moodle/IU/plugins_administrados}{IU-M11}{Administración de complementos}

\subsubsection{Elementos Relevantes}

    \begin{itemize}
        \item {\bf Configuración general}
            Opción que redirige a la pantalla \refElem{IU-F01}.
        \item {\bf Configuración financiera}
            Opción que redirige a la pantalla \refElem{IU-F02}.
    \end{itemize}


\clearpage
  % Configuraciones de plugins
    
\subsubsection{IU-F01: Configuración general del módulo financiero}

 Pantalla que se usa para configurar los elementos generales del módulo financiero.

    \IUfig{1}{modulos/finan/IU/configuracion_general}{IU-F01}{%
       Configuración general del módulo financiero}
\subsubsection{Elementos Relevantes}

    \begin{itemize}
        \item {\bf Activar complemento financiero}
            Esta opción activa y desactiva el módulo financiero.
    \end{itemize}

\subsubsection{Acciones relevantes}

    \begin{itemize}
        \item {\bf Configuración financiera}
            Esta opción redirige a la pantalla \refElem{IU-F02}.
        \item {\bf Guardar cambios}
            Esta opción guarda la configuración actual.
    \end{itemize}

\clearpage
  % Configuraciones Generales
    
\subsubsection{IU-F02: Configuración financiera}

 Pantalla que se utiliza para configurar cuántas monedas se dan, en qué eventos se dan y cuánto valen las monedas de oro respecto a las de plata.

    \IUfig{1}{modulos/finan/IU/configuracion_financiera}{IU-F02}{%
       Configuración financiera}
\subsubsection{Elementos Relevantes}

    \begin{itemize}
        \item {\bf Monedas de plata a oro}
            Esta opción especifica cuántas monedas de plata equivalen a una de oro.
        \item {\bf Evento competencia uno contra uno}
            Esta opción especifica cuántas monedas de plata se entregan a un usuario al haber derrotado a otro en las competencias uno contra uno.
        \item {\bf Evento competencia uno contra sistema}
            Esta opción especifica cuántas monedas de plata se entregan a un usuario al haber derrotado al sistema.
        \item {\bf Evento pregunta diaria}
            Esta opción especifica cuántas monedas de plata se entregan a un usuario al haber respondido correctamente la pregunta diaria.
    \end{itemize}

\subsubsection{Acciones relevantes}

    \begin{itemize}
        \item {\bf Activar evento competencia uno contra uno}
            Esta opción especifica si se dan monedas de plata a un usuario al haber derrotado a otro en las competencias uno contra uno.
        \item {\bf Activar evento competencia uno contra sistema}
            Esta opción especifica si se dan monedas de plata a un usuario al haber derrotado al sistema.
        \item {\bf Activar evento pregunta diaria}
            Esta opción especifica si se dan monedas de plata a un usuario al haber respondido correctamente la pregunta diaria.
    \end{itemize}

\clearpage
  % Configuraciones de finanzas


%

\subsubsection{Diseño de complementos}



A continuación se presenta cómo los submódulos de competencia
se implmenetan en moodle.\\


\noindent Resumiendo el módulo de competencia tiene 2 actividades establecidas, llamadas; 
competencia uno contra uno y competencia uno contra sistema. 
Ambas actividades deben aparecer dentro de la lista de actividades de moodle. Para ello 
moodle cuenta con un tipo de complemento que se denomina \textbf{'mod'}, este tipo de complemento al ser instalado 
en una plataforma de moodle, crea una nueva opción a la lista de actividades.\\

\noindent Tomando en consideración lo anterior y que existe el complemento gamedlemaster, se presenta en la figura \ref{fig:diseno-comp-comp}
los complementos contemplados y las dependencias entre los mismos.


    \addfigure{1}{modulos/comp/diagrams/diseno_complementos}{fig:diseno-comp-comp}{Implementación del modulo de competencia}


Cada complemento en la figura \ref{fig:diseno-comp-comp} está representado con una cadena que sigue el formato 'tipo\_de\_complemento:nombre\_de\_complemento'. Los tipos de complemento son;
\begin{itemize}
    \item \textbf{mod} - Este complemento permite crear una actividad que aparece en la lista de actividades a agregar a un curso.
    \item \textbf{local} -  Este complemento moodle lo iterpreta como un comdín, el cual puede ser usado para múltiples propósitos relacionados con la gestión de la información.
    \item \textbf{block} - Este complemento permite desplegar un cuadro en la mayoría de las páginas de moodle, el cuál puede representar valores 
\end{itemize}

La función de cada uno de los complementos presentados en la figura \ref{fig:diseno-comp-comp} son:


\begin{itemize}
    \item \textbf{gmcompcpu} Definir la competencia uno contra sistema.
    \item \textbf{gmcompvs} Definir la competencia uno contra uno.
    \item \textbf{gmcs} Entregar las monedas por ganar cada una de las competencias anteriores.
\end{itemize}

El complemento de tipo  \textbf{'mod'} tiene un requerimiento en su nombre, el cual es; 'El nombre del complemento a instalar debe ser igual a un nombre
de una de las tablas en la base de datos'. Debido a que moodle no soporta nombres de complementos que contengan quiones bajos, el
nombre de la tabla ya no puede llevarlos.






\subsection{Pruebas}

    %
\TestCase{CPC-C01}{Crear nuevas instancias de la actividad de competencia 1 contra 1}

    %
\TestCase{CPC-E02}{Realizar configuraciones del módulo de experiencia}

    %
\TestCase{CPC-E02-1}{Realizar configuración de visualización de niveles}

    %
\TestCase{CPI-E02-1a}{Realizar configuraciones visuales con todos los datos erroneos}

    %
\TestCase{CPI-E02-1b}{Configuraciones visuales con formato y nombre de imagen inválidos}


    %
\TestCase{CPC-E02-2a}{Realizar configuraciones del sistema de experiencia}

    %
\TestCase{CPC-E02-2b}{Realizar configuraciones con cursos iniciados}

    %
\TestCase{CPC-E02-2c}{Realizar configuraciones del sistema de experiencia con estudiantes con experiencia establecida}

    %
\TestCase{CPI-E02-2}{Realizar configuraciones del sistema de experiencia con datos inválidos}


    %
\TestCase{CPC-E02-3}{Realizar configuraciones del sistema de experiencia con datos correctoss}

    %
\TestCase{CPI-E02-3}{Realizar configuraciones del eventos con datos inválidos}


    %
\TestCase{CPC-E03}{Desinstalar plugins del módulo de experiencia}


    %
\TestCase{CPC-E04}{Crear un curso gamificado}

    %
\TestCase{CPI-E04}{Crear un curso gamificado con la experiencia deshabilitada}


    %
\TestCase{CPC-E05}{Eliminar un curso gamificado sin estudiantes inscritos}

    %
\TestCase{CPC-E05a}{Eliminar un curso gamificado con alumnos inscritos}


    %
\TestCase{CPC-E12}{Crear un usuario gamificado (administrador)}

    %
\TestCase{CPC-E12a}{Crear un usuario gamificado mediante el auto-registro}


    %
\TestCase{CPC-E13}{Eliminar usuario gamificado}

