
A continuación se enlistan los casos de prueba identificados
correspondientes a cada uno de los casos de uso especificados. Los casos
de prueba listados a continuación son de dos tipos, los correctos e
incorrectos identificados por los prefijos CPC y CPI respectivamente.

\subsubsection{\refElem{CU-F01}}

\begin{itemize}
  \TestCase{CPC-F01-1}{Modificar el esquema financiero}
  \TestCase{CPI-F01-2}{Intentar modifcar el esquema financiero sin tener isntalado el complemento}
  \TestCase{CPI-F01-3}{Ingresar valores incorrectos}
\end{itemize}



\subsubsection{\refElem{CU-F02}}

\begin{itemize}
  \TestCase{CPC-F02-1}{Instalar esquema financiero}
  \TestCase{CPI-F02-2}{Instalar esquema financiero sin tener el complemento gamedlemaster}
\end{itemize}


\subsubsection{\refElem{CU-F03}}

\begin{itemize}
  \TestCase{CPC-F03-1}{Comprar objeto}
  \TestCase{CPI-F03-2}{Intentar comprar objeto sin tener las monedas requeridas}
  \TestCase{CPI-F03-3}{Intentar comprar objeto ya adquirido}
\end{itemize}
