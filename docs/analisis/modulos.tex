
\section{Módulos identificados}
\label{analisis:modulos}

 Como se comentó en el capítulo de \nameref{ch:introduccion}, nuestro
 objetivo es crear una herramienta que permita implementar los principios de
 gamificación dentro de una plataforma web de aprendizaje. Con base en el objetivo
 se identificarón seís módulos principales a desarrollar, cada uno con sus
 respectivos submódulos (ver figura \ref{fig:modulos}).\\

 \noindent
 Los módulos identificados permitiran extender la funcionalidad de la plataforma web
 de aprendizaje Moodle, la cual fue elegida con base en la investiación realizada en
 la seccion \hyperrefx{sec:sistemasaprendizaje}. A continuación se describe cada uno de los módulos y
 submódulos identificados.\\

    \addfigure{1}{analisis/diagrams/modulosTT}{fig:modulos}{Diseño modular del sistema}


 % SOLO EXPLICAR QUE LOS MÓDULOS FUERON PLANEADOS PARA
 % BRINDAR SOPORTE A LOS PRINCIPIOS A,B,C,  Y TENER
 % CUIDADO CON EL
 %
 %   PRINCIPIO 3: Descubrimimento y Retroalimentación


 % FIND-QUOTE: Education must by itself enough engaging


\subsection{Módulo de Experiencia}
 \newcommand{\itemx}[1]{\item{\bf\color{primary}#1}}

 Este módulo brindará un mecanismo que permite a los usuarios medir su progreso
 como puntos de experiencia a nivel plataforma, además define la forma en que se
 obtendrán los puntos de experiencia, la forma en que se visualizará la información,
 el número requerido para superar cada nivel y la barra de progreso del nivel
 actual.

\subsubsection{Esquema de experiencia.}

 El esquema de experiencia es la configuración sobre cómo funciona el sistema
 de puntos de experiencia, incluyendo la cantidad de experiencia que tiene
 cada nivel, el tipo de incremento en los puntos de experiencia nivel a nivel,
 las restricciones sobre la experiencia y la forma en que se otorgarán los puntos.

\subsubsection{Niveles.}

 Es el mecanismo que permite mostrarle a los alumnos el progreso que han tenido
 a nivel plataforma mediante el nivel y los puntos de experiencia obtenidos en 
 los cursos, además contiene la configuración para establecer el cómo se vera el
 nivel y la experiencia obtenida de dicho nivel.

 % Los principios de gamificación a los cuales permite brindarles soporte son:
 % \principioII
 % \principioVI

\subsection{Módulo de Recompensa}

\subsubsection{Marcadores}
\subsubsection{Logros}
\subsubsection{Advertencias}

\subsection{Módulo de Personalización}

\subsubsection{Items de personalización}
\subsubsection{Personalización}
\subsubsection{Narrativa}

\subsection{Módulo Financiero}

\subsubsection{Esquema Financiero}
\subsubsection{Tienda}
\subsubsection{Loot Boxes}

\subsection{Módulo de Competencias}

\subsubsection{Retos}
\subsubsection{Retos al sistema (1 vs CPU)}
\subsubsection{Torneo}
\subsubsection{Apuestas}
\subsubsection{Torneo con ganancias}
\subsubsection{Competencias Poker}

\subsection{Módulo de Seguimiento}

\subsubsection{Preguntas diarias}
\subsubsection{Barras de progreso}
 

% =========================================
%  N O T E S
% =========================================

% DAN:
%   I THINK, IT'S A GOOD TO REMOVE THE SYSTEM REQUIREMENTS AND
%   USE THE 19 ''SUBMODULES'' AS THE PRODUCT BACKLOG ITEMS.
%

 % Los requerimientos presentes en el Product Backlog fueron agrupados en 6 módulos (ver
 % figura \ref{fig:modulos}): el módulo de competencia, módulo financiero, módulo de personalización,
 % módulo de seguimiento, módulo de experiencia y módulo de recompensa. Fueron identificados
 % 19 submódulos distribuidos en los módulos anteriormente mencionados.\\

 % \noindent Como se comentó en la sección \hyperrefx{subsec:plugins}, la manera más recomendable
 % para extender las funcionalidades de moodle es desarrollando o incluyendo plugins, debido a
 % esta razón, el análisis y diseño es realizado tomando en consideración de que se trabajará
 % desarrollando plugins.
 
