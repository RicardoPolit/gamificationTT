
\section{Módulos identificados}
\label{analisis:modulos}

 Como se comentó en el capítulo de \nameref{ch:introduccion}, nuestro
 objetivo es crear una herramienta que permita implementar los principios de
 gamificación dentro de una plataforma web de aprendizaje. Con base en el objetivo
 se identificaron seis módulos principales a desarrollar, cada uno con sus
 respectivos submódulos (ver figura \ref{fig:modulos}).\\

 \noindent
 Los módulos identificados permitirán extender la funcionalidad de la plataforma web
 de aprendizaje Moodle, la cual fue elegida con base en la investigación realizada en
 la sección \hyperrefx{sec:sistemasaprendizaje}. A continuación se describe cada uno de los módulos y
 submódulos identificados.\\

    \addfigure{1}{analisis/diagrams/modulosTT}{fig:modulos}{Diseño modular del sistema}


 % SOLO EXPLICAR QUE LOS MÓDULOS FUERON PLANEADOS PARA
 % BRINDAR SOPORTE A LOS PRINCIPIOS A,B,C,  Y TENER
 % CUIDADO CON EL
 %
 %   PRINCIPIO 3: Descubrimimento y Retroalimentación


 % FIND-QUOTE: Education must by itself enough engaging


\subsection{Módulo de Experiencia}

 Este módulo brindará un mecanismo que permite a los usuarios medir su progreso
 como puntos de experiencia a nivel plataforma, además define la forma en que se
 obtendrán los puntos de experiencia, la forma en que se visualizará la información,
 el número requerido para superar cada nivel y la barra de progreso del nivel
 actual.

\subsubsection{Esquema de experiencia.}

 El esquema de experiencia es la configuración sobre cómo funciona el sistema
 de puntos de experiencia, incluyendo la cantidad de experiencia que tiene
 cada nivel, el tipo de incremento en los puntos de experiencia nivel a nivel,
 las restricciones sobre la experiencia y la forma en que se otorgarán los puntos.

\subsubsection{Niveles.}

 Es el mecanismo que permite mostrarle a los estudiantes el progreso que han tenido
 a nivel plataforma mediante el nivel y los puntos de experiencia obtenidos en
 los cursos, además contiene la configuración para establecer el cómo se verá el
 nivel y la experiencia obtenida de dicho nivel.

 % Los principios de gamificación a los cuales permite brindarles soporte son:
 % \principioII
 % \principioVI
%

\subsection{Módulo de Competencias}

  Este módulo brindará dos opciones de competencia que otorgan al usuario
  la posibilidad de poner en practica sus conocimientos del curso y a su vez interactuar
  con otros usuarios inscritos al curso.

\subsubsection{Competencia uno contra uno}

La competencia uno contra uno permite al usuario desafiar a otro dentro del curso con
el objetivo de ayudarles a practicar los temas que contiene el curso, aportando retroalimentación
del puntaje obtenido en comparación al usuario que desafió.

\subsubsection{Competencia uno contra sistema}

La competencia uno contra sistema permite al usuario poner en practica sus conocimientos y evaluarlos
al desafiar al sistema en alguna de sus dificultades. Exigiendo un nivel más alto de conocimiento al
subir la dificultad.

%\subsection{Módulo de Recompensa}
%
%\subsubsection{Marcadores}
%\subsubsection{Logros}
%\subsubsection{Advertencias}

\subsection{Módulo de Personalización}

Este modulo pondrá a disposición de los usuarios una extensión a su perfil de moodle llamado "perfil gamificado". En el cual se podrá
seleccionar objetos para personalizar su perfil. Esta personalización será visible en los diferentes módulos del sistema que soporten
esta función.

\subsubsection{Objetos de personalización}

Existen diferentes tipos de objetos de personalización así como diferentes rarezas que tienen. Se proponen 3 tipos de objetos:

\begin{itemize}
  \item Imagen de perfil
  \item Tipo de marco
  \item Color de marco
\end{itemize}

Se proponen 4 categorías de rareza para cada tipo de objeto:

\begin{itemize}
  \item Común
  \item Rara
  \item Épica
  \item Legendaria
\end{itemize}

\subsubsection{Personalización}

Entre los objetos propuestos se podrá elegir un objeto por cada tipo para que sea mostrado en los diferentes módulos del sistema. La disponibilidad
o no disponibilidad de objetos a elegir dependerá de los módulos que se encuentren activados en el mismo sistema que el módulo de personalización.

%\subsubsection{Narrativa}

\subsection{Módulo Financiero}

Este módulo brindará un mecanismo que permite a los usuarios obtener monedas al realizar diferentes acciones en el sistema, estas acciones dependen
de los módulos que se encuentren instalados y activados en el mismo sistema que el módulo financiero. Es importante aclarar que es necesario que se
encuentre instalado el módulo de personalización para que tenga relevancia este módulo.

\subsubsection{Esquema Financiero}

Este esquema es la configuración sobre cómo funciona el módulo financiero, esto incluye la posibilidad elegir qué acciones otorgan monedas a los usuarios
y la cantidad que reciben por la acción en concreto.

\subsubsection{Tienda}

La función de la tienda es permitir utilizar las monedas para desbloquear objetos de personalización y así poder seleccionarlos. Lo cual otorga un grado de importancia a los objetos dependiendo de la cantidad necesaria para desbloquearlos.

%\subsubsection{Loot Boxes}

\subsection{Módulo de Seguimiento}

Este módulo permitirá a los usuarios tener un seguimiento de su progreso de mejora en el sistema. Lo cual otorga al usuario una manera de evaluar su rendimiento en el curso.

\subsubsection{Preguntas diarias}

Las preguntas diarias permiten al usuario contestar una pregunta elegida de manera aleatoria de un banco de preguntas seleccionado con anterioridad. Esto brinda la posibilidad de desafiar constantemente los conocimientos del usuario sobre el curso en el que se encuentra inscrito.

\subsubsection{Barras de progreso}

Las barras de progreso brindarán al usuario una manera visual de comparar su progreso y conocer fácilmente la cantidad de acciones que tiene que seguir realizando para llegar a la meta propuesta en dicha barra de progreso. Tomando como ejemplo las preguntas diarias se le mostrará al usuario cuantas preguntas diarias ha contestado y cuantas le faltan por contestar para llegar a la meta.

% =========================================
%  N O T E S
% =========================================

% DAN:
%   I THINK, IT'S A GOOD TO REMOVE THE SYSTEM REQUIREMENTS AND
%   USE THE 19 ''SUBMODULES'' AS THE PRODUCT BACKLOG ITEMS.
%

 % Los requerimientos presentes en el Product Backlog fueron agrupados en 6 módulos (ver
 % figura \ref{fig:modulos}): el módulo de competencia, módulo financiero, módulo de personalización,
 % módulo de seguimiento, módulo de experiencia y módulo de recompensa. Fueron identificados
 % 19 submódulos distribuidos en los módulos anteriormente mencionados.\\

 % \noindent Como se comentó en la sección \hyperrefx{subsec:plugins}, la manera más recomendable
 % para extender las funcionalidades de moodle es desarrollando o incluyendo plugins, debido a
 % esta razón, el análisis y diseño es realizado tomando en consideración de que se trabajará
 % desarrollando plugins.
