
\section{Módulos identificados}
\label{analisis:modulos}

 Como se comentó en el capítulo de \nameref{ch:introduccion}, nuestro
 objetivo es crear una herramienta que permita implementar los principios de
 gamificación dentro de una plataforma web de aprendizaje. Con base en el objetivo
 se identificaron seis módulos principales a desarrollar, cada uno con sus
 respectivos submódulos (ver figura \ref{fig:modulos}).\\

 \noindent
 Los módulos identificados permitirán extender la funcionalidad de la plataforma web
 de aprendizaje Moodle, la cual fue elegida con base en la investigación realizada en
 la sección \hyperrefx{sec:sistemasaprendizaje}. A continuación se describe cada uno de los módulos y
 submódulos identificados.\\

    \addfigure{1}{analisis/diagrams/modulosTT}{fig:modulos}{Diseño modular del sistema}


 % SOLO EXPLICAR QUE LOS MÓDULOS FUERON PLANEADOS PARA
 % BRINDAR SOPORTE A LOS PRINCIPIOS A,B,C,  Y TENER
 % CUIDADO CON EL
 %
 %   PRINCIPIO 3: Descubrimimento y Retroalimentación


 % FIND-QUOTE: Education must by itself enough engaging


\subsection{Módulo de Experiencia}

 Este módulo brindará un mecanismo que permite a los usuarios medir su progreso
 como puntos de experiencia a nivel plataforma, además define la forma en que se
 obtendrán los puntos de experiencia, la forma en que se visualizará la información,
 el número requerido para superar cada nivel y la barra de progreso del nivel
 actual.

\subsubsection{Esquema de experiencia.}

 El esquema de experiencia es la configuración sobre cómo funciona el sistema
 de puntos de experiencia, incluyendo la cantidad de experiencia que tiene
 cada nivel, el tipo de incremento en los puntos de experiencia nivel a nivel,
 las restricciones sobre la experiencia y la forma en que se otorgarán los puntos.

\subsubsection{Niveles.}

 Es el mecanismo que permite mostrarle a los estudiantes el progreso que han tenido
 a nivel plataforma mediante el nivel y los puntos de experiencia obtenidos en
 los cursos, además contiene la configuración para establecer el cómo se verá el
 nivel y la experiencia obtenida de dicho nivel.

 % Los principios de gamificación a los cuales permite brindarles soporte son:
 % \principioII
 % \principioVI
%

\subsection{Módulo de Competencias}

  Este módulo brindará dos opciones para competir que otorgan al usuario
  la posibilidad de poner en práctica sus conocimientos del curso y a su vez interactuar
  con otros usuarios inscritos al curso.

  \noindent El resultado de la competencia se mide respecto a puntos y estos son utilizados para determinar al ganador.

\subsubsection{Competencia uno contra uno}


La competencia uno contra uno es una actividad que se puede llevar acabo en un curso, 
la cual le permite al usuario competir con otro mientras responde un cuestionario.
Esto para permitirle practicar y medir sus conocimientos con el resto de sus compañeros.\\

\subsubsection{Competencia uno contra sistema}

  Esta competencia es una actividad que se puede llevar acabo en un curso, la cual consiste en competir con el sistema al responder un cuestionario.
  El sistema cuenta con varios niveles de dificultad, los cuales son; \\
  \begin{enumerate}
    \item Fácil.
    \item Normal.
    \item Difícil.
    \item Imposible.
  \end{enumerate}
  \noindent Mientras más alto sea el nivel de dificultad, es más probable que el sistema conteste correctamente las preguntas del cuestionario. 
  Así un estudiante, puede poner en práctica sus conocimientos mientras intenta vencer al sistema.

%\subsection{Módulo de Recompensa}
%
%\subsubsection{Marcadores}
%\subsubsection{Logros}
%\subsubsection{Advertencias}

\subsection{Módulo de Personalización}

Este módulo pondrá a disposición de los usuarios una extensión a su perfil de moodle llamado "perfil gamificado". En el cual se podrá
seleccionar objetos para personalizar su perfil. Esta personalización será visible en los diferentes módulos de este proyecto que soporten
esta función.

\subsubsection{Objetos de personalización}


Los objetos se clasifican mediante 2 características, las cuales son; el tipo y su rareza.\\

\noindent Se tienen 3 tipos de objetos:

\begin{multicols}{2}
  \begin{Titemize}
    \Titem Imagen de perfil
    \Titem Tipo de marco
    \Titem Color de marco
  \end{Titemize}

  \noindent Se tienen 4 rarezas:

  \begin{Titemize}
    \Titem Común
    \Titem Rara
    \Titem Épica
    \Titem Legendaria
  \end{Titemize}
\end{multicols}


\subsubsection{Perfil gamificado}

  El usuario tiene la posibilidad de personalizar su perfil gamificado al establecer 
  los objetos que quiere mostrar. Los objetos a disposición del usuario debieron haber sido desbloqueados
  previamente al intercambiar sus monedas por estos. 
  
  \noindent Esto otorga un grado de importancia a los objetos dependiendo de la cantidad de monedas necesarias para desbloquearlos,
  es por ello, que lo que determina la cantidad de monedas requeridas es la rareza del objeto.
%\subsubsection{Narrativa}

\subsection{Módulo Financiero}

Este módulo brindará un mecanismo que otorga la posiblidad a los usuarios de 
obtener monedas al realizar diferentes acciones en el sistema, las cuales son:

\begin{itemize}
  \item Derrotar al sistema en una competencia uno contra sistema.
  \item Derrotar a otro usuario en una competencia uno contra uno (Incluyendo las monedas apostadas).
  \item Contestar correctamente una pregunta diaria.
\end{itemize}

  \noindent El lugar donde se pueden utilizar las monedas adquiridas es en el módulo de personalización.

\subsubsection{Esquema Financiero}

Este esquema es la configuración del módulo financiero, lo cual incluye la posibilidad de habilitar o deshabilitar qué acciones anteriormente descritas
 otorgan monedas a los usuarios y la cantidad de monedas que reciben por esa acción en concreto.

\subsection{Módulo de Seguimiento}

Este módulo permitirá a los usuarios tener un seguimiento de su progreso en el sistema. 
Lo cual otorga al usuario una manera de evaluar su rendimiento.

\subsubsection{Preguntas diarias}

Las preguntas diarias permiten al usuario contestar únicamente una pregunta por día,
la cual es seleccionada de manera aleatoria de un banco de preguntas. 
Esto brinda la posibilidad de desafiar diariamente los conocimientos del usuario sobre el curso.

\subsubsection{Barras de progreso}

Las barras de progreso brindarán al usuario una manera visual de comparar su progreso y conocer
 la cantidad de acciones que tiene que seguir realizando para llegar a la meta establecida en dicha barra de progreso.
Tomando como ejemplo las preguntas diarias se le mostrará al usuario cuantas preguntas ha contestado 
y cuantas le faltan por contestar para llegar a la meta.

% =========================================
%  N O T E S
% =========================================

% DAN:
%   I THINK, IT'S A GOOD TO REMOVE THE SYSTEM REQUIREMENTS AND
%   USE THE 19 ''SUBMODULES'' AS THE PRODUCT BACKLOG ITEMS.
%

 % Los requerimientos presentes en el Product Backlog fueron agrupados en 6 módulos (ver
 % figura \ref{fig:modulos}): el módulo de competencia, módulo financiero, módulo de personalización,
 % módulo de seguimiento, módulo de experiencia y módulo de recompensa. Fueron identificados
 % 19 submódulos distribuidos en los módulos anteriormente mencionados.\\

 % \noindent Como se comentó en la sección \hyperrefx{subsec:plugins}, la manera más recomendable
 % para extender las funcionalidades de moodle es desarrollando o incluyendo plugins, debido a
 % esta razón, el análisis y diseño es realizado tomando en consideración de que se trabajará
 % desarrollando plugins.
