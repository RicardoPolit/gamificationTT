\section{Usuarios}
\label{analisis:usuarios}

 En esta sección se presentan los actores a los que va destinada la propuesta de
 solución, así como sus responsabilidades y perfil recomendado. Posteriormente se
 presentan por cada módulo el conjunto de funcionalidades que se le brindarán a
 cada actor especificando a que módulo pertenecen las funcionalidades.

 % =====================================
 %    A D M I N I S T R A D O R
 % =====================================

    \begin{actor}{aAdmininstrador}{Administrador del sitio}{%
    El adminsitrador del sitio es la persona con mayor jerarquía respecto a los
    permisos y funcionalidades que brinda moodle, desde configurar la visualización
    de la página principal del sitio hasta editar las políticas de seguridad. A
    continuación se describen las responsabilidades y cualidades que debe tener
    un administrador.\\}

    \item[Responsabilidades:] \hfill
        \begin{itemize}
        \item Instalar/desinstalar plugins
        \item configuración general de los plugins
        \end{itemize}

    \item[Perfil:] \hfill
        \begin{itemize}
        \item Contar con experiencia administrando moodle.
        \item Conocimientos de permisos en LAMP (Linux, apache, MySQL and PHP).
        \item Conocimientos básicos de gamificación.\\
        \end{itemize}
    \end{actor}

    \noindent
    Las funcionalidades extras que se le brindarán al adminitrador son las siguientes:

    \begin{quote}
    {\bf Funcionalidades Generales:}
        \begin{itemize}
        \item Elegir que conjunto de funcionalidades que desea instalar
              en la plataforma moodle que administra.
        \item Realizar las configuraciones generales (a nivel plataforma) de las
              funcionalidades de gamificación instaladas.
        \item Creación de usuarios gamificados de forma automática.
        \item Eliminar de moodle las funcionalidades de gamificación que desee.
        \end{itemize}

    {\bf Del módulo de experiencia:}
        \begin{itemize}
        \item Establecer la cantidad de experiencia que brindarán los cursos.
        \item Especificar la experiencia correspondiente al primer nivel.
        \item Elegir el tipo de incremento de la cantidad de experiencia en los niveles.
        \item Especificar el factor o valor de incremento asociado al tipo de incremento.
        \end{itemize}

    % TODO expecificar las funcionalidades de los otros módulos
    \end{quote}

 % =====================================
 %    P R O F E S O R
 % =====================================

    \begin{actor}{aProfesor}{Profesor}{%
    Un profesor, visto desde el punto de vista de este desarrollo, es la persona
    encargada de la creación y administración de los cursos desea impartir a través
    de moodle, por lo tanto además de tener rol de {\it profesor} el cual le permite
    gestionar y añadir contenido a los cursos así como el rol de {\it creador de
    curso} \cite{MoodleRoles} \\}
    % https://docs.moodle.org/all/es/Roles_est%C3%A1ndar

    \item[Responsabilidades:] \hfill
        \begin{itemize}
        \item Creación de los cursos que imparte
        \item Creación del contenido de los cursos que imparte
        \item Gestión del curso
        \item Establecer las formas de incripción en el curso
        \item Configuración de las funcionalidades escogidas para la gamifcación
        \end{itemize}

    \item[Perfil:] \hfill
        \begin{itemize}
        \item Contar con experiencia en moodle en creación y administración de cursos.
        \item Contar con conocimiento básicos de ludificación y/o gamificación.
        \end{itemize}
    \end{actor}

    \noindent
    Las funcionalidades extras que se le brindarán al profesor son las siguientes:

    \begin{quote}
    {\bf Funcionalidades generales:}
        \begin{itemize}
        \item Crear un curso con soporte da gamificación.
        \item Establecer las funcionalidades que desea incluir en cada curso.
        \item Convertir un curso no gamificado a un curso gamificado y viceversa.
        \item Soporte para todas las funcionalidades de administración de un curso
              normal.
        \end{itemize}

    {\bf Del módulo de experiencia:}
        \begin{itemize}
        \item Habilitar/Deshabilitar el soporte para entregar puntos de experiencia
              en su curso.
        \item Establecer la cantidad de experiencia que se le brindará a cada
              sección de su curso.
        \item Repartir la experiencia de forma igualitaria entre las secciones del
              curso.
        \item Establecer si se mostrará por defecto el nivel actual de experiencia
              correspondiente al usuario que está viendo la pantalla.
        \end{itemize}

    % TODO expecificar las funcionalidades de los otros módulos
    \end{quote}

 % =====================================
 %    A L U M N O
 % =====================================

    \begin{actor}{aAlumno}{Alumno}{%
    Un alumno, para este proyecto, es cualquier usuario de moodle que tenga el rol de
    estudiante. Este rol le permite ver los distintos cursos y al inscribirse participar
    en ellos \cite{MoodleRol}. Los distintos módulos de este proyecto fueron pensados para
    brindar soporte a los distintos tipos de usuarios en el ámbito de la gamificación
    propuestos por {\it Richard Bartle} \cite{TiposDeUsuario}\\}
    % https://docs.moodle.org/all/es/Rol_de_estudiante

    \item[Responsabilidades:] \hfill
        \begin{itemize}
        \item Cumplir con las actividades de los cursos
        \item Inscribirse a los cursos correspondientes
        \item Hacer uso correcto de la plataforma
        \end{itemize}

    \item[Perfil:] \hfill
        \begin{itemize}
        \item Haber utilizado como alumno anteriormente
        \end{itemize}
    \end{actor}

    \noindent
    Las funcionalidades extras que se le brindarán al alumno son las siguientes:

    \begin{quote}
    {\bf Del módulo de experiencia:}
        \begin{itemize}
        \item Reconocimiento/notificación al haber superado un nivel.
        \item Visualización del nivel actual en el que se encuentra.
        \item Visualización de los puntos que tiene del nivel actual.
        \end{itemize}
    % TODO expecificar las funcionalidades de los otros módulos
    \end{quote}

    \subsubsection{Tipos de usuarios de gamificación segun Richard Bartle}

    La clasificación de usuarios propuesta por Richard Bartle \cite{TiposDeUsuario}, define
    4 tipos de usuarios: triunfadores, exploradores, socializadores y asesinos, los cuales son
    descritos a continuación:
    
        % TODO Actualizar definición de tipos de usuarios
        \begin{quote}
        \begin{description}
        \item[Triunfadores]
                % son aquellos que les gusta recibir premios, en nuestro caso los logros
                % que tenemos contemplados.

        \item[Exploradores] 

        \item[Socializadores]
                % les gusta trabajar en equipo, por lo tanto nuestra propuesta de tener
                % grupos que estén compitiendo unos con otros va enfocada con este tipo
                % de usuario.

        \item[Asesinos]
                % son los usuarios competidores que sienten motivación al ganarle a otras
                % personas, las competencias 1 vs 1 y los otros tipos que tenemos
                % contemplados motivarían a este tipo de usuario.
        \end{description}
        \end{quote}

    \noindent 
    Al final de este capitulo, en la seccion \hyperrefx{analisis:principios} se detalla
    la relación que existe entre los distintos tipos de usuario, los principios de
    gamificación y los distintos tipos de usuarios en la clasificación propuesta por
    {\it Richart Bartle}.


\section{Características principales}

 Con base en las funcionalidades generales que se desean proporcionar a los distintos
 actores, se determinaron cuatro características se deben tener en consideración durante
 todo el desarrollo del trabajo terminal, dichas características son descritas a continuación:

% % TAMBIEN SE DESEA QUE SEAN VARIAS OPCIONES PARA QUE LOS PROFES TENGAN UN ABANICO
 % AMPLIO DE OPCIONES

    \begin{multicols}{2}
    \begin{itemize}
    \itemx{ Altamente configurable }

    \item[] Hace referencia a que se debe proporcionar al administrador, profesores
            y alumnos la flexbilidad para que puedan configurar, habilitar o deshabilitar
            los distintos componentes que se desarrollarán dependiendo de las funcionalidades
            correspondientes a cada rol.

    \itemx{ Escoge que quieres incluir }
        
    \item[] Se le debe brindar al administrador la opción de poder escojer únicamente los
            elementos requeridos dentro del conjunto de todos los componentes, de forma similar
            dentro de todos las elementos de gamificación que brindan los componentes instalados
            el profesor podrá escoger cuales desea incluir en sus cursos.

    %\vfill\null
    %\columnbreak
    %\vfill\null

    \itemx{ Bajo acoplamiento }

    \item[] El hecho de permitirle al administrador de la plataforma poder elegir los
            componentes que desea instalar, implica que los mismos deben estar diseñados
            para poder trabajar sin depender de las funciones de los demás componentes.

    \itemx{ Comunicación entre módulos }

    \item[] Que las cosas que pasen en módulo puedan impulsar acciones con otros módulos
            para que cuando se deseen multiples funcionalidades de distintos componentes
            estos puedan trabajar en conjunto como una única herramienta.
    \end{itemize}
    %\vfill\null
    \end{multicols}
