\section{Usuarios}
\label{analisis:usuarios}

 En esta sección se presentan los actores a los que va destinada la propuesta de
 solución, así como sus responsabilidades y perfil recomendado. Posteriormente se
 presentan por cada módulo el conjunto de funcionalidades que se le brindarán a
 cada actor especificando a que módulo pertenecen las funcionalidades.

 % =====================================
 %    A D M I N I S T R A D O R
 % =====================================

    \begin{actor}{aAdministrador}{Administrador del sitio}{%
    El administrador del sitio es la persona con mayor jerarquía respecto a los
    permisos y funcionalidades que brinda moodle, desde configurar la visualización
    de la página principal del sitio hasta editar las políticas de seguridad. A
    continuación se describen las responsabilidades y cualidades que debe tener
    un administrador.}

    \item[Responsabilidades:] \hfill
        \begin{itemize}
        \item Administrar el sitio de Moodle.
        \item Instalar/desinstalar plugins dependiendo de las necesidades
              institucionales.
        \item Realizar las configuraciones requeridas de los plugins instalados.
        \end{itemize}

    \item[Perfil:] \hfill
        \begin{itemize}
        \item Contar con experiencia administrando moodle.
        \item Conocimientos de permisos en LAMP (Linux, Apache, MySQL y PHP).
        \item Conocimientos básicos de gamificación.\\
        \end{itemize}
    \end{actor}

    \noindent
    Las funcionalidades extras que se le brindarán al administrador son las
    siguientes:

    \begin{quote}
    {\bf Funcionalidades generales:}
        \begin{itemize}
        \item Elegir que conjunto de funcionalidades que desea instalar
              en la plataforma moodle que administra.
        \item Realizar las configuraciones generales (a nivel plataforma) de las
              funcionalidades de gamificación instaladas.
        \item Creación de usuarios gamificados de forma automática.
        \item Eliminar de moodle las funcionalidades de gamificación que desee.
        \end{itemize}
    \end{quote}

    \begin{quote}
    {\bf Del módulo de experiencia:}
        \begin{itemize}
        \item Establecer la cantidad de experiencia que brindarán los cursos.
        \item Habilitar o deshabilitar de forma general el sistema de experiencia
              en toda la plataforma.
        \item Habilitar o deshabilitar la experiencia únicamente para los eventos
              soportados.
        \item Establecer el título, descripción, mensaje que se presentan en la
              pantalla emergente cuando un usuario alcanza un nuevo nivel.
        \item Configurar la imagen del nivel y los colores de la barra de progreso,
              y el color del nivel con base en los colores institucionales.
        \item Especificar la experiencia correspondiente al primer nivel.
        \item Elegir el tipo de incremento de la cantidad de experiencia en los
              niveles y el factor o valor de incremento asociado al tipo de
              incremento.
        \item Configurar la cantidad de experiencia que se brinda cuando ocurre
              algún evento emitido por otro módulo.
        \item Elegir en que pantallas de moodle se debería mostrar la
              visualizacion del nivel actual para cada usuario.
        %\item Configurar la cantidad de experiencia que brindará la actividad
        %      {\it uno contra uno}.
        %\item Establecer la cantidad de experiencia que brindará la actividad
        %      {\it uno contra sistema}.
        %\item Establecer la cantidad de experiencia que brindará la actividad 
        %      {\it preguntas diarias}.
        \end{itemize}
    \end{quote}

    \begin{quote}
    {\bf Del módulo financiero:}
        \begin{itemize}
        \item Establecer la cantidad de monedas de plata que brindará la actividad
              'uno contra uno'.
        \item Establecer la cantidad de monedas de plata que brindará la actividad
              'uno contra sistema'.
        \item Establecer la cantidad de monedas de plata que brindará la actividad
              'preguntas diarias'.
        \end{itemize}
    \end{quote}


 % =====================================
 %    P R O F E S O R
 % =====================================

    \begin{actor}{aProfesor}{Profesor}{%
    Un profesor, visto desde el punto de vista de este desarrollo, es la persona
    encargada de la creación y administración de los cursos que desea impartir a
    través de moodle, por lo tanto además de tener rol de {\it profesor} para
    gestionar y añadir contenido a los cursos, debe tener el rol de {\it creador de
    cursos} \cite{MoodleRoles} \\}

    \item[Responsabilidades:] \hfill
        \begin{itemize}
        \item Crear de los cursos de las materias que imparte.
        \item Crear del contenido de los cursos que imparte.
        \item Administrar las actividades, secciones y demás elementos dentro de un
              curso
        \item Establecer las formas de inscripción en el curso.
        \item Elegir y configurar las funcionalidades para la gamificación para el
              curso..
        \end{itemize}

    \item[Perfil:] \hfill
        \begin{itemize}
        \item Contar con experiencia en la creación y administración de cursos.
        \item Contar con conocimiento básicos de gamificaciÓn.
        \end{itemize}
    \end{actor}

    \noindent
    Las funcionalidades extras que se le brindarán al profesor son las siguientes:

    \begin{quote}
    {\bf Funcionalidades generales:}
        \begin{itemize}
        \item Crear un curso con soporte para gamificación.
        \item Establecer las funcionalidades que desea incluir en cada curso.
        \item Convertir un curso no gamificado a un curso gamificado y viceversa.
        \item Soporte para todas las funcionalidades de administración de un curso
              normal.
        \end{itemize}
    \end{quote}

    \begin{quote}
    {\bf Del módulo de experiencia:}
        \begin{itemize}
        \item Habilitar/Deshabilitar el soporte para entregar puntos de experiencia
              en su curso.
        \item Establecer la cantidad de experiencia que se le brindará a cada
              sección de su curso.
        \item Repartir la experiencia de forma igualitaria entre las secciones del
              curso.
        \item Establecer si se mostrará por defecto el nivel actual de experiencia
              correspondiente al usuario que está viendo la pantalla.
        \item Agregar una nueva sección a un curso con soporte para dar experiencia.
        \item Eliminar una sección de un curso con soporte para brindar experiencia.
        \item convertir un curso normal a un curso con experiencia.
        \item Agregar y/o eliminar cursos con soporte para experiencia.
        \end{itemize}
    \end{quote}

    \begin{quote}
    {\bf Del módulo de competencias:}
        \begin{itemize}
        \item Establecer si desea que haya apuestas en las competencias 'uno contra uno' dentro de los cursos que imparte.
        \item Establecer qué banco de preguntas se usará en la actividad 'uno contra uno' dentro de los cursos que imparte.
        \item Establecer cuántas veces un estudiante tiene que ganar para completar la actividad 'uno contra uno' dentro de los cursos que imparte.
        \item Establecer qué banco de preguntas se usará en la actividad 'uno contra sistema' dentro de los cursos que imparte.
        \item Establecer cuál es el nivel de dificultad mínimo a derrotar, para que un estudiante complete la actividad 'uno contra sistema' dentro de los cursos que imparte.
        \end{itemize}
    \end{quote}


    \begin{quote}
    {\bf Del módulo de seguimiento:}
        \begin{itemize}
        \item Establecer qué preguntas se usarán en la actividad 'preguntas diarias' dentro de los cursos que imparte.
        \end{itemize}
    \end{quote}

 % =====================================
 %    A L U M N O
 % =====================================

    \begin{actor}{aEstudiante}{Estudiante}{%
    Un estudiante, para este proyecto, es cualquier usuario de moodle que tenga el rol de
    estudiante. Este rol le permite ver los distintos cursos y al inscribirse participar
    en ellos \cite{MoodleRolEstudiante}. Los distintos módulos de este proyecto fueron pensados
    para brindar soporte a los distintos tipos de usuarios en el ámbito de la gamificación
    propuestos por {\it Richard Bartle} \cite{BartleUsuarios}.\\}

    \item[Responsabilidades:] \hfill
        \begin{itemize}
        \item Cumplir con las actividades de los cursos
        \item Inscribirse a los cursos correspondientes
        \item Hacer uso correcto de la plataforma
        \end{itemize}

    \item[Perfil:] \hfill
        \begin{itemize}
        \item Haber utilizado moodle como estudiante anteriormente.
        \end{itemize}
    \end{actor}

    \noindent
    Las funcionalidades extras que se le brindarán al estudiante son las siguientes:

    \begin{quote}
    {\bf Del módulo de experiencia:}
        \begin{itemize}
        \item Reconocimiento/notificación al haber superado un nivel.
        \item visualizar del nivel actual en el que se encuentra.
        \item Visualizar de los puntos que tiene del nivel actual.
        \item Elegir ocultar o mostrar la visualización del nivel de experiencia.

        \item Ganar puntos de experiencia cada vez que derrote al sistema en un
              nuevo nivel de dificultad en una competencia contra el sistema.

        \item Ganar puntos de experiencia cada vez que derrote a uno
              de sus compañeros en una competencia uno contra uno.

        \item Ganar puntos de experiencia cada vez que conteste correctamente
              una pregunta diaria.
        \end{itemize}
    \end{quote}

    \begin{quote}
    {\bf Del módulo de seguimiento:}
        \begin{itemize}
        \item Contestar de una pregunta de un cuestionario por día.
        \item Visualización del progreso de las preguntas que ha contestado.
        \item Visualización del progreso de las preguntas que han respondido sus compañeros.
        \end{itemize}
    \end{quote}

    \begin{quote}
    {\bf Del módulo de competencia:}
        \begin{itemize}
        \item Competir contra al sistema con diferente niveles de dificultad por medio de contestar un cuestionario.
        \item Competir contra sus compañeros de curso por medio de contestar un cuestionario (puede apostar monedas propias del sistema).
        \item Visualización del historial de competencia que ha tenido con sus compañeros.
        \item Visualización del historial de competencia que ha tenido contra el sistema.
        \item Visualización de la tabla de posiciones de la actividad de competencia uno contra uno.
        \item Visualización de las tablas de puntuaciones de la actividad de competencia uno contra sistema.
        \end{itemize}
    \end{quote}

    \begin{quote}
    {\bf Del módulo de personalización:}
        \begin{itemize}
        \item Adquirir imágenes de perfil con monedas ganadas en el sistema.
        \item Adquirir estilo y color de marco que rodea la imagen de perfil con monedas ganadas en el sistema.
        \item Personalizar su perfil con los objetos adquiridos previamente.
        \end{itemize}
    \end{quote}

    \begin{quote}
    {\bf Del módulo de financiero:}
        \begin{itemize}
        \item Ganar monedas cada vez que derrote al sistema en un nuevo nivel de dificultad en una competencia contra el sistema.
        \item Ganar monedas cada vez que derrote a uno de sus compañeros en una competencia uno contra uno.
        \item Ganar monedas cada vez que conteste correctamente una pregunta diaria.
        \end{itemize}
    \end{quote}

\subsubsection{Tipos de usuarios de gamificación según Richard Bartle}

 La clasificación de usuarios propuesta por Richard Bartle \cite{BartleUsuarios},
 define 4 tipos de usuarios: triunfadores, exploradores, socializadores y asesinos,
 los cuales son descritos a continuación:

    \begin{quote}
    \begin{description}
    \item[Triunfadores.]
        Son aquellos jugadores que establecen sus propios objetivos relacionados con
        el juego o sistema y se proponen vigorosamente alcanzarlos, lo cual
        generalmente se traduce en la acumulación y liberación de elementos de alto
        valor \cite[p. 3]{BartleUsuarios}.

    \item[Exploradores]
        Son aquellos que intentan y buscan cada acción que se puede realizar en el
        juego o sistema, comúnmente recorren y hacen un mapeo de la amplitud que se
        les proporciona, y posteriormente experimentan explorando la profundidad
        \cite[p. 4]{BartleUsuarios}.

    \item[Socializadores]
        Son aquellos que utilizan los distintos mecanismos del juego o sistema brinda
        con el propósito de comunicarse y naturalmente establecer roles interactuando
        con las demás personas presentes en el juego o sistema
        \cite[p. 4]{BartleUsuarios}.

    \item[Asesinos]
        Son aquellos que usan las herramientas del juego o sistema para obtener la
        historia sobre los demás jugadores, normalmente este implica utilizar las
        herramientas, armas, entre otras en contra de los demás y así ganar
        \cite[p. 4]{BartleUsuarios}.
    \end{description}
    \end{quote}

 \noindent
 Al final de este capitulo, en la sección \hyperrefx{analisis:principios} se detalla
 la relación que existe entre los distintos tipos de usuario, los principios de
 gamificación y los distintos tipos de usuarios en la clasificación propuesta por
 {\it Richart Bartle}.


\section{Características principales}

 Con base en las funcionalidades generales que se desean proporcionar a los distintos
 actores, se determinaron cuatro características se deben tener en consideración
 durante todo el desarrollo del trabajo terminal, dichas características son
 descritas a continuación:

% % TAMBIEN SE DESEA QUE SEAN VARIAS OPCIONES PARA QUE LOS PROFES TENGAN UN ABANICO
 % AMPLIO DE OPCIONES

    \begin{multicols}{2}
    \begin{itemize}
    \item{\bf\color{primary} Altamente configurable }

    \item[] Hace referencia a que se debe proporcionar al administrador, profesores
            y estudiantes la flexibilidad para que puedan configurar, habilitar o
            deshabilitar los distintos componentes que se desarrollarán dependiendo
            de las funcionalidades correspondientes a cada rol.

    \item{\bf\color{primary} Escoge qué quieres incluir }

    \item[] Se le debe brindar al administrador la opción de poder escoger únicamente
            los elementos requeridos dentro del conjunto de todos los componentes, de
            forma similar dentro de todos las elementos de gamificación que brindan
            los componentes instalados el profesor podrá escoger cuales desea incluir
            en sus cursos.

    \item{\bf\color{primary} Bajo acoplamiento }

    \item[] El hecho de permitirle al administrador de la plataforma poder elegir los
            componentes que desea instalar, implica que los mismos deben estar
            diseñados para poder trabajar sin depender de las funciones de los demás
            componentes.

    \item{\bf\color{primary} Comunicación entre módulos }

    \item[] Que las cosas que pasen en módulo puedan impulsar acciones con otros
            módulos para que cuando se deseen múltiples funcionalidades de distintos
            componentes estos puedan trabajar en conjunto como una única herramienta.
    \end{itemize}
    \end{multicols}
