
\section{Relación entre los módulos, principios y tipos de usuario}
\label{analisis:principios}

 Los módulos descritos en la sección \hyperrefx{analisis:modulos} fueron pensados para
 brindar soporte a distintos principios de gamificación. Los módulos están compuestos
 por submódulos los cuales están vinculados directamente a los principios de
 gamificación a los que brindan soporte (ver figura \ref{fig:modulosP}).

    \addfigure{0.85}{analisis/diagrams/modulosTTyP}{fig:modulosP}%
    {Relación entre los principios de Gamificación y los submódulos identificados}

 \clearpage
 \noindent
 En la tabla \ref{tbl:principiosxmodulos} se muestran los principios de gamificación
 a los que cada módulo brinda soporte. Como se puede observar se puede brindar
 soporte a la mayoría de los principios de gamificación desde distintos módulos,
 exceptuando los principios de {\it diferenciación y pertenencia} y
 {\it Personalización}.

    \addtable{|l|c|c|c|c|c|c|}{tbl:principiosxmodulos}{
        {\bf Principios de gamificación} &
        \rotatebox[origin=c]{45}{\bf Experiencia} &
        \rotatebox[origin=c]{45}{\bf Competencia} &
        \rotatebox[origin=c]{45}{\bf Personalización} &
        \rotatebox[origin=c]{45}{\bf Financiero} &
        \rotatebox[origin=c]{45}{\bf Seguimiento} \\\hline
        %\rotatebox[origin=c]{45}{\bf Recompensa}

        Diferenciación y pertenencia  &   &   & x &   &   \\\hline
        Desarrollo y competencia      & x &   &   &   & x \\\hline
        Personalización               &   &   & x &   &   \\\hline
        Motivo e impulso social       &   & x &   &   &  \\\hline
        Codicia                       & x & x &   & x &  \\\hline
        Impredecibilidad y curiosidad &   & x &   & x & x \\\hline
        Miedo a la perdida            &   & x & x &   &   \\\hline

    }{Relación entre los principios de gamificación y los módulos identificados}

 \noindent
 De acuerdo con {\it Octalysis} \cite[p. 414]{Octalysis} los principios de
 gamificación están relacionados con distintos tipos de usuarios dentro de la
 clasificación propuesta por {\it Richard Bartle}, de tal forma que:

    \begin{quote}
    \begin{itemize}
        \item Los {\bf triunfadores} son motivados en su mayoría por
              los principios de desarrollo y recompensa, codicia,
              personalización y descubrimiento y retroalimentación.

        \item Los {\bf exploradores} son motivados principalmente por
              el principio de impredictibilidad y curiosidad, así como
              por el desarrollo y recompensa, el descubrimiento y
              retroalimentación, y la codicia.

        \item Los {\bf socializadores} son motivados por los principios
              de motivo e impulso social, descubrimiento y retroalimentación,
              impredictibilidad y curiosidad, y personalización.

        \item Los {\bf asesinos} son motivados principalmente por una mezcla
              del principio de desarrollo y recompensa y el de motivo e impulso
              social, también son motivados en menor medida por el descubrimiento
              y retroalimentación, el miedo a la perdida y la personalización.
    \end{itemize}
    \end{quote}

 \noindent
 Como existe una relación entre los principios de gamificación y los tipos de usuarios
 propuestos por {\it Richard Bartle} y de la misma forma existe una relación entre
 los módulos identificados y los principios de {\it Octalysis}, es posible relacionar
 cada uno de los módulos identificados con los distintos tipos de usuario, dichas
 relaciones se encuentran en la tabla \ref{tbl:usuariosxmodulos}

    \addtable{|l|c|c|c|c|c|c|}{tbl:usuariosxmodulos}{
        {\bf Tipos de usuarios} &
        \rotatebox[origin=c]{40}{\bf Experiencia} &
        \rotatebox[origin=c]{40}{\bf Competencia} &
        \rotatebox[origin=c]{40}{\bf Personalización} &
        \rotatebox[origin=c]{40}{\bf Financiero} &
        \rotatebox[origin=c]{40}{\bf Seguimiento} \\\hline

        Triunfadores   & x & x &   &   & x \\\hline
        Exploradores   & x & x &   & x & x\\\hline
        Socializadores & x & x & x &   & x  \\\hline
        Asesinos       & x & x & x &   & x  \\\hline

    }{Relación entre los principios de gamificación y los tipos de usuarios}
