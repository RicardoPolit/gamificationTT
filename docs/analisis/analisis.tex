\chapter{Análisis general}
\label{ch:analisis}

 Este capítulo tiene como principal objetivo establecer las características o
 cualidades más importantes a considerar durante el diseño de los elementos de
 software y el desarrollo de los mismos durante este proyecto. Además presenta la
 forma en que se han seguido los marcos de trabajo para establecer los incrementos
 y entregables de cada iteración.

    

\section{Adaptación de la metodología}

 A continuación se presenta la forma en que ha sido configurado el marco de trabajo {\it Scrum}
 para trabajar en concordancia con los marcos de trabajo {\it Octalysis} y {\it For The Win}
 usados cómo guía para la diseño de los componentes creados para la implementación de
 Gamificación en una plataforma en línea.

\subsection{Roles}

 \noindent
 El marco de trabajo {\it Scrum} define los roles: scrum master, dueño del producto, stakeholders,
 y equipo de desarrollo. A continuación se especifica las personas que estarán cumpliendo los
 distintos roles en Scrum.

\subsubsection{Product Owner}

 Durante este proyecto el rol del dueño del producto (o {\it product owner}) lo llevarán a cabo
 los directores del trabajo terminal, ya que son a máxima autoridad entorno a la definición del
 alcance, ambos directores son listados a continuación:

    \begin{quote}
    \begin{itemize}
        \item {\it M. en C.} Sandra Ivette Bautista
        \item {\it M. en C.} Edgar Armando Catalán Salgado
    \end{itemize}
    \end{quote}

 \noindent
 A pesar de que en la guía oficial de scrum \cite{TheScrumGuide} se especifica que el {\it
 product owner} debe ser una persona, se decidió que este rol fuera llevado acabo mediante los
 directores del trabajo terminal, con la premisa de que para la toma de decisiones ambos
 directores deben estar de acuerdo.

\subsubsection{Equipo de desarrollo}

 Durante este trabajo el equipo de desarrollo estará conformado por un total de
 tres integrantes, siento estos los estudiantes que presentan el trabajo terminal. Los miembros del
 equipo de desarrollo team son listados a continuación:

    \begin{quote}
    \begin{itemize}
        \item David Flores Casanova
        \item Ricardo Naranjo Polit
        \item Daniel Isaí Ortega Zúñiga
    \end{itemize}
    \end{quote}

% ROLES - PRODUCT OWNER

 % \noindent
 % Para que las funciones del Product Owner sean exitosas, la organización debe respetar sus
 % decisiones. Nadie puede forzar al Development Team para trabajar en un conjunto distinto
 % de requerimientos establecidos en el Product Backlog.

% ROLES - EQUIPO DE DESARROLLO

 % durante el desarrollo de este proyecto el equipo de desarrollo estará conformado por:

 % \noindent El tamaño del equipo debe ser lo suficientemente pequeño para permanecer ágil y lo
 % suficientemente grande para realizar entregas significativas al final de cada {\em Sprint}.\\
 % Un equipo con menos de tres miembros disminuiría la interacción y por lo tanto la productividad,
 % por otro lado, con más de nueve miembros se requiere mayor coordinación con más complejidad.

\subsubsection{Maestro Scrum}

 El rol del {\it Scrum Master} durante este proyecto se llevará a cabo mediante
 dos personas con la finalidad de dividir las responsabilidades y no sobrecargar de trabajo
 a una persona. El {\it Scrum Master} estará conformado por:

    \begin{quote}
    \begin{itemize}
        \item M. en C. Edgar Armando Catalán {\it (Responsabilidades hacia el Product Owner)}
        \item Daniel Isaí Ortega {\it(Responsabilidades hacia el equipo de desarrollo)}
    \end{itemize}
    \end{quote}


\subsubsection{Stakeholders}

 Los Stakeholders son personas externas al equipo Scrum con un interés y/o conocimientos
 específicos del producto \cite{ScrumGlosary}. En relación a la naturaleza del proyecto se
 considera a los sinodales cómo los Stakeholders oficiales, los cuales son listados a
 continuación:

    \begin{quote}
    \begin{itemize}
        \item Dra. Fabiola Ocampo Botello
        \item M. en C. María del Socorro Téllez Reyes
        \item M. en C. José David Ortega Pacheco
    \end{itemize}
    \end{quote}

\subsection{Eventos}

 A continuación se detalla cómo han sido configurados los distintos eventos dentro del marco de trabajo
 Scrum, los cuales contemplan las iteraciones o los {\it Sprints}, reuniones diarias, las etapas de
 planeación, revisión y retroalimentación.

\subsubsection{Sprints}

 \noindent Para este proyecto los Sprints están configurados a una duración de 14 días con una
 estimación de 18 iteraciones, la duración de dos semanas se estableció con el propósito de:

    \begin{itemize}
    \item Incrementar la retroalimentación y detectar los impedimentos
          en la forma de trabajo lo más pronto posible, y

    \item Realizar incrementos más cortos y continuos considerando que el
          equipo de desarrollo está conformado por tres integrantes.
    \end{itemize}

 \noindent
 La figura \ref{tbl:sprints}, mostrada en la siguiente página, representa una estimación de fueron
 organizados los {\it sprints} a lo largo del proyecto. La primer etapa de {\bf investigación
 y análisis general} contiene forma de trabajo, la investigación acerca de la gamificación en la educación
 y finalmente la definición del alcance general. Posteriormente cada dos sprints se llevan a cabo las {\bf
 etapas de análisis, diseño, desarrollo y pruebas para cada módulo}. \clearpage

    \begin{sidewaysfigure}
        \centering
        \includegraphics[width=1\textwidth]{analisis/diagrams/cronograma}
        \caption{Cronograma de actividades}
        \label{fig:awesome_image}
    \end{sidewaysfigure}

\begin{comment}
 \clearpage
    {\centering\color{primary}\Huge THIS PAGE SHOULD BE REMOVED}

    \addtable{|c|c|c|l|}{tbl:sprints}{
        {\bf No Sprint} & {\bf Fecha Inicio} & {\bf Fecha Final}  & {\bf Tarea}\\\hline

        1  &  5 de Febrero    & 18 de Febrero    & Marco Teórico, Metodología, Definición del Alcance \\\hline
        2  & 19 de Febrero    &  4 de Marzo      & Investigación de Implementación \\\hline
        3  &  5 de Marzo      & 18 de Marzo      & Reporte Técnico del Trabajo Terminal \\\hline
        4  & 19 de Marzo      &  1 de Abril      & Pruebas de Concepto\\\hline

        \rowcolor{orange}
        5  &  2 de Abril      & 15 de Abril      & Mod. Experiencia: análisis y diseño \\\hline
        6  & 16 de Abril      & 29 de Abril      & Mod. Experiencia: desarrollo y pruebas\\\hline

        7  & 30 de Abril      & 13 de Mayo       & Mod. Recompensa: análisis y diseño \\\hline
        8  & 14 de Mayo       & 27 de Mayo       & Mod. Recompensa: desarrollo y pruebas \\\hline

        9  & 28 de Mayo       & 10 de Junio      & Mod. Personalización: analisis y diseño\\\hline % FIN SEMESTRE
        10 & 11 de Junio      & 24 de Junio      & Mod. Personalización: desarrollo y pruebas \\\hline

        11 & 25 de Junio      &  8 de Julio      & Mod. Financiero: analisis y diseño \\\hline
        12 &  9 de Julio      & 22 de Julio      & Mod. Financiero: desarrollo y pruebas \\\hline

        13 & 23 de Julio      &  5 de Agosto     & Mod. Competencia: analisis y diseño \\\hline % INICIO SEMESTRE
        14 &  6 de Agosto     & 19 de Agosto     & Mod. Competencia: desarrollo y pruebas \\\hline

        15 & 20 de Agosto     &  2 de Septiembre & Mod. Seguimiento: analisis y diseño \\\hline
        \rowcolor{green}
        16 &  3 de Septiembre & 16 de Septiembre & Mod. Seguimiento: desarrollo y pruebas \\\hline

        17 & 17 de Septiembre & 30 de Septiembre & Pruebas de Integración\\\hline
        18 &  1 de Octubre    & 14 de Octubre    & Caso de Estudio I (y correcciones)\\\hline
        19 & 15 de Octubre    & 28 de Octubre    & Caso de Estudio II \\\hline

        %20 & 29 de Octubre    & 17 de Noviembre  & Pruebas de Integración\\\hline
        %21 & 18 de Noviembre  & 25 de Noviembre  & Ejecución del Caso de Estudio\\\hline
        %22 & 26 de Noviembre  &  9 de Diciembre  & ''\\\hline % FIN DE SEMESTRE

    }{Detalle de Sprints (Estimación)}
\end{comment}
\clearpage

\subsubsection{Planeación}

 \noindent El día acordado para llevar a cabo esta reunión fueron los {\bf martes} cada dos semanas
 {\bf a la 1:30pm} en las instalaciones de la Escuela Superior de Cómputo (ESCOM). El horario fue
 acordado tomando en cuenta la disponibilidad de todos los miembros del equipo Scrum.

    \begin{quote}
    {\bf Nota:} En caso de que, por algún evento extraordinario, no se pueda
                llevar a cabo la etapa de planeación esta reunión se reagendará
                para que ocurra lo más pronto posible.
    \end{quote}

\subsubsection{Reunión diaria}

 \noindent
 Para llevar a cabo las reuniones diarias se establecieron por defecto días, lugar y hora detallados
 en la tabla \ref{tbl:daily}. Esta reunión sera realizada entre los miembros del equipo de desarrollo
 debido a la dificultad de hacer coincidir los horarios del equipo de desarrollo con los del {\it Product
 Owner}.

    \addtable{|c|c|c|}{tbl:daily}{
       {\bf Día de Trabajo} & {\bf Lugar} & {\bf Hora Inicio} \\\hline
       Lunes     & ESCOM Sala 21 N & 10:00am \\\hline
       Martes    & ESCOM Sala 21 N & 10:00am \\\hline
       Miércoles & ESCOM Sala 21 N & 10:00am \\\hline
       Jueves    & ESCOM Sala 21 N & 10:00am \\\hline
       Viernes   & ESCOM Sala 21 N & 10:00am \\\hline
       Domingo   & -               & 12:00pm \\\hline
    }{Horario y lugar acordado de las reuniones diarias}

    \begin{quote}
    Los días domingos la reunión diaria se llevará a cabo a través de una llamada grupal
    entre los miembros del equipo de desarrollo.
    \end{quote}

\subsubsection{Revisión}

 \noindent Debido a que en este proyecto, los stakeholders y el equipo de scrum tienen distintos horarios
 de disponibilidad, la revisión del {\it sprint} se divide en cuatro fases, aplicando la primer fase a los sprints
 impares y las cuatro fases para sprints pares. Las fases de describen a continuación:

    \begin{quote}
    \begin{itemize}
    \item[\it Fase 1]
        Consiste en realizar una primer reunión con el equipo scrum para obtener una
        retroalimentación y revisar el incremento entregado.

    \item[\it Fase 2]
        En esta fase el equipo de desarrollo tiene reuniones con los Stakeholders con la
        finalidad de obtener retroalimentación y observaciones acerca de la forma de
        trabajo y del incremento.

    \item[\it Fase 3]
        En esta fase los miembros del equipo scrum revisarán las observaciones y
        comentarios de los {\it Stakeholders} para saber cuales proceden.

    \item[\it Fase 4]
        Se avisa a los Stakeholders acerca de cuales observaciones procedieron y cuales no.\\
    \end{itemize}

    {\bf Nota:} Las reuniones de la fase 2, dependen de la disponibilidad que cada {\it stakeholder} tenga,
                en caso de que ningún stakeholder tenga disponibilidad para llevar a cabo la fase 2,
                el proceso de la revisión del {\it Sprint} terminará.
    \end{quote}

 \noindent \message{^^JWarning WE DONT RESPECT THE STABLISHED SPRINT REVIEW pp.\thepage}  %Ricardo: Da error de compilación, no entiendo porque pone eso, y luego en ingles

\subsubsection{Retroalimentación}

 \noindent La etapa de retroalimentación consiste en una reunión interna para discutir aquellas cosas que han
 sucedido correctamente y cuales deben mejorar. Esta reunión se realizará al final de cada {\it Sprint} estando
 presentes los directores del trabajo terminal y los miembros del equipo de desarrollo.

\subsection{Artefactos}

 A continuación se describe la forma en que serán se llevarán a cabo los artefactos definidos por el marco
 de trabajo Scrum, además se describe los atributos con los que serán especificados los elementos del
 {\it Product Backlog} y {\it Sprint Backlog}.

\subsubsection{Product Backlog}

 \noindent Debido a que el proyecto requería una etapa de investigación, se optó por tener dos tipos
 de {\it items} en el product backlog, los items de documentación/preparación del proyecto  y los {\it items}
 para desarrollo del mismo.\\

 \noindent{\bf Items de Documentación}\\
 Los items de preparación del proyecto y documentación deben ser especificados
 mediante los atributos presentes en la tabla \ref{attrPBpre}:

    \addtable{|l|l|}{attrPBpre}{
        {\bf Atributo} & {\bf Descripción}                                                             \\\hline
        id           &  Es una identificador de la forma ``Ax'' donde {\it x} es un número consecutivo \\\hline
        nombre       &  Nombre representativo de la actividad                                          \\\hline
        descripción  &  Detalle de lo que hay que hacer para llevar a cabo esta actividad.             \\\hline
        sprint       &  Indica el número de Sprint al cual ha sido asignada esta tarea.                \\\hline
        %estado      & Indica el estado ({\it por hacer, en proceso o concluida} de una actividad. \\\hline
        %estimación  & Especifica el periodo de tiempo estimado para la liberación de dicha actividad. \\\hline
    }{Atributos de los Items del P.B de Documentación}


 \noindent{\bf Items de Desarrollo del Proyecto}\\
 Describen las características del software que se desarrollará, estos items deben
 ser redactados de manera objetiva y como requerimientos del sistema, y deben contener
 los atributos presentes en la tabla \ref{attrPB}:

    \addtable{|l|p{0.62\textwidth}|}{attrPB}{
        {\bf Atributo} & {\bf Descripción}\\\hline
        id           &  Es una identificador de la forma '{\bf RFx}' o '{\bf RNFx}' para requerimientos    \par
                        funcionales y no funcionales respectivamente. {\it x} es un número consecutivo     \\\hline

        nombre       &  Nombre representativo del requerimiento del sistema.                               \\\hline
        descripción  &  Descripción concisa y objetiva acerca del requerimiento.                           \\\hline
        prioridad    &  Indica la prioridad de un requerimiento, los valores posibles son:                 \par
                        \qquad MA (muy alta), A (alta), M (Media), B (baja) y MB (muy baja)                \\\hline

        sprint       &  Indica el número de Sprint al cual ha sido asignado este requerimiento.            \\\hline
        tipo         &  Tipo de requerimiento no funcional según la clasificación propuesta por Frank Tsui \\\hline
        %estimación  & Especifica el periodo de tiempo estimado para la liberación de dicho requerimiento. \\\hline
 }{Atributos de los Items del P.B de Desarrollo del Proyecto}

 \begin{quote}
 {\bf Nota:} El atributo {\it Sprint} debe estar presente en todos los items correspondientes
              al sprint corriente y a los sprints anteriores a este. El atributo {\it Sprint}
              puede no estar presente en los items que no han sido vinculados a un Sprint.
 \end{quote}

\subsubsection{Sprint Backlog}

 \noindent Conforme los {\it items} del {\it product backlog} vayan siendo seleccionados para
 tratarse en un sprint, se les añadirá los atributos presentes en la tabla \ref{SBItems}, los
 cuales son una etiqueta que indique a qué sprint pertenecen y opcionalmente una etiqueta que
 indique cómo se evaluará la completitud de dicho item.

    \addtable{|l|l|}{SBItems}{
        {\bf Atributo} & {\bf Descripción}\\\hline
        sprint   &  Indica el número de Sprint al cual ha sido asignado el item.             \\\hline
        pruebas  &  (Opcional) Sentencia de cómo se evaluara que dicho ítem esté completado. \\\hline

    }{Atributos del Sprint Backlog}

% ======================================================
%   F O R   T H E   W I N
% ======================================================

\section{Marco de Trabajo For The Win}
\label{analisis:forthewin}

 Una de las herramientas más importantes que brinda el marco de trabajo {\it For The Win} es
 una guía pragmática para la implementación de la gamificación \cite[p. 8]{ForTheWin}, dicha guía consiste de seís
 pasos, el detalle de cómo se siguieron estos pasos se encuentra en las siguientes secciones
 de este capítulo de análisis.

    \begin{quote}
    \begin{itemize}
    \item El primer paso del marco de trabajo consiste en definir los objetivos del
          sistema, mientras que el segundo consiste en delimitar las acciones de los
          usuarios, especificando el comportamiento que tendrán dentro del sistema
          \cite[p. 61, 63]{ForTheWin}. Ambos puntos son tratados en la sección \hyperrefx{analisis:modulos}.\\

    \item El siguiente paso consiste en definir que usuarios interactuarán con el sistema
          y construir las funcionalidades con base en los tipos de usuarios que lo
          usarán \cite[p. 64]{ForTheWin}, para ello se destina la sección \hyperrefx{analisis:usuarios}.\\

    \item El cuarto paso del marco de trabajo consiste en definir los ciclos de actividades
          o las etapas por las que pasa un usuario para avanzar en el sistema gamificado.
          \cite[p. 66]{ForTheWin}
          Para este punto se presenta en la sección \hyperrefx{analisis:visionyuso}.\\

    \item El quinto paso tiene el titulo {\it``Piensa en la diversión''}, de acuerdo con el
          marco de trabajo para poder solventar este punto el diseño de los elementos de
          juego debe contemplar tanto elementos que apoyen a la motivación extrínseca cómo
          a la motivación intrínseca \cite[p. 68]{ForTheWin}. El mapeo entre los módulos
          planteados y el tipo de motivación a la que brindan soporte se puede ver en la
          sección \hyperrefx{analisis:principios}.\\

    \item El último paso consiste en escoger específicamente que elementos de juego se
          deciden utilizar en el diseño del sistema, probarlos, ajustarlos y ejecutar
          los cambios requeridos para mejorar la implementación \cite[p. 69]{ForTheWin}.
          Este punto es tratado en parte del documento destinada a los módulos.
    \end{itemize}
    \end{quote}

    % Explicar cómo es que está configurado scrum en este proyecto
    % detallar los entregables y las forma de trabajar.

    % Mencionar que se siguierón los pasos de ''For The Win'' y
    % que cada una de las secciones siguientes está vinculada a
    % los pasos de For The Win.

    \clearpage
    
\section{Módulos identificados}

    % Describir cuales son los módulos que fueron identificados
    % cual es el propósito de cada uno, y de que sub módulos
    % estan constituidos, solo poner la imagen principal.

    \clearpage
    \section{Usuarios}
\label{analisis:usuarios}

 En esta sección se presentan los actores a los que va destinada la propuesta de
 solución, así como sus responsabilidades y perfil recomendado. Posteriormente se
 presentan por cada módulo el conjunto de funcionalidades que se le brindarán a
 cada actor especificando a que módulo pertenecen las funcionalidades.

 % =====================================
 %    A D M I N I S T R A D O R
 % =====================================

    \begin{actor}{aAdmininstrador}{Administrador del sitio}{%
        El adminsitrador del sitio es la persona con mayor jerarquía respecto
        a los permisos y funcionalidades que brinda moodle, desde configurar la
        visualización de la página principal del sitio hasta editar las políticas
        de seguridad. A continuación se describen las responsabilidades y cualidades
        que debe tener un administrador.}

    \item[Responsabilidades:] \hfill
        \begin{itemize}
        \item Instalar/desinstalar plugins
        \item configuración general de los plugins
        \end{itemize}

    \item[Perfil:] \hfill
        \begin{itemize}
        \item Contar con experiencia administrando moodle.
        \item Conocimientos de permisos en LAMP (Linux, apache, MySQL and PHP).
        \item Conocimientos básicos de gamificación.\\
        \end{itemize}
    \end{actor}

 \noindent
 Las funcionalidades extras que se le brindarán al adminitrador son las siguientes:

    \begin{quote}

    {\bf Del módulo de experiencia:}
        \begin{itemize}
        \item Establecer la cantidad de experiencia que brindarán los cursos.
        \item Especificar la experiencia correspondiente al primer nivel.
        \item Elegir el tipo de incremento de la cantidad de experiencia en los niveles.
        \item Especificar el factor o valor de incremento asociado al tipo de incremento.
        \end{itemize}

    % TODO expecificar las funcionalidades de los otros módulos
    \end{quote}

 % =====================================
 %    P R O F E S O R
 % =====================================

    \begin{actor}{aProfesor}{Profesor}{%
        ...}

    \item[Responsabilidades:] \hfill
        \begin{itemize}
        \item ...
        \end{itemize}

    \item[Perfil:] \hfill
        \begin{itemize}
        \item ...
        \end{itemize}
    \end{actor}

 \noindent
 Las funcionalidades extras que se le brindarán al profesor son las siguientes:

    \begin{quote}

    {\bf Del módulo de experiencia:}
        \begin{itemize}
        \item ...
        \end{itemize}

    % TODO expecificar las funcionalidades de los otros módulos
    \end{quote}

 % =====================================
 %    A L U M N O
 % =====================================

    \begin{actor}{aAlumno}{Alumno}{%
        ...}

    \item[Responsabilidades:] \hfill
        \begin{itemize}
        \item ...
        \end{itemize}

    \item[Perfil:] \hfill
        \begin{itemize}
        \item ...
        \end{itemize}
    \end{actor}

 \noindent
 Las funcionalidades extras que se le brindarán al profesor son las siguientes:

    \begin{quote}

    {\bf Del módulo de experiencia:}
        \begin{itemize}
        \item ...
        \end{itemize}

    % TODO expecificar las funcionalidades de los otros módulos
    \end{quote}

 La división propuesta de Richard Bartle \cite{TiposDeUsuario}, los divide en 4 grupos, triunfadores,
 exploradores, socializadores y asesinos. En nuestro caso no nos enfocaremos en los exploradores
 puesto que los cursos de moodle son lineales.
    
    \begin{itemize}
    \item Los triunfadores son aquellos que les gusta recibir premios, en nuestro caso
          los logros que tenemos contemplados.

    \item Socializadores, les gusta trabajar en equipo, por lo tanto nuestra propuesta
          de tener grupos que estén compitiendo unos con otros va enfocada con este tipo
          de usuario.

    \item Los asesinos son los usuarios competidores que sienten motivación al ganarle
          a otras personas, las competencias 1 vs 1 y los otros tipos que tenemos
          contemplados motivarían a este tipo de usuario.
    \end{itemize}
    
 Es importante recalcar que en este momento las divisiones de los usuarios son mapeados en los principios
 del marco de trabajo Octalysis de la siguiente manera en el cuadro \ref{table:usuariosvprincipios} según
 el autor de Octalysis\cite[p. 414]{Octalysis}.
    
    \begin{table}[h!]
    \centering
    \begin{tabular}{|c|c|} \hline
        Triunfadores & Principio II, Principio VI \\ \hline
        Socializadores &  Principio V, Principio III, Principio VII\\\hline
        Asesinos & Principio II, Principio V, Principio VIII, Principio IV \\\hline
    \end{tabular}
    \caption{Tabla de mapeo de tipos de usuario y principios de Octalysis}
    \label{table:usuariosvprincipios}
    \end{table}


\section{Características principales}

 % TAMBIEN SE DESEA QUE SEAN VARIAS OPCIONES PARA QUE LOS PROFES TENGAN UN ABANICO
 % AMPLIO DE OPCIONES

 Las características más
 importantes que debe brindar nuestra propuesta de solución son las siguientes:

    \begin{multicols}{2}
    \begin{itemize}
    \itemx{ Altamente configurable }

    \item[] Hace referencia a que se debe proporcionar al administrador, profesores
            y alumnos la flexbilidad para que puedan configurar los valores por defecto
            y visualización de la herramienta que se desarrollará.

    \itemx{ Escoge que quieres incluir }

    \item[] Una de las principales caracteristicar que deben proporcionar estos
            componentes hacia los profesores es que les permitan escoger que elementos
            de gamificación desean incluir en sus cursos y cuales no.\\

    %\vfill\null
    %\columnbreak
    %\vfill\null

    \itemx{ Bajo acoplamiento }

    \item[] Respecto el poder elegir que características se desean incluir y cuales no,
            implica que los componentes que brindan dichas caracterñisticas hayan sido
            diseñados para que puedan trabajar tanto independientemente como
            colaborativamente.

    \itemx{ Comunicación entre módulos }

    \item[] Que las cosas que pasen en módulo puedan permitir realizar acciones con
            otros módulos para que cuando se deseém ocupar ambos, estos puedan trabajar
            en conjunto.

    \end{itemize}
    %\vfill\null
    \end{multicols}

    % Detallar el perfil de nuestros usuarios y los beneficios
    % que se brindarán a cada uno de ellos.

    % Además mencionar que se puede brindar soporte a distintos
    % tipos de jugadores a través de los principios de gamificación.
    % Presentar imagen de principios con módulos.

    \clearpage
    % FOLLOWING PART WAS REMOVED DUE TO DAVID-PACHECO WARNING
    % OF USIGN PROCESS
    % ---------------------------------------------
    
\section{Procesos}
\label{analisis:visionyuso}

    % Detallar cual es el proceso que se tiene planificado para
    % el uso de nuestra herramienta, describir el procesos desde
    % la instalación y configuración de plugins, hasta la aparición
    % de elementos de gamificación cuando los estudiantes realizan una
    % acción.

    
\section{Relacion entre los módulos y principios}
\section{Soporte para los principios de gamificación}

 Los módulos descritos anteriormente contemplan todas las funcionalidades que se desarrollarán
 durante el desarrollo de este trabajo terminal, en conjunto todos los módulos planteados contienen
 19 submódulos. Los módulos y submódulos se presentan organizados en la figura \ref{fig:modulos}.\\

 \noindent 
 % TAMBIEN SE DESEA QUE SEAN VARIAS OPCIONES PARA QUE LOS PROFES TENGAN UN ABANICO
 % AMPLIO DE OPCIONES

\section{Relación entre los módulos y los principios de gamificación}
\label{analisis:principios}
    
    La figura \ref{fig:modulosP} muestra la relación que cada submódulo tiene con los principios de gamificación.
    
    \addfigure{0.89}{analisis/diagrams/modulosTTyP}{fig:modulosP}{Relación entre los principios de Gamificación y los submódulos identificados}

    % Brindar un aspecto más amplio acerca de como se le brinda
    % soporte a distintas cosas a través de los principios del marco
    % de trabajo de octalysis.

    \clearpage
    
\section{Estudio de factibilidad de implementación sobre moodle}

    % Detalla el estudio de fatibilidad que se realizó inicialmente
    % para saber si era viable usar moodle o no.

    \clearpage
    \section{Modelo de información}
\label{ch:dominioDatos}

 Este capítulo se presenta el diccionario de datos que contiene la especificación de los
 atributos que se utilizarán a lo largo del mismo y se describe el modelo
 de información que le brinda soporte a la persistencia de los datos requeridos
 por los módulos y submódulos previamente establecidos.\\

 \noindent Debido a que este proyecto extiende la funcionalidad de moodle,
 se debe contemplar el esquema de base de datos de moodle para el diseño del modelo de información.
 El esquema de base de datos de moodle contiene alrededor de 400 entidades, por ello solo se hace mención de aquellas
 entidades relevantes para el desarrollo de este proyecto. Cabe recalcar que la información
 recopilada de la base de datos de moodle fue obtenida
 directamente del esquema de base de datos, ya que no existe documentación oficial acerca
 del esquema.

\subsection{Diccionario de datos}

 En esta sección se detalla el diccionario de datos del proyecto, en él se describe cada una
 de las entidades pertenecientes al modelo de información, así como el nombre, tipo de dato,
 descripción y restricciones sobre los atributos de las entidades.\\

 A continuación se listan y describen los tipos de
 datos usados en el diccionario.

    \begin{quote}
    \begin{bGlosario}
        \bTerm{tVarchar}{varchar} Cadena de caracteres de longitud variable
        \bTerm{tInt}{int}         Numeros enteros incluyendo, positivos y negativos.
        \bTerm{tBoolean}{boolean} Tipo de dato que solo tiene dos valores posibles,
                                  verdadero y falso.
        \bTerm{tNatural}{natural} Número entero mayores a cero.
        \bTerm{tDouble}{double}   Número real que contiene valores decimales.
        \bTerm{tColor}{color}     Cadena de texto que almacena el código de un color en hexadecimal,
                                  de la forma {\it ``\#RRGGBB''}.
        \bTerm{tImage}{imagen}    Tipo de dato que almacena la referencia a una imagen almacenada por
                                  moodle.
        \bTerm{tPath}{ruta}       Cadena que representa la ubicación de un archivo dentro del sistema
                                  de archivos.
        \bTerm{tVersion}{version} Conjunto de dígitos que representan la fecha de la ultima
                                  actualización del plugin.
        \bTerm{tObject}{objecto}  Es una colección de propiedades, y una propiedad es una
                                  asociación entre un nombre (o clave) y un valor
                                  %\cite{ObjectMozilla}.
        \bTerm{tText}{texto}      Cadena de caracteres que tiene una longitud máxima de $2^{(16)}$ caracteres.
        \bTerm{tTime}{tiempo}     Fecha y hora representadas como un número entero que contiene el valor en segundos de esa fecha y hora.
    \end{bGlosario}
    \end{quote}

 Además de los tipos de dato, se definen las
 siguientes restricciones:

    \begin{quote}
    \begin{bGlosario}

        \bTerm{tRequired}{Requerido}
            Indica que el dato es obligatorio y no puede existir un registro sin que tenga
            un valor.

        \bTerm{tUniqueKey}{Único}
            Indica que que los valores para un atributo o combinación de estos no se deben
            repetir en los diferentes registros.

        \bTerm{tPrimaryKey}{Llave primaria}
            Indica el atributo o combinación de estos que serán la llave primaria. La llave
            primaria es el identificador único de cada registro dentro de la tabla, por lo
            que de no puede haber duplicados y es obligatorio que tenga un valor.

        \bTerm{tForeignKey}{Llave foránea}
            Indica el atributo o conjunto de estos que son llave foránea y apuntan a otra
            tabla, esto es, sus valores posibles Sólo pueden ser aquellos que existan en la
            llave primaria a la que apuntan. FK(Tabla.atributo).

        \bTerm{tRange}{Rango}(a, b)
            Indica que el numero de valores esta restringido del valor $a$ al $b$.

        \bTerm{tAutoIncrement}{Auto-incremental}
            Indica que el número se incrementa en uno al insertarse un nuevo registro.

        \bTerm{tDefault}{Default} $:\ valor$
            Indica el valor por defecto que tendrá el atributo.

        \bTerm{tLength}{Length} $a-b$
            Indica la longitud que tendrá una \refElem{tVarchar}.

        \bTerm{tPositive}{Positivo} $a-b$
            Indica que el valor numérico debe ser positivo.
    \end{bGlosario}
    \end{quote}

\clearpage

\subsection{Entidades contempladas}

 Debido a que el esquema de la base de datos contiene las suficientes entidades para ser ilegible al ser mostradas en una imagen, se optó por
 enlistar en la imagen \ref{fig:BD-RDT} los nombres de las entidades que se contemplan en este proyecto y la especificación del esquema se va mostrando
 por cada módulo/submódulo en sus sendas secciones. Dicha imagen está dividida
 por cada submódulo contemplado, además de contar con una sección que lista las entidades de moodle.\\


% \noindent Las definiciones de cada una de las entidades de la imagen  \ref{fig:BD_RDT}, así como el esquema de base de datos de cada sección
% están especificados en cada una de ls siguientes secciones; \nameref{mod:exp}, \nameref{mod:recomp}, \nameref{mod:financ},
% \nameref{mod:pers}, \nameref{mod:comp} y \nameref{mod:seguim}.

    \addfigure{1}{analisis/diagrams/db_tables}{fig:BD-RDT}{Entidades contempladas en el proyecto}


    Las entidades que son de moodle tienen el prefijo \(\textbf{mdl\_}\), las entidades con el prefijo
 \(\textbf{gmdl\_}\) son las entidades diseñadas en este proyecto y por último, aquellas con el prefijo
 \(\textbf{gm}\) y sin guiones bajos, son aquellas que se utilizan para alvergar actividades de moodle.\\

 \noindent Todas las entidades creadas para este proyecto fueron diseñadas tomando en
 consideración que se debe extender el esquema de base de datos sin afectar las distintas
 funcionalidades que brinda de forma nativa. En los casos en los que se requería añadir
 atributos de moodle a una relación en particular, se agregaría una relación con los atributos
 requeridos, más la referencia a la relación del esquema de moodle.



\newcommand{\schemeName}[1]{%
% Esta macro se usa inmediatamente despues de una entidad
    \vspace{-1em}\hfill Nombre en el esquema: {\it #1}\ \ \par%
}

    
\subsection{Entidades de moodle}

Debido a que moodle cuenta con más de 400 entidades en su versión 3.5, se opta
por mostrar 2 subconjuntos que muestren las entidades que se utilizan para el proyecto.\\

\noindent El primer subconjunto es aquel que explica la forma en que moodle implementa los cursos,
secciones de curso, actividades, usuarios y roles (el cual se presenta en la figura \ref{fig:BD-ER-M1}),
mientras que el segundo subconjunto muestra como moodle maneja toda la
estructura de las preguntas creadas por el profesor y respondidas por el estudiante
(el cual se presenta en la figura \ref{fig:BD-ER-M2}).

\noindent El objetivo de ambos esquemas (\ref{fig:BD-ER-M1} y \ref{fig:BD-ER-M2}) es expresar la idea general que abarcan ambos subconjuntos.

\clearpage
\addfigure{0.7}{analisis/diagrams/db_module_structure}{fig:BD-ER-M1}{Esquema de la base de datos de moodle 'Cursos'}


\noindent Utilizando la figura \ref{fig:BD-ER-M1}, se obtuvieron las siguientes reglas y características que tiene moodle respecto a los usuarios en un curso y a la estructura de los cursos.
\begin{enumerate}
    \item Un usuario -{\it mdl\_user}- tiene un rol -{\it mdl\_role}- en un cierto contexto -{\it mdl\_context}-, cuyo  '{\it context\_level}' sea igual a cincuenta(50).
    \item Si el contexto '{\it context\_level}' es de 50, el atributo '{\it instance\_id}' hace referencia al atributo '{\it id}' de un curso -{\it mdl\_course}-.
    \item El curso -{\it mdl\_course}- tiene varias secciones -{\it mdl\_course\_sections}-.
    \item Cada seccion -{\it mdl\_course\_sections}- tiene varias actividades -{\it mdl\_course\_modules}- que pertenecen a un tipo de actividad -{\it mdl\_modules}-.
    \item Por cada registro en tipo de actividad -{\it mdl\_modules}-, se tiene una entidad que lleva el mismo nombre.
    \item El atributo '{\it instance\_id}' de una actividad  -{\it mdl\_course\_modules}- apunta a diferentes entidades. La entidad a la que apunta depende del nombre del tipo de actividad -{\it mdl\_modules}-.
    \item Un usuario -{\it mdl\_user}- se inscribió -{\it mdl\_user\_enrolments}- a un curso -{\it mdl\_course}-, por medio de un formato soportado de inscripción -{\it mdl\_enrol}-.
\end{enumerate}

\clearpage

 \addfigure{0.7}{analisis/diagrams/db_module_questions}{fig:BD-ER-M2}{Esquema de la base de datos de moodle 'Preguntas' }



\noindent Utilizando la figura \ref{fig:BD-ER-M2}, se obtuvieron las siguientes reglas y características que tiene moodle respecto a las preguntas.
\begin{enumerate}
    \item Las preguntas -{\it mdl\_question}- tienen versiones -{\it mdl\_question\_attempts}-.
    \item Una pregunta -{\it mdl\_question}- pertenece a un banco de preguntas -{\it mdl\_question\_categories}-.
    \item La versión de una pregunta -{\it mdl\_question\_attempts}- es contestada -{\it mdl\_question\_usages}- en un determinado contexto -{\it mdl\_context}-.
    \item Un usuario -{\it mdl\_user}- responde una versión de una pregunta -{\it mdl\_question\_attempt\_stepts}-.
    \item El responder una versión de una pregunta -{\it mdl\_question\_attempt\_stepts}- conlleva pasos\\ -{\it mdl\_question\_attempt\_stept\_data}-, los cuales son: cómo se muestra, si ya se terminó de responder y qué se respondió.
\end{enumerate}


 A continuación se presenta la especificación de las entidades del esquema de base
 de datos de moodle que son relevantes para el desarrollo de los módulos y submódulos
 de proyecto.

    \begin{cdtEntidad}{mdl-config-plugins}{Configuración de Plugin}{%
    Es una tabla del núcleo de moodle que almacena todas las configuraciones globales
    relacionadas a los plugins instalados, al iniciar moodle las configuraciones de los
    plugins instalados y habilitados se cargan en memoria.}

	    \brAttr{id}{Id}{tInt}{%
	        Es el dígito que representa el identificador único para una configuración
            específica de un plugin.\par

            \it Restricciones:
            \refElem{tPrimaryKey},
            \refElem{tAutoIncrement}.
        }

        \brAttr{plugin}{Plugin}{tVarchar}{%
            Cadena de caracteres del nombre identificador del plugin al cual pertenece
            la configuración.\par

            \it Restricciones:
            \refElem{tRequired},
            \refElem{tRange} (0,100),
            \refElem{tUniqueKey}
        }

        \brAttr{name}{Nombre}{tVarchar}{%
            Cadena de caracteres que representa el nombre de la configuración de un
            plugin en específico.\par

            \it Restricciones:
            \refElem{tUniqueKey},
            \refElem{tRange} (0,100),
            \refElem{tRequired}
        }

        \brAttr{value}{Valor}{tVarchar}{%
            Cadena que almacena el valor de una configuración perteneciente a alguno
            de los plugins instalados.\par

            \it Restricciones:
            \refElem{tRange} (0,4294967295),
            \refElem{tRequired}
        }
    \end{cdtEntidad}\schemeName{config\_plugins}

    \begin{cdtEntidad}{mdl-user}{Usuario de moodle}{%
    Es una tabla del núcleo de moodle que contiene toda la información que se
    almacena de los usuarios en la plataforma, independientemente del rol que
    estos contenga, esta relación contiene más de 53 atributos, sin embargo solo
    se detallan aquellos relevantes.}

	    \brAttr{id}{Id}{tInt}{%
	        Es el dígito que representa el identificador único para cada uno
            de los usuarios en moodle.\par

            \it Restricciones:
            \refElem{tPrimaryKey},
            \refElem{tAutoIncrement}.
        }
	    \brAttr{username}{nombre de usuario}{tVarchar}{%
	        .\par

            \it Restricciones:
            \refElem{tRequired},
            \refElem{tLength} 0-100
        }
	    \brAttr{password}{contraseña}{tVarchar}{%
	        .\par

            \it Restricciones:
            \refElem{tRequired},
            \refElem{tLength} 0-255.
        }
	    \brAttr{firstname}{nombre}{tVarchar}{%
	        .\par

            \it Restricciones:
            \refElem{tRequired},
            \refElem{tLength} 0-100
        }
	    \brAttr{lastname}{apellido}{tVarchar}{%
	        .\par

            \it Restricciones:
            \refElem{tRequired},
            \refElem{tLength} 0-100
        }
	    \brAttr{email}{correo}{tVarchar}{%
	        .\par

            \it Restricciones:
            \refElem{tRequired},
            \refElem{tLength} 0-100
        }
	    \brAttr{lastaccess}{último registro}{tInt}{%
	        .\par

            \it Restricciones:
            \refElem{tRequired},
            \refElem{tLength} 10
        }
	    \brAttr{city}{ciudad}{tVarchar}{%
	        .\par

            \it Restricciones:
            \refElem{tRequired},
            \refElem{tLength} 0-120
        }
	    \brAttr{country}{pais}{tVarchar}{%
	        .\par

            \it Restricciones:
            \refElem{tRequired},
            \refElem{tLength} 2
        }

    \end{cdtEntidad}\schemeName{user}

    \begin{cdtEntidad}{mdl-course}{Curso de moodle}{%
    Es una tabla del núcleo de moodle que contiene la información principal de cada
    curso registrado en moodle. Esta entidad contiene 31 atributos, a continuación se
    detallan los atributos relevantes para la especificación de este proyecto.}

	    \brAttr{id}{Id}{tInt}{%
	        Es el dígito que representa al identificador único para cada uno
            de los cursos en moodle.\par

            \it Restricciones:
            \refElem{tPrimaryKey},
            \refElem{tAutoIncrement}.
        }

	    \brAttr{format}{formato}{tVarchar}{%
	        Es el dígito que representa al identificador único para cada uno
            de los cursos en moodle.\par

            \it Restricciones:
            \refElem{tRequired}.
            \refElem{tDefault} topics,
            \refElem{tLength} 0-21.
        }

	    \brAttr{fullname}{nombre completo}{tVarchar}{%
	        Es el nombre completo que se le asigna al curso.\par

            \it Restricciones:
            \refElem{tRequired}.
            \refElem{tLength} 0-21.
        }

	    \brAttr{shortname}{nombre corto}{tVarchar}{%
            Es el nombre corto que se le asigna al curso.\par

            \it Restricciones:
            \refElem{tRequired}.
            \refElem{tLength} 0-21.
        }

    \end{cdtEntidad}\schemeName{course}

    \begin{cdtEntidad}{mdl-course-section}{Sección del curso de moodle}{%
    }
	    \brAttr{id}{Id}{tInt}{%
	        Es el dígito que representa al identificador único para cada sección
            de los cursos en moodle.\par

            \it Restricciones:
            \refElem{tPrimaryKey},
            \refElem{tAutoIncrement}.
        }

        \brAttr{name}{nombre}{tVarchar}{%
	        Es el dígito nombre que permite identificar a una sección dentro de un curso
            en moodle.\par

            \it Restricciones: ...
        }
    \end{cdtEntidad}\schemeName{course\_sections}

    \begin{cdtEntidad}{mdl-course-format-options}{Opciones del formato del curso}{%
    }
	    \brAttr{id}{Id}{tInt}{%
	        Es el dígito que representa al identificador único para cada uno
            de los cursos en moodle.\par

            \it Restricciones:
            \refElem{tPrimaryKey},
            \refElem{tAutoIncrement}.
        }

	    \brAttr{courseid}{Id}{tInt}{%
	        Es el dígito que representa al identificador único para cada uno
            de los cursos en moodle.\par

            \it Restricciones:
            \refElem{tForeignKey},
            \refElem{tRequired}
        }

	    \brAttr{format}{formato}{tVarchar}{%
	        Es el dígito que representa al identificador único para cada uno
            de los cursos en moodle.\par

            \it Restricciones:
            \refElem{tRequired}.
            \refElem{tDefault} topics,
            \refElem{tLength} 0-21.
        }

	    \brAttr{name}{opcion}{tVarchar}{%
	        Es el dígito que representa al identificador único para cada uno
            de los cursos en moodle.\par

            \it Restricciones:
            \refElem{tPrimaryKey},
            \refElem{tLength} 0-100
        }

	    \brAttr{value}{valor}{tVarchar}{%
	        Es el dígito que representa al identificador único para cada uno
            de los cursos en moodle.\par

            \it Restricciones:
            \refElem{tRequired}
        }

    \end{cdtEntidad}\schemeName{course\_format\_options}

    \begin{cdtEntidad}{mdl-course-category}{Categoria de curso}{%
      .}
        \brAttr{id}{id}{tInt}{%
        .}
        \brAttr{name}{nombre}{tInt}{%
        .}

    \end{cdtEntidad}\schemeName{course\_category}

    \begin{cdtEntidad}{mdl-course-module}{Actividad del curso}{%
    .}
	    \brAttr{id}{Id}{tInt}{%
	        Es el dígito que representa al identificador único para cada uno
            de los cursos en moodle.\par

            \it Restricciones:
            \refElem{tPrimaryKey},
            \refElem{tAutoIncrement}.
        }
        \brAttr{course}{curso}{tInt}{%
        .}
        \brAttr{module}{actividad}{tInt}{%
        .}
        \brAttr{section}{sección}{tInt}{%
        .}
    \end{cdtEntidad}\schemeName{course\_module}

    \begin{cdtEntidad}{mdl-course-module-completion}{Actividad del curso para alumno}{%
    .}
	    \brAttr{id}{Id}{tInt}{%
	        Es el dígito que representa al identificador único para cada uno
            de los cursos en moodle.\par

            \it Restricciones:
            \refElem{tPrimaryKey},
            \refElem{tAutoIncrement}.
        }
        \brAttr{coursemoduleid}{actividad}{tInt}{%
        .}
        \brAttr{userid}{usuario}{tInt}{%
        .}
        \brAttr{completionstate}{completitud}{tBoolean}{%
        .}
    \end{cdtEntidad}\schemeName{course\_module}

    \begin{cdtEntidad}{Plugin}{Plugin}{%
    La forma en que moodle obtiene información acerca de los plugins es analizando
    los archivos internos de cada uno, a pesar de que los plugins no forman parte
    del esquema de base de datos, si forman parte del modelo de información que
    utiliza Moodle.}

	    \brAttr{componente}{Componente}{tVarchar}{%
	        Cadena compuesta por el tipo de plugin y el nombre del mismo, que
            representa a la clase principal del plugin que contiene los métodos
            principales del plugin.\par

            \it Restricciones: Ninguna
        }

	    \brAttr{pluginname}{Nombre}{tVarchar}{%
	        Es el nombre del plugin obtenido de los archivos de
            internacionalización presentes en el plugin, el valor de esta cadena
            depende del lenguaje seleccionado en moodle.\par

            \it Restricciones: Ninguna
        }

	    \brAttr{fullpath}{Ruta absoluta}{tPath}{%
	        La ruta absoluta de un plugin denota la ubicación del plugin en el
            sistema de archivos, esta ruta está compuesta por la ruta absoluta
            de la instalación de moodle, la carpeta correspondiente al tipo del
            plugin y el nombre del plugin.\par

            \it Restricciones: Formato ``/path/to/moodle/plugintype/pluginname''
        }

	    \brAttr{path}{Ruta relativa}{tPath}{%
	        La ruta relativa denota la ubicación del plugin dentro de la carpeta
            donde se encuentran los archivos de moodle, esta ruta está compuesta
            por la carpeta correspondiente al tipo del plugin y el nombre del
            plugin.\par

            \it Restricciones: Formato ``plugintype/pluginname''
        }

	    \brAttr{version}{Versión}{tVersion}{%
	        Numero entero de longitud de 10 dígitos que representa la versión del
            plugin.\par

            \it Restricciones: Ninguna adicional al tipo de dato
        }

	    \brAttr{moodle}{Versión de Moodle}{tVersion}{%
	        Número entero de longitud de 10 dígitos que representa la versión de
            moodle en la que se puede instalar el plugin.\par

            \it Restricciones: Ninguna adicional al tipo de dato
        }

        \brAttr{dependencies}{Dependencias}{tObject}{%
            Objeto que almacena un conjunto de claves con sus respectivos valores,
            donde cada clave representa el nombre del componente del plugin y el valor
            es la \refElem{Plugin.version} requerida del mismo.

            \it Restricciones: Ninguna
        }

        \brAttr{icon}{ícono}{tImage}{%
            Imagen para el ícono del plugin, debe estar contenida en el directorio
            {\it pix/} del plugin y tener como nombre {\it icon.png} o {\it icon.svg},
            moodle recomienda tener ambos archivos por si los navegadores no soportan
            algún tipo de archivo \cite{moodlePluginfiles}.\par

            \it Restricciones: El nombre debe ser icono con extensiones png o svg
        }

    \end{cdtEntidad}


    
En este capítulo se especifica cómo se implementó el modelo de información utilizando los complementos de moodle
y la especificación del análisis, diseño, desarrollo y pruebas
por cada uno de los módulos del proyecto. \\

\noindent Moodle permite definir una base de datos mediante los archivos db/install.xml, db/upgrade.php y db/install.php de cada complemento.
Esto permite que cada complemento pueda especificar las entidades que estos utilizen por separado. 
Sin embargo, se optó por crear un complemento cuya función sea la implementación del modelo de información
y que en caso de que un complemento necesite usar alguna entidad, este pueda sin problemas acceder a ella 
sin la necesidad de ocuparse de la creación o las versiones que se tengan
del modelo de información.\\

\noindent En otras palabras, la definciión del modelo de información está centralizada en
un complemento del cual todos los otros complementos dependen. Esto se logra gracias al archivo version.php que establece
las dependencias que tiene un complemento de moodle. Este componente que crea la base de datos,
siguiendo el modelo de infomación es denominado gamedlemaster.\\

\noindent El complemento gamedlemaster también se encarga de definir
los eventos con los que los complementos se puedan comunicar entre si.


    
\subsection{Análisis}

 Este apartado contiene el análisis requerido para la elaboración de módulo de competencia,
 contiene la especificación del alcance de este módulo, la descripción de las funcionalidades
 a desarrollar, la reglas de negocio que rigen el comportamiento del módulo, y por último la
 especificación de los casos de uso a los que brinda soporte.

%\subsubsection{Submódulo de competencia 1 contra 1}
%\subsubsection{Funcionalidades}

\subsubsection{Reglas de negocio} %========================================================

 En esta sección se especifican todas las reglas de negocio relevantes para el módulo de
 experiencia. Las reglas de negocio que establece moodle son diferenciadas por tener la letra {\it M}
 antecediendo al número consecutivo en su identificador.

    %
\subsection{Entidades de moodle}

Debido a que moodle cuenta con más de 400 entidades en su versión 3.5, se opta
por mostrar 2 subconjuntos que muestren las entidades que se utilizan para el proyecto.\\

\noindent El primer subconjunto es aquel que explica la forma en que moodle implementa los cursos,
secciones de curso, actividades, usuarios y roles (el cual se presenta en la figura \ref{fig:BD-ER-M1}),
mientras que el segundo subconjunto muestra como moodle maneja toda la
estructura de las preguntas creadas por el profesor y respondidas por el estudiante
(el cual se presenta en la figura \ref{fig:BD-ER-M2}).

\noindent El objetivo de ambos esquemas (\ref{fig:BD-ER-M1} y \ref{fig:BD-ER-M2}) es expresar la idea general que abarcan ambos subconjuntos.

\clearpage
\addfigure{0.7}{analisis/diagrams/db_module_structure}{fig:BD-ER-M1}{Esquema de la base de datos de moodle 'Cursos'}


\noindent Utilizando la figura \ref{fig:BD-ER-M1}, se obtuvieron las siguientes reglas y características que tiene moodle respecto a los usuarios en un curso y a la estructura de los cursos.
\begin{enumerate}
    \item Un usuario -{\it mdl\_user}- tiene un rol -{\it mdl\_role}- en un cierto contexto -{\it mdl\_context}-, cuyo  '{\it context\_level}' sea igual a cincuenta(50).
    \item Si el contexto '{\it context\_level}' es de 50, el atributo '{\it instance\_id}' hace referencia al atributo '{\it id}' de un curso -{\it mdl\_course}-.
    \item El curso -{\it mdl\_course}- tiene varias secciones -{\it mdl\_course\_sections}-.
    \item Cada seccion -{\it mdl\_course\_sections}- tiene varias actividades -{\it mdl\_course\_modules}- que pertenecen a un tipo de actividad -{\it mdl\_modules}-.
    \item Por cada registro en tipo de actividad -{\it mdl\_modules}-, se tiene una entidad que lleva el mismo nombre.
    \item El atributo '{\it instance\_id}' de una actividad  -{\it mdl\_course\_modules}- apunta a diferentes entidades. La entidad a la que apunta depende del nombre del tipo de actividad -{\it mdl\_modules}-.
    \item Un usuario -{\it mdl\_user}- se inscribió -{\it mdl\_user\_enrolments}- a un curso -{\it mdl\_course}-, por medio de un formato soportado de inscripción -{\it mdl\_enrol}-.
\end{enumerate}

\clearpage

 \addfigure{0.7}{analisis/diagrams/db_module_questions}{fig:BD-ER-M2}{Esquema de la base de datos de moodle 'Preguntas' }



\noindent Utilizando la figura \ref{fig:BD-ER-M2}, se obtuvieron las siguientes reglas y características que tiene moodle respecto a las preguntas.
\begin{enumerate}
    \item Las preguntas -{\it mdl\_question}- tienen versiones -{\it mdl\_question\_attempts}-.
    \item Una pregunta -{\it mdl\_question}- pertenece a un banco de preguntas -{\it mdl\_question\_categories}-.
    \item La versión de una pregunta -{\it mdl\_question\_attempts}- es contestada -{\it mdl\_question\_usages}- en un determinado contexto -{\it mdl\_context}-.
    \item Un usuario -{\it mdl\_user}- responde una versión de una pregunta -{\it mdl\_question\_attempt\_stepts}-.
    \item El responder una versión de una pregunta -{\it mdl\_question\_attempt\_stepts}- conlleva pasos\\ -{\it mdl\_question\_attempt\_stept\_data}-, los cuales son: cómo se muestra, si ya se terminó de responder y qué se respondió.
\end{enumerate}


 A continuación se presenta la especificación de las entidades del esquema de base
 de datos de moodle que son relevantes para el desarrollo de los módulos y submódulos
 de proyecto.

    \begin{cdtEntidad}{mdl-config-plugins}{Configuración de Plugin}{%
    Es una tabla del núcleo de moodle que almacena todas las configuraciones globales
    relacionadas a los plugins instalados, al iniciar moodle las configuraciones de los
    plugins instalados y habilitados se cargan en memoria.}

	    \brAttr{id}{Id}{tInt}{%
	        Es el dígito que representa el identificador único para una configuración
            específica de un plugin.\par

            \it Restricciones:
            \refElem{tPrimaryKey},
            \refElem{tAutoIncrement}.
        }

        \brAttr{plugin}{Plugin}{tVarchar}{%
            Cadena de caracteres del nombre identificador del plugin al cual pertenece
            la configuración.\par

            \it Restricciones:
            \refElem{tRequired},
            \refElem{tRange} (0,100),
            \refElem{tUniqueKey}
        }

        \brAttr{name}{Nombre}{tVarchar}{%
            Cadena de caracteres que representa el nombre de la configuración de un
            plugin en específico.\par

            \it Restricciones:
            \refElem{tUniqueKey},
            \refElem{tRange} (0,100),
            \refElem{tRequired}
        }

        \brAttr{value}{Valor}{tVarchar}{%
            Cadena que almacena el valor de una configuración perteneciente a alguno
            de los plugins instalados.\par

            \it Restricciones:
            \refElem{tRange} (0,4294967295),
            \refElem{tRequired}
        }
    \end{cdtEntidad}\schemeName{config\_plugins}

    \begin{cdtEntidad}{mdl-user}{Usuario de moodle}{%
    Es una tabla del núcleo de moodle que contiene toda la información que se
    almacena de los usuarios en la plataforma, independientemente del rol que
    estos contenga, esta relación contiene más de 53 atributos, sin embargo solo
    se detallan aquellos relevantes.}

	    \brAttr{id}{Id}{tInt}{%
	        Es el dígito que representa el identificador único para cada uno
            de los usuarios en moodle.\par

            \it Restricciones:
            \refElem{tPrimaryKey},
            \refElem{tAutoIncrement}.
        }
	    \brAttr{username}{nombre de usuario}{tVarchar}{%
	        .\par

            \it Restricciones:
            \refElem{tRequired},
            \refElem{tLength} 0-100
        }
	    \brAttr{password}{contraseña}{tVarchar}{%
	        .\par

            \it Restricciones:
            \refElem{tRequired},
            \refElem{tLength} 0-255.
        }
	    \brAttr{firstname}{nombre}{tVarchar}{%
	        .\par

            \it Restricciones:
            \refElem{tRequired},
            \refElem{tLength} 0-100
        }
	    \brAttr{lastname}{apellido}{tVarchar}{%
	        .\par

            \it Restricciones:
            \refElem{tRequired},
            \refElem{tLength} 0-100
        }
	    \brAttr{email}{correo}{tVarchar}{%
	        .\par

            \it Restricciones:
            \refElem{tRequired},
            \refElem{tLength} 0-100
        }
	    \brAttr{lastaccess}{último registro}{tInt}{%
	        .\par

            \it Restricciones:
            \refElem{tRequired},
            \refElem{tLength} 10
        }
	    \brAttr{city}{ciudad}{tVarchar}{%
	        .\par

            \it Restricciones:
            \refElem{tRequired},
            \refElem{tLength} 0-120
        }
	    \brAttr{country}{pais}{tVarchar}{%
	        .\par

            \it Restricciones:
            \refElem{tRequired},
            \refElem{tLength} 2
        }

    \end{cdtEntidad}\schemeName{user}

    \begin{cdtEntidad}{mdl-course}{Curso de moodle}{%
    Es una tabla del núcleo de moodle que contiene la información principal de cada
    curso registrado en moodle. Esta entidad contiene 31 atributos, a continuación se
    detallan los atributos relevantes para la especificación de este proyecto.}

	    \brAttr{id}{Id}{tInt}{%
	        Es el dígito que representa al identificador único para cada uno
            de los cursos en moodle.\par

            \it Restricciones:
            \refElem{tPrimaryKey},
            \refElem{tAutoIncrement}.
        }

	    \brAttr{format}{formato}{tVarchar}{%
	        Es el dígito que representa al identificador único para cada uno
            de los cursos en moodle.\par

            \it Restricciones:
            \refElem{tRequired}.
            \refElem{tDefault} topics,
            \refElem{tLength} 0-21.
        }

	    \brAttr{fullname}{nombre completo}{tVarchar}{%
	        Es el nombre completo que se le asigna al curso.\par

            \it Restricciones:
            \refElem{tRequired}.
            \refElem{tLength} 0-21.
        }

	    \brAttr{shortname}{nombre corto}{tVarchar}{%
            Es el nombre corto que se le asigna al curso.\par

            \it Restricciones:
            \refElem{tRequired}.
            \refElem{tLength} 0-21.
        }

    \end{cdtEntidad}\schemeName{course}

    \begin{cdtEntidad}{mdl-course-section}{Sección del curso de moodle}{%
    }
	    \brAttr{id}{Id}{tInt}{%
	        Es el dígito que representa al identificador único para cada sección
            de los cursos en moodle.\par

            \it Restricciones:
            \refElem{tPrimaryKey},
            \refElem{tAutoIncrement}.
        }

        \brAttr{name}{nombre}{tVarchar}{%
	        Es el dígito nombre que permite identificar a una sección dentro de un curso
            en moodle.\par

            \it Restricciones: ...
        }
    \end{cdtEntidad}\schemeName{course\_sections}

    \begin{cdtEntidad}{mdl-course-format-options}{Opciones del formato del curso}{%
    }
	    \brAttr{id}{Id}{tInt}{%
	        Es el dígito que representa al identificador único para cada uno
            de los cursos en moodle.\par

            \it Restricciones:
            \refElem{tPrimaryKey},
            \refElem{tAutoIncrement}.
        }

	    \brAttr{courseid}{Id}{tInt}{%
	        Es el dígito que representa al identificador único para cada uno
            de los cursos en moodle.\par

            \it Restricciones:
            \refElem{tForeignKey},
            \refElem{tRequired}
        }

	    \brAttr{format}{formato}{tVarchar}{%
	        Es el dígito que representa al identificador único para cada uno
            de los cursos en moodle.\par

            \it Restricciones:
            \refElem{tRequired}.
            \refElem{tDefault} topics,
            \refElem{tLength} 0-21.
        }

	    \brAttr{name}{opcion}{tVarchar}{%
	        Es el dígito que representa al identificador único para cada uno
            de los cursos en moodle.\par

            \it Restricciones:
            \refElem{tPrimaryKey},
            \refElem{tLength} 0-100
        }

	    \brAttr{value}{valor}{tVarchar}{%
	        Es el dígito que representa al identificador único para cada uno
            de los cursos en moodle.\par

            \it Restricciones:
            \refElem{tRequired}
        }

    \end{cdtEntidad}\schemeName{course\_format\_options}

    \begin{cdtEntidad}{mdl-course-category}{Categoria de curso}{%
      .}
        \brAttr{id}{id}{tInt}{%
        .}
        \brAttr{name}{nombre}{tInt}{%
        .}

    \end{cdtEntidad}\schemeName{course\_category}

    \begin{cdtEntidad}{mdl-course-module}{Actividad del curso}{%
    .}
	    \brAttr{id}{Id}{tInt}{%
	        Es el dígito que representa al identificador único para cada uno
            de los cursos en moodle.\par

            \it Restricciones:
            \refElem{tPrimaryKey},
            \refElem{tAutoIncrement}.
        }
        \brAttr{course}{curso}{tInt}{%
        .}
        \brAttr{module}{actividad}{tInt}{%
        .}
        \brAttr{section}{sección}{tInt}{%
        .}
    \end{cdtEntidad}\schemeName{course\_module}

    \begin{cdtEntidad}{mdl-course-module-completion}{Actividad del curso para alumno}{%
    .}
	    \brAttr{id}{Id}{tInt}{%
	        Es el dígito que representa al identificador único para cada uno
            de los cursos en moodle.\par

            \it Restricciones:
            \refElem{tPrimaryKey},
            \refElem{tAutoIncrement}.
        }
        \brAttr{coursemoduleid}{actividad}{tInt}{%
        .}
        \brAttr{userid}{usuario}{tInt}{%
        .}
        \brAttr{completionstate}{completitud}{tBoolean}{%
        .}
    \end{cdtEntidad}\schemeName{course\_module}

    \begin{cdtEntidad}{Plugin}{Plugin}{%
    La forma en que moodle obtiene información acerca de los plugins es analizando
    los archivos internos de cada uno, a pesar de que los plugins no forman parte
    del esquema de base de datos, si forman parte del modelo de información que
    utiliza Moodle.}

	    \brAttr{componente}{Componente}{tVarchar}{%
	        Cadena compuesta por el tipo de plugin y el nombre del mismo, que
            representa a la clase principal del plugin que contiene los métodos
            principales del plugin.\par

            \it Restricciones: Ninguna
        }

	    \brAttr{pluginname}{Nombre}{tVarchar}{%
	        Es el nombre del plugin obtenido de los archivos de
            internacionalización presentes en el plugin, el valor de esta cadena
            depende del lenguaje seleccionado en moodle.\par

            \it Restricciones: Ninguna
        }

	    \brAttr{fullpath}{Ruta absoluta}{tPath}{%
	        La ruta absoluta de un plugin denota la ubicación del plugin en el
            sistema de archivos, esta ruta está compuesta por la ruta absoluta
            de la instalación de moodle, la carpeta correspondiente al tipo del
            plugin y el nombre del plugin.\par

            \it Restricciones: Formato ``/path/to/moodle/plugintype/pluginname''
        }

	    \brAttr{path}{Ruta relativa}{tPath}{%
	        La ruta relativa denota la ubicación del plugin dentro de la carpeta
            donde se encuentran los archivos de moodle, esta ruta está compuesta
            por la carpeta correspondiente al tipo del plugin y el nombre del
            plugin.\par

            \it Restricciones: Formato ``plugintype/pluginname''
        }

	    \brAttr{version}{Versión}{tVersion}{%
	        Numero entero de longitud de 10 dígitos que representa la versión del
            plugin.\par

            \it Restricciones: Ninguna adicional al tipo de dato
        }

	    \brAttr{moodle}{Versión de Moodle}{tVersion}{%
	        Número entero de longitud de 10 dígitos que representa la versión de
            moodle en la que se puede instalar el plugin.\par

            \it Restricciones: Ninguna adicional al tipo de dato
        }

        \brAttr{dependencies}{Dependencias}{tObject}{%
            Objeto que almacena un conjunto de claves con sus respectivos valores,
            donde cada clave representa el nombre del componente del plugin y el valor
            es la \refElem{Plugin.version} requerida del mismo.

            \it Restricciones: Ninguna
        }

        \brAttr{icon}{ícono}{tImage}{%
            Imagen para el ícono del plugin, debe estar contenida en el directorio
            {\it pix/} del plugin y tener como nombre {\it icon.png} o {\it icon.svg},
            moodle recomienda tener ambos archivos por si los navegadores no soportan
            algún tipo de archivo \cite{moodlePluginfiles}.\par

            \it Restricciones: El nombre debe ser icono con extensiones png o svg
        }

    \end{cdtEntidad}
 % Archivo de plugin
    
\begin{BusinessRule}[%
Autor/Ricard Naranjo Polit,%
Version/0.1,%
Estado/revision]%
%
{BR-C01}{Restricciones del tiempo que se tiene para completar el desafío una vez iniciado en competencia uno contra uno}
 % El archivo de instalación debe ser un archivo ZIP, el cual debe contener
 % exactamente un directorio que coincida con el nombre del plugin.
     \BRitem[control]{Revisor}{Sin asignar.}

 \BRsection[control]{Atributos}

    \BRitem[admin]{Clase}{\bcCondition}%
    %\BRitem[admin]{Clase}{\bcIntegridad}%
    %\BRitem[admin]{Clase}{\bcAutorizacion}%
    %\BRitem[admin]{Clase}{\bcDerivacion}%

    \BRitem[admin]{Tipo}{\btTimer}%
    %\BRitem[admin]{Tipo}{\btTimer}%
    %\BRitem[admin]{Tipo}{\btExecutive}%

    \BRitem[admin]{Nivel}{\blControlling}
    %\BRitem[admin]{Nivel}{\blInfluencing}

    \BRitem{Descripción}{%
        Cuando un usuario desafía a un \refElem{aEstudiante} o acepta un desafío tiene un día para completarlo.
        Al terminar este tiempo se terminará el desafió y se le pondrá una puntuación de 0.
        % debido a que se el directorio donde se guardará será el directorio
        % para almacenar las imágenes del plugin.
    }

    \BRitem{Ejemplo positivo}{\hfill\par%
        \begin{itemize}
        \item El usuario desafía a un estudiante y termina el desafío en menos de un día.

        \item El usuario acepta un desafío y lo termina en menos de un día.
        \end{itemize}
    }

    \BRitem{Ejemplo negativo}{\hfill\par%
        \begin{itemize}
          \item El usuario desafía a un estudiante y no termina el desafío en menos de un día.

          \item El usuario acepta un desafío y no lo termina en menos de un día.
        \end{itemize}
    }%

 \end{BusinessRule}
 % Restricciones sobre de imagen del nivel.
    
\begin{BusinessRule}[%
Autor/Ricard Naranjo Polit,%
Version/0.1,%
Estado/revision]%
%
{BR-C02}{Un usuario no puede desafiar a otro con el que tenga un desafío pendiente}
 % El archivo de instalación debe ser un archivo ZIP, el cual debe contener
 % exactamente un directorio que coincida con el nombre del plugin.
     \BRitem[control]{Revisor}{Sin asignar.}

 \BRsection[control]{Atributos}

    \BRitem[admin]{Clase}{\bcCondition}%
    %\BRitem[admin]{Clase}{\bcIntegridad}%
    %\BRitem[admin]{Clase}{\bcAutorizacion}%
    %\BRitem[admin]{Clase}{\bcDerivacion}%

    \BRitem[admin]{Tipo}{\btEnabler}%
    %\BRitem[admin]{Tipo}{\btTimer}%
    %\BRitem[admin]{Tipo}{\btExecutive}%

    \BRitem[admin]{Nivel}{\blControlling}
    %\BRitem[admin]{Nivel}{\blInfluencing}

    \BRitem{Descripción}{%
        Cuando un usuario desafía a un \refElem{aEstudiante} no lo podrá volver a desafiar hasta que el desafiante
        y desafiado terminen hayan completado la competencia.
        % debido a que se el directorio donde se guardará será el directorio
        % para almacenar las imágenes del plugin.
    }

    \BRitem{Ejemplo positivo}{\hfill\par%
        \begin{itemize}
        \item El usuario desafía a un estudiante, los dos terminan la competencia y el usuario vuelve a desafiar al mismo estudiante.

        \end{itemize}
    }

    \BRitem{Ejemplo negativo}{\hfill\par%
        \begin{itemize}
          \item El usuario desafía a un estudiante y no alguno de los dos no termina la competencia,
          el usuario no puede volver a desafiar al mismo estudiante.

        \end{itemize}
    }%

 \end{BusinessRule}
 % Permanencia del nivel de comperiencia.
    %
\begin{BusinessRule}[%
Autor/Daniel Isai Ortega Zúñiga,%
Version/0.1,%
Estado/revision]%
%
{BR-E03}{Tipos de Incremento}
    \BRitem[control]{Revisor}{Sin asignar.}

 \BRsection[control]{Atributos}
    
    \BRitem[admin]{Clase}{\bcCondition}%
    %\BRitem[admin]{Clase}{\bcIntegridad}%
    %\BRitem[admin]{Clase}{\bcAutorizacion}%
    %\BRitem[admin]{Clase}{\bcDerivacion}%
        
    \BRitem[admin]{Tipo}{\btEnabler}%
    %\BRitem[admin]{Tipo}{\btTimer}%
    %\BRitem[admin]{Tipo}{\btExecutive}%
        
    \BRitem[admin]{Nivel}{\blControlling}
    %\BRitem[admin]{Nivel}{\blInfluencing}
    
    \BRitem{Descripción}{%
    Cuando se modifiquen el \refElem{xp-scheme-settings} o la \refElem{levelXP} de las
    \refElem{xp-scheme-settings}
    }

    \BRitem{Ejemplo positivo}{\hfill\par%
        \begin{itemize}
        \item ...
        \end{itemize}
    }

    \BRitem{Ejemplo negativo}{\hfill\par%
        \begin{itemize}
        \item ...
        \end{itemize}
    }% 
    
 \end{BusinessRule}
 % Tipos de incremento
    %\begin{BusinessRule}[%
Autor/Daniel Isai Ortega Zúñiga,%
Version/0.1,%
Estado/revision]%
%
{BR-E04}{Calculo de experiencia del nivel con incremento porcentual}
    \BRitem[control]{Revisor}{Sin asignar.}

 \BRsection[control]{Atributos}
    
    \BRitem[admin]{Clase}{\bcCondition}%
    %\BRitem[admin]{Clase}{\bcIntegridad}%
    %\BRitem[admin]{Clase}{\bcAutorizacion}%
    %\BRitem[admin]{Clase}{\bcDerivacion}%
        
    \BRitem[admin]{Tipo}{\btEnabler}%
    %\BRitem[admin]{Tipo}{\btTimer}%
    %\BRitem[admin]{Tipo}{\btExecutive}%
        
    \BRitem[admin]{Nivel}{\blControlling}
    %\BRitem[admin]{Nivel}{\blInfluencing}
    
    \BRitem{Descripción}{%
        El calculo para obtener la experiencia del nivel $i$ uando el tipo de
        incremento es porcentual está dado por la siguiente fórmula: Sea {\it exp()}
        la función que optiene la experiencia de un nivel en específico, sea tambien
        $i$ el nivel del cual se calcula la experiencia, sea $inc$ el factor de
        incremento de nivel a nivel, y finalmente sea $round()$ una función de
        redondeo a números enteros, entonces:

            $$ exp(i) = round( exp(1) * (inc)^{(i-1)})$$
    }

%   \BRitem{Sentencia}{%
%       Si $fecha$ 
%   }%

    \BRitem{Ejemplo positivo}{\hfill\par%
        \begin{itemize}
        \item La experiencia requerida para superar el nivel 1 es de 2000 puntos y el
              factor de incremento entre los niveles es 1.1, entonces la experiencia
              requerida para pasar el nivel 5 es de 2928 puntos.
        \end{itemize}
    }

    \BRitem{Ejemplo negativo}{\hfill\par%
        \begin{itemize}
        \item La experinecia requerida para superar el nivel 1 es de 2000 puntos y el
              factor de incremento entre los niveles es 1.1, entonces la experiencia
              requerida para pasar el nivel 5 es de 2300 puntos.
        \end{itemize}
    }% 
    
\end{BusinessRule}
 % Incremento porcentual
    %\begin{BusinessRule}[%
Autor/Daniel Isai Ortega Zúñiga,%
Version/0.1,%
Estado/revision]%
%
{BR-E05}{Cálculo de experiencia del nivel con incremento linea} % Cuando están iniciados
    \BRitem[control]{Revisor}{Sin asignar.}

 \BRsection[control]{Atributos}
    
    \BRitem[admin]{Clase}{\bcCondition}%
    %\BRitem[admin]{Clase}{\bcIntegridad}%
    %\BRitem[admin]{Clase}{\bcAutorizacion}%
    %\BRitem[admin]{Clase}{\bcDerivacion}%
        
    \BRitem[admin]{Tipo}{\btEnabler}%
    %\BRitem[admin]{Tipo}{\btTimer}%
    %\BRitem[admin]{Tipo}{\btExecutive}%
        
    \BRitem[admin]{Nivel}{\blControlling}
    %\BRitem[admin]{Nivel}{\blInfluencing}
    
    \BRitem{Descripción}{%
    }

%   \BRitem{Sentencia}{%
%       Si $fecha$ 
%   }%

    \BRitem{Ejemplo positivo}{\hfill\par%
        \begin{itemize}
        \item ...
        \end{itemize}
    }

    \BRitem{Ejemplo negativo}{\hfill\par%
        \begin{itemize}
        \item ...
        \end{itemize}
    }% 
    
\end{BusinessRule}
 % Incremento lineal
    %\begin{BusinessRule}[%
Autor/Daniel Isai Ortega Zúñiga,%
Version/0.1,%
Estado/revision]%
%
{BR-E06}{Eliminación de cursos gamificados} % Cuando están iniciados
    \BRitem[control]{Revisor}{Sin asignar.}

 \BRsection[control]{Atributos}

    \BRitem[admin]{Clase}{\bcCondition}%
    %\BRitem[admin]{Clase}{\bcIntegridad}%
    %\BRitem[admin]{Clase}{\bcAutorizacion}%
    %\BRitem[admin]{Clase}{\bcDerivacion}%

    \BRitem[admin]{Tipo}{\btEnabler}%
    %\BRitem[admin]{Tipo}{\btTimer}%
    %\BRitem[admin]{Tipo}{\btExecutive}%

    \BRitem[admin]{Nivel}{\blControlling}
    %\BRitem[admin]{Nivel}{\blInfluencing}

    \BRitem{Descripción}{%
        Debido a que el \refElem{mdl-course.format} por defecto para los cursos de
        moodle es el formato de tópicos/temas El formato de curso gamificado extiende
        las funcionalidades de este formato para facilitar la migración de un curso
        gamificado a uno no gamificado y viceversa. % TODO Pasar a analisis.
        La desinstalación del módulo de experiencia implica que los cursos con el
        \refElem{xp-course.format} gamificado (gamedle) se migren a cursos no
        gamificados, por compatibilidad en este migración se deben realizar las
        siguientes acciones de forma transaccional:

        \begin{itemize}
        \item El formato de los \refElem[cursos gamificados]{xp-course} debe
              cambiarse por el formato por defecto que tienen los cursos en moodle
              el cual es el de tópicos/temas.

        \item Se deben eliminar las \refElem[opciones del formato del curso]%
              {mdl-course-format-options} gamificado (gamedle).

        \item Se deben establecer los valores para las opciones del formato de
              tópicos/temas, que tienen por \refElem[nombre]%
              {mdl-course-format-options.name} secciones ocultas y aspecto del curso.
        \end{itemize}
    }

%   \BRitem{Sentencia}{%
%       Si $fecha$
%   }%

    \BRitem{Ejemplo positivo}{\hfill\par%
        \begin{itemize}
        \item Cuando se desinstala los plugins correspondientes al módulo de
              experiencia los cursos que estan vinculados con el formato gamificado
              son cambiados al formato de topicos/temas (formato por defecto de
              moodle), y también se establecen las opciones equivalentes al formato
              gamificado.
        \end{itemize}
    }

    \BRitem{Ejemplo negativo}{\hfill\par%
        \begin{itemize}
        \item Cuando se desinstala los plugins correspondientes al módulo de
              experiencia los cursos que estan vinculados con el formato gamificado
              no son cambiados al formato de topicos/temas ocasionando inconsistencia
              entre los cursos y los formatos.
        \end{itemize}
    }%

\end{BusinessRule}
 % Eliminación de cursos gamificados
    %\begin{BusinessRule}[%
Autor/Daniel Isai Ortega Zúñiga,%
Version/0.1,%
Estado/edicion]%
%
{BR-E07}{Valores iniciales de experiencia}

     \BRitem[control]{Revisada por}{Pendiente.}

 \BRsection[control]{Atributos}
    % Clases: \bcCondition, \bcIntegridad, \bcAutorization o \bcDerivation
    % Tipos: \btEnabler, \btTimer o \btExecutive
    % Niveles: \blControlling o \blInfluencing.

    \BRitem[admin]{Clase}{\bcIntegridad}%

    \BRitem[admin]{Tipo}{\btTimer}%

    \BRitem[admin]{Nivel}{\blControlling}

    \BRitem{Descripción}{%
        Cuando un \refElem{xp-user} es creado este debe de empezar a ganar puntos
        de experiencia a partir del \refElem{xp-user.level} uno, tenido cero puntos
        de experiencia en la \refElem{xp-user.levelxp} y \refElem{xp-user.xp}. Ningún
        usuario puede comenzar con valores distintos a los indicados anteriormente.
    }

%   \BRitem{Sentencia}{%
%       Si $fecha$ 
%   }%

    \BRitem{Ejemplo positivo}{\hfill\par%
        \begin{itemize}
        \item ...
        \end{itemize}
    }

    \BRitem{Ejemplo negativo}{\hfill\par%
        \begin{itemize}
        \item ...
        \end{itemize}
    }

 \end{BusinessRule}
 % Valores iniciales de comperiencia
    %\begin{BusinessRule}[%
Autor/Daniel Isai Ortega Zúñiga,%
Version/0.1,%
Estado/edicion]%
%
{BR-E08}{Valores iniciales de experiencia de un curso}

     \BRitem[control]{Revisada por}{Pendiente.}

 \BRsection[control]{Atributos}
    % Clases: \bcCondition, \bcIntegridad, \bcAutorization o \bcDerivation
    % Tipos: \btEnabler, \btTimer o \btExecutive
    % Niveles: \blControlling o \blInfluencing.

    \BRitem[admin]{Clase}{\bcIntegridad}%

    \BRitem[admin]{Tipo}{\btTimer}%

    \BRitem[admin]{Nivel}{\blControlling}

    \BRitem{Descripción}{%
        Cuando un \refElem{xp-course} es creado la \refElem[experiencia total del curso]%
        {xp-scheme-settings.courseXP} de ser dividida uniformemente entre las
        \refElem[secciones del curso gamificado]{xp-course-section}. Si la división del
        total de experiencia entre el número de secciones genera un residuo entonces este
        se deberá agregan a la última sección del curso.
    }

%   \BRitem{Sentencia}{%
%       Si $fecha$
%   }%

    \BRitem{Ejemplo positivo}{\hfill\par%
        \begin{itemize}
        \item ...
        \end{itemize}
    }

    \BRitem{Ejemplo negativo}{\hfill\par%
        \begin{itemize}
        \item ...
        \end{itemize}
    }

 \end{BusinessRule}
 % Valores iniciales de experiencia del curso

    % INPUT: Cursos Igualitarios.
    % INPUT: Otorgar experiencia
    % INPUT: Administración de experiencia en el curso

\clearpage
\subsubsection{Casos de uso} % ============================================================

 En este apartado se especifican todos los casos de usos contemplados para el módulo de
competencia, para cada caso de uso se especifica su tabla de atributos la cual indica que casos
 de prueba deberán ejecutarse correctamente para corroborar la completitud del caso de uso.

\subsubsection*{Diagrama de casos de uso}

 En la figura \ref{comp:usecases} se detalla el diagrama de casos de uso correspondiente al módulo
 de competencia. Los casos de uso de moodle (en color blanco) son modelados como casos de uso
 abstractos, mientras que los casos de uso del módulo de competencia son diferenciados por el
 color azul, en total el desarrollo de este módulo consiste en 17 casos de uso principales.

    \addfigure{0.6}{modulos/comp/diagrams/UseCases}{comp:usecases}{%
        Diagrama de casos de uso del módulo de competencia}

 \noindent
 Debido a que los plugins a desarrollar son elementos opcionales para Moodle, solo se puede
 acceder a los casos de uso del módulo de competencia a través de puntos de extensión de los
 casos de uso de moodle. Por otra parte los casos de uso que serán documentados en esta sección
 serán los del módulo de competencia debido a que Moodle proporciona en su página oficial, guías
 e instructivos como documentación de las funcionalidades que brinda.

    % MODULO DE EXPERIENCIA

\input{modulos/comp/CU/CU-C01}   % Instalar plugin del esquema de comperiencia

% \ucstEnEdicion     Al terminar una revisión/aprobación con observaciones
%                    y al inicio del CU.
%
% \ucstEnRevision    Al terminar la edición del CU (version += 0.1).
% \ucstEnAprobacion  Al pasar la revision sin observaciones.
% \ucstAprobado      Al ser aprobado por el usuario (version += 1.0)

\begin{UseCase}[%
Autor/Ricardo Naranjo,%
Version/0.1,%
Estado/\ucstEnRevision]%
%
{CU-C02}{Actualizar instancia (Competencia uno contra uno)}{%
%
 Permite al \refElem{aProfesor} y al \refElem{aAdministrador} actualizar una instancia de la actividad competencia uno contra uno en su curso.
 Este caso de uso es una extensión del caso de uso {\it Ver curso} que es propio de moodle.}

	\UCitem[control]{Revisor}{ Sin asignar }
	\UCitem[control]{Último cambio}{ 13/NOV/19 }

 \UCsection{Atributos}

    \UCitem{Actor(es)}{%
        \refElem{aProfesor},
        \refElem{aAdministrador}
    }

	\UCitems{Propósito}{%
        \Titem Permitir al \refElem{aProfesor} y al \refElem{aAdministrador} actualizar una instancia de la actividad de competencia uno contra uno.

        \Titem Permitir al \refElem{aEstudiante}, \refElem{aProfesor} y \refElem{aAdministrador} con acceso al curso utilizar la instancia actualizada de la actividad de competencia uno contra uno creada por el \refElem{aProfesor} o \refElem{aAdministrador}.
	}

	\UCitem{Entradas}{\imprimeUC{entrada}}

	\UCitems{Origen}{%
        \Titem Mouse
        \Titem Teclado
	}

	\UCitem{Salidas}{\imprimeUC{salida}}

	\UCitem{Destino}{%
		\refElem{IU-M07}
	}

	\UCitems{Precondiciones}{%
        \Titem El plugin de competencia uno contra uno debe estar instalado en moodle.
        \Titem La instancia de la actividad de competencia uno contra uno debe estar creada.
        % Realizar el caso de uso "listar actividades disponibles?"
        % \Titem Si se trata de una actualización de un plugin la versión de este debe
               % cumplir con la regla \refElem{BR-M02}.
	}

	\UCitems{Postcondiciones}{%
        \Titem La instancia actualizada de la actividad debe mostrarse en la pantalla \refElem{IU-M07}.%

	}

	\UCitem{Reglas de negocio}{\imprimeUC{regla}}

	\UCitems{Errores}{%
        \Titem \UCerr{Err1}{%
            No se ingresó un campo requerido en el formulario de creación de la actividad,}{% CAUSA
            no se puede actualizar la instancia de la actividad}% EFECTO
	}

	% \UCitem{Viene de}{% Indicar si el Caso de uso es primario o se extiende de otro. La mayoría se
					  % extienden de Login.
		% EJEMPLO: \refIdElem{PY-CU1} o Caso de uso primario.
	% 	\TODO Especificar.
	% }

 \UCsection[design]{Datos de Diseño}

	\UCitems[design]{Casos de Prueba}{%
        \Titem \refElem{CPC-C01}
	}

 \UCsection[admin]{Datos de Administración de Requerimiento}

	\UCitem[admin]{Observaciones}{}

\end{UseCase}

\subsubsection{Trayectorias del caso de uso}

\begin{UCtrayectoria}%
%

    \Actor Activa la edición del curso en la pantalla \refElem{IU-M07}.
    \Sistema Redirige a la pantalla de edición del curso \refElem{IU-M07aa}.
    \Actor Presiona el botón {\bf Editar} de la instancia que desea actualizar.
    \Sistema Despliega el menú \refElem{IU-M07b}.
    \Actor Presiona el botón {\bf Editar ajustes} del menú desplegable \refElem{IU-M07b}.

    \Sistema Redirige a la pantalla \refElem{IU-C06} y carga los valores de la instancia \refElem{comp-1v1-gmcompvs} (
      \salida{comp-1v1-gmcompvs.name},
      \salida{comp-1v1-gmcompvs.mdl-question-categories-id},
      \salida{comp-1v1-gmcompvs.apuestas-activas},
      \salida{comp-1v1-gmcompvs.completionnumwon}).

    \label{CU-C02-muestra-pantalla}

    \Actor Actualiza los datos correspondientes en el formulario.

    \Actor Presiona el botón {\bf Guardar cambios y regresar al curso}.\refTray{A} \refTray{B}

    \Sistema Valida que los campos ingresados sean válidos. \refTray{C} \refErr{Err1}

    \Sistema Actualiza los valores ingresados para la instancia \refElem{comp-1v1-gmcompvs} (
      \entrada{comp-1v1-gmcompvs.name},
      \entrada{comp-1v1-gmcompvs.mdl-question-categories-id},
      \entrada{comp-1v1-gmcompvs.apuestas-activas},
      \entrada{comp-1v1-gmcompvs.completionnumwon}), especificadas en el modelo de información.

    \Sistema Redirige a la pantalla \refElem{IU-M07} y muestra la instancia actualizada en el curso.

\end{UCtrayectoria}

\begin{UCtrayectoriaA}[Fin del caso de uso]{A}{El \refElem{aProfesor} o \refElem{aAdministrador} desea ver la instancia actualizada de la actividad}

    \Actor Presiona el botón {\bf Guardar cambios y mostrar} de la pantalla \refElem{IU-C06}.

    \Sistema Valida que los campos ingresados sean válidos. \refTray{C} \refErr{Err1}

    \Sistema Actualiza los valores ingresados para la instancia \refElem{comp-1v1-gmcompvs} (
      \refElem{comp-1v1-gmcompvs.name},
      \refElem{comp-1v1-gmcompvs.mdl-question-categories-id},
      \refElem{comp-1v1-gmcompvs.apuestas-activas},
      \refElem{comp-1v1-gmcompvs.completionnumwon}), especificadas en el modelo de información.

    \Sistema Redirige a la pantalla \refElem{IU-C02}.

\end{UCtrayectoriaA}

\begin{UCtrayectoriaA}[Fin del caso de uso]%
  {B}{El \refElem{aProfesor} o \refElem{aAdministrador} desea cancelar la actualización de la instancia después de mostrar el formulario de actualización}

  \Actor Presiona el botón {\bf cancelar} en la pantalla \refElem{IU-C06}.
  \Sistema Redirige a la pantalla \refElem{IU-C01}.

\end{UCtrayectoriaA}

\begin{UCtrayectoriaA}{C}{Algún dato ingresado por el \refElem{aProfesor} o \refElem{aAdministrador} es inválido}

  \Sistema Muestra un mensaje de error "-Usted debe poner un valor aquí", en los campos de la pantalla \refElem{IU-C06} que sean requeridos.
  \Sistema Regresa al paso \ref{CU-C02-muestra-pantalla}

\end{UCtrayectoriaA}


% \ucstEnEdicion     Al terminar una revisión/aprobación con observaciones
%                    y al inicio del CU.
%
% \ucstEnRevision    Al terminar la edición del CU (version += 0.1).
% \ucstEnAprobacion  Al pasar la revision sin observaciones.
% \ucstAprobado      Al ser aprobado por el usuario (version += 1.0)

\begin{UseCase}[%
Autor/Ricardo Naranjo,%
Version/0.1,%
Estado/\ucstEnRevision]%
%
{CU-C03}{Eliminar instancia (Competencia uno contra uno)}{%
%
 Permite al \refElem{aProfesor} y al \refElem{aAdministrador} eliminar una instancia de la actividad competencia uno contra uno en su curso.
 Este caso de uso es una extensión del caso de uso {\it Ver curso} que es propio de moodle.}

	\UCitem[control]{Revisor}{ Sin asignar }
	\UCitem[control]{Último cambio}{ 13/NOV/19 }

 \UCsection{Atributos}

    \UCitem{Actor(es)}{%
        \refElem{aProfesor},
        \refElem{aAdministrador}
    }

	\UCitems{Propósito}{%
        \Titem Permitir al \refElem{aProfesor} y al \refElem{aAdministrador} eliminar una instancia de la actividad de competencia uno contra uno.
	}

	\UCitem{Entradas}{\imprimeUC{entrada}}

	\UCitems{Origen}{%
        \Titem Mouse
	}

	\UCitem{Salidas}{\imprimeUC{salida}}

	\UCitem{Destino}{%
		\refElem{IU-M08}
	}

	\UCitems{Precondiciones}{%
        \Titem El plugin de competencia uno contra uno debe estar instalado en moodle.
        \Titem La instancia de la actividad de competencia uno contra uno debe estar creada.
        % Realizar el caso de uso "listar actividades disponibles?"
        % \Titem Si se trata de una actualización de un plugin la versión de este debe
               % cumplir con la regla \refElem{BR-M02}.
	}

	\UCitems{Postcondiciones}{%
        \Titem La instancia de la actividad eliminada no debe mostrarse en la pantalla \refElem{IU-M08}.%

	}

	\UCitem{Reglas de negocio}{\imprimeUC{regla}}

	\UCitems{Errores}{%
	}

	% \UCitem{Viene de}{% Indicar si el Caso de uso es primario o se extiende de otro. La mayoría se
					  % extienden de Login.
		% EJEMPLO: \refIdElem{PY-CU1} o Caso de uso primario.
	% 	\TODO Especificar.
	% }

 \UCsection[design]{Datos de Diseño}

	\UCitems[design]{Casos de Prueba}{%
        \Titem \refElem{CPC-C01}
	}

 \UCsection[admin]{Datos de Administración de Requerimiento}

	\UCitem[admin]{Observaciones}{}

\end{UseCase}

\subsubsection{Trayectorias del caso de uso}

\begin{UCtrayectoria}%
%

    \Actor Activa la edición del curso en la pantalla \refElem{IU-M08}.

    \Sistema Redirige a la pantalla de edición del curso \refElem{IU-M08aa}.

    \Actor Presiona el botón {\bf Editar} de la instancia que desea eliminar.

    \Sistema Despliega el menú \refElem{IU-M08b}.

    \Actor Presiona el botón {\bf Eliminar} del menú desplegable \refElem{IU-M08b}.

    \Sistema Despliega mensaje de confirmación de eliminación. \refElem{IU-M08c}

    \Actor Presiona el botón {\bf Si}. \refTray{A}

    \Sistema Redirige a la pantalla \refElem{IU-M08} y elimina la instancia y los valores de la instancia \refElem{comp-1v1-gmcompvs}, así como los datos que dependen de la instancia en las siguientes entidades: \refElem{comp-1v1-gmdl-partida}, \refElem{comp-1v1-gmdl-participacion} y \refElem{comp-1v1-gmdl-partida}.

\end{UCtrayectoria}

\begin{UCtrayectoriaA}[Fin del caso de uso]{A}{El \refElem{aProfesor} o \refElem{aAdministrador} desea cancelar la eliminación después de mostrar el mensaje de confirmación}

  \Actor Presiona el botón {\bf No} en la mensaje de confirmación \refElem{IU-M08C}.
  \Sistema Redirige a la pantalla \refElem{IU-M08}.

\end{UCtrayectoriaA}


% \ucstEnEdicion     Al terminar una revisión/aprobación con observaciones
%                    y al inicio del CU.
%
% \ucstEnRevision    Al terminar la edición del CU (version += 0.1).
% \ucstEnAprobacion  Al pasar la revision sin observaciones.
% \ucstAprobado      Al ser aprobado por el usuario (version += 1.0)

\begin{UseCase}[%
Autor/Ricardo Naranjo,%
Version/0.1,%
Estado/\ucstEnRevision]%
%
{CU-C04}{Crear instancia (Competencia uno contra sistema)}{%
%
 Permite al \refElem{aProfesor} y al \refElem{aAdministrador} crear una nueva instancia de la actividad competencia uno contra sistema en su curso.
 La conclusión de la trayectoria principal de esta caso de uso es una precondición para que
 algunos casos de uso del módulo de competencia puedan ejecutarse.\\%
 Este caso de uso es una extensión del caso de uso {\it Listar actividades disponibles} que es propio de moodle.}

	\UCitem[control]{Revisor}{ Sin asignar }
	\UCitem[control]{Último cambio}{ 13/NOV/19 }

 \UCsection{Atributos}

    \UCitem{Actor(es)}{%
        \refElem{aProfesor},
        \refElem{aAdministrador}
    }

	\UCitems{Propósito}{%
        \Titem Permitir al \refElem{aProfesor} y al \refElem{aAdministrador} incluir en su curso una nueva instancia de la actividad de competencia uno contra sistema.

        \Titem Permitir al \refElem{aEstudiante}, \refElem{aProfesor} y \refElem{aAdministrador} con acceso al curso utilizar la instancia de la actividad de competencia uno contra sistema creada por el \refElem{aProfesor} o \refElem{aAdministrador}.
	}

	\UCitem{Entradas}{\imprimeUC{entrada}}

	\UCitems{Origen}{%
        \Titem Mouse
        \Titem Teclado
	}

	\UCitem{Salidas}{\imprimeUC{salida}}

	\UCitem{Destino}{%
		\refElem{IU-M07}
	}

	\UCitems{Precondiciones}{%
        \Titem El plugin de competencia 1 contra sistema debe estar instalado en moodle.
        % Realizar el caso de uso "listar actividades disponibles?"
        % \Titem Si se trata de una actualización de un plugin la versión de este debe
               % cumplir con la regla \refElem{BR-M02}.
	}

	\UCitems{Postcondiciones}{%
        \Titem La nueva instancia de la actividad debe mostrarse en la pantalla \refElem{IU-M07}.%

	}

	\UCitem{Reglas de negocio}{\imprimeUC{regla}}

	\UCitems{Errores}{%
        \Titem \UCerr{Err1}{%
            No se ingresó un campo requerido en el formulario de creación de la actividad,}{% CAUSA
            no se puede crear la nueva instancia de la actividad}% EFECTO
	}

	% \UCitem{Viene de}{% Indicar si el Caso de uso es primario o se extiende de otro. La mayoría se
					  % extienden de Login.
		% EJEMPLO: \refIdElem{PY-CU1} o Caso de uso primario.
	% 	\TODO Especificar.
	% }

 \UCsection[design]{Datos de Diseño}

	\UCitems[design]{Casos de Prueba}{%
        \Titem \refElem{CPC-C01}
	}

 \UCsection[admin]{Datos de Administración de Requerimiento}

	\UCitem[admin]{Observaciones}{}

\end{UseCase}

\subsubsection{Trayectorias del caso de uso}

\begin{UCtrayectoria}%
%
    \Actor Selecciona la actividad Gamedle - Competencia 1 contra sistema en la pantalla \refElem{IU-M07a}.
    \Sistema Muestra la descripción de la actividad Gamedle - Competencia 1 contra sistema en la pantalla.

    \Actor Presiona el botón {\bf Agregar} en la pantalla. \refTray{A}
    \Sistema Redirige a la pantalla \refElem{IU-C03}.
    \label{CU-C04-muestra-pantalla}

    \Actor Ingresa los datos correspondientes en el formulario.

    \Actor Presiona el botón {\bf Guardar cambios y regresar al curso}.\refTray{B} \refTray{C}

    \Sistema Valida que los campos ingresados sean válidos. \refTray{D} \refErr{Err1}

    \Sistema Establece los valores ingresados para la nueva instancia \refElem{comp-cpu-gmcompcpu} (
      \entrada{comp-cpu-gmcompcpu.name},
      \entrada{comp-cpu-gmcompcpu.mdl-question-categories-id},
      \entrada{comp-cpu-gmcompcpu.completioncpudiff}), especificadas en el modelo de información.

    \Sistema Redirige a la pantalla \refElem{IU-M07} y muestra la nueva instancia creada en el curso.

\end{UCtrayectoria}

\begin{UCtrayectoriaA}[Fin del caso de uso]%
  {A}{El \refElem{aProfesor} o \refElem{aAdministrador} desea cancelar la creación de la nueva instancia después que se le muestra la descripción de la actividad}

  \Actor Presiona el botón {\bf cancelar} en la pantalla \refElem{IU-M07a}.
  \Sistema Cierra la pantalla \refElem{IU-M07a} y redirige a la pantalla \refElem{IU-M07}.

\end{UCtrayectoriaA}

\begin{UCtrayectoriaA}[Fin del caso de uso]{B}{El \refElem{aProfesor} o \refElem{aAdministrador} desea ver la nueva instancia de la actividad}

    \Actor Presiona el botón {\bf Guardar cambios y mostrar} de la pantalla \refElem{IU-C03}.

    \Sistema Valida que los campos ingresados sean válidos. \refTray{D} \refErr{Err1}

    \Sistema Establece los valores ingresados para la nueva instancia \refElem{comp-cpu-gmcompcpu} (
      \refElem{comp-cpu-gmcompcpu.name},
      \refElem{comp-cpu-gmcompcpu.mdl-question-categories-id},
      \refElem{comp-cpu-gmcompcpu.completioncpudiff}), especificadas en el modelo de información.

    \Sistema Redirige a la pantalla \refElem{IU-C04}.

\end{UCtrayectoriaA}

\begin{UCtrayectoriaA}[Fin del caso de uso]%
  {C}{El \refElem{aProfesor} desea cancelar la creación de la nueva instancia después de mostrar el formulario de creación}

  \Actor Presiona el botón {\bf cancelar} en la pantalla \refElem{IU-C03}.
  \Sistema Redirige a la pantalla \refElem{IU-M07}.

\end{UCtrayectoriaA}

\begin{UCtrayectoriaA}{D}{Algún dato ingresado por el \refElem{aProfesor} o \refElem{aAdministrador} es inválido}

  \Sistema Muestra un mensaje de error "-Usted debe poner un valor aquí", en los campos de la pantalla \refElem{IU-C03} que sean requeridos.
  \Sistema Regresa al paso \ref{CU-C04-muestra-pantalla}

\end{UCtrayectoriaA}
   % Instalar plugin del esquema de comperiencia

% \ucstEnEdicion     Al terminar una revisión/aprobación con observaciones
%                    y al inicio del CU.
%
% \ucstEnRevision    Al terminar la edición del CU (version += 0.1).
% \ucstEnAprobacion  Al pasar la revision sin observaciones.
% \ucstAprobado      Al ser aprobado por el usuario (version += 1.0)

\begin{UseCase}[%
Autor/Ricardo Naranjo,%
Version/0.1,%
Estado/\ucstEnRevision]%
%
{CU-C05}{Actualizar instancia (Competencia uno contra sistema)}{%
%
 Permite al \refElem{aProfesor} y al \refElem{aAdministrador} actualizar una instancia de la actividad competencia uno contra sistema en su curso.
 Este caso de uso es una extensión del caso de uso {\it Ver curso} que es propio de moodle.}

	\UCitem[control]{Revisor}{ Sin asignar }
	\UCitem[control]{Último cambio}{ 13/NOV/19 }

 \UCsection{Atributos}

    \UCitem{Actor(es)}{%
        \refElem{aProfesor},
        \refElem{aAdministrador}
    }

	\UCitems{Propósito}{%
        \Titem Permitir al \refElem{aProfesor} y al \refElem{aAdministrador} actualizar una instancia de la actividad de competencia uno contra sistema.

        \Titem Permitir al \refElem{aEstudiante}, \refElem{aProfesor} y \refElem{aAdministrador} con acceso al curso utilizar la instancia actualizada de la actividad de competencia uno contra sistema creada por el \refElem{aProfesor} o \refElem{aAdministrador}.
	}

	\UCitem{Entradas}{\imprimeUC{entrada}}

	\UCitems{Origen}{%
        \Titem Mouse
        \Titem Teclado
	}

	\UCitem{Salidas}{\imprimeUC{salida}}

	\UCitem{Destino}{%
		\refElem{IU-M08}
	}

	\UCitems{Precondiciones}{%
        \Titem El plugin de competencia uno contra sistema debe estar instalado en moodle.
        \Titem La instancia de la actividad de competencia uno contra sistema debe estar creada.
        % Realizar el caso de uso "listar actividades disponibles?"
        % \Titem Si se trata de una actualización de un plugin la versión de este debe
               % cumplir con la regla \refElem{BR-M02}.
	}

	\UCitems{Postcondiciones}{%
        \Titem La instancia actualizada de la actividad debe mostrarse en la pantalla \refElem{IU-M08}.%

	}

	\UCitem{Reglas de negocio}{\imprimeUC{regla}}

	\UCitems{Errores}{%
        \Titem \UCerr{Err1}{%
            No se ingresó un campo requerido en el formulario de creación de la actividad,}{% CAUSA
            no se puede actualizar la instancia de la actividad}% EFECTO
	}

	% \UCitem{Viene de}{% Indicar si el Caso de uso es primario o se extiende de otro. La mayoría se
					  % extienden de Login.
		% EJEMPLO: \refIdElem{PY-CU1} o Caso de uso primario.
	% 	\TODO Especificar.
	% }

 \UCsection[design]{Datos de Diseño}

	\UCitems[design]{Casos de Prueba}{%
        \Titem \refElem{CPC-C01}
	}

 \UCsection[admin]{Datos de Administración de Requerimiento}

	\UCitem[admin]{Observaciones}{}

\end{UseCase}

\subsubsection{Trayectorias del caso de uso}

\begin{UCtrayectoria}%
%

    \Actor Activa la edición del curso en la pantalla \refElem{IU-M08}.
    \Sistema Redirige a la pantalla de edición del curso \refElem{IU-M08aa}.
    \Actor Presiona el botón {\bf Editar} de la instancia que desea actualizar.
    \Sistema Despliega el menú \refElem{IU-M08b}.
    \Actor Presiona el botón {\bf Editar ajustes} del menú desplegable \refElem{IU-M08b}.

    \Sistema Redirige a la pantalla \refElem{IU-C05} y carga los valores de la instancia \refElem{comp-cpu-gmcompcpu} (
      \salida{comp-cpu-gmcompcpu.name},
      \salida{comp-cpu-gmcompcpu.mdl-question-categories-id},
      \salida{comp-cpu-gmcompcpu.completioncpudiff}), especificadas en el modelo de información.

    \label{CU-C05-muestra-pantalla}

    \Actor Actualiza los datos correspondientes en el formulario.

    \Actor Presiona el botón {\bf Guardar cambios y regresar al curso}.\refTray{A} \refTray{B}

    \Sistema Valida que los campos ingresados sean válidos. \refTray{C} \refErr{Err1}

    \Sistema Actualiza los valores ingresados para la instancia \refElem{comp-cpu-gmcompcpu} (
      \entrada{comp-cpu-gmcompcpu.name},
      \entrada{comp-cpu-gmcompcpu.mdl-question-categories-id},
      \entrada{comp-cpu-gmcompcpu.completioncpudiff}), especificadas en el modelo de información.

    \Sistema Redirige a la pantalla \refElem{IU-M08} y muestra la instancia actualizada en el curso.

\end{UCtrayectoria}

\begin{UCtrayectoriaA}[Fin del caso de uso]{A}{El \refElem{aProfesor} o \refElem{aAdministrador} desea ver la instancia actualizada de la actividad}

    \Actor Presiona el botón {\bf Guardar cambios y mostrar} de la pantalla \refElem{IU-C05}.

    \Sistema Valida que los campos ingresados sean válidos. \refTray{C} \refErr{Err1}

    \Sistema Actualiza los valores ingresados para la instancia \refElem{comp-cpu-gmcompcpu} (
      \refElem{comp-cpu-gmcompcpu.name},
      \refElem{comp-cpu-gmcompcpu.mdl-question-categories-id},
      \refElem{comp-cpu-gmcompcpu.completioncpudiff}), especificadas en el modelo de información.

    \Sistema Redirige a la pantalla \refElem{IU-C02}.

\end{UCtrayectoriaA}

\begin{UCtrayectoriaA}[Fin del caso de uso]%
  {B}{El \refElem{aProfesor} o \refElem{aAdministrador} desea cancelar la actualización de la instancia después de mostrar el formulario de actualización}

  \Actor Presiona el botón {\bf cancelar} en la pantalla \refElem{IU-C05}.
  \Sistema Redirige a la pantalla \refElem{IU-C01}.

\end{UCtrayectoriaA}

\begin{UCtrayectoriaA}{C}{Algún dato ingresado por el \refElem{aProfesor} o \refElem{aAdministrador} es inválido}

  \Sistema Muestra un mensaje de error "-Usted debe poner un valor aquí", en los campos de la pantalla \refElem{IU-C05} que sean requeridos.
  \Sistema Regresa al paso \ref{CU-C05-muestra-pantalla}

\end{UCtrayectoriaA}

\input{modulos/comp/CU/CU-C06}
\input{modulos/comp/CU/CU-C07}

% \ucstEnEdicion     Al terminar una revisión/aprobación con observaciones
%                    y al inicio del CU.
%
% \ucstEnRevision    Al terminar la edición del CU (version += 0.1).
% \ucstEnAprobacion  Al pasar la revision sin observaciones.
% \ucstAprobado      Al ser aprobado por el usuario (version += 1.0)

%\addfigure[(adaptado de {\it For The Win} \cite{ForTheWin})]%
%    {.4}{investigacion/images/forthewin}{fig:ForTheWin}%
%    {Jerarquía de elementos de juego segun For The Win}

\begin{UseCase}[%
Autor/Ricardo Naranjo,%
Version/0.1,%
Estado/\ucstEnRevision]%
%
{CU-C08}{Ver historial de sus partidas (Competencia uno contra uno)}{%
%
 Permite al \refElem{aEstudiante}, \refElem{aProfesor} y al \refElem{aAdministrador} ver su historial de una instancia de la actividad competencia uno contra uno en el curso.
 Este caso de uso es una extensión del caso de uso \refElem{CU-C07}.}

	\UCitem[control]{Revisor}{ Sin asignar }
	\UCitem[control]{Último cambio}{ 13/NOV/19 }

 \UCsection{Atributos}

    \UCitem{Actor(es)}{%
        \refElem{aEstudiante},
        \refElem{aProfesor},
        \refElem{aAdministrador}
    }

	\UCitems{Propósito}{%
        \Titem Permitir al \refElem{aEstudiante}, \refElem{aProfesor} y al \refElem{aAdministrador} ver su historial de una instancia de la actividad de competencia uno contra uno.
	}

	\UCitem{Entradas}{\imprimeUC{entrada}}

	\UCitems{Origen}{%
        \Titem Mouse
	}

	\UCitem{Salidas}{\imprimeUC{salida}}

	\UCitem{Destino}{%
		\refElem{IU-C07}
	}

	\UCitems{Precondiciones}{%
        \Titem El plugin de competencia uno contra uno debe estar instalado en moodle.
        \Titem La instancia de la actividad de competencia uno contra uno debe estar creada.
        % Realizar el caso de uso "listar actividades disponibles?"
        % \Titem Si se trata de una actualización de un plugin la versión de este debe
               % cumplir con la regla \refElem{BR-M02}.
	}

	\UCitems{Postcondiciones}{%
        \Titem La pantalla de historial de la instancia de la actividad de competencia uno contra uno \refElem{IU-C07} debe mostrar los datos pertinentes al usuario que realizó el caso de uso.%

	}

	\UCitem{Reglas de negocio}{\imprimeUC{regla}}

	\UCitems{Errores}{%
	}

	% \UCitem{Viene de}{% Indicar si el Caso de uso es primario o se extiende de otro. La mayoría se
					  % extienden de Login.
		% EJEMPLO: \refIdElem{PY-CU1} o Caso de uso primario.
	% 	\TODO Especificar.
	% }

 \UCsection[design]{Datos de Diseño}

	\UCitems[design]{Casos de Prueba}{%
        \Titem \refElem{CPC-C01}
	}

 \UCsection[admin]{Datos de Administración de Requerimiento}

	\UCitem[admin]{Observaciones}{}

\end{UseCase}

\subsubsection{Trayectorias del caso de uso}

\begin{UCtrayectoria}%
%

    \Actor Presiona el botón {\bf Historial} de la pantalla \refElem{IU-C02}.

    \Sistema Redirige a la pantalla historial de la instancia \refElem{IU-C07}.

    \Sistema Muestra en la sección {\bf Resumen de los estados de los desafíos} los siguientes números: \salida{Victorias}, \salida{Empates}, \salida{Derrotas}, \salida{En curso}, \salida{Retirada}.

    \Sistema Muestra en la sección inferior un bloque por cada partida iniciada, en el bloque se muestra: \salida{Imagen de perfil de contrincante}, \salida{Nombre de contrincante}, \salida{Los puntos del actor}, \salida{Monedas apostadas}, \salida{Puntos contrincante}, \salida{Monedas apostadas contrincante}, \salida{Estado de desafío}.

\end{UCtrayectoria}


% \ucstEnEdicion     Al terminar una revisión/aprobación con observaciones
%                    y al inicio del CU.
%
% \ucstEnRevision    Al terminar la edición del CU (version += 0.1).
% \ucstEnAprobacion  Al pasar la revision sin observaciones.
% \ucstAprobado      Al ser aprobado por el usuario (version += 1.0)

%\addfigure[(adaptado de {\it For The Win} \cite{ForTheWin})]%
%    {.4}{investigacion/images/forthewin}{fig:ForTheWin}%
%    {Jerarquía de elementos de juego segun For The Win}

\begin{UseCase}[%
Autor/Ricardo Naranjo,%
Version/0.1,%
Estado/\ucstEnRevision]%
%
{CU-C09}{Ver tabla de posiciones (Competencia uno contra uno)}{%
%
 Permite al \refElem{aEstudiante}, \refElem{aProfesor} y al \refElem{aAdministrador} ver la tabla de posiciones de una instancia de la actividad competencia uno contra uno en el curso.
 Este caso de uso es una extensión del caso de uso \refElem{CU-C07}.}

	\UCitem[control]{Revisor}{ Sin asignar }
	\UCitem[control]{Último cambio}{ 13/NOV/19 }

 \UCsection{Atributos}

    \UCitem{Actor(es)}{%
        \refElem{aEstudiante},
        \refElem{aProfesor},
        \refElem{aAdministrador}
    }

	\UCitems{Propósito}{%
        \Titem Permitir al \refElem{aEstudiante}, \refElem{aProfesor} y al \refElem{aAdministrador} ver la tabla de posiciones de una instancia de la actividad de competencia uno contra uno.
	}

	\UCitem{Entradas}{\imprimeUC{entrada}}

	\UCitems{Origen}{%
        \Titem Mouse
	}

	\UCitem{Salidas}{\imprimeUC{salida}}

	\UCitem{Destino}{%
		\refElem{IU-C08}
	}

	\UCitems{Precondiciones}{%
        \Titem El plugin de competencia uno contra uno debe estar instalado en moodle.
        \Titem La instancia de la actividad de competencia uno contra uno debe estar creada.
        % Realizar el caso de uso "listar actividades disponibles?"
        % \Titem Si se trata de una actualización de un plugin la versión de este debe
               % cumplir con la regla \refElem{BR-M02}.
	}

	\UCitems{Postcondiciones}{%
        \Titem Se muestra la pantalla de tabla de posiciones de la instancia de la actividad de competencia uno contra uno \refElem{IU-C08} de acuerdo a su número de victorias.%

	}

	\UCitem{Reglas de negocio}{\imprimeUC{regla}}

	\UCitems{Errores}{%
	}

	% \UCitem{Viene de}{% Indicar si el Caso de uso es primario o se extiende de otro. La mayoría se
					  % extienden de Login.
		% EJEMPLO: \refIdElem{PY-CU1} o Caso de uso primario.
	% 	\TODO Especificar.
	% }

 \UCsection[design]{Datos de Diseño}

	\UCitems[design]{Casos de Prueba}{%
        \Titem \refElem{CPC-C01}
	}

 \UCsection[admin]{Datos de Administración de Requerimiento}

	\UCitem[admin]{Observaciones}{}

\end{UseCase}

\subsubsection{Trayectorias del caso de uso}

\begin{UCtrayectoria}%
%

    \Actor Presiona el botón {\bf Tabla posiciones} de la pantalla \refElem{IU-C08}.

    \Sistema Redirige a la pantalla historial de la instancia \refElem{IU-C08}.

    \Sistema Muestra un bloque por cada usuario con al menos una partida terminada, en el bloque se muestra: \salida{posición del usuario} \salida{Imagen de perfil de usuario}, \salida{Nombre de usuario}, \salida{Número de victorias del usuario}.

\end{UCtrayectoria}



% =========================================================
\clearpage
\subsection{Diseño}

\subsubsection{Interfaces del módulo de competencia}

    \subsubsection{IU-M08 Pantalla principal del curso}

 En esta pantalla se muestran las instancias de las actividades
 que han sido creadas para el curso.

    \IUfig{1}{modulos/moodle/IU/p_principal_curso}{IU-M08}{Pantalla principal del curso}

\subsubsection{Elementos Relevantes}

    \begin{itemize}
    \item {\bf Instancias de las actividades}
    \item {\bf Menú de acciones}
    \item {\bf Activar edición}
    \end{itemize}

\subsubsection{Acciones relevantes}

    \begin{itemize}
    \item {\bf Activar la edición del curso para poder agregar,actualizar o eliminar instancias}
    \item {\bf Tener acceso a las instancias de las actividades}
    \end{itemize}

\clearpage

    
\subsubsection{IU-M08aa Pantalla principal del curso con la edición activada}

 Descripción ...

    \IUfig{1}{modulos/moodle/IU/p_principal_curso_editar}{IU-M08aa}{ Pantalla principal del curso con la edición activada}

\subsubsection{Elementos Relevantes}

    \begin{itemize}
    \item {\bf Lorem ipsum}
        ...
    \end{itemize}

\subsubsection{Acciones relevantes}

    \begin{itemize}
    \item {\bf Lorem ipsum}
        ...
    \end{itemize}

\clearpage
  % Configuraciones
    
\subsubsection{IU-M08a Pantalla de listado de actividades disponibles}

 Esta pantalla se activa cuando se presiona el botón de {\bf Añadir una actividad o recurso} en la pantalla \refElem{IU-M08aa}. La pantalla muestra las actividades disponibles que se pueden agregar al curso.

    \IUfig{.5}{modulos/moodle/IU/p_listado_actividades}{IU-M08a}{Pantalla de listado de actividades disponibles}

\subsubsection{Elementos Relevantes}

    \begin{itemize}
    \item {\bf Actividades}
    \item {\bf Agregar}
    \item {\bf Cancelar}
    \end{itemize}

\subsubsection{Acciones relevantes}

    \begin{itemize}
    \item {\bf Seleccionar la actividad que se desea agregar en el curso}
    \item {\bf Iniciar el proceso para crear una instancia de la actividad seleccionada}
    \end{itemize}

\clearpage

    
\subsubsection{IU-M08b Menú desplegable al presionar el botón Editar}

 Este menú desplegable muestra una variedad de opciones que se pueden usar para editar la instancia de la actividad elegida.

    \IUfig{.3}{modulos/moodle/IU/md_boton_editar_instancia}{IU-M08b}{Menú desplegable al presionar el botón Editar}

\subsubsection{Elementos Relevantes}

    \begin{itemize}
    \item {\bf Editar ajustes}
    \item {\bf Eliminar}

    \end{itemize}

\subsubsection{Acciones relevantes}

    \begin{itemize}
    \item {\bf Presionar el botón Editar ajustes}, nos permite actualizar la instancia de la actividad.
    \item {\bf Presionar el botón Eliminar}, nos permite eliminar la instancia de la actividad.
    \end{itemize}

\clearpage

    
\subsubsection{IU-M08c Mensaje de confirmación al presionar el botón Eliminar}

 Este mensaje de confirmación se muestra para reducir la cantidad de errores que ocurren cuando se presiona el botón equivocado en el menú \refElem{IU-M08b}.

    \IUfig{.3}{modulos/moodle/IU/m_confirmacion_eliminar_instancia}{IU-M08c}{Mensaje de confirmación al presionar el botón Eliminar}

\subsubsection{Elementos Relevantes}

    \begin{itemize}
    \item {\bf Si}
    \item {\bf No}

    \end{itemize}

\subsubsection{Acciones relevantes}

    \begin{itemize}
    \item {\bf Confirmar la eliminación de la instancia}
    \item {\bf Cancelar la eliminación de la instancia}

    \end{itemize}

\clearpage
  % Configuraciones
    
\subsubsection{IU-C01: Pantalla creación de nueva instancia de actividad de competencia uno contra uno}

Esta pantalla es el formulario que se le muestra al usuario cuando quiere crear una nueva instancia de la actividad uno contra uno. No es necesario agregar una descripción.

    \IUfig{1}{modulos/comp/IU/p_creacion_competencia1vs1}{IU-C01}{%
        Pantalla de creación de nueva instancia de actividad de competencia uno contra uno}

\subsubsection{Elementos Relevantes}

    \begin{itemize}
    \item {\bf Nombre de actividad}
    \item {\bf Categoría de preguntas}
    \item {\bf Apuestas}
    \item {\bf Competencias ganadas requeridas}
    \item {\bf Los estudiantes deben de haber ganado un mínimo de competencias de:}
    \item {\bf Guardar cambios y regresar al curso}
    \item {\bf Guardar cambios y mostrar}
    \item {\bf Cancelar}
    \end{itemize}

\subsubsection{Acciones relevantes}

    \begin{itemize}
    \item {\bf Se puede crear una nueva instancia y mostrar su creación en la pantalla principal del curso}
    \item {\bf Se puede crear una nueva instancia y mostrar su pantalla principal}
    \item {\bf Se puede elegir como se finalizará la actividad}
    \end{itemize}

\clearpage
  % Configuraciones Generales
    
\subsubsection{IU-C02: Pantalla principal de actividad de competencia uno contra uno}

Esta pantalla es la principal de la competencia uno contra uno, en ella se puede ver el número de victorias que tiene el usuario actual y se le muestran los estudiantes a los que puede desafiar, también, si tiene desafíos pendientes se le muestran para que los pueda aceptar.

    \IUfig{1}{modulos/comp/IU/p_principal_competencia1vs1}{IU-C02}{%
        Pantalla principal de actividad de competencia uno contra uno}

\subsubsection{Elementos Relevantes}

    \begin{itemize}
    \item {\bf Número de victorias}
    \item {\bf Compañeros del curso}
    \item {\bf Desafíos pendientes}
    \item {\bf Tabla de posiciones}
    \item {\bf Historial}
    \item {\bf Desafiar}
    \item {\bf Aceptar desafío}
    \item {\bf Monedas a apostar}
    \end{itemize}

\subsubsection{Acciones relevantes}

    \begin{itemize}
    \item {\bf Desafiar a un estudiante}
    \item {\bf Aceptar un desafío de un usuario}
    \item {\bf Abrir la pestaña de tabla de posiciones}
    \item {\bf Abrir la pestaña de historial}
    \end{itemize}

\clearpage
  % Configuraciones Visuales
    
\subsubsection{IU-C03: Pantalla de creación de nueva instancia de actividad de competencia uno contra sistema}

 Descripción ...

    \IUfig{1}{modulos/comp/IU/p_creacion_competencia1vscpu}{IU-C03}{%
        Pantalla de creación de nueva instancia de actividad de competencia uno contra sistema}

\subsubsection{Elementos Relevantes}

    \begin{itemize}
    \item {\bf Lorem ipsum}
        ...
    \end{itemize}

\subsubsection{Acciones relevantes}

    \begin{itemize}
    \item {\bf Guardar cambios}
        ...

    \item {\bf Cancelar}
        ...
    \end{itemize}

\clearpage
  % Configuraciones Generales
    
\subsubsection{IU-C04: Pantalla principal de actividad de competencia uno contra sistema}

Esta pantalla es la principal de la competencia uno contra sistema, en ella se pueden ver las dificultades del sistema que han sido derrotadas, así como elegir el desafiar a una dificultad.

    \IUfig{1}{modulos/comp/IU/p_principal_competencia1vscpu}{IU-C04}{%
        Pantalla principal de actividad de competencia uno contra sistema}

        \subsubsection{Elementos Relevantes}

            \begin{itemize}
            \item {\bf Puntuación}
            \item {\bf Historial}
            \item {\bf Seleccione una dificultad}
            \item {\bf Empezar}
            \end{itemize}

        \subsubsection{Acciones relevantes}

            \begin{itemize}
            \item {\bf Desafiar a una dificultad}
            \item {\bf Abrir la pestaña de puntuaciones}
            \item {\bf Abrir la pestaña de historial}
            \end{itemize}

        \clearpage

    
\subsubsection{IU-C05: Pantalla Actualización de instancia de actividad de competencia uno contra sistema}

 Descripción ...

    \IUfig{1}{modulos/comp/IU/p_actualizacion_competencia1vscpu}{IU-C05}{%
        Pantalla de actualizacion instancia de actividad de competencia uno contra sistema}

\subsubsection{Elementos Relevantes}

    \begin{itemize}
    \item {\bf Lorem ipsum}
        ...
    \end{itemize}

\subsubsection{Acciones relevantes}

    \begin{itemize}
    \item {\bf Lorem ipsum}
        ...
    \end{itemize}

\clearpage

    
\subsubsection{IU-C06: Pantalla Actualización de instancia de actividad de competencia uno contra uno}

 Esta pantalla es el formulario que se le muestra al usuario cuando quiere actualizar una instancia de la actividad uno contra uno.

    \IUfig{1}{modulos/comp/IU/p_actualizacion_competencia1vs1}{IU-C06}{%
        Pantalla de actualización instancia de actividad de competencia uno contra uno}

        \subsubsection{Elementos Relevantes}

            \begin{itemize}
            \item {\bf Nombre de actividad}
            \item {\bf Categoría de preguntas}
            \item {\bf Apuestas}
            \item {\bf Competencias ganadas requeridas}
            \item {\bf Los estudiantes deben de haber ganado un mínimo de competencias de:}
            \item {\bf Guardar cambios y regresar al curso}
            \item {\bf Guardar cambios y mostrar}
            \item {\bf Cancelar}
            \end{itemize}

        \subsubsection{Acciones relevantes}

            \begin{itemize}
            \item {\bf Se puede actualizar una instancia y mostrar su creación en la pantalla principal del curso}
            \item {\bf Se puede actualizar una instancia y mostrar su pantalla principal}
            \item {\bf Se puede elegir como se finalizará la actividad}
            \end{itemize}

        \clearpage

    
\subsubsection{IU-C07: Pantalla de historial de actividad de competencia uno contra uno}

 Descripción ...

    \IUfig{1}{modulos/comp/IU/p_historial_competencia1vs1}{IU-C07}{%
        Pantalla de historial de actividad de competencia uno contra uno}

\subsubsection{Elementos Relevantes}

    \begin{itemize}
    \item {\bf Lorem ipsum}
        ...
    \end{itemize}

\subsubsection{Acciones relevantes}

    \begin{itemize}
    \item {\bf Lorem ipsum}
        ...
    \end{itemize}

\clearpage

    
\subsubsection{IU-C08: Pantalla de tabla de puntuaciones de actividad de competencia uno contra uno}

 Descripción ...

    \IUfig{1}{modulos/comp/IU/p_tablap_competencia1vs1}{IU-C08}{%
        Pantalla de tabla de puntuaciones de actividad de competencia uno contra uno}

\subsubsection{Elementos Relevantes}

    \begin{itemize}
    \item {\bf Lorem ipsum}
        ...
    \end{itemize}

\subsubsection{Acciones relevantes}

    \begin{itemize}
    \item {\bf Lorem ipsum}
        ...
    \end{itemize}

\clearpage



    

\subsubsection{Diseño de complementos}



A continuación se presenta cómo los submódulos de competencia
se implementan en moodle.\\


\noindent Resumiendo el módulo de competencia tiene 2 actividades establecidas, llamadas;
competencia uno contra uno y competencia uno contra sistema.
Ambas actividades deben aparecer dentro de la lista de actividades de moodle. Para ello
moodle cuenta con un tipo de complemento que se denomina \textbf{'mod'}, este tipo de complemento al ser instalado
en una plataforma de moodle, crea una nueva opción a la lista de actividades.\\

\noindent Tomando en consideración lo anterior y que existe el complemento gamedlemaster, se presenta en la figura \ref{fig:diseno-comp-comp}
los complementos contemplados y las dependencias entre los mismos.


    \addfigure{1}{modulos/comp/diagrams/diseno_complementos}{fig:diseno-comp-comp}{Implementación del modulo de competencia}


Cada complemento en la figura \ref{fig:diseno-comp-comp} está representado con una cadena que sigue el formato 'tipo\_de\_complemento:nombre\_de\_complemento'. Los tipos de complemento son;
\begin{itemize}
    \item \textbf{mod} - Este complemento permite crear una actividad que aparece en la lista de actividades a agregar a un curso.
    \item \textbf{local} -  Este complemento puede ser usado para múltiples propósitos relacionados con la gestión de la información.
    \item \textbf{block} - Este complemento permite desplegar una sección en la mayoría de las páginas de moodle, la cuál puede representar código html.
\end{itemize}

La función de cada uno de los complementos presentados en la figura \ref{fig:diseno-comp-comp} son:


\begin{itemize}
    \item \textbf{gamedlemaster} Definir la base de datos y los eventos a manejar.
    \item \textbf{gmcompcpu} Definir la competencia uno contra sistema.
    \item \textbf{gmcompvs} Definir la competencia uno contra uno.
    \item \textbf{gmcs} Entregar las monedas por ganar cada una de las competencias anteriores.
\end{itemize}

El complemento de tipo  \textbf{'mod'} tiene un requerimiento en su nombre, el cual es; 'El nombre del complemento a instalar debe ser igual a un nombre
de una de las tablas en la base de datos'. Debido a esto y que moodle no soporta nombres de complementos que contengan guiones bajos, el
nombre de la tabla ya no puede llevarlos.\\


%\subsubsection{Diagrama de componentes} % TODO CHANGE FOR INPUTS
%\subsubsection{Diagrama de clases} % TODO CHANGE FOR INPUTS
\clearpage
\subsubsection{Comportamiento del sistema de la competencia uno contra sistema}
    

A continuación se presenta de manera general el algoritmo que sigue el
sistema para responder un cuestionario.

\noindent Al diseñar el algoritmo se tuvo en cuenta que no respondiera únicamente
de manera correcta o incorrecta determinado por una probabilidad.
Se diseño para que cada nivel de dificultad del sistema hiciera las mismas acciones,
pero con más oportunidades o más probabilidades de contestar correctamente.
Es por ello que se consideran 4 factores:

\begin{enumerate}
    \item Saber cuántas respuestas se deben elegir para responder correctamente la pregunta.
    \item El número de intentos para responder la pregunta correctamente.
    \item La probabilidad de que una respuesta sea elegida ante las otras respuestas.
    \item La posibilidad de reducir el número de respuestas que se tienen.
\end{enumerate}

El flujo general que no depende de la dificultad del sistema está representado en la figura \ref{fig:algoritmo-cpu-1},
en dicha figura se presentan cuadros azules cuya especificación se encuentran en las figuras:


\begin{enumerate}
    \item 'Calcular intentos [i]' está representado en la figura \ref{fig:algoritmo-cpu-2}.
    \item 'Calcular [c]' está representado en la figura \ref{fig:algoritmo-cpu-3} .
    \item 'Ponderar las respuestas' está representado en la figura \ref{fig:algoritmo-cpu-4} .
    \item 'Intentar descartar la respuesta [r] como una opción' está representado en la figura \ref{fig:algoritmo-cpu-5} .
\end{enumerate}

\clearpage
    \addfigure{0.8}{modulos/comp/diagrams/algoritmo_parte_1}{fig:algoritmo-cpu-1}{Diagrama de flujo del algoritmo, General}

\clearpage
    \addfigure{0.8}{modulos/comp/diagrams/algoritmo_parte_2}{fig:algoritmo-cpu-2}{Diagrama de flujo del algoritmo, 'Obtener intentos'}


\clearpage
    El algoritmo debe tomar en cuenta los casos en que las preguntas tienen más de una respuesta correcta
    y los casos en que la respuesta correcta a una pregunta es una combinación de 2 o más respuestas.\\
    \addfigure{0.8}{modulos/comp/diagrams/algoritmo_parte_3}{fig:algoritmo-cpu-3}{Diagrama de flujo del algoritmo, 'Respuestas a elegir'}
\clearpage
    Debido a que el sistema elige una pregunta de manera al azar,
    dependiendo de la dificultad se hace más probable que elija una respuesta correcta o que se elija una respuesta incorrecta.
    \addfigure{0.8}{modulos/comp/diagrams/algoritmo_parte_4}{fig:algoritmo-cpu-4}{Diagrama de flujo del algoritmo, 'Ponderación de respuestas'}
\clearpage
    Para aprovechar los intentos que tiene un sistema,
    se hace que cada nivel de dificultad pueda reducir las respuestas posibles una vez haya seleccionado una respuesta.
    \addfigure{0.8}{modulos/comp/diagrams/algoritmo_parte_5}{fig:algoritmo-cpu-5}{Diagrama de flujo del algoritmo, 'Reducir opciones'}
\clearpage


\noindent Para corroborar el funcionamiento del algoritmo y que cada dificultad tenga más probabilidad de obtener una mejor calificación se hicieron pruebas.
Dichas pruebas consistían en eligir una dificultad y que el sistema contestara 10,000 veces el mismo cuestionario y con ello obtener
el promedio de su calificación en ese cuestionario \ref{table:resultados-calificaciones-algoritmo-sistema},
el promedio de obtención de cada una de las calificaciones  \ref{table:resultados-calificacion-promedio-algoritmo-sistema}
y la desviación estándar que se tiene \ref{table:resultados-desviacion-algoritmo-sistema}.\\

\begin{table}[h!]
    \centering
    \begin{tabular}{|c|c|c|c|c|} \hline
        Dificultad del sistema & Fácil &     Normal &    Difícil &   Imposible \\\hline
        Calificación de 0 &  0.73\% &    0.01\% &   0.0\%  &  0.0\%   \\\hline
        Calificación de 1 &  3.76\% &    0.14\% &   0.0\%  &  0.0\%\\\hline
        Calificación de 2 &  5.6\% &    0.26\% &   0.0\%  &  0.0\%\\\hline
        Calificación de 3 &  12.33\% &    1.55\% &   0.0\%  &  0.0\%\\\hline
        Calificación de 4 &  20.79\% &    4.68\% &   0.07\%  &  0.0\%\\\hline
        Calificación de 5 &  17.69\% &    8.93\% &   0.31\%  &  0.01\%\\\hline
        Calificación de 6 &  21.55\% &    21.32\% &   3.15\%  &  0.56\%\\\hline
        Calificación de 7 &  10.44\% &    19.99\% &   3.68\%  &  0.63\%\\\hline
        Calificación de 8 &  5.01\% &    21.95\% &   21.86\%  &  10.35\%\\\hline
        Calificación de 9 &  1.81\% &    12.08\% &   11.87\%  &  5.99\%\\\hline
        Calificación de 10 &  0.29\% &    9.09\% &   59.06\%  &  82.46\%\\\hline
    \end{tabular}
    \caption{Tabla de resultados- Prueba algoritmo del sistema 'Porcentaje de obtención de calificación'}
    \label{table:resultados-calificaciones-algoritmo-sistema}
\end{table}


\begin{table}[h!]
    \centering
    \begin{tabular}{|c|c|c|c|c|} \hline
        Dificultad del sistema &                 Fácil &     Normal &    Difícil &   Imposible \\\hline
        Promedio de calificación &  6.0142 &    7.8878 &    9.4805 &    9.8186 \\ \hline
    \end{tabular}
    \caption{Tabla de resultados- Prueba algoritmo del sistema 'Calificación promedio'}
    \label{table:resultados-calificacion-promedio-algoritmo-sistema}
\end{table}


\begin{table}[h!]
    \centering
    \begin{tabular}{|c|c|c|c|c|} \hline
        Dificultad del sistema &                 Fácil &     Normal &    Difícil &   Imposible \\\hline
        Porcentaje de confianza &  2.0210884097436 &    3.2341593158042 &    4.7286951952952 &    5.1423616092209 \\\hline
    \end{tabular}
    \caption{Tabla de resultados- Prueba algoritmo del sistema 'Desviación estándar'}
    \label{table:resultados-desviacion-algoritmo-sistema}
\end{table}

Se grafica los resultados de las dificultades para una mejor evaluación de rendimiento en la imagen \ref{fig:algoritmo-resultados-grafica}.

\addfigure{0.8}{modulos/comp/diagrams/grafica_dificultades}{fig:algoritmo-resultados-grafica}{Gráfica de resultados de probabilidad}


\subsection{Pruebas}

    
\TestCase{CPC-C01}{Crear nuevas instancias de la actividad de competencia 1 contra 1}


\subsection{Funcionalidades de moodle}

En esta sección se abordan las funcionalidades propias de moodle que fueron utilizadas para el desarrollo
de los módulos de competencia, así como una descripción de su objetivo y problemas encontrados al utilizarlas.\\

  
\subsection{Entidades de moodle}

Debido a que moodle cuenta con más de 400 entidades en su versión 3.5, se opta
por mostrar 2 subconjuntos que muestren las entidades que se utilizan para el proyecto.\\

\noindent El primer subconjunto es aquel que explica la forma en que moodle implementa los cursos,
secciones de curso, actividades, usuarios y roles (el cual se presenta en la figura \ref{fig:BD-ER-M1}),
mientras que el segundo subconjunto muestra como moodle maneja toda la
estructura de las preguntas creadas por el profesor y respondidas por el estudiante
(el cual se presenta en la figura \ref{fig:BD-ER-M2}).

\noindent El objetivo de ambos esquemas (\ref{fig:BD-ER-M1} y \ref{fig:BD-ER-M2}) es expresar la idea general que abarcan ambos subconjuntos.

\clearpage
\addfigure{0.7}{analisis/diagrams/db_module_structure}{fig:BD-ER-M1}{Esquema de la base de datos de moodle 'Cursos'}


\noindent Utilizando la figura \ref{fig:BD-ER-M1}, se obtuvieron las siguientes reglas y características que tiene moodle respecto a los usuarios en un curso y a la estructura de los cursos.
\begin{enumerate}
    \item Un usuario -{\it mdl\_user}- tiene un rol -{\it mdl\_role}- en un cierto contexto -{\it mdl\_context}-, cuyo  '{\it context\_level}' sea igual a cincuenta(50).
    \item Si el contexto '{\it context\_level}' es de 50, el atributo '{\it instance\_id}' hace referencia al atributo '{\it id}' de un curso -{\it mdl\_course}-.
    \item El curso -{\it mdl\_course}- tiene varias secciones -{\it mdl\_course\_sections}-.
    \item Cada seccion -{\it mdl\_course\_sections}- tiene varias actividades -{\it mdl\_course\_modules}- que pertenecen a un tipo de actividad -{\it mdl\_modules}-.
    \item Por cada registro en tipo de actividad -{\it mdl\_modules}-, se tiene una entidad que lleva el mismo nombre.
    \item El atributo '{\it instance\_id}' de una actividad  -{\it mdl\_course\_modules}- apunta a diferentes entidades. La entidad a la que apunta depende del nombre del tipo de actividad -{\it mdl\_modules}-.
    \item Un usuario -{\it mdl\_user}- se inscribió -{\it mdl\_user\_enrolments}- a un curso -{\it mdl\_course}-, por medio de un formato soportado de inscripción -{\it mdl\_enrol}-.
\end{enumerate}

\clearpage

 \addfigure{0.7}{analisis/diagrams/db_module_questions}{fig:BD-ER-M2}{Esquema de la base de datos de moodle 'Preguntas' }



\noindent Utilizando la figura \ref{fig:BD-ER-M2}, se obtuvieron las siguientes reglas y características que tiene moodle respecto a las preguntas.
\begin{enumerate}
    \item Las preguntas -{\it mdl\_question}- tienen versiones -{\it mdl\_question\_attempts}-.
    \item Una pregunta -{\it mdl\_question}- pertenece a un banco de preguntas -{\it mdl\_question\_categories}-.
    \item La versión de una pregunta -{\it mdl\_question\_attempts}- es contestada -{\it mdl\_question\_usages}- en un determinado contexto -{\it mdl\_context}-.
    \item Un usuario -{\it mdl\_user}- responde una versión de una pregunta -{\it mdl\_question\_attempt\_stepts}-.
    \item El responder una versión de una pregunta -{\it mdl\_question\_attempt\_stepts}- conlleva pasos\\ -{\it mdl\_question\_attempt\_stept\_data}-, los cuales son: cómo se muestra, si ya se terminó de responder y qué se respondió.
\end{enumerate}


 A continuación se presenta la especificación de las entidades del esquema de base
 de datos de moodle que son relevantes para el desarrollo de los módulos y submódulos
 de proyecto.

    \begin{cdtEntidad}{mdl-config-plugins}{Configuración de Plugin}{%
    Es una tabla del núcleo de moodle que almacena todas las configuraciones globales
    relacionadas a los plugins instalados, al iniciar moodle las configuraciones de los
    plugins instalados y habilitados se cargan en memoria.}

	    \brAttr{id}{Id}{tInt}{%
	        Es el dígito que representa el identificador único para una configuración
            específica de un plugin.\par

            \it Restricciones:
            \refElem{tPrimaryKey},
            \refElem{tAutoIncrement}.
        }

        \brAttr{plugin}{Plugin}{tVarchar}{%
            Cadena de caracteres del nombre identificador del plugin al cual pertenece
            la configuración.\par

            \it Restricciones:
            \refElem{tRequired},
            \refElem{tRange} (0,100),
            \refElem{tUniqueKey}
        }

        \brAttr{name}{Nombre}{tVarchar}{%
            Cadena de caracteres que representa el nombre de la configuración de un
            plugin en específico.\par

            \it Restricciones:
            \refElem{tUniqueKey},
            \refElem{tRange} (0,100),
            \refElem{tRequired}
        }

        \brAttr{value}{Valor}{tVarchar}{%
            Cadena que almacena el valor de una configuración perteneciente a alguno
            de los plugins instalados.\par

            \it Restricciones:
            \refElem{tRange} (0,4294967295),
            \refElem{tRequired}
        }
    \end{cdtEntidad}\schemeName{config\_plugins}

    \begin{cdtEntidad}{mdl-user}{Usuario de moodle}{%
    Es una tabla del núcleo de moodle que contiene toda la información que se
    almacena de los usuarios en la plataforma, independientemente del rol que
    estos contenga, esta relación contiene más de 53 atributos, sin embargo solo
    se detallan aquellos relevantes.}

	    \brAttr{id}{Id}{tInt}{%
	        Es el dígito que representa el identificador único para cada uno
            de los usuarios en moodle.\par

            \it Restricciones:
            \refElem{tPrimaryKey},
            \refElem{tAutoIncrement}.
        }
	    \brAttr{username}{nombre de usuario}{tVarchar}{%
	        .\par

            \it Restricciones:
            \refElem{tRequired},
            \refElem{tLength} 0-100
        }
	    \brAttr{password}{contraseña}{tVarchar}{%
	        .\par

            \it Restricciones:
            \refElem{tRequired},
            \refElem{tLength} 0-255.
        }
	    \brAttr{firstname}{nombre}{tVarchar}{%
	        .\par

            \it Restricciones:
            \refElem{tRequired},
            \refElem{tLength} 0-100
        }
	    \brAttr{lastname}{apellido}{tVarchar}{%
	        .\par

            \it Restricciones:
            \refElem{tRequired},
            \refElem{tLength} 0-100
        }
	    \brAttr{email}{correo}{tVarchar}{%
	        .\par

            \it Restricciones:
            \refElem{tRequired},
            \refElem{tLength} 0-100
        }
	    \brAttr{lastaccess}{último registro}{tInt}{%
	        .\par

            \it Restricciones:
            \refElem{tRequired},
            \refElem{tLength} 10
        }
	    \brAttr{city}{ciudad}{tVarchar}{%
	        .\par

            \it Restricciones:
            \refElem{tRequired},
            \refElem{tLength} 0-120
        }
	    \brAttr{country}{pais}{tVarchar}{%
	        .\par

            \it Restricciones:
            \refElem{tRequired},
            \refElem{tLength} 2
        }

    \end{cdtEntidad}\schemeName{user}

    \begin{cdtEntidad}{mdl-course}{Curso de moodle}{%
    Es una tabla del núcleo de moodle que contiene la información principal de cada
    curso registrado en moodle. Esta entidad contiene 31 atributos, a continuación se
    detallan los atributos relevantes para la especificación de este proyecto.}

	    \brAttr{id}{Id}{tInt}{%
	        Es el dígito que representa al identificador único para cada uno
            de los cursos en moodle.\par

            \it Restricciones:
            \refElem{tPrimaryKey},
            \refElem{tAutoIncrement}.
        }

	    \brAttr{format}{formato}{tVarchar}{%
	        Es el dígito que representa al identificador único para cada uno
            de los cursos en moodle.\par

            \it Restricciones:
            \refElem{tRequired}.
            \refElem{tDefault} topics,
            \refElem{tLength} 0-21.
        }

	    \brAttr{fullname}{nombre completo}{tVarchar}{%
	        Es el nombre completo que se le asigna al curso.\par

            \it Restricciones:
            \refElem{tRequired}.
            \refElem{tLength} 0-21.
        }

	    \brAttr{shortname}{nombre corto}{tVarchar}{%
            Es el nombre corto que se le asigna al curso.\par

            \it Restricciones:
            \refElem{tRequired}.
            \refElem{tLength} 0-21.
        }

    \end{cdtEntidad}\schemeName{course}

    \begin{cdtEntidad}{mdl-course-section}{Sección del curso de moodle}{%
    }
	    \brAttr{id}{Id}{tInt}{%
	        Es el dígito que representa al identificador único para cada sección
            de los cursos en moodle.\par

            \it Restricciones:
            \refElem{tPrimaryKey},
            \refElem{tAutoIncrement}.
        }

        \brAttr{name}{nombre}{tVarchar}{%
	        Es el dígito nombre que permite identificar a una sección dentro de un curso
            en moodle.\par

            \it Restricciones: ...
        }
    \end{cdtEntidad}\schemeName{course\_sections}

    \begin{cdtEntidad}{mdl-course-format-options}{Opciones del formato del curso}{%
    }
	    \brAttr{id}{Id}{tInt}{%
	        Es el dígito que representa al identificador único para cada uno
            de los cursos en moodle.\par

            \it Restricciones:
            \refElem{tPrimaryKey},
            \refElem{tAutoIncrement}.
        }

	    \brAttr{courseid}{Id}{tInt}{%
	        Es el dígito que representa al identificador único para cada uno
            de los cursos en moodle.\par

            \it Restricciones:
            \refElem{tForeignKey},
            \refElem{tRequired}
        }

	    \brAttr{format}{formato}{tVarchar}{%
	        Es el dígito que representa al identificador único para cada uno
            de los cursos en moodle.\par

            \it Restricciones:
            \refElem{tRequired}.
            \refElem{tDefault} topics,
            \refElem{tLength} 0-21.
        }

	    \brAttr{name}{opcion}{tVarchar}{%
	        Es el dígito que representa al identificador único para cada uno
            de los cursos en moodle.\par

            \it Restricciones:
            \refElem{tPrimaryKey},
            \refElem{tLength} 0-100
        }

	    \brAttr{value}{valor}{tVarchar}{%
	        Es el dígito que representa al identificador único para cada uno
            de los cursos en moodle.\par

            \it Restricciones:
            \refElem{tRequired}
        }

    \end{cdtEntidad}\schemeName{course\_format\_options}

    \begin{cdtEntidad}{mdl-course-category}{Categoria de curso}{%
      .}
        \brAttr{id}{id}{tInt}{%
        .}
        \brAttr{name}{nombre}{tInt}{%
        .}

    \end{cdtEntidad}\schemeName{course\_category}

    \begin{cdtEntidad}{mdl-course-module}{Actividad del curso}{%
    .}
	    \brAttr{id}{Id}{tInt}{%
	        Es el dígito que representa al identificador único para cada uno
            de los cursos en moodle.\par

            \it Restricciones:
            \refElem{tPrimaryKey},
            \refElem{tAutoIncrement}.
        }
        \brAttr{course}{curso}{tInt}{%
        .}
        \brAttr{module}{actividad}{tInt}{%
        .}
        \brAttr{section}{sección}{tInt}{%
        .}
    \end{cdtEntidad}\schemeName{course\_module}

    \begin{cdtEntidad}{mdl-course-module-completion}{Actividad del curso para alumno}{%
    .}
	    \brAttr{id}{Id}{tInt}{%
	        Es el dígito que representa al identificador único para cada uno
            de los cursos en moodle.\par

            \it Restricciones:
            \refElem{tPrimaryKey},
            \refElem{tAutoIncrement}.
        }
        \brAttr{coursemoduleid}{actividad}{tInt}{%
        .}
        \brAttr{userid}{usuario}{tInt}{%
        .}
        \brAttr{completionstate}{completitud}{tBoolean}{%
        .}
    \end{cdtEntidad}\schemeName{course\_module}

    \begin{cdtEntidad}{Plugin}{Plugin}{%
    La forma en que moodle obtiene información acerca de los plugins es analizando
    los archivos internos de cada uno, a pesar de que los plugins no forman parte
    del esquema de base de datos, si forman parte del modelo de información que
    utiliza Moodle.}

	    \brAttr{componente}{Componente}{tVarchar}{%
	        Cadena compuesta por el tipo de plugin y el nombre del mismo, que
            representa a la clase principal del plugin que contiene los métodos
            principales del plugin.\par

            \it Restricciones: Ninguna
        }

	    \brAttr{pluginname}{Nombre}{tVarchar}{%
	        Es el nombre del plugin obtenido de los archivos de
            internacionalización presentes en el plugin, el valor de esta cadena
            depende del lenguaje seleccionado en moodle.\par

            \it Restricciones: Ninguna
        }

	    \brAttr{fullpath}{Ruta absoluta}{tPath}{%
	        La ruta absoluta de un plugin denota la ubicación del plugin en el
            sistema de archivos, esta ruta está compuesta por la ruta absoluta
            de la instalación de moodle, la carpeta correspondiente al tipo del
            plugin y el nombre del plugin.\par

            \it Restricciones: Formato ``/path/to/moodle/plugintype/pluginname''
        }

	    \brAttr{path}{Ruta relativa}{tPath}{%
	        La ruta relativa denota la ubicación del plugin dentro de la carpeta
            donde se encuentran los archivos de moodle, esta ruta está compuesta
            por la carpeta correspondiente al tipo del plugin y el nombre del
            plugin.\par

            \it Restricciones: Formato ``plugintype/pluginname''
        }

	    \brAttr{version}{Versión}{tVersion}{%
	        Numero entero de longitud de 10 dígitos que representa la versión del
            plugin.\par

            \it Restricciones: Ninguna adicional al tipo de dato
        }

	    \brAttr{moodle}{Versión de Moodle}{tVersion}{%
	        Número entero de longitud de 10 dígitos que representa la versión de
            moodle en la que se puede instalar el plugin.\par

            \it Restricciones: Ninguna adicional al tipo de dato
        }

        \brAttr{dependencies}{Dependencias}{tObject}{%
            Objeto que almacena un conjunto de claves con sus respectivos valores,
            donde cada clave representa el nombre del componente del plugin y el valor
            es la \refElem{Plugin.version} requerida del mismo.

            \it Restricciones: Ninguna
        }

        \brAttr{icon}{ícono}{tImage}{%
            Imagen para el ícono del plugin, debe estar contenida en el directorio
            {\it pix/} del plugin y tener como nombre {\it icon.png} o {\it icon.svg},
            moodle recomienda tener ambos archivos por si los navegadores no soportan
            algún tipo de archivo \cite{moodlePluginfiles}.\par

            \it Restricciones: El nombre debe ser icono con extensiones png o svg
        }

    \end{cdtEntidad}



    
\clearpage
\subsection{Entidades del módulo de seguimiento}

 En esta sección se presentan las entidades del módulo de seguimiento,
 las cuales están representadas en la figura  \ref{fig:BD_ER_S_PD},
 el cual muestra en color amarillo las entidades propias del módulo,
 en azul las entidades que se utilizan de moodle  y en blanco las entidades generales.

        \addfigure{0.7}{analisis/diagrams/db_seguimiento}{fig:BD_ER_S_PD}{Esquema de la base de datos del módulo 'Seguimiento'}

        \begin{cdtEntidad}{seg-gmpregdiarias}{Actividad preguntas diarias}{

        Esta entidad es requerida por moodle para poder guardar los registros de cada instancia de la actividad creadas a lo largo de los cursos. Los requisitos que debe cumplir esta entidad son:\\
            1.- El nombre de la tabla debe contener el mismo nombre que el nombre del plugin.\\
            2.- Se deben contar con los atributos \{id, course, name, intro, introformat, timemodified.\}
            }

            \brAttr{id}{id}{tInt}{%
                Es el número que representa el identificador único.\par

                \it Restricciones:
                \refElem{tPrimaryKey},
                \refElem{tAutoIncrement}
            }

            \brAttr{course}{Curso}{tInt}{%
                Identificador del curso donde se creó la instancia.\par

                \it Restricciones:
                \refElem{tForeignKey}
                \refElem{tRequired}.
            }

            \brAttr{name}{Nombre}{tVarchar}{%
                El nombre que lleva la instancia de la actividad.\par

                \it Restricciones:
                \refElem{tLength} $1-255$,
                \refElem{tRequired}.
            }

            \brAttr{intro}{Introducción}{tText}{%
                La introducción de la instancia de la actividad.\par

                \it Restricciones:
                \refElem{tRequired}.
            }
            \brAttr{introformat}{Formato de la introducción}{tInt}{%
                Entero que especifica en qué formato ha sido escrita la introducción del curso.\par

                \it Restricciones:
                \refElem{tNatural}
                \refElem{tRequired}.
            }


            \brAttr{timemodified}{Tiempo de la última modificación}{tTime}{%
                Fecha en la que se actualizó la instancia de la actividad.
                \it Restricciones:
                \refElem{tRequired}.

            }
            \brAttr{mdl-question-categories-id}{banco de preguntas}{tInt}{%
                Identificador del banco de preguntas del cual se obtendrán las preguntas a mostrar en la instancia de la actividad.\par

                \it Restricciones:
                \refElem{tForeignKey},
                \refElem{tRequired}.
            }

        \end{cdtEntidad}
        \schemeName{gmpregdiarias}


        \begin{cdtEntidad}{seg-gmdl-intento-diario}{Intento diario}{
            Esta entidad contiene los intentos diarios que se llevan a cabo por cada instancia.}

            \brAttr{id}{id}{tInt}{%
                Es el número que representa el identificador único.\par

                \it Restricciones:
                \refElem{tPrimaryKey},
                \refElem{tAutoIncrement}
            }

            \brAttr{gmdl-usuario-id}{Usuario gamificado}{tInt}{%
                Identificador del usuario que respondió el intento diario.

                \it Restricciones:
                \refElem{tForeignKey}
                \refElem{tRequired}.
            }
            \brAttr{gmdl-preg-diarias-id}{Instancia de la actividad}{tInt}{%
                Identificador de la instancia en la que se llevó acabo el intento.\par

                \it Restricciones:
                \refElem{tForeignKey}
                \refElem{tRequired}.
            }

            \brAttr{fecha}{Fecha}{tTime}{%
                Fecha en la que se realizó el intento.

                \it Restricciones:
                \refElem{tRequired}.
            }

            \brAttr{calificacion}{Calificación}{tDouble}{%
                Calificación que obtuvo el usuario al responder la pregunta diaria.

                \it Restricciones:
                \refElem{tRange} (-1.0 a 1.0).
                \refElem{tRequired}.
            }

            \brAttr{mdl-question-id}{Pregunta}{tInt}{%
                Identificador de la pregunta que se respondió en el intento.

                \it Restricciones:
                \refElem{tForeignKey}
                \refElem{tRequired}.
            }

        \end{cdtEntidad}
        \schemeName{gmdl\_intento\_diario}

    
\subsection{Análisis}

 Este apartado contiene el análisis requerido para la elaboración de módulo de personalización,
 contiene la especificación del alcance de este módulo, la descripción de las funcionalidades
 a desarrollar, la reglas de negocio que rigen el comportamiento del módulo, y por último la
 especificación de los casos de uso a los que brinda soporte.



\subsubsection{Reglas de negocio} %========================================================

 En esta sección se especifican todas las reglas de negocio relevantes para el módulo de
 experiencia. Las reglas de negocio que establece moodle son diferenciadas por tener la letra {\it M}
 antecediendo al número consecutivo en su identificador.

    % No se puede tener seleccionado  más de un objeto de un mismo tipo
    % Los objetos solo pueden ser adquiridos una vez
    % En caso de no tener un objeto seleccionado de un cierto tipo, se debe cargar una configuración por defecto
    %\begin{BusinessRule}[%
Autor/Daniel Isai Ortega Zúñiga,%
Version/0.1,%
Estado/edicion]%
%
{BR-E08}{Valores iniciales de experiencia de un curso}

     \BRitem[control]{Revisada por}{Pendiente.}

 \BRsection[control]{Atributos}
    % Clases: \bcCondition, \bcIntegridad, \bcAutorization o \bcDerivation
    % Tipos: \btEnabler, \btTimer o \btExecutive
    % Niveles: \blControlling o \blInfluencing.

    \BRitem[admin]{Clase}{\bcIntegridad}%

    \BRitem[admin]{Tipo}{\btTimer}%

    \BRitem[admin]{Nivel}{\blControlling}

    \BRitem{Descripción}{%
        Cuando un \refElem{xp-course} es creado la \refElem[experiencia total del curso]%
        {xp-scheme-settings.courseXP} de ser dividida uniformemente entre las
        \refElem[secciones del curso gamificado]{xp-course-section}. Si la división del
        total de experiencia entre el número de secciones genera un residuo entonces este
        se deberá agregan a la última sección del curso.
    }

%   \BRitem{Sentencia}{%
%       Si $fecha$
%   }%

    \BRitem{Ejemplo positivo}{\hfill\par%
        \begin{itemize}
        \item ...
        \end{itemize}
    }

    \BRitem{Ejemplo negativo}{\hfill\par%
        \begin{itemize}
        \item ...
        \end{itemize}
    }

 \end{BusinessRule}
 % Valores iniciales de experiencia del curso

    % INPUT: Cursos Igualitarios.
    % INPUT: Otorgar experiencia
    % INPUT: Administración de experiencia en el curso
    \begin{BusinessRule}[%
Autor/David Flores Casanova,%
Version/0.1,%
Estado/edicion]%
%
{BR-P01}{Solo se puede tener un objeto seleccionado de cada tipo de objeto}

     \BRitem[control]{Revisada por}{Pendiente.}

 \BRsection[control]{Atributos}
    % Clases: \bcCondition, \bcIntegridad, \bcAutorization o \bcDerivation
    % Tipos: \btEnabler, \btTimer o \btExecutive
    % Niveles: \blControlling o \blInfluencing.
    
    \BRitem[admin]{Clase}{\bcIntegridad}%
        
    \BRitem[admin]{Tipo}{\btEnabler}%
        
    \BRitem[admin]{Nivel}{\blControlling}
    
    \BRitem{Descripción}{%
       Un usuario solo puede tener un objeto de un mismo tipo seleccionado para mostrarse en el perfil gamificado.
    }

    \BRitem{Ejemplo positivo}{\hfill\par%
        Un usuario tiene seleccionado para su perfil;
        \begin{itemize}
            \item Un objeto de tipo imagen.
            \item Un objeto de tipo estilo de marco.
            \item Un objeto de tipo color de marco.
        \end{itemize}
    }

    \BRitem{Ejemplo negativo}{\hfill\par%
        Un usuario tiene seleccionado para su perfil;
        \begin{itemize}
            \item Tres objetos de tipo imagen.
        \end{itemize}
        }
    
 \end{BusinessRule}

\clearpage

\subsubsection{Mensajes}
  

    \begin{mensaje2}{MSG-P01}{Cambios guardados}{Operación exitosa}
        \item[Redacción:] ¡Cambios guardados!
    \end{mensaje2}
\subsubsection{Casos de uso} % ============================================================

 En este apartado se especifican todos los casos de usos contemplados para el módulo de
 personalización, para cada caso de uso se especifica su tabla de atributos la cual indica que casos
 de prueba deberán ejecutarse correctamente para corroborar la completitud del caso de uso.

\subsubsection*{Diagrama de casos de uso}

 En la figura \ref{personalizacion:usecases} se detalla el diagrama de casos de uso correspondiente al módulo
 de personalización. Los casos de uso de moodle (en color blanco) son modelados como casos de uso
 abstractos, mientras que los casos de uso del módulo de personalización son diferenciados por el
 color azul, en total el desarrollo de este módulo consiste en 17 casos de uso principales.

    \addfigure{0.6}{modulos/person/diagrams/UseCases}{personalizacion:usecases}{%
        Diagrama de casos de uso del módulo de personalización}

 \noindent
 Debido a que los plugins a desarrollar son elementos opcionales para Moodle, solo se puede
 acceder a los casos de uso del módulo de competencia a través de puntos de extensión de los
 casos de uso de moodle. Por otra parte los casos de uso que serán documentados en esta sección
 serán los del módulo de competencia debido a que Moodle proporciona en su página oficial, guías
 e instructivos como documentación de las funcionalidades que brinda.

    % MODULO DE EXPERIENCIA


% \ucstEnEdicion     Al terminar una revisión/aprobación con observaciones
%                    y al inicio del CU.
%
% \ucstEnRevision    Al terminar la edición del CU (version += 0.1).
% \ucstEnAprobacion  Al pasar la revision sin observaciones.
% \ucstAprobado      Al ser aprobado por el usuario (version += 1.0)

\begin{UseCase}[%
Autor/David Flores Casanova,%
Version/0.1,%
Estado/\ucstEnRevision]%
%
{CU-P01}{Ver perfil gamificado}{%
%
Permite a un usuario (Ya sea un \refElem{aProfesor}, un \refElem{aAdministrador} o un \refElem{aEstudiante})
 de moodle ver las configuraciones de personalización que tiene activas en su perfil así como las monedas que tiene disponibles.
 La conclusión de la trayectoria principal de esta caso de uso es una precondición para que
 algunos casos de uso del módulo de personalización puedan ejecutarse.\\%
 Este caso de uso es una extensión del caso de uso {\it Iniciar sesión} que es propio de moodle.}

	\UCitem[control]{Revisor}{ Sin asignar }
	\UCitem[control]{Último cambio}{ 17/NOV/19 }

 \UCsection{Atributos}

    \UCitem{Actor(es)}{%
        \refElem{aProfesor},
        \refElem{aAdministrador},
        \refElem{aEstudiante}
    }

	\UCitems{Propósito}{%
        \Titem El usuario quiere agregar un objeto más a su colección, para poder usarlo en la personalización de si perfil.
	}

	\UCitem{Entradas}{\imprimeUC{entrada}}

	\UCitems{Origen}{%
        \Titem Mouse
	}

	\UCitem{Salidas}{
        \imprimeUC{salida}}

	\UCitem{Destino}{%
		\refElem{IU-P01}
	}

	\UCitems{Precondiciones}{%
        \Titem El usuario debió de haber ejecutado el CU {\it Iniciar sesión} que es propio de moodle.
	}

	\UCitems{Postcondiciones}{%

	}

	\UCitem{Reglas de negocio}{\imprimeUC{regla}}

	\UCitems{Errores}{%
	}

 \UCsection[design]{Datos de Diseño}

	\UCitems[design]{Casos de Prueba}{%
        \Titem \refElem{CPC-P01-1}
        \Titem \refElem{CPC-P01-2}
        \Titem \refElem{CPC-P01-3}
	}

 \UCsection[admin]{Datos de Administración de Requerimiento}

	\UCitem[admin]{Observaciones}{}

\end{UseCase}

\subsubsection{Trayectorias del caso de uso}

\begin{UCtrayectoria}%
%
    \Actor Selecciona dando clic a la opción \textbf{Perfil gamificado} de la pantalla \refElem{IU-M09}.
    \Sistema Corrobora que el actor haya iniciado sesión. \refTray{A}
    \Sistema Corrobora que el actor esté registrado en la entidad \refElem{xp-user}.  \refTray{B}
    \Sistema Redirige a la pantalla \refElem{IU-P01}.
    \label{CU-P01-cargar-informacion}
    \Sistema Carga el nombre del actor (\salida{Nombre de usuario}) y lo muestra en pantalla.
    \Sistema Carga la configuración actual del actor,
    cargando los objetos  (\salida{tienda-gmdl-objeto})
    desbloqueados cuyo \refElem{tienda-gmdl-objeto-desbloqueado.elegido} sea igual a 1  y los muestra en pantalla.
    \Sistema Comprueba que el módulo financiero esté desactivado. \refTray{C}
    \Sistema Carga todos los objetos en \refElem{tienda-gmdl-objeto} y los muestra en pantalla.


\end{UCtrayectoria}

\begin{UCtrayectoriaA}[Fin del caso de uso]%
  {A}{El actor aun no ha iniciado sesión}

  \Sistema Cierra la pantalla \refElem{IU-P01} y redirige a la pantalla \refElem{IU-M00b}.

\end{UCtrayectoriaA}

\begin{UCtrayectoriaA}%
{B}{El actor no está registrado en \refElem{xp-user}}

    \Sistema Registra al actor en \refElem{xp-user}.
    \item Se regresa al paso \ref{CU-P01-cargar-informacion}.

\end{UCtrayectoriaA}


\begin{UCtrayectoriaA}[Fin del caso de uso]%
{C}{El módulo financiero está activado}
    \Sistema Carga todos los objetos en \refElem{tienda-gmdl-objeto} y muestra la opción \textbf{'Comprar'} (utilizando el ícono \IUMonedas{})
     en aquellos objetos que el actor
    aun no tiene en \refElem{tienda-gmdl-objeto-desbloqueado}  y los muestra en pantalla.
    \Sistema Carga (\salida{xp-user.monedas-plata}) y lo muestra en pantalla.

\end{UCtrayectoriaA}




\UCExtensionPoint{Modificar perfil gamificado}{%

    El actor desea modificar qué objetos mostrar en su perfil gamificado.
%
    }{Al final de la trayectoria principal del caso de uso.
%
    }{\refElem{CU-P02}}

\UCExtensionPoint{Comprar objeto}{%

    El actor desea agregar un nuevo objeto a su colección..
  %
      }{Al final de la trayectoria alternativa C
  %
      }{\refElem{CU-F03}}
   % Ver perfil gamificado

% \ucstEnEdicion     Al terminar una revisión/aprobación con observaciones
%                    y al inicio del CU.
%
% \ucstEnRevision    Al terminar la edición del CU (version += 0.1).
% \ucstEnAprobacion  Al pasar la revision sin observaciones.
% \ucstAprobado      Al ser aprobado por el usuario (version += 1.0)

\begin{UseCase}[%
Autor/David Flores Casanova,%
Version/0.1,%
Estado/\ucstEnRevision]%
%
{CU-P02}{Modificar perfil gamificado}{%
%
    Permite al usuario (Ya sea un \refElem{aProfesor}, un \refElem{aAdministrador} o un \refElem{aEstudiante})
 de moodle seleccionar qué objetos quiere que se muestren en su perfil.
 Este caso de uso es una extensión del caso de uso {\it \refElem{CU-P01}}.}

	\UCitem[control]{Revisor}{ Sin asignar }
	\UCitem[control]{Último cambio}{ 17/NOV/19 }

 \UCsection{Atributos}

    \UCitem{Actor(es)}{%
        \refElem{aProfesor},
        \refElem{aAdministrador},
        \refElem{aEstudiante}
    }

	\UCitems{Propósito}{%
        \Titem El usuario quiere modificar que objetos se mostrarán cuando se visualice su perfil gamificado en las actividades.
	}

	\UCitem{Entradas}{\imprimeUC{entrada}}

	\UCitems{Origen}{%
        \Titem Mouse
	}

	\UCitem{Salidas}{
        \imprimeUC{salida}
        \Titem ''¡Cambios guardados!''%\refElem{MSG-P01}
        }

	\UCitem{Destino}{%
		\refElem{IU-P01}
	}

	\UCitems{Precondiciones}{%
        \Titem El usuario debió de haber ejecutado el \refElem{CU-P01}.
        \Titem El usuario debe tener al menos un objeto desbloqueado.
        \Titem Los objetos seleccionados deben estar desbloqueado para el usuario.
	}

	\UCitems{Postcondiciones}{%
        \Titem Al mostrar el perfil gamificado del usuario se verán las nuevas opciones que seleccionó.
	}

	\UCitem{Reglas de negocio}{\imprimeUC{regla}}

	\UCitems{Errores}{%
	}

 \UCsection[design]{Datos de Diseño}

	\UCitems[design]{Casos de Prueba}{%
        \Titem \refElem{CPC-P02-1}
        \Titem \refElem{CPI-P02-2}
        \Titem \refElem{CPI-P02-3}
	}

 \UCsection[admin]{Datos de Administración de Requerimiento}

	\UCitem[admin]{Observaciones}{}

\end{UseCase}

\subsubsection{Trayectorias del caso de uso}

\begin{UCtrayectoria}%
%
    \Actor Da clic en algún objeto en la pantalla \refElem{IU-P01}.
    \label{CU-P03-eligiendo-objeto}
    \Sistema Carga la previsualización siguiendo la regla de negocio \regla{BR-P01}.
    \Sistema Comprueba que el actor tenga adquirido el objeto. \refTray{A}
    \Sistema Habilita la opción \textbf{Guardar} de la pantalla \refElem{IU-P01}.
    \Actor Ve el resultado en la previsualización. \refTray{B}
    \Actor Presiona el botón \textbf{Guardar}.
    \Sistema Cambia los objetos seleccionados usando el atributo \refElem{tienda-gmdl-objeto-desbloqueado.elegido}.
    \Sistema Muestra el mensaje ''¡Cambios guardados!''%\refElem{MSG-P01}
\end{UCtrayectoria}

\begin{UCtrayectoriaA}%
  {A}{El actor no tiene el objeto desbloqueado }
    \Sistema Inhabilita la opción \textbf{Guardar} de la pantalla \refElem{IU-P01}.
    \item Se regresa al paso \ref{CU-P03-eligiendo-objeto}.

\end{UCtrayectoriaA}


\begin{UCtrayectoriaA}%
  {B}{El actor desea deshacer los cambios }
    \Actor  Presiona el botón \textbf{Deshacer}.
    \Sistema Carga la configuración actual del actor,
        cargando los objetos  (\salida{tienda-gmdl-objeto}) desbloqueados cuyo
        \refElem{tienda-gmdl-objeto-desbloqueado.elegido} sea igual a 1  y lo muestra en pantalla.
    \Sistema Habilita la opción \textbf{Guardar} de la pantalla \refElem{IU-P01}.
    \item Se regresa al paso \ref{CU-P03-eligiendo-objeto}.

\end{UCtrayectoriaA}
   % Modificar perfil gamificado

% \ucstEnEdicion     Al terminar una revisión/aprobación con observaciones
%                    y al inicio del CU.
%
% \ucstEnRevision    Al terminar la edición del CU (version += 0.1).
% \ucstEnAprobacion  Al pasar la revision sin observaciones.
% \ucstAprobado      Al ser aprobado por el usuario (version += 1.0)

\begin{UseCase}[%
Autor/David Flores Casanova,%
Version/0.1,%
Estado/\ucstEnRevision]%
%
{CU-P03}{Eliminar datos del complemento}{%
%
    Si ya no se quiere contar con el complemento de la tienda, se puede desinstalar y se borrarán los datos guardados en la base de datos.
    Este caso de uso es una extensión del caso de uso {\it Desinstalar un complemento} que es propio de moodle.}

	\UCitem[control]{Revisor}{ Sin asignar }
	\UCitem[control]{Último cambio}{ 17/NOV/19 }

 \UCsection{Atributos}

    \UCitem{Actor(es)}{%
        \refElem{aAdministrador}.
    }

	\UCitems{Propósito}{%
        \Titem El actor ya no quiere utilizar el complemento de tienda.
	}

	\UCitem{Entradas}{\imprimeUC{entrada}}

	\UCitems{Origen}{%
        \Titem Mouse
	}

	\UCitem{Salidas}{
        \imprimeUC{salida}}

	\UCitem{Destino}{%
		\refElem{IU-P01}
	}

	\UCitems{Precondiciones}{%
        \Titem se debió de haber ejecutado el CU 'Desinstalar complemento'.
	}

	\UCitems{Postcondiciones}{%
        \Titem Los datos guardados relacionados a qué usuario debloqueó qué objeto estarán eliminados.
        \Titem La opción Perfil gamificado ahora arrojará un error de pagina no encontrada.
	}

	\UCitem{Reglas de negocio}{\imprimeUC{regla}}

	\UCitems{Errores}{%
	}

 \UCsection[design]{Datos de Diseño}

	\UCitems[design]{Casos de Prueba}{%
	}

 \UCsection[admin]{Datos de Administración de Requerimiento}

	\UCitem[admin]{Observaciones}{La postcondición del error 'paǵina no encontrada' se debe a que el caso de uso que agrega la opción 'Perfil gamificado',
    es soportado únicamente por moodle.}

\end{UseCase}

\subsubsection{Trayectorias del caso de uso}

\begin{UCtrayectoria}%
%

    \Sistema Elimina toda la información de la entidad \refElem{tienda-gmdl-objeto-desbloqueado}.

\end{UCtrayectoria}
   % Eliminar datos del complemento

% \ucstEnEdicion     Al terminar una revisión/aprobación con observaciones
%                    y al inicio del CU.
%
% \ucstEnRevision    Al terminar la edición del CU (version += 0.1).
% \ucstEnAprobacion  Al pasar la revision sin observaciones.
% \ucstAprobado      Al ser aprobado por el usuario (version += 1.0)

\begin{UseCase}[%
Autor/David Flores Casanova,%
Version/0.1,%
Estado/\ucstEnRevision]%
%
{CU-P04}{Activar complemento}{%
%
    Para poder empezar a utilizar el plugin, se necesita que el actor\refElem{aAdministrador} haga unas modificaciones
    en la configuración de moodle. Esto ya es soportado por moodle, solo se indica qué es lo que debe de ingresar
    el actor.  Este caso de uso es una extensión del caso de uso {\it Acceder a la administración del sistio} propio de moodle.}

	\UCitem[control]{Revisor}{ Sin asignar }
	\UCitem[control]{Último cambio}{ 17/NOV/19 }

 \UCsection{Atributos}

    \UCitem{Actor(es)}{%
        \refElem{aAdministrador}.
    }

	\UCitems{Propósito}{%
        \Titem El actor quiere utilizar las funciones que le brinda el compelemento de 'tienda'.
	}

	\UCitem{Entradas}{\imprimeUC{entrada}}

	\UCitems{Origen}{%
        \Titem Mouse
        \Titem Teclado
	}

	\UCitem{Salidas}{
        \imprimeUC{salida}}

	\UCitem{Destino}{%
		\refElem{IU-P01}
	}

	\UCitems{Precondiciones}{%
        \Titem se debió de haber ejecutado el CU 'Acceder a la administración del sistio'.
	}

	\UCitems{Postcondiciones}{%
        \Titem La cabezera de moodle contará con la opción de 'Perfil gamificado'.
        \Titem El submenú del usuario de moodle contará con la opción de 'Perfil gamificado'.
	}

	\UCitem{Reglas de negocio}{\imprimeUC{regla}}

	\UCitems{Errores}{%
	}

 \UCsection[design]{Datos de Diseño}

	\UCitems[design]{Casos de Prueba}{%
	}

 \UCsection[admin]{Datos de Administración de Requerimiento}

	\UCitem[admin]{Observaciones}{}

\end{UseCase}

\subsubsection{Trayectorias del caso de uso}

\begin{UCtrayectoria}%
%
    \Actor Selecciona la opción \textbf{Apariencia}  y presiona la opción -> 'Ajustes de temas'.
    \Sistema Redirige a la pantalla \refElem{IU-M10}.
    \Actor Ingresa el texto \textbf{Perfil gamificado|/local/gmtienda/perfilgamificado.php} en la caja de texto \textbf{Ítems del menú personalizado}.
    \Actor Ingresa el texto \textbf{Perfil gamificado|/local/gmtienda/perfilgamificado.php} en la caja de texto \textbf{Ítems del menú de usuario}.
    \Actor Presiona el botón \textbf{Guardar cambios}.
    \Sistema Guarda la configuración.
    \Sistema Recarga la pestaña  \refElem{IU-M10}.
    
\end{UCtrayectoria}
   % Activar complemento
\input{modulos/person/CU/CU-P05}   % Desactivar complemento

% =========================================================
\clearpage
\subsection{Diseño}

\subsubsection{Interfaces del módulo de competencia}

    %
\subsubsection{IU-M00b Ingreso a moodle}

    Esta pantalla es de moodle. Esta pantalla brinda acceso al sistema.

    \IUfig{1}{modulos/moodle/IU/Login}{IU-M00b}{Ingreso a moodle}

\subsubsection{Elementos Relevantes}

    \begin{itemize}
    \item {\bf Lorem ipsum}
        ...
    \end{itemize}

\subsubsection{Acciones relevantes}

    \begin{itemize}
    \item {\bf Lorem ipsum}
        ...
    \end{itemize}

\clearpage

    
\subsubsection{IU-M09 Cabezera de moodle}

    \IUfig{1}{modulos/moodle/IU/cabezera_moodle}{IU-M09}{Cabezera de moodle}

\subsubsection{Elementos Relevantes}

    \begin{itemize}
        \item {\bf Nombre de la plataforma}
            En este caso mostrado como 'Gamedle', es el nombre que recibe la plataforma.
        \item {\bf Botón menú lateral}
            Este botón permite mostrar o ocultar el menú lateral de moodle.
        \item {\bf Idioma preferente}
            Esta opción permite cambiar el idioma con el cual el usuario configura cómo se muestran los textos de moodle en las páginas.
        \item {\bf Notificaciones}
            Esta opción permite ver qué notificaciones aún no se han atendido y ver el historial de las notificaciones.
        \item {\bf Mensajes}
            Esta opción permite ver las conversaciones que se tienen con otros usuarios.
        \item {\bf Perfil moodle}
            Esta opción permite ver el usuario de moodle.
    \end{itemize}


\clearpage

    
\subsubsection{IU-M09a Cabezera de moodle con el perfil gamificado}

    Esta pantalla se muestra en lugar de \refElem{IU-M09}, una vez se haya ejecutado el \refElem{CU-P04}.

    \IUfig{1}{modulos/moodle/IU/cabezera_moodle_gmdl}{IU-M09a}{Cabezera de moodle con perfil gamificado}

\subsubsection{Elementos Relevantes}

    \begin{itemize}
        \item {\bf Nombre de la plataforma}
            En este caso mostrado como 'Gamedle', es el nombre que recibe la plataforma.
        \item {\bf Botón menú lateral}
            Este botón permite mostrar o ocultar el menú lateral de moodle.
        \item {\bf Perfil gamificado}
            Este botón permite acceder ala pantalla \refElem{IU-P01}.
        \item {\bf Idioma preferente}
            Esta opción permite cambiar el idioma con el cual el usuario configura cómo se muestran los textos de moodle en las páginas.
        \item {\bf Notificaciones}
            Esta opción permite ver qué notificaciones aún no se han atendido y ver el historial de las notificaciones.
        \item {\bf Mensajes}
            Esta opción permite ver las conversaciones que se tienen con otros usuarios.
        \item {\bf Perfil moodle}
            Esta opción permite ver el usuario de moodle.
    \end{itemize}


\clearpage

    


\subsubsection{IU-M10 Ajustes de tema - Sección submenús}


    \IUfig{1}{modulos/moodle/IU/elementos_importantes_configuracion}{IU-M10}{Ajustes de tema - sección submenús}

\subsubsection{Elementos Relevantes}

    \begin{itemize}
        \item {\bf Ítems del menú personalizado}
            Campo de texto que establece la cantidad de opciones que están alojados en la cabecera de moodle.
        \item {\bf Ítems del menú del usuario}
            Campo de texto que establece la cantidad de opciones que están alojados en el submenú del usuario de moodle.
    \end{itemize}


\clearpage

    
\subsubsection{IU-P01: Perfil gamificado}

 Pantalla que muestra el estado actual del perfil gamificado del usuario.

    \IUfig{1}{modulos/person/IU/perfil_gamificado}{IU-P01}{%
        Pantalla del perfil gamificado sin módulo financiero}

    %\IUfig{1}{modulos/finan/IU/perfil_gamificado}{IU-F01}{%
    %   Pantalla del perfil gamificado con módulo financiero}

\subsubsection{Elementos Relevantes}

    \begin{itemize}
        \item {\bf Previzualización de configuración}
            En esta sección se muestra cómo se vería el perfil, utilizando los objetos seleccionados por el usuario.
        \item {\bf Monedas usuario}
            En esta sección se muestra la cantida de monedas que tiene a su disposición el usuario.
        \item {\bf Objetos}
            Los objetos a disposición del usuario a ser comprados.
    \end{itemize}

\subsubsection{Acciones relevantes}

    \begin{itemize}
        \item {\bf Deshacer}
            Esta opción borra las configuraciones cambiadas por el usuaario y vuelve a las ya precargadas para la pantalla.
        \item {\bf Guardar}
            Esta opción guarda la configuración que se tiene en la previsualización, para ser  mostrada en las otras activiedades gamificadas.
        \item {\bf Comprar}
            Botón de cada objeto indicado por el ícono \IUMonedas{}, permite comparar el objeto correspondiente.
    \end{itemize}

\clearpage





\subsubsection{Diseño de complementos}



A continuación se presenta cómo los submódulos de competencia
se implementan en moodle.\\


\noindent Resumiendo el módulo de competencia tiene 2 actividades establecidas, llamadas;
competencia uno contra uno y competencia uno contra sistema.
Ambas actividades deben aparecer dentro de la lista de actividades de moodle. Para ello
moodle cuenta con un tipo de complemento que se denomina \textbf{'mod'}, este tipo de complemento al ser instalado
en una plataforma de moodle, crea una nueva opción a la lista de actividades.\\

\noindent Tomando en consideración lo anterior y que existe el complemento gamedlemaster, se presenta en la figura \ref{fig:diseno-comp-comp}
los complementos contemplados y las dependencias entre los mismos.


    \addfigure{1}{modulos/comp/diagrams/diseno_complementos}{fig:diseno-comp-comp}{Implementación del modulo de competencia}


Cada complemento en la figura \ref{fig:diseno-comp-comp} está representado con una cadena que sigue el formato 'tipo\_de\_complemento:nombre\_de\_complemento'. Los tipos de complemento son;
\begin{itemize}
    \item \textbf{mod} - Este complemento permite crear una actividad que aparece en la lista de actividades a agregar a un curso.
    \item \textbf{local} -  Este complemento puede ser usado para múltiples propósitos relacionados con la gestión de la información.
    \item \textbf{block} - Este complemento permite desplegar una sección en la mayoría de las páginas de moodle, la cuál puede representar código html.
\end{itemize}

La función de cada uno de los complementos presentados en la figura \ref{fig:diseno-comp-comp} son:


\begin{itemize}
    \item \textbf{gamedlemaster} Definir la base de datos y los eventos a manejar.
    \item \textbf{gmcompcpu} Definir la competencia uno contra sistema.
    \item \textbf{gmcompvs} Definir la competencia uno contra uno.
    \item \textbf{gmcs} Entregar las monedas por ganar cada una de las competencias anteriores.
\end{itemize}

El complemento de tipo  \textbf{'mod'} tiene un requerimiento en su nombre, el cual es; 'El nombre del complemento a instalar debe ser igual a un nombre
de una de las tablas en la base de datos'. Debido a esto y que moodle no soporta nombres de complementos que contengan guiones bajos, el
nombre de la tabla ya no puede llevarlos.\\



\subsection{Pruebas}


    
\subsection{Análisis}

 Este apartado contiene el análisis requerido para la elaboración de módulo financiero,
 contiene la especificación del alcance de este módulo, la descripción de las funcionalidades
 a desarrollar, la reglas de negocio que rigen el comportamiento del módulo, y por último la
 especificación de los casos de uso a los que brinda soporte.

%\subsubsection{Submódulo de competencia 1 contra 1}
%\subsubsection{Funcionalidades}

\subsubsection{Reglas de negocio} %========================================================

 En esta sección se especifican todas las reglas de negocio relevantes para el módulo de
 experiencia. Las reglas de negocio que establece moodle son diferenciadas por tener la letra {\it M}
 antecediendo al número consecutivo en su identificador.

    %
\subsection{Entidades de moodle}

Debido a que moodle cuenta con más de 400 entidades en su versión 3.5, se opta
por mostrar 2 subconjuntos que muestren las entidades que se utilizan para el proyecto.\\

\noindent El primer subconjunto es aquel que explica la forma en que moodle implementa los cursos,
secciones de curso, actividades, usuarios y roles (el cual se presenta en la figura \ref{fig:BD-ER-M1}),
mientras que el segundo subconjunto muestra como moodle maneja toda la
estructura de las preguntas creadas por el profesor y respondidas por el estudiante
(el cual se presenta en la figura \ref{fig:BD-ER-M2}).

\noindent El objetivo de ambos esquemas (\ref{fig:BD-ER-M1} y \ref{fig:BD-ER-M2}) es expresar la idea general que abarcan ambos subconjuntos.

\clearpage
\addfigure{0.7}{analisis/diagrams/db_module_structure}{fig:BD-ER-M1}{Esquema de la base de datos de moodle 'Cursos'}


\noindent Utilizando la figura \ref{fig:BD-ER-M1}, se obtuvieron las siguientes reglas y características que tiene moodle respecto a los usuarios en un curso y a la estructura de los cursos.
\begin{enumerate}
    \item Un usuario -{\it mdl\_user}- tiene un rol -{\it mdl\_role}- en un cierto contexto -{\it mdl\_context}-, cuyo  '{\it context\_level}' sea igual a cincuenta(50).
    \item Si el contexto '{\it context\_level}' es de 50, el atributo '{\it instance\_id}' hace referencia al atributo '{\it id}' de un curso -{\it mdl\_course}-.
    \item El curso -{\it mdl\_course}- tiene varias secciones -{\it mdl\_course\_sections}-.
    \item Cada seccion -{\it mdl\_course\_sections}- tiene varias actividades -{\it mdl\_course\_modules}- que pertenecen a un tipo de actividad -{\it mdl\_modules}-.
    \item Por cada registro en tipo de actividad -{\it mdl\_modules}-, se tiene una entidad que lleva el mismo nombre.
    \item El atributo '{\it instance\_id}' de una actividad  -{\it mdl\_course\_modules}- apunta a diferentes entidades. La entidad a la que apunta depende del nombre del tipo de actividad -{\it mdl\_modules}-.
    \item Un usuario -{\it mdl\_user}- se inscribió -{\it mdl\_user\_enrolments}- a un curso -{\it mdl\_course}-, por medio de un formato soportado de inscripción -{\it mdl\_enrol}-.
\end{enumerate}

\clearpage

 \addfigure{0.7}{analisis/diagrams/db_module_questions}{fig:BD-ER-M2}{Esquema de la base de datos de moodle 'Preguntas' }



\noindent Utilizando la figura \ref{fig:BD-ER-M2}, se obtuvieron las siguientes reglas y características que tiene moodle respecto a las preguntas.
\begin{enumerate}
    \item Las preguntas -{\it mdl\_question}- tienen versiones -{\it mdl\_question\_attempts}-.
    \item Una pregunta -{\it mdl\_question}- pertenece a un banco de preguntas -{\it mdl\_question\_categories}-.
    \item La versión de una pregunta -{\it mdl\_question\_attempts}- es contestada -{\it mdl\_question\_usages}- en un determinado contexto -{\it mdl\_context}-.
    \item Un usuario -{\it mdl\_user}- responde una versión de una pregunta -{\it mdl\_question\_attempt\_stepts}-.
    \item El responder una versión de una pregunta -{\it mdl\_question\_attempt\_stepts}- conlleva pasos\\ -{\it mdl\_question\_attempt\_stept\_data}-, los cuales son: cómo se muestra, si ya se terminó de responder y qué se respondió.
\end{enumerate}


 A continuación se presenta la especificación de las entidades del esquema de base
 de datos de moodle que son relevantes para el desarrollo de los módulos y submódulos
 de proyecto.

    \begin{cdtEntidad}{mdl-config-plugins}{Configuración de Plugin}{%
    Es una tabla del núcleo de moodle que almacena todas las configuraciones globales
    relacionadas a los plugins instalados, al iniciar moodle las configuraciones de los
    plugins instalados y habilitados se cargan en memoria.}

	    \brAttr{id}{Id}{tInt}{%
	        Es el dígito que representa el identificador único para una configuración
            específica de un plugin.\par

            \it Restricciones:
            \refElem{tPrimaryKey},
            \refElem{tAutoIncrement}.
        }

        \brAttr{plugin}{Plugin}{tVarchar}{%
            Cadena de caracteres del nombre identificador del plugin al cual pertenece
            la configuración.\par

            \it Restricciones:
            \refElem{tRequired},
            \refElem{tRange} (0,100),
            \refElem{tUniqueKey}
        }

        \brAttr{name}{Nombre}{tVarchar}{%
            Cadena de caracteres que representa el nombre de la configuración de un
            plugin en específico.\par

            \it Restricciones:
            \refElem{tUniqueKey},
            \refElem{tRange} (0,100),
            \refElem{tRequired}
        }

        \brAttr{value}{Valor}{tVarchar}{%
            Cadena que almacena el valor de una configuración perteneciente a alguno
            de los plugins instalados.\par

            \it Restricciones:
            \refElem{tRange} (0,4294967295),
            \refElem{tRequired}
        }
    \end{cdtEntidad}\schemeName{config\_plugins}

    \begin{cdtEntidad}{mdl-user}{Usuario de moodle}{%
    Es una tabla del núcleo de moodle que contiene toda la información que se
    almacena de los usuarios en la plataforma, independientemente del rol que
    estos contenga, esta relación contiene más de 53 atributos, sin embargo solo
    se detallan aquellos relevantes.}

	    \brAttr{id}{Id}{tInt}{%
	        Es el dígito que representa el identificador único para cada uno
            de los usuarios en moodle.\par

            \it Restricciones:
            \refElem{tPrimaryKey},
            \refElem{tAutoIncrement}.
        }
	    \brAttr{username}{nombre de usuario}{tVarchar}{%
	        .\par

            \it Restricciones:
            \refElem{tRequired},
            \refElem{tLength} 0-100
        }
	    \brAttr{password}{contraseña}{tVarchar}{%
	        .\par

            \it Restricciones:
            \refElem{tRequired},
            \refElem{tLength} 0-255.
        }
	    \brAttr{firstname}{nombre}{tVarchar}{%
	        .\par

            \it Restricciones:
            \refElem{tRequired},
            \refElem{tLength} 0-100
        }
	    \brAttr{lastname}{apellido}{tVarchar}{%
	        .\par

            \it Restricciones:
            \refElem{tRequired},
            \refElem{tLength} 0-100
        }
	    \brAttr{email}{correo}{tVarchar}{%
	        .\par

            \it Restricciones:
            \refElem{tRequired},
            \refElem{tLength} 0-100
        }
	    \brAttr{lastaccess}{último registro}{tInt}{%
	        .\par

            \it Restricciones:
            \refElem{tRequired},
            \refElem{tLength} 10
        }
	    \brAttr{city}{ciudad}{tVarchar}{%
	        .\par

            \it Restricciones:
            \refElem{tRequired},
            \refElem{tLength} 0-120
        }
	    \brAttr{country}{pais}{tVarchar}{%
	        .\par

            \it Restricciones:
            \refElem{tRequired},
            \refElem{tLength} 2
        }

    \end{cdtEntidad}\schemeName{user}

    \begin{cdtEntidad}{mdl-course}{Curso de moodle}{%
    Es una tabla del núcleo de moodle que contiene la información principal de cada
    curso registrado en moodle. Esta entidad contiene 31 atributos, a continuación se
    detallan los atributos relevantes para la especificación de este proyecto.}

	    \brAttr{id}{Id}{tInt}{%
	        Es el dígito que representa al identificador único para cada uno
            de los cursos en moodle.\par

            \it Restricciones:
            \refElem{tPrimaryKey},
            \refElem{tAutoIncrement}.
        }

	    \brAttr{format}{formato}{tVarchar}{%
	        Es el dígito que representa al identificador único para cada uno
            de los cursos en moodle.\par

            \it Restricciones:
            \refElem{tRequired}.
            \refElem{tDefault} topics,
            \refElem{tLength} 0-21.
        }

	    \brAttr{fullname}{nombre completo}{tVarchar}{%
	        Es el nombre completo que se le asigna al curso.\par

            \it Restricciones:
            \refElem{tRequired}.
            \refElem{tLength} 0-21.
        }

	    \brAttr{shortname}{nombre corto}{tVarchar}{%
            Es el nombre corto que se le asigna al curso.\par

            \it Restricciones:
            \refElem{tRequired}.
            \refElem{tLength} 0-21.
        }

    \end{cdtEntidad}\schemeName{course}

    \begin{cdtEntidad}{mdl-course-section}{Sección del curso de moodle}{%
    }
	    \brAttr{id}{Id}{tInt}{%
	        Es el dígito que representa al identificador único para cada sección
            de los cursos en moodle.\par

            \it Restricciones:
            \refElem{tPrimaryKey},
            \refElem{tAutoIncrement}.
        }

        \brAttr{name}{nombre}{tVarchar}{%
	        Es el dígito nombre que permite identificar a una sección dentro de un curso
            en moodle.\par

            \it Restricciones: ...
        }
    \end{cdtEntidad}\schemeName{course\_sections}

    \begin{cdtEntidad}{mdl-course-format-options}{Opciones del formato del curso}{%
    }
	    \brAttr{id}{Id}{tInt}{%
	        Es el dígito que representa al identificador único para cada uno
            de los cursos en moodle.\par

            \it Restricciones:
            \refElem{tPrimaryKey},
            \refElem{tAutoIncrement}.
        }

	    \brAttr{courseid}{Id}{tInt}{%
	        Es el dígito que representa al identificador único para cada uno
            de los cursos en moodle.\par

            \it Restricciones:
            \refElem{tForeignKey},
            \refElem{tRequired}
        }

	    \brAttr{format}{formato}{tVarchar}{%
	        Es el dígito que representa al identificador único para cada uno
            de los cursos en moodle.\par

            \it Restricciones:
            \refElem{tRequired}.
            \refElem{tDefault} topics,
            \refElem{tLength} 0-21.
        }

	    \brAttr{name}{opcion}{tVarchar}{%
	        Es el dígito que representa al identificador único para cada uno
            de los cursos en moodle.\par

            \it Restricciones:
            \refElem{tPrimaryKey},
            \refElem{tLength} 0-100
        }

	    \brAttr{value}{valor}{tVarchar}{%
	        Es el dígito que representa al identificador único para cada uno
            de los cursos en moodle.\par

            \it Restricciones:
            \refElem{tRequired}
        }

    \end{cdtEntidad}\schemeName{course\_format\_options}

    \begin{cdtEntidad}{mdl-course-category}{Categoria de curso}{%
      .}
        \brAttr{id}{id}{tInt}{%
        .}
        \brAttr{name}{nombre}{tInt}{%
        .}

    \end{cdtEntidad}\schemeName{course\_category}

    \begin{cdtEntidad}{mdl-course-module}{Actividad del curso}{%
    .}
	    \brAttr{id}{Id}{tInt}{%
	        Es el dígito que representa al identificador único para cada uno
            de los cursos en moodle.\par

            \it Restricciones:
            \refElem{tPrimaryKey},
            \refElem{tAutoIncrement}.
        }
        \brAttr{course}{curso}{tInt}{%
        .}
        \brAttr{module}{actividad}{tInt}{%
        .}
        \brAttr{section}{sección}{tInt}{%
        .}
    \end{cdtEntidad}\schemeName{course\_module}

    \begin{cdtEntidad}{mdl-course-module-completion}{Actividad del curso para alumno}{%
    .}
	    \brAttr{id}{Id}{tInt}{%
	        Es el dígito que representa al identificador único para cada uno
            de los cursos en moodle.\par

            \it Restricciones:
            \refElem{tPrimaryKey},
            \refElem{tAutoIncrement}.
        }
        \brAttr{coursemoduleid}{actividad}{tInt}{%
        .}
        \brAttr{userid}{usuario}{tInt}{%
        .}
        \brAttr{completionstate}{completitud}{tBoolean}{%
        .}
    \end{cdtEntidad}\schemeName{course\_module}

    \begin{cdtEntidad}{Plugin}{Plugin}{%
    La forma en que moodle obtiene información acerca de los plugins es analizando
    los archivos internos de cada uno, a pesar de que los plugins no forman parte
    del esquema de base de datos, si forman parte del modelo de información que
    utiliza Moodle.}

	    \brAttr{componente}{Componente}{tVarchar}{%
	        Cadena compuesta por el tipo de plugin y el nombre del mismo, que
            representa a la clase principal del plugin que contiene los métodos
            principales del plugin.\par

            \it Restricciones: Ninguna
        }

	    \brAttr{pluginname}{Nombre}{tVarchar}{%
	        Es el nombre del plugin obtenido de los archivos de
            internacionalización presentes en el plugin, el valor de esta cadena
            depende del lenguaje seleccionado en moodle.\par

            \it Restricciones: Ninguna
        }

	    \brAttr{fullpath}{Ruta absoluta}{tPath}{%
	        La ruta absoluta de un plugin denota la ubicación del plugin en el
            sistema de archivos, esta ruta está compuesta por la ruta absoluta
            de la instalación de moodle, la carpeta correspondiente al tipo del
            plugin y el nombre del plugin.\par

            \it Restricciones: Formato ``/path/to/moodle/plugintype/pluginname''
        }

	    \brAttr{path}{Ruta relativa}{tPath}{%
	        La ruta relativa denota la ubicación del plugin dentro de la carpeta
            donde se encuentran los archivos de moodle, esta ruta está compuesta
            por la carpeta correspondiente al tipo del plugin y el nombre del
            plugin.\par

            \it Restricciones: Formato ``plugintype/pluginname''
        }

	    \brAttr{version}{Versión}{tVersion}{%
	        Numero entero de longitud de 10 dígitos que representa la versión del
            plugin.\par

            \it Restricciones: Ninguna adicional al tipo de dato
        }

	    \brAttr{moodle}{Versión de Moodle}{tVersion}{%
	        Número entero de longitud de 10 dígitos que representa la versión de
            moodle en la que se puede instalar el plugin.\par

            \it Restricciones: Ninguna adicional al tipo de dato
        }

        \brAttr{dependencies}{Dependencias}{tObject}{%
            Objeto que almacena un conjunto de claves con sus respectivos valores,
            donde cada clave representa el nombre del componente del plugin y el valor
            es la \refElem{Plugin.version} requerida del mismo.

            \it Restricciones: Ninguna
        }

        \brAttr{icon}{ícono}{tImage}{%
            Imagen para el ícono del plugin, debe estar contenida en el directorio
            {\it pix/} del plugin y tener como nombre {\it icon.png} o {\it icon.svg},
            moodle recomienda tener ambos archivos por si los navegadores no soportan
            algún tipo de archivo \cite{moodlePluginfiles}.\par

            \it Restricciones: El nombre debe ser icono con extensiones png o svg
        }

    \end{cdtEntidad}
 % Archivo de plugin
    %
\begin{BusinessRule}[%
Autor/Ricard Naranjo Polit,%
Version/0.1,%
Estado/revision]%
%
{BR-C01}{Restricciones del tiempo que se tiene para completar el desafío una vez iniciado en competencia uno contra uno}
 % El archivo de instalación debe ser un archivo ZIP, el cual debe contener
 % exactamente un directorio que coincida con el nombre del plugin.
     \BRitem[control]{Revisor}{Sin asignar.}

 \BRsection[control]{Atributos}

    \BRitem[admin]{Clase}{\bcCondition}%
    %\BRitem[admin]{Clase}{\bcIntegridad}%
    %\BRitem[admin]{Clase}{\bcAutorizacion}%
    %\BRitem[admin]{Clase}{\bcDerivacion}%

    \BRitem[admin]{Tipo}{\btTimer}%
    %\BRitem[admin]{Tipo}{\btTimer}%
    %\BRitem[admin]{Tipo}{\btExecutive}%

    \BRitem[admin]{Nivel}{\blControlling}
    %\BRitem[admin]{Nivel}{\blInfluencing}

    \BRitem{Descripción}{%
        Cuando un usuario desafía a un \refElem{aEstudiante} o acepta un desafío tiene un día para completarlo.
        Al terminar este tiempo se terminará el desafió y se le pondrá una puntuación de 0.
        % debido a que se el directorio donde se guardará será el directorio
        % para almacenar las imágenes del plugin.
    }

    \BRitem{Ejemplo positivo}{\hfill\par%
        \begin{itemize}
        \item El usuario desafía a un estudiante y termina el desafío en menos de un día.

        \item El usuario acepta un desafío y lo termina en menos de un día.
        \end{itemize}
    }

    \BRitem{Ejemplo negativo}{\hfill\par%
        \begin{itemize}
          \item El usuario desafía a un estudiante y no termina el desafío en menos de un día.

          \item El usuario acepta un desafío y no lo termina en menos de un día.
        \end{itemize}
    }%

 \end{BusinessRule}
 % Restricciones sobre de imagen del nivel.
    %
\begin{BusinessRule}[%
Autor/Ricard Naranjo Polit,%
Version/0.1,%
Estado/revision]%
%
{BR-C02}{Un usuario no puede desafiar a otro con el que tenga un desafío pendiente}
 % El archivo de instalación debe ser un archivo ZIP, el cual debe contener
 % exactamente un directorio que coincida con el nombre del plugin.
     \BRitem[control]{Revisor}{Sin asignar.}

 \BRsection[control]{Atributos}

    \BRitem[admin]{Clase}{\bcCondition}%
    %\BRitem[admin]{Clase}{\bcIntegridad}%
    %\BRitem[admin]{Clase}{\bcAutorizacion}%
    %\BRitem[admin]{Clase}{\bcDerivacion}%

    \BRitem[admin]{Tipo}{\btEnabler}%
    %\BRitem[admin]{Tipo}{\btTimer}%
    %\BRitem[admin]{Tipo}{\btExecutive}%

    \BRitem[admin]{Nivel}{\blControlling}
    %\BRitem[admin]{Nivel}{\blInfluencing}

    \BRitem{Descripción}{%
        Cuando un usuario desafía a un \refElem{aEstudiante} no lo podrá volver a desafiar hasta que el desafiante
        y desafiado terminen hayan completado la competencia.
        % debido a que se el directorio donde se guardará será el directorio
        % para almacenar las imágenes del plugin.
    }

    \BRitem{Ejemplo positivo}{\hfill\par%
        \begin{itemize}
        \item El usuario desafía a un estudiante, los dos terminan la competencia y el usuario vuelve a desafiar al mismo estudiante.

        \end{itemize}
    }

    \BRitem{Ejemplo negativo}{\hfill\par%
        \begin{itemize}
          \item El usuario desafía a un estudiante y no alguno de los dos no termina la competencia,
          el usuario no puede volver a desafiar al mismo estudiante.

        \end{itemize}
    }%

 \end{BusinessRule}
 % Permanencia del nivel de comperiencia.
    %
\begin{BusinessRule}[%
Autor/Daniel Isai Ortega Zúñiga,%
Version/0.1,%
Estado/revision]%
%
{BR-E03}{Tipos de Incremento}
    \BRitem[control]{Revisor}{Sin asignar.}

 \BRsection[control]{Atributos}
    
    \BRitem[admin]{Clase}{\bcCondition}%
    %\BRitem[admin]{Clase}{\bcIntegridad}%
    %\BRitem[admin]{Clase}{\bcAutorizacion}%
    %\BRitem[admin]{Clase}{\bcDerivacion}%
        
    \BRitem[admin]{Tipo}{\btEnabler}%
    %\BRitem[admin]{Tipo}{\btTimer}%
    %\BRitem[admin]{Tipo}{\btExecutive}%
        
    \BRitem[admin]{Nivel}{\blControlling}
    %\BRitem[admin]{Nivel}{\blInfluencing}
    
    \BRitem{Descripción}{%
    Cuando se modifiquen el \refElem{xp-scheme-settings} o la \refElem{levelXP} de las
    \refElem{xp-scheme-settings}
    }

    \BRitem{Ejemplo positivo}{\hfill\par%
        \begin{itemize}
        \item ...
        \end{itemize}
    }

    \BRitem{Ejemplo negativo}{\hfill\par%
        \begin{itemize}
        \item ...
        \end{itemize}
    }% 
    
 \end{BusinessRule}
 % Tipos de incremento
    %\begin{BusinessRule}[%
Autor/Daniel Isai Ortega Zúñiga,%
Version/0.1,%
Estado/revision]%
%
{BR-E04}{Calculo de experiencia del nivel con incremento porcentual}
    \BRitem[control]{Revisor}{Sin asignar.}

 \BRsection[control]{Atributos}
    
    \BRitem[admin]{Clase}{\bcCondition}%
    %\BRitem[admin]{Clase}{\bcIntegridad}%
    %\BRitem[admin]{Clase}{\bcAutorizacion}%
    %\BRitem[admin]{Clase}{\bcDerivacion}%
        
    \BRitem[admin]{Tipo}{\btEnabler}%
    %\BRitem[admin]{Tipo}{\btTimer}%
    %\BRitem[admin]{Tipo}{\btExecutive}%
        
    \BRitem[admin]{Nivel}{\blControlling}
    %\BRitem[admin]{Nivel}{\blInfluencing}
    
    \BRitem{Descripción}{%
        El calculo para obtener la experiencia del nivel $i$ uando el tipo de
        incremento es porcentual está dado por la siguiente fórmula: Sea {\it exp()}
        la función que optiene la experiencia de un nivel en específico, sea tambien
        $i$ el nivel del cual se calcula la experiencia, sea $inc$ el factor de
        incremento de nivel a nivel, y finalmente sea $round()$ una función de
        redondeo a números enteros, entonces:

            $$ exp(i) = round( exp(1) * (inc)^{(i-1)})$$
    }

%   \BRitem{Sentencia}{%
%       Si $fecha$ 
%   }%

    \BRitem{Ejemplo positivo}{\hfill\par%
        \begin{itemize}
        \item La experiencia requerida para superar el nivel 1 es de 2000 puntos y el
              factor de incremento entre los niveles es 1.1, entonces la experiencia
              requerida para pasar el nivel 5 es de 2928 puntos.
        \end{itemize}
    }

    \BRitem{Ejemplo negativo}{\hfill\par%
        \begin{itemize}
        \item La experinecia requerida para superar el nivel 1 es de 2000 puntos y el
              factor de incremento entre los niveles es 1.1, entonces la experiencia
              requerida para pasar el nivel 5 es de 2300 puntos.
        \end{itemize}
    }% 
    
\end{BusinessRule}
 % Incremento porcentual
    %\begin{BusinessRule}[%
Autor/Daniel Isai Ortega Zúñiga,%
Version/0.1,%
Estado/revision]%
%
{BR-E05}{Cálculo de experiencia del nivel con incremento linea} % Cuando están iniciados
    \BRitem[control]{Revisor}{Sin asignar.}

 \BRsection[control]{Atributos}
    
    \BRitem[admin]{Clase}{\bcCondition}%
    %\BRitem[admin]{Clase}{\bcIntegridad}%
    %\BRitem[admin]{Clase}{\bcAutorizacion}%
    %\BRitem[admin]{Clase}{\bcDerivacion}%
        
    \BRitem[admin]{Tipo}{\btEnabler}%
    %\BRitem[admin]{Tipo}{\btTimer}%
    %\BRitem[admin]{Tipo}{\btExecutive}%
        
    \BRitem[admin]{Nivel}{\blControlling}
    %\BRitem[admin]{Nivel}{\blInfluencing}
    
    \BRitem{Descripción}{%
    }

%   \BRitem{Sentencia}{%
%       Si $fecha$ 
%   }%

    \BRitem{Ejemplo positivo}{\hfill\par%
        \begin{itemize}
        \item ...
        \end{itemize}
    }

    \BRitem{Ejemplo negativo}{\hfill\par%
        \begin{itemize}
        \item ...
        \end{itemize}
    }% 
    
\end{BusinessRule}
 % Incremento lineal
    %\begin{BusinessRule}[%
Autor/Daniel Isai Ortega Zúñiga,%
Version/0.1,%
Estado/revision]%
%
{BR-E06}{Eliminación de cursos gamificados} % Cuando están iniciados
    \BRitem[control]{Revisor}{Sin asignar.}

 \BRsection[control]{Atributos}

    \BRitem[admin]{Clase}{\bcCondition}%
    %\BRitem[admin]{Clase}{\bcIntegridad}%
    %\BRitem[admin]{Clase}{\bcAutorizacion}%
    %\BRitem[admin]{Clase}{\bcDerivacion}%

    \BRitem[admin]{Tipo}{\btEnabler}%
    %\BRitem[admin]{Tipo}{\btTimer}%
    %\BRitem[admin]{Tipo}{\btExecutive}%

    \BRitem[admin]{Nivel}{\blControlling}
    %\BRitem[admin]{Nivel}{\blInfluencing}

    \BRitem{Descripción}{%
        Debido a que el \refElem{mdl-course.format} por defecto para los cursos de
        moodle es el formato de tópicos/temas El formato de curso gamificado extiende
        las funcionalidades de este formato para facilitar la migración de un curso
        gamificado a uno no gamificado y viceversa. % TODO Pasar a analisis.
        La desinstalación del módulo de experiencia implica que los cursos con el
        \refElem{xp-course.format} gamificado (gamedle) se migren a cursos no
        gamificados, por compatibilidad en este migración se deben realizar las
        siguientes acciones de forma transaccional:

        \begin{itemize}
        \item El formato de los \refElem[cursos gamificados]{xp-course} debe
              cambiarse por el formato por defecto que tienen los cursos en moodle
              el cual es el de tópicos/temas.

        \item Se deben eliminar las \refElem[opciones del formato del curso]%
              {mdl-course-format-options} gamificado (gamedle).

        \item Se deben establecer los valores para las opciones del formato de
              tópicos/temas, que tienen por \refElem[nombre]%
              {mdl-course-format-options.name} secciones ocultas y aspecto del curso.
        \end{itemize}
    }

%   \BRitem{Sentencia}{%
%       Si $fecha$
%   }%

    \BRitem{Ejemplo positivo}{\hfill\par%
        \begin{itemize}
        \item Cuando se desinstala los plugins correspondientes al módulo de
              experiencia los cursos que estan vinculados con el formato gamificado
              son cambiados al formato de topicos/temas (formato por defecto de
              moodle), y también se establecen las opciones equivalentes al formato
              gamificado.
        \end{itemize}
    }

    \BRitem{Ejemplo negativo}{\hfill\par%
        \begin{itemize}
        \item Cuando se desinstala los plugins correspondientes al módulo de
              experiencia los cursos que estan vinculados con el formato gamificado
              no son cambiados al formato de topicos/temas ocasionando inconsistencia
              entre los cursos y los formatos.
        \end{itemize}
    }%

\end{BusinessRule}
 % Eliminación de cursos gamificados
    %\begin{BusinessRule}[%
Autor/Daniel Isai Ortega Zúñiga,%
Version/0.1,%
Estado/edicion]%
%
{BR-E07}{Valores iniciales de experiencia}

     \BRitem[control]{Revisada por}{Pendiente.}

 \BRsection[control]{Atributos}
    % Clases: \bcCondition, \bcIntegridad, \bcAutorization o \bcDerivation
    % Tipos: \btEnabler, \btTimer o \btExecutive
    % Niveles: \blControlling o \blInfluencing.

    \BRitem[admin]{Clase}{\bcIntegridad}%

    \BRitem[admin]{Tipo}{\btTimer}%

    \BRitem[admin]{Nivel}{\blControlling}

    \BRitem{Descripción}{%
        Cuando un \refElem{xp-user} es creado este debe de empezar a ganar puntos
        de experiencia a partir del \refElem{xp-user.level} uno, tenido cero puntos
        de experiencia en la \refElem{xp-user.levelxp} y \refElem{xp-user.xp}. Ningún
        usuario puede comenzar con valores distintos a los indicados anteriormente.
    }

%   \BRitem{Sentencia}{%
%       Si $fecha$ 
%   }%

    \BRitem{Ejemplo positivo}{\hfill\par%
        \begin{itemize}
        \item ...
        \end{itemize}
    }

    \BRitem{Ejemplo negativo}{\hfill\par%
        \begin{itemize}
        \item ...
        \end{itemize}
    }

 \end{BusinessRule}
 % Valores iniciales de comperiencia
    %\begin{BusinessRule}[%
Autor/Daniel Isai Ortega Zúñiga,%
Version/0.1,%
Estado/edicion]%
%
{BR-E08}{Valores iniciales de experiencia de un curso}

     \BRitem[control]{Revisada por}{Pendiente.}

 \BRsection[control]{Atributos}
    % Clases: \bcCondition, \bcIntegridad, \bcAutorization o \bcDerivation
    % Tipos: \btEnabler, \btTimer o \btExecutive
    % Niveles: \blControlling o \blInfluencing.

    \BRitem[admin]{Clase}{\bcIntegridad}%

    \BRitem[admin]{Tipo}{\btTimer}%

    \BRitem[admin]{Nivel}{\blControlling}

    \BRitem{Descripción}{%
        Cuando un \refElem{xp-course} es creado la \refElem[experiencia total del curso]%
        {xp-scheme-settings.courseXP} de ser dividida uniformemente entre las
        \refElem[secciones del curso gamificado]{xp-course-section}. Si la división del
        total de experiencia entre el número de secciones genera un residuo entonces este
        se deberá agregan a la última sección del curso.
    }

%   \BRitem{Sentencia}{%
%       Si $fecha$
%   }%

    \BRitem{Ejemplo positivo}{\hfill\par%
        \begin{itemize}
        \item ...
        \end{itemize}
    }

    \BRitem{Ejemplo negativo}{\hfill\par%
        \begin{itemize}
        \item ...
        \end{itemize}
    }

 \end{BusinessRule}
 % Valores iniciales de experiencia del curso

    % INPUT: Cursos Igualitarios.
    % INPUT: Otorgar experiencia
    % INPUT: Administración de experiencia en el curso

\clearpage
\subsubsection{Casos de uso} % ============================================================

 En este apartado se especifican todos los casos de usos contemplados para el financiero,
 para cada caso de uso se especifica su tabla de atributos la cual indica que casos
 de prueba deberán ejecutarse correctamente para corroborar la completitud del caso de uso.

\subsubsection*{Diagrama de casos de uso}

 En la figura \ref{financiero:usecases} se detalla el diagrama de casos de uso correspondiente al módulo
financiero. Los casos de uso de moodle (en color blanco) son modelados como casos de uso
 abstractos, mientras que los casos de uso del módulo de competencia son diferenciados por el
 color azul, en total el desarrollo de este módulo consiste en 3 casos de uso.

    \addfigure{0.6}{modulos/finan/diagrams/UseCases}{financiero:usecases}{%
        Diagrama de casos de uso del módulo financiero}

 \noindent
 Debido a que los plugins a desarrollar son elementos opcionales para Moodle, solo se puede
 acceder a los casos de uso del módulo financiero a través de puntos de extensión de los
 casos de uso de moodle. Por otra parte los casos de uso que serán documentados en esta sección
 serán los del módulo financiero, debido a que Moodle proporciona en su página oficial guías,
 instructivos y documentación de las funcionalidades que brinda.




\subsubsection{Mensajes}

    \begin{mensaje2}{MSG-F01}{Compra exitosa}{Operación exitosa}
        \item[Redacción:] ¡Objeto [NombreDelObjeto] comprado!
        \item[Parámetros:]
        \begin{Citemize}
            \item NombreDelObjeto: \refElem(tienda-gmdl-objeto.nombre).
        \end{Citemize}
       \item[Ejemplo:] ¡Objeto compañero comprado!
    \end{mensaje2}

    \begin{mensaje2}{MSG-F02}{No se tienen suficientes monedas}{Operación exitosa}
        \item[Redacción:] ¡Objeto [NombreDelObjeto] comprado!
        \item[Parámetros:]
        \begin{Citemize}
            \item NombreDelObjeto: \refElem(tienda-gmdl-objeto.nombre).
        \end{Citemize}
       \item[Ejemplo:] ¡Objeto compañero comprado!
    \end{mensaje2}

    \begin{mensaje2}{MSG-F03}{Objeto ya adquirido}{Mensaje de advertencia}
        \item[Redacción:] ¡Ya tenías desbloqueado el objeto [NombreDelObjeto]!
        \item[Parámetros:]
        \begin{Citemize}
            \item NombreDelObjeto: \refElem(tienda-gmdl-objeto.nombre).
        \end{Citemize}
       \item[Ejemplo:] ¡Objeto compañero comprado!
    \end{mensaje2}

    \begin{mensaje2}{MSG-F04}{Campo erróneo, monedas}{Mensaje de error}
        \item[Redacción:] Las monedas deben ser un número natural.
    \end{mensaje2}



    % MODULO FINANCIERO


% \ucstEnEdicion     Al terminar una revisión/aprobación con observaciones
%                    y al inicio del CU.
%
% \ucstEnRevision    Al terminar la edición del CU (version += 0.1).
% \ucstEnAprobacion  Al pasar la revision sin observaciones.
% \ucstAprobado      Al ser aprobado por el usuario (version += 1.0)

\begin{UseCase}[%
Autor/David Flores Casanova,%
Version/0.1,%
Estado/\ucstEnRevision]%
%
{CU-F01}{Modificar esquema financiero }{%
%
    Permite al \refElem{aAdministrador} modificar la cantidad de monedas que otorga cada evento, 
    así como, si desea que el esquema financiero esté activado o no.
    Este caso de uso es una extensión del caso de uso {\it Entrar a administración de complementos} que es propio de moodle }

	\UCitem[control]{Revisor}{ Sin asignar }
	\UCitem[control]{Último cambio}{ 17/NOV/19 }

 \UCsection{Atributos}

    \UCitem{Actor(es)}{%
        \refElem{aAdministrador}
    }

	\UCitems{Propósito}{%
        \Titem El actor desea cambiar la cantidad de monedas que da un evento o quiere activar  o desactivar el esquema financiero.
	}

	\UCitem{Entradas}{\imprimeUC{entrada}}

	\UCitems{Origen}{%
        \Titem Mouse
	}

	\UCitem{Salidas}{
        \imprimeUC{salida}}

	\UCitem{Destino}{%
		\refElem{IU-F01}
	}

	\UCitems{Precondiciones}{%
        \Titem El complemento del módulo financiero debe estar instalado.
	}

	\UCitems{Postcondiciones}{%
        \Titem Se guardará la nueva configuración del actor.
	}

	\UCitem{Reglas de negocio}{\imprimeUC{regla}}

	\UCitems{Errores}{%
        \Titem Alguno de los campos fueron ingresados de manera erronea.
	}

 \UCsection[design]{Datos de Diseño}

	\UCitems[design]{Casos de Prueba}{%
	}

 \UCsection[admin]{Datos de Administración de Requerimiento}

	\UCitem[admin]{Observaciones}{}

\end{UseCase}

\subsubsection{Trayectorias del caso de uso}

\begin{UCtrayectoria}%
%

    \Actor Selecciona dando clic a la opción \textbf{Gamedle: Módulo financiero} en la pantalla \refElem{IU-M11}. \refTray{A}
    \Sistema Redirige a la pantalla \refElem{IU-F01}.
    \Actor Modifica las opciones que desea. \refTray{B}
    \label{CU-F01-hacer-cambios}
    \Actor Presiona el botón \textbf{Guardar cambios}. 
    \label{CU-F01-guardar-cambios}
    \Sistema Valida que los valores ingresados sean correctos. \refTray{C}
    \Sistema Guarda la nueva configuración.

\end{UCtrayectoria}

\begin{UCtrayectoriaA}[Fin del caso de uso]%
  {A}{El complemento del  módulo financiero no se encuentra instalado}

  \Actor No encuentra la opción en la pantalla  \textbf{Gamedle: Módulo financiero}, porque el sistema no la generó.

\end{UCtrayectoriaA}

\begin{UCtrayectoriaA}%
{B}{El actor desea modificar la configuración financiera}
    \Actor Selecciona dando clic a la opción \textbf{Gamedle: Módulo financiero} en la pantalla \refElem{IU-F01}.
    \Sistema Redirige a la pantalla \refElem{IU-F02}.
    \Actor Modifica las opciones que desea. 
    \item Se regresa al paso \ref{CU-F01-guardar-cambios} de la trayectoria principal.

\end{UCtrayectoriaA}

\begin{UCtrayectoriaA}%
{B}{El actor desea modificar la configuración financiera}
    \Actor Selecciona dando clic a la opción \textbf{Gamedle: Módulo financiero} en la pantalla \refElem{IU-F01}.
    \Sistema Redirige a la pantalla \refElem{IU-F02}.
    \Actor Modifica las opciones que desea. 
    \item Se regresa al paso \ref{CU-F01-guardar-cambios} de la trayectoria principal.

\end{UCtrayectoriaA}


\begin{UCtrayectoriaA}%
{C}{Se han ingresaron datos erróneos}
    \Sistema Muestra el mensaje \refElem{MSG-F04} en cada campo con valores erróneos.. 
    \item Se regresa al paso \ref{CU-F01-hacer-cambios}.

\end{UCtrayectoriaA}   % Modificar esquema financiero

% \ucstEnEdicion     Al terminar una revisión/aprobación con observaciones
%                    y al inicio del CU.
%
% \ucstEnRevision    Al terminar la edición del CU (version += 0.1).
% \ucstEnAprobacion  Al pasar la revision sin observaciones.
% \ucstAprobado      Al ser aprobado por el usuario (version += 1.0)

\begin{UseCase}[%
Autor/David Flores Casanova,%
Version/0.1,%
Estado/\ucstEnRevision]%
%
{CU-F02}{Instalar esquema financiero }{%
%
    El \refElem{aAdministrador} instala el complemento como lo especifica moodle y al instalarlo moodle le permite
    configurar las opciones una vez instalado.
    Este caso de uso es una extensión del caso de uso {\it Instalar complemento.}}

	\UCitem[control]{Revisor}{ Sin asignar }
	\UCitem[control]{Último cambio}{ 17/NOV/19 }

 \UCsection{Atributos}

    \UCitem{Actor(es)}{%
        \refElem{aAdministrador}
    }

	\UCitems{Propósito}{%
        \Titem El actor desea utilizar las funciones que brinda el módulo financiero .
	}

	\UCitem{Entradas}{\imprimeUC{entrada}}

	\UCitems{Origen}{%
        \Titem Mouse
	}

	\UCitem{Salidas}{
        \imprimeUC{salida}}

	\UCitem{Destino}{%
		\refElem{IU-F01}
	}

	\UCitems{Precondiciones}{%
        \Titem El complemento del módulo financiero no debe de estar instalado.
	}

	\UCitems{Postcondiciones}{%
        \Titem El módulo financiero ahora estará funcionando en la plataforma del actor.
        \Titem Se guardará la configuración del actor.
	}

	\UCitem{Reglas de negocio}{\imprimeUC{regla}}

	\UCitems{Errores}{%
	}

 \UCsection[design]{Datos de Diseño}

	\UCitems[design]{Casos de Prueba}{%
	}

 \UCsection[admin]{Datos de Administración de Requerimiento}

	\UCitem[admin]{Observaciones}{}

\end{UseCase}

\subsubsection{Trayectorias del caso de uso}

\begin{UCtrayectoria}%
%

    \Actor Modifica si desea que el módulo financiero esté activado o no en la plataforma, usando pantalla \refElem{IU-F02}.
    \Actor Presiona el botón \textbf{Guardar cambios}.
    \label{CU-F01-guardar-cambios}
    \Sistema Redirige a la pantalla \refElem{IU-F01}.

\end{UCtrayectoria}
   % Instalar esquema financiero

% \ucstEnEdicion     Al terminar una revisión/aprobación con observaciones
%                    y al inicio del CU.
%
% \ucstEnRevision    Al terminar la edición del CU (version += 0.1).
% \ucstEnAprobacion  Al pasar la revision sin observaciones.
% \ucstAprobado      Al ser aprobado por el usuario (version += 1.0)

\begin{UseCase}[%
Autor/David Flores Casanova,%
Version/0.1,%
Estado/\ucstEnRevision]%
%
{CU-F03}{Comprar objeto}{%
%
Permite al usuario (Ya sea un \refElem{aProfesor}, un \refElem{aAdministrador} o un \refElem{aEstudiante})
 de moodle adquirir un objeto para la personalización de su perfil utilizando sus monedas disponibles.
 Este caso de uso es una extensión del caso de uso {\it \refElem{CU-P01}}.}

	\UCitem[control]{Revisor}{ Sin asignar }
	\UCitem[control]{Último cambio}{ 17/NOV/19 }

 \UCsection{Atributos}

    \UCitem{Actor(es)}{%
        \refElem{aProfesor},
        \refElem{aAdministrador},
        \refElem{aEstudiante}
    }

	\UCitems{Propósito}{%
        \Titem El usuario quiere saber el estado actual de su perfil, así como las monedas que tiene disponibles.
	}

	\UCitem{Entradas}{\imprimeUC{entrada}}

	\UCitems{Origen}{%
        \Titem Mouse
	}

	\UCitem{Salidas}{
        \imprimeUC{salida}
        \Titem ''¡Objeto [NombreDelObjeto] comprado!''%\refElem{MSG-F01}
        \Titem ''¡Oh no!, No tienes suficiente dinero para comprar el objeto [NombreDelObjeto]!''%\refElem{MSG-F02}
        \Titem ''¡Ya tenías desbloqueado el objeto [NombreDelObjeto]!''%\refElem{MSG-F03}

        }

	\UCitem{Destino}{%
		\refElem{IU-P01}
	}

	\UCitems{Precondiciones}{%
        \Titem El usuario debió de haber ejecutado el \refElem{CU-P01}.
        \Titem El usuario debe contener las monedas suficientes para comprar el objeto.
        \Titem El objeto que se quiere  comprar no debe estar ya adquirido.
	}

	\UCitems{Postcondiciones}{%
        \Titem El usuario tendrá desbloqueado el objeto para usarlo.
        \Titem El usuario se le restarán las \refElem{xp-user.monedas-plata} del objeto que adquirió.
	}

	\UCitem{Reglas de negocio}{\imprimeUC{regla}}

	\UCitems{Errores}{%
        \Titem El usuario no cuenta con las suficientes monedas para adquirir el objeto.
	}

 \UCsection[design]{Datos de Diseño}

	\UCitems[design]{Casos de Prueba}{%
        \Titem \refElem{CPC-F03-1}
        \Titem \refElem{CPI-F03-2}
        \Titem \refElem{CPI-F03-3}
	}

 \UCsection[admin]{Datos de Administración de Requerimiento}

	\UCitem[admin]{Observaciones}{}

\end{UseCase}

\subsubsection{Trayectorias del caso de uso}

\begin{UCtrayectoria}%
%

    \Actor Selecciona dando clic a la opción \textbf{Comprar} (Indicado por el ícono de las monedas \IUMonedas{}) del objeto que desea comprar en la pantalla \refElem{IU-P01}.
    \Sistema Comprueba que el actor cuenta con suficientes \refElem{xp-user.monedas-plata} para comprar el objeto. \refTray{A}
    \Sistema Comprueba que el actor no cuente con ese objeto todavía. \refTray{B}
    \Sistema Desbloquea el objeto seleccionado para el actor, guardándolo en \refElem{tienda-gmdl-objeto-desbloqueado}.
    \Sistema Le resta al actor las \refElem{xp-user.monedas-plata} dependiendo el \refElem{tienda-gmdl-rareza-objeto.costo-adquisicion} del objeto seleccionado.
    \Sistema Muestra el mensaje ''¡Objeto [NombreDelObjeto] comprado!''%\refElem{MSG-P01}

\end{UCtrayectoria}

\begin{UCtrayectoriaA}[Fin del caso de uso]%
  {A}{El actor no cuenta con las monedas necesarias }

  \Sistema Muestra el mensaje  ''¡Oh no!, No tienes suficiente dinero para comprar el objeto [NombreDelObjeto]!''%\refElem{MSG-F02}.

\end{UCtrayectoriaA}

\begin{UCtrayectoriaA}[Fin del caso de uso]%
{B}{El actor ya tiene comprado el objeto}

    \Sistema Muestra el mensaje ''¡Ya tenías desbloqueado el objeto [NombreDelObjeto]!''%\refElem{MSG-F03}

\end{UCtrayectoriaA}
   % Comprar objeto

% =========================================================
\clearpage
\subsection{Diseño}

\subsubsection{Interfaces del módulo de competencia}

    \subsubsection{IU-M11 Administración de complementos}


    \IUfig{1}{modulos/moodle/IU/plugins_administrados}{IU-M11}{Administración de complementos}

\subsubsection{Elementos Relevantes}

    \begin{itemize}
        \item {\bf Configuración general}
            Opción que redirige a la pantalla \refElem{IU-F01}.
        \item {\bf Configuración financiera}
            Opción que redirige a la pantalla \refElem{IU-F02}.
    \end{itemize}


\clearpage
  % Configuraciones de plugins
    
\subsubsection{IU-F01: Configuración general del módulo financiero}

 Pantalla que se usa para configurar los elementos generales del módulo financiero.

    \IUfig{1}{modulos/finan/IU/configuracion_general}{IU-F01}{%
       Configuración general del módulo financiero}
\subsubsection{Elementos Relevantes}

    \begin{itemize}
        \item {\bf Activar complemento financiero}
            Esta opción activa y desactiva el módulo financiero.
    \end{itemize}

\subsubsection{Acciones relevantes}

    \begin{itemize}
        \item {\bf Configuración financiera}
            Esta opción redirige a la pantalla \refElem{IU-F02}.
        \item {\bf Guardar cambios}
            Esta opción guarda la configuración actual.
    \end{itemize}

\clearpage
  % Configuraciones Generales
    
\subsubsection{IU-F02: Configuración financiera}

 Pantalla que se utiliza para configurar cuántas monedas se dan, en qué eventos se dan y cuánto valen las monedas de oro respecto a las de plata.

    \IUfig{1}{modulos/finan/IU/configuracion_financiera}{IU-F02}{%
       Configuración financiera}
\subsubsection{Elementos Relevantes}

    \begin{itemize}
        \item {\bf Monedas de plata a oro}
            Esta opción especifica cuántas monedas de plata equivalen a una de oro.
        \item {\bf Evento competencia uno contra uno}
            Esta opción especifica cuántas monedas de plata se entregan a un usuario al haber derrotado a otro en las competencias uno contra uno.
        \item {\bf Evento competencia uno contra sistema}
            Esta opción especifica cuántas monedas de plata se entregan a un usuario al haber derrotado al sistema.
        \item {\bf Evento pregunta diaria}
            Esta opción especifica cuántas monedas de plata se entregan a un usuario al haber respondido correctamente la pregunta diaria.
    \end{itemize}

\subsubsection{Acciones relevantes}

    \begin{itemize}
        \item {\bf Activar evento competencia uno contra uno}
            Esta opción especifica si se dan monedas de plata a un usuario al haber derrotado a otro en las competencias uno contra uno.
        \item {\bf Activar evento competencia uno contra sistema}
            Esta opción especifica si se dan monedas de plata a un usuario al haber derrotado al sistema.
        \item {\bf Activar evento pregunta diaria}
            Esta opción especifica si se dan monedas de plata a un usuario al haber respondido correctamente la pregunta diaria.
    \end{itemize}

\clearpage
  % Configuraciones de finanzas


%

\subsubsection{Diseño de complementos}



A continuación se presenta cómo los submódulos de competencia
se implementan en moodle.\\


\noindent Resumiendo el módulo de competencia tiene 2 actividades establecidas, llamadas;
competencia uno contra uno y competencia uno contra sistema.
Ambas actividades deben aparecer dentro de la lista de actividades de moodle. Para ello
moodle cuenta con un tipo de complemento que se denomina \textbf{'mod'}, este tipo de complemento al ser instalado
en una plataforma de moodle, crea una nueva opción a la lista de actividades.\\

\noindent Tomando en consideración lo anterior y que existe el complemento gamedlemaster, se presenta en la figura \ref{fig:diseno-comp-comp}
los complementos contemplados y las dependencias entre los mismos.


    \addfigure{1}{modulos/comp/diagrams/diseno_complementos}{fig:diseno-comp-comp}{Implementación del modulo de competencia}


Cada complemento en la figura \ref{fig:diseno-comp-comp} está representado con una cadena que sigue el formato 'tipo\_de\_complemento:nombre\_de\_complemento'. Los tipos de complemento son;
\begin{itemize}
    \item \textbf{mod} - Este complemento permite crear una actividad que aparece en la lista de actividades a agregar a un curso.
    \item \textbf{local} -  Este complemento puede ser usado para múltiples propósitos relacionados con la gestión de la información.
    \item \textbf{block} - Este complemento permite desplegar una sección en la mayoría de las páginas de moodle, la cuál puede representar código html.
\end{itemize}

La función de cada uno de los complementos presentados en la figura \ref{fig:diseno-comp-comp} son:


\begin{itemize}
    \item \textbf{gamedlemaster} Definir la base de datos y los eventos a manejar.
    \item \textbf{gmcompcpu} Definir la competencia uno contra sistema.
    \item \textbf{gmcompvs} Definir la competencia uno contra uno.
    \item \textbf{gmcs} Entregar las monedas por ganar cada una de las competencias anteriores.
\end{itemize}

El complemento de tipo  \textbf{'mod'} tiene un requerimiento en su nombre, el cual es; 'El nombre del complemento a instalar debe ser igual a un nombre
de una de las tablas en la base de datos'. Debido a esto y que moodle no soporta nombres de complementos que contengan guiones bajos, el
nombre de la tabla ya no puede llevarlos.\\


\subsection{Pruebas}

    %
\TestCase{CPC-C01}{Crear nuevas instancias de la actividad de competencia 1 contra 1}

    %
\TestCase{CPC-E02}{Realizar configuraciones del módulo de experiencia}

    %
\TestCase{CPC-E02-1}{Realizar configuración de visualización de niveles}

    %
\TestCase{CPI-E02-1a}{Realizar configuraciones visuales con todos los datos erroneos}

    %
\TestCase{CPI-E02-1b}{Configuraciones visuales con formato y nombre de imagen inválidos}


    %
\TestCase{CPC-E02-2a}{Realizar configuraciones del sistema de experiencia}

    %
\TestCase{CPC-E02-2b}{Realizar configuraciones con cursos iniciados}

    %
\TestCase{CPC-E02-2c}{Realizar configuraciones del sistema de experiencia con estudiantes con experiencia establecida}

    %
\TestCase{CPI-E02-2}{Realizar configuraciones del sistema de experiencia con datos inválidos}


    %
\TestCase{CPC-E02-3}{Realizar configuraciones del sistema de experiencia con datos correctoss}

    %
\TestCase{CPI-E02-3}{Realizar configuraciones del eventos con datos inválidos}


    %
\TestCase{CPC-E03}{Desinstalar plugins del módulo de experiencia}


    %
\TestCase{CPC-E04}{Crear un curso gamificado}

    %
\TestCase{CPI-E04}{Crear un curso gamificado con la experiencia deshabilitada}


    %
\TestCase{CPC-E05}{Eliminar un curso gamificado sin alumnos inscritos}

    %
\TestCase{CPC-E05a}{Eliminar un curso gamificado con estudiantes inscritos}


    %
\TestCase{CPC-E12}{Crear un usuario gamificado (administrador)}

    %
\TestCase{CPC-E12a}{Crear un usuario gamificado mediante el auto-registro}


    %
\TestCase{CPC-E13}{Eliminar usuario gamificado}


    
\subsection{Análisis}

 Este apartado contiene el análisis requerido para la elaboración de módulo de experiencia,
 contiene la especificación del alcance de este módulo, la descripción de las funcionalidades
 a desarrollar, la reglas de negocio que rigen el comportamiento del módulo, y por último la
 especificación de los casos de uso a los que brinda soporte.

\subsubsection{Esquema de experiencia}

 El esquema de experiencia le proporciona al \refElem{aAdministrador} y a los \refElem[Profesores]%
 {aProfesor} un mecanismo mediante el cual pueden configurar la forma en que se obtienen los puntos
 de experiencia, la cantidad a otorgar, el número de puntos de cada nivel y finalmente la
 visualización del nivel y de los puntos de cada usuario.\\

 \noindent
 Las configuraciones fueron organizadas en dos grupos: las {\it configuraciones a
 nivel plataforma} las cuales definen valores de forma global, y las {\it configuraciones a nivel
 curso} las cuales definen valores para un curso en específico. A continuación se describen cada
 uno de los grupos.\\

\subsubsection{Submódulo de niveles}
\subsubsection{Funcionalidades}

% \subsubsection{Esquema de experiencia.}

 % El esquema de experiencia es la configuración sobre cómo funciona el sistema
 % de puntos de experiencia, incluyendo la cantidad de experiencia que tiene
 % cada nivel, el tipo de incremento en los puntos de experiencia nivel a nivel,
 % las restricciones sobre la experiencia y la forma en que se otorgarán los puntos.

% \subsubsection{Niveles.}

 % Es el mecanismo que permite mostrarle a los estudiantes el progreso que han tenido
 % a nivel plataforma mediante el nivel y los puntos de experiencia obtenidos en
 % los cursos, además contiene la configuración para establecer el cómo se verá el
 % nivel y la experiencia obtenida de dicho nivel.

 % Los principios de gamificación a los cuales permite brindarles soporte son:
 % \principioII
 % \principioVI
%
\subsubsection{Reglas de negocio} %========================================================

 En esta sección se especifican todas las reglas de negocio relevantes para el módulo de
 experiencia. Las reglas de negocio que establece moodle son diferenciadas por tener la letra {\it M}
 antecediendo al número consecutivo en su identificador.

    
\subsection{Entidades de moodle}

Debido a que moodle cuenta con más de 400 entidades en su versión 3.5, se opta
por mostrar 2 subconjuntos que muestren las entidades que se utilizan para el proyecto.\\

\noindent El primer subconjunto es aquel que explica la forma en que moodle implementa los cursos,
secciones de curso, actividades, usuarios y roles (el cual se presenta en la figura \ref{fig:BD-ER-M1}),
mientras que el segundo subconjunto muestra como moodle maneja toda la
estructura de las preguntas creadas por el profesor y respondidas por el estudiante
(el cual se presenta en la figura \ref{fig:BD-ER-M2}).

\noindent El objetivo de ambos esquemas (\ref{fig:BD-ER-M1} y \ref{fig:BD-ER-M2}) es expresar la idea general que abarcan ambos subconjuntos.

\clearpage
\addfigure{0.7}{analisis/diagrams/db_module_structure}{fig:BD-ER-M1}{Esquema de la base de datos de moodle 'Cursos'}


\noindent Utilizando la figura \ref{fig:BD-ER-M1}, se obtuvieron las siguientes reglas y características que tiene moodle respecto a los usuarios en un curso y a la estructura de los cursos.
\begin{enumerate}
    \item Un usuario -{\it mdl\_user}- tiene un rol -{\it mdl\_role}- en un cierto contexto -{\it mdl\_context}-, cuyo  '{\it context\_level}' sea igual a cincuenta(50).
    \item Si el contexto '{\it context\_level}' es de 50, el atributo '{\it instance\_id}' hace referencia al atributo '{\it id}' de un curso -{\it mdl\_course}-.
    \item El curso -{\it mdl\_course}- tiene varias secciones -{\it mdl\_course\_sections}-.
    \item Cada seccion -{\it mdl\_course\_sections}- tiene varias actividades -{\it mdl\_course\_modules}- que pertenecen a un tipo de actividad -{\it mdl\_modules}-.
    \item Por cada registro en tipo de actividad -{\it mdl\_modules}-, se tiene una entidad que lleva el mismo nombre.
    \item El atributo '{\it instance\_id}' de una actividad  -{\it mdl\_course\_modules}- apunta a diferentes entidades. La entidad a la que apunta depende del nombre del tipo de actividad -{\it mdl\_modules}-.
    \item Un usuario -{\it mdl\_user}- se inscribió -{\it mdl\_user\_enrolments}- a un curso -{\it mdl\_course}-, por medio de un formato soportado de inscripción -{\it mdl\_enrol}-.
\end{enumerate}

\clearpage

 \addfigure{0.7}{analisis/diagrams/db_module_questions}{fig:BD-ER-M2}{Esquema de la base de datos de moodle 'Preguntas' }



\noindent Utilizando la figura \ref{fig:BD-ER-M2}, se obtuvieron las siguientes reglas y características que tiene moodle respecto a las preguntas.
\begin{enumerate}
    \item Las preguntas -{\it mdl\_question}- tienen versiones -{\it mdl\_question\_attempts}-.
    \item Una pregunta -{\it mdl\_question}- pertenece a un banco de preguntas -{\it mdl\_question\_categories}-.
    \item La versión de una pregunta -{\it mdl\_question\_attempts}- es contestada -{\it mdl\_question\_usages}- en un determinado contexto -{\it mdl\_context}-.
    \item Un usuario -{\it mdl\_user}- responde una versión de una pregunta -{\it mdl\_question\_attempt\_stepts}-.
    \item El responder una versión de una pregunta -{\it mdl\_question\_attempt\_stepts}- conlleva pasos\\ -{\it mdl\_question\_attempt\_stept\_data}-, los cuales son: cómo se muestra, si ya se terminó de responder y qué se respondió.
\end{enumerate}


 A continuación se presenta la especificación de las entidades del esquema de base
 de datos de moodle que son relevantes para el desarrollo de los módulos y submódulos
 de proyecto.

    \begin{cdtEntidad}{mdl-config-plugins}{Configuración de Plugin}{%
    Es una tabla del núcleo de moodle que almacena todas las configuraciones globales
    relacionadas a los plugins instalados, al iniciar moodle las configuraciones de los
    plugins instalados y habilitados se cargan en memoria.}

	    \brAttr{id}{Id}{tInt}{%
	        Es el dígito que representa el identificador único para una configuración
            específica de un plugin.\par

            \it Restricciones:
            \refElem{tPrimaryKey},
            \refElem{tAutoIncrement}.
        }

        \brAttr{plugin}{Plugin}{tVarchar}{%
            Cadena de caracteres del nombre identificador del plugin al cual pertenece
            la configuración.\par

            \it Restricciones:
            \refElem{tRequired},
            \refElem{tRange} (0,100),
            \refElem{tUniqueKey}
        }

        \brAttr{name}{Nombre}{tVarchar}{%
            Cadena de caracteres que representa el nombre de la configuración de un
            plugin en específico.\par

            \it Restricciones:
            \refElem{tUniqueKey},
            \refElem{tRange} (0,100),
            \refElem{tRequired}
        }

        \brAttr{value}{Valor}{tVarchar}{%
            Cadena que almacena el valor de una configuración perteneciente a alguno
            de los plugins instalados.\par

            \it Restricciones:
            \refElem{tRange} (0,4294967295),
            \refElem{tRequired}
        }
    \end{cdtEntidad}\schemeName{config\_plugins}

    \begin{cdtEntidad}{mdl-user}{Usuario de moodle}{%
    Es una tabla del núcleo de moodle que contiene toda la información que se
    almacena de los usuarios en la plataforma, independientemente del rol que
    estos contenga, esta relación contiene más de 53 atributos, sin embargo solo
    se detallan aquellos relevantes.}

	    \brAttr{id}{Id}{tInt}{%
	        Es el dígito que representa el identificador único para cada uno
            de los usuarios en moodle.\par

            \it Restricciones:
            \refElem{tPrimaryKey},
            \refElem{tAutoIncrement}.
        }
	    \brAttr{username}{nombre de usuario}{tVarchar}{%
	        .\par

            \it Restricciones:
            \refElem{tRequired},
            \refElem{tLength} 0-100
        }
	    \brAttr{password}{contraseña}{tVarchar}{%
	        .\par

            \it Restricciones:
            \refElem{tRequired},
            \refElem{tLength} 0-255.
        }
	    \brAttr{firstname}{nombre}{tVarchar}{%
	        .\par

            \it Restricciones:
            \refElem{tRequired},
            \refElem{tLength} 0-100
        }
	    \brAttr{lastname}{apellido}{tVarchar}{%
	        .\par

            \it Restricciones:
            \refElem{tRequired},
            \refElem{tLength} 0-100
        }
	    \brAttr{email}{correo}{tVarchar}{%
	        .\par

            \it Restricciones:
            \refElem{tRequired},
            \refElem{tLength} 0-100
        }
	    \brAttr{lastaccess}{último registro}{tInt}{%
	        .\par

            \it Restricciones:
            \refElem{tRequired},
            \refElem{tLength} 10
        }
	    \brAttr{city}{ciudad}{tVarchar}{%
	        .\par

            \it Restricciones:
            \refElem{tRequired},
            \refElem{tLength} 0-120
        }
	    \brAttr{country}{pais}{tVarchar}{%
	        .\par

            \it Restricciones:
            \refElem{tRequired},
            \refElem{tLength} 2
        }

    \end{cdtEntidad}\schemeName{user}

    \begin{cdtEntidad}{mdl-course}{Curso de moodle}{%
    Es una tabla del núcleo de moodle que contiene la información principal de cada
    curso registrado en moodle. Esta entidad contiene 31 atributos, a continuación se
    detallan los atributos relevantes para la especificación de este proyecto.}

	    \brAttr{id}{Id}{tInt}{%
	        Es el dígito que representa al identificador único para cada uno
            de los cursos en moodle.\par

            \it Restricciones:
            \refElem{tPrimaryKey},
            \refElem{tAutoIncrement}.
        }

	    \brAttr{format}{formato}{tVarchar}{%
	        Es el dígito que representa al identificador único para cada uno
            de los cursos en moodle.\par

            \it Restricciones:
            \refElem{tRequired}.
            \refElem{tDefault} topics,
            \refElem{tLength} 0-21.
        }

	    \brAttr{fullname}{nombre completo}{tVarchar}{%
	        Es el nombre completo que se le asigna al curso.\par

            \it Restricciones:
            \refElem{tRequired}.
            \refElem{tLength} 0-21.
        }

	    \brAttr{shortname}{nombre corto}{tVarchar}{%
            Es el nombre corto que se le asigna al curso.\par

            \it Restricciones:
            \refElem{tRequired}.
            \refElem{tLength} 0-21.
        }

    \end{cdtEntidad}\schemeName{course}

    \begin{cdtEntidad}{mdl-course-section}{Sección del curso de moodle}{%
    }
	    \brAttr{id}{Id}{tInt}{%
	        Es el dígito que representa al identificador único para cada sección
            de los cursos en moodle.\par

            \it Restricciones:
            \refElem{tPrimaryKey},
            \refElem{tAutoIncrement}.
        }

        \brAttr{name}{nombre}{tVarchar}{%
	        Es el dígito nombre que permite identificar a una sección dentro de un curso
            en moodle.\par

            \it Restricciones: ...
        }
    \end{cdtEntidad}\schemeName{course\_sections}

    \begin{cdtEntidad}{mdl-course-format-options}{Opciones del formato del curso}{%
    }
	    \brAttr{id}{Id}{tInt}{%
	        Es el dígito que representa al identificador único para cada uno
            de los cursos en moodle.\par

            \it Restricciones:
            \refElem{tPrimaryKey},
            \refElem{tAutoIncrement}.
        }

	    \brAttr{courseid}{Id}{tInt}{%
	        Es el dígito que representa al identificador único para cada uno
            de los cursos en moodle.\par

            \it Restricciones:
            \refElem{tForeignKey},
            \refElem{tRequired}
        }

	    \brAttr{format}{formato}{tVarchar}{%
	        Es el dígito que representa al identificador único para cada uno
            de los cursos en moodle.\par

            \it Restricciones:
            \refElem{tRequired}.
            \refElem{tDefault} topics,
            \refElem{tLength} 0-21.
        }

	    \brAttr{name}{opcion}{tVarchar}{%
	        Es el dígito que representa al identificador único para cada uno
            de los cursos en moodle.\par

            \it Restricciones:
            \refElem{tPrimaryKey},
            \refElem{tLength} 0-100
        }

	    \brAttr{value}{valor}{tVarchar}{%
	        Es el dígito que representa al identificador único para cada uno
            de los cursos en moodle.\par

            \it Restricciones:
            \refElem{tRequired}
        }

    \end{cdtEntidad}\schemeName{course\_format\_options}

    \begin{cdtEntidad}{mdl-course-category}{Categoria de curso}{%
      .}
        \brAttr{id}{id}{tInt}{%
        .}
        \brAttr{name}{nombre}{tInt}{%
        .}

    \end{cdtEntidad}\schemeName{course\_category}

    \begin{cdtEntidad}{mdl-course-module}{Actividad del curso}{%
    .}
	    \brAttr{id}{Id}{tInt}{%
	        Es el dígito que representa al identificador único para cada uno
            de los cursos en moodle.\par

            \it Restricciones:
            \refElem{tPrimaryKey},
            \refElem{tAutoIncrement}.
        }
        \brAttr{course}{curso}{tInt}{%
        .}
        \brAttr{module}{actividad}{tInt}{%
        .}
        \brAttr{section}{sección}{tInt}{%
        .}
    \end{cdtEntidad}\schemeName{course\_module}

    \begin{cdtEntidad}{mdl-course-module-completion}{Actividad del curso para alumno}{%
    .}
	    \brAttr{id}{Id}{tInt}{%
	        Es el dígito que representa al identificador único para cada uno
            de los cursos en moodle.\par

            \it Restricciones:
            \refElem{tPrimaryKey},
            \refElem{tAutoIncrement}.
        }
        \brAttr{coursemoduleid}{actividad}{tInt}{%
        .}
        \brAttr{userid}{usuario}{tInt}{%
        .}
        \brAttr{completionstate}{completitud}{tBoolean}{%
        .}
    \end{cdtEntidad}\schemeName{course\_module}

    \begin{cdtEntidad}{Plugin}{Plugin}{%
    La forma en que moodle obtiene información acerca de los plugins es analizando
    los archivos internos de cada uno, a pesar de que los plugins no forman parte
    del esquema de base de datos, si forman parte del modelo de información que
    utiliza Moodle.}

	    \brAttr{componente}{Componente}{tVarchar}{%
	        Cadena compuesta por el tipo de plugin y el nombre del mismo, que
            representa a la clase principal del plugin que contiene los métodos
            principales del plugin.\par

            \it Restricciones: Ninguna
        }

	    \brAttr{pluginname}{Nombre}{tVarchar}{%
	        Es el nombre del plugin obtenido de los archivos de
            internacionalización presentes en el plugin, el valor de esta cadena
            depende del lenguaje seleccionado en moodle.\par

            \it Restricciones: Ninguna
        }

	    \brAttr{fullpath}{Ruta absoluta}{tPath}{%
	        La ruta absoluta de un plugin denota la ubicación del plugin en el
            sistema de archivos, esta ruta está compuesta por la ruta absoluta
            de la instalación de moodle, la carpeta correspondiente al tipo del
            plugin y el nombre del plugin.\par

            \it Restricciones: Formato ``/path/to/moodle/plugintype/pluginname''
        }

	    \brAttr{path}{Ruta relativa}{tPath}{%
	        La ruta relativa denota la ubicación del plugin dentro de la carpeta
            donde se encuentran los archivos de moodle, esta ruta está compuesta
            por la carpeta correspondiente al tipo del plugin y el nombre del
            plugin.\par

            \it Restricciones: Formato ``plugintype/pluginname''
        }

	    \brAttr{version}{Versión}{tVersion}{%
	        Numero entero de longitud de 10 dígitos que representa la versión del
            plugin.\par

            \it Restricciones: Ninguna adicional al tipo de dato
        }

	    \brAttr{moodle}{Versión de Moodle}{tVersion}{%
	        Número entero de longitud de 10 dígitos que representa la versión de
            moodle en la que se puede instalar el plugin.\par

            \it Restricciones: Ninguna adicional al tipo de dato
        }

        \brAttr{dependencies}{Dependencias}{tObject}{%
            Objeto que almacena un conjunto de claves con sus respectivos valores,
            donde cada clave representa el nombre del componente del plugin y el valor
            es la \refElem{Plugin.version} requerida del mismo.

            \it Restricciones: Ninguna
        }

        \brAttr{icon}{ícono}{tImage}{%
            Imagen para el ícono del plugin, debe estar contenida en el directorio
            {\it pix/} del plugin y tener como nombre {\it icon.png} o {\it icon.svg},
            moodle recomienda tener ambos archivos por si los navegadores no soportan
            algún tipo de archivo \cite{moodlePluginfiles}.\par

            \it Restricciones: El nombre debe ser icono con extensiones png o svg
        }

    \end{cdtEntidad}
 % Archivo de plugin
    \input{modulos/exp/BR/BR-E01} % Restricciones sobre de imagen del nivel.
    
\begin{BusinessRule}[%
Autor/Daniel Isai Ortega Zúñiga,%
Version/0.1,%
Estado/revision]%
%
{BR-E02}{Permanencia de los puntos de experiencia}
    \BRitem[control]{Revisor}{Sin asignar.}

 \BRsection[control]{Atributos}
    
    %\BRitem[admin]{Clase}{\bcCondition}%
    \BRitem[admin]{Clase}{\bcIntegridad}%
    %\BRitem[admin]{Clase}{\bcAutorizacion}%
    %\BRitem[admin]{Clase}{\bcDerivacion}%
        
    \BRitem[admin]{Tipo}{\btEnabler}%
    %\BRitem[admin]{Tipo}{\btTimer}%
    %\BRitem[admin]{Tipo}{\btExecutive}%
        
    %\BRitem[admin]{Nivel}{\blControlling}
    \BRitem[admin]{Nivel}{\blInfluencing}
    
    \BRitem{Descripción}{%
    Los puntos de experiencia una vez que son obtenidos no pueden ser quitados bajo
    ninguna condicion exceptuando únicamente la acción de eliminación de un usuario y
    la desinstalación de los plugins que formen parte del esquema de experiencia.}

    \BRitem{Ejemplo positivo}{\hfill\par%
        \begin{itemize}
        \item Un usuario conforme va completando las secciones de los cursos obtiene
              puntos de experiencia, si el curso es eliminado entonces el usuario 
              deberá permanecer con los puntos de experiencia obtenidos.

        \item Un usuario con 300 puntos de experiencia es eliminado del sitio, y en
              consecuencia se eliminan sus puntos de experiencia.
        \end{itemize}
    }

    \BRitem{Ejemplo negativo}{\hfill\par%
        \begin{itemize}
        \item Un curso es eliminado y a todos los estudiantes se les resta de sus
              puntos de experiencia la cantidad de experiencia obtenida durante el
              curso.
        \end{itemize}
    }% 
    
 \end{BusinessRule}

 % Permanencia del nivel de experiencia.
    
\begin{BusinessRule}[%
Autor/Daniel Isai Ortega Zúñiga,%
Version/0.1,%
Estado/revision]%
%
{BR-E03}{Tipos de Incremento}
    \BRitem[control]{Revisor}{Sin asignar.}

 \BRsection[control]{Atributos}
    
    \BRitem[admin]{Clase}{\bcCondition}%
    %\BRitem[admin]{Clase}{\bcIntegridad}%
    %\BRitem[admin]{Clase}{\bcAutorizacion}%
    %\BRitem[admin]{Clase}{\bcDerivacion}%
        
    \BRitem[admin]{Tipo}{\btEnabler}%
    %\BRitem[admin]{Tipo}{\btTimer}%
    %\BRitem[admin]{Tipo}{\btExecutive}%
        
    \BRitem[admin]{Nivel}{\blControlling}
    %\BRitem[admin]{Nivel}{\blInfluencing}
    
    \BRitem{Descripción}{%
    Cuando se modifiquen el \refElem{xp-scheme-settings} o la \refElem{levelXP} de las
    \refElem{xp-scheme-settings}
    }

    \BRitem{Ejemplo positivo}{\hfill\par%
        \begin{itemize}
        \item ...
        \end{itemize}
    }

    \BRitem{Ejemplo negativo}{\hfill\par%
        \begin{itemize}
        \item ...
        \end{itemize}
    }% 
    
 \end{BusinessRule}
 % Tipos de incremento
    \begin{BusinessRule}[%
Autor/Daniel Isai Ortega Zúñiga,%
Version/0.1,%
Estado/revision]%
%
{BR-E04}{Calculo de experiencia del nivel con incremento porcentual}
    \BRitem[control]{Revisor}{Sin asignar.}

 \BRsection[control]{Atributos}
    
    \BRitem[admin]{Clase}{\bcCondition}%
    %\BRitem[admin]{Clase}{\bcIntegridad}%
    %\BRitem[admin]{Clase}{\bcAutorizacion}%
    %\BRitem[admin]{Clase}{\bcDerivacion}%
        
    \BRitem[admin]{Tipo}{\btEnabler}%
    %\BRitem[admin]{Tipo}{\btTimer}%
    %\BRitem[admin]{Tipo}{\btExecutive}%
        
    \BRitem[admin]{Nivel}{\blControlling}
    %\BRitem[admin]{Nivel}{\blInfluencing}
    
    \BRitem{Descripción}{%
        El calculo para obtener la experiencia del nivel $i$ uando el tipo de
        incremento es porcentual está dado por la siguiente fórmula: Sea {\it exp()}
        la función que optiene la experiencia de un nivel en específico, sea tambien
        $i$ el nivel del cual se calcula la experiencia, sea $inc$ el factor de
        incremento de nivel a nivel, y finalmente sea $round()$ una función de
        redondeo a números enteros, entonces:

            $$ exp(i) = round( exp(1) * (inc)^{(i-1)})$$
    }

%   \BRitem{Sentencia}{%
%       Si $fecha$ 
%   }%

    \BRitem{Ejemplo positivo}{\hfill\par%
        \begin{itemize}
        \item La experiencia requerida para superar el nivel 1 es de 2000 puntos y el
              factor de incremento entre los niveles es 1.1, entonces la experiencia
              requerida para pasar el nivel 5 es de 2928 puntos.
        \end{itemize}
    }

    \BRitem{Ejemplo negativo}{\hfill\par%
        \begin{itemize}
        \item La experinecia requerida para superar el nivel 1 es de 2000 puntos y el
              factor de incremento entre los niveles es 1.1, entonces la experiencia
              requerida para pasar el nivel 5 es de 2300 puntos.
        \end{itemize}
    }% 
    
\end{BusinessRule}
 % Incremento porcentual
    \begin{BusinessRule}[%
Autor/Daniel Isai Ortega Zúñiga,%
Version/0.1,%
Estado/revision]%
%
{BR-E05}{Cálculo de experiencia del nivel con incremento linea} % Cuando están iniciados
    \BRitem[control]{Revisor}{Sin asignar.}

 \BRsection[control]{Atributos}
    
    \BRitem[admin]{Clase}{\bcCondition}%
    %\BRitem[admin]{Clase}{\bcIntegridad}%
    %\BRitem[admin]{Clase}{\bcAutorizacion}%
    %\BRitem[admin]{Clase}{\bcDerivacion}%
        
    \BRitem[admin]{Tipo}{\btEnabler}%
    %\BRitem[admin]{Tipo}{\btTimer}%
    %\BRitem[admin]{Tipo}{\btExecutive}%
        
    \BRitem[admin]{Nivel}{\blControlling}
    %\BRitem[admin]{Nivel}{\blInfluencing}
    
    \BRitem{Descripción}{%
    }

%   \BRitem{Sentencia}{%
%       Si $fecha$ 
%   }%

    \BRitem{Ejemplo positivo}{\hfill\par%
        \begin{itemize}
        \item ...
        \end{itemize}
    }

    \BRitem{Ejemplo negativo}{\hfill\par%
        \begin{itemize}
        \item ...
        \end{itemize}
    }% 
    
\end{BusinessRule}
 % Incremento lineal
    \begin{BusinessRule}[%
Autor/Daniel Isai Ortega Zúñiga,%
Version/0.1,%
Estado/revision]%
%
{BR-E06}{Eliminación de cursos gamificados} % Cuando están iniciados
    \BRitem[control]{Revisor}{Sin asignar.}

 \BRsection[control]{Atributos}

    \BRitem[admin]{Clase}{\bcCondition}%
    %\BRitem[admin]{Clase}{\bcIntegridad}%
    %\BRitem[admin]{Clase}{\bcAutorizacion}%
    %\BRitem[admin]{Clase}{\bcDerivacion}%

    \BRitem[admin]{Tipo}{\btEnabler}%
    %\BRitem[admin]{Tipo}{\btTimer}%
    %\BRitem[admin]{Tipo}{\btExecutive}%

    \BRitem[admin]{Nivel}{\blControlling}
    %\BRitem[admin]{Nivel}{\blInfluencing}

    \BRitem{Descripción}{%
        Debido a que el \refElem{mdl-course.format} por defecto para los cursos de
        moodle es el formato de tópicos/temas El formato de curso gamificado extiende
        las funcionalidades de este formato para facilitar la migración de un curso
        gamificado a uno no gamificado y viceversa. % TODO Pasar a analisis.
        La desinstalación del módulo de experiencia implica que los cursos con el
        \refElem{xp-course.format} gamificado (gamedle) se migren a cursos no
        gamificados, por compatibilidad en este migración se deben realizar las
        siguientes acciones de forma transaccional:

        \begin{itemize}
        \item El formato de los \refElem[cursos gamificados]{xp-course} debe
              cambiarse por el formato por defecto que tienen los cursos en moodle
              el cual es el de tópicos/temas.

        \item Se deben eliminar las \refElem[opciones del formato del curso]%
              {mdl-course-format-options} gamificado (gamedle).

        \item Se deben establecer los valores para las opciones del formato de
              tópicos/temas, que tienen por \refElem[nombre]%
              {mdl-course-format-options.name} secciones ocultas y aspecto del curso.
        \end{itemize}
    }

%   \BRitem{Sentencia}{%
%       Si $fecha$
%   }%

    \BRitem{Ejemplo positivo}{\hfill\par%
        \begin{itemize}
        \item Cuando se desinstala los plugins correspondientes al módulo de
              experiencia los cursos que estan vinculados con el formato gamificado
              son cambiados al formato de topicos/temas (formato por defecto de
              moodle), y también se establecen las opciones equivalentes al formato
              gamificado.
        \end{itemize}
    }

    \BRitem{Ejemplo negativo}{\hfill\par%
        \begin{itemize}
        \item Cuando se desinstala los plugins correspondientes al módulo de
              experiencia los cursos que estan vinculados con el formato gamificado
              no son cambiados al formato de topicos/temas ocasionando inconsistencia
              entre los cursos y los formatos.
        \end{itemize}
    }%

\end{BusinessRule}
 % Eliminación de cursos gamificados
    \begin{BusinessRule}[%
Autor/Daniel Isai Ortega Zúñiga,%
Version/0.1,%
Estado/edicion]%
%
{BR-E07}{Valores iniciales de experiencia}

     \BRitem[control]{Revisada por}{Pendiente.}

 \BRsection[control]{Atributos}
    % Clases: \bcCondition, \bcIntegridad, \bcAutorization o \bcDerivation
    % Tipos: \btEnabler, \btTimer o \btExecutive
    % Niveles: \blControlling o \blInfluencing.

    \BRitem[admin]{Clase}{\bcIntegridad}%

    \BRitem[admin]{Tipo}{\btTimer}%

    \BRitem[admin]{Nivel}{\blControlling}

    \BRitem{Descripción}{%
        Cuando un \refElem{xp-user} es creado este debe de empezar a ganar puntos
        de experiencia a partir del \refElem{xp-user.level} uno, tenido cero puntos
        de experiencia en la \refElem{xp-user.levelxp} y \refElem{xp-user.xp}. Ningún
        usuario puede comenzar con valores distintos a los indicados anteriormente.
    }

%   \BRitem{Sentencia}{%
%       Si $fecha$ 
%   }%

    \BRitem{Ejemplo positivo}{\hfill\par%
        \begin{itemize}
        \item ...
        \end{itemize}
    }

    \BRitem{Ejemplo negativo}{\hfill\par%
        \begin{itemize}
        \item ...
        \end{itemize}
    }

 \end{BusinessRule}
 % Valores iniciales de experiencia
    \begin{BusinessRule}[%
Autor/Daniel Isai Ortega Zúñiga,%
Version/0.1,%
Estado/edicion]%
%
{BR-E08}{Valores iniciales de experiencia de un curso}

     \BRitem[control]{Revisada por}{Pendiente.}

 \BRsection[control]{Atributos}
    % Clases: \bcCondition, \bcIntegridad, \bcAutorization o \bcDerivation
    % Tipos: \btEnabler, \btTimer o \btExecutive
    % Niveles: \blControlling o \blInfluencing.

    \BRitem[admin]{Clase}{\bcIntegridad}%

    \BRitem[admin]{Tipo}{\btTimer}%

    \BRitem[admin]{Nivel}{\blControlling}

    \BRitem{Descripción}{%
        Cuando un \refElem{xp-course} es creado la \refElem[experiencia total del curso]%
        {xp-scheme-settings.courseXP} de ser dividida uniformemente entre las
        \refElem[secciones del curso gamificado]{xp-course-section}. Si la división del
        total de experiencia entre el número de secciones genera un residuo entonces este
        se deberá agregan a la última sección del curso.
    }

%   \BRitem{Sentencia}{%
%       Si $fecha$
%   }%

    \BRitem{Ejemplo positivo}{\hfill\par%
        \begin{itemize}
        \item ...
        \end{itemize}
    }

    \BRitem{Ejemplo negativo}{\hfill\par%
        \begin{itemize}
        \item ...
        \end{itemize}
    }

 \end{BusinessRule}
 % Valores iniciales de experiencia del curso
    \begin{BusinessRule}[%
Autor/Daniel Isai Ortega Zúñiga,%
Version/0.1,%
Estado/edicion]%
%
{BR-E09}{Secciones editables de un curso con experiencia}

     \BRitem[control]{Revisada por}{Pendiente.}

 \BRsection[control]{Atributos}
    % Clases: \bcCondition, \bcIntegridad, \bcAutorization o \bcDerivation
    % Tipos: \btEnabler, \btTimer o \btExecutive
    % Niveles: \blControlling o \blInfluencing.

    \BRitem[admin]{Clase}{\bcCondition}%

    \BRitem[admin]{Tipo}{\btEnabler}%

    \BRitem[admin]{Nivel}{\blControlling}

    \BRitem{Descripción}{%
        Para todas aquellas secciones de un curso gamificado que ya hayan sido
        completadas por al menos un estudiante, se debe bloquear la edición de los
        puntos de experiencia que otorgan con el propósito de impedir que una sección
        en distintos momentos para distintos alumnos otorge puntos de experiencia.
    }

%   \BRitem{Sentencia}{%
%       Si $fecha$
%   }%

    \BRitem{Ejemplo positivo}{\hfill\par%
        \begin{itemize}
        \item Al abrir la edición de un curso con experiencia con cinco secciones,
              se deshabilita la edición de las primeras tres secciones del curso
              ya que estas han sido completadas por almenos un estudiante.
        \end{itemize}
    }

    \BRitem{Ejemplo negativo}{\hfill\par%
        \begin{itemize}
        \item Al abrir la edición de un curso con experiencia con cinco secciones,
              no se deshabilitan la edición de las secciones del curso ya que hayan
              sido completadas con anterioridad.
        \end{itemize}
    }

 \end{BusinessRule}
 % Secciones editables de un curso con experiencia
    \begin{BusinessRule}[%
Autor/Daniel Isai Ortega Zúñiga,%
Version/0.1,%
Estado/edicion]%
%
{BR-E10}{Administración de la experiencia de un curso}

     \BRitem[control]{Revisada por}{Pendiente.}

 \BRsection[control]{Atributos}
    % Clases: \bcCondition, \bcIntegridad, \bcAutorization o \bcDerivation
    % Tipos: \btEnabler, \btTimer o \btExecutive
    % Niveles: \blControlling o \blInfluencing.

    \BRitem[admin]{Clase}{\bcCondition}%

    \BRitem[admin]{Tipo}{\btEnabler}%

    \BRitem[admin]{Nivel}{\blControlling}

    \BRitem{Descripción}{%
        Un evento puede ...
    }

%   \BRitem{Sentencia}{%
%       Si $fecha$
%   }%

    \BRitem{Ejemplo positivo}{\hfill\par%
        \begin{itemize}
        \item ...
        \end{itemize}
    }

    \BRitem{Ejemplo negativo}{\hfill\par%
        \begin{itemize}
        \item ...
        \end{itemize}
    }

 \end{BusinessRule}
 % Administración de la experiencia en un curso
    \begin{BusinessRule}[%
Autor/Daniel Isai Ortega Zúñiga,%
Version/0.1,%
Estado/edicion]%
%
{BR-E11}{Eliminación de las secciones con experiencia}

     \BRitem[control]{Revisada por}{Pendiente.}

 \BRsection[control]{Atributos}
    % Clases: \bcCondition, \bcIntegridad, \bcAutorization o \bcDerivation
    % Tipos: \btEnabler, \btTimer o \btExecutive
    % Niveles: \blControlling o \blInfluencing.

    \BRitem[admin]{Clase}{\bcCondition}%

    \BRitem[admin]{Tipo}{\btEnabler}%

    \BRitem[admin]{Nivel}{\blControlling}

    \BRitem{Descripción}{%
        Para poder eliminar una sección de un curso con soporte para experiencia
        esta sección debe configurar con un valor de cero puntos de experiencia.
        con propósito de que la eliminación de una sección no colisione con el
        cumplimiento de la regla \refElem{BR-E10}.
    }

%   \BRitem{Sentencia}{%
%       Si $fecha$
%   }%

    \BRitem{Ejemplo positivo}{\hfill\par%
        \begin{itemize}
        \item Se procede a eliminar un sección que tiene cero puntos de experiencia
              en un curso gamificado.
        \end{itemize}
    }

    \BRitem{Ejemplo negativo}{\hfill\par%
        \begin{itemize}
        \item Se procede a eliminar un sección que tiene 300 puntos de experiencia
              en un curso gamificado.
        \end{itemize}
    }

 \end{BusinessRule}
 % Eliminación de secciones con experiencia

    % INPUT: Cursos Igualitarios.
    % INPUT: Otorgar experiencia

\clearpage
\subsubsection{Casos de uso} % ============================================================

 En este apartado se especifican todos los casos de usos contemplados para el módulo de
 experiencia, para cada caso de uso se especifica su tabla de atributos la cual indica que casos
 de prueba deberán ejecutarse correctamente para corroborar la completitud del caso de uso.

\subsubsection*{Diagrama de casos de uso}

 En la figura \ref{exp:usecases} se detalla el diagrama de casos de uso correspondiente al módulo
 de experiencia. Los casos de uso de moodle (en color blanco) son modelados como casos de uso
 abstractos, mientras que los casos de uso del módulo de experiencia son diferenciados por el
 color azul, en total el desarrollo de este módulo consiste en 13 casos de uso principales.

    \addfigure{0.9}{modulos/UseCases}{exp:usecases}{%
        Diagrama de casos de uso del módulo de experiencia}

 \noindent
 Debido a que los plugins a desarrollar son elementos opcionales para Moodle, solo se puede
 acceder a los casos de uso del módulo de experiencia a través de puntos de extensión de los
 casos de uso de moodle. Por otra parte los casos de uso que serán documentados en esta sección
 serán los del módulo de experiencia debido a que Moodle proporciona en su página oficial, guías
 e instructivos como documentación de las funcionalidades que brinda.

    % CASOS DE USO DE MOODLE
    
% \ucstEnEdicion     Al terminar una revisión/aprobación con observaciones
%                    y al inicio del CU.
%
% \ucstEnRevision    Al terminar la edición del CU (version += 0.1).
% \ucstEnAprobacion  Al pasar la revision sin observaciones.
% \ucstAprobado      Al ser aprobado por el usuario (version += 1.0)

\begin{UseCase}[%
Autor/Daniel Ortega,%
Version/0.1,%
Estado/\ucstEnEdicion]%
%
{CU-M01}{Acceder a la administración del sitio}{% TODO; Deberia se Instalar/Actualizar ???
%
 Permite al \refElem{aAdministrador} acceder a la pantalla \refElem{IU-M01} realizar las
 distintas tareas que incluye la administración del sitio de moodle. Esta caso de uso es realizado
 debido a que es requerido para la ejecución de la mayoría de los casos de uso cuyo actor es el
 administrador.}

	\UCitem[control]{Revisor}{ Sin asignar }
	\UCitem[control]{Último cambio}{ 14/OCT/19 }

 \UCsection{Atributos}

    \UCitem{Actor(es)}{%
        \refElem{aAdministrador}
    }

	\UCitem{Propósito}{%
        Permitir al administrador acceder a la administración del moodle que administra.
	}

	\UCitem{Entradas}{-}

	\UCitem{Origen}{-}

	\UCitem{Salidas}{-}

	\UCitems{Destino}{%
        \refElem{IU-M01}
	}

	\UCitem{Precondiciones}{-}

	\UCitem{Postcondiciones}{-}

	\UCitem{Reglas de negocio}{-}

	\UCitem{Errores}{-}

	% \UCitem{Viene de}{% Indicar si el Caso de uso es primario o se extiende de otro. La mayoría se
					  % extienden de Login.
		% EJEMPLO: \refIdElem{PY-CU1} o Caso de uso primario.
	% 	\TODO Especificar.
	% }

 \UCsection[design]{Datos de Diseño}

	\UCitem[design]{Casos de Prueba}{%
        Incluidos en la ejecución de los casos de uso que incluyen a este caso de uso
    }

 \UCsection[admin]{Datos de Administración de Requerimiento}

	\UCitem[admin]{Observaciones}{-}

\end{UseCase}

\subsubsection{Trayectorias del caso de uso}

\begin{UCtrayectoria}%
%
    \Actor Presiona el botón \IUMenu de la pantalla \refElem{IU-M00}
    \Sistema Despliega el menú de navegación lateral.

    \Actor Selecciona la opción {\bf \IUAdminSitio Administración del sitio}
    \Sistema Carga la pantalla \refElem{IU-M01} con la pestaña de administración del
             sitio preseleccionada.

    \Actor Selecciona la pestaña {\bf plugins}. \refTray{A}
    \Sistema Carga la pantalla \refElem{IU-M01a}

\end{UCtrayectoria}

\begin{UCtrayectoriaA}{A}{El \refElem{aAdministrador} desea administrar los usuarios
dentro de la plataforma}

    \Actor Selecciona la pestaña {\bf Usuarios}.
    \Sistema Carga la pantalla \refElem{IU-M01b}.
\end{UCtrayectoriaA}

\subsubsection{Puntos de extensión}

\UCExtensionPoint{Instalación de un plugin}{%

    El \refElem{aAdministrador} desea extender la funcionalidad
    de moodle mediante la instalación de plugins.%
    }{Al inicio la trayectoria principal}{\refElem{CU-E01}}

\UCExtensionPoint{Configuraciones generales del módulo de experiencia}{%

    El \refElem{aAdministrador} desea cambiar las configuraciones
    generales del módulo de experiencia.%
    }{Al inicio la trayectoria principal}{\refElem{CU-E02}}

\UCExtensionPoint{Configuraciones visuales del módulo de experiencia}{%

    El \refElem{aAdministrador} desea establecer las configuraciones
    de la visualización de los niveles del módulo de experiencia.%
    }{Al inicio la trayectoria principal}{\refElem{CU-E02-1}}

\UCExtensionPoint{Configuraciones de comportamiento del módulo de experiencia}{%

    El \refElem{aAdministrador} desea establecer el comportamiento del
    sistema de experiencia que incluye el módulo de experiencia.%
    }{Al inicio la trayectoria principal}{\refElem{CU-E02-2}}

\UCExtensionPoint{Configuraciones de eventos del módulo de experiencia}{%

    El \refElem{aAdministrador} desea establecer la cantidad de experiencia
    que brindarán los eventos que soporta el módulo de experiencia.%
    }{Al inicio la trayectoria principal}{\refElem{CU-E02-3}}

\UCExtensionPoint{Desinstalación de un plugin}{%

    El \refElem{aAdministrador} desea desinstalar un plugin en
    moodle debido a que ya no requiere de las funcionalidades
    que este brinda%
    }{Al inicio la trayectoria principal}{\refElem{CU-E03}}

\UCExtensionPoint{Crear usuario gamificado}{%

    El \refElem{aAdministrador} desea crear un usuario en moodle
    para otorgarle acceso a las distintas funcionalidades que brinda
    moodle.%
    }{Al inicio la trayectoria principal}{\refElem{CU-E12}}

 % Acceder a la administración del sitio

    % MODULO DE EXPERIENCIA
    
% \ucstEnEdicion     Al terminar una revisión/aprobación con observaciones
%                    y al inicio del CU.
%
% \ucstEnRevision    Al terminar la edición del CU (version += 0.1).
% \ucstEnAprobacion  Al pasar la revision sin observaciones.
% \ucstAprobado      Al ser aprobado por el usuario (version += 1.0)

\begin{UseCase}[%
Autor/Daniel Ortega,%
Version/0.1,%
Estado/\ucstEnRevision]%
%
{CU-E01}{Instalar plugin del módulo de experiencia}{% TODO; Deberia se Instalar/Actualizar ???
%
 Permite al \refElem{aAdministrador} incluir todas las funcionalidades que brinda el módulo de
 experiencia al moodle que administra mediante la instalación de los plugins correspondientes.
 La conclusión de la trayectoria principal de esta caso de uso es una precondición para que los
 demás casos de uso del módulo de experiencia puedan ejecutarse.}

	\UCitem[control]{Revisor}{ Sin asignar }
	\UCitem[control]{Último cambio}{ 13/OCT/19 }

 \UCsection{Atributos}

    \UCitem{Actor(es)}{%
        \refElem{aAdministrador}
    }

	\UCitems{Propósito}{%
        \Titem Permitir al administrador incluir todas las funcionalidades que brinda el módulo de
        experiencia al moodle que administra.

        \Titem Permitir a los usuarios de moodle ver su progreso en la plataforma mediante puntos
        de experiencia.
	}

	\UCitem{Entradas}{\imprimeUC{entrada}}

	\UCitems{Origen}{%
        \Titem Mouse
	}

	\UCitem{Salidas}{\imprimeUC{salida}}

	\UCitem{Destino}{%
		\refElem{IU-M01}
	}

	\UCitems{Precondiciones}{%
        \Titem La carpeta comprimida que contiene los archivos del plugin
        \Titem El plugin debe cumplir con la regla \refElem{BR-M01} para poder ser
               instalado.
        % \Titem Si se trata de una actualización de un plugin la versión de este debe
               % cumplir con la regla \refElem{BR-M02}.
	}

	\UCitems{Postcondiciones}{%
        \Titem El plugin debe permanecer instalado en moodle.%
        \Titem La actualización de las \refElem{xp-general-settings} del módulo de experiencia
               deben persistirse en el sistema.
        \Titem Los \refElem[usuarios]{mdl-user} registrados en moodle deberán tener la
               asociada la información de un \refElem{xp-user}.
	}

	\UCitem{Reglas de negocio}{\imprimeUC{regla}}

	\UCitems{Errores}{%
        \Titem \UCerr{Err1}{%
            El archivo zip del plugin seleccionado está roto,}{% CAUSA
            no se puede continuar con la instalación del plugin}% EFECTO

        \Titem \UCerr{Err2}{%
            Alguna de las dependencias del plugin no se satisface con
            los plugins instalados}{% CAUSE
            no se puede continuar con las instalación del plugin}% EFECTO

        \Titem \UCerr{Err3}{%
        % CAUSA
            Durante la ejecución de las tareas de instalación del nuevo plugin ocurre
            un error,}{%
        % EFECTO
            las tareas de instalación del propio plugin no pudieron concluir apropiadamente,
            el plugin sigue }
	}

	% \UCitem{Viene de}{% Indicar si el Caso de uso es primario o se extiende de otro. La mayoría se
					  % extienden de Login.
		% EJEMPLO: \refIdElem{PY-CU1} o Caso de uso primario.
	% 	\TODO Especificar.
	% }

 \UCsection[design]{Datos de Diseño}

	\UCitems[design]{Casos de Prueba}{%
        \Titem \refElem{CPC-E01}
	}

 \UCsection[admin]{Datos de Administración de Requerimiento}

	\UCitem[admin]{Observaciones}{}

\end{UseCase}

\subsubsection{Trayectorias del caso de uso}

\begin{UCtrayectoria}%
%
    \includeUC{CU-M01}

    \Actor Selecciona la opción {\bf Instalar plugins}
    \Sistema Carga la pantalla \refElem{IU-M02} con el formulario para seleccionar el
             plugin a instalar. \label{CU-E01-formulario-instalacion}

    \Actor Presiona la opción {\bf Seleccione un archivo}
    \Sistema Despliega la pantalla \refElem{IU-M00a} como pantalla emergente
             \label{CU-E01-seleccion-archivo}

    \Actor Selecciona la opción {\it Subir un archivo} en el menu izquierdo de la pantalla
           emergente y posteriormente presiona el botón {\it Browse}.
    \Actor Selecciona el archivo que contiene al \entrada{Plugin} del módulo de experiencia.
    \Actor Presiona el botón {\bf Subir este archivo}.
    \Sistema Valida que el archivo del plugin sea de tipo {\it ZIP}. \refTray{A}
    \Sistema Cierra la pantalla emergente y muestra el nombre del archivo seleccionado en la
             pantalla \refElem{IU-M02}

    \Actor Presiona el botón {\bf Instalar plugin desde archivo ZIP}
    \Sistema Valida que el archivo {\it ZIP} cumpla con las restricciones dictadas por la
             \regla{BR-M01}. \refErr{Err1}
    \Sistema Obtiene el \salida{Plugin.componente}, la \salida{Plugin.fullpath} y el
             \salida{Plugin.pluginname} del plugin a ser instalado.
    \Sistema Carga la pantalla \refElem{IU-M02a}, mostrando los datos anteriormente obtenidos.

    \Actor Continua con la instalación de plugin presionando la opción {\bf continuar}. \refTray{B}

    \Sistema Obtiene tambien la \salida{Plugin.moodle}, la lista de \salida{Plugin.dependencies}
             y la \salida{Plugin.version} del plugin a instalar. \refErr{Err2}
             \label{CU-E01-comprobacion}
    \Sistema Despliega los datos obtenidos en la pantalla \refElem{IU-M02b}

    \Actor Presiona el botón {\bf Actualizar base de datos Moodle ahora}. \refTray{C}
    \Sistema Procesa las tareas de instalación de moodle.

    \Sistema Obtiene la lista de los \refElem[identificadores]{mdl-user.id} de los
             usuarios de moodle.
    \Sistema Asocia mediante los identificadores datos de un \refElem{xp-user}
             estableciendo el \refElem{xp-user.level} actual igual a $1$,
             la \refElem{xp-user.xp} igual a $0$.
    \Sistema Establece los valores por defecto para las \refElem{xp-visual-settings} (
              \entrada{xp-visual-settings.title},
              \entrada{xp-visual-settings.description},
              \entrada{xp-visual-settings.message},
              \entrada{xp-visual-settings.colorLvl},
              \entrada{xp-visual-settings.colorBar} e
              \entrada{xp-visual-settings.image}), especificadas en el modelo de información.

    \Sistema Carga la interfaz \refElem{IU-M02d} informando que la instalación
             ha sido llevaba a cabo de forma correcta. \refErr{Err3}

    \Actor Presiona el botón {\bf continuar}.

    \includeUC{CU-E02} a partir del paso \ref{CU-E03-formulario}
\end{UCtrayectoria}

\begin{UCtrayectoriaA}{A}{Cuando el archivo seleccionado es distinto de un ZIP}

  \Sistema Emite en una ventana emergente el mensaje {\it Error: ``El tipo de
           archivo \$EXT no se acepta.''} siendo {\it\$EXT} la extensión del
           archivo seleccionado.
  \Sistema Regresa al paso \ref{CU-E01-seleccion-archivo}.

\end{UCtrayectoriaA}

\begin{UCtrayectoriaA}[Fin del caso de uso]%
{B}{El \refElem{aAdministrador} desea cancelar la instalación después de la validación del archivo ZIP}

    \Actor Presiona el botón {\bf cancelar} de la pantalla \refElem{IU-M02a}.
    \Sistema Cancela la instalación del plugin y redirige a la pantalla \refElem{IU-M02}

\end{UCtrayectoriaA}

\begin{UCtrayectoriaA}[Fin del caso de uso]%
{C}{El \refElem{aAdministrador} desea cancelar la instalación después de ver la comprobación de plugins a instalar}

    \Actor Presiona el botón {\bf cancelar esta instalación} o {\bf cancelar las nuevas instalaciones}
    \Sistema Redirige a la pantalla \refElem{IU-M02c}

    \Actor Si el actor presiona el botón {\bf Continuar} entonces
    \UCpaso[--] el caso de uso terminará, (en caso contrario)

    \Actor Si el actor presiona el botón {\bf Cancelar}
    \Sistema Regresa al paso \ref{CU-E01-comprobacion}
\end{UCtrayectoriaA}

%\subsubsection{Puntos de extensión}

%\UCExtensionPoint{Nombre del punto de extensión}{%

%    El \refElem{aAdministrador} desea/requiere/necesita ....%
%
%    }{En el paso \ref{CU-ET-1x} de la trayectoria principal  ...%
%
%    }{\refElem{CU-E2-T}}
   % Instalar plugin del esquema de experiencia
    
% \ucstEnEdicion     Al terminar una revisión/aprobación con observaciones 
%                    y al inicio del CU.
%
% \ucstEnRevision    Al terminar la edición del CU (version += 0.1).
% \ucstEnAprobacion  Al pasar la revision sin observaciones.
% \ucstAprobado      Al ser aprobado por el usuario (version += 1.0)

\begin{UseCase}[%
Autor/Daniel Ortega,%
Version/0.1,%
Estado/\ucstEnEdicion]%
%
{CU-E02}{Configurar esquema de experiencia}{%
%
 Permite al \refElem{aAdministrador} configurar el esquema de experiencia, especificando
 la forma en que se obtienen los puntos de experiencia, la cantidad de puntos a otorgar, 
 el número de puntos de cada nivel y finalmente la visualización del nivel y de los puntos
 de cada usuario.}

	\UCitem[control]{Revisor}{ Sin asignar }
	\UCitem[control]{Último cambio}{ \today }

 \UCsection{Atributos}

    \UCitem{Actor(es)}{%
        \refElem{aAdministrador}
    }

	\UCitem{Propósito}{%
        Acceder a las menús de configuración del esquema de experiencia.
		% \Titem Establecer la cantidad de experiencia que brindarán los cursos.
        % \Titem Establecer la forma en que incrementa la cantidad de experiencia de cada nivel con respecto al anterior.
        % \Titem Configurar la visualización de la pantalla emergente al subir de nivel
        % \Titem Configurar la visualización del bloque que muestra el nivel actual del usuario.
        % \Titem Establecer visualización agrupando niveles
	}
	
%% BEGIN-BLOQUE PARA AGREGAR UNA REVISION ------------------------------------->
%% Copiar y descomentar este bloque por cada revision que se realice
%	\UCsection[control]{% Indicar la versión objeto de la revisión.
%		Revisión de la Versión \TODO X.X
%	}
%	\UCitem[control]{Revisó}{% Coloque el nombre de quien realizó la revisión
%		\TODO Especificar
%	}
%	\UCitem[control]{Fecha}{% Coloque la fecha de la revisión
%		% EJEMPLO: 21 de Septiembre de 2019.
%		\TODO Especificar
%	}
%	\UCitem[control]{Resultado}{% Las opciones son: 
%								% Pendiente: se pasa a EnEdicion y se agregan las observaciones
%								% Aprobado: Se pasa a EnAprobacion.
%		\TODO Especificar
%	}
%	\UCitems[control]{Observaciones}{
%		% Agregar las observaciones resultado de la revision o la palabra ``Ninguna''
%		\Titem \TODO Agregar observaciones en cada viñeta, usar el comando \TODO %\TOCHK \DONE.
%	}
%% <------------------------------------------ END-BLOQUE PARA AGREGAR UNA REVISION
	
	\UCitems{Entradas}{%
		\Titem Selección de la opción \IUMenu menu.
        \Titem Selección de la opción \IUAdminSitio Administración del Sitio.
        \Titem Selección de la pestaña {\it Plugins}.
	}

	\UCitems{Origen}{%
        \Titem Mouse
	}

	\UCitems{Salidas}{%
		\imprimeUC{salida}
	}

	\UCitems{Destino}{%
		\Titem \refElem{IU-M02a}
	}
	
	\UCitems{Precondiciones}{%
        % TODO: ¿Se deben especificar aunque eso sea diseño?
        \Titem Que los plugins del módulo de experiencia se encuentren instalados
	}

	\UCitem{Postcondiciones}{%
        Ninguna
	}

	\UCitem{Reglas de negocio}{%
		Ninguna
	}

	\UCitems{Errores}{%
        \Titem \UCerr{Err1}{%
        % CAUSA
            Los plugins del módulo de experiencia no se encuentran instalados,}{%
        % EFECTO
            No se presenta en el menu de opciones las opciones para modificar %
            el esquema de experiencia.}
	}

	% \UCitem{Viene de}{% Indicar si el Caso de uso es primario o se extiende de otro. La mayoría se 
					  % extienden de Login.
		% EJEMPLO: \refIdElem{PY-CU1} o Caso de uso primario.
	% 	\TODO Especificar.
	% }	

 \UCsection[design]{Datos de Diseño}

	\UCitems[design]{Casos de Prueba}{%
        \Titem \refElem{CPC-E02}
        \Titem \refElem{CPI-E02}
	}

 \UCsection[admin]{Datos de Administración de Requerimiento}

	\UCitem[admin]{Observaciones}{%
        Ninguna
	}

\end{UseCase}

\clearpage
\subsubsection{Trayectorias del caso de uso}

\begin{UCtrayectoria}%
%
 \Actor Presiona el botón \IUMenu en la esquina superior izquierda de la pantalla \refElem{IU-M01}
        para abrir el menu de navegación.

 \Actor Selecciona la opción {\it \IUAdminSitio Administración del sitio}

 \Sistema Carga la pantalla \refElem{IU-M02}

 \Actor Selecciona ver las opciones para administrar plugins presionando la pestaña
        {\it Plugins}

 \Sistema Obtiene las categorias de las opciones que se muestran en la pantalla.

 \Sistema En la categoría {\it Bloques} agrega la subcategoría \salida{tExpSettingsGeneral} con
          los enlaces a las configuraciones \salida{tExpSettingsVisual} y 
          \salida{tExpSettingsComportamiento} del módulo de experiencia. \refErr{Err1}

 \Sistema Carga la pantalla \refElem{IU-M02a}.

    % \Sistema ... \label{CU-E2-Menu}
    % \Actor ...  \refTray{B} \ref{CU-E2-Menu}
\end{UCtrayectoria}


\subsubsection{Puntos de extensión}

\UCExtensionPoint{Configuraciones generales}{%

    El \refElem{aAdministrador} desea cambiar las configuraciones generales del módulo de 
    experiencia, las cuales incluyen habilitar/deshabilitar el módulo de experiencia o
    la entrega de recompensa a otros eventos de Gamedle.%
%
    }{En el paso \ref{CU-E2-1x} de la trayectoria principal%
%
    }{\refElem{CU-E02-1}}

\UCExtensionPoint{Configuraciones Visuales}{%

    El \refElem{aAdministrador} desea cambiar las configuraciones generales del módulo de 
    experiencia, las cuales incluyen habilitar/deshabilitar el módulo de experiencia o
    la entrega de recompensa a otros eventos de Gamedle.%
%
    }{En el paso \ref{CU-E2-1x} de la trayectoria principal%
%
    }{\refElem{CU-E02-2}}

\UCExtensionPoint{Configuraciones de Comportamiento}{%

    El \refElem{aAdministrador} desea cambiar las configuraciones generales del módulo de 
    experiencia, las cuales incluyen habilitar/deshabilitar el módulo de experiencia o
    la entrega de recompensa a otros eventos de Gamedle.%
%
    }{En el paso \ref{CU-E2-1x} de la trayectoria principal%
%
    }{\refElem{CU-E02-3}}

   % Configuraciones generales
    
% \ucstEnEdicion     Al terminar una revisión/aprobación con observaciones
%                    y al inicio del CU.
%
% \ucstEnRevision    Al terminar la edición del CU (version += 0.1).
% \ucstEnAprobacion  Al pasar la revision sin observaciones.
% \ucstAprobado      Al ser aprobado por el usuario (version += 1.0)

\begin{UseCase}[%
Autor/Daniel Ortega,%
Version/0.1,%
Estado/\ucstEnRevision]%
%
{CU-E02-1}{Configurar visualización de niveles}{%
%
 Permite al \refElem{aAdministrador} establecer y modificar los colores, textos e
 imagenes que se muestran en la visualización del nivel actual y en las ventanas
 emergentes de un nuevo nivel alcanzado. Cuando un administrador actualice los valores
 de la configuración los usuarios serán capaces de ver los cambios al renderizarse la
 siguiente página.}

	\UCitem[control]{Revisor}{ Sin asignar }
	\UCitem[control]{Último cambio}{ \today }

 \UCsection{Atributos}

    \UCitem{Actor(es)}{%
        \refElem{aAdministrador}
    }

	\UCitem{Propósito}{%
      \Titem Cambiar las \refElem{xp-visual-settings} del módulo de experiencia.
      \Titem Cambiar el color de la barra de progreso del nivel actual que ve cada
             usuario.
      \Titem Establecer la imagen que se presenta en cada nivel de la plataforma.
      \Titem El mensaje de felicitaciones que se presenta al alcanzar un nuevo nivel
      \Titem Establecer el color que tendrá el número del nivel
      \Titem Cambiar la descripción de los mensajes visible cuando se alcanza un
             nuevo nivel.
      \Titem Cambiar el nombre de los niveles.
	}

	\UCitem{Entradas}{\imprimeUC{entrada}}

	\UCitems{Origen}{%
        \Titem Mouse
        \Titem Teclado
	}

	\UCitem{Salidas}{\imprimeUC{salida}}

	\UCitems{Destino}{%
		\Titem \refElem{IU-M01}
	}

	\UCitems{Precondiciones}{%
        \Titem Que los plugins del módulo de experiencia se encuentren instalados,
        \Titem El módulo de experiencia haya sido habilitando en el caso uso
               \refElem{CU-E02}
	}

	\UCitem{Postcondiciones}{%
        Los nuevos valores de las \refElem{xp-visual-settings} deben ser actualizados
        para todos los usuarios y además deben persistirse en el sistema.
	}

	\UCitem{Reglas de negocio}{\imprimeUC{regla}}

	\UCitems{Errores}{%
        \Titem \UCerr{Err1}{%
        % CAUSA
            Los plugins del módulo de experiencia no se encuentran instalados,}{%
        % EFECTO
            No se presentan en el menú las opciones para acceder a la pantalla
            de la configuración y no se puede llevar a cabo el caso de uso.}

        \Titem \UCerr{Err2}{%
        % CAUSA
            La imagen no cumple con las restricciones de nombre de la regla de negocio
            \refElem{BR-E01},}{%
        % EFECTO
            No se remplaza la imagen de la configuración actual, se realizan
            las demás actualizaciones con los datos ingresados por el usuario,
            y se emite el mensage que indica que el nombre es inválido.}

        \Titem \UCerr{Err3}{%
        % CAUSA
            Ocurre un fallo durante la persistencia de la imagen en el sistema,}{%
        % EFECTO
            Se interrumpe la actualización de la imagen, se procede con las
            demás actualizaciones y se emite el mensaje de error conrrespondiente}
	}

 \UCsection[design]{Datos de Diseño}

	\UCitems[design]{Casos de Prueba}{%
        \Titem \refElem{CPC-E02-1}
        \Titem \refElem{CPI-E02-1a}
        \Titem \refElem{CPI-E02-1b}
	}

 \UCsection[admin]{Datos de Administración de Requerimiento}

	\UCitem[admin]{Observaciones}{%
        Ninguna
	}

\end{UseCase}

\subsubsection{Trayectorias del caso de uso}

\begin{UCtrayectoria}%
  \includeUC{CU-M01} \refErr{Err1}

  \Actor Presiona la opción {\bf \refElem{tExpSettingsVisual}} en la categoría
         \refElem{tExpCategoria}.
  \Sistema Obtiene el valor de si el módulo de experiencia está \refElem[activado]%
           {xp-general-settings.activated} o no. \refTray{A}
  \Sistema Obtiene los valores actuales de la configuración:
           \salida{xp-visual-settings.title},
           \salida{xp-visual-settings.description},
           \salida{xp-visual-settings.message},
           \salida{xp-visual-settings.colorLvl} y
           \salida{xp-visual-settings.colorBar}.

  \Sistema Carga la pantalla \refElem{IU-E03} estableciendo como valores por defecto
           las \refElem{tExpSettingsVisual} actuales obtenidas.

  \Actor Ingresa los valores de \entrada{xp-visual-settings.title},
         \entrada{xp-visual-settings.description} y
         \entrada{xp-visual-settings.message} para los campos requeridos.
         \label{CU-E02-1.formulario}
  \Actor Ingresa los valores para el \entrada{xp-visual-settings.colorLvl} y
         \entrada{xp-visual-settings.colorBar} seleccionando el color y la
         tonalidad mediante el \refElem{tSelectColor}. \refTray{B}
         \label{CU-E02-1.color}
  \Actor Presiona la opción {\bf Seleccione un archivo}. \refTray{C}
  \Sistema Despliega la pantalla \refElem{IU-M00a} como pantalla emergente
           \label{CU-E02-1.seleccion-archivo}

  \Actor Selecciona la opción {\it Subir un archivo} en el menu izquierdo de la
         pantalla emergente y posteriormente presiona el botón {\it Browse}.
  \Actor Selecciona el archivo de la \entrada{xp-visual-settings.image}.
  \Actor Presiona el botón {\bf Subir este archivo}.
  \Sistema Valida que el archivo tenga alguna de las extensiones de indicadas
           por la regla \regla{BR-E01}. \refTray{D}
  \Sistema Cierra la pantalla emergente y muestra el nombre del archivo seleccionado
           en la pantalla \refElem{IU-E03}.


  \Actor Presiona el botón {\bf Guardar Cambios}. \refTray{E} \label{CU-E02-1.validacion}

  \Sistema Valida que las opciones ingresadas cumplan con las restricciones
           especificadas en el modelo de información. \refTray{F}.

  \Sistema Valida que la \refElem{xp-visual-settings.image} proporcionada cumpla con
           las restricciones de nombre de archivo establecida por la regla
           \refElem{BR-E01}. \refErr{Err2}
  \Sistema Remplaza la imagen de los niveles por la imagen propocionada por el usuario,
           actualiza los demás valores de las configuraciones. \refErr{Err3}
  \Sistema Despliega la pantalla \refElem{IU-E03} con el mensaje que indica que
           los datos han sido actualizados exitosamente.
\end{UCtrayectoria}

\begin{UCtrayectoriaA}{A}{El módulo de experiencia no se encuentra activado}

  \Sistema Carga la pantalla \refElem{IU-E03a}.

  \Actor Presiona el botón {\bf Activar módulo de experiencia}
  \includeUC{CU-E02} a partir del paso \ref{CU-E02-ir-a-formulario},
                     para activar el módulo de experiencia.

  \Sistema Regresa al inicio de la trayectoria principal.

\end{UCtrayectoriaA}

\begin{UCtrayectoriaA}{B}{%
El \refElem{aAdministrador} desea especificar el valor del \refElem{tColor}
directamente}

    \Actor Ingresa el valor hexadecimal del color para el
           \refElem{xp-visual-settings.colorLvl} o
           \refElem{xp-visual-settings.colorBar}.

    \Sistema Continua en el paso \ref{CU-E02-1.color} de la trayectoria principal.

\end{UCtrayectoriaA}

\begin{UCtrayectoriaA}{C}{%
No desea cambiar la \refElem{xp-visual-settings.image} actual de los niveles}

    \Sistema Continua en el paso \ref{CU-E02-1.validacion} de la trayectoria principal
\end{UCtrayectoriaA}

\begin{UCtrayectoriaA}{D}{Cuando el archivo seleccionado es distinto de {\it png o jpg}}

  \Sistema Emite en una ventana emergente el mensaje {\it Error: ``El tipo de
           archivo \$EXT no se acepta.''} siendo {\it\$EXT} la extensión del
           archivo seleccionado.
  \Sistema Regresa al paso \ref{CU-E02-1.seleccion-archivo}.

\end{UCtrayectoriaA}

\begin{UCtrayectoriaA}[Fin del caso de uso]{E}{%
Desea cancelar la actualización de las configuraciones}

  \Actor Presiona el botón cancelar.
  \Sistema Redirige a la pantala \refElem{IU-M01}.

\end{UCtrayectoriaA}

\begin{UCtrayectoriaA}{F}{%
Alguno de los datos ingresados no cumple con las restricciones en el modelo de
información}

    \Sistema Imprime los mensajes de error abajo de los campos con valores incorrectos.

    \Actor Ingresa nuevamente los valores para los campos marcados como incorrectos.
    \Sistema Regresa al paso \ref{CU-E02-1.formulario}

\end{UCtrayectoriaA}

 % Configurar visualización de niveles
    
% \ucstEnEdicion     Al terminar una revisión/aprobación con observaciones 
%                    y al inicio del CU.
%
% \ucstEnRevision    Al terminar la edición del CU (version += 0.1).
% \ucstEnAprobacion  Al pasar la revision sin observaciones.
% \ucstAprobado      Al ser aprobado por el usuario (version += 1.0)

\begin{UseCase}[%
Autor/Daniel Ortega,%
Version/0.1,%
Estado/\ucstEnEdicion]%
%
{CU-E02-2}{Configurar sistema de experiencia}{%
%
 Permite al \refElem{aAdministrador} establecer y modificar las cantidades de puntos
 de experiencia que brindan los cursos en la plataforma y la forma en que aumenta
 la cantidad de experiencia requerida para pasar de un nivel al siguiente. Se 
 definieron dos formas en la que la experiencia de los niveles se determina, ambas
 estan definidas por la regla de negocio \refElem{BR-E02}}

	\UCitem[control]{Revisor}{ Sin asignar }
	\UCitem[control]{Último cambio}{ \today }

 \UCsection{Atributos}

    \UCitem{Actor(es)}{%
        \refElem{aAdministrador}
    }

	\UCitems{Propósito}{%
        \Titem Permitir al administrador configurar el sistema de experiencia.
        \Titem Establecer o modificar la cantidad de experiencia que brindan los
               cursos.
        \Titem Establecer la cantidad de experiencia requerida para pasar el primer
               nivel usada como calculo para los demás niveles.
        \Titem Cambiar la forma en cómo se incrementa la cantidad de experiencia
               requerida para avanzar de un nivel a otro.
	}
	
	\UCitem{Entradas}{\imprimeUC{entrada}}

	\UCitems{Origen}{%
        \Titem Mouse
        \Titem Teclado
	}

	\UCitem{Salidas}{\imprimeUC{salida}}

	\UCitems{Destino}{%
		\Titem \refElem{IU-E04}
	}
	
	\UCitems{Precondiciones}{%
        \Titem Que los plugins del módulo de experiencia se encuentren instalados
        \Titem El módulo de experiencia debe estár habilitado en el caso de uso
               \refElem{CU-E02}.
	}

	\UCitem{Postcondiciones}{%
        Los nuevos valores de las \refElem{xp-scheme-settings} deber ser
        estár actualizados para todos los usuarios, además de persistirse en el
        sistema.
	}

	\UCitem{Reglas de negocio}{\imprimeUC{regla}}

	\UCitems{Errores}{%
        \Titem \UCerr{Err1}{%
        % CAUSA
            Los plugins del módulo de experiencia no se encuentran instalados,}{%
        % EFECTO
            no se presentan las opciones en el menú y por lo tanto no se puede
            acceder a las configuraciones}
	}

	% \UCitem{Viene de}{% Indicar si el Caso de uso es primario o se extiende de otro. La mayoría se 
					  % extienden de Login.
		% EJEMPLO: \refIdElem{PY-CU1} o Caso de uso primario.
	% 	\TODO Especificar.
	% }	

 \UCsection[design]{Datos de Diseño}

	\UCitems[design]{Casos de Prueba}{%
        \Titem \refElem{CPC-E02-2}
        \Titem \refElem{CPI-E02-2a}
        \Titem \refElem{CPI-E02-2b}
        \Titem \refElem{CPI-E02-2c}
	}

 \UCsection[admin]{Datos de Administración de Requerimiento}

	\UCitem[admin]{Observaciones}{%
        Ninguna
	}

\end{UseCase}

\subsubsection{Trayectorias del caso de uso}

\begin{UCtrayectoria}%
%
  \includeUC{CU-M01} \refErr{Err1}

  \Actor Presiona la opción {\bf \refElem{tExpSettingsComportamiento}} en la categoría
         \refElem{tExpCategoria}. \refTray{A} 
  \Sistema Obtiene el valor de si el módulo de experiencia está \refElem[activado]%
           {xp-general-settings.activated} o no. \refTray{B} \label{CU-E02-2-loading}
  \Sistema Obtiene los valores actuales de la configuración del sistema de experiencia:
           \salida{xp-scheme-settings.increment},
           \salida{xp-scheme-settings.incrementValue},
           \salida{xp-scheme-settings.levelXP} y
           \salida{xp-scheme-settings.courseXP}.
  \Sistema Carga la pantalla \refElem{IU-E04} estableciendo como valores por defecto
           las \refElem{xp-scheme-settings} obtenidas en el anterior paso.
  \Sistema Muestra la descripción de la regla \regla{BR-E03} para informar al 
           \refElem{aAdministrador} del comportamiento que tendrá el sistema en 
           caso de modificarse la cantidad de experiencia correspondiente al primer
           nivel.

  \Sistema Muestra la descripción de la regla \regla{BR-E04} para informar al 
           \refElem{aAdministrador} del comportamiento que tendrá el sistema si se 
           modificarse la cantidad de experiencia que otorgan los cursos.

  \Actor Especifica si el \entrada{xp-scheme-settings.increment} en la cantidad de
         experiencia de los niveles será {\it Lineal} o {\it Percentual}.
  \Actor Ingresa el valor para el \entrada{xp-scheme-settings.incrementValue} con
         base en la regla \regla{BR-E02}.
  \Actor Ingresa los valores para la \entrada{xp-scheme-settings.levelXP} y la
         \entrada{xp-scheme-settings.courseXP}.
  \Actor Presiona la opción {\bf Guardar Cambios}. \refTray{C} \label{CU-E02-2-submit}

  \Sistema Valida que los valores ingresados por el usuario cumplan con las
           restricciones especificadas en el modelo de información.
  \Sistema Verifica que el \refElem{xp-scheme-settings.incrementValue} cumpla
           con la regla \refElem{BR-E02}. \refTray{D}
  \Sistema Actualiza los valores de las \refElem{xp-scheme-settings} con los
           ingresados por el usuario.
  \Sistema Despliega la pantalla \refElem{IU-E04} con el mensaje de que los datos
           han sido actualizados exitosamente.

\end{UCtrayectoria}

\begin{UCtrayectoriaA}{A}{
El \refElem{aAdministrador} selecciona la categoría \refElem{tExpCategoria}}
  \Sistema Carga la pantalla \refElem{IU-E01}
  \Actor Regresa al paso \ref{CU-E02-2-loading}
\end{UCtrayectoriaA}

\begin{UCtrayectoriaA}{B}{
El módulo de experiencia no se encuentra activado}
  \Sistema Carga la pantalla \refElem{IU-E03a}.
  \Actor Presiona el botón {\bf Activar módulo de experiencia}
  \includeUC{CU-E02} a partir del paso \ref{CU-E02-ir-a-formulario},
                     para activar el módulo de experiencia.

  \Sistema Regresa al inicio de la trayectoria principal.

\end{UCtrayectoriaA}

\begin{UCtrayectoriaA}{C}{
El \refElem{aAdministrador} desea cancelar la modificación en el sistema de
experiencia}

  \Actor Presiona el botón {\bf Cancelar}.
  \Sistema Redirige a la pantalla \refElem{IU-M01}.
\end{UCtrayectoriaA}

\begin{UCtrayectoriaA}{D}{
Alguno de los valores ingresados por el usuario son incorrectos.}
  \Sistema Imprime los mensajes de error abajo de los campos con los valores
           incorrectos.
  \Actor Ingresa nuevamente los valores en los campos marcados como incorrectos.
  \Sistema Regresa al paso \ref{CU-E02-2-submit}.

\end{UCtrayectoriaA}
 % Configurar esquema de experiencia
    
% \ucstEnEdicion     Al terminar una revisión/aprobación con observaciones 
%                    y al inicio del CU.
%
% \ucstEnRevision    Al terminar la edición del CU (version += 0.1).
% \ucstEnAprobacion  Al pasar la revision sin observaciones.
% \ucstAprobado      Al ser aprobado por el usuario (version += 1.0)

\begin{UseCase}[%
Autor/Daniel Ortega,%
Version/0.1,%
Estado/\ucstEnEdicion]%
%
{CU-E02-3}{Configurar esquema de experiencia}{%
%
 Permite al \refElem{aActor} .}

	\UCitem[control]{Revisor}{ Sin asignar }
	\UCitem[control]{Último cambio}{ \today }

 \UCsection{Atributos}

    \UCitem{Actor(es)}{%
        \refElem{aActor}
    }

	\UCitem{Propósito}{%
        ...
	}
	
	\UCitem{Entradas}{\imprimeUC{entrada}}

	\UCitems{Origen}{%
        \Titem Mouse
        \Titem Teclado
	}

	\UCitem{Salidas}{\imprimeUC{salida}}

	\UCitems{Destino}{%
		\Titem \refElem{IU-M02a}
	}
	
	\UCitems{Precondiciones}{%
        \Titem ...
	}

	\UCitem{Postcondiciones}{%
        Ninguna
	}

	\UCitem{Reglas de negocio}{%
		Ninguna
	}

	\UCitems{Errores}{%
        \Titem \UCerr{Err1}{%
        % CAUSA
            ...,}{%
        % EFECTO
            ...}
	}

	% \UCitem{Viene de}{% Indicar si el Caso de uso es primario o se extiende de otro. La mayoría se 
					  % extienden de Login.
		% EJEMPLO: \refIdElem{PY-CU1} o Caso de uso primario.
	% 	\TODO Especificar.
	% }	

 \UCsection[design]{Datos de Diseño}

	\UCitems[design]{Casos de Prueba}{%
        \Titem \refElem{CPC-E0Y}
        \Titem \refElem{CPI-E0Y}
	}

 \UCsection[admin]{Datos de Administración de Requerimiento}

	\UCitem[admin]{Observaciones}{%
        Ninguna
	}

\end{UseCase}

\clearpage
\subsubsection{Trayectorias del caso de uso}

\begin{UCtrayectoria}%
%
 \Actor Presiona el botón \IUMenu en la esquina superior izquierda de la pantalla \refElem{IU-M01}
        para abrir el menu de navegación.

 \Actor Selecciona la opción {\it \IUAdminSitio Administración del sitio}

 \Sistema Carga la pantalla \refElem{IU-M02}

\end{UCtrayectoria}


\subsubsection{Puntos de extensión}

\UCExtensionPoint{Nombre del punto de extensión}{%

    El \refElem{aAdministrador} desea/requiere/necesita ....%
%
    }{En el paso \ref{CU-ET-1x} de la trayectoria principal  ...%
%
    }{\refElem{CU-E2-T}}

 % Configurar Eventos de experiencia
    
% \ucstEnEdicion     Al terminar una revisión/aprobación con observaciones 
%                    y al inicio del CU.
%
% \ucstEnRevision    Al terminar la edición del CU (version += 0.1).
% \ucstEnAprobacion  Al pasar la revision sin observaciones.
% \ucstAprobado      Al ser aprobado por el usuario (version += 1.0)

\begin{UseCase}[%
Autor/Daniel Ortega,%
Version/0.1,%
Estado/\ucstEnEdicion]%
%
{CU-E03}{Desinstalar plugins del esquema de experiencia}{%
%
 Permite al \refElem{aAdministrador} desinstalr el o los plugins correspondientes
 al módulo de experiencia cuando desee remover las funcionalidades que estos otorgan
 y eliminar los registros de los puntos de experiencia de los usuarios y cursos.}

	\UCitem[control]{Revisor}{ Sin asignar }
	\UCitem[control]{Último cambio}{ \today }

 \UCsection{Atributos}

    \UCitem{Actor(es)}{%
        \refElem{aAdministrador}
    }

	\UCitem{Propósito}{%
        Remover el módulo de experiencia junto con las funcionalidades que
        brinda y los datos generados por el mismo.
	}
	
	\UCitem{Entradas}{\imprimeUC{entrada}}

	\UCitems{Origen}{%
        \Titem Mouse
	}

	\UCitem{Salidas}{Ninguna}

	\UCitem{Destino}{%
		\refElem{IU-M03}
	}
	
	\UCitem{Precondiciones}{%
        Los plugins que se desean instalar deben estar previamente 
        instalados.
	}

	\UCitem{Postcondiciones}{%
        Los datos de generados por el esquema de experiencia deben ser 
        eliminados una vez concluida la desinstalación del o de los plugins.
	}

	\UCitem{Reglas de negocio}{Ninguna}

	\UCitems{Errores}{%
        \Titem \UCerr{Err1}{%
        % CAUSA
            El plugin a desinstalar no se encuentra instalado}{%
        % EFECTO
            no se puede proceder con la ejecución debido a que ya está 
            desinstalado}
	}

 \UCsection[design]{Datos de Diseño}

	\UCitems[design]{Casos de Prueba}{%
        \Titem \refElem{CPC-E03}
	}

 \UCsection[admin]{Datos de Administración de Requerimiento}

	\UCitem[admin]{Observaciones}{%
        Ninguna
	}

\end{UseCase}

\subsubsection{Trayectorias del caso de uso}

\begin{UCtrayectoria}%
%
  \includeUC{CU-M01}
  \Actor Presiona la opción {\bf Vista General de Plugins}.
  \Sistema Carga la pantalla \refElem{IU-M03}.

  \Actor Pide consultar únicamente los plugins adicionales instalados presionando
         la opción {\bf Plugins adicionales}..
  \Sistema Carga la pantalla \refElem{IU-M03a}.

  \Actor Presiona la opción {\bf Desinstalar} correspondiente al
         \entrada[plugin]{Plugin} del módulo de experiencia que desea desinstalar.
  \Sistema Muestra la pantalla \refElem{IU-M04}.
  \Sistema Pide la confirmación del usuario para continuar con la desinstalación.

  \Actor Presiona el botón de {\bf Aceptar}. \refTray{A}
  \Sistema ...
  \Sistema ...
  \Sistema ...
  \Sistema ...

  \Sistema Muestra la pantalla \refElem{IU-M04a} pidiendo al usuario la confirmación
           para eliminar los archivos del plugin. \refTray{B}
  \Sistema Elimina los archivos correspondientes al plugin.

\end{UCtrayectoria}

\begin{UCtrayectoriaA}{A}{%
El \refElem{aAdministrador} desea cancelar la desinstalación del plugin
}
\end{UCtrayectoriaA}

\begin{UCtrayectoriaA}{B}{%
El \refElem{aAdministrador} no desea eliminar los archivos del plugin}
\end{UCtrayectoriaA}

   % Desinstalar plugin del esquema de experiencia
    
% \ucstEnEdicion     Al terminar una revisión/aprobación con observaciones
%                    y al inicio del CU.
%
% \ucstEnRevision    Al terminar la edición del CU (version += 0.1).
% \ucstEnAprobacion  Al pasar la revision sin observaciones.
% \ucstAprobado      Al ser aprobado por el usuario (version += 1.0)

\begin{UseCase}[%
Autor/Daniel Ortega,%
Version/0.1,%
Estado/\ucstEnEdicion]%
%
{CU-E04}{Crear curso con experiencia}{%
%
 Permite al \refElem{aProfesor} .}

	\UCitem[control]{Revisor}{ Sin asignar }
	\UCitem[control]{Último cambio}{ \today }

 \UCsection{Atributos}

    \UCitem{Actor(es)}{%
        \refElem{aProfesor}, \refElem{aAdministrador}.
    }

	\UCitem{Propósito}{%
        Crear un curso con soporte para brindar puntos de experiencia conforme
        van completando las secciones del curso para que los estudiantes los acumulen
        y para subir de nivel.
	}

	\UCitem{Entradas}{\imprimeUC{entrada}}

	\UCitems{Origen}{%
        \Titem Mouse
        \Titem Teclado
	}

	\UCitem{Salidas}{\imprimeUC{salida}}

	\UCitem{Destino}{%
		\refElem{IU-E06a}
	}

	\UCitems{Precondiciones}{%
        \Titem Los plugins correspondientes al módulo de experiencia deben de estar
               habilitados.
        \Titem El módulo de experiencia debe de estar habilitado por el
               \refElem{aAdministrador}.
	}

	\UCitem{Postcondiciones}{%
        Las configuraciones particulares para este curso deben ser almacenadas
        en el sistema.
	}

	\UCitem{Reglas de negocio}{\imprimeUC{regla}}

	\UCitems{Errores}{%
        \Titem \UCerr{Err1}{%
        % CAUSA
            Los plugins correspondientes al módulo de experiencia no se encuentran
            instalados,}{%
        % EFECTO
            no aparece la opción para el formato gamificado, termina el caso de uso.}

        \Titem \UCerr{Err2}{%
        % CAUSA
            El módulo de experiencia se encuentra deshabilitado en la plataforma,}{%
        % EFECTO
            se le solicita al \refElem{aAdministrador} que habilite la experiencia}
	}

	% \UCitem{Viene de}{% Indicar si el Caso de uso es primario o se extiende de otro. La mayoría se
					  % extienden de Login.
		% EJEMPLO: \refIdElem{PY-CU1} o Caso de uso primario.
	% 	\TODO Especificar.
	% }

 \UCsection[design]{Datos de Diseño}

	\UCitems[design]{Casos de Prueba}{%
        \Titem \refElem{CPC-E04} % Creado chingon
        \Titem \refElem{CPI-E04} % Módulo de experiencia dehabilitado
	}

 \UCsection[admin]{Datos de Administración de Requerimiento}

	\UCitem[admin]{Observaciones}{%
        Ninguna
	}

\end{UseCase}

\subsubsection{Trayectorias del caso de uso}

\begin{UCtrayectoria}%
  \includeUC{CU-M01} presionando la pestaña {\bf Cursos}

  \Actor Presiona la opción {\bf Gestionar cursos y categorías}.
  \Sistema Obtiene las \refElem[categorias]{mdl-course-category} y
           \refElem{mdl-course.fullname} de los cursos presentes en la plataforma.
  \Sistema Muestra la pantalla \refElem{IU-M06}.

  \Actor Selecciona el botón nuevo curso de la categoría seleccionada por defecto.
  \Sistema Muestra la pantalla \refElem{IU-E06}.

  \Actor Introduce el \entrada{mdl-course.fullname}, \entrada{mdl-course.shortname} y
         la categoría a la cual pertenecerá el curso.

  \Actor Despliega la sección para el formato del curso.
  \Actor Selecciona el \entrada{xp-course.format} de curso gamificado {\it(gamedle)}.
         \refErr{Err1} \label{CU-E04-format}

  \Actor Especifica las opciones \entrada{xp-course.sections},
         \entrada{xp-course.hiddensections} y  \entrada{xp-course.coursedisplay}
         del formato de curso gamificado. \refTray{A} \refErr{Err2}

  \Actor Opcionalmente incluye valores en los demás campos opcionales del formulario.
  \Actor Presiona el botón {\bf Guardar cambios y mostrar}. \refTray{B} \refTray{C}
         \refTray{D} \label{CU-E04-submit}.

  \Sistema Obtiene los datos ingresados por el usuario.
  \Sistema Crea un curso \salida{mdl-course} junto con la cantidad de
           \salida[secciones]{mdl-course-section} especificadas por el usuario.
  \Sistema Crea un \salida{xp-course} con los valores de  \salida{xp-course.hiddensections}
           y \salida{xp-course.coursedisplay} ingresados por el usuario.
  \Sistema Por cada una de las \salida[secciones]{xp-course.sections} del curso gamificado
           crea las \salida[secciones gamificadas]{xp-course-section} correspondientes.
  \Sistema Establece la \salida{xp-course-section.xp} de cada seccion del curso con base
           en la regla de negocio \regla{BR-E08}. \label{CU-E04-finish}
  \Sistema Muestra la pantalla con \refElem{IU-E06a} con los datos del curso recién
           creado.

\end{UCtrayectoria}

\begin{UCtrayectoriaA}{A}{%
El módulo de experiencia se encuentra deshabilitado
}
  \Sistema Muestra el mensaje informando que el módulo de experiencia está
           deshabilitado.
  \Actor Si es que desea continuar y crear un curso gamificado sin la opción
         de brindar experiencia, prosigue en el paso \ref{CU-E04-format}

\end{UCtrayectoriaA}

\begin{UCtrayectoriaA}{B}{%
Alguno de los campos introducidos por el usuario es erróneo.
}

  \Sistema Muestra los mensajes de error correspondientes en los campos que
           contienen datos inválidos.
  \Actor Ingresa de nuevo los campos que contiene errores.
  \Sistema Regresa al paso \ref{CU-E04-submit}

\end{UCtrayectoriaA}

\begin{UCtrayectoriaA}{C}{%
El usuario desea regresar a la pantalla \refElem{IU-M06}.
}

  \Actor Presiona el botón {\bf Guardar y regresar}
  \Sistema Ejecuta los pasos a partir del paso \ref{CU-E04-submit} hasta el
           paso \ref{CU-E04-finish}.
  \Sistema Redirige a la pantalla \refElem{IU-M06}

\end{UCtrayectoriaA}

\begin{UCtrayectoriaA}{D}{%
El usuario desea cancelar la creación del curso
}

  \Actor Presiona el botón {\bf Cancelar}
  \Sistema Redirige a la pantalla \refElem{IU-M06}

\end{UCtrayectoriaA}
   % Crear un curso gamificado
    
% \ucstEnEdicion     Al terminar una revisión/aprobación con observaciones 
%                    y al inicio del CU.
%
% \ucstEnRevision    Al terminar la edición del CU (version += 0.1).
% \ucstEnAprobacion  Al pasar la revision sin observaciones.
% \ucstAprobado      Al ser aprobado por el usuario (version += 1.0)

\begin{UseCase}[%
Autor/Daniel Ortega,%
Version/0.1,%
Estado/\ucstEnEdicion]%
%
{CU-E05}{Habilitar el soporte para experiencia en un curso}{%
%
 Permite al \refElem{aProfesor} incluir el soporte para que las secciones del curso
 que administra brinden experiencia a los alumnos conforme estos las vayan completando.}

	\UCitem[control]{Revisor}{ Sin asignar }
	\UCitem[control]{Último cambio}{ \today }

 \UCsection{Atributos}

    \UCitem{Actor(es)}{%
        \refElem{aProfesor}
    }

	\UCitem{Propósito}{%
        Agregar soporte para brindar experiencia en cualquier curso que haya sido creado
        en moodle.
	}
	
	\UCitem{Entradas}{\imprimeUC{entrada}}

	\UCitems{Origen}{%
        \Titem Mouse
	}

	\UCitem{Salidas}{\imprimeUC{salida}}

	\UCitem{Destino}{%
		\refElem{IU-E06a}
	}
	
	\UCitems{Precondiciones}{%
        \Titem Los plugins correspondientes al módulo de experiencia deben de estar
               habilitados.
	}

	\UCitem{Postcondiciones}{%
        \Titem Se debe de brindar la experiencia correspondiente de las secciones que
               hayan completado los alumnos inscritos en el curso.
	}

	\UCitem{Reglas de negocio}{\imprimeUC{regla}}

	\UCitems{Errores}{%
        \Titem \UCerr{Err1}{%
        % CAUSA
            Los plugins corespondientes al módulo de experiencia no se encuentran
            instalados,}{%
        % EFECTO
            no aparece la opción para cambiar el formato del curso al formato gamificado}

        \Titem \UCerr{Err2}{%
        % CAUSA
            El módulo de experiencia se encuentra deshabilitado en la plataforma,}{%
        % EFECTO
            se le solicita al \refElem{aAdministrador} que habilite la experiencia}
    }

	% \UCitem{Viene de}{% Indicar si el Caso de uso es primario o se extiende de otro. La mayoría se 
					  % extienden de Login.
		% EJEMPLO: \refIdElem{PY-CU1} o Caso de uso primario.
	% 	\TODO Especificar.
	% }	

 \UCsection[design]{Datos de Diseño}

	\UCitems[design]{Casos de Prueba}{%
        \Titem \refElem{CPC-E05}
        \Titem \refElem{CPC-E05a}
	}

 \UCsection[admin]{Datos de Administración de Requerimiento}

	\UCitem[admin]{Observaciones}{%
        Ninguna
	}

\end{UseCase}

\clearpage
\subsubsection{Trayectorias del caso de uso}

\begin{UCtrayectoria}%
%
 \includeUC{CU-M01} presionando la pestaña {\bf Cursos}.

  \Actor Presiona la opción {\bf Gestionar cursos y categorías}.
  \Sistema Obtiene las \refElem[categorias]{mdl-course-category} y 
           \refElem{mdl-course.fullname} de los cursos presentes en la plataforma.
  \Sistema Muestra en la pantalla \refElem{IU-M06} la lista de cursos. \refTray{A}.
           \label{CU-E05-course-list}

  \Actor Presiona el botón  

\end{UCtrayectoria}

\begin{UCtrayectoriaA}{A}{%
El curso al cual el \refElem{aProfesor} desea brindar soporte para experiencia 
se encuentra en una categoria distinta a la mostrada por defecto.
}
  \Actor Selecciona la \refElem{course.category} a la que pertenece el curso
         que desea agregarle experiencia.

  \Sistema Obtiene las \refElem[categorias]{mdl-course-category} y 
           \refElem{mdl-course.fullname} de los cursos presentes en la plataforma.

  \Sistema Redirige al paso \ref{CU-E05-course-list} de la trayectoria principal.

\end{UCtrayectoriaA}

   % Convertir un curso ordinario a uno gamificado
    
% \ucstEnEdicion     Al terminar una revisión/aprobación con observaciones 
%                    y al inicio del CU.
%
% \ucstEnRevision    Al terminar la edición del CU (version += 0.1).
% \ucstEnAprobacion  Al pasar la revision sin observaciones.
% \ucstAprobado      Al ser aprobado por el usuario (version += 1.0)

\begin{UseCase}[%
Autor/Daniel Ortega,%
Version/0.1,%
Estado/\ucstEnEdicion]%
%
{CU-E06}{Quitar el soporte de brindar experiencia a un curso gamificado}{%
%
 Permite al \refElem{aProfesor} quitar el soporte para brindar experiencia a un
 curso gamificado, se recomienda que para que no cambie la organización del curso
 se ocupe el formato de tópicos/temas debido a que en este formato esta basado el
 formato gamificado.}

	\UCitem[control]{Revisor}{ Sin asignar }
	\UCitem[control]{Último cambio}{ \today }

 \UCsection{Atributos}

    \UCitem{Actor(es)}{%
        \refElem{aProfesor}
    }

	\UCitem{Propósito}{%
        Permitir al profesor remover el soporte brindar puntos de experiencia de
        un curso que previamente gamificado.
	}
	
	\UCitem{Entradas}{\imprimeUC{entrada}}

	\UCitems{Origen}{%
        \Titem Mouse
	}

	\UCitem{Salidas}{\imprimeUC{salida}}

	\UCitem{Destino}{%
		\refElem{IU-M02a}
	}
	
	\UCitems{Precondiciones}{%
        \Titem El curso al que se le debe quitar el soporte de experiencia debe tener
               el formato gamificado.
        \Titem Los plugins del módulo de experiencia deben de estar instalados.
	}

	\UCitem{Postcondiciones}{%
        Ninguna
	}

	\UCitem{Reglas de negocio}{\imprimeUC{regla}}

	\UCitems{Errores}{%
        \Titem \UCerr{Err1}{%
        % CAUSA
            El curso elegido no es un curso gamificado con soporte para experiencia,}{%
        % EFECTO
            termina el caso de uso}
	}

	% \UCitem{Viene de}{% Indicar si el Caso de uso es primario o se extiende de otro. La mayoría se 
					  % extienden de Login.
		% EJEMPLO: \refIdElem{PY-CU1} o Caso de uso primario.
	% 	\TODO Especificar.
	% }	

 \UCsection[design]{Datos de Diseño}

	\UCitems[design]{Casos de Prueba}{%
        \Titem \refElem{CPC-E0Y}
        \Titem \refElem{CPI-E0Y}
	}

 \UCsection[admin]{Datos de Administración de Requerimiento}

	\UCitem[admin]{Observaciones}{%
        Ninguna
	}

\end{UseCase}

\clearpage
\subsubsection{Trayectorias del caso de uso}

\begin{UCtrayectoria}%
%
 \includeUC{CU-M01} presionando la pestaña {\bf Cursos}.

  \Actor Presiona la opción {\bf Gestionar cursos y categorías}.
  \Sistema Obtiene las \refElem[categorias]{mdl-course-category} y 
           \refElem{mdl-course.fullname} de los cursos presentes en la plataforma.
  \Sistema Muestra en la pantalla \refElem{IU-M06} la lista de cursos.
           \label{CU-E05-course-list}

  \Actor Presiona el botón \IUConfigurar del curso que desea editar.
  \Sistema Obtiene el \salida{mdl-course.fullname}, \salida{mdl-course.shortname} y
           \salida{mdl-course.format}.
  \Sistema Obtiene el valor de \salida{xp-course.hiddensections} y
           \salida{xp-course.coursedisplay}. \refErr{Err1}
  \Sistema Obtiene los demás datos del curso.
  \Sistema Muestra los datos obtenidos del curso en la pantalla \refElem{IU-E05}.

  \Actor Despliega la sección para el formato del curso.
  \Actor Selecciona UN \entrada{mdl-course.format} distinto al 
         \refElem[formato gamificado]{xp-course.format}.
  \Sistema Carga la información correspondiente al formato del curso elegido.

  \Actor Opcionalmente edita los demás campos opcionales del formulario.
  \Actor Presiona el botón {\bf Guardar Cambios y mostrar}. \refTray{B}, \refTray{C},
         \refTray{D} \label{CU-E05-submit}
  \Sistema Actualiza los datos del curso de moodle con los ingresados por el usuario.
  \Sistema Elimina el \entrada{xp-course} vinculado al \refElem{mdl-course} y las
           configuraciones del mismo.
  \Sistema Elimina las \entrada{xp-course-section} vinculadas al curso.
  \Sistema Elimina las \entrada[recompensas]{xp-section-reward} de las secciones del 
           curso sin alterar la experiencia recibida por los alumnos de acuerdo con 
           la regla \regla{BR-E02}.
  \Sistema Muestra la pantalla \refElem{IU-M06a}.

\end{UCtrayectoria}


\subsubsection{Puntos de extensión}

\UCExtensionPoint{Nombre del punto de extensión}{%

    El \refElem{aAdministrador} desea/requiere/necesita ....%
%
    }{En el paso \ref{CU-ET-1x} de la trayectoria principal  ...%
%
    }{\refElem{CU-E2-T}}

   % Convertir Curso gamificado -> curso ordinario
    
% \ucstEnEdicion     Al terminar una revisión/aprobación con observaciones 
%                    y al inicio del CU.
%
% \ucstEnRevision    Al terminar la edición del CU (version += 0.1).
% \ucstEnAprobacion  Al pasar la revision sin observaciones.
% \ucstAprobado      Al ser aprobado por el usuario (version += 1.0)

\begin{UseCase}[%
Autor/Daniel Ortega,%
Version/0.1,%
Estado/\ucstEnEdicion]%
%
{CU-E07}{Administrar experiencia de un curso}{%
%
 Permite al \refElem{aProfesor} establecer la cantidad de experiencia que cada una de las
 secciones de un curso gamificado brindará a los alunos cuando estos la hayan completado}

	\UCitem[control]{Revisor}{ Sin asignar }
	\UCitem[control]{Último cambio}{ \today }

 \UCsection{Atributos}

    \UCitem{Actor(es)}{%
        \refElem{aProfesor}
    }

	\UCitems{Propósito}{%
        \Titem Permitirle al profesor especificar la cantidad de experiencia que cada
               sección del curso brindará.
        \Titem Permitirle al profesor ponderar de acuerdo a su criterio que secciones
               del curso deben brindar mayor o menor experiencia.
	}
	
	\UCitem{Entradas}{\imprimeUC{entrada}}

	\UCitems{Origen}{%
        \Titem Mouse
        \Titem Teclado
	}

	\UCitem{Salidas}{\imprimeUC{salida}}

	\UCitem{Destino}{%
		\refElem{IU-E06b}
	}
	
	\UCitems{Precondiciones}{%
        \Titem Los plugins del módulo de experiencia deben de estar habilitados
        \Titem El curso debe debe de tener el \refElem{xp-course.format} gamificado.
	}

	\UCitem{Postcondiciones}{%
        Los valores de experiencia para cada sección del curso deben ser
        almacenados en el sistema.
	}

	\UCitem{Reglas de negocio}{\imprimeUC{regla}}

	\UCitems{Errores}{%
        \Titem \UCerr{Err1}{%
        % CAUSA
            El curso elegido no es un curso gamificado con soporte para experiencia,}{%
        % EFECTO
            no se puede administrar la experiencia y termina el caso de uso}
	}

	% \UCitem{Viene de}{% Indicar si el Caso de uso es primario o se extiende de otro. La mayoría se 
					  % extienden de Login.
		% EJEMPLO: \refIdElem{PY-CU1} o Caso de uso primario.
	% 	\TODO Especificar.
	% }	

 \UCsection[design]{Datos de Diseño}

	\UCitems[design]{Casos de Prueba}{%
        \Titem \refElem{CPC-E07}
        \Titem \refElem{CPC-E07a}
        \Titem \refElem{CPI-E07}
	}

 \UCsection[admin]{Datos de Administración de Requerimiento}

	\UCitem[admin]{Observaciones}{%
        Ninguna
	}

\end{UseCase}

\subsubsection{Trayectorias del caso de uso}

\begin{UCtrayectoria}%
%
  \Actor Presiona el botón \IUMenu de la pantalla \refElem{IU-M00}
  \Sistema Despliega el menú de navegación lateral
  \Actor Selecciona la opción \IUHome {\bf Página Inicial del Sitio}

  \Sistema Obtiene el \salida{mdl-course.fullname} de los cursos disponibles en la
           plataforma.

  \Sistema Muestra la lista de cursos disponibles en la pantalla \refElem{IU-M07}.

  \Actor Selecciona el \entrada{mdl-course} del cual desea administrar su experiencia.
  \Sistema Obtiene el \entrada{mdl-course.fullname}, \entrada{mdl-course.shortname}
           así como el \salida[secciones]{mdl-course-section.name} de las secciones 
           del curso junto con las \salida[actividades]{mdl-course-module} 
           correspondients a cada sección y el estado de \refElem[completitud]%
           {mdl-course-module.completionstate}.

  \Sistema Muestra los datos obtenidos en la pantalla \refElem{IU-M07}.
           \label{CU-E07-pantalla}

  \Actor Presiona el botón \IUAdminSitio en la parte superior izquierda de la pantalla
  \Sistema Muestra el menu desplegable de la administración del curso

  \Actor Presiona el botón \IUEditar {\bf Activar Edición}. \refErr{Err1}
  \Sistema Obtiene la \salida{xp-course-section.xp} de la secciones gamificadas del 
           curso y revisa hay \refElem[recompensas]{xp-section-reward} que hayan sido
           entregadas correspondientes a dicha sección.
  \Sistema Obtiene la cantidad total de \salida{xp-scheme-settings.courseXP}.
  \Sistema Muestra la experiencia de cada sección y el total de experiencia 
           en la pantalla \refElem{IU-E06b}.
  \Sistema Habilita o deshabilita los campos para editar la experiencia de cada 
           sección con base en la regla \regla{BR-E09}.

  \Actor Introduce la experiencia de las secciones de acuerdo con la regla
         \regla{BR-E10}. \refTray{A}
  \Actor Presiona el botón {\bf Guardar cambios}. \refTray{B} \label{CU-E07-submit}
  \Sistema Obtiene los valores de experiencia ingresados por el usuario
           correspondientes a las secciones del curso.
  \Sistema Valida que los valores de experiencia ingresados cumplan con las
           restricciones especificadas en el modelo de información. \refTray{C}
  \Sistema Valida que los valores de experiencia cumplan con la regla
           \refElem{BR-E10} \refTray{D}.
  \Sistema Actualiza los valores de experiencia correspondientes a cada sección.
           \label{CU-E07-finish}
  \Sistema Muestra la pantalla \refElem{IU-E06b} con la \refElem{xp-course-section.xp}
           actualizada de las secciones del curso.
           

\end{UCtrayectoria}

\begin{UCtrayectoriaA}{A}{%
Alguna de las secciones gamificadas ha sido completada por almenos un alumno.
}
  \Sistema Detecta que todas las \refElem{mdl-course-module} de una sección
           han sido completadas por al menos un \refElem{aAlumno}.
  \Sistema Deshabilita los campos de edición en todas las secciones que entre
           en el anterior supuesto.
  \Sistema Continua en el paso \ref{CU-E07-pantalla}.
\end{UCtrayectoriaA}

\begin{UCtrayectoriaA}{B}{%
El \refElem{aProfesor} desea distribuir uniformemente la experiencia del curso
disponible entre las distintas secciones de acuerdo con la regla.}

  \Actor Presiona el botón {\bf Distribuir uniformemente}
  \Sistema Divide la cantidad de experiencia entre las secciones editables
           de acuerdo con las reglas \refElem{BR-E10} y \regla{BR-E08}.
  \Sistema Continua en el paso \ref{CU-E07-finish} de la trayectoria principal.
\end{UCtrayectoriaA}

\begin{UCtrayectoriaA}{C}{%
Alguno de los campos para establecer la experiencia de las %
\refElem[secciones del curso]{xp-course-section} es incorrecto.}

  \Sistema Muestra el mensaje de error debajo de los campos con un valor de
           experiencia incorrecto.

  \Actor Ingresa de nuevo los valores de experiencia de los campos de experiencia
         incrrectos.
  \Sistema Regresa la paso \ref{CU-E07-submit} de la trayectoria principal
\end{UCtrayectoriaA}

\begin{UCtrayectoriaA}{D}{%
Los valores de experiencia introducidos por el \refElem{aProfesor} no cumplen con
la regla \refElem{BR-E09}.}

  \Sistema Muestra el mensaje de error de que la suma de la experiencia no es igual
           al total de experiencia del curso.
  \Actor Ingresa de nuevo los valores de experiencia correspondientes a cada sección
         del curso.
  \Sistema Regresa la paso \ref{CU-E07-submit} de la trayectoria principal
\end{UCtrayectoriaA}
   % Administrar Curso gamificado
    
% \ucstEnEdicion     Al terminar una revisión/aprobación con observaciones
%                    y al inicio del CU.
%
% \ucstEnRevision    Al terminar la edición del CU (version += 0.1).
% \ucstEnAprobacion  Al pasar la revision sin observaciones.
% \ucstAprobado      Al ser aprobado por el usuario (version += 1.0)

\begin{UseCase}[%
Autor/Daniel Ortega,%
Version/0.1,%
Estado/\ucstEnEdicion]%
%
{CU-E08}{Agregar sección con experiencia}{%
%
 Permite al \refElem{aProfesor} agregar una sección con soporte para experiencia
 en un curso gamificado y que pueda configurar la cantidad de experiencia que brinda
 dicha sección mediante el caso de uso \refElem{CU-E07}.}

	\UCitem[control]{Revisor}{ Sin asignar }
	\UCitem[control]{Último cambio}{ \today }

 \UCsection{Atributos}

    \UCitem{Actor(es)}{%
        \refElem{aProfesor}
    }

	\UCitems{Propósito}{%
        \Titem Permitirle al profesor agregar una sección con soporte para experiencia
               en curso gamificado.
        \Titem Permitirle al profesor tener la misma funcionalidad de agregar nuevas
               secciones en como en un curso normal en moodle.
	}

	\UCitem{Entradas}{\imprimeUC{entrada}}

	\UCitems{Origen}{%
        \Titem Mouse
	}

	\UCitem{Salidas}{\imprimeUC{salida}}

	\UCitem{Destino}{%
		\refElem{IU-E06b}
	}

	\UCitems{Precondiciones}{%
        \Titem El módulo de experiencia debe estar habilitado.
        \Titem El curso debe debe de tener el \refElem{xp-course.format} gamificado.
	}

	\UCitem{Postcondiciones}{%
        Los valores de experiencia para cada sección del curso deben ser
        almacenados en el sistema.
	}

	\UCitem{Reglas de negocio}{\imprimeUC{regla}}

	\UCitems{Errores}{%
        \Titem \UCerr{Err1}{%
        % CAUSA
            El curso elegido no es un curso gamificado con soporte para experiencia,}{%
        % EFECTO
            no se puede crear una sección con experiencia y termina el caso de uso}
	}

	% \UCitem{Viene de}{% Indicar si el Caso de uso es primario o se extiende de otro. La mayoría se
					  % extienden de Login.
		% EJEMPLO: \refIdElem{PY-CU1} o Caso de uso primario.
	% 	\TODO Especificar.
	% }

 \UCsection[design]{Datos de Diseño}

	\UCitems[design]{Casos de Prueba}{%
        \Titem \refElem{CPC-E08}
	}

 \UCsection[admin]{Datos de Administración de Requerimiento}

	\UCitem[admin]{Observaciones}{%
        Ninguna
	}

\end{UseCase}

\subsubsection{Trayectorias del caso de uso}

\begin{UCtrayectoria}%
%
  \Actor Presiona el botón \IUMenu de la pantalla \refElem{IU-M00}
  \Sistema Despliega el menú de navegación lateral
  \Actor Selecciona la opción \IUHome {\bf Página Inicial del Sitio}

  \Sistema Obtiene el \salida{mdl-course.fullname} de los cursos disponibles en la
           plataforma.

  \Sistema Muestra la lista de cursos disponibles en la pantalla \refElem{IU-M07}.

  \Actor Selecciona el \entrada{mdl-course} al cual desea agregar una sección con
         soporte de gamificación.
  \Sistema Obtiene el \entrada{mdl-course.fullname}, \entrada{mdl-course.shortname}
           así como el \salida[secciones]{mdl-course-section.name} de las secciones
           del curso junto con las \salida[actividades]{mdl-course-module}
           correspondientes a cada sección y el estado de \refElem[completitud]%
           {mdl-course-module-completion.completionstate}.

  \Sistema Muestra los datos obtenidos en la pantalla \refElem{IU-M07}.
           \label{CU-E07-pantalla}

  \Actor Presiona el botón \IUAdminSitio en la parte superior izquierda de la pantalla
  \Sistema Muestra el menú desplegable de la administración del curso

  \Actor Presiona el botón \IUEditar {\bf Activar Edición}. \refErr{Err1}
  \Sistema Obtiene la \salida{xp-course-section.xp} de la secciones gamificadas del
           curso y revisa hay \refElem[recompensas]{xp-section-reward} que hayan sido
           entregadas correspondientes a dicha sección.
  \Sistema Obtiene la cantidad total de \salida{xp-scheme-settings.courseXP}.
  \Sistema Muestra la experiencia de cada sección y el total de experiencia
           en la pantalla \refElem{IU-E06b}.
  \Sistema Habilita o deshabilita los campos para editar la experiencia de cada
           sección con base en la regla \regla{BR-E09}.

  \Actor Presiona el botón \IUAnadir {\bf Misión}.
  \Sistema Crea una sección \entrada{mdl-course-section} perteneciente al
           \refElem{mdl-course}.
  \Sistema Crea una sección \entrada{xp-course-section} asociada a la sección
           recientemente creada, indicando que dicha sección tiene cero puntos de
           \refElem{xp-course-section.xp}.
  \Sistema Muestra la pantalla \refElem{IU-E06b} con los datos de la nueva sección creada.
\end{UCtrayectoria}
   % Agregar seccion con experiencia
    %
% \ucstEnEdicion     Al terminar una revisión/aprobación con observaciones 
%                    y al inicio del CU.
%
% \ucstEnRevision    Al terminar la edición del CU (version += 0.1).
% \ucstEnAprobacion  Al pasar la revision sin observaciones.
% \ucstAprobado      Al ser aprobado por el usuario (version += 1.0)

\begin{UseCase}[%
Autor/Daniel Ortega,%
Version/0.1,%
Estado/\ucstEnEdicion]%
%
{CU-E09}{Eliminar sección con experiencia}{%
%
 Permite al \refElem{aProfesor} eliminar una seccion de un curso gamificado con soporte
 para experiencia con base en la regla \refElem{BR-E11}, la cual indica para eliminar una
 sección de un curso gamificado esta debe brindar 0 puntos de experiencia.}

	\UCitem[control]{Revisor}{ Sin asignar }
	\UCitem[control]{Último cambio}{ \today }

 \UCsection{Atributos}

    \UCitem{Actor(es)}{%
        \refElem{aProfesor}
    }

	\UCitems{Propósito}{%
        \Titem Permitirle al profesor eliminar una sección de un curso con soporte 
               para experiencia.
        \Titem Permitirle al profesor eliminar secciones del curso de manera controlada
               sin afectar la cantidad de experiencia total que brinda el curso.
	}
	
	\UCitem{Entradas}{\imprimeUC{entrada}}

	\UCitems{Origen}{%
        \Titem Mouse
	}

	\UCitem{Salidas}{\imprimeUC{salida}}

	\UCitem{Destino}{%
		\refElem{IU-E06b}
	}
	
	\UCitems{Precondiciones}{%
        \Titem El curso debe debe de tener el \refElem{xp-course.format} gamificado.
	}

	\UCitem{Postcondiciones}{%
        Las eliminación de las secciones de experiencia debe permanecer en el sistema.
	}

	\UCitem{Reglas de negocio}{\imprimeUC{regla}}

	\UCitems{Errores}{%
        \Titem \UCerr{Err1}{%
        % CAUSA
            El curso elegido no es un curso gamificado con soporte para experiencia,}{%
        % EFECTO
            termina el caso de uso.}
        \Titem \UCerr{Err2}{%
        % CAUSE
            La sección que se desea eliminar ya ha sido completada por alguno
            de los \refElem{aAlumno},}{%
        % EFECTO
            la opción de eliminar sección no se muestra, termina el caso de uso.}
	}

	% \UCitem{Viene de}{% Indicar si el Caso de uso es primario o se extiende de otro. La mayoría se 
					  % extienden de Login.
		% EJEMPLO: \refIdElem{PY-CU1} o Caso de uso primario.
	% 	\TODO Especificar.
	% }	

 \UCsection[design]{Datos de Diseño}

	\UCitems[design]{Casos de Prueba}{%
        \Titem \refElem{CPC-E08}
        \Titem \refElem{CPI-E08}
	}

 \UCsection[admin]{Datos de Administración de Requerimiento}

	\UCitem[admin]{Observaciones}{%
        Ninguna
	}

\end{UseCase}

\subsubsection{Trayectorias del caso de uso}

\begin{UCtrayectoria}%
%
  \Actor Presiona el botón \IUMenu de la pantalla \refElem{IU-M00}
  \Sistema Despliega el menú de navegación lateral
  \Actor Selecciona la opción \IUHome {\bf Página Inicial del Sitio}

  \Sistema Obtiene el \salida{mdl-course.fullname} de los cursos disponibles en la
           plataforma.

  \Sistema Muestra la lista de cursos disponibles en la pantalla \refElem{IU-M07}.

  \Actor Selecciona el \entrada{mdl-course} al cual desea agregar una sección con
         soporte de gamificación.
  \Sistema Obtiene el \entrada{mdl-course.fullname}, \entrada{mdl-course.shortname}
           así como el \salida[secciones]{mdl-course-section.name} de las secciones 
           del curso junto con las \salida[actividades]{mdl-course-module} 
           correspondients a cada sección y el estado de \refElem[completitud]%
           {mdl-course-module.completionstate}.

  \Sistema Muestra los datos obtenidos en la pantalla \refElem{IU-M07}.
           \label{CU-E07-pantalla}

  \Actor Presiona el botón \IUAdminSitio en la parte superior izquierda de la pantalla
  \Sistema Muestra el menu desplegable de la administración del curso

  \Actor Presiona el botón \IUEditar {\bf Activar Edición}. \refErr{Err1}
  \Sistema Obtiene la \salida{xp-course-section.xp} de la secciones gamificadas del 
           curso y revisa hay \refElem[recompensas]{xp-section-reward} que hayan sido
           entregadas correspondientes a dicha sección.
  \Sistema Obtiene la cantidad total de \salida{xp-scheme-settings.courseXP}.
  \Sistema Muestra la experiencia de cada sección y el total de experiencia 
           en la pantalla \refElem{IU-E06b}.
  \Sistema Habilita o deshabilita los campos para editar la experiencia de cada 
           sección con base en la regla \regla{BR-E09}.

  \Actor Presiona la opción {\bf Editar} correspondiente a la \entrada{xp-course-section}
         seccion que desea eliminar. \label{CU-E09-options}.

  \Sistema Muestra el menu acciones de edición que se pueden realizar a dicha sección de
           acuerdo con la regla \regla{BR-E11}. \refTray{A} \refErr{Err2}

  \Actor Selecciona la opción \IUEliminar {\bf Eliminar sección}.

  \Sistema Elimina la \refElem{mdl-course-section} especificada, así como la
           \refElem{xp-course-section}.
  \Sistema Muestra la pantalla \refElem{IU-E06b} con los datos de las secciones restantes
           del curso.

\end{UCtrayectoria}

\begin{UCtrayectoriaA}{A}{%
La cantidad de \refElem{xp-course-section.xp} de la sección a eliminar es distinta de cero.
}
  \Sistema Redirige al paso \ref{CU-E09-options} de la trayectoria principal.
\end{UCtrayectoriaA}

   % Eliminar seccion con experiencia
    
% \ucstEnEdicion     Al terminar una revisión/aprobación con observaciones
%                    y al inicio del CU.
%
% \ucstEnRevision    Al terminar la edición del CU (version += 0.1).
% \ucstEnAprobacion  Al pasar la revision sin observaciones.
% \ucstAprobado      Al ser aprobado por el usuario (version += 1.0)

\begin{UseCase}[%
Autor/Daniel Ortega,%
Version/0.1,%
Estado/\ucstEnEdicion]%
%
{CU-E10}{Eliminar un curso gamificado}{%
%
 Permite al \refElem{aAdministrador} eliminar un curso gamificado. }

	\UCitem[control]{Revisor}{ Sin asignar }
	\UCitem[control]{Último cambio}{ \today }

 \UCsection{Atributos}

    \UCitem{Actor(es)}{%
        \refElem{aAdministrador}
    }

	\UCitem{Propósito}{%
        Permitir al administrador eliminar un curso que está gamificado junto
        con los datos de gamificación que este ha otorgado.
	}

	\UCitem{Entradas}{\imprimeUC{entrada}}

	\UCitems{Origen}{%
        \Titem Mouse
	}

	\UCitem{Salidas}{\imprimeUC{salida}}

	\UCitem{Destino}{%
		\refElem{IU-M02a}
	}

	\UCitems{Precondiciones}{%
        \Titem El curso que se desea eliminar no debe haber sido eliminado
               anteriormente.
	}

	\UCitem{Postcondiciones}{%
        Ninguna
	}

	\UCitem{Reglas de negocio}{\imprimeUC{regla}}

	\UCitems{Errores}{%
        \Titem \UCerr{Err1}{%
        % CAUSA
            El curso que se desea eliminar ha sido eliminado previamente,}{%
        % EFECTO
            termina el caso de uso.}
	}

	% \UCitem{Viene de}{% Indicar si el Caso de uso es primario o se extiende de otro. La mayoría se
					  % extienden de Login.
		% EJEMPLO: \refIdElem{PY-CU1} o Caso de uso primario.
	% 	\TODO Especificar.
	% }

 \UCsection[design]{Datos de Diseño}

	\UCitems[design]{Casos de Prueba}{%
        \Titem \refElem{CPC-E05}
        \Titem \refElem{CPC-E05a}
	}

 \UCsection[admin]{Datos de Administración de Requerimiento}

	\UCitem[admin]{Observaciones}{%
        Ninguna
	}

\end{UseCase}

\clearpage
\subsubsection{Trayectorias del caso de uso}

\begin{UCtrayectoria}%
%
  \Actor Presiona la opción {\bf Gestionar cursos y categorías}.
  \Sistema Obtiene las \salida[categorias]{mdl-course-category} y
           \salida{mdl-course.fullname} de los cursos presentes en la plataforma.
  \Sistema Muestra la pantalla \refElem{IU-M06}. \refErr{Err1}

  \Actor Presiona el botón \IUEliminar correspondiente al curso que desea eliminar.
         \label{CU-E05-delete-button}. \refTray{A}
  \Sistema Despliega la pantalla \refElem{IU-M06a} pidiendo la confirmación del
           usuario para continuar con la eliminación.

  \Actor Presiona la opción {\bf Eliminar}. \refTray{B}
  \Sistema Elimina las \entrada{xp-section-reward} de experiencia que se han
           entregado respetando la regla \regla{BR-E02}.
  \Sistema Elimina las \entrada[secciones gamificadas]{xp-course-section} del curso.
  \Sistema Elimina las opciones del \entrada{xp-course.format}.
  \Sistema Procede con las demás tareas de eliminación para concluir con la
           eliminación del curso.
  \Sistema Muestra la pantalla \refElem{IU-M06b}.

  \Actor Presiona el botón {\bf Continuar}.
  \Sistema Obtiene las \refElem[categorias]{mdl-course-category} y
           \refElem{mdl-course.fullname} de los cursos restantes en la plataforma.
  \Sistema Muestra la pantalla \refElem{IU-M06}.
\end{UCtrayectoria}

\begin{UCtrayectoriaA}{A}{%
El \refElem{mdl-course} que el \refElem{aAdministrador} desea eliminar se encuentra
en otra categoría diferente a la seleccionada por defecto.}

  \Actor Selecciona la \refElem{mdl-course-category} donde se encuentra el curso
         que desea eliminar.

  \Sistema Obtiene las \refElem{mdl-course.fullname} de los cursos pertenecientes a
           dichas categorías.
  \Sistema Regresa al paso \ref{CU-E05-delete-button}.
\end{UCtrayectoriaA}

\begin{UCtrayectoriaA}{B}{%
El \refElem{aAdministrador} desea cancelar la eliminación del curso.}

  \Actor Presiona el botón {\bf Cancelar}
  \Sistema Obtiene las \refElem[categorias]{mdl-course-category} y
           \refElem{mdl-course.fullname} de los cursos presentes en la plataforma.
\end{UCtrayectoriaA}
   % Eliminar un curso gamificado
    %\input{modulos/exp/CU/CU-E11}   % Recibir Experiencia                                  %RICARDO: Por qué está comentado este caso de uso?
    
% \ucstEnEdicion     Al terminar una revisión/aprobación con observaciones
%                    y al inicio del CU.
%
% \ucstEnRevision    Al terminar la edición del CU (version += 0.1).
% \ucstEnAprobacion  Al pasar la revision sin observaciones.
% \ucstAprobado      Al ser aprobado por el usuario (version += 1.0)

\begin{UseCase}[%
Autor/Daniel Ortega,%
Version/0.1,%
Estado/\ucstEnEdicion]%
%
{CU-E12}{Crear cuenta de usuario gamificado}{%
%
 Permite al módulo de experiencia recibir el evento, emitido por moodle, de cuando
 un usuario es creado, con base en este evento el módulo de experiencia asignará los
 datos por defecto de un usuario gamificado al usuario que se acaba de crear.}

	\UCitem[control]{Revisor}{ Sin asignar }
	\UCitem[control]{Último cambio}{ \today }

 \UCsection{Atributos}

    \UCitem{Actor(es)}{%
        \refElem{aAdministrador} o \refElem{aEstudiante}.
    }

	\UCitem{Propósito}{%
        Asociar una cuenta de un \refElem{xp-user} a un \refElem{mdl-user}
        cuando es creado por el administrador.
	}

	\UCitem{Entradas}{\imprimeUC{entrada}}

	\UCitems{Origen}{%
        \Titem Teclado
        \Titem Mouse
    }

	\UCitem{Salidas}{\imprimeUC{salida}}

	\UCitem{Destino}{\refElem{IU-M05a}}

	\UCitems{Precondiciones}{%
        \Titem Los plugins del módulo de experiencia deben de estar instalados.
        \Titem Cuando el actor es un \refElem{aEstudiante} \refTray{A}, la opción
               de permitir autoregistros debe estar habilitada y las
               configuraciones para poder realizar envío de correos mediante SMTP
               deben ser las correctas.
	}

	\UCitems{Postcondiciones}{%
        \Titem Los datos de un usuario gamificado deben permanecer vinculados con
               el usuario de moodle que se creó.
	}

	\UCitem{Reglas de negocio}{\imprimeUC{regla}}

	\UCitems{Errores}{%
        \Titem \UCerr{Err1}{%
        % CAUSA
            Los plugins del módulo de experiencia no se encuentran instalados y}{%
        % EFECTO
            no se puede crear un usuario gamificado, termina el caso de uso}
        \Titem \UCerr{Err2}{%
        % CAUSA
            La opción de creación de cuentas mediante autoregistro se encuentra
            deshabilitada}{%
        % EFECTO
            no se muestra la opción para que el estudiante cree su propia contraseña,
            termina el caso de uso.}
        \Titem \UCerr{Err3}{%
        % CAUSA
            La configuración de envío de correos para autoregistrarse no permite
            enviar correos}{%
        % EFECTO
            no se puede enviar el correo de confirmación, pero la cuenta ha sido
            creada, el \refElem{aEstudiante} contacta con soporte para que validen su
            cuenta manualmente}
	}

	% \UCitem{Viene de}{% Indicar si el Caso de uso es primario o se extiende de
    % otro. La mayoría se extienden de Login.
		% EJEMPLO: \refIdElem{PY-CU1} o Caso de uso primario.
	% 	\TODO Especificar.
	% }

 \UCsection[design]{Datos de Diseño}

	\UCitems[design]{Casos de Prueba}{%
        \Titem \refElem{CPC-E12}
        \Titem \refElem{CPC-E12a}
	}

 \UCsection[admin]{Datos de Administración de Requerimiento}

	\UCitem[admin]{Observaciones}{%
        Ninguna
	}

\end{UseCase}

\subsubsection{Trayectorias del caso de uso}

\begin{UCtrayectoria}%
%
  \includeUC{CU-M01} presionando la pestaña {\bf Usuarios}. \refTray{A}
  \Actor Presiona la opción {\bf Agregar un usuario} de la pantalla \refElem{IU-M01b}.
  \Sistema Carga la pantalla \refElem{IU-M05}.
  \Actor Ingresa el \entrada{mdl-user.username}, \entrada{mdl-user.password},
         \entrada{mdl-user.firstname} y \entrada{mdl-user.lastname}, que son los
         atributos requeridos para el nuevo usuario. \label{IU-E12-input-data}

  \Actor Opcionalmente ingresa los valores de los demás campos.

  \Actor Presiona el botón de {\bf Crear Usuario}. \refTray{B} \refTray{C}
         \label{CU-E12-submit}.

  \Sistema Obtiene lo valores ingresados para el nuevo usuario y crea la cuenta de un
           \refElem{mdl-user}.

  \Sistema Vincula los datos de un \salida{xp-user} con el \refElem{mdl-user} obtenido,
           estableciendo los valores iniciales para el \refElem{xp-user.level},
           \refElem{xp-user.levelxp} y \refElem{xp-user.xp} de acuerdo con la regla
           \regla{BR-E07}. \refErr{Err1} \label{IU-E12}

  \Sistema Obtiene el \salida{mdl-user.firstname}, \salida{mdl-user.lastname},
           \salida{mdl-user.email}, \salida{mdl-user.lastaccess},
           \salida{mdl-user.city} y \salida{mdl-user.country} de los usuarios
           presentes en moodle.

  \Sistema Despliega la información de los usuarios en la pantalla \refElem{IU-M05a}.
\end{UCtrayectoria}


\begin{UCtrayectoriaA}{A}{%
El usuario que ejecuta este caso de uso es el \refElem{aEstudiante} para registrarse
así mismo.
}
  \Actor Se encuentra en la pantalla \refElem{IU-M00b}. \refErr{Err2}
  \Actor Presiona el botón {\bf Comience ahora creando una cuenta}.
  \Sistema Carga la pantalla \refElem{IU-M05c}.
  \Sistema Ejecuta del paso \ref{IU-E12-input-data} al
  \Sistema Envía el correo de confirmación al \refElem{aEstudiante}. \refErr{Err3}
  \Actor Revisa su correo y accede al enlace para confirmar el registro de su cuenta.
\end{UCtrayectoriaA}

\begin{UCtrayectoriaA}{B}{%
Algunos de los campos ingresados por el \refElem{aAdministrador} son erróneos
}
  \Sistema Imprime los mensaje de error debajo de los campos con valores incorrectos.
  \Actor Ingresa nuevamente los valores en los campos marcados como incorrectos.
  \Sistema Regresa al paso \ref{CU-E12-submit}.
\end{UCtrayectoriaA}


\begin{UCtrayectoriaA}{C}{%
El \refElem{aAdministrador} desea cancelar la creación de un nuevo usuario
}
  \Actor Presiona el botón {\bf Cancelar}.
  \Sistema Redirige a la pantalla \refElem{IU-M01b}
\end{UCtrayectoriaA}
   % Crear cuenta de un usuario gamificado
    
% \ucstEnEdicion     Al terminar una revisión/aprobación con observaciones
%                    y al inicio del CU.
%
% \ucstEnRevision    Al terminar la edición del CU (version += 0.1).
% \ucstEnAprobacion  Al pasar la revision sin observaciones.
% \ucstAprobado      Al ser aprobado por el usuario (version += 1.0)

\begin{UseCase}[%
Autor/Daniel Ortega,%
Version/0.1,%
Estado/\ucstEnEdicion]%
%
{CU-E13}{Eliminar usuario gamificado}{%
%
 Permite al \refElem{aAdministrador} que cuando se elimine a algún estudiante de moodle
 también se eliminen los datos correspondientes al módulo de experiencia de dicho
 usuario.}

	\UCitem[control]{Revisor}{ Sin asignar }
	\UCitem[control]{Último cambio}{ \today }

 \UCsection{Atributos}

    \UCitem{Actor(es)}{%
        \refElem{aAdministrador}
    }

	\UCitem{Propósito}{%
        Eliminar los datos de experiencia de un usuario gamificado cuando se vaya al
        usuario de moodle vinculado a este.
	}

	\UCitem{Entradas}{\imprimeUC{entrada}}

	\UCitems{Origen}{%
        \Titem Mouse
	}

	\UCitem{Salidas}{\imprimeUC{salida}}

	\UCitem{Destino}{%
		\refElem{IU-M05a}
	}

	\UCitems{Precondiciones}{%
        \Titem Los plugins correspondientes al módulo de experiencia deben de estar
               instalados.
        \Titem El \refElem{mdl-user} que se desea eliminar no debe haber sido
               eliminado previamente.
	}

	\UCitem{Postcondiciones}{%
        \Titem Los datos de experiencia vinculados al \refElem{xp-user} deben de ser
               removidos del sistema.
	}

	\UCitem{Reglas de negocio}{\imprimeUC{regla}}

	\UCitems{Errores}{%
        \Titem \UCerr{Err1}{%
        % CAUSA
            Los plugins correspondientes al módulo de experiencia no se encuentran
            instalados,}{%
        % EFECTO
            continúaen el penúltimo paso de la trayectoria principal}

        \Titem \UCerr{Err2}{%
        % CAUSA
            El usuario que se desea eliminar ha sido eliminado previamente,}{%
        % EFECTO
            no se muestra la entrada del estudiante en la pantalla \refElem{IU-M05a},
            termina el caso de uso}
	}

	% \UCitem{Viene de}{% Indicar si el Caso de uso es primario o se extiende de otro. La mayoría se
					  % extienden de Login.
		% EJEMPLO: \refIdElem{PY-CU1} o Caso de uso primario.
	% 	\TODO Especificar.
	% }

 \UCsection[design]{Datos de Diseño}

	\UCitems[design]{Casos de Prueba}{%
        \Titem \refElem{CPC-E13}
	}

 \UCsection[admin]{Datos de Administración de Requerimiento}

	\UCitem[admin]{Observaciones}{%
        Ninguna
	}

\end{UseCase}

\subsubsection{Trayectorias del caso de uso}

\begin{UCtrayectoria}%
%
  \includeUC{CU-M01} presionando la pestaña {\bf Usuarios}

  \Actor Presiona la opción {\bf Mirar lista de usuarios}
  \Sistema Obtiene el \salida{mdl-user.firstname}, \salida{mdl-user.lastname},
           \salida{mdl-user.email}, \salida{mdl-user.lastaccess},
           \salida{mdl-user.city} y \salida{mdl-user.country} de los usuarios
           presentes en moodle.

  \Sistema Despliega la información de los usuarios en la pantalla \refElem{IU-M05a}.

  \Actor Presiona el botón \IUEliminar correspondiente al usuario que desea eliminar.
         \refErr{Err2}.

  \Sistema Pide la confirmación para eliminar al usuario seleccionado mediante la
           pantalla \refElem{IU-M05b}.
  \Actor Presiona el botón {\bf Eliminar} \refTray{A}

  \Sistema Elimina los datos de experiencia obtenidos en los cursos gamificados en
           los que el usuario ha sido \entrada{xp-section-reward}.

  \Sistema Elimina los datos del \entrada{xp-user}.

  \Sistema Elimina al \entrada{mdl-user}.

  \Sistema Obtiene el \refElem{mdl-user.firstname}, \refElem{mdl-user.lastname},
           \refElem{mdl-user.email}, \refElem{mdl-user.lastaccess},
           \refElem{mdl-user.city} y \refElem{mdl-user.country} de los usuarios
           presentes en moodle.

  \Sistema Despliega la información de los usuarios en la pantalla \refElem{IU-M05a}.


\end{UCtrayectoria}

\begin{UCtrayectoriaA}{A}{%
El \refElem{aAdministrador} desea cancelar la eliminación del usuario seleccionado
}%

  \Actor Presiona el botón {\bf Cancelar}
  \Sistema Obtiene el \refElem{mdl-user.firstname}, \refElem{mdl-user.lastname},
           \refElem{mdl-user.email}, \refElem{mdl-user.lastaccess},
           \refElem{mdl-user.city} y \refElem{mdl-user.country} de los usuarios
           presentes en moodle.
  \Sistema Redirige a la pantalla \refElem{IU-M05a}.
\end{UCtrayectoriaA}
   % Eliminar cuenta de un usuario gamificado

% =========================================================
\clearpage
\subsection{Diseño}

 En esta sección se presenta los distintos diagramas que ayudan a modelar el
 software a desarrollar, de la misma forma todas las decisiones y consideraciones
 relevantes para la implementación de las funcionalidades se encuentra documentada
 en esta sección.

    
\subsection*{Interfaces de Moodle}

 Como primera instancia se presentan las tanto las interfaces utilizadas de moodle
 que son utilizadas a lo largo de los distintos casos de uso del módulo de
 experiencia. La especificación de las interfaces de moodle contiene una imagen de
 la interfaz y una descripción de los elementos y acciones más relevantes por cada
 interfaz.

    % INTERFACES DE MOODLE
    \input{modulos/moodle/IU/IU-M00}  % Tablero (Dashboard)
    \input{modulos/moodle/IU/IU-M00a} % Form: Selector de archivos
    \input{modulos/moodle/IU/IU-M00b} % Login

    \input{modulos/moodle/IU/IU-M01}  % Administración del sitio
    \input{modulos/moodle/IU/IU-M01a} % Administración del sitio plugins
    \input{modulos/moodle/IU/IU-M01b} % Administración del sitio usuarios
    \input{modulos/moodle/IU/IU-M01c} % Administración del sitio Cursos

    \input{modulos/moodle/IU/IU-M02}  % Instalación de Plugin
    \input{modulos/moodle/IU/IU-M02a} % Validación de archivo ZIP
    \input{modulos/moodle/IU/IU-M02b} % Comprobación de plugibs
    \input{modulos/moodle/IU/IU-M02c} % Cancelación Instalación Plugin
    \input{modulos/moodle/IU/IU-M02d} % Resultado Instalación Plugin

    \input{modulos/moodle/IU/IU-M03}  % 
    \input{modulos/moodle/IU/IU-M03a} % 

    \input{modulos/moodle/IU/IU-M04}  % 
    \input{modulos/moodle/IU/IU-M04a} % 

    \input{modulos/moodle/IU/IU-M05}  % 
    \input{modulos/moodle/IU/IU-M05a} % 
    \input{modulos/moodle/IU/IU-M05b} % 
    \input{modulos/moodle/IU/IU-M05c} % 

    \input{modulos/moodle/IU/IU-M06}  % Cursos y Categorías
    \input{modulos/moodle/IU/IU-M06a} % Confirmación eliminar curso
    \input{modulos/moodle/IU/IU-M06b} % Estado de eliminación de un curso
    \input{modulos/moodle/IU/IU-M06c} % Curso de moodle

    \input{modulos/moodle/IU/IU-M07} % Pagina Inicial del Sitio

\subsection*{Interfaces diseñadas}

 A continuación se presentan todas las interfaces que fueron requeridas agregar
 a moodle para brindar las funcionalidades planteadas por los casos de uso. Al igual
 que la especificación de las interfaces de moodle, la especificación de estas
 interfaces contiene la imagen de la interfaz y una descripción de los elementos
 y acciones más relevantes por cada interfaz.

    \input{modulos/exp/IU/IU-E01}  % Configuraciones
    \input{modulos/exp/IU/IU-E02}  % Configuraciones Generales
    \input{modulos/exp/IU/IU-E03}  % Configuraciones Visuales
    \input{modulos/exp/IU/IU-E03a} % Configuraciones Mod Exp desactivado
    \input{modulos/exp/IU/IU-E04}  % Configuraciones Esquema
    \input{modulos/exp/IU/IU-E05}  % Configuraciones de Eventos
    \input{modulos/exp/IU/IU-E05a} % Configuraciones Events desativados
    \input{modulos/exp/IU/IU-E06}  % Curso Gamificado
    \input{modulos/exp/IU/IU-E06a}  % Curso Gamificado
    \input{modulos/exp/IU/IU-E06b} % Curso Gamificado (editing)

    %\input{modulos/modExpIU/IU-E04} % Configuraciones de Comportamiento
    %\input{modulos/modExpIU/IU-E05} % Configuraciones de eventos

\subsection*{Diseño de plugins} % TODO CHANGE FOR INPUTS

 Para poder desarrollar las funcionalidades especificadas mediante los casos de uso
 es requerido el desarrollo de tres plugins distintos. Dichos plugins son listados a
 continuación:

    \begin{itemize}
    \item {\bf\color{primary}%
    \hypertarget{local:gamedlemaster}{Gamedlemaster} (local)}

        Este plugin ayudará a extender el esquema de la base de datos de moodle
        para que los plugins de experiencia y los de los demás módulos puedan
        acceder a los datos sin la necesidad de depender explícitamente de otro
        módulo.

    \item {\bf\color{primary}%
    \hypertarget{format:gamedle}{Curso con puntos de experiencia} (course format)}

        Este es un plugin de tipo {\it course format} lo cual permite cambiar
        la estructura de cómo se presentan las secciones y actividades del curso
        \cite{MoodleCourseFormat}. Este tipo de plugin fue escogido para que los
        cursos en moodle puedan otorgar puntos de experiencia.
        % https://docs.moodle.org/dev/Course_formats

    \item {\bf\color{primary}%
    \hypertarget{block:gmxp}{Nivel Gamedle} (block)}

        Los plugins de tipo {\it block} permiten desplegar información y/o brindar una 
        funcionalidad a lo largo de distintas páginas de moodle \cite{MoodleBlocks},
        además permite a cualquier usuario agregarlo u ocultarlo en las distintas
        pantallas a las que tiene acceso, debido a estar razones este plugin se
        encargará de mostrar el nivel y puntos de experiencia de dicho nivel que
        tiene un usuario.
        % https://docs.moodle.org/dev/Plugin_types
    \end{itemize}

\subsubsection{Diagrama de componentes} % TODO CHANGE FOR INPUTS
\subsubsection{Diagrama de clases} % TODO CHANGE FOR INPUTS


\begin{comment} % ==================================================================
\subsubsection{Interfaces de Moodle}

    % INTERFACES DE MOODLE
    
\subsubsection{IU-M00 Pantalla principal}

 El tablero o {\it Dashboard} (ver figura~\ref{IU-M02}) es la primer página que ve un usuario de
 inmediatamente despues iniciar sesión, esta página muestra a los usuarios detalles de su progreso
 y fechas límite próximas \cite{MoodleTablero} . Los elementos que tiene esta página con las demás
 páginas del sitio de moodle es el menú de navegación a la izquierda y la columna derecha de
 bloques.

    \IUfig{1}{modulos/moodle/IU/Dashboard.png}{IU-M00}{Pantalla Principal de Moodle}

\subsubsection{Elementos relevantes}

    \begin{itemize}
    \item
    {\bf Menú Superior}
        Como su nombre lo indica se encuentra en la parte superior, este elemento se
        encuentra en la mayoría de las pantallas de moodle.

    \item
    {\bf Menú de Navegación}
        Cuando esta visible se encuentra en la parte izquierda de la parte izquierda
        de la mayoría de las pantallas de moodle. Se puede ocultar o mostrar con la
        acción \IUMenu[].

    \item
    {\bf Contenido}
        Tiene todos los demás elementos que conforman el contenido de la pantalla.

    \end{itemize}

\subsubsection{Acciones relevantes}

    \begin{itemize}
    
    \item
    {\bf \IUMenu{} (Desplegar el menú)}
        Si el menú está oculto, cuando el usuario presione el botón \IUMenu{} el menú de
        navegación se desplegará.

    \item {\bf \IUMenu{} (Ocultar Menu)}
        Si el menú está visible, cuando el usuario presione el botón \IUMenu{} el menú de
        navegación se ocultará.

    \item {\bf \IUAdminSitio{} Administración del sitio }
        Cuando el menú está visible, el botón de administración del sitio nos permitirá
        navegar a la pantalla \refElem{IU-M01}

    \end{itemize}
  % Tablero (Dashboard)
    
\subsubsection{IU-M00a Selector de archivos}

 El selector de archivos permite a los usuarios de moodle seleccionar un archivo desde los archivos
 del servidor, archivos recientes, archivos privados, desde la computadora o incluso buscar imágenes
 para ser seleccionadas \cite{MoodleSelectorArchivos}.

    \IUfig{1}{modulos/moodle/IU/PopUpArchivos.png}{IU-M00a}{Selector de archivos}

\subsubsection{Elementos relevantes}

    \begin{itemize}
    \item {\bf Menú izquierdo}
        Permite al usuario escoger desde que medio seleccionará el archivo a elegir.
    \end{itemize}

\subsubsection{Acciones relevantes}

    \begin{itemize}
    \item {\bf Browse (Subir un archivo)}
        Cuando se presione el botón \fbox{Browse}, el navegador desplegará una ventana
        emergente para seleccionar un archivo desde el sistema de archivos.

    \item {\bf Subir este archivo}
        Cuando el usuario presione este botón el usuario confirmará la acción de subir
        el archivo que previamente a seleccionado.
    \end{itemize}
 % Form: Selector de archivos
    
\subsubsection{IU-M00b Ingreso a moodle}

    Esta pantalla es de moodle. Esta pantalla brinda acceso al sistema.

    \IUfig{1}{modulos/moodle/IU/Login}{IU-M00b}{Ingreso a moodle}

\subsubsection{Elementos Relevantes}

    \begin{itemize}
    \item {\bf Lorem ipsum}
        ...
    \end{itemize}

\subsubsection{Acciones relevantes}

    \begin{itemize}
    \item {\bf Lorem ipsum}
        ...
    \end{itemize}

\clearpage
 % Login

    
\subsection{IU-M01 Pantalla principal}

 La página de portada, o página principal mostrada en la figura \ref{IU-M01}, es la
 página inicial que ve alguien que llega a un sitio Moodle antes o después de entrar al sitio.
 Típicamente un estudiante verá los cursos, algunos bloques de información, mostrados en un tema.
 En la Barra de navegación y en el menú de navegación (esquina superior izquierda).\\

 \noindent 
 La combinación de las políticas del sitio, autenticación del usuario y configuraciones de la
 portada determinan quién puede llegar a la portada, los elementos que pueden ver y acciones
 que pueden realizar \cite{MoodlePortada}.
    % https://docs.moodle.org/all/es/Portada

    \IUfig{1}{modulos/IUMoodle/IU-M01-moodle.png}{IU-M01}{Pantalla Principal de Moodle}

\subsubsection{Elementos Relevantes}

    \begin{itemize}
    \item
    {\bf Menú Superior}
        Como su nombre lo indica se encuentra en la parte superior, este elemento se
        encuentra en la mayoría de las pantallas de moodle.

    \item
    {\bf Menú de Navegación}
        Cuando esta visible se encuentra en la parte izquierda de la parte izquierda
        de la mayoría de las pantallas de moodle. Se puede ocultar o mostrar con la
        acción \IUMenu[].

    \item
    {\bf Contenido}
        Tiene todos los demás elementos que conforman el contenido de la pantalla.

    \end{itemize}

\subsubsection{Acciones relevantes}

    \begin{itemize}
    
    \item
    {\bf \IUMenu (Desplegar el menú)}
        Si el menú está oculto, cuando el usuario presione el botón \IUMenu el menú de
        navegación se desplegará.

    \item {\bf \IUMenu (Ocultar Menu)}
        Si el menú está visible, cuando el usuario presione el botón \IUMenu el menú de
        navegación se ocultará.

    \item {\bf \IUAdminSitio Administración del sitio }
        Cuando el menú está visible, el botón de administración del sitio nos permitirá
        navegar a la pantalla \refElem{IU-M02}

    \end{itemize}
  % Administración del sitio
    
\subsection{IU-M01a: Resultado de instalación del plugin}

 Esta pantalla muestra el resultado de la instalación del plugin. Si los plugins
 que son instalados vienen de una fuente confiable, esta pantalla siempre debería
 de aparecer mostrando un resultado existoso, en caso contrario dira que la instalación
 no pudo llevarse a cabo de forma exitosa.

    \IUfig{1}{modulos/IUMoodle/InstallResult.png}{IU-M01a}{Resultado de instalación del plugin}

\subsubsection{Elementos relevantes}

    \begin{itemize}
    \item {\bf Mensaje de estado de instalación}
        El mensaje se pinta de color verde o color rojo dependiendo si el 
    \end{itemize}

\subsubsection{Acciones relevantes}

    \begin{itemize}
    \item {\bf Aceptar}
        Si el plugin possé configuraciones para el administrador entonces esta acción
        redirigirá a la pantalla de configuración correspondiente al plugin, en caso
        contrario redirigirá a la anterior página del sitio de moodle mostrada.
    \end{itemize}
 % Administración del sitio plugins
    
\subsubsection{IU-M01b Administración del sitio (Usuarios)}

 Esta pantalla permite al \refElem{aAdministrador} acceder a las configuraciones
 específicas de usuarios dentro del moodle que administra. En esta pantalla se
 encuentran tres categorías principales de configuraciones, dichas categorías con
 usuarios, cuentas y permisos, cada una con sus correspondientes configuraciones
 específicas.

    \IUfig{1}{modulos/moodle/IU/AdministracionSitioUsers}%
        {IU-M01b}{Administración del sitio (Usuarios)}

\subsubsection{Elementos Relevantes}

    \begin{itemize}
    \item {\bf Menu de opciones}
        En la parte principal de la pantalla se encuentra la lista de las
        opciones agrupadas en las tres categorías de usuarios, cuentas y permisos.
    \end{itemize}

\subsubsection{Acciones relevantes}

    \begin{itemize}
    \item {\bf Agregar un usuario}
        Permite acceder al formulario para crear cuentas para distintos usuarios.

    \item {\bf Mirar la lista de usuarios}
        Permite acceder a la pantalla para gestionar las distintas cuentas de usuarios
        de la plataforma.
    \end{itemize}

\clearpage
 % Administración del sitio usuarios
    
\subsubsection{IU-M01c Administración del sitio (Cursos)}

 Esta pantalla permite al \refElem{aProfesor} acceder a las configuraciones
 específicas de los cursos. El acceso a esta pantalla es indispensable para que el 
 profesor cree un curso, lo edite y tambien administre su contenido.

    \IUfig{1}{modulos/moodle/IU/AdministracionSitioCursos}%
        {IU-M01c}{Administración del sitio (módulos)}

\subsubsection{Elementos Relevantes}

    \begin{itemize}
    \item {\bf Menú de opciones}
        Para la cuenta del profesor se muestran las opciones mínimas requeridas
        para acceder a la gestión de los cursos y de las categorías a las que
        están vinculadas los cursos.
    \end{itemize}

\subsubsection{Acciones relevantes}

    \begin{itemize}
    \item {\bf Selección de la opción gestionar cursos y categorías}
        Redirige a la pantalla para la gestión de cursos y categorías.
    \end{itemize}

\clearpage
 % Administración del sitio Cursos

    
\subsection{IU-M02: Instalador de plugin}

 La página del instalador de plugins permite al \refElem{aAdministrador} instalar nuevos plugins al 
 moodle que administra de una forma sencilla y sin tener que manipular los archivos en el servidor
 donde se tenga moodle instalado, para ello cada \refElem[plugin]{Plugin} a instalar debe estar
 compresos en un archivo {\it ZIP} cumpliendo con la regla \refElem{BR-M1}.

 % TODO: BR-M1: Restricciones el archivo de instalación.
 % El archivo de instalación debe ser un archivo ZIP, el cual debe contener exactamente un
 % directorio que coincida con el nombre del plugin.

    \IUfig{1}{modulos/moodleIU/InstallPlugin.png}{IU-M02}{Instalador de plugin}

\subsubsection{Elementos relevantes}

    \begin{itemize}
    \item {\bf Selector de archivos.}
        Permite elegir un archivo y prepararlo para subirlo a moodle
        y realizar las acciones correspondientes.
    \end{itemize}

\subsubsection{Acciones relevantes}

    \begin{itemize}
    \item {\bf Selección de un archivo}
        Permite seleccionar un \refElem{Plugin} compreso en un archivo {\it ZIP} para
        ser instalado en moodle.

    \item {\bf Instalar plugin desde un archivo ZIP}
        Confirma el envió del formulario que contiene principalmente al archivo compreso con 
        el plugin que será instalado. Redirige a la pantalla \refElem{IU-M02a}.
    \end{itemize}

\clearpage
  % Instalación de Plugin
    
\subsection{IU-M02a Validación del plugin a instalar}

 La pantalla de validación del plugin a instalar presenta el resultado de la validación de un
 archivo de plugin compreso con base en la regla \refElem{BR-M01}. Esta pantalla dira Si el archivo
 esta formado correctamente o no, además de las acciones adicionales que se llevarán a cabo.

    \IUfig{1}{modulos/moodleIU/PluginZIPValidacion}{IU-M02a}{Validación del plugin a instalar}

\subsubsection{Elementos Relevantes}

    \begin{itemize}
    \item {\bf Validación del plugin}
        Contiene el resultado de la validación del plugin más las acciones
        a realizar para proceder con la instalación del plugin.
        
    \end{itemize}

\subsubsection{Acciones relevantes}

    \begin{itemize}
    \item {\bf Continuar}
        En caso correcto de la validación, este botón permite continuar con la instalación
        rediriginedo a la página \refElem{IU-M02b}.

    \item {\bf Cancelar}
        El botón de cancelar interrumpe el proceso de instalación del plugin, redirigiendo
        a la anterior pantalla \refElem{IU-M01}.
    \end{itemize}

\clearpage
 % Validación de archivo ZIP
    
\subsubsection{IU-M02b Comprobación de plugins}

 Esta página muestra los plugins que pueden requerir su atención durante un actualización al sitio
 de moodle, como una la instalación o actualización de plugins. La documentación de esta pantalla
 contempla únicamente el caso de instalación/actualización de un plugin a la vez.

    \IUfig{1}{modulos/moodle/IU/InstallConfirm}{IU-M02b}{Comprobación de plugins}

\subsubsection{Elementos relevantes}

   \begin{itemize}
   \item {\bf Lista de plugins a instalar}
        La lista de \refElem[Plugins]{Plugin} que se van a instalar, incluyendo
        sus atributos.
   \end{itemize}

\subsubsection{Acciones relevantes}

    \begin{itemize}
    \item {\bf Actualizar base de datos de Moodle ahora}
        Esta acción permite instalar el plugin en Moodle y correr la secuencia de
        instrucciones establecida por el plugin a instalar. Redirige a la pantalla
        \refElem{IU-M02d}.

    \item {\bf Cancelar las nuevas instalaciones}
        Esta acción permite cancelar las instalaciones o actualizaciones de los plugins,
        redirige a la pantalla \refElem{IU-M02c}

    \item {\bf Cancelar esta instalación}
        A diferencia de la acción anterior, esta acción permite cancelar la instalación
        o actualización de un plugin en particular anterior. Redirige a la pantalla
        \refElem{IU-M02c}.
    \end{itemize}

\clearpage
 % Comprobación de plugibs
    
\subsubsection{IU-M02c: Cancelación de instalación de plugins}

 Esta pantalla tiene el propósito de notificar al \refElem{aAdministrador} de las acciones a
 llevar a cabo en caso de proseguir con la cancelación de la instalación de los plugins. Esta es
 la última confirmación que se le pregunta al administrador antes de la cancelación.

    \IUfig{1}{modulos/moodle/IU/InstallCancelled.png}{IU-M02c}{Comprobación de pluginc}

\subsubsection{Elementos relevantes}

    \begin{itemize}
    \item {\bf Lista de los plugins}
        Contiene la lista de los plugins y la ubicación absoluta de la carpeta
        que contiene todos los archivos de cada plugin que será eliminado.
    \end{itemize}

\subsubsection{Acciones relevantes}

    \begin{itemize}
    \item {\bf Continuar}
        Esta acción confirma la cancelación de los plugins y eliminación de los
        archivos de los mismos. Redirige a la pantalla \refElem{IU-M02}.

    \item {\bf Cancelar}
        Esta acción regresa a la pantalla \refElem{IU-M01} para continuar con la instalación
        de plugins.
    \end{itemize}

\clearpage
 % Cancelación Instalación Plugin
    
\subsubsection{IU-M02d: Resultado de instalación del plugin}

 Esta pantalla muestra el resultado de la instalación del plugin. Si los plugins
 que son instalados vienen de una fuente confiable, esta pantalla siempre debería
 de aparecer mostrando un resultado existoso, en caso contrario dira que la instalación
 no pudo llevarse a cabo de forma exitosa.

    \IUfig{1}{modulos/moodle/IU/InstallResult.png}{IU-M02d}{Resultado de instalación del plugin}

\subsubsection{Elementos relevantes}

    \begin{itemize}
    \item {\bf Mensaje de estado de instalación}
        El mensaje se pinta de color verde o color rojo dependiendo si el 
    \end{itemize}

\subsubsection{Acciones relevantes}

    \begin{itemize}
    \item {\bf Aceptar}
        Si el plugin poseé configuraciones para el administrador entonces esta acción
        redirigirá a la pantalla de configuración correspondiente al \refElem{Plugin},
        en caso contrario redirigirá a la anterior página del sitio de moodle mostrada.
    \end{itemize}

\clearpage
 % Resultado Instalación Plugin

    
\subsubsection{IU-M03 Vista General de Plugins}

 Esta pantalla permite al \refElem{aAdministrador} visualizar la lista de los plugins
 instalados en moodle incluyen los que vienen incluidos en modole, aśi como los
 adicionales, tambien permite acceder a la pantalla \refElem{IU-M03a} para ver la
 lista de todos los plugins.

    \IUfig{1}{modulos/moodle/IU/VistaGeneralPlugins}{IU-M03}{Vista General de Plugins}

\subsubsection{Elementos Relevantes}

    \begin{itemize}
    \item {\bf Lista de todos los plugins}
        Contiene la lista de todos los plugins agrupados por el tipo de plugin, 
        se pueden visualizar y realizar acciones de configuración y eliminación
        a cada elemento de lista.
    \end{itemize}

\subsubsection{Acciones relevantes}

    \begin{itemize}
    \item {\bf Cambiar la lista mostrada de los plugins}
        Permite seleccionar la pestaña correspondiente a la lista de plugins
        que se desea ver.
    \item {\bf Acceder a la configuración de un plugin}
        Si el plugin tiene configuración mediante este enlace se puede acceder
        a la configuración del plugin correspondiente.
    \item {\bf Desinstalar}
        Permite la desinstalación de los plugins que no sean requeridos en la
        plataforma.
    \end{itemize}

\clearpage
  %
    
\subsubsection{IU-M03a Vista de Plugins Adicionales}

 Esta pantalla permite al \refElem{aAdministrador} visualizar la lista de los plugins
 instalados adicionales instalados, tambien permite acceder a la pantalla
 \refElem{IU-M03} para ver la lista de todos los plugins.

    \IUfig{1}{modulos/moodle/IU/VistaPluginsAdicionales}%
        {IU-M03a}{Vista de Plugins Adicionales}

\subsubsection{Elementos Relevantes}

    \begin{itemize}
    \item {\bf Lista de plugins adicionales}
        Contiene la lista de todos los plugins externos a moodle, instalados por
        el \refElem{aAdministrador} agrupados por el tipo de plugin, de la misma
        forma se pueden visualizar y realizar acciones de configuración y eliminación
        a cada elemento de lista.
    \end{itemize}

\subsubsection{Acciones relevantes}

    \begin{itemize}
    \item {\bf Cambiar la lista mostrada de los plugins}
        Permite seleccionar la pestaña correspondiente a la lista de plugins
        que se desea ver.
    \item {\bf Acceder a la configuración de un plugin}
        Si el plugin tiene configuración mediante este enlace se puede acceder
        a la configuración del plugin correspondiente.
    \item {\bf Desinstalar}
        Permite la desinstalación de los plugins que no sean requeridos en la
        plataforma.
    \end{itemize}

\clearpage
 %

    
\subsubsection{IU-M04 Desinstalar plugin}

 Esta pantalla muestra un mensaje de confirmación para la acción de eliminar un
 plugin en específico.

    \IUfig{1}{modulos/moodle/IU/DesinstalarPluginConfirm}{IU-M04}{Desinstalar plugin}

\subsubsection{Elementos Relevantes}

    \begin{itemize}
    \item {\bf Mensaje de confirmación}
        Muestra un mensaje solicitando la confirmación del usuario acerca de la
        desinstalación de un plugin.
    \end{itemize}

\subsubsection{Acciones relevantes}

    \begin{itemize}
    \item {\bf Continuar}
        Permite al usuario confirmar a operación de desinstalar un plugin.
    \item {\bf Cancelar}
        Permite al usuario cancelar la operación de desinstalación de un plugin.
    \end{itemize}

\clearpage
  %
    
\subsubsection{IU-E04a Desinstalando plugin}

 Cuando se desinstala un plugin moodle pide la confirmación del usuario para eliminar
 los archivos correspondientes al plugin. Si el usuario decide no eliminar los
 archivos entonces el plugin se detectará como una plugin para ser instalado desde
 cero.

    \IUfig{1}{modulos/moodle/IU/DesinstalandoPlugin}%
        {IU-M04a}{Desinstalando plugin}

\subsubsection{Elementos Relevantes}

    \begin{itemize}
    \item {\bf Mensaje de confirmación}
        Contiene un mensaje que pide la confirmación del usuario para proceder con
        la eliminación de los archivos de un plugin y si dejar los archivos para que
        este sea instalado de nuevo.
    \end{itemize}

\subsubsection{Acciones relevantes}

    \begin{itemize}
    \item {\bf Continuar}
        Permite confirmar la eliminación de los archivos del plugin que fue
        previamente desinstalado.

    \item {\bf Cancelar}
        Permite indicarle a moodle que no elimine los archivos correspondientes al
        plugin se acaba de eliminar, esto permite que el plugin se pueda instalar
        desde cero.
    \end{itemize}

\clearpage
 %

    
\subsubsection{IU-M05 Creación de usuario}

 Descripción ...

    \IUfig{1}{modulos/moodle/IU/CreateUser}{IU-M05}{Creación de usuario}

\subsubsection{Elementos Relevantes}

    \begin{itemize}
    \item {\bf Lorem ipsum}
        ...
    \end{itemize}

\subsubsection{Acciones relevantes}

    \begin{itemize}
    \item {\bf Lorem ipsum}
        ...
    \end{itemize}

\clearpage
  %
    
\subsubsection{IU-M05a Lista de usuarios}

 Descripción ...

    \IUfig{1}{modulos/moodle/IU/ListUsers}{IU-M05a}{Lista de Usuarios}

\subsubsection{Elementos Relevantes}

    \begin{itemize}
    \item {\bf Lorem ipsum}
        ...
    \end{itemize}

\subsubsection{Acciones relevantes}

    \begin{itemize}
    \item {\bf Lorem ipsum}
        ...
    \end{itemize}

\clearpage
 %
    
\subsubsection{IU-M05b Confirmación acerca de eliminar un usuario}

 Mensaje de confirmación acerca de la eliminación de un usuario.

    \IUfig{1}{modulos/moodle/IU/DeleteUser}%
        {IU-M05b}{Confirmación acerca de eliminar un usuario}

\subsubsection{Elementos Relevantes}

    \begin{itemize}
    \item {\bf Mensaje de confirmación}
        Presenta el mensaje de confirmación para continuar con la operación de
        eliminación de un usuario.
    \end{itemize}

\subsubsection{Acciones relevantes}

    \begin{itemize}
    \item {\bf Eliminar}
        Permite continuar con las operaciones de eliminación de un usuario
        de moodle.
        
    \item {\bf Cancelar}
        Permite cancelar las operaciones de eliminación de un usuario de moodle.
    \end{itemize}

\clearpage
 %
    
\subsubsection{IU-M05c Pantalla de auto-registro}

 Ofrece a los usuarios que no estan registrados en moodle un mecanismo mediante
 el cual pueden acceder a crear su usuario de moodle de forma por su propia cuenta.

    \IUfig{1}{modulos/moodle/IU/CreateOwnUser}{IU-M05c}{Pantalla de auto-registro}

\subsubsection{Elementos Relevantes}

    \begin{itemize}
    \item {\bf Formulario de registro}
        Contiene los campos para el registro de un usuario de moodle
    \end{itemize}

\subsubsection{Acciones relevantes}

    \begin{itemize}
    \item {\bf Crear nueva cuenta}
        Permite al usuario confirmar la creación de se nueva cuenta en el
        sitio de moodle

    \item {\bf Cancelar}
        Permite al usuario cancelar el registro de su cuenta de usuario.
    \end{itemize}

\clearpage
 %

    
\subsubsection{IU-M06 Cursos y categorías}

 Esta pantalla permite acceder a la gestión de cursos y categorías existentes en
 moodle, partiendo de este punto se pueden editar, configurar, eliminar y crear
 cursos y categorías presentes en moodle. Los actores que pueden acceder a esta
 pantalla son el \refElem{aProfesor} y el \refElem{aAdministrador}.

    \IUfig{1}{modulos/moodle/IU/Cursos-Categories}{IU-M06}{Cursos y categorías}

\subsubsection{Elementos Relevantes}

    \begin{itemize}
    \item {\bf Lista de categorías}
        Muestra la lista de categorías existentes en el sitio de moodle.
    \item {\bf Lista de cursos}
        Contiene la lista de los cursos correspondientes a la categoría
        seleccionada, sobre esta lista se puden ejecutar las acciones
        relevantes..
    \end{itemize}

\subsubsection{Acciones relevantes}

    \begin{itemize}
    \item {\bf Crear un nuevo curso}
        Permite crear un nuevo curso con la categoría seleccionada como
        la categoría por defecto.
    \item {\bf Visualizar curso}
        Permite acceder a la pantalla de la visualización de un nuevo curso
        para administrarlo y/o editarlo.
        la categoría por defecto.
    \item {\bf Eliminar un curso}
        Permite seleccionar un curso que se deseé eliminar.
    \end{itemize}

\clearpage
  % Cursos y Categorías
    
\subsubsection{IU-M06a Confirmación para eliminar un curso}

 %Descripción ...

    \IUfig{1}{modulos/moodle/IU/Cursos-Delete}%
        {IU-M06a}{Confirmación para eliminar un curso}

\begin{comment}
\subsubsection{Elementos Relevantes}

    \begin{itemize}
    \item {\bf Lorem ipsum}
        ...
    \end{itemize}

\subsubsection{Acciones relevantes}

    \begin{itemize}
    \item {\bf Lorem ipsum}
        ...
    \end{itemize}
\end{comment}

\clearpage
 % Confirmación eliminar curso
    
\subsubsection{IU-M06b Estado de la eliminación de un curso}

% Descripción ...

    \IUfig{1}{modulos/moodle/IU/Cursos-Delete-Status}%
        {IU-M06b}{Estado de la eliminación de un curso}

\begin{comment}
\subsubsection{Elementos Relevantes}

    \begin{itemize}
    \item {\bf Lorem ipsum}
        ...
    \end{itemize}

\subsubsection{Acciones relevantes}

    \begin{itemize}
    \item {\bf Lorem ipsum}
        ...
    \end{itemize}
\end{comment}

\clearpage
 % Estado de eliminación de un curso
    
\subsubsection{IU-M06c Curso de moodle}

 Descripción ...

    \IUfig{1}{modulos/moodle/IU/Cursos-normal}{IU-M06c}{Curso de moodle}

\subsubsection{Elementos Relevantes}

    \begin{itemize}
    \item {\bf Lorem ipsum}
        ...
    \end{itemize}

\subsubsection{Acciones relevantes}

    \begin{itemize}
    \item {\bf Lorem ipsum}
        ...
    \end{itemize}

\clearpage
 % Curso de moodle

    
\subsubsection{IU-M07 Pantalla principal del curso}

 Descripción ...

    \IUfig{1}{modulos/moodle/IU/p_principal_curso}{IU-M07}{Pantalla principal del curso}

\subsubsection{Elementos Relevantes}

    \begin{itemize}
    \item {\bf Lorem ipsum}
        ...
    \end{itemize}

\subsubsection{Acciones relevantes}

    \begin{itemize}
    \item {\bf Lorem ipsum}
        ...
    \end{itemize}

\clearpage
 % Pagina Inicial del Sitio

\subsubsection{Interfaces del módulo de experiencia}

    
\subsubsection{IU-E01 Configuraciones del módulo de experiencia}

 Descripción ...

    \IUfig{1}{modulos/exp/IU/Settings}{IU-E01}{Configuraciones del módulo de experiencia}

\subsubsection{Elementos Relevantes}

    \begin{itemize}
    \item {\bf Lorem ipsum}
        ...
    \end{itemize}

\subsubsection{Acciones relevantes}

    \begin{itemize}
    \item {\bf Lorem ipsum}
        ...
    \end{itemize}

\clearpage
  % Configuraciones
    
\subsection{IU-E02 Configuraciones generales}

 Descripción ...

    \IUfig{1}{modulos/modExpIU/SettingsGenerales}{IU-E02}{Configuraciones generales}

\subsubsection{Elementos Relevantes}

    \begin{itemize}
    \item {\bf Lorem ipsum}
        ...
    \end{itemize}

\subsubsection{Acciones relevantes}

    \begin{itemize}
    \item {\bf Lorem ipsum}
        ...
    \end{itemize}

\clearpage
  % Configuraciones Generales
    
\subsection{IU-E03: Configuraciones Visuales del módulo de experiencia}

 Descripción ...

    \IUfig{1}{modulos/exp/IU/settingsVisuales}{IU-E03}{%
        configuraciones Visuales del módulo de experiencia}

\subsubsection{Elementos Relevantes}

    \begin{itemize}
    \item {\bf Lorem ipsum}
        ...
    \end{itemize}

\subsubsection{Acciones relevantes}

    \begin{itemize}
    \item {\bf Guardar cambios}
        ...

    \item {\bf Cancelar}
        ...
    \end{itemize}

\clearpage
  % Configuraciones Visuales
    
\subsection{IU-E03a Módulo de experiencia desactivado}

 Descripción ...

    \IUfig{1}{modulos/modExpIU/settingsExperienceDisabled}{IU-E03a}{%
        Módulo de experiencia desactivado}

\subsubsection{Elementos Relevantes}

    \begin{description}
    \bTerm{tSelectColor}{panel de colores} ...
    \end{description}

\subsubsection{Acciones relevantes}

    \begin{itemize}
    \item {\bf Lorem ipsum}
        ...
    \end{itemize}

\clearpage
 % Configuraciones Mod Exp desactivado
    
\subsubsection{IU-E04: Configuraciones del sistema de experiencia}

 Descripción ...

    \IUfig{1}{modulos/exp/IU/settingsEsquema}{IU-E04}{%
        Configuraciones del sistema de experiencia}

\subsubsection{Elementos Relevantes}

    \begin{itemize}
    \item {\bf Lorem ipsum}
        ...
    \end{itemize}

\subsubsection{Acciones relevantes}

    \begin{itemize}
    \item {\bf Lorem ipsum}
        ...
    \end{itemize}

\clearpage
  % Configuraciones Esquema
    
\subsubsection{IU-E05 Configuración de eventos con experiencia}

% Descripción ...

    \IUfig{1}{modulos/exp/IU/SettingsEventos}{IU-E05}%
        {Configuración de eventos con experiencia}

\begin{comment}
\subsubsection{Elementos Relevantes}

    \begin{itemize}
    \item {\bf Lorem ipsum}
        ...
    \end{itemize}

\subsubsection{Acciones relevantes}

    \begin{itemize}
    \item {\bf Lorem ipsum}
        ...
    \end{itemize}
\end{comment}

\clearpage
  % Configuraciones de Eventos
    
\subsection{IU-E05a Eventos con experiencia desactivados}

 Descripción ...

    \IUfig{1}{modulos/exp/IU/SettingsEventsDisabled}{IU-E05a}%
    {Eventos con experiencia desactivados}

\subsubsection{Elementos Relevantes}

    \begin{itemize}
    \item {\bf Lorem ipsum}
        ...
    \end{itemize}

\subsubsection{Acciones relevantes}

    \begin{itemize}
    \item {\bf Lorem ipsum}
        ...
    \end{itemize}

\clearpage
 % Configuraciones Events desativados
    
\subsubsection{IU-E06 Creación de un curso gamificado}

 Descripción ...

    \IUfig{1}{modulos/exp/IU/CursoCreate}{IU-E06}{Creación curso gamificado}

\subsubsection{Elementos Relevantes}

    \begin{itemize}
    \item {\bf Lorem ipsum}
        ...
    \end{itemize}

\subsubsection{Acciones relevantes}

    \begin{itemize}
    \item {\bf Lorem ipsum}
        ...
    \end{itemize}

\clearpage
  % Curso Gamificado
    
\subsubsection{IU-E06a Curso Gamificado}

 Descripción ...

    \IUfig{1}{modulos/exp/IU/CursoView}{IU-E06a}{Curso Gamificado}

\subsubsection{Elementos Relevantes}

    \begin{itemize}
    \item {\bf Lorem ipsum}
        ...
    \end{itemize}

\subsubsection{Acciones relevantes}

    \begin{itemize}
    \item {\bf Lorem ipsum}
        ...
    \end{itemize}

\clearpage
  % Curso Gamificado
    
\subsection{IU-E06b Curso gamificado en edición}

 Descripción ...

    \IUfig{1}{modulos/exp/IU/CursoEdit}{IU-E06b}{Curso gamificado en edición}

\subsubsection{Elementos Relevantes}

    \begin{itemize}
    \item {\bf Lorem ipsum}
        ...
    \end{itemize}

\subsubsection{Acciones relevantes}

    \begin{itemize}
    \item {\bf Lorem ipsum}
        ...
    \end{itemize}

\clearpage
 % Curso Gamificado (editing)

    %
\subsubsection{IU-E04: Configuraciones del sistema de experiencia}

 Descripción ...

    \IUfig{1}{modulos/exp/IU/settingsEsquema}{IU-E04}{%
        Configuraciones del sistema de experiencia}

\subsubsection{Elementos Relevantes}

    \begin{itemize}
    \item {\bf Lorem ipsum}
        ...
    \end{itemize}

\subsubsection{Acciones relevantes}

    \begin{itemize}
    \item {\bf Lorem ipsum}
        ...
    \end{itemize}

\clearpage
 % Configuraciones de Comportamiento
    %
\subsubsection{IU-E05 Configuración de eventos con experiencia}

% Descripción ...

    \IUfig{1}{modulos/exp/IU/SettingsEventos}{IU-E05}%
        {Configuración de eventos con experiencia}

\begin{comment}
\subsubsection{Elementos Relevantes}

    \begin{itemize}
    \item {\bf Lorem ipsum}
        ...
    \end{itemize}

\subsubsection{Acciones relevantes}

    \begin{itemize}
    \item {\bf Lorem ipsum}
        ...
    \end{itemize}
\end{comment}

\clearpage
 % Configuraciones de eventos

\subsubsection{Diseño de plugins} % TODO CHANGE FOR INPUTS
\subsubsection{Diagrama de componentes} % TODO CHANGE FOR INPUTS
\subsubsection{Diagrama de clases} % TODO CHANGE FOR INPUTS

\end{comment} % >>>>>>> d14b3c5df86665f3cc11f090d68bc3001772b4f5 ============asdsds

\subsection{Pruebas}

    
A continuación se enlistan los casos de prueba identificados
correspondientes a cada uno de los casos de uso especificados. Los casos
de prueba listados a continuación son de dos tipos, los correctos e
incorrectos identificados por los prefijos CPC y CPI respectivamente.

\subsubsection{\refElem{CU-P01}}

\begin{itemize}
  \TestCase{CPC-P01-1}{Ver perfil con objetos ya elegidos}
  \TestCase{CPC-P01-2}{Ver perfil con valores por defecto}
  \TestCase{CPC-P01-3}{Usuario no registrado como usuario gamificado}
\end{itemize}


\subsubsection{\refElem{CU-P02}}

\begin{itemize}
  \TestCase{CPC-P02-1}{Modificar perfil con objetos desbloqueados}
  \TestCase{CPI-P02-2}{Intentar modificar pefil con objetos no desbloqueados}
  \TestCase{CPI-P02-3}{Intentar modificar perfil con 2 objetos de un mismo tipo}
\end{itemize}


\subsubsection{\refElem{CU-P03}}

\begin{itemize}
  \TestCase{CPC-P03-1}{Eliminar datos del complemento}
\end{itemize}


\subsubsection{\refElem{CU-P04}}

\begin{itemize}
  \TestCase{CPC-P04-1}{Activar complemento}
  \TestCase{CPI-P04-2}{Activar complemento sin tener el complemento de la tienda}
  \TestCase{CPI-P04-3}{Ingresar valores incorrectos}
\end{itemize}


\subsubsection{\refElem{CU-P05}}

\begin{itemize}
  \TestCase{CPC-P05-1}{Desactivar complemento}
  \TestCase{CPI-P05-2}{Desactivar complemento sin tener el complemento de la tienda}
  \TestCase{CPI-P05-3}{Ingresar valores incorrectos}
\end{itemize}



\begin{comment}
\section{Submódulos}

\subsubsection{Esquema configurable}

 Se quiere que el administrador de la página pueda configurar:
 \begin{quote}
 \begin{itemize}
    \item{La ''\hyperref[table:METerminosExperiencia1]{experiencia del nivel}'' del nivel 1.}
    \item {El tipo de incremento.}
    \item {La cantidad de los puntos de experiencia en el incremento.}
    \item {La ''\hyperref[table:METerminosExperiencia1]{experiencia otorgada}'' que da resolver cualquier actividad.}
 \end{itemize}
 \end{quote}

\subsection{Submódulo de Niveles}

 Presenta a los estudiantes su progreso utilizando un sistema de niveles que se van alcanzado
 obteniendo puntos de experiencia. Al alcanzar un nuevo nivel la barra que muestra la
 cantidad de experiencia del nivel se actualizará.
 % y cada vez que se alcanza un nivel, los puntos de experiencia se regresan a cero.

    \begin{quote}
    \begin{description}
    \item[Objetivo] \hfill\\
        Mostrar a los estudiantes el nivel actual de experiencia que tienen y el avance que tienen de ese mismo nivel.
        %Mostrar el nivel actual que tienen los estudiantes, así como el avance que tienen en ese mismo nivel.

        %Proveer información al estudiante que indique la cantidad de tiempo y esfuerzo que le ha dedicado a la plataforma.

    \item[Principios a los que brinda soporte:] \hfill
        \begin{itemize}
            \item 2 \principioII
            \item 6 \principioVI
        \end{itemize}
    \end{description}
    \end{quote}

%\begin{comment}%

\subsection{Submódulo de Barra de Progreso}

Muestra al estudiante el progreso que lleva en un curso usando un valor de 0\% a 100\% dependiendo de los ejercicios que haya hecho del curso o del tiempo que haya transcurrido.

    \begin{quote}
    \begin{description}
        \item[Objetivo] \hfill\\
            Proveer información al estudiante que indique el tiempo y esfuerzo que le ha dedicado a un curso, así como el que le falta por dedicar.

        \item[Principios a los que brinda soporte:] \hfill
        \begin{itemize}
            \item 2 \principioII
        \end{itemize}
    \end{description}
    \end{quote}
%\end{comment}%

\subsection{Diagrama de Clases}

    En la figura \ref{fig:classesXP} se muestra el diagrama de clases, los archivos {\it lib, events, settings, version} y los {\it módulos AMD} son representados mediante el uso de clases. Para facilitar la lectura del diagrama se representa a moodle como un paquete completo, el cual lee los distintos archivos y clases que requiere el plugin para funcionar.

%    \addfigure{1}{diagrams/classesExp}{fig:classesXP}{Diagrama de clases del Módulo de Experiencia}
\clearpage

\subsection{Diagrama de componentes}

    En la figura \ref{fig:bloques1} se muestra el diagrama de componentes del Módulo de experiencia que contiene como interactúa el Módulo con la plataforma Moodle.

%    \addfigure{1}{diagrams/bloques1}{fig:bloques1}{Diagrama de componentes del Módulo de Experiencia}

\clearpage
\subsection*{DS-E2: Crear curso con experiencia}

    Para diseñar la forma en que se ejecuta el caso de uso CU-E2, se tomó en consideración el flujo normal de eventos emitidos cuando se crea un curso en moodle. Los eventos emitidos en orden cronológico son {\it course\_created}, {\it course\_section\_created} y {\it enrol\_instance\_created}.\\

    \noindent En la figura \ref{ds:e2} se detalla la interacción entre el core de moodle, los eventos emitidos, y las clases del plugin {\bf Format Gamedle}.

\end{comment}



% ==================================================================
%   La lista de configuraciones guardadas no pertenece a analisis
%   sino a diseño.
%                   To Do: Mover a diseño en el módulo de experiencia
% ===================================================================

    %\noindent Por el momento se están guardando las siguientes configuraciones:
    %\begin{center}
    %\begin{tabular}{| m{0.25\textwidth} | m{0.30\textwidth} | m{0.15\textwidth} | m{0.18\textwidth}|}\hline
        %{\bf Nombre} & {\bf Descripción} & {\bf Módulo} & {\bf Sub-módulo}  \\\hline

        %Modulo de exp. activo &
        %    Bandera que nos indica si el módulo está activado o no &
        %    Experiencia & Esquema de \newline experiencia\\\hline

        %Tipo de incremento &
        %    Si el incremento de experiencia necesaria por nivel es lineal o porcentual &
        %    Experiencia & Esquema de \newline experiencia \\\hline

        %Cantidad de incremento &
        %    Número decimal que indica cuánto se incrementa la experiencia necesaria por cada nivel &
        %    Experiencia & Esquema de \newline experiencia \\\hline

        %Experiencia por actividad &
        %    Cantidad de experiencia que da cualquier actividad al ser completada &
        %    Experiencia & Esquema de \newline experiencia \\\hline

        %Experiencia del nivel 1 &
        %    Cantidad de experiencia asociada al nivel 1 &
        %    Experiencia & Esquema de \newline experiencia \\\hline

        %Nombre del nivel &
        %    Nombre que reciben los niveles por defecto. &
        %    Experiencia & Niveles \\\hline

        %Mensaje de felicitaciones &
        %    Mensaje que aparece cuando un usuario sube de nivel. &
        %    Experiencia & Niveles \\\hline

        %Descripción del rango &
        %    Mensaje que aparece por defecto, si el nivel no pertenece a ningún rango. &
        %    Experiencia & Niveles \\\hline

        %Imagen del nivel &
        %    Imagen del nivel por defecto (Se almacena la ruta de la imagen, después de haber sido copiada). &
        %    Experiencia & Niveles \\\hline

        %Color del número del nivel &
        %    Color que tendrá el número del nivel por defecto. &
        %    Experiencia & Niveles \\\hline

        %Color de la barra de progreso &
        %    Color que tendrá la barra de progreso de los niveles por defecto. &
        %    Experiencia & Niveles \\\hline
    %\end{tabular}%
    %\end{center}%


 % PASAR EL ANALISIS COMO ANEXO LAS FORMAS NORMALES

 % y finalmente un análisis de las formas normales sobre la base de datos propuesta.\\

 % Con lo anteriormente especificado y de acuerdo a las 2 primeras formas normales
 % especificadas por Edgar Frank Codd \cite[ paǵ. 161-182]{libroBaseDeDatosEspaniol}
 % se diseñó el esquema de la base de datos. A continuación se presentan de manera
 % muy resumida dichas formas normales.


 % CONCLUSIÓN: PASAR AL CAPITULO DE CURVA DE APRENDIZAJE, CÓMO CONCLUSION.

 % \noindent Debido a la capa de abstracción que moodle tiene con respecto al acceso a datos
 % y a que las nuevas funcionalidades se desarrollaran mediante desarrollo de plugins. se
 % decidió utilizar las herramientas que proporciona moodle para la creación de las tablas
 % requeridas para implementar Gamificación.

    %[V] Libro que me pasó CAT (David)

    % Una vez que se decidio moodle y que se realizó el estudio de
    % factibilidad se procedio a documentar el modelo de datos que
    % se usará contemplando los módulos analisados diseñados hasta
    % el momento.

%\par\bigskip\noindent\colorbox{yellow}{%
%{\bf NOTA:} Aquí termina este versión del documento}

\begin{comment}
% =================================================================
%   Pasar las siguientes lineas en los archivos presentes en la
%   carpeta de analisis
% =================================================================

% Y SI SE DOCUMENTA COMO FUE??:
%   TOP-DOWN => PRIMERO SE PLANEARON LOS MÓDULOS Y DESPUES SE BAJO A REQUERIMIENTOS

El alcance de este proyecto es representado mediante el Product Backlog (artefacto de Scrum). El product backlog incluye dos tipos de items: los items de documentación, denotados por ls clave {\it {\bf A}x}; y los items de desarrollo del proyecto, denotados por las claves {\it {\bf RF}x} y {\it{\bf RNF}x}.\\

\noindent A continuación se menciona la lista completa de actividades y requerimientos recopilados, debido
%debido a la larga lista de requerimientos no es posible cumplir para el término del trabajo terminal,
se realizarán los que tienen mayor prioridad. Esta lista puede ser ampliada o reducida bajo indicaciones estrictas de los directores del trabajo terminal.\\

\noindent Al final de este documento se incluye el documento de Metodología como anexo, el cual detalla las características que deben tener los artefactos de Scrum, así como la configuración de Scrum para este proyecto.\\

\section{Product Backlog}

    \begin{multicols}{2}
    
\newcounter{theActivity}\stepcounter{theActivity}
\newcounter{theRF}\stepcounter{theRF}
\newcounter{theRNF}\stepcounter{theRNF}


\newenvironment{Actividad}[2]{\begin{itemize}\item[\bf A\arabic{theActivity}] {\bf #1:} {#2}}{\end{itemize}\stepcounter{theActivity}}

\newenvironment{RF}[2]{
    \begin{itemize}
        \item[\bf \hypertarget{RF\arabic{theRF}}{RF\arabic{theRF}}] {\bf #1:} {#2}
}{
    \end{itemize}
    \stepcounter{theRF}
}

\newenvironment{RNF}[2]{\begin{itemize}\item[\bf RNF\arabic{theRNF}] {\bf #1:} {#2}}{\end{itemize}\stepcounter{theRNF}}
\newcommand{\Sprint}[1]{{\color{primary}\fbox{Sprint #1}}}
\newcommand{\PBitem}{\item[] \quad}

%\noindent\bb{Sprint 1: Marco Teorico, Metodología, Alcance TT-I}
\begin{Actividad}{Investigar Scrum}{%
    Redactar el capítulo de I del documento de metodología el cual describe el marco de trabajo Scrum, basándose en la guía oficial.}
    \PBitem \Sprint{1}%Estimación 2 dias. \Sprint{1}
\end{Actividad}

\begin{Actividad}{Adaptación de Scrum}{%
    Especificar como es configurado Scrum para el proyecto, definir roles, eventos, y artefactos.}
    \PBitem \Sprint{1}%Estimación 2 dias. \Sprint{1}
\end{Actividad}

\begin{Actividad}{Adquirir Actionable Gamificación}{%
    Adquisición del libro de Yu-kai Chou, {\it Actionable Gamification: Beyond Points, Badges, and Leaderboards}}
    \PBitem \Sprint{1}%Estimación 2 dias. \Sprint{1}
\end{Actividad}

\vfill\null\columnbreak

\begin{Actividad}{Investigar Gamificación}{%
    Ampliar la investigación de gamificación, definiciones, sus inicios, uso en la educación.}
    \PBitem \Sprint{1}%Estimación 2 dias. \Sprint{1}
\end{Actividad}

%\begin{Actividad}{Principios de Gamificación}{%
    %Investigar cada uno de los principios de gamificación de acuerdo con el marco de referencia Octalysis,  buscar técnicas para poder soportar los principios.}
    %\PBitem Estimación 2 dias. \Sprint{1}
%\end{Actividad}

\begin{Actividad}{Estado del arte}{%
    Investigar el estado del arte en relación a desarrollo de funcinalidades a una plataforma de aprendizaje.}
    \PBitem \Sprint{1}%Estimación 2 dias. \Sprint{1}
\end{Actividad}

%\begin{Actividad}{Redactar Marco Teórico}{%
    %Invetigar distintos papers que definan la gamificación, describan sus inicios}
    %\PBitem Estimación 2 dias. \Sprint{1}
%\end{Actividad}

\begin{Actividad}{Establecer los módulos}{%
    Plantear una propuesta integral la cual divida en módulos principales las funcionalidades que tendrá el producto final.}
    \PBitem \Sprint{1}%Estimación 2 dias. \Sprint{1}
\end{Actividad}

\pagebreak

\begin{Actividad}{Alcance TT-I}{%
    Definir el alcance que tendrá el proyecto para la presentación del trabajo terminal I.}
    \PBitem \Sprint{2}%Estimación 2 dias. \Sprint{1}
\end{Actividad}

%\hfill\bigskip\\\noindent\bb{Sprint 2: Investigación de Implementación}

\begin{Actividad}{Módulo I y II}{%
    Especificar el funcionamiento y el análisis inicial que se realiza en el módulo de Recompensa.}
    \PBitem \Sprint{2}%Estimación 2 dias. \Sprint{2}
\end{Actividad}

%\begin{Actividad}{Módulo II}{%
    %Especificar el funcionamiento y el análisis inicial que se realiza en el módulo Financiero.}
    %\PBitem Estimación 2 dias. \Sprint{2}
%\end{Actividad}

\begin{Actividad}{Módulo III y IV}{%
    Especificar el funcionamiento y el análisis inicial que se realiza en el módulo de Seguimiento.}
    \PBitem \Sprint{2}%Estimación 2 dias. \Sprint{2}
\end{Actividad}

%\begin{Actividad}{Módulo IV}{%
    %Especificar el funcionamiento y el análisis inicial que se realiza en el módulo de Competencia.}
    %\PBitem Estimación 2 dias. \Sprint{2}
%\end{Actividad}

\begin{Actividad}{Módulo V y VI}{%
    Especificar el funcionamiento y el análisis inicial que se realiza en el módulo de Personalización.}
    \PBitem \Sprint{2}%Estimación 2 dias. \Sprint{2}
\end{Actividad}

%\begin{Actividad}{Modulo VI}{%
    %De la propuesta de solución, describir cada uno de los módulos y herramientas (submódulos) que contienen.}
    %\PBitem Estimación 2 dias. \Sprint{2}
%\end{Actividad}

\begin{Actividad}{Alternativas a Moodle}{%
    Investigar otros sistemas gestores de aprendizaje en los que se puedan desarrollar nuevas funcionalidades.}
    \PBitem \Sprint{2}%Estimación 2 días \Sprint{2}
\end{Actividad}


\begin{Actividad}{Implementación Gamificación}{%
    Investigar distintas publicaciónes (papers) en donde se describa la forma en que se implementó gamificación y los resultados obtenidos}
    \PBitem \Sprint{2}%Estimación 2 días
\end{Actividad}
%trabajo los desarrollos, investigaciones y trabajos ya existentes acerca de la gamificación en una plataforma de aprendizaje.


%\hfill\bigskip\\\noindent\bb{Sprint 3: Reporte Técnico del trabajo Terminal}

\begin{Actividad}{Problema}{%
    Redactar el problema que se pretende atacar con este trabajo terminal.}
    \PBitem \Sprint{3}%Estimación 2 días \Sprint{3}
\end{Actividad}

\begin{Actividad}{Propuesta de Solución}{%
    redactar la propuesta de solución, que se pretendar dar ante el problema}
    \PBitem \Sprint{3}%Estimación 2 días \Sprint{3}
\end{Actividad}

\begin{Actividad}{Justificación}{%
    Redactar por que la justificación de porque surge el proyecto y porque se optó por esa propuesta de solución.}
    \PBitem \Sprint{3}%Estimación 2 días \Sprint{3}
\end{Actividad}

\vfill\null\columnbreak

\begin{Actividad}{Alcances y Limitaciones}{%
    Establecer los alcances y limitaciones que tiene el trabajo terminal}
    \PBitem \Sprint{3}%Estimación 2 días \Sprint{3}
\end{Actividad}

\begin{Actividad}{Instalar Moodle}{%
    Realizar la instalación de Moodle de forma local, en las computadoras de los miembros del equipo.}
    \PBitem \Sprint{3}%Estimación 2 días \Sprint{3}
\end{Actividad}

\begin{Actividad}{Usar Moodle}{%
    Familiarizarse con el uso de Moodle en especifico con las funcioalidades de un administrador, gestionar cursos, gestionar grupos, crear usuarios, etc}
    \PBitem \Sprint{3}%Estimación 2 días \Sprint{3}
\end{Actividad}

%\hfill\bigskip\\\noindent\bb{Sprint 4: Pruebas de Concepto}

\begin{Actividad}{Entorno de desarrollo}{%
    Establecer el entorno de desarrollo sobre el cual se trabajará, incluyendo características de instalación}
    \PBitem \Sprint{4}%Estimación 3 hrs \Sprint{4}
\end{Actividad}

\begin{Actividad}{Filtrar plugins}{%
    Escoger los plugins de los cuales se realizarán las pruebas de concepto y documentar los criterior de discrimnación ocupados.}
    \PBitem \Sprint{4}%Estimación 3 hrs \Sprint{4}
\end{Actividad}

\begin{Actividad}{P1: Database Fields}{%
    Realizar la prueba de database fields}
    \PBitem \Sprint{4}%Estimación 3 hrs \Sprint{4}
\end{Actividad}

\begin{Actividad}{P2: Database Presets}{%
    Realizar la prueba de database presets}
    \PBitem \Sprint{4}%Estimación 3 hrs \Sprint{4}
\end{Actividad}

\begin{Actividad}{P3: User Profile Fields}{%
    Realizar la prueba de user profile fields}
    \PBitem \Sprint{4}%Estimación 3 hrs \Sprint{4}
\end{Actividad}

\begin{Actividad}{P4: Block}{%
    Realizar la prueba de block}
    \PBitem \Sprint{4}%Estimación 3 hrs \Sprint{4}
\end{Actividad}

\begin{Actividad}{Reporte de Pruebas}{%
    Realizar el reporte de pruebas de concepto para entregar al profesor de seguimiento. }
    \PBitem \Sprint{4}%Estimación 3 hrs \Sprint{4}
\end{Actividad}

    
\clearpage

\begin{RF}{Logros en curso}{%
    El sistema debe permitir premiar o otorgar un elemento distintivo (logro) cuando un estudiante realice alguna acción positiva dentro de un curso}
    \item[] Prior. MA. %Estimación 2 días.
\end{RF}

\begin{RF}{Logros en Plataforma}{%
    El sistema debe permitir premiar o otorgar un elemento distintivo (logro) cuando un estudiante realice alguna acción positiva a nivel plataforma}
    \item[] Prior. A. %Estimación 2 días.
\end{RF}

\begin{RF}{Advertencias}{%
    El sistema debe permitir premiar o otorgar un elemento distintivo cuando un estudiante realice alguna acción negativa, como si fuera una advertencia.}
    \item[] Prior. M. %Estimación 2 días.
\end{RF}

\begin{RF}{Marcadores}{%
    El sistema deberá permitir a los usuarios visualizar la lista de los mejores estudiantes respecto al uso en la plataforma (cuantificado mediante los puntos de experiencia), la mejor calificación ponderada, el mayor numero de preguntas diarias, ...}
    \item[] Prior. MA. %Estimación 2 días.
\end{RF}

\begin{RF}{Configurar Logros}{%
    El sistema deberá permitir al administrador cambiar el título, imagen y mensaje de los logros y advertencias que se otorgan a los estudiantes.}
    \item[] Prior. B. %Estimación 2 días.
\end{RF}

\begin{RF}{Habilitar Logros}{%
    El sistema deberá permitir al administrador habilitar y deshabilitar los logros y advertencias que el sistema pone a disposición}
    \item[] Prior. M. %Estimación 2 días.
\end{RF}

\begin{RF}{Experiencia}{%
    El sistema deberá cuantificar como puntos de experiencia, qué tanto usan la plataforma de acuerdo con las actividades/acciones dentro y fuera de los cursos. }
    \item[] Prior. MA. \Sprint{5}%Estimación 2 días.
\end{RF}

\begin{RF}{Configurar Experiencia}{%
    El sistema deberá contar con un mecanismo mediante el cual el administrador defina la cantidad de experiencia que se otorga al terminar un curso y al realizar distintas actividades/acciones.}
    \item[] Prior. A. \Sprint{5}%Estimación 2 días.
\end{RF}

\begin{RF}{Niveles}{%
    El sistema deberá asignar a los estudiantes un nivel de experiencia correspondiente a los incrementos de experiencia configurados y a la cantidad de experiencia recibida. }
    \item[] Prior. A. \Sprint{5}%Estimación 2 días.
\end{RF}

\begin{RF}{Incremento de Niveles}{%
    El sistema deberá permitir al administrador configurar la forma en que incrementan los niveles (lineal o porcentual) y el valor de incremento. }
    \item[] Prior. M. \Sprint{5}%Estimación 2 días.
\end{RF}

\begin{RF}{Configurar Niveles}{%
    El sistema deberá permitir al administrador configurar la imagen, título, descripción y mensaje de los niveles. }
    \item[] Prior. A. \Sprint{5}%Estimación 2 días.
\end{RF}

\begin{RF}{Incrementar Nivel}{%
    El sistema deberá notificar a un estudiante cuando aumente su nivel de experiencia.}
    \item[] Prior. MA. \Sprint{5}%Estimación 2 días.
\end{RF}

\begin{RF}{Progreso}{%
    El sistema deberá mostrarle al un estudiante el progreso que el mismo tiene del curso, mediante una barra que indique el porcentaje que lleva realizado de un curso.}
    \item[] Prior. A. %Estimación 2 días.
\end{RF}

\begin{RF}{Configurar Progreso}{%
    El sistema deberá permitir al administrador o al profesor elegir el color de la barra de progreso para los estudiantes dentro de un curso.}
    \item[] Prior. M. %Estimación 2 días.
\end{RF}

\begin{RF}{Narrativa}{%
    El sistema deberá permitir al administrador y profesores incluir una narrativa que se vaya contando conforme el curso vaya avanzando}
    \item[] Prior. MA. %Estimación 2 días.
\end{RF}

\begin{RF}{Personalización de Curso}{%
    El sistema deberá permitir al administrador o profesor elegir el tema  o visualización del curso que está diseñando. }
    \item[] Prior. M. %Estimación 2 días.
\end{RF}

\begin{RF}{Plantillas de Narrativas}{%
    El sistema deberá brindar al administrador plantillas de narrativas. }
    \item[] Prior. M. %Estimación 2 días.
\end{RF}

\begin{RF}{Personaje de Narrativa}{%
    El sistema deberá permitir al administrador o profesor especificar los datos (nombre,  imagen, etc) de los personajes principales que forman parte de la narrativa de un curso}
    \item[] Prior. A. %Estimación 2 días.
\end{RF}

\begin{RF}{Monedas}{%
    El sistema deberá de contar una moneda principal y otra secundaría para la adquisición de items mediante la tienda. }
    \item[] Prior. MA. %Estimación 2 días.
\end{RF}

\begin{RF}{Configurar Esquema Financiero}{%
    El sistema deberá permitir al administrador indicar la cantidad de monedas que es otorgada en determinadas acciones, el precio que tienen los items de la tienda y las equivalencias entre la moneda principal y secundaria.}
    \item[] Prior. A. %Estimación 2 días.
\end{RF}

\begin{RF}{Tienda}{%
    El sistema deberá de contar con una tienda virtual mediante la cual se puedan adquirir items utilizando las monedas }
    \item[] Prior. MA. %Estimación 2 días.
\end{RF}

\begin{RF}{Añadir Item}{%
    El sistema deberá permitir al administrador añadir items para que estén disponibles en la plataforma, precio de moneda irreal, su categoría y demás atributos. }
    \item[] Prior. A. %Estimación 2 días.
\end{RF}

\begin{RF}{Modificar Item}{%
    El sistema deberá permitir al administrador modificar si el precio de moneta irreal, categoría y demás atributos de los items disponibles en la plataforma. }
    \item[] Prior. A. %Estimación 2 días.
\end{RF}

\begin{RF}{Bloquear Items}{%
    El sistema deberá permitir al administrador bloquear los items para que, posterior a ese momento no se pueda acceder a ellos.}
    \item[] Prior. M. %Estimación 2 días.
\end{RF}

\begin{RF}{Desbloquear Items}{%
    El sistema deberá permitir al administrador desbloquear los items bloqueados para que estos vuelvan a estar disponibles en la plataforma y se pueda acceder a ellos.}
    \item[] Prior. M. %Estimación 2 días.
\end{RF}

\begin{RF}{Exportar Items}{%
    El sistema deberá permitir al administrador exportar los items que ha creado con el propósito de guardarlos para posteriormente ser incluidos en otra plataforma con gamificación}
    \item[] Prior. B. %Estimación 2 días.
\end{RF}

\begin{RF}{Avatar inicial}{%
    El sistema deberá brindarle a los usuarios un avatar inicial y genérico}
    \item[] Prior. MB. %Estimación 2 días.
\end{RF}

\begin{RF}{Configuración Avatar inicial}{%
    El sistema deberá permitir al administrador establecer la apariencia del avatar que se otorga inicialmente a los usuarios }
    \item[] Prior. MB. %Estimación 2 días.
\end{RF}

\begin{RF}{Item: Temas}{%
    El sistema deberá contar con items de tipo tema, los cuales permitan cambiar la visualización que un usuario tiene de la plataforma siempre y cuando tenga dicho item}
    \item[] Prior. MA. %Estimación 2 días.
\end{RF}

\begin{RF}{Item: Skin Avatar}{%
    El sistema deberá contar con items de tipo Skin los cuales contengan un conjunto de elementos que cambien la apariencia del avatar. }
    \item[] Prior. MB. %Estimación 2 días.
\end{RF}

\begin{RF}{Item: Ropa del Avatar}{%
    El sistema deberá contar con items de tipo Ropa, los cuales permitan cambiar una prenda al avatar que los usuarios tienen.}
    \item[] Prior. MB. %Estimación 2 días.
\end{RF}

\begin{RF}{Item: Loot-Box}{%
    El sistema deberá con un tipo de item LootBox el cual otorge cualquier otro items utilizando la probabilidad y aleatoridad de acuerdo con las categorias de los items}
    \item[] Prior. A. %Estimación 2 días.
\end{RF}

\begin{RF}{Configuración de Loot Boxes}{%
    El sistema deberá permitir al administrador cambiar los valores de probabilidad de obtener items de una categoría en especifico mediante un lootBox}
    \item[] Prior. M. %Estimación 2 días.
\end{RF}

\begin{RF}{Monedas en Curso}{%
    El sistema deberá permitir al administrador/profesor ponerle un alias (nombre e imagen) a las monedas (principal y secundaria) dentro de un curso.}
    \item[] Prior. M. %Estimación 2 días.
\end{RF}

\begin{RF}{Personalización}{%
    El sistema debe contar con una página de personalización donde el usuario pueda configurar su avatar, además de la visualización que el tiene de la plataforma.}
    \item[] Prior. M. %Estimación 2 días.
\end{RF}

\begin{RF}{Retar a compañero}{%
    El sistema deberá permitir a un estudiante desafiar a otro a una sesión de preguntas acerca de los temas de un curso que tengan en común. }
    \item[] Prior. M. %Estimación 2 días.
\end{RF}

\begin{RF}{Apuestas en retos}{%
    El sistema deberá permitir a los competidores %que participan
    apostar una cantidad en mutuo acuerdo entre los estudiantes que forma parte de un reto. }
    \item[] Prior. A. %Estimación 2 días.
\end{RF}

\begin{RF}{Retar al sistema}{%
    El sistema deberá permitir a un estudiante desafiar al sistema a una sesión de preguntas, escogiendo un nivel de dificultad }
    \item[] Prior. A. %Estimación 2 días.
\end{RF}

\begin{RF}{Recompensas en retos con el sistema}{%
    El sistema deberá dar recompensas de acuerdo con el nivel de dificulta elegido por el estudiante.}
    \item[] Prior. MA. %Estimación 2 días.
\end{RF}

\begin{RF}{Torneos}{%
    El sistema deberá permitir al profesor organizar un torneo entre los estudiantes de un curso, con el propósito de comparar el aprovechamiento de los estudiantes}
    \item[] Prior. A. %Estimación 2 días.
\end{RF}

\begin{RF}{Recompensa en Torneos}{%
    El sistema deberá otorgar una recompensa al primer, segundo y tercer lugar, distintiva. }
    \item[] Prior. A. %Estimación 2 días.
\end{RF}

\begin{RF}{Poker}{%
    El sistema deberá permitir a los usuarios iniciar una sesión de poker entre distintos estudiantes, donde los mismos puedan apostar las monedas ficticias del sistema.}
    \item[] Prior. M. %Estimación 2 días.
\end{RF}



\begin{RF}{Animación de Personajes}{%
    El sistema deberá de contener animaciones para los distintos elementos de interfaz de usuario. }
    \item[] Prior. MB. %Estimación 2 días.
\end{RF}

\begin{RNF}{Bajo Acoplamiento}{%
    El sistema deberá trabajar con el menor acoplamiento}
    \item[Tipo] Propiedad de Software
    \item[] Prior. A. %Estimación 2 días. \Sprint{*}
\end{RNF}

\begin{comment}
\begin{RNF}{Robustez}{%
    El sistema debe, INSERTAR AQUI MÉTRICA DE ROBUSTEZ. }
    \item[Tipo] Propiedad de Software
    \item[] Prior. MA. %Estimación 2 días. \Sprint{*}
\end{RNF}
\end{comment}

\begin{RNF}{Modularidad}{%
    El sistema deberá permitir al administrador habilitar únicamente las herramientas que el decida incluir en la plataforma, y deshabilitar las que no requiera.}
    \item[Tipo.] Regla de Negocio
    \item[] Prior. A. %Estimación 2 días. \Sprint{*}
\end{RNF}

\begin{RF}{Preguntas diarias}{%
    El sistema deberá contar con un ejercicio que podrá ser contestado diariamente por el estudiante.}
    \item[] Prior. A. %Estimación 2 días.
\end{RF}
    \end{multicols}

    %\begin{quote}
    %\noindent {\bf Nota:} El número de {\it Sprint} debe estar presente en todos los items correspondientes al sprint corriente y a los sprints anteriores a este. El atributo {\it Sprint} puede no estar presente en los items que no han sido vinculados a un Sprint.
    \noindent {\bf Nota:} El número de {\it Sprint} debe estar presente en todos los ítems que ya hayan sido agregados a un Sprint Backlog.
    %\end{quote}

\section{Relación módulos-requerimientos}

En la tabla \ref{tab:modreq} se relacionan lo módulos definidos contra los requerimientos encontrados en el Backlog, se muestran sólo los identificadores de los requerimientos para una mayor legibilidad.

\newcommand{\Refr}[1]{{\hyperlink{#1}{#1}}}
\begin{table}[h!]
    \centering
    \begin{tabular}{|c|c|}
    \hline
        Competencia & \Refr{RF36}, \Refr{RF37}, \Refr{RF38}, \Refr{RF40}, \Refr{RF41}, \Refr{RF39}, \Refr{RF42}\\
    \hline
        Seguimiento & \Refr{RF13}, \Refr{RF14}, \Refr{RF44}\\
    \hline
        Financiero & \Refr{RF19}, \Refr{RF20}, \Refr{RF21}, \Refr{RF22}, \Refr{RF23}, \Refr{RF24}, \Refr{RF25}, \Refr{RF26}, \Refr{RF32}, \Refr{RF33}\\
    \hline
        Experiencia & \Refr{RF9}, \Refr{RF7}, \Refr{RF8}, \Refr{RF10}, \Refr{RF11}, \Refr{RF12}\\
    \hline
        Recompensa & \Refr{RF1}, \Refr{RF2}, \Refr{RF4}, \Refr{RF3}, \Refr{RF5}, \Refr{RF6}\\
    \hline
        Personalización & \Refr{RF15}, \Refr{RF16}, \Refr{RF17}, \Refr{RF18}, \Refr{RF27}, \Refr{RF28}, \Refr{RF29}, \Refr{RF30}, \Refr{RF31}, \Refr{RF34}, \Refr{RF35}\\
    \hline
    \end{tabular}
    \caption{Relación entre los módulos y requerimientos}
    \label{tab:modreq}
\end{table}


\end{comment}
