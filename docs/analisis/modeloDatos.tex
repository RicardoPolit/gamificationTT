\chapter{Modelo de Dominio de Datos}
\label{ch:dominioDatos}

 Este capítulo describe las decisiones más relevantes tomadas en relación al esquema de
 base de datos de moodle, el modelo de información que le brinda soporte a la persistencia
 de los datos usados y requeridos por los módulos y submódulos previamente establecidos, 
 finalmente se presenta el diccionario de datos que contiene la especificación de los
 atributos de las distintas relaciones requeridas.\\

 \noindent Debido a que el esquema de base de datos de moodle contiene alrededor de 400
 relaciones, solo se hace mención de aquellas relevantes para el desarrollo de este proyecto,
 cabe recalcar que la información recopilada de la base de datos de moodle fue obtenida
 directamente del esquema de base de datos, ya que no existe documentación oficial acerca
 del esquema.

\section{Esquema de la base de datos}

 % DONDE SE DOCUMENTA QUE SE VA A USAR MYSQL ???

 En la figura \ref{fig:EGBD} se muestran aquellas relaciones relevantes para el desarrollo
 de este proyecto, tanto del esquema de la base de datos de moodle cómo de las relaciones
 que fueron requeridas crear, el diagrama presentado contiene las relaciones agrupadas
 de la siguiente forma:\\
  
    \begin{quote}
    \begin{description}
        \item[Relaciones de Moodle] Contiene un subconjunto de las relaciones del esquema de
            base de datos de moodle que son relevantes para el desarrollo de este proyecto.\\

        \item[General] Contiene a todas las relaciones que no pertenecen propiamente a un
            módulo pero que son esenciales para el funcionamiento de los distintos módulos
            propuestos.\\

    \end{description}
    \end{quote}

 \noindent Las demás relaciones se encuentran organizadas en seís grupos correspondientes
 a los módulos de este proyecto (\nameref{mod:exp}, \nameref{mod:recomp}, \nameref{mod:financ},
 \nameref{mod:pers}, \nameref{mod:comp} y \nameref{mod:seguim}).

 \clearpage

 \noindent Todas las relaciones creadas para este proyecto fueron diseñadas tomando en
 consideración que se debía extender el esquema de base de datos sin afectar las distintas
 funcionalidades que brinda de forma nativa. En los casos en los que se requería añadir
 atributos de moodle a una relación en particular, se agregaría una relación con los atributos
 requeridos, más la referencia a la relación del esquema de moodle.

    \addfigure{1}{analisis/diagrams/db}{fig:EGBD}{Esquema de la base de datos}

\clearpage
\section{Diccionario de datos}

 En esta sección se detalla el diccionario de datos del proyecto, en él se describe cada una
 de las entidades pertenecientes al modelo de información, así como el nombre, tipo de dato,
 descripción y restricciones sobre los atributos de las entidades.\\

 A continuación se listan y describen los tipos de
 datos usados en el diccionario.

    \begin{quote}
    \begin{bGlosario}
        \bTerm{tVarchar}{varchar} Cadena de caracteres de longitud variable
        \bTerm{tInt}{int}         Numero enteros incluyendo, positivos y negativos.
    \end{bGlosario}
    \end{quote}

 Además de los tipos de dato, se definen las
 siguientes restriccioness:
    
    \begin{quote}
    \begin{bGlosario}
    
        \bTerm{tRequerido}{Requerido}
            Indica que el dato es obligatorio y no puede existir un registro sin que tenga
            un valor.
            
        \bTerm{tUnico}{Único}
            Indica que que los valores para un atributo o combinacion de estos no se deben
            repetir en los diferentes registros.
            
        \bTerm{tPrimaryKey}{Llave primaria}
            Indica el atributo o combinacion de estos que seran la llave primaria. La llave
            primaria es el identificador unico de cada registro dentro de la tabla, por lo
            que de no puede haber duplicados y es obligatorio que tenga un valor.
            
        \bTerm{tForeignKey}{Llave foranea}
            Indica el atributo o conjunto de estos que son llaver foranea y apuntan a otra
            tabla, esto es, sus valores posibles Sólo pueden ser aquellos que existan en la
            llave primaria a la que apuntan. FK(Tabla.atributo).
            
        \bTerm{tRango}{Rango}(a, b)
            Indica que el numero de valores esta restringido del valor $a$ al $b$.

        \bTerm{tAutoIncrement}{Auto-incremental}
            Indica que el número se incrementa en uno al insertarse un nuevo registro.

        \bTerm{tDefault}{Default} $:\ valor$
            Indica el valor por defecto que tendrá el atributo
            
    \end{bGlosario}
    \end{quote}
    
\clearpage

\section{Relaciones de moodle}

 A continuacíón se presenta la especificación de las relaciones del esquema de base
 de datos de moodle que son relevantes para el desarrollo de los módulos y submódulos
 de proyecto.

    \begin{cdtEntidad}{Mdl-configuration}{Configuración de Plugin}{%
    Es una tabla del nucleo de moodle que almacena todas las configuraciones globales
    relacionadas a los plugins instalados, al iniciar moodle las configuraciones de los
    plugins instalados y habilitados se cargan en memoria.}

	    \brAttr{id}{Id}{tInt}{%
	        Es el dígito que representa el identificador único para una configuración
            específica de un plugin.\par

            \it Restricciones:
            \refElem{tPrimaryKey},
            \refElem{tAutoIncrement}.
        }

        \brAttr{plugin}{Plugin}{tVarchar}{%
            Cadena de caracteres del nombre identificador del plugin al cual pertenece
            la configuración.\par

            \it Restricciones:
            \refElem{tRequerido},
            \refElem{tRango} (0,100),
            \refElem{tUnico}
        }
		
        \brAttr{name}{Nombre}{tVarchar}{%
            Cadena de caracteres que representa el nombre de la configuración de un
            plugin en específico.\par

            \it Restricciones:
            \refElem{tUnico},
            \refElem{tRango} (0,100),
            \refElem{tRequerido}
        }

        \brAttr{value}{Valor}{tVarchar}{%
            Cadena que almacena el valor de una configuración perteneciente a alguno
            de los plugins instalados.\par
            
            \it Restricciones:
            \refElem{tRango} (0,4294967295),
            \refElem{tRequerido}
        }
	%\cdtEntityRelSection
	%\brRel{\brRelComposition}{Ejemplo}{Escribe el enunciado que denota la composición.}
	%\brRel{\brRelAgregation}{Ejemplo}{Tambien puede ser con agregaciones.}
	%\brRel{\brRelGeneralization}{Ejemplo}{De la misma forma con Generalizaciones.}
    \end{cdtEntidad}
    \vspace{-1em}\hfill Nombre en el esquema: {\it mdl\_configuration}\ \ \par%


    \begin{cdtEntidad}{mdl-usuario}{Usuario de moodle}{%
    Es una tabla del núcleo de moodle que contiene toda la información que se
    almacena de los usuarios en la plataforma, independientemente del rol que
    estos contenga, esta relación contiene más de 53 atributos, sin embargo solo
    se detallan aquellos relevantes.}

	    \brAttr{id}{Id}{tInt}{%
	        Es el dígito que representa el identificador único para cada uno 
            de los usuarios en moodle.\par

            \it Restricciones:
            \refElem{tPrimaryKey},
            \refElem{tAutoIncrement}.
        }

    \end{cdtEntidad}
    \vspace{-1em}\hfill Nombre en el esquema: {\it mdl\_configuration}\ \ \par%

\begin{comment}

    \begin{Entidad}{usuario}{%
            \attrM{id\_usuario}
            {Atributo que relaciona un usuario de Moodle con uno nuestro}
                \Titem{ Llave foránea 
                \Titem{ No nulo}
            {General}
            
            \attr{nivel\_actual}
            {Nivel actual del usuario.}
                \Titem{ No nulo.}
            {Experiencia}
            
            \attr{experiencia\_ actual}
            {Experiencia actual del nivel del usuario.}
                \Titem{ Su valor es $ \geq 0$}
                \Titem{ No nulo.}
            {Experiencia}
    \end{Entidad}
    
    \begin{Entidad}
        {alumno}
            \attrG{id\_ usuario}
            {Atributo que relaciona un usuario con un papel de alumno. }
                \Titem{ Llave foránea a \textbf{gmdl\_usuario (mdl\_id\_ usuario)}}
                \Titem{ No nulo}
            {General}
            
            \attrM{id\_ curso}
            {Atributo que relaciona un curso con un papel de alumno.}
                \Titem{ Llave foránea  a \textbf{mdl\_course (id)}.}
                \Titem{ No nulo.}
            {General}
            
            \attr{experiencia\_ total\_ recibida}
            {Toda la experiencia que ha recibido un usuario haciendo las actividades de un determinado curso.}
                \Titem{ Su valor es $ \geq 0$}
                \Titem{ No nulo.}
            {Experiencia}
    \end{Entidad}

 \noindent La combinación de los atributos ( \textbf{gmdl\_id\_usuario } y \textbf{mdl\_id\_curso} ) es un índice único. 
    
    \begin{Entidad}
        {nivel\_categoria\_curso}
            \attrG{id\_ usuario}
            {Atributo que relaciona un usuario con un nivel. }
                \Titem{ Llave foránea  a \textbf{gmdl\_usuario (mdl\_id\_usuario)}}
                \Titem{ No nulo}
            {General}
            
            \attrM{id\_ categoria\_ curso}
            {Atributo que relaciona un nivel con una categoría de curso de Moodle.}
                \Titem{ Llave foránea  a \textbf{mdl\_course\_categories (id)}.}
                \Titem{ No nulo.}
            {General}
            
            \attr{nivel\_actual}
            {Nivel actual del usuario en una categoría de cursos.}
                \Titem{ No nulo.}
            {Experiencia}
            
            \attr{experiencia\_ actual}
            {Número entero  \newline \textbf{Tamaño:}\newline 10 dígitos }
            {Experiencia actual del usuario de su nivel actual en una categoría de cursos.}
                \Titem{ Su valor es $ \geq 0$}
                \Titem{ No nulo.}
            {Experiencia}
    \end{Entidad}

 \noindent La combinación de los atributos ( \textbf{gmdl\_id\_usuario } y \textbf{mdl\_categoria\_curso} ) es un índice único. 
    
    \begin{Entidad}
        {rango\_nivel}
            
            \attr{nombre}
            {Nivel actual del usuario en una categoría de cursos.}
                \Titem{ No nulo.}
            {Experiencia}
            
            \attr{nivel\_inferior}
            {Nivel mínimo requerido que puede tener un usuario para estar dentro del rango.}
                \Titem{ No nulo.}
            {Experiencia}
            
            \attr{nivel\_superior}
            {Nivel máximo que puede tener un usuario para estar dentro del rango.}
                \Titem{ No nulo.}
            {Experiencia}
            
            \attr{imagen}
            {Ruta donde se guarda la imagen asociada al rango.}
                \Titem{ No nulo.}
            {Experiencia}
            
            \attr{mensaje}
            {Mensaje de felicitaciones que se le muestra al usuario al subir de nivel.}
                \Titem{ No nulo.}
            {Experiencia}
            
            \attr{descripcion}
            {Descripción del rango, que se le muestra al usuario al subir de nivel.}
                \Titem{ No nulo.}
            {Experiencia}
            
    \end{Entidad}

    \begin{Entidad}
        {logro}
            \attr{icono}
            {Ruta donde se guardará el icono del logro..}
                \Titem{ No nulo.}
            {Recompensa}
            
            \attr{nombre}
            { Nombre  \textbf{único} del logro.}
                \Titem{ Índice único.}
                \Titem{ No nulo.}
            {Recompensa}
            
            \attr{descripción}
            {Descripción que le indica al usuario, cómo se desbloquea el logro.}
                \Titem{ No nulo.}
            {Recompensa}
            
            \attr{modulo}
            {Nombre del módulo al que pertenece el logro.}
                \Titem{ No nulo.}
            {Recompensa}
            
            \attr{experiencia\_ de\_logro}
            {Experiencia que otorga el logro al ser desbloqueado.}
                \Titem{ No nulo.}
            {Experiencia}
            
            \attr{tipo}
                Caracter que nos indica que tipo de logro es.
                    \newline ''A'': El logro es a nivel curso.
                    \newline ''B'': El logro es a nivel curso y es una advertencia.
                    \newline ''C'': El logro es a nivel plataforma.
                    \newline ''D'': El logro es a nivel plataforma y es una advertencia.
                \Titem{ Sus únicos valores posibles son: \{'A','B','C','D'\}.}
                \Titem{ No nulo.}
            {Recompensa}
            
    \end{Entidad}
    
    \begin{Entidad}
        {evento}
            
            \attr{nombre\_ evento}
            { Nombre  \textbf{único} del evento.}
                \Titem{ Índice único.}
                \Titem{ No nulo.}
            {Recompensa}
            
    \end{Entidad}
    
    \begin{Entidad}
        {evento\_logro}
            
            \attrG{id\_ logro}
            { Atributo que relaciona un logro con una lista de eventos. }
                \Titem{ Llave foránea a \textbf{gmdl\_logro(id)}}
                \Titem{ No nulo}
            {General}
            
            \attrG{id\_ evento}
            { Atributo que relaciona un evento con una lista de logros.}
                \Titem{ Llave foránea a \textbf{gmdl\_evento(id)}}
                \Titem{ No nulo}
            {General}
            
    \end{Entidad}
    
    \noindent La combinación de los atributos ( \textbf{gmdl\_id\_logro } y \textbf{gmdl\_id\_evento } ) es un índice único. 
    
    \begin{Entidad}
        {condicion}
            
            \attr{tabla}
            { Nombre de la tabla, la cual contiene la información para saber si la condición se cumple o no.}
                \Titem{ No nulo.}
            {Recompensa}
            
            \attr{atributo}
            { Nombre del atributo que se usará para ver si la condición se cumple o no.}
                \Titem{ No nulo.}
            {Recompensa}
            
            \attr{expresión}
            { Expresión a usar con el atributo y el valor.}
                \Titem{ No nulo.}
            {Recompensa}
            
            \attr{valor}
            {Valor que establece la referencia para saber si la condición de cumple o no. }
                \Titem{ No nulo.}
            {Recompensa}
            
    \end{Entidad}
    
    \begin{Entidad}
        {condicion\_logro}
            
            \attrG{id\_ logro}
            {Atributo que relaciona a un logro con una lista de condiciones. }
                \Titem{ Llave foránea a \textbf{gmdl\_logro(id)}}
                \Titem{ No nulo}
            {General}
            
            \attrG{id\_ condicion}
            {Atributo que relaciona a una condición con una lista de logros. }
                \Titem{ Llave foránea a \textbf{gmdl\_condicion(id)}}
                \Titem{ No nulo}
            {General}
            
    \end{Entidad}

    \noindent La combinación de los atributos ( \textbf{gmdl\_id\_logro } y \textbf{gmdl\_id\_condicion } ) es un índice único. 
    
    \begin{Entidad}
        {usuario\_logro\_curso}
            
            \attrG{id\_ logro}
            {Atributo que relaciona un logro con usuarios y cursos. }
                \Titem{ Llave foránea a \textbf{gmdl\_logro(id)}.}
                \Titem{ No nulo.}
            {General}
            
            \attrG{id\_ usuario}
            {Atributo que relaciona a un usuario con logros y cursos. }
                \Titem{ Llave foránea a \textbf{gmdl\_usuario(mdl\_ id\_usuario)} .}
                \Titem{ No nulo.}
            {General}
            
            \attrM{id\_ curso}
            {Atributo que relaciona a un curso con usuarios y logros. }
                \Titem{ Llave foránea a \textbf{mdl\_curso(id)}.}
                \Titem{ No nulo.}
            {General}
            
            \attr{cuando}
            {  Fecha que nos indica cuándo el logro fue desbloqueado.        }
                \Titem{ No nulo.}
            {Recompensa}
            
    \end{Entidad}
    
    \begin{Entidad}
        {usuario\_logro\_global}
            
            \attrG{id\_ logro}
            {Número entero positivo. \newline {\bf Tamaño:}\newline 10 dígitos}
            {Atributo que relaciona un logro con usuarios y cursos. }
                \Titem{ Llave foránea a \textbf{gmdl\_logro(id)}.}
                \Titem{ No nulo.}
            {General}
            
            \attrG{id\_ usuario}
            {Atributo que relaciona a un usuario con logros y cursos. }
                \Titem{ Llave foránea a \textbf{gmdl\_usuario(mdl\_ id\_usuario)} .}
                \Titem{ No nulo.}
            {General}
            
            \attr{cuando}
            {  Fecha que nos indica cuándo el logro fue desbloqueado.        }
                \Titem{ No nulo.}
            {Recompensa}
            
    \end{Entidad}

\section{Formas normales}

    %En esta sección se analiza el cumplimiento de las seis formas normales propuestas por Frank Codd y de la forma normal Boyce-Codd, con el propósito a que se quiere reducir en lo mayor posible la redundancia de datos. Si existe alguna forma normal que no sea cumplida, %se analizarán los cambios requeridos para su cumplimiento. En caso de que no se pueda, se especificará el por qué.  
%
    Debido a que las pautas de Moodle entorpecen el diseño de la base de datos, se analizará si afectan en redundancia. Para ello se tomarán en cuenta las 6+2 formas normales con el esquema de la base de datos propuesto.
%   
\subsection*{Primera forma normal}
    
    Cada tabla en un esquema de base de datos se considera en primera forma normal si y solo si, cumple las siguientes condiciones \cite[pág. 154]{libroBaseDeDatosIngles}: 
    
    \begin{enumerate}
        \item Tiene un dominio atómico. Un dominio es atómico si y solo si, cada elemento del dominio es indivisible \cite[pág. 161]{libroBaseDeDatosEspaniol}.
        \item Cada registro debe de tener el mismo número de valores.
        \item Cada registro debe ser único.
    \end{enumerate}  
    
    \noindent Un ejemplo de un dominio no atómico es tener todos los teléfonos de un cliente en un mismo elemento. "\textit{Teléfono\_A},  \textit{Teléfono\_B}".  Haciendo lo anterior el elemento puede albergar más de un valor, haciendo que ya no sea indivisible.\\
    
    \noindent Continuando con el ejemplo anterior, si en lugar de guardar todos los teléfonos en un atributo, se crean más atributos para poder guardar todos los teléfonos del cliente que tiene más teléfonos. Se incumple con el segundo punto de la primera forma normal.\\
    
    
    \noindent \textbf{Se cumple} con esta forma normal en nuestro esquema de la base de datos porqué no existe ningún atributo en ninguna tabla que no sea atómico.
    
    
\subsection*{Segunda forma normal}
    
    Cada tabla en un esquema de base de datos se considera en segunda forma normal si y solo si, cumple las siguientes 2 condiciones \cite[pág. 159]{libroBaseDeDatosIngles}:
    \begin{enumerate}
        \item Cumple con la primera forma normal.
        \item Todo atributo que no forma parte de la clave primaria debe depender funcionalmente de toda la clave primaria.
    \end{enumerate}
    
    \noindent \textbf{Se cumple} con esta forma normal gracias a que no se cuenta con ninguna clave primaria compuesta. Esto gracias a las pautas de moodle, que nos hacen diferenciar un registro únicamente con un número entero positivo. 
    
\clearpage    
\subsection*{Tercera forma normal}
    
    Cada tabla en un esquema de base de datos se considera en tercera forma normal si y solo si, cumple con las siguientes 2 condiciones \cite[pág. 163]{libroBaseDeDatosIngles}:
    
    \begin{enumerate}
        \item Cumple con la segunda forma normal.
        \item No existen dependencias funcionales transitivas a la clave primaria.
    \end{enumerate}
    
    \noindent \textbf{No se cumple} con esta forma normal en las siguientes tablas de la base de datos.
    \begin{itemize}
        \item \textbf{gmdl\_usuario}
        \item \textbf{gmdl\_alumno}
        %\item \textbf{gmdl\_logro}
        %\item \textbf{gmdl\_condicion}
        %\item \textbf{gmdl\_usuario\_logro\_curso}
        %\item \textbf{gmdl\_usuario\_logro\_global}
        %\item \textbf{gmdl\_evento\_logro}
        %\item \textbf{gmdl\_condicion\_logro}
    \end{itemize}
  
    \noindent Se tiene el mismo problema en cada una de las tablas anteriores, porque en sus atributos contienen una llave candidata que terminó no formando parte de la llave primaria. Gracias a esto se crea una dependencia funcional a la llave candidata y la llave candidata depende funcionalmente de la llave primaria, generando así el no cumplimiento de esta forma normal.
    
    \noindent Dicho problema se puede solventar tomando en consideración la forma normal de Boyce-Codd, es por eso que no se realizarán cambios a dichas tablas de la base de datos.
 
\subsection*{Forma normal de Boyce-Codd (FNBC)}
    
    Cada tabla en un esquema de base de datos se considera en la forma normal de Boyce-Codd si y solo si, cumple con las siguientes condiciones \cite[pág. 168]{libroBaseDeDatosIngles}:
    
    \begin{enumerate}
        \item Todos los atributos no claves de la tabla deben depender funcionalmente de toda una clave candidata.
        \item Toda dependencia funcional de la tabla debe ser hacia una clave candidata.
    \end{enumerate}
    
    \noindent Esta forma normal extiende las anteriores formas normales diciendo que en una entidad puede haber más de una clave candidata y que todos los atributos de esa entidad deben depender de una de esas claves. Gracias a esto se considera a la forma normal de Boyce-Codd como una alternativa a la segunda y tercer forma normal.\\
    
    
    \noindent \textbf{Se cumple} con la forma normal de Boyce-Codd. Si recordamos las tablas que no cumplieron con la tercera forma normal, tienen el mismo problema de tener una dependencia funcional hacia la llave candidata, y con esta forma normal no existe ese problema. \\
    
     
     
    
\clearpage    
\subsection*{Cuarta forma normal}
    
    Cada tabla en un esquema de base de datos se considera en cuarta forma normal si y solo si, cumple con las siguientes 2 condiciones \cite[pág. 182]{libroBaseDeDatosIngles}:
    
    \begin{enumerate}
        \item Cumple con la tercera forma normal o con la forma normal de Boyce-Codd.
        \item No existen dependencias multivaloradas.
    \end{enumerate}
    
    \noindent Una dependencia multivalorada $ A \twoheadrightarrow B$ se cumple si en una relación legal \textit{r(R)}, para todo par de tuplas $t_1$ y $t_2$    tales que $t_1[A] = t_2[A]$, existen unas tuplas $t_3$ y $t_4$ de \textit{r} tales que \cite[pág. 180]{libroBaseDeDatosEspaniol}:
    
    \begin{center}
           $t_1[A] = t_2[A] = t_3[A] = t_4[A]$\\
           $t_3[B] = t_1[B]$\\
           $t_3[R - B] = t_2[R - B]$\\
           $t_4[B] = t_2[B]$\\
           $t_4[R - B] = t_1[R - B]$
    \end{center}
    
    \noindent En otras palabras, si existe una relación ternaria cuyas claves foráneas A, B y C, donde C y B están relacionadas con A, pero son independientes entre sí, dicha relación tiene una dependencia multivalorada \cite[pág. 118]{libroBaseDeDatosInglesCuarteEnAdelante}.\\
    
    \noindent \textbf{Se cumple} con esta forma normal, debido a que no se cuentan con relaciones ternarias en la base de datos.
    
\end{comment}

\begin{comment}
    \noindent Si se ve nuestro esquema de la base de datos, solo se cuenta con una relación que contempla 3 tablas que es \textbf{gmdl\_usuario\_logro\_curso}. Sus claves foráneas son \textbf{gmdl\_id\_curso} que hace referencia a un curso, \textbf{gmdl\_id\_logro} que hace referencia a un logro y \textbf{gmdl\_id\_usuario} que hace referencia a un usuario.\\
    
    \noindent En dicha tabla, las claves se relacionan de la siguiente manera:
    \begin{itemize}
        \item Un usuario desbloquea logros de un curso.
        \item Un curso cuenta con logros que fueron desbloqueados por usuarios.
        \item Un logro fue desbloqueado por usuarios en uno o más cursos.
    \end{itemize}
    
    \noindent \textbf{Se cumple} con la cuarta forma normal. Observando las relaciones anteriores podemos observar que no existe una dependencia multivalorada porque cada atributo se relaciona/depende de los otros 2.
\end{comment}    
    
 
\begin{comment}
\subsection*{Quinta forma normal}

    Cada tabla en un esquema de base de datos se considera en quinta forma normal si y solo si, cumple con las siguientes 2 condiciones \cite[pág. 124]{libroBaseDeDatosInglesCuarteEnAdelante}:
    
    \begin{enumerate}
        \item Cumple con la cuarta forma normal.
        \item Todas las dependencias de unión ''JOIN'' están implicadas por las claves candidatas.
    \end{enumerate}
    
    \noindent Una dependencia de unión ''JOIN'' ocurre cuando una tabla puede volver a ser formada correctamente (refiriéndonos a que los registros permanezcan intactos) uniendo 2 o más tablas cuyos atributos sean de la tabla original \cite[pág. 124]{libroBaseDeDatosInglesCuarteEnAdelante}.\\
    
    \noindent Respecto a que las uniones ''JOIN'' estén implicadas por las claves candidatas, se refiere a que los atributos que se utilicen como pivote (para hacer dicha separación) no sean otros que los que son claves candidatas.\\
    
\end{comment}
    
\begin{comment}
    \noindent  La única tabla que puede llegar a darnos problemas con esta forma normal es la misma que se analizó en la cuarta forma normal. Para analizar si esta tabla (\textbf{gmdl\_usuario\_logro\_curso}) cumple con esta forma tenemos que listar sus atributos.
    \begin{itemize}
        \item id
        \item gmdl\_id\_logro
        \item gmdl\_id\_usuario
        \item gmdl\_id\_curso
        \item cuando
    \end{itemize}
    
    \noindent  Las únicas 2 formas en las que se puede dividir la anterior tabla, conservando los mismos registros son las siguientes:
    \begin{multicols}{2}
    
    \noindent Utilizando como pivote el atributo ''id''. 
    \begin{itemize}
        \item A1(id, gmdl\_id\_usuario)
        \item A2(id, gmdl\_id\_logro)
        \item A3(id, gmdl\_id\_curso)
        \item A4(id, cuando)
    \end{itemize}
    \noindent \textbf{Nota:} Se pueden desarrollar distintas combinaciones con lo anterior, sin embargo, siempre se usa como pivote al atributo ''id'' es por eso que no se contemplan.
    \columnbreak
    
    \noindent Utilizando como pivote la llave candidata ''gmdl\_id\_usuario, gmdl\_id\_logro , gmdl\_id\_curso''.
    \begin{itemize}
        \item B1(id, gmdl\_id\_usuario, gmdl\_id\_logro , gmdl\_id\_curso)
        \item B2(gmdl\_id\_usuario, gmdl\_id\_logro , gmdl\_id\_curso, cuando)
    \end{itemize}
    
    \end{multicols}
\end{comment}    
    
\begin{comment}
    \noindent \textbf{Se cumple} con esta forma normal, debido a que ninguna de las tablas en el esquema puede ser dividida sin contemplar como pivote a una clave candidata.
    
    
\clearpage
\subsection*{Sexta forma normal}
    
    No se cuenta con una definición formal de esta forma normal.\\
    
    \noindent La sexta forma normal trata de datos que son temporales, refiriéndose a que estos pueden cambiar en un futuro. Siendo más específicos, si la base de datos necesita llevar un historial para poder hacer estadísticas, dichas estadísticas no deberían presentar diferentes resultados sin importar cuándo se consulten \cite[pág. 125-126]{libroBaseDeDatosInglesCuarteEnAdelante}.\\
    
    \noindent Lo anterior se puede ver más a detalle con el ejemplo de; ¿cuánto se ha vendido el mes ''X''?.
    
    \noindent Si el único atributo que se tiene para calcular lo anterior es el ''precio'' en una tabla ''producto''. Esto incumple la sexta forma normal, ya que, cuando el precio de dicho producto cambie, la consulta para determinar cuánto se ha ganado el mes ''X'' con ese producto también cambiaría.\\
    
    
    \noindent \textbf{Se cumple} con la sexta forma normal, debido a que no tenemos datos temporales que puedan llegar a afectar de esa forma a nuestra interpretación de los datos. \\
    
    \noindent Los únicos datos temporales que se tienen son; El nivel\_actual y la experiencia\_actual tanto del usuario como del tipo de categoría del curso. Ambos en sus sendas tablas \textbf{gmdl\_usuario} y \textbf{gmdl\_nivel\_categoria\_curso}
    
    
\subsection*{Forma normal de dominio clave}

    Una tabla cumple con la forma normal de dominio/clave si todas las restricciones que se tienen en los datos están asociados con una clave o un dominio \cite[p. 193]{libroBaseDeDatosIngles}.
    
    \begin{itemize}
        \item Un dominio es cualquier limitación que se tengan en los datos para ser guardados en una cierta columna en una base de datos. Esto refiriéndonos a las limitaciones como las llaves foráneas o el tipo de dato.
        \item La clave es una clave única.
    \end{itemize}
    
    \noindent Esta forma normal es considerada como la perfecta forma normal.\\
    
    \noindent \textbf{No se cumple} con esta formal, gracias a los atributos nivel\_actual y la experiencia\_actual tanto del usuario como del tipo de categoría del curso. Ambos en sus sendas tablas \textbf{gmdl\_usuario} y \textbf{gmdl\_nivel\_categoria\_curso}.\\ 
    
    \noindent El atributo experiencia\_actual tiene un límite superior dependiendo del valor del atributo nivel\_actual. La forma para reparar este problema es creando una tabla ''nivel'', sin embargo, se desea que los niveles sean un infinito simbólico.
    
    \noindent \textbf{No se cumple} con esta forma normal en la tabla \textbf{gmdl\_condicion} debido a que  los valores del atributo ''atributo'' dependen del valor del atributo ''tabla''.
    
    \noindent Se considera que el incumplimiento anterior no será atendido, esto por que para resolverlo se necesitaría crear 2 tablas extras que contengan todas las tablas relacionadas con Moodle y todos los atributos asociados con esas tablas. Estas tablas actuarían como un catálogo para la tabla \textbf{gmdl\_condicion} y se tendrían que estar actualizando si se quiere que los plugins a desarrollar tengan compatibilidad con versiones posteriores de Moodle.
    
    \noindent Otro motivo por el cual no se atiende es que los datos de la tabla \textbf{gmdl\_condicion} nunca cambiarán una vez que se haya creado la base de datos y llenado los valores por defecto.
\end{comment}



% ==================================================================
%   La lista de configuraciones guardadas no pertenece a analisis
%   sino a diseño.
%                   To Do: Mover a diseño en el módulo de experiencia
% ===================================================================
    
    %\noindent Por el momento se están guardando las siguientes configuraciones:
    %\begin{center}
    %\begin{tabular}{| m{0.25\textwidth} | m{0.30\textwidth} | m{0.15\textwidth} | m{0.18\textwidth}|}\hline
        %{\bf Nombre} & {\bf Descripción} & {\bf Módulo} & {\bf Sub-módulo}  \\\hline
        
        %Modulo de exp. activo &
        %    Bandera que nos indica si el módulo está activado o no &
        %    Experiencia & Esquema de \newline experiencia\\\hline 
     
        %Tipo de incremento &
        %    Si el incremento de experiencia necesaria por nivel es lineal o porcentual &
        %    Experiencia & Esquema de \newline experiencia \\\hline 
         
        %Cantidad de incremento &
        %    Número decimal que indica cuánto se incrementa la experiencia necesaria por cada nivel &
        %    Experiencia & Esquema de \newline experiencia \\\hline
        
        %Experiencia por actividad &
        %    Cantidad de experiencia que da cualquier actividad al ser completada &
        %    Experiencia & Esquema de \newline experiencia \\\hline 
        
        %Experiencia del nivel 1 &
        %    Cantidad de experiencia asociada al nivel 1 &
        %    Experiencia & Esquema de \newline experiencia \\\hline 
        
        %Nombre del nivel &
        %    Nombre que reciben los niveles por defecto. &
        %    Experiencia & Niveles \\\hline 
        
        %Mensaje de felicitaciones &
        %    Mensaje que aparece cuando un usuario sube de nivel. &
        %    Experiencia & Niveles \\\hline 
        
        %Descripción del rango &
        %    Mensaje que aparece por defecto, si el nivel no pertenece a ningún rango. &
        %    Experiencia & Niveles \\\hline 
            
        %Imagen del nivel &
        %    Imagen del nivel por defecto (Se almacena la ruta de la imagen, después de haber sido copiada). &
        %    Experiencia & Niveles \\\hline 
        
        %Color del número del nivel &
        %    Color que tendrá el número del nivel por defecto. &
        %    Experiencia & Niveles \\\hline 
        
        %Color de la barra de progreso &
        %    Color que tendrá la barra de progreso de los niveles por defecto. &
        %    Experiencia & Niveles \\\hline 
    %\end{tabular}%
    %\end{center}%


 % PASAR EL ANALISIS COMO ANEXO LAS FORMAS NORMALES

 % y finalmente un análisis de las formas normales sobre la base de datos propuesta.\\
     
 % Con lo anteriormente especificado y de acuerdo a las 2 primeras formas normales
 % especificadas por Edgar Frank Codd \cite[ paǵ. 161-182]{libroBaseDeDatosEspaniol}
 % se diseñó el esquema de la base de datos. A continuación se presentan de manera
 % muy resumida dichas formas normales.


 % CONCLUSIÓN: PASAR AL CAPITULO DE CURVA DE APRENDIZAJE, CÓMO CONCLUSION.

 % \noindent Debido a la capa de abstracción que moodle tiene con respecto al acceso a datos
 % y a que las nuevas funcionalidades se desarrollaran mediante desarrollo de plugins. se
 % decidió utilizar las herramientas que proporciona moodle para la creación de las tablas
 % requeridas para implementar Gamificación.

    %[V] Libro que me pasó CAT (David)   
