
\newcounter{theActivity}\stepcounter{theActivity}
\newcounter{theRF}\stepcounter{theRF}
\newcounter{theRNF}\stepcounter{theRNF}


\newenvironment{Actividad}[2]{\begin{itemize}\item[\bf A\arabic{theActivity}] {\bf #1:} {#2}}{\end{itemize}\stepcounter{theActivity}}

\newenvironment{RF}[2]{
    \begin{itemize}
        \item[\bf \hypertarget{RF\arabic{theRF}}{RF\arabic{theRF}}] {\bf #1:} {#2}
}{
    \end{itemize}
    \stepcounter{theRF}
}

\newenvironment{RNF}[2]{\begin{itemize}\item[\bf RNF\arabic{theRNF}] {\bf #1:} {#2}}{\end{itemize}\stepcounter{theRNF}}
\newcommand{\Sprint}[1]{{\color{primary}\fbox{Sprint #1}}}
\newcommand{\PBitem}{\item[] \quad}

%\noindent\bb{Sprint 1: Marco Teorico, Metodología, Alcance TT-I}
\begin{Actividad}{Investigar Scrum}{% 
    Redactar el capítulo de I del documento de metodología el cual describe el marco de trabajo Scrum, basándose en la guía oficial.} 
    \PBitem \Sprint{1}%Estimación 2 dias. \Sprint{1} 
\end{Actividad}

\begin{Actividad}{Adaptación de Scrum}{%
    Especificar como es configurado Scrum para el proyecto, definir roles, eventos, y artefactos.}
    \PBitem \Sprint{1}%Estimación 2 dias. \Sprint{1} 
\end{Actividad}

\begin{Actividad}{Adquirir Actionable Gamificación}{%
    Adquisición del libro de Yu-kai Chou, {\it Actionable Gamification: Beyond Points, Badges, and Leaderboards}}
    \PBitem \Sprint{1}%Estimación 2 dias. \Sprint{1} 
\end{Actividad}

\vfill\null\columnbreak

\begin{Actividad}{Investigar Gamificación}{%
    Ampliar la investigación de gamificación, definiciones, sus inicios, uso en la educación.}
    \PBitem \Sprint{1}%Estimación 2 dias. \Sprint{1} 
\end{Actividad}

%\begin{Actividad}{Principios de Gamificación}{%
    %Investigar cada uno de los principios de gamificación de acuerdo con el marco de referencia Octalysis,  buscar técnicas para poder soportar los principios.}
    %\PBitem Estimación 2 dias. \Sprint{1} 
%\end{Actividad}

\begin{Actividad}{Estado del arte}{%
    Investigar el estado del arte en relación a desarrollo de funcinalidades a una plataforma de aprendizaje.}
    \PBitem \Sprint{1}%Estimación 2 dias. \Sprint{1} 
\end{Actividad}

%\begin{Actividad}{Redactar Marco Teórico}{%
    %Invetigar distintos papers que definan la gamificación, describan sus inicios}
    %\PBitem Estimación 2 dias. \Sprint{1} 
%\end{Actividad}

\begin{Actividad}{Establecer los módulos}{%
    Plantear una propuesta integral la cual divida en módulos principales las funcionalidades que tendrá el producto final.}
    \PBitem \Sprint{1}%Estimación 2 dias. \Sprint{1} 
\end{Actividad}

\pagebreak

\begin{Actividad}{Alcance TT-I}{%
    Definir el alcance que tendrá el proyecto para la presentación del trabajo terminal I.}
    \PBitem \Sprint{2}%Estimación 2 dias. \Sprint{1} 
\end{Actividad}

%\hfill\bigskip\\\noindent\bb{Sprint 2: Investigación de Implementación}

\begin{Actividad}{Módulo I y II}{%
    Especificar el funcionamiento y el análisis inicial que se realiza en el módulo de Recompensa.}
    \PBitem \Sprint{2}%Estimación 2 dias. \Sprint{2} 
\end{Actividad}

%\begin{Actividad}{Módulo II}{%
    %Especificar el funcionamiento y el análisis inicial que se realiza en el módulo Financiero.}
    %\PBitem Estimación 2 dias. \Sprint{2} 
%\end{Actividad}

\begin{Actividad}{Módulo III y IV}{%
    Especificar el funcionamiento y el análisis inicial que se realiza en el módulo de Seguimiento.}
    \PBitem \Sprint{2}%Estimación 2 dias. \Sprint{2} 
\end{Actividad}

%\begin{Actividad}{Módulo IV}{%
    %Especificar el funcionamiento y el análisis inicial que se realiza en el módulo de Competencia.}
    %\PBitem Estimación 2 dias. \Sprint{2} 
%\end{Actividad}

\begin{Actividad}{Módulo V y VI}{%
    Especificar el funcionamiento y el análisis inicial que se realiza en el módulo de Personalización.}
    \PBitem \Sprint{2}%Estimación 2 dias. \Sprint{2} 
\end{Actividad}

%\begin{Actividad}{Modulo VI}{%
    %De la propuesta de solución, describir cada uno de los módulos y herramientas (submódulos) que contienen.}
    %\PBitem Estimación 2 dias. \Sprint{2} 
%\end{Actividad}

\begin{Actividad}{Alternativas a Moodle}{%
    Investigar otros sistemas gestores de aprendizaje en los que se puedan desarrollar nuevas funcionalidades.}
    \PBitem \Sprint{2}%Estimación 2 días \Sprint{2}
\end{Actividad}


\begin{Actividad}{Implementación Gamificación}{%
    Investigar distintas publicaciónes (papers) en donde se describa la forma en que se implementó gamificación y los resultados obtenidos}
    \PBitem \Sprint{2}%Estimación 2 días 
\end{Actividad}
%trabajo los desarrollos, investigaciones y trabajos ya existentes acerca de la gamificación en una plataforma de aprendizaje.


%\hfill\bigskip\\\noindent\bb{Sprint 3: Reporte Técnico del trabajo Terminal}

\begin{Actividad}{Problema}{%
    Redactar el problema que se pretende atacar con este trabajo terminal.}
    \PBitem \Sprint{3}%Estimación 2 días \Sprint{3}
\end{Actividad}

\begin{Actividad}{Propuesta de Solución}{%
    redactar la propuesta de solución, que se pretendar dar ante el problema}
    \PBitem \Sprint{3}%Estimación 2 días \Sprint{3}
\end{Actividad}

\begin{Actividad}{Justificación}{%
    Redactar por que la justificación de porque surge el proyecto y porque se optó por esa propuesta de solución.}
    \PBitem \Sprint{3}%Estimación 2 días \Sprint{3}
\end{Actividad}

\vfill\null\columnbreak

\begin{Actividad}{Alcances y Limitaciones}{%
    Establecer los alcances y limitaciones que tiene el trabajo terminal}
    \PBitem \Sprint{3}%Estimación 2 días \Sprint{3}
\end{Actividad}

\begin{Actividad}{Instalar Moodle}{%
    Realizar la instalación de Moodle de forma local, en las computadoras de los miembros del equipo.}
    \PBitem \Sprint{3}%Estimación 2 días \Sprint{3}
\end{Actividad}

\begin{Actividad}{Usar Moodle}{%
    Familiarizarse con el uso de Moodle en especifico con las funcioalidades de un administrador, gestionar cursos, gestionar grupos, crear usuarios, etc}
    \PBitem \Sprint{3}%Estimación 2 días \Sprint{3}
\end{Actividad}

%\hfill\bigskip\\\noindent\bb{Sprint 4: Pruebas de Concepto}

\begin{Actividad}{Entorno de desarrollo}{%
    Establecer el entorno de desarrollo sobre el cual se trabajará, incluyendo características de instalación}
    \PBitem \Sprint{4}%Estimación 3 hrs \Sprint{4}
\end{Actividad}

\begin{Actividad}{Filtrar plugins}{%
    Escoger los plugins de los cuales se realizarán las pruebas de concepto y documentar los criterior de discrimnación ocupados.}
    \PBitem \Sprint{4}%Estimación 3 hrs \Sprint{4}
\end{Actividad}

\begin{Actividad}{P1: Database Fields}{%
    Realizar la prueba de database fields}
    \PBitem \Sprint{4}%Estimación 3 hrs \Sprint{4}
\end{Actividad}

\begin{Actividad}{P2: Database Presets}{%
    Realizar la prueba de database presets}
    \PBitem \Sprint{4}%Estimación 3 hrs \Sprint{4}
\end{Actividad}

\begin{Actividad}{P3: User Profile Fields}{%
    Realizar la prueba de user profile fields}
    \PBitem \Sprint{4}%Estimación 3 hrs \Sprint{4}
\end{Actividad}

\begin{Actividad}{P4: Block}{%
    Realizar la prueba de block}
    \PBitem \Sprint{4}%Estimación 3 hrs \Sprint{4}
\end{Actividad}

\begin{Actividad}{Reporte de Pruebas}{%
    Realizar el reporte de pruebas de concepto para entregar al profesor de seguimiento. }
    \PBitem \Sprint{4}%Estimación 3 hrs \Sprint{4}
\end{Actividad}
