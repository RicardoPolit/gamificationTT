
\section{Proceso completo de implementación}
\label{analisis:visionyuso}

 Esta sección presenta la secuencia lógica en la que, partiendo de una instalación
 de Moodle con la versión 3.5, se pueden agregar y utilizar los distintos módulos
 que se desarrollarán a lo largo de este trabajo terminal. A continuación se
 describen los pasos:

% Definición de Macro ---------------------------------
\newcommand{\Recomendacion}[2][Recomendación:]{%
    %\begin{description}\item[#1] #2\end{description}
}
% ------------------------------------------------------

 \begin{enumerate}
  % \item El \refElem{aAdministrador} realiza las configuraciones extras como
        % método de autenticación, correos de entrada y salida, restricciones de
        % seguridad y demás configuraciones que requiera del sitio que administra.
        % Este paso puede ser omitido si es que la institución ya posee una
        % plataforma Moodle con la versión 3.5.

        % \Recomendacion{%
        %     Si se tiene acceso a un servidor de correo SMTP entonces se
        %     recomienda habilitar el método automático de autenticación y
        %     configurar el correo de salida, para permitir automatizar el
        %     registro de los alumnos. % (ver anexo \ref{anexo:install:smtp}).
        % }

  % \item Se lleva a cabo el registro de los \refElem[profesores]{aProfesor} y
        % \refElem[alumnos]{aEstudiante} que utilizarán la plataforma. Posteriormente
        % el administrador otorga los permisos de {\it creador de curso} a los
        % profesores dados de alta en la plataforma. Este paso puede ser omitido
        % si es que la institución ya tiene definido un mecanismo para la creación
        % y administración de cursos.
        % (ver avexo \ref{anexo:install:permissions})

  \item El \refElem{aAdministrador} procede a instalar los distintos plugins
        correspondientes a los módulos identificados (experiencia, recompensa,
        financiero, seguimiento, personalización y competencias).

  \item El \refElem{aAdministrador} realiza las configuraciones de los plugins
        de gamificación que haya instalado con base en las peticiones que requiera
        la institución o que demanden los profesores. Dependiendo de los plugins
        que se hayan instalado las configuraciones serán distintas.

  \item Los \refElem[profesores]{aProfesor} crean los cursos correspondientes a
        los grupos y materias que imparten, y de cada curso el \refElem[administrador]%
        {aAdministrador} asigna a un profesor titular. Si la institución tiene
        otro mecanismo para la creación y administración del curso este paso se
        puede omitir.

  \item El \refElem{aProfesor} crea el contenido del curso y establecer las
        actividades pertenecientes a cada sección del curso, también procede
        a realizar las configuraciones de aquellos elementos de gamificación
        que haya decidido agrega a su curso.

  \item Los \refElem[estudiantes]{aEstudiante} se inscriben a los cursos y
        comienzan a realizar las actividades especificadas por los profesores,
        si los módulos de experiencia y módulo financiero se encuentran instalados
        y habilitados entonces el \refElem[estudiante]{aEstudiante} conforme vaya
        completando actividades y secciones de los cursos comenzará a acumular puntos
        de experiencia y monedas las cuales puede canjear por otros objetos.

 \end{enumerate}

 \noindent
 Adicionalmente los \refElem[estudiantes]{aEstudiante} pueden configurar ciertos
 tipos de plugins para que estos se muestren u oculten. Así mismo también se otorga
 la flexibilidad para que a los cursos que hayan sido creados anteriormente puedan
 ser editados para que se les agreguen elementos de gamificación.
