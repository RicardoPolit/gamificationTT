\chapter{Glosario} \label{ch:glosario}

\begin{bGlosario}

  \bTerm{Clavecandidata}{Clave candidata}
  Una clave candidata es una súper clave, pero cuyos subconjuntos propios no son súper clave \cite[p. 24]{libroBaseDeDatosEspaniol}.


  \bTerm{Claveprimaria}{Clave primaria}
  La clave primaria es una clave candidata, pero que fue elegida por el diseñador de la base de datos para ser la clave que diferenciará a los registros de una tabla \cite[p. 24]{libroBaseDeDatosEspaniol}.

  \bTerm{Complemento}{Complemento}
  Permiten extender la funcionalidad de moodle (plugins).

  \bTerm{Dependenciafuncional}{Dependencia funcional}
  Se dice que un conjunto de atributos $B$ depende funcionalmente de un conjunto de atributos $A$ si y solo si, para todos los pares de tuplas $t_1$ y $t_2$ de una entidad tales que $t_1[A] = t_2[A]$, también ocurre que  $t_1[B] = t_2[B]$ \cite[p. 163]{libroBaseDeDatosEspaniol}.

  \bTerm{Experiencia}{Experiencia}
  La experiencia o puntos de experiencia son un valor que permite cuantificar la expertiz que un usuario tiene en un juego. Comunmente son denotados por las abreviaciones 'xp' o 'exp'.

  \bTerm{gamificacion}{gamificación}
  El uso de mecánicas de juegos en un entorno no lúdico para involucrar y motivar digitalmente a las personas para que logren sus objetivos.

  \bTerm{Niveles}{Niveles}
  A lo largo de este trabajo terminal al usar \textbf{nivel de experiencia} nos referiremos a la posición conseguida por una cantidad de puntos de experiencia representada por un número entero.

  \bTerm{Plugin}{Plugin}
  Componentes que permiten añadir características y funcionalidades adicionales a las que proporciona Moodle de forma nativa.

  \bTerm{Puntos}{Puntos}
  Los videojuegos al representar la experiencia utilizan puntos (números enteros positivos), los cuales son otorgados por realizar acciones determinadas. Un tipo de acción siempre otorga la misma cantidad de puntos y diferentes tipos de acciones pueden o no dar la misma cantidad de puntos.

  \bTerm{Subconjunto}{Subconjunto propio}
  Un subconjunto propio $B$ de $A$ ($B \subset A$) es aquel conjunto que contiene elementos del conjunto $A$, pero no todos ellos ($B \neq A$) \cite[p. 79]{libroMatematicasDiscretas}.

  \bTerm{Superclave}{Súper clave}
  Una súper clave es aquel conjunto de uno o más atributos que nos permiten diferenciar entre un registro y otro en una misma tabla \cite[p. 24]{libroBaseDeDatosEspaniol}.


\end{bGlosario}
