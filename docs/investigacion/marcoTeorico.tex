\chapter{Marco Teórico}
\label{ch:marcoTeorico}

 Este capítulo tiene como propósito establecer el soporte conceptual y documental del proyecto,
 especifica la metodología incluyendo roles, eventos y artefactos, además se especifican los
 marcos de trabajo de gamificación, los sistemas de aprendizaje contemplados y finalmente
 la elección de la plataforma sobre la cual se desarrollaran las distintas funcionalidades.

    \section{Metología}

 El desarrollo de este proyecto se realizará mediante un desarrollo iterativo
 utilizando el marco de referencia Scrum. Este aparatado está destinado a
 presentar los roles, eventos y artefactos de scrum, así como a describir
 la forma en que han sido configurado para este proyecto.\\

    \noindent Los creadores Schwaber y Sutherland definen scrum de la siguiente manera:
        \begin{quote}
        Scrum es un marco de trabajo en el cual las personas pueden abordar
        problemas complejos de adaptación mientras que, productiva y creativamente
        desarrollan productos con el mayor valor posible \cite{Sutherland17}.
        \end{quote}

    \noindent De acuerdo con Michele Sliger PMP (Project Management Professional) y CST (Certified Scrum Trainer):
        \begin{quote}
        Scrum es un método ágil de entrega iterativa e incremental de productos que
        utiliza comentarios frecuentes y toma de decisiones en colaboración \cite{Sliger1}.
        \end{quote}

    \noindent Pete Deemer, CEO de GoodAgile (Certificadora para Scrum Masters y Product Owners) lo define cómo:% y lider de ScrumPrimer.
        \begin{quote}
        Scrum es un marco de trabajo en el que equipos multifuncionales pueden crear productos
        o desarrollar proyectos de una forma iterativa e incremental \cite{ScrumPrimer}.
        \end{quote}


\subsection{Equipo de Scrum}

 \noindent El equipo de Scrum está conformado por el dueño del producto {\em(Product Owner)},
 el equipo de desarrollo {\em(Development team)} y el maestro scrum {\em(Scrum Master)}.

\subsubsection{Product Owner} 

 El dueño del producto o {\em Product Owner} es responsable de maximizar el valor del producto
 resultante del trabajo del equipo de desarrollo, su principal responsabilidad es la gestión
 del artefacto {\em``Product Backlog''}. Para que las funciones del {\it Product Owner} sean exitosas,
 todos los involucrados en el proyecto deben respetar sus decisiones.\\
 % Nadie puede forzar al equipo de desarrollo a trabajar en requerimientos distintos a los del product owner.\\

 \noindent En este proyecto el rol del Product Owner lo llevarán a cabo los directores del trabajo terminal:
     
    \begin{quote}
    \begin{itemize}
        \item M. en C. Sandra Ivette Bautista, y el
        \item M. en C. Edgar Armando Catalán
    \end{itemize}
    \end{quote}
                                         
 \noindent A pesar de que en la guía oficial de Scrum \cite{Sutherland17} se especifica que el {\it Product
 Owner} debe ser una persona, se decidió que este rol fuera llevado a cabo mediante los directores del trabajo
 terminal con la premisa de que para la toma de decisiones ambos directores deben de estar de acuerdo.

\subsubsection{Equipo de desarrollo}

 El equipo de desarrollo consiste en un grupo de profesionales que realizan el trabajo para entregar
 los incrementos del producto al final de cada {\it Sprint}. El equipo de desarrollo es un equipo
 auto-organizado y multifuncional, durante el desarrollo de este proyecto el equipo estará conformado por:

 % \noindent El tamaño del equipo debe ser lo suficientemente pequeño para permanecer ágil y lo
 % suficientemente grande para realizar entregas significativas al final de cada {\em Sprint}.\\
 % Un equipo con menos de tres miembros disminuiría la interacción y por lo tanto la productividad,
 % por otro lado, con más de nueve miembros se requiere mayor coordinación con más complejidad.

 % \noindent El equipo de desarrollo estará conformado por los estudiantes que cursan el trabajo terminal:
        
    \begin{quote}
    \begin{itemize}
        \item David Flores Casanova
        \item Ricardo Naranjo Polit
        \item Daniel Isaí Ortega Zúñiga
    \end{itemize}
    \end{quote}

\subsubsection{Maestro Scrum}

 El maestro Scrum o {\it Scrum Master} es el líder que está al servicio del equipo,
 se encarga de ayudar al equipo scrum a maximizar el valor y mejorar continuamente la
 forma de trabajo, además de guiar a todos los involucrados hacia una mejor implementación
 del marco de trabajo.\\

 \noindent El rol del {\it Scrum Master} durante este proyecto se llevará a cabo mediante
 dos personas con la finalidad de dividir las responsabilidades y no sobrecargar de trabajo
 a una persona. El {\it Scrum Master} estará conformado por:

    \begin{quote}
    \begin{itemize}
        \item M. en C. Edgar Armando Catalán {\it (Responsabilidades hacia el Product Owner)}
        \item Daniel Isaí Ortega {\it(Responsabilidades hacia el equipo de desarrollo)}
    \end{itemize}                        
    \end{quote}

\subsubsection{Stakeholders}

 Los Stakeholders son personas externas al equipo Scrum con un interés y/o conocimientos
 específicos del producto \cite{ScrumGlosary}. Durante el desarrollo del trabajo terminal
 se consideran a los sinodales cómo los Stakeholders oficiales, los cuales son listados a
 continuación:

    \begin{quote}
    \begin{itemize}
        \item Dra. Fabiola Ocampo Botello
        \item M. en C. María del Socorro Téllez Reyes
        \item M. en C. José David Ortega Pacheco
    \end{itemize}
    \end{quote}

\subsection{Eventos}

 Los eventos prescritos en Scrum son usados para crear regularidad en el proceso, además
 minimizan la necesidad de juntas no definidas. Todos los eventos tienen un tiempo establecido, 
 para los {\it sprints} la duración es fija y no puede ser acotada o alargada, los demás eventos
 pueden concluir una vez que su propósito es cumplido.

\subsubsection{Sprint}

 Un {\it Sprint} es el lapso de tiempo en el cual un incremento del producto es creado, los sprints
 son secuenciales, es decir, inician inmediatamente después del término de otro. Internamente 
 cada Sprint consiste en las etapas de planeación, reuniones diarias ({\it Daily Scrum}),
 desarrollo, revisión y retroalimentación.
    
    % \begin{itemize}
    %    \item No se pueden hacer cambios a la definición del Sprint Goal. (objetivo del Sprint)
    %    \item Los objetivos de calidad no disminuyen.
    %    \item El alcance debe ser clarificado por el Product Owner y negociado entre él y el Team.
    % \end{itemize}
    
 % \noindent Solo el Product Owner tiene la autoridad de cancelar un Sprint. Otra situación que puede
 % cancelar el Sprint es que el Sprint Goal se vuelva obsoleto.

 \begin{quote}
    \noindent Para este proyecto los Sprints están configurados a una duración de 14 días con una
    estimación de 18 iteraciones, la duración de dos semanas se estableció con el propósito de:
    
    \begin{itemize}
        \item Incrementar la retroalimentación y detectar los impedimentos
              en la forma de trabajo lo más pronto posible, y

        \item Realizar incrementos más cortos y continuos considerando que el
              equipo de desarrollo está conformado por tres integrantes.
    \end{itemize} 
 \end{quote}
    
\subsubsection{Planeación}

 La planeación del {\it Sprint} se realiza con todos los miembros que forman parte del equipo scrum,
 en dicha reunión se establece cuál será el incremento entregado al final del sprint, y la forma
 en que se logrará el objetivo del mismo.\\

 \noindent El día acordado para llevar a cabo esta reunión son los {\bf martes} cada dos semanas
 {\bf a la 1:30pm} en las instalaciones de la ESCOM. El horario fue acordado tomando en cuenta
 la disponibilidad de todos los miembros del equipo Scrum.
    
    \begin{quote}
    {\bf Nota:} En caso de que, por algun evento extraordinario, no se pueda
                llevar a cabo el Sprint Planning este reunión se reagendará para
                que ocurra lo más pronto posible.
    \end{quote}
  
\subsubsection{Daily Scrum}

 Las reuniones diarias o Daily Scrum se realiza día a día durante la ejecución de los {\it Sprint}, deben
 durar como máximo 15 minutos, en esta reunión el equipo de desarrollo planea como trabajá durante el día.
 En la tabla \ref{tbl:daily} muestra los días acordados, lugar y hora pre-establecidos para la reunión.

 % En la reunión cada miembro debe responder las siguientes preguntas:
     
    % \begin{itemize}
    %    \item ¿Qué hice ayer para ayudar al Team a alcanzar el Sprint Goal?
    %    \item ¿Qué hare hoy para ayudar al Team a lograr el Sprint Goal?
    %    \item ¿Veo algun obstaculo que me impida o impida al Team lograr el Sprint Goal?
    % \end{itemize} 
   
    \addtable{|c|c|c|}{tbl:daily}{
        {\bf Día de Trabajo} & {\bf Lugar} & {\bf Hora Inicio} \\\hline
        Lunes     & ESCOM Sala 21 N & 10:00am \\\hline
        Martes    & ESCOM Sala 21 N & 10:00am \\\hline
        Miércoles & ESCOM Sala 21 N & 10:00am \\\hline
        Jueves    & ESCOM Sala 21 N & 10:00am \\\hline
        Viernes   & ESCOM Sala 21 N & 10:00am \\\hline
        Domingo   & -               & 12:00pm \\\hline
    }{Horario de Daily Scrum}
    
 \noindent Debido a la dificultad de hacer coincidir los horarios del equipo de desarrollo  con los
 del {\it Product Owner}, cuando se requiera de sus decisiones, opinión o retroalimentación se le contactará
 a través de mensajería instantánea.
    
\subsubsection{Revisión}

 Esta reunión se realiza al final de cada {\it Sprint} para de revisar el incremento, discutir los inconvenientes
 encontrados y en dado caso adaptar la planeación del proyecto. El equipo scrum y los stakeholder deben
 revisar el incremento y establecer cuáles serán los cambios a ejecutar.\\

 \noindent Debido a que en este proyecto, los stakeholders y el equipo de scrum tienen distintos horarios
 de disponibilidad, la revisión del {\it sprint} se divide en cuatro fases, aplicando la primer fase a los sprints
 impares y las cuatro fases para sprints pares. Las fases de describen a continuación:
 
    \begin{quote}
    \begin{itemize}
    \item[\it Fase 1]
        Consiste en realizar una primer reunión con el equipo scrum para obtener una
        retroalimentación y revisar el incremento entregado.

    \item[\it Fase 2]
        En esta fase el equipo de desarrollo tiene reuniones con los Stakeholders con la
        finalidad de obtener retroalimentación y observaciones acerca de la forma de
        trabajo y del incremento.

    \item[\it Fase 3] 
        En esta fase los miembros del equipo scrum revisarán las observaciones y
        comentarios de los {\it Stakeholders} para saber cuales proceden.

    \item[\it Fase 4]
        Se avisa a los Stakeholders acerca de cuales observaciones procedieron y cuales no.\\
    \end{itemize}    
    
    {\bf Nota:} Las reuniones de la fase 2, dependen de la disponibilidad que cada {\it stakeholder} tenga,
                en caso de que ningún stakeholder tenga disponibilidad para llevar a cabo la fase 2,
                el proceso de la revisión del {\it Sprint} terminará.
    \end{quote}

\subsubsection{Retroalimentación}

 La etapa de retroalimentación consiste en una reunión del equipo scrum con el objetivo de crear un plan para
 las mejoras en la forma de trabajo, esta reunión ocurre después de la revisión y antes de la planeación del
 siguiente {\it Sprint}. En esta reunión los miembros del equipo scrum ven cómo atender las debilidades y áreas
 de oportunidad.

\subsection{Artefactos}

 Los artefactos en Scrum proporcionan transparencia en toda la aplicación de Scrum
 y además funcionan como herramientas para la inspección de la implementación del
 marco de trabajo y adaptación del mismo. Tener los artefactos organizados brinda
 una mayor visibilidad acerca del avance del proyecto y del producto final.
 
\clearpage

\subsubsection{Product Backlog}

 La cartera de producto o {\it product Backlog} lista todas las características, funcionalidades, requerimientos,
 y mejoras necesarias para la creación del producto. El responsable del contenido, disponibilidad y organización
 del {\it Product Backlog} es el {\it Product Owner}.\\
       
 \noindent Debido a que el proyecto requería una etapa de investigación, se optó por tener dos tipos
 de {\it items} en el product backlog, los items de documentación/preparación del proyecto  y los {\it items}
 para desarrollo del mismo.\\
    
    \noindent{\bf Items de Documentación}\\
    Los items de preparación del proyecto y documentación deben ser especificados
    mediante los atributos presentes en la tabla \ref{attrPBpre}:
    
    \addtable{|l|l|}{attrPBpre}{
        {\bf Atributo} & {\bf Descripción}                                                             \\\hline
        id           &  Es una identificador de la forma ``Ax'' donde {\it x} es un número consecutivo \\\hline
        nombre       &  Nombre representativo de la actividad                                          \\\hline
        descripción  &  Detalle de lo que hay que hacer para llevar a cabo esta actividad.             \\\hline
        sprint       &  Indica el número de Sprint al cual ha sido asignada esta tarea.                \\\hline
        %estado      & Indica el estado ({\it por hacer, en proceso o concluida} de una actividad. \\\hline
        %estimación  & Especifica el periodo de tiempo estimado para la liberación de dicha actividad. \\\hline
    }{Atributos de los Items del P.B de Documentación}


    \noindent{\bf Items de Desarrollo del Proyecto}\\
    Describen las características del software que se desarrollará, estos items deben
    ser redactados de manera objetiva y como requerimientos del sistema, y deben contener
    los atributos presentes en la tabla \ref{attrPB}:
    
    \addtable{|l|p{0.62\textwidth}|}{attrPB}{
        {\bf Atributo} & {\bf Descripción}\\\hline
        id           &  Es una identificador de la forma '{\bf RFx}' o '{\bf RNFx}' para requerimientos    \par
                        funcionales y no funcionales respectivamente. {\it x} es un número consecutivo     \\\hline

        nombre       &  Nombre representativo del requerimiento del sistema.                               \\\hline
        descripción  &  Descripción concisa y objetiva acerca del requerimiento.                           \\\hline
        prioridad    &  Indica la prioridad de un requerimiento, los valores posibles son:                 \par
                        \qquad MA (muy alta), A (alta), M (Media), B (baja) y MB (muy baja)                \\\hline

        sprint       &  Indica el número de Sprint al cual ha sido asignado este requerimiento.            \\\hline
        tipo         &  Tipo de requerimiento no funcional según la clasificación propuesta por Frank Tsui \\\hline
        %estimación  & Especifica el periodo de tiempo estimado para la liberación de dicho requerimiento. \\\hline
    }{Atributos de los Items del P.B de Desarrollo del Proyecto}
    
    \begin{quote}
    {\bf Nota:} El atributo {\it Sprint} debe estar presente en todos los items correspondientes
                al sprint corriente y a los sprints anteriores a este. El atributo {\it Sprint}
                puede no estar presente en los items que no han sido vinculados a un Sprint. 
    \end{quote}
    
\subsubsection{Sprint Backlog}
    
 El {\it Sprint Backlog} está formado por los {\it items} del {\it Product Backlog} seleccionados
 para el {\it Sprint} más el plan para entregarlos, y así cumplir con el Sprint Goal. El {\it Sprint
 Backlog} es una estimación de las funcionalidades que serán entregadas en el siguiente incremento
 del producto.

 % \noindent Conforme los {\it items} del {\it product backlog} vayan siendo seleccionados para
 % tratarse en un sprint, se les añadirá una etiqueta que indique a qué sprint pertenecen.
    
    % \addtable{|l|l|}{SBItems}{
    %    {\bf Atributo} & {\bf Descripción}\\\hline
    %    sprint   &  Indica el número de Sprint al cual ha sido asignado el item.             \\\hline
    %    pruebas  &  (Opcional) Sentencia de cómo se evaluara que dicho ítem esté completado. \\\hline
    %
    % }{Atributos del Sprint Backlog }


\section{Marcos de trabajo para la Gamificación}

 Como se comentó en el capítulo \hyperrefx{ch:introduccion}, el crear una experiencia
 gamificada exitosa no solo consiste en aplicar mecánicas de juegos a una actividad
 específica, tambien requiere del seguimiento de un marco de instrucción apropiado \cite[p. 1110]{GamInE-Learning}.
 Se ha decidido utilizar dos marcos de trabajo como guía en el diseño e implementación
 de los componentes que se desarrollarán. Los marcos de trabajo elegidos son {\it Octalysis}
 \cite{Octalysis} y {\it For The Win} \cite{ForTheWin}.

    \input{investigacion/marcoTeorico/Octalysis}
    \subsection{For The Win} \label{sec:ForTheWin}

 Dan Hunter y Kevin Werbach crearon un marco de trabajo que se centra en aplicar la gamificación en
 los negocios y empresas, siguiendo una sería de pasos y conociendo a detalle los elementos de juego.
 El nombre ``{\it For The Win}'', es por que dicho marco de trabajo es presentado en el libro
 ``{\it For The Win:  How game thinking can revolutionize your business}''.
    
\subsubsection{Elementos de juego}
    
 \noindent De acuerdo con For The Win, para implementar gamificación se necesitan contemplar
 los tres tipos de elementos de juego, Dinámicas, Mecánicas y Componentes. Los tipos de
 elementos son organizados en una pirámide (figura \ref{fig:FTW_Piramide}) de acuerdo con
 su nivel de abstracción y el objetivo que tienen \cite[pp. 55-57]{FrameWorkForTheWin}.
    
    \addfigure[(adaptado de {\it For The Win} \cite{FrameWorkForTheWin})]%
        {.35}{investigacion/images/ForTheWin_Piramide}{fig:FTW_Piramide}%
        {Niveles de clasificación de elementos de juego según For The Win}
    
    \begin{multicols}{2}
    \noindent
    {\bf Nivel: Dinámicas}.
        Las dinámicas son lo más abstracto, es la temática que envuelve a todo el sistema.
        Existen 5 dinámicas, las cuales son:
        
        \begin{enumerate}
            \item Restricciones
            \item Emociones
            \item Historia
            \item Progresión
            \item Relaciones sociales
        \end{enumerate}

    \vfill\null
    \columnbreak

    \noindent
    {\bf Nivel: Mecánicas}.
        Las mecánicas son el motivo para que se haga alguna acción, son las que mantienen
        enganchado al jugador. Existen 10 mecánicas, las cuales son:
        
        \begin{enumerate}
            \item Desafíos
            \item Suerte 
            \item Competencia
            \item Cooperación
            \item Retroalimentación 
            \item Obtención de elementos
            \item Recompensas
            \item Transacciones
            \item Turnos
            \item Ganadores y perdedores\\
        \end{enumerate}

    \end{multicols}
\clearpage

        \noindent\textbf{Nivel: Componentes} Los componentes son la forma de implementar las mecánicas y las dinámicas. Existen 15 componentes, los cuales son:
        
    \begin{multicols}{2}
        \begin{enumerate}
            \item Logros
            \item Avatares
            \item Insignias
            \item Peleas de jefes finales
            \item Colecciones
            \item Combates
            \item Desbloqueo de contenido
            \item Regalos e intercambios
            \item Tablas de líderes
            \item Niveles de personaje (Experiencia)
            \item Puntos
            \item Misiones
            \item Esquemas sociales
            \item Equipos
            \item Moneda virtual
        \end{enumerate}
    \end{multicols}
    
    
    \noindent  For The Win establece que para cumplir con gamificación no es necesario tener cada uno de los elementos anteriores, ya que establece que antes de cantidad se necesita calidad, refiriéndose a que los elementos tengan coherencia entre sí.
    
    \subsubsection{ Proceso de implementación}
    
    For the Win indica que el proceso consta de 6 pasos que especifican cómo introducir la gamififación, cada uno de los pasos se describen a continuación \cite[pp. 60-70]{FrameWorkForTheWin}.\\
    
    \noindent \textbf{1.- Definir los objetivos del negocio}\\
    
    \noindent Los objetivos no se refieren a los planteados en la visión y misión de la empresa, sino al ''¿Por qué?'' se está haciendo este sistema que tiene implementada la gamificación.\\
    
    \noindent \textbf{2.- Delimita las acciones de tus usuarios}\\
    
    \noindent Ya definido el objetivo, se tiene que ver que acciones tus usuarios podrán desarrollar en el sistema, dichas acciones deben ser concretas y específicas. Por ejemplo: Iniciar sesión el la página web, compartir la información del trabajo vía twitter y comentar en una publicación de facebook. 
    
    \noindent Dichas acciones tienen que estar relacionadas con el ''¿Por qué?''.\\
    
    \noindent \textbf{3.- Describe a tus usuarios}\\
    
    \noindent ¿Qué usuarios estarán usando tu sistema? y aún más importante, ¿cuál es tu relación con ellos? y/o ¿qué tanto sabes de ellos? Esto para poder conocer qué podría motivarlos.\\
    
    
    \noindent \textbf{4.- Define tus actividades de inicio a fin}\\
    
    \noindent Conociendo a tus usuarios y tus objetivos ya puedes diseñar que actividades tendrá tu sistema y cómo es el flujo en cada una de ellas. En los juegos siempre las actividades tienen un inicio y a veces tienen un final. Y hay veces que se tienen ciclos antes de llegar al final. Por eso mismo se debe tomar en cuenta que hay 2 posibles formas de crear tu flujo de actividad: de forma de ciclo y forma de escaleras.\\
    
    
    
    
    \noindent \textbf{5.- Nunca olvides la diversión}\\
    
    \noindent Antes de empezar a usar el sistema se tiene que dar un paso atrás y preguntarte si al menos tú consideras que es divertido, si a ti te gustaría probar el hacer dichas actividades voluntariamente.\\
    
    
    \noindent \textbf{6.- Utiliza las herramientas adecuadas para el trabajo}\\
    
    \noindent En esta paso se tiene que especificar qué elementos de juego se utilizarán a lo largo de las actividades diseñadas anteriormente y empezar a codificarlas en tu sistema.\\
    
%\subsection{Marco de trabajo C}
% TODO: Agregar el paper que habla d epapers aquí.

\begin{comment}
\section{Elección de Marco de Trabajo}

%Hemos decidido utilizar  a Yukai-Cho(Octalysis)

    Gracias a que Octalysis divide la implementación de la Gamificación en 8 principios, da una flexibilidad mayor en su implementación puesto que se pueden elegir diferentes herramientas para implementar sus principios, a diferencia de los otros autores que solo enumeran las herramientas más usadas (puntos, insignias y tablas de clasificación.) y no dan cabida al uso de otras distintas decidimos utilizar 0ctalysis por las ventajas.
     
     - Modularidad de principios
     - 
     
     
    Que principios se tendrán.
\end{comment}



\clearpage
\section{Sistemas de aprendizaje en línea}
\label{sec:sistemasaprendizaje}

 En esta sección presenta los distintos sistemas de aprendizaje en linea, sus características
 principales, restricciones de uso y acceso a la documentación, posteriormente se constrastan
 distintos aspectos entre las plataformas y finalmente se detalla y argumenta la elección de 
 la plataforma sobre la cual se desarrollarán los distintos componentes de gamificación.
 A continuación se detallan las plataformas investigadas.

    

% \begin{comment}

\section{Acerca de la investigación}

La primer etapa de la investigación se centro en investigar acerca de qué es la \gls{gamificacion}, sus implicaciones y como debería ser implementada la busqueda permitió encontrar diversas definiciones de la gamificación, las bases sobre la cual está desarrollada y distintas recomendaciones su implementación.\\

\noindent En segunda instancia, se buscaron diversos casos de estudio y documentos de investigación que documentaran el proceso de implementación y los resultados de la investigación realizada, cada uno de los documentos encontrados nos sirve para conocer las fortalezas y áreas de oportunidad de otras implementación de gamificación.\\

\noindent Como tercer etapa, se investigaron distintos marcos de trabajo para el diseño e implementación de la gamificación, dicha busqueda obtuvo como resultado la elección de dos marcos de trabajo {\em ``Octalysis''} \cite{Octalysis} y {\em ``For The Win''} \cite{FrameWorkForTheWin} (descritos en el capítulo \nameref{ch:marcoTeorico}).\\

\noindent Posteriormente se analizaron los principales sistemas de aprendizaje en linea
buscando qué elementos de gamificación de forma nativa tenían, las restricciones de uso y
características principales. Esta etapa de investicación permitió filtrar los sistemas
que brindaban soporte a la gamificación, de forma nativa o mediante componentes externos
(plugins), de los demás sistemas de aprendizaje en línea.

\clearpage


   % El estado del arte se compone de 2 partes. Una que compara las soluciones con gamificación en sistemas dedicados al aprendizaje y la otra que compara las soluciones con gamificación en plugins de Moodle.
A continuación se describen cada uno de los sistemas:


Con nuestra investigación encontramos varios sistemas dedicados al aprendizaje que cuentan con gamificación. A continuación se presenta una tabla que indica cómo es que dichos sistemas cuentan con gamificación.
   
    \def\aux{90}
    \addtable{|c|c|c|c|c|c|c|c|c|}{table:LMS_GMFC}{& %
        
        \rotatebox[origin=c]{\aux}{Duolingo   \cite{PagDuolingo}}  &
        \rotatebox[origin=c]{\aux}{Moodle     \cite{PagMoodle}}    &
        \rotatebox[origin=c]{\aux}{Docebo     \cite{PagDocebo}}    &
        \rotatebox[origin=c]{\aux}{SAP Litmos \cite{PagSAPLitmos}} &
        \rotatebox[origin=c]{\aux}{ATutor     \cite{PagATutor}}    & 
        \rotatebox[origin=c]{\aux}{ALEKS      \cite{PagALEKS}}     &
        \rotatebox[origin=c]{\aux}{Udemy      \cite{PagUdemy}}     &
        \rotatebox[origin=c]{\aux}{TalentLMS  \cite{PagTalentLMS}} \\\hline 
        
        Propia    & X & X &   & X &   & X & X & X \\\hline
        Extendida &   & X & X & X & X &   &   &   \\\hline
        
    }{Implementación de gamificación}
       
\noindent Al decir \textbf{Propia} en el cuadro \ref{table:LMS_GMFC} nos referimos a que el sistema gestor de aprendizaje ya tiene integrado en su funcionalidad la implementación  de gamificación. Y al decir \textbf{Extendida} nos referimos a que existen componentes externos (plugins) que implementan gamificación.\\
    
\noindent A continuación se describen los sistemas en el cuadro \ref{table:LMS_GMFC} y los elementos de gamificación con los que cuentan, de acuerdo con los marcos de referencia \nameref{sec:ForTheWin} y \nameref{sec:octalysis}.
\clearpage

    
\begin{multicols*}{2}    
\subsection*{Duolingo}
    
Duolingo es un sistema de aprendizaje dedicado a los idiomas, es un servicio web que
te brinda la posibilidad de crearte una cuenta y seleccionar entre 9 idiomas para aprender,
los cuales son: Inglés, guaraní, francés, alemán, catalán, espartano, italiano, portugués y ruso.\\

\noindent Duolingo divide un idioma en secciones y cada sección contiene sub-secciones,
que a su vez contienen unidades que se dividen en 5 niveles cada una. Al inicio Duolingo
solo te permite empezar una unidad.
    
\noindent Al completar el primer nivel de todas las unidades de una sub-sección,
Duolingo te permite avanzar a la siguiente sub-sección. Y para poder acceder a la siguiente
sección Duolingo te pide que completes una cierta cantidad de niveles de unidades.\\
    
    \noindent Duolingo cuenta con varios módulos que están orientados a la gamificación, se utilizaron
    los elementos de juego definidos por el marco de trabajo ``For the Win'' para formar la siguiente lista:
    
    \begin{itemize}
        \item {\bf Logros:} Cuenta con un sistema de logros o en este caso ``insignias''
            que están divididas en 3 niveles, y cada vez que alcanzas un nivel se 
            desbloquea una estrella que se muestra en el icono del logro.
            
        \item {\bf Desbloqueo de contenido:} Al dividir el contenido de la forma
            anteriormente explicada, Duolingo permite visualizar tu progreso viendo
            la cantidad de unidades completadas y desbloqueadas.
            
        \item {\bf Puntos y niveles de experiencia:} Cada que completas un nivel de una
        unidad se te otorgan puntos de experiencia y esto te permite subir de nivel.
        Cabe aclarar que la experiencia y el nivel están relacionados con el idioma,
        esto quiere decir que puedo ser nivel 10 en inglés pero también ser nivel 1 en francés.
        
        \item {\bf Tablas de líderes:} Si agregas a alguien como tu amigo en Duolingo ambos 
        podrán ver su progreso semanal, mensual y total. El resultado de que el sistema
         los compara genera la tabla de líderes. 
         
        \item {\bf Misiones:} Duolingo permite que te pongas una meta diaria y una meta semanal.
        
    \end{itemize}


\subsection*{Docebo}
    
Docebo es un servicio web que se enfoca en la creación de dominios donde se brinda
un sistema gestor de aprendizaje, es decir, que uno pueda tener su página en línea
donde pueda crear y gestionar sus cursos y los alumnos puedan entrar a tomarlos.\\
    
\noindent Docebo no cuenta con gamificación de raíz, sino que se necesita instalar
plugins que se desarrollan con la API de Docebo, dichos plugins hasta el momento 
solo cuentan con:\\

    \begin{itemize}
        \item {\bf Logros:} Se cuenta con un sistema de logros,
        que se desbloquean si la persona cumple con sus condiciones.
    \end{itemize}
    
    
    
\subsection*{SAP Litmos}
    
SAP Litmos es un sistema que te permite crear cursos para tu equipo de trabajo,
así como delegar tareas y ver el progreso de las mismas. Esta orientado a
fortalecer el capital humano de una empresa.\\
    
    \noindent SAP Litmos cuenta con 3 módulos de gamificación, los cuales son:
    
    \begin{itemize}
        \item {\bf Insignias:} A diferencia que con los logros, estos
        no son otorgados cuando se cumple una cierta condición, sino 
        que el administrador crea una insignia y se le otorga a un usuario.
        
        \item {\bf Equipos: } Debido a que está orientado al capital humano
         de una empresa, uno puede crear equipos que sean por área de la
          empresa y así ver si las áreas están cumpliendo con sus tareas.
          
        \item {\bf Tablas de líderes y puntos: } SAP Litmos te muestra una
         gráfica de que tanto han avanzado los usuarios en un cierto curso
         o en sus tareas. Esto mediante una gráfica y asignación de puntos.
         
    \end{itemize}


\clearpage
\subsection*{ATutor}

ATutor es un un sitema gestor de aprendizaje de software libre. Para poder
utilizarlo se necesita tener un servidor web y montar dicho código en el servidor.\\
    
    \noindent ATutor no cuenta con gamificación de raíz,
    pero cuenta con un plugin llamado \textbf{GameMe} que agrega:
    
    \begin{itemize}
        \item {\bf Logros:} Dichos logros son estáticos y se
        desbloquean cuando se un usuario cumple las condiciones.
        
        \item {\bf Puntos y niveles de experiencia:} Hay definidos 10
        niveles de experiencia y cada que ocurre un evento que tenga
        que ver con un usuario, se le otorga experiencia.
        
    \end{itemize}



\subsection*{ALEKS}

ALEKS es un servicio web que ofrece un sistema gestor de aprendizaje
que adapta el contenido al usuario utilizando inteligencia artificial.
Esto lo mantienen controlado utilizando únicamente ciertos tipos de cursos.\\
    
    \noindent ALEKS cuenta con gamificación de raíz, los elementos con los que cuenta son:
    
    \begin{itemize}
        \item {\bf Progresión:} El fuerte de ALEKS es utilizar la inteligencia
        artificial y algoritmos de predicción así que tiene un montón de datos del
        usuario que aprovecha desplegándolos en gráficos que muestran el progreso
        en diversos temas de un curso, así como el porcentaje del curso que se ha
        tomado, dominado o que falta por revisar. Cabe destacar que un profesor puede
        ver los gráficos de cada alumno, pero los alumnos no pueden ver el de los demás.
        
    \end{itemize}
    
\vfill\null
\columnbreak
\subsection*{Udemy}

Udemy es un servicio web que te permite tomar cursos y/o subir tus cursos.
El formato de los cursos es siempre un video. Cuanta con muchos temas
gracias a que cualquiera puede crear su curso.\\

    \noindent Usando como referencia al marco de trabajo octalysis,
    Udemy cuenta con los siguientes principios de gamificación:
    
    \begin{itemize}
        \item {\bf \principioIII} Debido a que cualquiera puede subir
        sus cursos y recibir retroalimentación de los que lo tomaron,
        se cumple este principio, pero dico principio está orientado
        hacia los creadores de cursos.
        
    \end{itemize}
    
    
    
\subsection*{TalentLMS}

TalentLMS es un servicio web que se enfoca en la creación de dominios donde se
brinda un sistema gestor de aprendizaje, es decir, que uno pueda tener su página
en línea donde pueda crear y gestionar sus cursos y los alumnos puedan entrar a tomarlos.\\
    
    \noindent TalentLMS cuenta con gamificación de raíz,
    y los elementos de juego con los que cuenta, son:
    
    \begin{itemize} 
    
        \item {\bf Logros:} Se cuenta con un sistema de logros o en este caso
        ``insignias'' que están divididas en 8 niveles, y cada vez que alcanzas
        un nivel se desbloquea la insignia en su color correspondiente.
        
        \item {\bf Puntos y niveles de experiencia:} Cada que ocurre un
        determinado evento que tenga que ver con un usuario, se le otorga experiencia.
        
        \item {\bf Tablas de líderes y puntos:} TalentLMS muestra la
        tabla de líderes por categoría de curso, esto a nivel ''plataforma''.
        
    \end{itemize}
    
\end{multicols*}



\clearpage
\subsection{Moodle}
   
En la sección del \nameref{ch:marcoTeorico} se especifica Moodle más a fondo y el cómo se desarrollan plugins para el mismo. Es por eso que nos limitamos a hacer comparativas de los elementos de gamificación con los que cuentan los diferentes plugins en la tabla \ref{tbl:pluginscreated}

%\noindent En la tabla \ref{table:pluginComp} se comparan los diferentes plugins que existen en el sistema Moodle.
   
%Antes de poder presentar que otras soluciones existen, se necesita establecer un contexto. Es por ello que se tiene que definir en que sistema gestor de aprendizaje se trabajará en este trabajo terminal.  Existen varios sistemas gestores de aprendizaje disponibles para su uso actualmente, sin embargo, para determinar cuál se usará a lo largo de este trabajo terminal se realizó la siguiente tabla comparativa.

%Ya sabiendo que se utilizará Moodle, podemos fijarnos en las soluciones existentes, que en este caso serían los componentes (traducción de plugins del inglés) que cuentan con gamificación. La tabla \ref{table:1} compara de acuerdo a sus elementos los diferentes  componentes que implementan gamficación en la plataforma Moodle \cite{arte1}.
 

    \addtable{|l|c|c|c|c|c|c|c|c|c|c|}{tbl:pluginscreated}{%
        Elementos/Plugins &
        \rotatebox[origin=c]{\aux}{LevelUp!         \cite{arte2}}  &
        \rotatebox[origin=c]{\aux}{Ranking block    \cite{arte3}}  &
        \rotatebox[origin=c]{\aux}{Game             \cite{arte4}}  &
        \rotatebox[origin=c]{\aux}{Quizventure      \cite{arte5}}  &
        \rotatebox[origin=c]{\aux}{Stash            \cite{arte6}}  &
        \rotatebox[origin=c]{\aux}{Mootivated       \cite{arte7}}  &
        \rotatebox[origin=c]{\aux}{UNEDrivial       \cite{arte8}}  &
        \rotatebox[origin=c]{\aux}{Stamp collection \cite{arte9}}  &
        \rotatebox[origin=c]{\aux}{Exabis games     \cite{arte10}} &
        \rotatebox[origin=c]{\aux}{Badge leader     \cite{arte11}} \\\hline
        
        Competencias            &   &   & X & X &   &   & X &   & X &   \\\hline
        Niveles                 & X &   &   &   &   &   &   &   &   &   \\\hline
        Desbloqueo de contenido & X &   &   &   & X &   &   &   &   &   \\\hline
        Logros                  & X &   &   &   &   &   & X &   &   & X \\\hline
        Esquema financiero      &   &   &   &   &   &   &   &   &   &   \\\hline
        Cajas de botín          &   &   &   &   &   &   &   &   &   &   \\\hline
        Puntos                  & X & X &   &   &   &   &   &   & X &   \\\hline
        Tienda                  &   &   &   &   &   &   &   &   &   &   \\\hline
        Tabla lideres           & X &   &   &   &   &   & X & X &   & X \\\hline
        Barra de progreso       & X &   &   &   &   &   &   &   &   &   \\\hline
        
    }{Tabla de comparación de componentes externos (plugins) en Moodle}

\end{comment}



\clearpage
\section{Moodle}

 Moodle es una plataforma de aprendizaje diseñada para brindar a los educadores, administradores
 y alumnos un único sistema sólido, seguro e integrado para crear entornos de aprendizaje personalizados \cite{aboutMoodle}.
 Moodle inicialmente hace referencia al acrónimo en inglés {\it Modular Object Oriented Dynamic
 Learning Environment} o en español Entorno de Aprendizaje Dinamico y Modular Orientado a Objetos \cite{aboutMoodle19}.\\

    % https://docs.moodle.org/all/es/Acerca_de_Moodle
    % https://docs.moodle.org/all/es/19/Acerca_de_Moodle

 \noindent Moodle es proporcionado gratuitamente como programa de código Abierto, bajo la GNU-GPL
 (GNU General Public License), esta licencia permite que Moodle sea adecuado y personalizado libremente
 ya que su configuración modular y diseño inter-operable permite a los desarrolladores crear plugins
 e integrar aplicaciones externas para lograr funcionalidades específicas \cite{aboutMoodle}.

 % La Licencia GPL indica que que cualquier persona puede adaptar, extender o modificar, tanto para
 % proyectos comerciales como no-comerciales, sin pago de cuotas por licenciamiento, bajo la condición
 % de que, al darse una copia de la aplicación se brinde tambien él código fuente de la misma. \cite{GPL}
 % http://mchapman.com/amb/soft/gpl.pdf

    \begin{quote}
    Durante el desarrollo del trabajo terminal se utiliza la versión 3.5 de moodle,
    debido a que es la versión más reciente con soporte a largo plazo (Moodle 3.5 LTS)
    al mes de febrero de 2019. \cite{moodleHistorial}
    \end{quote}

    % https://docs.moodle.org/all/es/dev/Historia_de_las_versiones#Moodle_3.6

\subsection{Arquitectura de Moodle}

 Moodle trabaja sobre una arquitectura cliente-servidor, específicamente requiere de un servidor web
 con soporte para PHP y acceso a una base de datos (MySQL, PostgreSQL, Microsoft SQL Server, MariaDB
 u Oracle).\\
    
 \noindent Como se puede ver en la figura \ref{moodle:arch}, la estructura interna que tiene Moodle está
 divida en los {\it componentes requeridos}, que incluyen el núcleo y los subsistemas; y los {\it elementos
 opcionales} que incluyen propiamente los plugins con sus respectivos subplugins. Moodle está diseñado para
 ser altamente extensible y personalizable a través del desarrollo de plugins sin la necesidad de modificar
 el núcleo o los subsistemas. \cite{moodleArch}.

    % https://docs.moodle.org/dev/Moodle_architecture
    
    \addfigure[(adaptado de {\it Moodle Architecture}  \cite{moodleArch})]%
        {0.4}{investigacion/images/MoodleArch}{moodle:arch}{Componentes que conforman la estructura interna de Moodle}
    
 \noindent Debido a que Moodle está conformado tanto de elementos requeridos (núcleo/core y subsistemas)
 como opcionales (plugins), los tipos de comunicación permitidos estan regidos por un conjunto de reglas
 descritas a continuación \cite{moodleComponets}.

    % https://docs.moodle.org/dev/Communication_Between_Components
    
    \begin{itemize}
    \item Es permitido que cualquier componente se puede comunicar con los componentes
          requeridos de moodle (núcleo y los subsistemas).

    \item Cualquier componente puede comunicarse con sí mismo.

    \item Es permitido comunicarse con otros componentes de los cuales se especifique
          explicitamente la dependencia.

    \item Los subplugins pueden comunicarse con el plugin que los contiene, y con cualquier
          otro plugin del cual dependan explícitamente.

    \item Todas las demás comunicaciones entre componentes están prohibidas.
    \end{itemize}


\subsection{Núcleo de Moodle}

 El núcleo de Moodle contiene las bibliotecas que proporcionan funcionalidades que requieren todas
 las demás partes de Moodle. El código del núcleo no puede ser eliminado sin comprometer la funcionalidad
 básica de Moodle, El núcleo de Moodle siempre está disponible y se puede llamar de forma segura desde
 cualquier otro componente \cite{moodleComponets}.\\

 \noindent El núcleo proporciona un conjunto de 51 APIs que forman parte del núcleo \cite{moodleCoreAPIs},
 las 51 API son listadas a continuación. % https://docs.moodle.org/dev/Core_APIs

    \begin{multicols}{3}
        \newcommand{\API}[2]{\item #1 }
% Argument 1 = Nombre del API
% Argument 2 = Descripción del API

\begin{itemize}
    \API{Access API (access)}{%
    The Access API gives you functions so you can determine what the current user is allowed to do, and it allows modules to extend Moodle with new capabilities.}
    
    \API{Data manipulation API (dml)}{%
    The Data manipulation API allows you to read/write to databases in a consistent and safe way.}
    
    \API{File API (files)}{%
    The File API controls the storage of files in connection to various plugins.}
    
    \API{Form API (form)}{% 
    The Form API defines and handles user data via web forms.}
    
    \API{Logging API (log)}{%
    The Event 2 API allows you to log events in Moodle, while Logging 2 describes how logs are stored and retrieved.}
    
    \API{Navigation API (navigation)}{%
    The Navigation API allows you to manipulate the navigation tree to add and remove items as you wish.}
    
    \API{Page API (page)}{%
    The Page API is used to set up the current page, add JavaScript, and configure how things will be displayed to the user.}
    
    \API{Output API (output)}{%
    The Output API is used to render the HTML for all parts of the page.}
    
    \API{String API (string)}{%
    The String API is how you get language text strings to use in the user interface. It handles any language translations that might be available.}
    
    \API{Upgrade API (upgrade)}{%
    The Upgrade API is how your module installs and upgrades itself, by keeping track of its own version.}
    
    \API{Moodlelib API (core)}{%
    The Moodlelib API is the central library file of miscellaneous general-purpose Moodle functions. Functions can over the handling of request parameters, configs, user preferences, time, login, mnet, plugins, strings and others. There are plenty of defined constants too.}
    
    \API{Admin settings (admin)}{%
    The Admin settings API deals with providing configuration options for each plugin and Moodle core.}
    
    \API{Analytics API (analytics)}{%
    The Analytics API allow you to create prediction models and generate insights.}
    
    \API{Availability (availability)}{%
    The Availability API controls access to activities and sections.}
    
    \API{Backup API (backup)}{%
    The Backup API defines exactly how to convert course data into XML for backup purposes, and the Restore API describes how to convert it back the other way.}
    
    \API{Cache API (cache)}{%
    The The Moodle Universal Cache (MUC) is the structure for storing cache data within Moodle. Cache_API explains some of what is needed to use a cache in your code.}
    
    \API{Calendar API (calendar)}{%
    The Calendar API allows you to add and modify events in the calendar for user, groups, courses, or the whole site.}
    
    \API{Comment API (comment)}{%
    The Comment API allows you to save and retrieve user comments, so that you can easily add commenting to any of your code.}
    
    \API{Competency API (competency)}{%
    The Competency API allows you to list and add evidence of competencies to learning plans, learning plan templates, frameworks, courses and activities.}
    
    \API{Data definition API (ddl)}{%
    The Data definition API is what you use to create, change and delete tables and fields in the database during upgrades.}
    
    \API{Editor API}{%
    The Editor API is used to control HTML text editors.}
    
    \API{Enrolment API (enrol)}{%
    The Enrolment API deals with course participants.}
    
    \API{Events API (event)}{%
    The Event 2 allows to define "events" with payload data to be fired whenever you like, and it also allows you to define handlers to react to these events when they happen. This is the recommended form of inter-plugin communication. This also forms the basis for logging in Moodle.}
    
    \API{External functions API (external)}{%
    The External functions API allows you to create fully parametrised methods that can be accessed by external programs (such as Web services).}
    
    \API{Favourites API}{%
    The Favourites API allows you to mark items as favourites for a user and manage these favourites. This is often referred to as 'Starred'.}
    
    \API{Lock API (lock)}{%
    The Lock API lets you synchronise processing between multiple requests, even for separate nodes in a cluster.}
    
    \API{Message API (message)}{%
    The Message API lets you post messages to users. They decide how they want to receive them.}
    
    \API{Media API (media)}{%
    The Media API can be used to embed media items such as audio, video, and Flash.}

    \API{My profile API}{%
    The My profile API is used to add things to the profile page.}

    \API{OAuth 2 API (oauth2)}{%
    The OAuth 2 API is used to provide a common place to configure and manage external systems using OAuth 2.}

    \API{Preference API (preference)}{%
    The Preference API is a simple way to store and retrieve preferences for individual users.}

    \API{Portfolio API (portfolio)}{%
    The Portfolio API allows you to add portfolio interfaces on your pages and allows users to package up data to send to their portfolios.}
    
    \API{Privacy API (privacy)}{%
    The Privacy API allows you to describe the personal data that you store, and provides the means for that data to be discovered, exported, and deleted on a per-user basis. This allows compliance with regulation such as the General Data Protection Regulation (GDPR) in Europe.}
    
    \API{Rating API (rating)}{%
    The Rating API lets you create AJAX rating interfaces so that users can rate items in your plugin. In an activity module, you may choose to aggregate ratings to form grades.}

    \API{RSS API (rss)}{%
    The RSS API allows you to create secure RSS feeds of data in your module.}

    \API{Search API (search)}{%
    The Search API allows you to index contents in a search engine and query the search engine for results.}

    \API{Tag API (tag)}{%
    The Tag API allows you to store tags (and a tag cloud) to items in your module.}

    \API{Task API (task)}{%
    The Task API lets you run jobs in the background. Either once off, or on a regular schedule.}

    \API{Time API (time)}{%
    The Time API takes care of translating and displaying times between users in the site.}

    \API{Testing API (test)}{%
    The testing API contains the Unit test API (PHPUnit) and Acceptance test API (Acceptance testing). Ideally all new code should have unit tests written FIRST.}

    \API{User-related APIs (user)}{%
    This is a rather informal grouping of miscellaneous User-related APIs relating to sorting and searching lists of users.}

    \API{Web services API (webservice)}{%
    The Web services API allows you to expose particular functions (usually external functions) as web services.}

    \API{Badges API (OpenBadges)}{%
    The Badges user documentation (is a temp page until we compile a proper page with all the classes and APIs that allows you to manage particular badges and OpenBadges Backpack).}

    \API{Custom fields API}{%
    The Custom fields API allows you to configure and add custom fields for different entities}

    \API{Activity module APIs}{%
    Activity modules are the most important plugin in Moodle. There are several core APIs that service only Activity modules.}

    \API{Activity completion API (completion)}{%
    The Activity completion API is to indicate to the system how activities are completed.}

    \API{Advanced grading API (grading)}{%
    The Advanced grading API allows you to add more advanced grading interfaces (such as rubrics) that can produce simple grades for the gradebook.}

    %\API{Conditional activities API (condition) - deprecated in 2.7
    %The deprecated Conditional activities API used to provide conditional access to modules and sections in Moodle 2.6 and below. It has been replaced by the Availability API.}

    \API{Groups API (group)}{%
    The Groups API allows you to check the current activity group mode and set the current group.}

    \API{Gradebook API (grade)}{%
    The Gradebook API allows you to read and write from the gradebook. It also allows you to provide an interface for detailed grading information.}

    \API{Plagiarism API (plagiarism)}{%
    The Plagiarism API allows your activity module to send files and data to external services to have them checked for plagiarism.}

    \API{Question API (question)}{%
    The Question API (which can be divided into the Question bank API and the Question engine API), can be used by activities that want to use questions from the question bank.}
    
\end{itemize}

    \end{multicols}


\subsection{Subsistemas}

 Los subsistemas son grupos de funciones y clases que forman parte del núcleo, pero se agrupan
 lógicamente al mismo. Están vinculados a una función particular y bajo condiciones especificas
 pueden desactivarse/habilitarse \cite{moodleComponets}.


\subsection{Plugins y subplugins}\label{subsec:plugins}

 Los plugins son componentes opcionales que permiten extender las funcionalidades de Moodle.
 Hay muchos tipos diferentes de plugins, y cada plugin permite brindar distintas funcionalidades
 correspondientes al tipo de plugin. El desarrollo de plugins es la manera recomendada para extender
 la funcionalidad de Moodle.\\

 \noindent Actualmente Moodle menciona en su documentación 54 tipos de plugins los cuales son
 listados a continuación.

    \begin{multicols}{3}
        \newcommand{\plugin}[2]{\item #1 }
% Argument 1 = Nombre del plugin
% Argument 2 = Descripción del plugin

\begin{itemize}
        \plugin{Activity Modules}{%
	  }

	\plugin{Questions Types}{%
	  }

	\plugin{Course Reports}{%
	  }

	\plugin{Antivirus plugins}{%
	  }

	\plugin{Question Behaviours}{%
	  }

	\plugin{Gradebook export}{%
	  }

	\plugin{Assignment submission plugins}{%
	  }

	\plugin{Questions Import/Export Formats}{%
	  }

	\plugin{Gradebook import}{%
	  }

	\plugin{Assignment feedback plugins}{%
	  }

	\plugin{Text Filters}{%
	  }

	\plugin{Gradebook reports}{%
	  }

	\plugin{Book tools}{%
	  }

	\plugin{Editors}{%
	  }

	\plugin{Advanced Grading Methods}{%
	  }

	\plugin{Database Fields}{%
	  }

	\plugin{Atto Editor Plugins}{%
	  }

	\plugin{MNET Services}{%
	  }

	\plugin{Database Presets}{%
	  }

	\plugin{TinyMCE editor Plugins}{%
	  }

	\plugin{Web Service Protocols}{%
	  }

	\plugin{LTI sources}{%
	  }

	\plugin{Enrolment Plugins}{%
	  }

	\plugin{Repository Plugins}{%
	  }

	\plugin{File Converters}{%
	  }

	\plugin{Authentication Plugins}{%
	  }

	\plugin{Portfolio plugins}{%
	  }

	\plugin{LTI services}{%
	  }

	\plugin{Admin Tools}{%
	  }

	\plugin{Search Engines}{%
	  }

	\plugin{Machine Learning Backends}{%
	  }

	\plugin{Log Stores}{%
	  }

	\plugin{Media Players}{%
	  }

	\plugin{Quiz Reports}{%
	  }

	\plugin{Availability Conditions}{%
	  }

	\plugin{Plagiarism Plugins}{%
	  }

	\plugin{Quiz Access Rules}{%
	  }

	\plugin{Calendar Types}{%
	  }

	\plugin{Cache Store}{%
	  }

	\plugin{SCORM Reports}{%
	  }

	\plugin{Messaging Consumers}{%
	  }

	\plugin{Cache Locks}{%
	  }

	\plugin{Workshop Grading Strategies}{%
	  }

	\plugin{Course Formats}{%
	  }

	\plugin{Themes}{%
	  }

	\plugin{Workshop Allocations Methods}{%
	  }

	\plugin{Data Formats}{%
	  }

	\plugin{Local Plugins}{%
	  }

	\plugin{Workshop Evaluaction Methods}{%
	  }

	\plugin{User Profile Fields}{%
	  }

	\plugin{Legacy Assignment Types}{%
	  }

	\plugin{Blocks}{%
	  }

	\plugin{Reports}{%
	  }

	\plugin{Legacy Admin Reports}{%
	  }
\end{itemize}
    \end{multicols}

 \clearpage

 \noindent Para la mayoría de los tipos plugins, Moodle tiene una estructura estandarizada para
 los archivos que debe contener un plugin. En la figura \ref{fig:pluginFiles} se representa dicha
 estructura. Los archivos y directorios son descritos a continuación \cite{moodlePluginfiles}:

     \addfigure{0.8}{investigacion/images/PluginFiles}{fig:pluginFiles}%
        {Organización de los archivos presentes en la mayoría de los plugins}

    \begin{quote}
        \begin{description}
    \item[version.php]
    Contiene los metadatos acerca del plugin como el número de versión o las
    dependencias hacia versiones de moodle y otros plugins.

    \item[lang/] Contiene las cadenas utilizadas por el plugin por defecto y
    las traducciones a utilizar (si son especificadas).

    \item[lib.php] Define la interfaz entre el núcleo de moodle y el plugin.
    El contenido de este archivo depende del tipo de plugin que se vaya a
    desarrollar.

    \item[db/install.xml] Contiene el esquema de las tablas, campos, índices
    y llaves que se deben crear al instalarse el plugin. Este archivo debería
    crearse mediante la herramienta XMLDB integrada en moodle.

    \item[db/upgrade.php] Contiene los pasos para actualizar una instalación de
    un plugin, como los cambios en la base de datos, de la misma forma puede
    contener otras acciones requeridas al momento de una actualización de un
    plugin.

    \item[db/access.php]
    Define las acciones que un usuario tiene permitido
    hacer acerca del plugin que se desarrolla.

    \item[db/install.php]
    Permite ejecutar código PHP inmediatamente después
    de que el esquema presente en install.xml ha sido creado.

    \item[db/uninstall.php]
    Permite ejecutar código PHP después de que las tablas
    y datos correspondientes al plugin hayan sido eliminados durante la
    desinstalación.

    \item[db/events.php]
    Contiene las suscripciones a los eventos que el plugin a desarrollar procesará.

    \item[db/messages.php]
    Permite declarar o publicar el plugin como un proveedor de mensajes.

    \item[db/services.php]
    Contiene las funciones externas o servicios web que proporciona el plugin.

    \item[db/renamedclasses.php]
    Detalla las clases que han sido renombradas para su carga automática.

    \item[classes/]
    Contiene las distintas clases que son necesarias para el funcionamiento del
    plugin. Estos son cargadas de forma automática siguiendo las reglas de
    nomenclatura.

    \item[cli/]
    Contiene los scripts que permiten configurar el plugin desde la linea de comandos.

    \item[settings.php]
    Describe la configuración que el administrador puede realizar sobre el plugin.

    \item[amd/]
    Contiene código de JavaScript de los módulos asíncronos AMD (Asynchronous
    Module Definition)

    \item[yui/]
    Contiene los módulos YUI (Yahoo User Interface), usados en versión anteriores
    para incluir CSS y Javascript

    \item[jquery/]
    Contiene los módulos de JQuery para Javascript

    \item[styles.css]
    Contiene las hojas de estilos del plugin

    \item[pix/icon.svg]
    Contiene el icono del plugin, en la dimensión correspondiente al tipo de plugin.

    \item[thirdpartylibs.xml]
    Contiene la lista de todas las bibliotecas de terceros incluidas en el plugin.

    \item[readme\_moodle.txt]
    Este archivo debe contener instrucciones detalladas acerca de como importar las
    librearias presentes en ''thirdpartylibs.xml''.

    \item[environment.xml]
    Define sus requerimientos adicionales del entorno en donde se ejecuta moodle,
    como estensiones específicas de PHP.

    \item[README]
    (README.md o README.txt) debe contener información relevante acerca del plugin.

    \item[CHANGES]
    (CHANGES.md, CHANGES.txt, CHANGES.html o CHANGES) es el archivo encontrado cuando
    se sube una nueva versión del plugin al repositorio de plugins.
\end{description}


    \end{quote}

\subsection{Requerimientos}

 Moodle es desarrollado principalmente utilizando Linux como sistema operativo usando Apache
 como servidor web; PostgreSQL, MySQL o MariaDB como gestores de bases de datos; y PHP como
 lenguaje principal del lado del servidor. Se recomienda que Moodle sea instalado utilizando
 un entorno con las mismas tecnologías \cite{moodleInstall}. \\

 \noindent Los requisitos básicos de hardware son los siguientes:

    \begin{itemize}
    \item 200MB de Disco duro para el código de moodle más el espacio requerido
          para almacenar el contenido, moodle como mínimo recomienda 5GB.

    \item Procesador 1GHz como mínimo. Recomendado 2GHz dual-core o mayor.

    \item 512 MB de memoria RAM, 1GB o más recomendado, y para servidores en entorno
          de producción se recomiendan 8GB.
    \end{itemize}

 \clearpage

 \noindent Los requisitos de software varían dependiendo de la versión de moodle, para la
 versión 3.5 LTS son los siguientes \cite{moodleReleaseNotes}:
 % https://docs.moodle.org/dev/Moodle_3.5_release_notes
    
    \begin{itemize}
        \item PHP Versión 7.0 como mínimo, PHP 7.1.x and 7.2.x también son soportados.
        %PHP 7.x could have some engine limitations
        \item Extensión {\it Intl} de PHP
        \item Bases de datos
            \begin{itemize}
            \item PostgreSQL v9.3 o mayor
            \item MySQL v5.5.31 o mayor
            \item MariaDB v5.5.31 o mayor
            \item Microsoft SQL Server 2008 o mayor
            \item Oracle Database v10.2 o mayor
            \end{itemize}
        \item[] (Recomendación), Si se usa MySQL o MariaDB, deberán estar configurados para soportar en conjunto de caracteres {\it utf8mb4}.
    \end{itemize}
    
