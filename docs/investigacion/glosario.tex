

\newcommand{\term}[3]{%
    \newglossaryentry{#1}{name={#2},description={#3}}%
}

\makeglossaries

    \term{gamificacion}{gamificación}{%
    El uso de mecánicas de juegos en un entorno no lúdico}

\begin{comment}
\chapter{Glosario}

% ===============================================================================
%   D E F I N I C I O N E S   O R D E N A D A S   A L F A B E T I C A M E N T E
% ===============================================================================

\newcommand{\term}[2]{\item[\hypertarget{#1}{#2}]}
\newenvironment{glosario}{\begin{description}}{\end{description}}

    \term{tGamificacion}{Gamificación} La experiencia o puntos de experiencia son un
    valor que permite cuantificar la expertiz que un usuario tiene en un juego.
    Comunmente son denotados por las abreviaciones 'xp' o 'exp'.

    \item[Experiencia] La experiencia o puntos de experiencia son un valor que
    permite cuantificar la expertiz que un usuario tiene en un juego. Comunmente
    son denotados por las abreviaciones 'xp' o 'exp'.

    \item[Logro] Elemento distintivo que señala la completitud de una acción o conjunto de
    acciones relevantes en un juego.

    \item[Nivel] Posición que un jugador tiene basada en la cantidad de puntos de
    experiencia \cite[p. 197]{conveptosVJNiveles}.

    \item[Plugin] Componentes que permiten añadir características y funcionalidades
    adicionales a las que proporciona Moodle de forma nativa.
    % https://docs.moodle.org/37/en/Installing_plugins

%\begin{comment}
\subsection*{Logros}

    Según el Diccionario del Español de México (DEM) logro es un ''Acto de lograr o conseguir algo''  y según el Diccionario del Español de México (DEM) lograr es, ''Llegar a tener algo que se deseaba, o conseguir el resultado que se buscaba''. \\

    \noindent Los logros en los videojuegos son recompensas o reconocimientos obtenidos por los jugadores para la realización dentro del juego \cite{conceptoDJLogros}.\\

    \noindent Aún con las definiciones anteriores, existe una idea ambigua de lo que es un logro, es por eso que se propone la siguiente definición: \\ \\ \noindent Un logro es un elemento que cuenta con la siguiente estructura:

    \begin{itemize}
        \item \textbf{Nombre}: Un texto que le permite al usuario identificar el logro.
        \item \textbf{Estado}: Una bandera que indica si el usuario tiene o no el logro.
        \item \textbf{Condición}: Lo que necesita ocurrir para que el logro sea desbloqueado para el usuario .
        \item \textbf{Descripción}: Un texto que le explica al usuario la condición del logro.
    \end{itemize}

    %\noindent Se puede notar que el 'Acto de lograr' sería el acto de cumplir con la condición del logro del contexto de los videojuegos. \\

    \noindent A lo largo de este trabajo terminal, se usará la palabra \textbf{logro} para referirse a la propuesta antes mencionada.

\subsection*{Experiencia}

    Según el Diccionario del Español de México  experiencia es ''Conocimiento al que se llega por la práctica o después de muchos años de vida''.\\

    \begin{description}
        \item[Puntos de experiencia] Es la unidad que representa la cantidad de actividades completadas por un usuario.
        \item[Nivel] Posición conseguida por una cantidad de puntos de experiencia representada por un número natural.
    \end{description}

    Para la definición del nivel, nos apoyamos en la definición del diseñador del juego "Pac Man World" que define al nivel en el contexto de los videojuegos como 'Posición del jugador basado en su puntaje, cantidad de puntos de experiencia o habilidad' \cite[p. 197]{conveptosVJNiveles}.

    \subsubsection{Puntos}

    \noindent  Los videojuegos al representar la experiencia utilizan puntos (números enteros positivos), los cuales son otorgados por realizar acciones determinadas. Un tipo de acción siempre otorga la misma cantidad de puntos y diferentes tipos de acciones pueden o no dar la misma cantidad de puntos.\\

    \noindent Una persona tiene cierta cantidad de puntos dependiendo de la cantidad de acciones que realizó. Sí dos personas (\textit{A} y \textit{B}) realizan acciones que otorgan los mismos puntos de experiencia y la persona \textit{A} tiene más puntos de experiencia, quiere decir que la persona \textit{A} ha realizado más acciones que la persona \textit{B}.\\

    \noindent A lo largo de este trabajo terminal al usar \textbf{puntos de experiencia} nos referiremos a .  \\

    \subsubsection{Niveles}

    \noindent El diseñador del juego "Pac Man World" define al nivel en el contexto de los videojuegos como 'Posición del jugador basado en su puntaje, cantidad de puntos de experiencia o habilidad' \cite[p. 197]{conveptosVJNiveles}.\\

    \noindent A lo largo de este trabajo terminal al usar \textbf{nivel de experiencia} nos referiremos a la posición conseguida por una cantidad de puntos de experiencia representada por un número entero.


    \subsection{Términos relacionados con base de datos}

    \begin{description}
        \item[Súper clave] Una súper clave es aquel conjunto de uno o más atributos que nos permiten diferenciar entre un registro y otro en una misma tabla \cite[p. 24]{libroBaseDeDatosEspaniol}.
        \item[Clave candidata] Una clave candidata es una súper clave, pero cuyos subconjuntos propios no son súper clave \cite[p. 24]{libroBaseDeDatosEspaniol}.
        \item[Subconjunto propio] Un subconjunto propio $B$ de $A$ ($B \subset A$) es aquel conjunto que contiene elementos del conjunto $A$, pero no todos ellos ($B \neq A$) \cite[p. 79]{libroMatematicasDiscretas}.
        \item[Clave primaria] La clave primaria es una clave candidata, pero que fue elegida por el diseñador de la base de datos para ser la clave que diferenciará a los registros de una tabla \cite[p. 24]{libroBaseDeDatosEspaniol}.
        \item[Dependencia funcional] Se dice que un conjunto de atributos $B$ depende funcionalmente de un conjunto de atributos $A$ si y solo si, para todos los pares de tuplas $t_1$ y $t_2$ de una entidad tales que $t_1[A] = t_2[A]$, también ocurre que  $t_1[B] = t_2[B]$ \cite[p. 163]{libroBaseDeDatosEspaniol}.
    \end{description}

\end{comment}
