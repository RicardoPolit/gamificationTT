

\begin{comment}

\section{Acerca de la investigación}

 La primer etapa de la investigación se centro en investigar acerca de qué es la \gls{gamificacion},
 sus implicaciones y como debería ser implementada la busqueda permitió encontrar diversas
 definiciones de la gamificación, las bases sobre la cual está desarrollada y distintas
 recomendaciones su implementación.\\

 \noindent En segunda instancia, se buscaron diversos casos de estudio y documentos de investigación
 que documentaran el proceso de implementación y los resultados de la investigación realizada, cada
 uno de los documentos encontrados nos sirve para conocer las fortalezas y áreas de oportunidad de
 otras implementación de gamificación.\\

 \noindent Como tercer etapa, se investigaron distintos marcos de trabajo para el diseño e implementación
 de la gamificación, dicha busqueda obtuvo como resultado la elección de dos marcos de trabajo
 {\em ``Octalysis''} \cite{Octalysis} y {\em ``For The Win''} \cite{ForTheWin} (descritos en el
 capítulo \nameref{ch:marcoTeorico}).\\

 \noindent Posteriormente se analizaron los principales sistemas de aprendizaje en linea
 buscando qué elementos de gamificación, las restricciones de uso y características principales.
 Esta etapa de investicación permitió filtrar los sistemas que brindaban soporte a la gamificación,
 de forma nativa o mediante componentes externos (plugins), de los demás sistemas de aprendizaje
 en línea.

\clearpage
\end{comment}


\def\aux{90}

\begin{multicols*}{2}
\subsection*{Duolingo}

 Duolingo \cite{PagDuolingo} es un sistema de aprendizaje dedicado a los idiomas, es un servicio web que
 te brinda la posibilidad de crearte una cuenta y seleccionar entre 9 idiomas para aprender,
 los cuales son: Inglés, guaraní, francés, alemán, catalán, espartano, italiano, portugués y ruso.\\

 \noindent Duolingo divide un idioma en secciones y cada sección contiene sub-secciones,
 que a su vez contienen unidades que se dividen en 5 niveles cada una. Al inicio Duolingo
 solo te permite empezar una unidad.\\

 \noindent Al completar el primer nivel de todas las unidades de una subsección duolingo
 te permite avanzar a la siguiente subsección, para poder acceder a la siguiente es requerido
 completar la cantidad de niveles de unidades especificada.\\

 \noindent Duolingo cuenta con varios módulos orientados a la gamificación, los elementos
 de juego, de acuerdo con {\em For The Win}, que tiene son:

    \begin{itemize}
    \item {\bf Logros:} Cuenta con un sistema de logros o en este caso ``insignias''
        que están divididas en 3 niveles, y cada vez que alcanzas un nivel se
        desbloquea una estrella que se muestra en el icono del logro.

    \item {\bf Desbloqueo de contenido:} Al dividir el contenido de la forma
        anteriormente explicada, Duolingo permite visualizar tu progreso viendo
        la cantidad de unidades completadas y desbloqueadas.

    \item {\bf Puntos y niveles de experiencia:} Al completar un nivel de una unidad
        se otorgan puntos de experiencia usados para subir de nivel en un idioma.

    \item {\bf Tablas de líderes:} Si agregas a alguien como tu amigo en Duolingo ambos
        podrán ver su progreso semanal, mensual y total. El resultado de que el sistema
         los compara genera la tabla de líderes.

    \item {\bf Misiones:} Duolingo permite que te pongas una meta diaria y una meta semanal.

    \end{itemize}


\subsection*{Docebo}

 Docebo \cite{PagDocebo} es un servicio web que se enfoca en la creación de dominios donde se brinda
 un sistema gestor de aprendizaje, es decir, que uno pueda tener su página en línea
 donde pueda crear y gestionar sus cursos y los alumnos puedan entrar a tomarlos.\\

 \noindent Docebo no cuenta con gamificación de raíz, sino que se necesita instalar
 plugins que se desarrollan con la API de Docebo, dichos plugins hasta el momento
 solo cuentan con:

    \begin{itemize}
        \item {\bf Logros:} Se cuenta con un sistema de logros,
        que se desbloquean si la persona cumple con sus condiciones.
    \end{itemize}



\subsection*{SAP Litmos}

 SAP Litmos \cite{PagSAPLitmos} es un sistema que te permite crear cursos para tu equipo de trabajo,
 así como delegar tareas y ver el progreso de las mismas. Esta orientado a
 fortalecer el capital humano de una empresa.\\

    \noindent SAP Litmos cuenta con 3 módulos de gamificación, los cuales son:

    \begin{itemize}
        \item {\bf Insignias:} A diferencia que con los logros, estos
        no son otorgados cuando se cumple una cierta condición, sino
        que el administrador crea una insignia y se le otorga a un usuario.

        \item {\bf Equipos: } Debido a que está orientado al capital humano
         de una empresa, uno puede crear equipos que sean por área de la
          empresa y así ver si las áreas están cumpliendo con sus tareas.

        \item {\bf Tablas de líderes y puntos: } SAP Litmos te muestra una
         gráfica de que tanto han avanzado los usuarios en un cierto curso
         o en sus tareas. Esto mediante una gráfica y asignación de puntos.

    \end{itemize}


\subsection*{ATutor}

 ATutor \cite{PagATutor} es un un sitema gestor de aprendizaje de software libre. Para poder
 utilizarlo se necesita tener un servidor web y montar dicho código en el servidor.\\

    \noindent ATutor no cuenta con gamificación de raíz,
    pero cuenta con un plugin llamado \textbf{GameMe} que agrega:

    \begin{itemize}
        \item {\bf Logros:} Dichos logros son estáticos y se
        desbloquean cuando se un usuario cumple las condiciones.

        \item {\bf Puntos y niveles de experiencia:} Hay definidos 10
        niveles de experiencia y cada que ocurre un evento que tenga
        que ver con un usuario, se le otorga experiencia.

    \end{itemize}



\subsection*{ALEKS}

 ALEKS \cite{PagALEKS} es un servicio web que ofrece un sistema gestor de aprendizaje
 que adapta el contenido al usuario utilizando inteligencia artificial.
 Esto lo mantienen controlado utilizando únicamente ciertos tipos de cursos.\\

    \noindent ALEKS cuenta con gamificación de raíz, los elementos con los que cuenta son:

    \begin{itemize}
        \item {\bf Progresión:} El fuerte de ALEKS es utilizar la inteligencia
        artificial y algoritmos de predicción así que tiene un montón de datos del
        usuario que aprovecha desplegándolos en gráficos que muestran el progreso
        en diversos temas de un curso, así como el porcentaje del curso que se ha
        tomado, dominado o que falta por revisar. Cabe destacar que un profesor puede
        ver los gráficos de cada alumno, pero los alumnos no pueden ver el de los demás.

    \end{itemize}

\vfill\null
\columnbreak
\subsection*{Udemy}

 Udemy \cite{PagUdemy} es un servicio web que te permite tomar cursos y/o subir tus cursos.
 El formato de los cursos es siempre un video. Cuanta con muchos temas
 gracias a que cualquiera puede crear su curso.\\

    \noindent Usando como referencia al marco de trabajo octalysis,
    Udemy cuenta con los siguientes principios de gamificación:

    \begin{itemize}
        \item {\bf Creatividad} Debido a que cualquiera puede subir
        sus cursos y recibir retroalimentación de los que lo tomaron,
        se cumple este principio, pero dicho principio está orientado
        hacia los creadores de cursos.

    \end{itemize}



\subsection*{TalentLMS}

 TalentLMS \cite{PagTalentLMS} es un servicio web que se enfoca en la creación de dominios donde se
 brinda un sistema gestor de aprendizaje, es decir, que uno pueda tener su página
 en línea donde pueda crear y gestionar sus cursos y los alumnos puedan entrar a tomarlos.\\

    \noindent TalentLMS cuenta con gamificación de raíz,
    y los elementos de juego con los que cuenta, son:

    \begin{itemize}

        \item {\bf Logros:} Se cuenta con un sistema de logros o en este caso
        ``insignias'' que están divididas en 8 niveles, y cada vez que alcanzas
        un nivel se desbloquea la insignia en su color correspondiente.

        \item {\bf Puntos y niveles de experiencia:} Cada que ocurre un
        determinado evento que tenga que ver con un usuario, se le otorga experiencia.

        \item {\bf Tablas de líderes y puntos:} TalentLMS muestra la
        tabla de líderes por categoría de curso, esto a nivel ''plataforma''.

    \end{itemize}

\end{multicols*}



\clearpage
\subsection*{Moodle}

 Moodle \cite{PagMoodle} es una plataforma de aprendizaje diseñada para proporcionarle a educadores,
 administradores y estudiantes un sistema integrado único, robusto y seguro para crear
 ambientes de aprendizaje personalizados. Los elementos con los que cuenta moodle sin
 la adición de plugins son los siguientes:

 \begin{quote}
 \begin{itemize}
    \item {\bf Insignias}, pueden ser otorgadas dependiendo de múltiples variados
                criterios, existen insignias a nivel plataforma y a nivel curso.

    \item {\bf Desbloqueo de contenido}, los profesores o administradores pueden
                definir que las secciones de un curso se vayan desbloqueando
                conforme se vayan cumpliendo ciertas condiciones.

    \item {\bf Creatividad}, permite a los profesores crear distintos cursos
                experimentando con la inclusión de distintos tipos de ejercicios.
 \end{itemize}
 \end{quote}

 \noindent Una de las fortalezas más grandes de moodle es que fue diseñado para ser altamente
 extensible en cuanto a funcionalidades, dentro del inmenso catálogo de componentes {\it plugins}
 para moodle, se encontraron diez plugins que agregan funcionalidades de gamificación. En la tabla
 \ref{tbl:pluginscreated} se presenta a qué elementos de juego están vinculados estos {\it plugins}.


    \addtable{|l|c|c|c|c|c|c|c|c|c|c|}{tbl:pluginscreated}{%
        Elementos/Plugins &
        \rotatebox[origin=c]{\aux}{LevelUp!         \cite{LevelUp}}  &
        \rotatebox[origin=c]{\aux}{Ranking block    \cite{RankingBlock}}  &
        \rotatebox[origin=c]{\aux}{Game             \cite{Game}}  &
        \rotatebox[origin=c]{\aux}{Quizventure      \cite{QuizVenture}}  &
        \rotatebox[origin=c]{\aux}{Stash            \cite{Stash}}  &
        \rotatebox[origin=c]{\aux}{Mootivated       \cite{Mootivated}}  &
        \rotatebox[origin=c]{\aux}{UNEDrivial       \cite{UNEDTrivial}}  &
        \rotatebox[origin=c]{\aux}{\ Stamp collection \cite{StampCollection}\ } &
        \rotatebox[origin=c]{\aux}{Exabis games     \cite{ExabisGames}} &
        \rotatebox[origin=c]{\aux}{Badge leader     \cite{BagdeLadder}} \\\hline

        Competencias            &   &   & X & X &   &   & X &   & X &   \\\hline
        Niveles                 & X &   &   &   &   &   &   &   &   &   \\\hline
        Desbloqueo de contenido & X &   &   &   & X &   &   &   &   &   \\\hline
        Logros                  & X &   &   &   &   &   & X &   &   & X \\\hline
        Esquema financiero      &   &   &   &   &   &   &   &   &   &   \\\hline
        Cajas de botín          &   &   &   &   &   &   &   &   &   &   \\\hline
        Puntos                  & X & X &   &   &   &   &   &   & X &   \\\hline
        Tienda                  &   &   &   &   &   &   &   &   &   &   \\\hline
        Tabla lideres           & X &   &   &   &   &   & X & X &   & X \\\hline
        Barra de progreso       & X &   &   &   &   &   &   &   &   &   \\\hline

    }{Tabla de comparación de componentes (plugins) en Moodle}


\clearpage
\subsection{Elección de la plataforma}

 Para poder determinar cual es la plataforma de aprendizaje sobre la cual se trabaja durante
 el desarrollo de este trabajo terminal se realizó una investigación para saber cómo las
 plataformas de aprendizaje anteriormente mencionadas brindaban soporte a la gamificación.\\

 \noindent En la tabla \ref{table:lmsGam} se presenta la forma en que se encuentran o se
 pueden añadir elementos de gamificación a las plataformas, se consideran principalmente
 que las plataformas pueden brindar soporte a la gamificación de forma nativa o mediante
 la adición de componentes externos.

    \addtable{|c|c|c|c|c|c|c|c|c|}{table:lmsGam}{& %

        \rotatebox[origin=c]{\aux}{Duolingo   \cite{PagDuolingo}}  &
        \rotatebox[origin=c]{\aux}{Moodle     \cite{PagMoodle}}    &
        \rotatebox[origin=c]{\aux}{Docebo     \cite{PagDocebo}}    &
        \rotatebox[origin=c]{\aux}{SAP Litmos \cite{PagSAPLitmos}} &
        \rotatebox[origin=c]{\aux}{ATutor     \cite{PagATutor}}    &
        \rotatebox[origin=c]{\aux}{ALEKS      \cite{PagALEKS}}     &
        \rotatebox[origin=c]{\aux}{Udemy      \cite{PagUdemy}}     &
        \rotatebox[origin=c]{\aux}{TalentLMS  \cite{PagTalentLMS}} \\\hline

        Nativa    & X & X &   & X &   & X & X & X \\\hline
        Externa   &   & X & X & X & X &   &   &   \\\hline

    }{Implementación de gamificación}

 \noindent Como se puede ver, tan solo cuatro de las ocho plataformas de aprendizaje
 en línea permiten brindar soporte a la gamificación mediante componentes externos.
 Sin embargo solo dos de esas cuatro ({\it Moodle} y {\it SAP Litmos}) brindan soporte
 tanto de forma nativa como externa con componentes externos.\\

 \noindent A partir del resultado encontrado se prosiguió a ampliar los criterios para
 la elección de la plataforma haciendo una nueva comparativa entre las seis plataformas
 más relevantes. A continuación se describen los criterios considerados:

    \begin{quote}
    \begin{description}
    \item[Documentación de código.] El tener acceso a la documentación de código de
            la plataforma permite saber el la forma en que interactúan los distintos
            módulos de la plataforma y la forma en que se puede extender las distintas
            funcionalidades.

    \item[Idioma español o Inglés] Un aspecto importante es que la plataforma a utilizar
            tenga soporte para el idioma español o al menos en inglés, puesto que un
            idioma distinto a esos dos implicaría una barrera en el entendimiento requerido
            para su uso.

    \item[Licencia] Establece las restricciones que se tienen acerca del uso, modificación
            y distribución de la plataforma sobre la cual se trabajará.

    \item[Extensible] Hace referencia a si la plataforma permite extender su funcionalidad
            mediante el desarrollo de componentes externos.
    \end{description}
    \end{quote}

 \noindent La comparación entre las distintas plataformas con base en los criterios
 anteriormente definidos, puede ser vista en la tabla \ref{tbl:LMSs}, como se puede
 ver Tanto {\it Moodle} como {\it ATutor} cumplen de forma positiva con los criterios
 establecidos.

 \clearpage

    \addtable{|l|c|c|c|c|c|c|}{tbl:LMSs}{
        {\bf Características} &
        {\bf Moodle} &
        {\bf ATutor} &
        {\bf Docebo} &
        {\bf SAP Litmos} &
        {\bf Gnosis Connect} &
        {\bf TalentLMS} \\

        & \cite{PagMoodle} & \cite{PagATutor} & \cite{PagDocebo} & \cite{PagSAPLitmos}
        & \cite{PagGnosisConnect} & \cite{PagTalentLMS} \\\hline

        % Características       % Moodle  ATutor  Docebo   SAPLit   Gnosis   Talent
        Documentación de código & Sí      & Sí    & Sí     & Sí     & No     & No     \\\hline
        Idioma Español o Inglés & Sí      & Sí    & Sí     & Sí     & Sí     & Sí     \\\hline
        Tipo de Licencia        & GPLv3   & GPL   & Propia & Propia & Propia & Propia \\\hline
        Extensible              & Sí      & Sí    & Sí     & Sí     & No     & No     \\\hline
    }{Comparativa de las plataformas de aprendizaje}


 \noindent Un punto importante es que Moodle y ATutor se encuentran bajo la licencia GPL,
 la cual nos permite hacer uso y realizar modificaciones, además de que la aplicación es
 distribuida junto con el código fuente, sin embargo se optó por utilizar moodle considerando
 los siguientes puntos:

    \begin{itemize}
        \item Moodle ha sido y sigue siendo utilizado dentro de la comunidad de ESCOM,
              abriendo la posibilidad de resolver dudas específicas con profesores con
              mayor expertíz en el uso de moodle.

        \item Vinculado con el punto anterior, el utilizar Moodle amplia la oportunidad
              de que el producto final de este trabajo terminal sea utilizado minimizando
              los costos de interacción.

        \item A diferencia de ATutor, Moodle especifica la versión de la licencia GPL
              bajo la cual se encuentra.

        \item ATutor aparenta no estar siendo actualizado constantemente ya que se
              encontraron varios enlaces rotos en su página oficial.
    \end{itemize}

 %   Sistema de gestión de aprendizaje es la traducción de Learning Managment System (LMS)
 %   del Inglés. A continuación se enlistan las definiciones de distintos autores acerca
 %   de los sistemas gestores de aprendizaje:

 %   Un sistema de gestión de aprendizaje es:
 %
 %      - Un software que incluye  una lista de servicios que le permiten y
 %        ayudan al profesor con la gestión de sus cursos. \cite{LMS_1}
 %
 %      - Una aplicación de software basada en web diseñada para manejar
 %        material didáctico, interacción con el estudiante, herramientas
 %        de evaluación y reportes del progreso de aprendizaje de los estudiantes.
 %        \cite{LMS_2}
 %
 %      - Un software para el manejo y presentación de materiales didácticos en
 %        la internet, además de ofrecer funcionalidades para la colaboración
 %        en línea. \cite{LMS_3}


 % \noindent Existen varios sistemas gestores de aprendizaje disponibles para su uso
 % actualmente. Se realizó la siguiente tabla comparativa para poder determinar que
 % sistema gestor de aprendizaje se utilizará.

 % Todos los LMS investigados:
 %      Moodle*
 %      Claroline
 %      Docebo*
 %      SAP Litmos*
 %      Gnosis Conect*
 %      TalentLMS*
 %      eFront
 %      Sakai
 %      Edmodo
 %      ATutor*

 % Con lo anterior se entiende que Moodle y ATutor cumplen con nuestras necesidades, sin embargo, por preferencia de nuestros directores se decidió que Moodle será el que se usará para este trabajo terminal.
