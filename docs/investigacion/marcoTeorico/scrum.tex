\section{Metodología}

 Este proyecto es realizado mediante un desarrollo iterativo utilizando como referencia
 el marco de trabajo {\it Scrum}. A lo largo de esta sección se presenta el marco de trabajo,
 roles, eventos y artefactos del mismo. A continuación se mencionan tres definiciones
 de distintos autores para brindar una perspectiva más completa acerca del marco de trabajo.\\

    \noindent Sus creadores, K. Schwaber y J. Sutherland lo definen de la siguiente manera:
        \begin{quote}
        Scrum es un marco de trabajo en el cual las personas pueden abordar
        problemas complejos de adaptación mientras que, productiva y creativamente
        desarrollan productos con el mayor valor posible \cite{TheScrumGuide}.
        \end{quote}

    \noindent De acuerdo con Michele Sliger PMP (Project Management Professional) y CST (Certified
     Scrum Trainer):
        \begin{quote}
        Scrum es un método ágil de entrega iterativa e incremental de productos que
        utiliza una comunicación constante y tomas de decisiones en conjunto \cite{Sliger1}.
        \end{quote}

    \noindent Pete Deemer, CEO de GoodAgile (Certificadora de Scrum) lo define como:
            % y lider de ScrumPrimer.
        \begin{quote}
        Scrum es un marco de trabajo en el que equipos multifuncionales pueden crear productos
        o desarrollar proyectos de una forma iterativa e incremental \cite{ScrumPrimer}.
        \end{quote}


\clearpage

\subsection{Equipo de Scrum}

 El equipo de Scrum está conformado por el dueño del producto {\em(Product Owner)},
 el equipo de desarrollo {\em(Development Team)} y el maestro scrum
 {\em(Scrum Master)}. A continuación se especifican las responsabilidades de cada uno.

\subsubsection{Product Owner}

 \noindent El dueño del producto o {\em Product Owner} es responsable de maximizar el valor
 del producto resultante del trabajo del equipo de desarrollo, su principal responsabilidad
 es la gestión del artefacto llamado {\em``Product Backlog''}, dicha gestión incluye:

    \begin{quote}
    \begin{itemize}
        \item Expresar claramente los elementos del {\it Product Backlog}.
        \item Ordenar los elementos del {\it Product Backlog} para lograr mejor las metas y
               objetivos.
        \item Optimizar el valor del trabajo realizado por el equipo de desarrollo.
        \item Asegurarse de que el {\it Product Backlog} es visible, transparente y claro para todos.
        \item Asegurarse de que el equipo de desarrollo entienda los elementos del {\it Product
               Backlog} al nivel requerido.\\
    \end{itemize}
    \end{quote}

 \noindent El dueño del producto debe trabajar en conjunto con el equipo de desarrollo para
 cumplir con los puntos anteriormente mencionados. Además, este debe ser una persona, no un comité,
 en caso de existir un comité el dueño del producto será el representante de dicho comité.

\subsubsection{Equipo de Desarrollo}

 El equipo de desarrollo consiste en un grupo de profesionales que realizan el 
 trabajo requerido para entregar los incrementos del producto al final de cada 
 iteración o {\it Sprint}. El equipo de desarrollo debe cumplir con los siguientes 
 puntos:

    \begin{quote}
    \begin{itemize}
    \item Es un equipo auto-organizado, no es requerido que alguien les guíe
          acerca de cómo llevar a cabo los incrementos.

    \item El equipo debe ser multi-funcional y contener como equipo todas las
          habilidades necesarias para crear los incrementos del producto.

    \item No hay títulos o etiquetas para los miembros del equipo de desarrollo.

    \item No hay equipos internos en el equipo de desarrollo.

    \item Los miembros del equipo pueden tener habilidades y áreas de conocimiento
          especializadas, pero el conjunto de habilidades pertenecen al equipo.
    \end{itemize}
    \end{quote}

 \noindent El equipo de desarrollo debe ser lo suficientemente pequeño para permanecer ágil y lo
 suficientemente grande para completar un trabajo significativo en cada iteración.\\

 % TODO: Mejorar la redacción al final del párrafo
 \noindent Un equipo con menos de tres miembros disminuiría la interacción, lo que implica una
 menor productividad, mientras que tener más de nueve miembros requiere demasiada coordinación.
 Los equipos de desarrollo grandes generan demasiada complejidad para que un proceso empírico
 sea útil.

\clearpage

\subsubsection{Maestro Scrum}

 El maestro Scrum o {\it Scrum Master} es el líder que está al servicio del equipo 
 scrum, se encarga de ayudar al equipo a maximizar el valor y mejorar continuamente 
 la forma de trabajo, además es el encargado de guiar a todos los involucrados hacia
 una mejor implementación del marco de trabajo.

    \begin{quote}
    {\bf Responsabilidades relacionadas con el Product Owner }
    \begin{itemize}
    \item Asegurar que los objetivos, alcance y definición del producto sea
          entendido por todos los miembros del equipo de desarrollo.

    \item Encontrar técnicas para una gestión efectiva del artefacto
          {\it Product Backlog}.

    \item Ayudar al equipo de desarrollo a entender la necesidad de la claridad
          y objetividad de los elementos del {\it Product Backlog}.

    % TODO: Mejorar la redacción de este punto
    \item Entender la planeación del producto en un ambiente empírico.

    \item Asegurarse de que el dueño del producto organice el {\it Product
          Backlog} para maximizar su valor.

    \item Entender y practicar la agilidad.

    \item Facilitar los eventos de Scrum cuando sean solicitados o necesarios.\\
    \end{itemize}


    {\bf Responsabilidades relacionadas con el equipo de desarrollo}
    \begin{itemize}
    \item Entrenar al equipo para ser auto-organizado y multi-funcional.
    \item Ayudar al equipo a crear incrementos de alto valor.
    \item Resolver/remover los impedimentos que frenen el progreso del equipo.
    \item Facilitar los eventos de {\it Scrum} cuando sean solicitados o necesarios.
    \item Entrenar al equipo en entornos donde {\it Scrum} no puede ser completamente
          adoptado y/o entendido.\\
    \end{itemize}

    %\noindent {\bf Responsabilidades relacionadas con la organización}
    %\begin{itemize}
    %\item Liderar y entrenar a la organización en la adopción del marco de trabajo Scrum.
    %\item Planear las implementaciones de Scrum en la organización.
    %\item Ayudar a los empleados y a los stakeholders a entender Scrum.
    %\item Trabajar con otros Scrum Masters para incrementar la efectividad de la aplicación de Scrum en la organización.
    %\end{itemize}
    \end{quote}

\subsection{Eventos}

 Los eventos prescritos en {\it Scrum} son usados para crear regularidad en el proceso y minimizar
 la necesidad de juntas no definidas, todos los eventos tienen un tiempo establecido y pueden
 concluir una vez que su propósito es cumplido. El único evento cuya duración es fija y no puede
 ser acotado o alargado, son las iteraciones o {\it Sprints}.
 
\subsubsection{Sprint}

 Una iteración o {\it Sprint} es el lapso de tiempo en el cual un incremento del producto es 
 creado, las iteraciones son secuenciales, es decir, inician inmediatamente después del término
 de otro. Internamente cada Sprint consiste en las etapas de planeación, reuniones diarias
 ({\it Daily Scrum}), desarrollo, revisión y retroalimentación.\\

 \noindent A continuación se mencionan las características principales que debe tener este evento:

    \begin{itemize}
    \item Los sprints deben tener una duración máxima de un mes.
    \item No se pueden hacer cambios a la definición del objetivo del {\it Sprint}
          o {\it Sprint Goal}.
    \item Los objetivos de calidad no disminuyen.
    \item El alcance debe ser clarificado por el dueño del producto y negociado entre 
          él y el equipo de desarrollo.
    \end{itemize}


\subsubsection{Planeación}

 La planeación de cada iteración se realiza incluyendo a todos los miembros del equipo scrum,
 en dicha reunión se establece cuál será el incremento entregado al final del {\it sprint}, y la
 forma en que se logrará el objetivo del mismo.\\

 \noindent La duración de la etapa de planeación o {\it Sprint Planning} está 
 relacionada con la duración del {\it Sprint} y la cantidad de miembros en el equipo 
 scrum; para {\it Sprints} de un mes la reunión debe durar cómo máximo 8 horas. Dicha
 reunión debe producir como resultado la definición el alcance durante el {\it Sprint},
 el objetivo del mismo y las funcionalidades específicas que se desarrollarán.\\

 \noindent El conjunto de funcionalidades a desarrollar deben ser tomadas del artefacto {\it Product Backlog} para constituir el artefacto {\it Sprint Backlog} que contiene 
 la descripción de las funcionalidades específicas a desarrollar, cual define del 
 alcance del sprint. Para llevar a cabo esta reunión es necesario contar con el 
 incremento del {\it sprint} anterior más la base de conocimiento de cómo trabajó 
 el equipo.\\

 % El Team se auto-organiza para llevar acabo el trabajo del Sprint Backlog. Las estimaciones
 % de tiempo y esfuerzo se miden en unidades de un día o menos.

 % Al finalizar el Sprint Planning, el Team debe ser capaz de explicar al Product Owner y al Scrum
 % Master como se organizarán y la forma de trabajo que ocuparan para alcanzar el Sprint Goal y crear
 % el incremento.

 \noindent El objetivo de cada iteración o {\it Sprint Goal} debe ser claro y 
 entendible para todos los miembros del equipo scrum y redactado de tal forma que 
 oriente al equipo de desarrollo sobre el propósito por el cual se está creando el
 incremento.

\subsubsection{Reunión diaria}

 La reunión diaria o {\it Daily Scrum} se realiza día a día durante la ejecución de cada iteración
 y debe durar como máximo 15 minutos. Esta reunión tiene el propósito de optimizar la colaboración
 y el aprovechamiento haciendo una inspección del trabajo realizado desde la anterior reunión diaria
 e indicando que acciones se realizarán durante el día.\\

 En la reunión cada miembro debe responder las siguientes preguntas:

    \begin{quote}
    \begin{itemize}
    \item ¿Qué hice ayer para ayudar al equipo de desarrollo a alcanzar el {\it Sprint Goal}?
    \item ¿Qué haré hoy para ayudar al equipo a lograr el {\it Sprint Goal}?
    \item ¿Véo algún obstáculo que me impida (o al equipo de desarrollo) lograr el objetivo
           del sprint?
    \end{itemize}
    \end{quote}

 \noindent En la reunión diaria participan únicamente los miembros del equipo de desarrollo. En
 caso de que otras personas estén presentes el maestro scrum debe asegurar la fluidez de la reunión.
 A menudo los miembros del equipo suelen reunirse al termino de la reunión para discutir, adaptar
 o replantear aspectos mencionados en la reunión.

\subsubsection{Revisión}

 Esta reunión se realiza al final de cada iteración para revisar el incremento,
 discutir los inconvenientes encontrados y en dado caso adaptar la planeación del
 proyecto. El equipo {\it scrum} y los stakeholders deben revisar el incremento
 y establecer cuáles serán los cambios a implementar con la finalidad de optimizar
 el valor de los siguientes incrementos.\\

 \noindent La reunión consiste en presentar el incremento realizado y obtener retroalimentación de
 todos los involucrados en el proyecto, para {\it Sprints} de 1 mes la reunión debe durar cómo
 máximo cuatro horas. A continuación se detallan los puntos que se deben cumplir en esta reunión.

    \begin{quote}
    \begin{itemize}
    \item El equipo de desarrollo presenta el incremento, los problemas encontrados y las
            soluciones tomadas.

    \item El dueño del producto confirma cuales elementos del {\it Product Backlog} han sido
            realizados.

    \item El equipo {\it Scrum} comenta las cosas a realizar en el siguiente incremento.

    \item Se revisan los tiempos de entrega y presupuestos para las entregas posteriores.
    \end{itemize}
    \end{quote}


\subsubsection{Retroalimentación}

 La etapa de retroalimentación consiste en una reunión del equipo {\it Scrum} con el objetivo de
 crear un plan para las mejoras en la forma de trabajo, esta reunión ocurre después de la revisión
 y antes de la planeación de la siguiente iteración. En esta reunión los miembros del equipo
 {\it Scrum} ven cómo atender las debilidades y áreas de oportunidad encontradas.\\

 \noindent La duración de esta reunión depende directamente de la duración 
 establecida para las iteraciones del proyecto, para {\it Sprints} de un mes esta 
 reunión debe durar como máximo tres horas. Los objetivos principales de la reunión 
 son:

    \begin{quote}
    \begin{itemize}
    \item Inspeccionar el último {\it Sprint} en relación a las personas, relaciones,
            procesos y herramientas.

    \item Identificar y ordenar las cosas que ocurrieron bien durante el Sprint y las
            cosas que hay que mejorar.

    \item Crear un plan para la aplicación de las mejoras para tener una mejor implementación
            de Scrum.
    \end{itemize}
    \end{quote}

\subsection{Artefactos}

 Los artefactos en Scrum proporcionan transparencia en toda la aplicación de {\it Scrum}
 y además funcionan como herramientas para la inspección de la implementación del
 marco de trabajo y adaptación del mismo. A continuación se presentan los artefactos
 definidos por el macro de trabajo.

 %Tener los artefactos organizados brinda
 % una mayor visibilidad acerca del avance del proyecto y del producto final.

\subsubsection{Product Backlog}

 % La cartera de producto o Product Backlog
 El {\it Product Backlog} lista todas las características, funcionalidades, 
 requerimientos y mejoras necesarias para la creación del producto. El responsable 
 del contenido, disponibilidad y organización del {\it Product Backlog} es el dueño
 del producto.\\

 \noindent El {\it Product Backlog} es dinámico, es decir, puede cambiar 
 constantemente quitando o añadiendo requerimientos del producto que deben ser 
 cumplidos, cada uno de los requerimientos está vinculado con uno o más elementos 
 del {\it Product Backlog}, dichos elementos debe contener {\bf descripción, orden,
 estimación y valor}; opcionalmente se suele incluir una descripción acerca de como
 probar que un elemento haya sido completado del {\it Product Backlog}.

\subsubsection{Sprint Backlog}

 El {\it Sprint Backlog} está formado por los elementos del {\it Product Backlog} seleccionados
 para cada iteración, incluyendo el plan para entregarlos y así cumplir con el objetivo planteado.
 En otras palabras, el {\it Sprint Backlog} es una estimación de las funcionalidades que serán
 entregadas en el siguiente incremento del producto.\\

 \noindent Los elementos del {\it Product Backlog} seleccionados para formar parte del {\it Sprint
 Backlog} pueden ser redefinidos con la finalidad de establecer el alcance durante el {\it Sprint}
 de dicho elemento. El {\it Sprint Backlog} hace visible el trabajo necesario para cumplir el
 objetivo del sprint.

\clearpage
