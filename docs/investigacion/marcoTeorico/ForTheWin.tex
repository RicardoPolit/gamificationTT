\subsection{For The Win} \label{sec:ForTheWin}

 {\em For The Win} es un marco de trabajo realizado por Dan Hunter y Kevin Werbach, de acuerdo
 con los actores una de las motivaciones por las cuales fue creado este marco de trabajo es que
 había una necesidad real de una guía pragmática basada en la investigación que explicara cómo
 implementar gamificación de forma correcta \cite{ForTheWin}.\\

 \noindent {\em For The Win} describe el pensamiento de diseñador de juegos, establece una
 jerarquía entre los distintos elementos de juego en la gamificación (ver figura \ref{fig:ForTheWin}),
 y presenta una serie de pasos para su implementación, siendo estas características las razones
 principales por las cuales fue elegido este marco de trabajo.

    \addfigure[(adaptado de {\it For The Win} \cite{ForTheWin})]%
        {.4}{investigacion/images/forthewin}{fig:ForTheWin}%
        {Jerarquía de elementos de juego segun For The Win}

 \noindent El pensamiento de diseñador de juegos consiste en usar todos los recursos para
 crear una experiencia atractiva capaz de motivar comportamientos deseados. El hecho de
 aplicar gamificación para resolver problemas ajenos a los juegos pone a la persona que
 diseñará los elementos de gamificación en el rol de un diseñador de juegos \cite[p. 29]{ForTheWin}.


\subsubsection{El rol del diseñador de juegos}

 \noindent A diferencia de los jugadores cuyo objetivo es ganar, el diseñador de juegos
 tiene el objetivo de hacer que los jugadores permanezcan jugando. El diseñador de juegos
 define los objetivos del juego para que estos proporcionen los beneficios hacia el negocio
 \cite[p. 29]{ForTheWin}.\\

 \noindent La persona que esté ocupando el rol del diseñador de juegos debe saber que la
 gamificación no es una solución absoluta para los problemas del negocio, la gamificación
 funciona apropiadamente en contextos que son o pueden ser atractivos y que estén vinculados
 con los objetivos del negocio \cite[p. 30]{ForTheWin}. Para saber si la gamificación encaja
 en un contexto en particular {\em For The Win} define los siguientes puntos a analizar:

    \begin{itemize}
    \item
    {\bf Motivación.}
        Existen principalmente tres tipos de actividades en las cuales la motivación
        es particularmente importante: el {\em trabajo creativo}, en el cual cuando la gente
        está comprometida producen un mejor resultado; {\em tareas no atractivas}, donde las
        personas deben cumplir tareas que son repetitivas o aburridas; y la creación de
        {\em hábitos}, en los cuales las personas saber que deben realizar tareas de formas
        constante durante un tiempo significativo \cite[p. 31]{ForTheWin}.
        \clearpage

    \item
    {\bf Elecciones significativas.}
        Hacen referencia a que se le pueda proporcionar a los jugadores un sentido de libertad
        y autonomía brindando distintas opciones a elegir, las cuales impliquen consecuencias
        significativas. Si un sistema gamificado ofrece recompensas sin brindar al usuario
        distintas opciones, rápidamente se volverá aburrido para la mayoría de los jugadores
        \cite[p. 32]{ForTheWin}.

    \item {\bf Estructura}
        La gamificación requiere de algoritmos para medir y responder a las acciones de los
        usuarios. Además, debe ser fácil registrar o rastrear las actividades, para que los
        datos relevantes que estos se administren dentro del sistema. Si el contexto no permite
        organizar de forma estructurada las acciones de los usuarios, difícilmente será un
        entorno donde la gamificación pueda brindar frutos \cite[p. 32]{ForTheWin}.

    \item {\bf Conflictos potenciales}
        El diseñador de juegos debe analizar las distintas mecánicas de juego que utilizará
        ya que estas pueden desmotivar a los jugadores o ser incongruentes con la percepción
        que se tiene del negocio. Es importante identificar las formas en que se quiere motivar
        a la población objetivo y pensar cómo funcionarían junto con la gamificación
        \cite[p. 33]{ForTheWin}.

    \end{itemize}


    \noindent En la mayoría de los casos los entornos en los que se desea implementar la gamificación
    no encajan totalmente en los puntos mencionados, es importante que la persona bajo el rol del
    diseñador de juegos intente resolver cada punto de forma positiva y analizar como cada punto
    afecta a los demás \cite[p. 34]{ForTheWin}.



\subsubsection{Elementos de juego}

 \noindent Como se mencionó anteriormente, {\em For The Win} establece una jerarquía entre los
 elementos de juego. De acuerdo con el marco de trabajo, para implementar gamificación se
 necesitan contemplar distintos elementos de juego, estos elementos han sido clasificados por
 el marco de trabajo en trés grupos: dinámicas, mecánicas y componentes \cite[pp. 55-57]{ForTheWin}.\\

 \noindent Los tipos de elementos  están organizados de forma decreciente de acuerdo con su nivel
 de abstracción como se muestra en la figura \ref{fig:ForTheWin}, de tal forma que cada mecánica
 está ligada a una o más dinámicas y cada componente esta ligado a uno o más mecánicas y
 dinámicas.\\


 \noindent {\bf Dinámicas}
 \begin{quote}
    Las dinámicas son el elemento de juego más abstracto, estas deben de ser consideradas y
    manejadas, sin embargo estas no pueden ser introducidas directamente en los juegos debido
    a su nivel de abstracción, por ejemplo, el desarrollo de los empleados, la creación de una
    cultura de innovación, entre otras. Las cinco dinámicas más importantes de acuerdo con el
    marco de trabajo son:

    \begin{enumerate}
        \item Restricciones, limitaciones y reglas
        \item Emociones de los jugadores (curiosidad, competitividad, frustración, etc.)
        \item Historia (la narrativa del juego o sistema)
        \item Progreso (desarrollo y crecimiento de los jugadores)
        \item Interacciones sociales (compañerismo, altruismo, etc.)
    \end{enumerate}

 \end{quote}

 \clearpage

 \noindent {\bf Mecánicas}
 \begin{quote}
    Las mecánicas son el motivo para que se realice alguna acción, estas mantienen
    motivado y comprometido al jugador. Cada mecánica puede ayudar a que se logren una o
    más dinámicas. Las diez mecánicas consideradas más relevantes en el marco de trabajo
    son:

    %\begin{multicols}{2}
    \begin{enumerate}
        \item Desafíos (retos o actividades que requieren esfuerzo para resolverse)
        \item Suerte (recompensas y elementos aleatorios)
        \item Competencias (los jugadores se esfuerzan por ser los ganadores)
        \item Cooperación (los jugadores deben trabajar en conjunto para cumplir el objetivo)
        \item Retroalimentación (información acerca de las acciones del jugador)
        \item Obtención de recursos (adquisición de utilerías o coleccionables)
        \item Recompensas (beneficios de realizar una acción u obtener un logro)
        \item Transacciones (intercambio entre los jugadores o mediante mediante intermediarios)
        \item Turnos (participación alternante y secuencial de los jugadores)
        \item Victorias (objetivos que hacen ganador a un jugador o equipo)\\
    \end{enumerate}
    %\end{multicols}
 \end{quote}


 \noindent {\bf Componentes}
 \begin{quote}
    Los componentes son la base para la implementación de distintas mecánicas y las dinámicas,
    los componentes son los elementos de juego más tangibles. Los quince componentes más importantes
    de acuerdo con el marco de trabajo son:

    %\begin{multicols}{3}
    \begin{enumerate}
        \item Logros (objetivos definidos)
        \item Avatares (representación visual del carácter del jugador)
        \item Insignias (representación visual de logros)
        \item Peleas de jefes finales (desafíos difíciles al término de un nivel)
        \item Colecciones (conjunto de elementos acumulables)
        \item Combates (batalla efímera)
        \item Desbloqueables (artículos condicionados)
        \item Regalos e intercambios (oportunidad de compartir recursos con otros)
        \item Tablas de líderes (representación visual de progreso de los jugadores)
        \item Niveles de experiencia (representación la cantidad de actividades realizadas)
        \item Puntos (representación numérica de progreso de un jugador)
        \item Misiones (retos predefinidos con objetivos y recompensas)
        \item Vínculos sociales (representación gráfica de los vínculos sociales en el juego)
        \item Equipos (grupos definidos de jugadores trabajando para un objetivo común)
        \item Bienes virtuales (activos en el juego con un valor monetario virtual o real)
    \end{enumerate}
    %\end{multicols}
 \end{quote}


\subsubsection{Proceso de implementación}

 {\it For the Win} propone un marco de trabajo el cual contiene un conjunto de pasos para diseñar
 sistemas gamificados, combinando conceptos como la diversión, experiencias de usuario y jugabilidad,
 junto la ingeniería de sistemas para cumplir objetivos específicos del negocio
 \cite[p. 70]{ForTheWin}. A continuación se describe cada uno de los pasos del marco de trabajo.\\

    \noindent {\bf 1.- Define los objetivos del negocio}

    \begin{quote}
        Para una implementación efectiva de la gamificación es crítico entender de los
        objetivos por los cuales se implementará la gamificación cómo ``incrementar la
        permanencia de los clientes'', ``generar lealtad a la empresa'', etcétera. Los
        objetivos deben estar ordenados por prioridad y para cada uno se debe especificar
        cómo beneficiarían a la organización \cite[p. 62]{ForTheWin}.
    \end{quote}


    \noindent {\bf 2.- Delimita las acciones de tus usuarios}

    \begin{quote}
        Este paso consiste en definir el comportamiento que tendrán los jugadores y las
        métricas a utilizar. Los comportamientos deben ser concretos y específicos,
        debido a que eventualmente se desarrollarán métricas para traducir los comportamientos
        en resultados cuantificables, dichas métricas deben definir cuando y cómo se obtiene
        los estados de victoria accesibles a todos los usuarios \cite[pp. 63-64]{ForTheWin}.
    \end{quote}


    \noindent {\bf 3.- Describe a tus usuarios}

    \begin{quote}
        Es importante identificar los distintos tipos de usuarios que usarán el sistema
        gamificado y respecto al tipo de usuario listar las distintas acciones que los
        motiva y desmotiva. Finalmente se debe brindar soporte a cada una de las etapas
        durante el ciclo de vida del jugador, desde ser novato a convertirse en experto
        \cite[pp. 64-65]{ForTheWin}.
        % una clasificación util es dividirlos en triunfadores, exploradores
        % socializadores y asesinos \cite{BartleUsers},
    \end{quote}


    \noindent {\bf 4.- Define ciclos de actividades}

    \begin{quote}
        El conocer a los usuarios y objetivos permite diseñar las actividades que tendrá
        el sistema y cómo será el flujo de ellas. Es necesario contemplar los ciclos puesto
        que es común que los usuarios repitan ciertas acciones para subir de nivel o para
        lograr un meta en el sistema \cite[p. 66]{ForTheWin}.
    \end{quote}


    \noindent {\bf 5.- Piensa en la diversión}

    \begin{quote}
        Uno de los principales requerimientos de los sistemas con gamificación es que brinden
        algún tipo de diversión a los usuarios. A pesar de que la diversión es subjetiva, el
        marco de trabajo plantea que para llevar a cabo este punto de forma concreta se
        debe revisar que el diseño del sistema contemple elementos que apoyen tanto a la
        motivación extrínseca de los jugadores como a la intrínseca \cite[p. 68]{ForTheWin}.
    \end{quote}


    \noindent {\bf 6.- Utiliza las herramientas adecuadas para el trabajo}
    \begin{quote}
        Este paso se centra en la etapa de implementación, es requerido escoger las herramientas
        apropiadas para implementar las dinámicas y mecánicas diseñadas, para esto se tendrán
        que seleccionar qué incluir o excluir, ver que es lo que no funciona y cambiarlo,
        finalmente esta etapa consiste en hacer pruebas, iterar, aprender y ejecutar los cambios
        necesarios para mejorar la implementación de la gamificación \cite[p. 69]{ForTheWin}.
    \end{quote}
