
\chapter{Módulo de Experiencia}
\label{mod:experiencia}

\section{Análisis}

Este módulo contiene la especificación de %experiencia que indica;
cómo se obtienen los puntos de experiencia, la cantidad a otorgar, el número requerido para alcanzar cada nivel en la plataforma, la forma en que el usuario puede visualizar su nivel y la barra de progreso del nivel actual del alumno.

\subsection{Esquema de Experiencia}
    
Es la especificación de los conceptos relacionados con los puntos de experiencia, cuales son los tipos de incremento y cómo se usan y cuales restricciones se aplican para la implementación de los puntos de experiencia, niveles y la acción subir de nivel. % y cuántos niveles hay.\\

\subsubsection{Conceptos de puntos de experiencia}
\noindent Como se especificó en el marco teórico, los puntos de experiencia son una unidad que representa la cantidad de actividades completadas por un usuario, sin embargo, se puede referir en diversos contextos a estos puntos. Es por ello que se utilizarán conceptos definidos por los cuadros  \hyperref[table:METerminosExperiencia1]{6.1} y  \hyperref[table:METerminosExperiencia2]{6.2} .\\

\noindent Para poder explicar mejor los conceptos que se tienen para los puntos de experiencia, se utilizará un ejemplo y cada concepto en los cuadros  \hyperref[table:METerminosExperiencia1]{6.1} y  \hyperref[table:METerminosExperiencia2]{6.2} , tendrá su relación con este ejemplo.\\

\noindent Se tiene un usuario ''\textit{U}'', y dicho usuario está actualmente en el nivel ''5'' con ''1000'' puntos de experiencia, y además los puntos de experiencia relacionados con los niveles son los siguientes:
\begin{itemize}
    \item Estando en el nivel 1 se necesitan 1000 puntos de experiencia para subir al nivel 2.
    \item Estando en el nivel 2 se necesitan 1250 puntos de experiencia para subir al nivel 3.
    \item Estando en el nivel 3 se necesitan 1500 puntos de experiencia para subir al nivel 4.
    \item Estando en el nivel 4 se necesitan 1750 puntos de experiencia para subir al nivel 5.
    \item Estando en el nivel 5 se necesitan 2000 puntos de experiencia para subir al nivel 6.
\end{itemize}

\noindent Agreguemos que existe una actividad ''\textit{A}'' que al ser completada otorga 1050 puntos de experiencia.\\

\noindent Con el ejemplo anterior, podemos proceder a definir nuestros conceptos y relacionarlos con el ejemplo para que quede todo más claro.

\begin{table}[h!]
    \label{table:METerminosExperiencia1}
    \centering   
        \begin{tabular}{|m{0.2 \textwidth}|m{0.32 \textwidth}|m { 0.42\textwidth}|}\hline
        \textbf{Nombre} & \textbf{Definición} & \textbf{Ejemplo} \\\hline
        
        Experiencia actual  & 
        Es la cantidad de puntos de experiencia que un usuario ha conseguido mientras tiene asociado un cierto nivel. &
        ''\textit{U}'' tiene una \textbf{experiencial actual} de 1000 puntos de experiencia. Y el nivel al cual está asociado es el nivel 5.
        \\\hline
        
        Experiencia del nivel &
        Es la cantidad de puntos de experiencia que se requieren para subir de un nivel ''\textit{A}'' a un nivel ''\textit{B}'', sin contemplar la \textbf{experiencia actual} del usuario.&
        Con nuestro ejemplo la \textbf{experiencia del nivel} 1, es 1000 mientras que la del nivel 5 es 2000. \\\hline
        
        Porcentaje actual &
        Es un número entero positivo con rango del 1 al 100, que se calcula de la forma $ \frac{experiencia\_actual}{experiencia\_del\_nivel} * 100 $ &
        ''\textit{U}'' actualmente está en el nivel 5, por lo tanto el cálculo sería: $ \frac{1000}{2000} * 100 = 50 $
        \\\hline
        
        Experiencia otorgada &
        Es la cantidad de puntos de experiencia que se otorgan al completar una actividad. &
        La actividad ''\textit{A}'' tiene una \textbf{experiencia otorgada} de 1050.
        \\\hline
        
        Experiencia acumulada &
        Es la cantidad de puntos de experiencia que un usuario ha conseguido a través de los niveles. &
        ''\textit{U}'' está actualmente en el nivel 5, esto quiere decir que ha pasado por los niveles 1, 2, 3 y 4. Cada uno de estos últimos tiene su \textbf{experiencia del nivel}, por lo tanto, nuestro usuario ''\textit{U}'' a conseguido $ 1000 + 1250 + 1500 + 1750 $ puntos para llegar al nivel 5. Además, ''\textit{U}'' tiene una \textbf{experiencia actual} de 1000 puntos.\newline
        
        Sumando todo lo anterior $ 1000 + 1250 + 1500 + 1750 + 1000  = 6500 $, por lo tanto el usuario ''\textit{U}'' tiene una \textbf{experiencia acumulada} de 6500 \\\hline
        
        \end{tabular}
    \caption{Conceptos referentes a los puntos de experiencia (parte 1).}
\end{table}
\clearpage 
\begin{table}[h!]
    \label{table:METerminosExperiencia2}
    \centering   
        \begin{tabular}{|m{0.2 \textwidth}|m{0.32 \textwidth}|m { 0.42\textwidth}|}\hline
        \textbf{Nombre} & \textbf{Definición} & \textbf{Ejemplo} \\\hline
        
        Experiencia necesaria &
        Es la cantidad de puntos de experiencia que un usuario necesita para que su \textbf{experiencia actual} alcance la \textbf{experiencia del nivel}. Expresado de forma matemática: 
        \begin{center}
            experiencia del nivel\newline 
            \underline{ - experiencia actual} \newline 
            experiencia necesaria \newline 
        \end{center}& 
       
        Para  ''\textit{U}'' su \textbf{experiencia necesaria} sería.
        
        \begin{center}
              2000\newline 
            \underline{ - 1000} \newline 
            1000\newline
        \end{center}
        \\\hline
        
        Experiencia sobrante &
        Es la cantidad de puntos de experiencia que rebasan la \textbf{experiencia del nivel} cuando a un usuario recibe \textbf{experiencia otorgada}. Expresado de forma matemática:  
        \begin{center}
            experiencia otorgada\newline 
            + experiencia actual\newline 
            \underline{ - experiencia del nivel} \newline 
            experiencia sobrante \newline 
        \end{center}& 
        Si el alumno ''U'' realiza la actividad ''A'', recibiría una \textbf{experiencia otorgada} de 1050 , sin embargo, su \textbf{experiencia actual} es de 1000. Por lo tanto, tendría una \textbf{experiencia sobrante} de 50 puntos de experiencia.
         
        \begin{center}
            1050\newline 
            + 1000\newline 
            \underline{ - 2000} \newline 
            50 \newline 
        \end{center}\\\hline
        \end{tabular}
    \caption{Conceptos referentes a los puntos de experiencia (parte 2).}
\end{table}

\noindent A lo largo de este capítulo se utilizaran los conceptos anteriores.\\

\subsubsection{Tipos de incremento}

    Un tipo de incremento define cómo se calcula la diferencia entre los valores de ''\hyperref[table:METerminosExperiencia1]{experiencia del nivel}'' de un nivel \textit{$n_i$} y el siguiente a él \textit{$n_(i+1)$}.\\
    
    \noindent Se da soporte a los siguientes tipos de incremento entre los niveles:
    
    
    \begin{quote}
    \begin{description}
        \item[Lineal] Establece que la diferencia es una cantidad fija.\\
        Sea $f(n_i)$ una función la cual indica la ''\hyperref[table:METerminosExperiencia1]{experiencia del nivel}'' de un nivel $n_i$. Y sea $e$ una constante que representa la diferencia de la ''\hyperref[table:METerminosExperiencia1]{experiencia del nivel}'' entre 2 niveles continuos. Entonces.
            $$\forall n_i \in Nivel\ | \left(\ f(n_{i+1}) - f(n_i)\ \right) = e$$
        
        \item[Porcentual] Establece que la diferencia está regida por la siguiente regla:\\
        Sea $n_i$ un nivel de experiencia,  $f(n_i)$ una función la cual indica la ''\hyperref[table:METerminosExperiencia1]{experiencia del nivel}'' de un nivel $n_i$ y  $c$ una constante tal que $1 \leq c \leq 2$ , entonces
            $$\forall n_i \in Nivel\ |\ (c)f(n_i) = f(n_{i+1}).$$
    \end{description}
    \end{quote}
    
    
\begin{comment}
    \begin{quote}
    \begin{description}
        \item[Lineal] Establece que la diferencia es una cantidad fija.\\
        Sea $f(n_i)$ una función la cual indica la cantidad de experiencia requerida para subir al nivel $n_i$. y sea $e$ una constante que representa el incremento de la cantidad de experiencia requerida entre niveles. Entonces.
            $$\forall n_i \in Nivel\ | \left(\ f(n_{i+1}) - f(n_i)\ \right) = e$$
        
        \item[Porcentual] Establece que la diferencia entre cantidad necesaria de experiencia para subir del nivel $i$ al nivel $i+1$ está regido por la siguiente regla:\\
        Sea $n_i$ un nivel de experiencia y $n_{i+1}$ el siguiente nivel, y sea $c$ una constante tal que $1 \leq c \leq 2$ , entonces
            $$\forall n_i \in Nivel\ |\ (c)f(n_i) = f(n_{i+1}).$$
    \end{description}
    \end{quote}

\end{comment}    

%\noindent El esquema de experiencia también contiene una lista de todas las restricciones que deben de existir para que una acción en particular entregue experiencia.

\subsubsection{Esquema configurable}

    Se quiere que el administrador de la página pueda configurar:
    \begin{quote}
    \begin{itemize}
        \item{La ''\hyperref[table:METerminosExperiencia1]{experiencia del nivel}'' del nivel 1.}
        \item {El tipo de incremento.}
        \item {La cantidad de los puntos de experiencia en el incremento.}
        \item {La ''\hyperref[table:METerminosExperiencia1]{experiencia otorgada}'' que da resolver cualquier actividad.}
    \end{itemize}
    \end{quote}

\subsection{Submódulo de Niveles}

Presenta a los alumnos su progreso utilizando un sistema de niveles que se van alcanzado obteniendo puntos de experiencia. Al alcanzar un nuevo nivel la barra que muestra la cantidad de experiencia del nivel se actualizará.
% y cada vez que se alcanza un nivel, los puntos de experiencia se regresan a cero.

\begin{quote}
\begin{description}
    \item[Objetivo] \hfill\\
        Mostrar a los alumnos el nivel actual de experiencia que tienen y el avance que tienen de ese mismo nivel.
        %Mostrar el nivel actual que tienen los alumnos, así como el avance que tienen en ese mismo nivel.
        
        %Proveer información al alumno que indique la cantidad de tiempo y esfuerzo que le ha dedicado a la plataforma.
        
    \item[Principios a los que brinda soporte:] \hfill
    \begin{itemize}
        \item 2 \principioII
        \item 6 \principioVI
    \end{itemize}
\end{description}
\end{quote}

\begin{comment}

\subsection{Submódulo de Barra de Progreso}

Muestra al alumno el progreso que lleva en un curso usando un valor de 0\% a 100\% dependiendo de los ejercicios que haya hecho del curso o del tiempo que haya transcurrido.
    
    \begin{quote}
    \begin{description}
        \item[Objetivo] \hfill\\
            Proveer información al alumno que indique el tiempo y esfuerzo que le ha dedicado a un curso, así como el que le falta por dedicar.
            
        \item[Principios a los que brinda soporte:] \hfill
        \begin{itemize}
            \item 2 \principioII
        \end{itemize}
    \end{description}
    \end{quote}
\end{comment}
    
\subsection{Comportamiento en Moodle}

%\subsubsection{Plugin}
%\subsubsection{Opciones para el administrador}

\clearpage
%\subsection{Reglas de Uso} % Reglas de Negocio | But there's no business
\subsection{Casos de Uso}
    
    %\addfigure{0.8}{CU/CU_E}{fig:CUEDiagram}{Diagrama de casos de uso del Módulo de Experiencia} \clearpage
    
    En la figura \ref{fig:expUsecase} se detalla el diagrama de casos de uso correspondientes al módulo de experiencia. Los casos de uso de moodle (en color blanco) son modelados como casos de uso abstractos, mientras que los casos de uso relevantes para el módulo de experiencia son diferenciados por el color azul.\\
    
    \addfigure{1}{diagrams/UseCases}{fig:expUsecase}{Diagrama de casos de uso}
    
    \noindent Debido a que los plugins a desarrollar son elementos opcionales para moodle, solo se puede acceder a los casos de uso del módulo de experiencia (en color azul) a través de puntos de extensión de los casos de uso de moodle (en color blanco).\\
    
    \noindent Solo son documentados los casos de uso del módulo de experiencia porque son propiamente las funcionalidades o características que se desarrollaran. Los pasos de los casos de uso de moodle, guías e instructivos pueden ser consultados en la página oficial de moodle.
    \clearpage
    
    \begin{UseCase}{CU-E1}{Configurar esquema de experiencia}{
El administrador desea configurar el esquema de experiencia, entra a la interfaz  \hyperref[IUE03]{IU-E03 Configuración del esquema de experiencia}, modifica lo que desea, el sistema valida las entradas y las guarda.
}
	\UCrow{Versión}{\color{gray} 0.2 (Revisado)}
    \UCrow{Autor}{\color{gray}	David Flores Casanova}
    \UCrow{Supervisa}{\color{gray}	Ricardo Naranjo Polit}
    \UCrow{Actor}{Administrador} % Gerente, Instructor
    \UCrow{Propósito}{Modificar esquema de experiencia.}
    \UCrow{Entradas}{
        Selección en el componente ''Gamedle Level'' de la interfaz \hyperref[IUM04]{IU-M04 Sección de plugins}.\newline
        El actor puede modificar cualquiera y cuantas quiera de las opciones de la interfaz \hyperref[IUE03]{IU-E03 Configuración del esquema de experiencia}, dichas opciones se enlistan a continuación:\par\vspace{1.5em}
		\begin{Titemize}
		    \Titem{Selección en el parámetro ''Componente activado''.}
	        \Titem{Selección en el tipo de incremento.}
	        \Titem{Selección en el color del nivel.}
	        \Titem{Selección en el color de la barra de progreso.}
            \Titem{La cantidad del incremento (Número real $ \geq $ 0).}
            \Titem{La '' \hyperref[table:METerminosExperiencia1]{experiencia del nivel} '' del nivel 1 (Número entero $ > $ 0).}
            \Titem{La '' \hyperref[table:METerminosExperiencia1]{experiencia otorgada} '' de todas las actividades (Número entero $ > $ 0).}
	        \Titem{Nombre del nivel (Cadena de caracteres, longitud $\leq$ 60)}
	        \Titem{Mensaje de felicitaciones del nivel (Cadena de caracteres, longitud $\leq$ 50)}
	        \Titem{Imagen del nivel (Imagen, formato '.png')}
	        \Titem{Descripción del nivel (Cadena de caracteres, longitud $\leq$ 200)}
		\end{Titemize}
        
   	}
    \UCrow{Origen}{Ratón y teclado de la computadora}
	\UCrow{Salidas}{ 
	    El sistema carga las configuraciones actuales o las asignadas por defecto de las siguientes opciones:
		\begin{CUTitemize}
		    \CUTitem{Selección en el parámetro ''Componente activado''.}
	        \CUTitem{Selección en el tipo de incremento.}
	        \CUTitem{Selección en el color del nivel.}
	        \CUTitem{Selección en el color de la barra de progreso.}
            \CUTitem{La cantidad del incremento (Número real  $ > $ 0).}
            \CUTitem{La ''\hyperref[table:METerminosExperiencia1]{experiencia del nivel}'' del nivel 1 (Número entero  $ > $ 0).}
            \CUTitem{La '' \hyperref[table:METerminosExperiencia1]{experiencia otorgada} '' de todas las actividades (Número entero $ > $ 0).}
	        \CUTitem{Nombre del nivel (Cadena de caracteres, longitud $\leq$ 60)}
	        \CUTitem{Mensaje de felicitaciones del nivel (Cadena de caracteres, longitud $\leq$ 50)}
	        \CUTitem{Imagen del nivel (Imagen, formato '.png')}
	        \CUTitem{Descripción del nivel (Cadena de caracteres, longitud $\leq$ 200)}
		\end{CUTitemize}
		Mensajes de error:
		\begin{CUTitemize}
		    \CUTitem{MS1: Algunos ajustes no se han cambiado debido a un error.}
		    \CUTitem{MS2: Este valor no es válido.}
		\end{CUTitemize}
		
    }
    \UCrow{Destino}{Pantalla}
    \UCrow{Precondiciones}{
		\begin{CUTitemize}
	        \CUTitem{Tener instalado el componente ''Gamedle level''.}
            \CUTitem{Que el actor esté ingresado en una cuenta de administrador.}
		\end{CUTitemize}
    }
    \UCrow{Postcondiciones}{
		\begin{CUTitemize}
			\CUTitem{Se actualiza la configuración global del componente ''Gamedle Level''.}
		\end{CUTitemize}
    }
	\UCrow{Errores}{E1: Formato de entradas no válido.}
    \UCrow{Observaciones}{Las opciones al  tienen un valor por defecto, es por eso que no es necesario que el actor modifique ninguna de las entradas respecto a este punto, debido a que las configuraciones por defecto se cargan en la pantalla. }
\end{UseCase}

%\textbullet{Trayectorias}

\begin{UCtrayectoria}{Principal}
    \actor se encuentra en la interfaz \hyperref[IUM04]{IU-M04 Sección de plugins}.
    \actor Busca la sección de ''Bloques'' .
    \actor selecciona el componente ''Gamedle level''.
    \sistema carga la interfaz \hyperref[IUE03]{IU-E03 Configuración del esquema de experiencia}.
    \actor modifica las opciones que desea ({\it Trayectoria alternativa A}).
    \actor le da al botón \#1 ''Guardar cambios''  ({\it Trayectoria alternativa B}).
    \item[- -] - - {\em El caso de uso termina.}
    
\end{UCtrayectoria}

\begin{UCtrayectoria}{alternativa A}
    \item[- -] - - {\em El actor no quiere cambiar ninguna configuración.}
    \actor selecciona alguna opción en el menú de Moodle o en el directorio de la página
    \sistema carga la interfaz correspondiente a la selección del actor.
    \item[- -] - - {\em El caso de uso termina.}
\end{UCtrayectoria}


\begin{UCtrayectoria}{alternativa B}
    \item[- -] - - {\em Alguna de las entradas no es válida.}
    \sistema despliega el mensaje MS1.
    \sistema despliega el mensaje MS2 en cada una de las entradas que ha encontrado como inválidas.
    
    \item[- -] - - {\em se continúa en el paso \#5 de la trayectoria principal.}
\end{UCtrayectoria}


\vfill\clearpage\clearpage % Configurar esquema de experiencia
    
\begin{UseCase}{CU-E2}{Crear curso con experiencia}{%
El profesor desea crear un nuevo curso en moodle con soporte para brindar puntos de experiencia a los alumnos, partiendo de la interfaz \IUref{moodle:nuevoCurso} llena los campos del curso, escoge el formato {\it Gamedle} y habilita la opción de experiencia, finalmente presiona el botón para crear el curso.}

	\UCrow{Versión}{\color{gray} 0.1 (Edición)}
    \UCrow{Autor}{\color{gray}	Daniel Ortega}
    \UCrow{Supervisa}{\color{gray}}
    \UCrow{Actor}{Profesor}
    \UCrow{Propósito}{Que el profesor pueda crear un curso que tenga soporte para brindar puntos de experiencia en las distintas secciones del curso.}
    \UCrow{Entradas}{
		\begin{Titemize}
		    \Titem{ Nombre completo y nombre corto del curso }
		    \Titem{ Datos generales y de descripción del curso }
		    \Titem{ Elección del formato del curso }
		    \Titem{ Numero de secciones }
		    \Titem{ Visibilidad de secciones ocultas }
		    \Titem{ Aspecto del curso }
		    \Titem{ Casilla de experiencia }
		    \Titem{ Conjunto de datos restantes }
		    \Titem{ Botón de confirmación \textit{Guardar y regresar} o \textit{Guardar cambios y mostrar} }
		\end{Titemize}
   	}
    \UCrow{Origen}{ Ratón para las acciones y elecciones, teclado para los campos de texto}
	\UCrow{Salidas}{ \IUref{moodle:} o \IUref{moodle:} } %\begin{Titemize}\Titem{Ninguna}\end{Titemize}
    \UCrow{Destino}{ Pantalla }
    \UCrow{Precondiciones}{
		\begin{Titemize}
	        \Titem{ Contar con los permisos necesarios para crear cursos }
	        \Titem{ Tener instalado el plugin ''Gamedle Level'' }
	        \Titem{ Tener instalado el plugin ''Gamedle Format'' }
	        \Titem{ Que el actor haya elegido el formato de curso {\it Gamedle} en el caso de uso {\it Crear curso} }
		\end{Titemize}
    }
    \UCrow{Postcondiciones}{%
        \begin{Titemize}
            \Titem{ Se crea un curso con soporte para brindar experiencia. }
            \Titem{ Las secciones del curso tienen experiencia predeterminada }
            \Titem{ El curso y las secciones se muestran de acuerdo a las configuraciones realizadas }
        \end{Titemize}
    }
	\UCrow{Errores}{ No se encuentra la opción ''Gamedle Format'' en los formatos del curso, debido a que los plugins no han sido instalados }
    \UCrow{Observaciones}{  }
\end{UseCase}
\clearpage

%\textbullet{Trayectorias}

\begin{UCtrayectoria}{Principal}
    \moodle Muestra la interfaz \IUref{moodle:nuevoCurso}. \UCnote{CU: Crear curso}
    \actor Especifica el ''nombre completo'' y ''nombre corto'' además de la ''descripción'' y los ''datos generales'' del curso.\\
    
    \setcounter{enumi}{0}
    \actor Selecciona el {\it formato del curso} {\bf Curso Gamedle}. \IUref{exp:format} \UCnote{CU: Crear curso con experiencia}
    \sistema Carga los nuevos datos para el formulario del formato: Curso Gamedle.
    \actor Selecciona el ''número de secciones'' que tendrá por defecto el curso.
    \actor Selecciona la ''visibilidad'' de forma colapsada u no visible de las secciones ocultas.
    \actor Especifica si el ''aspecto del curso'' es mostrar una sección por página o mostrar todas.
    \actor Habilita la ''casilla de experiencia''. \UCnote{\bf Trayectoria A}
    \actor Especifica el conjunto de datos restantes para la configuración del curso.
    \actor Presiona botón \fbox{Guardar y regresar} \UCnote{\bf Trayectoria B} \UCnote{\bf Trayectoria C} %\fbox{Guardar cambios}.
    \sistema Crea el curso y las secciones del mismo.
    \sistema Obtiene del esquema de experiencia la cantidad de experiencia para los cursos.
    \sistema Divide la cantidad de experiencia del curso entre las secciones creadas.
    \sistema Guarda los valores de experiencia que le corresponden a cada sección.
    \sistema Muestra la pantalla \IUref{moodle:}
        % en caso de que la división no sea entera, la última sección tendrá la cantidad para completar
    \item[- -] - - {\em Fin del caso de uso.}
\end{UCtrayectoria}

%\begin{UCtrayectoria}[Formato distinto a Gamedle.]{Alternativa A}
    %\actor Selecciona un formato de curso distinto a ''Gamedle Format''
    %\actor Especifica el conjunto de datos restantes para la configuración del curso.
    %\actor Presiona botón \fbox{Guardar y regresar} o \fbox{Guardar cambios y mostrar}.
    %\item[- -] - - {\em Fin del caso de uso}
%\end{UCtrayectoria}

\begin{UCtrayectoria}[Formato Gamedle sin experiencia]{Alternativa A}
    \actor Deshabilita la ''casilla de experiencia''
    \actor Especifica el conjunto de datos restantes para la configuración del curso.
    \item[- -] - - {\em Fin del caso de uso}
\end{UCtrayectoria}

\begin{UCtrayectoria}[Guardar cambios y mostrar]{Alternativa B}
    \actor Presiona botón \fbox{Guardar cambios y mostrar}.
    \sistema Crea el curso y las secciones del mismo.
    \sistema Obtiene del esquema de experiencia la cantidad de experiencia para los cursos.
    \sistema Divide la cantidad de experiencia del curso entre las secciones creadas.
    \sistema Guarda los valores de experiencia que le corresponden a cada sección.
    \sistema Muestra la pantalla \IUref{moodle:}
    \item[- -] - - {\em Fin del caso de uso}
\end{UCtrayectoria}

\begin{UCtrayectoria}[Cancelar]{Alternativa C}
    \actor Presiona botón \fbox{Cancelar}.
    \item[- -] - - {\em Fin del caso de uso}
\end{UCtrayectoria}

%\UserStory{Crear curso con experiencia}{Como {\bf administrador} me gustaría que la instalación de un ... por lo que ...}

\clearpage\clearpage % Crear curso con experiencia
    \begin{UseCase}{CU-E9}{Recibir experiencia}{
Cuando un alumno conteste un ejercicio y suba un intento para revisión, el sistema le otorgará la experiencia correspondiente. Si recibe experiencia suficiente, el usuario subirá de nivel.
}
	\UCrow{Versión}{\color{gray} 0.2 (Revisado)}
    \UCrow{Autor}{\color{gray}	David Flores Casanova}
    \UCrow{Supervisa}{\color{gray}	Daniel Isaí Ortega Zúñiga}
    \UCrow{Actor}{Alumno} % Gerente, Instructor
    \UCrow{Propósito}{Otorgarle experiencia al actor por completar una actividad.}
    \UCrow{Entradas}{
        Selección en el botón \#2 ''Enviar todo y terminar'' de la interfaz \hyperref[IUM01]{IU-M01 Ver intento de examen} .\newline
        Selección en el botón \#1 ''Enviar todo y terminar'' de la interfaz \hyperref[IUM02]{IU-M02 Confirmación de envío de intento} .\newline
        Presión de la tecla ''Enter'' o la tecla ''espacio''.
   	}
    \UCrow{Origen}{Ratón y teclado de la computadora }
	\UCrow{Salidas}{
	    \begin{Titemize}
        \Titem{''\hyperref[table:METerminosExperiencia1]{Experiencia otorgada}''.}
        \Titem{''\hyperref[table:METerminosExperiencia1]{Experiencia actual}'' del actor.}
        \Titem{''\hyperref[table:METerminosExperiencia1]{Experiencia del nivel}'' del nivel actual del actor.}
        \Titem{Barra de progreso, mostrada según el \hyperref[table:METerminosExperiencia1]{porcentaje actual}.}
        \Titem{Nombre del nivel (Cadena de caracteres, longitud $\leq$ 60)}
        \Titem{Mensaje de felicitaciones del nivel (Cadena de caracteres, longitud $\leq$ 50)}
        \Titem{Imagen del nivel (Imagen, formato '.png')}
        \Titem{Descripción del nivel (Cadena de caracteres, longitud $\leq$ 200)}
	    \end{Titemize}
    }
    \UCrow{Destino}{Pantalla}
    \UCrow{Precondiciones}{
		\begin{CUTitemize}
	        \CUTitem{El actor está registrado como alumno del curso.}
            \CUTitem{El actor no tiene registrados intentos anteriores.}
			\CUTitem{La actividad del curso está creada.} 
            \CUTitem{El módulo de experiencia está habilitado.}
		\end{CUTitemize}
    }
    \UCrow{Postcondiciones}{
		\begin{CUTitemize}
	        \CUTitem{Al actor se le registra su nueva cantidad de experiencia.}
			\CUTitem{Se actualiza la cantidad de experiencia que ha recibido el actor en ese curso.}
		\end{CUTitemize}
    }
	\UCrow{Errores}{E1: El actor no está registrado como alumno del curso}
    \UCrow{Observaciones}{}
\end{UseCase}

%\textbullet{Trayectorias}

\begin{UCtrayectoria}{Principal}
    \actor se encuentra en la interfaz \hyperref[IUM01]{IU-M01 Ver intento de examen}.
    \actor selecciona el botón \#2 ''Enviar todo y terminar'' .
    \sistema muestra la ventana emergente \hyperref[IUM02]{IU-M02 Confirmación de envío de intento}.
    \actor selecciona el botón \#1 ''Enviar todo y terminar'' ({\it Trayectoria alternativa A})
    \sistema comprueba que el módulo de experiencia esté habilitado ({\it Trayectoria alternativa B}).
    \sistema comprueba que el actor que subió el intento es un alumno del curso ({\it Trayectoria alternativa C}).
    \sistema comprueba que el actor no tenga registrados intentos anteriores ({\it Trayectoria alternativa D}).
    \sistema calcula si la experiencia que se le dará al actor provoca que este ''suba de nivel'' ({\it Trayectoria alternativa E}).
    \sistema carga la interfaz \hyperref[IUM03]{IU-M03 Intento calificado}.
    \item[- -] - - {\em El caso de uso termina.}
\end{UCtrayectoria}

\begin{UCtrayectoria}{alternativa A}
    \item[- -] - - {\em El usuario presionó el botón \#2 \fbox{Cancelar} o  el botón \#3 \fbox{X}.}
    \sistema cierra la interfaz \hyperref[IUM02]{IU-M02 Confirmación de envío de intento}.
    \item[- -] - - {\em El caso de uso termina.}
\end{UCtrayectoria}

\begin{UCtrayectoria}{alternativa B}
    \item[- -] - - {\em El módulo de experiencia no está habilitado.}
    \item[- -] - - {\em El caso de uso termina.}
\end{UCtrayectoria}

\begin{UCtrayectoria}{alternativa C}
    \item[- -] - - {\em El actor no está registrado como alumno del curso.}
    \item[- -] - - {\em El caso de uso termina.}
\end{UCtrayectoria}

\begin{UCtrayectoria}{alternativa D}
    \item[- -] - - {\em El actor ya había hecho intentos anteriores.}
    \item[- -] - - {\em El caso de uso termina.}
\end{UCtrayectoria}

\begin{UCtrayectoria}{alternativa E}
    \item[- -] - - {\em La ''\hyperref[table:METerminosExperiencia1]{experiencia otorgada}'' que recibe el actor es suficiente para ''subir de nivel''.}
    %\item[- -] - - {\em Se extiende al CU-E02.}
    \sistema calcula la ''\hyperref[table:METerminosExperiencia2]{experiencia sobrante}''.
    \sistema incrementa el nivel actual del actor en una unidad.
    \sistema cambia el valor de la ''\hyperref[table:METerminosExperiencia1]{experiencia actual}'' por la de la ''\hyperref[table:METerminosExperiencia1]{experiencia sobrante}''
    \sistema guarda el nivel actual del actor y la ''\hyperref[table:METerminosExperiencia1]{experiencia actual}''.
    \sistema comprueba si el nivel actual del actor existe dentro de un rango de niveles ({\it Trayectoria alternativa F}). 
    \sistema carga la interfaz \hyperref[IUE02]{IU-E02 Subir de nivel} con la información por defecto de niveles.
    \actor presiona la tecla ''enter'' o ''espacio''.
    \sistema cierra la interfaz \hyperref[IUE02]{IU-E02 Subir de nivel}.
    \item[- -] - - {\em Se continua en el paso \#9 de la trayectoria principal.}
\end{UCtrayectoria}


\begin{UCtrayectoria}{alternativa F}
    \item[- -] - - {\em El nivel actual del actor está dentro de un rango de niveles.}
    \sistema carga la interfaz \hyperref[IUE02]{IU-E02 Subir de nivel}  con la información especificada en el rango de niveles.
    \item[- -] - - {\em Se continua en el paso \#7 de la trayectoria alternativa E.}
\end{UCtrayectoria}

\vfill\clearpage\clearpage % Recibir experiencia
    
\subsection{Interfaces}
    
\subsection*{IU-E01 Bloque de experiencia}
\label{IUE01}

    Visualización en Moodle del bloque de experiencia.

    \addfigure{1}{IU/IU_E01_Bloque_Experiencia}{fig:IUE01}{IU-E01: Bloque de experiencia.}
    
    \noindent {\bf Elementos:}
    \begin{quote}
    \begin{description}
    	\item[Nombre del bloque] Nombre para diferenciar los bloques en las interfaces de Moodle.
    	\item[Imagen del nivel] Imagen del nivel configurada por el administrador.
    	\item[Número del nivel] Número entero positivo que representa el nivel actual del usuario.
    	\item[Barra de progreso] Barra que se llena según el ''\hyperref[table:METerminosExperiencia1]{porcentaje actual}'' del usuario
    	\item[Experiencia actual del nivel] Número entero positivo que representa la ''\hyperref[table:METerminosExperiencia1]{experiencia actual}'' del usuario.
    	\item[Experiencia total del nivel] Número entero positivo que representa la ''\hyperref[table:METerminosExperiencia1]{experiencia del nivel}''.
    	\item[Experiencia acumulada] Número entero positivo que representa la cantidad de ''\hyperref[table:METerminosExperiencia1]{experiencia acumulada}''.
    \end{description}
    \end{quote}
	\clearpage
    
\subsection*{IU-E02 Subir de nivel}
\label{IUE02}

    Esta interfaz es una ventana emergente que sale siempre que el usuario tenga ''\hyperref[table:METerminosExperiencia1]{experiencia sobrante}'' al ejecutar el \hyperref[CU-E01]{CU-E01 Recibir experiencia}.    

    \addfigure{1}{IU/IU_E02_PopUp_SubirNivel}{fig:IUE02}{IU-E02: Ventana emergente que aparece al subir de nivel.}
    
    \noindent {\bf Elementos:}
    \begin{quote}
    \begin{description}
    	\item[Mensaje] Mensaje de felicitaciones configurado por el administrador.
    	\item[Imagen del nivel] Imagen del nivel configurada por el administrador.
    	\item[Número del nivel] Número entero positivo que representa el nivel actual del usuario.
    	\item[Nombre] Nombre que reciben los niveles, que es configurado por el administrador.
    	\item[Descripción] Descripción asociada a los niveles, que es configurada por el administrador.
    \end{description}
    \end{quote}
	\clearpage

\subsection*{IU-E03 Configuración del esquema de experiencia}
\label{IUE03}
    Esta interfaz es definida por el archivo \textbf{settings.php}, sin embargo, está es generada por Moodle.\\
    En esta interfaz el administrador puede modificar los aspectos visuales que tienen las interfaces \hyperref[IUE01]{IU-E01 Bloque de experiencia} y \hyperref[IUE02]{IU-E02 Subir de nivel} , y configurar el tipo de incremento, cuanto incremento hay por nivel, la ''\hyperref[table:METerminosExperiencia1]{experiencia del nivel}'' del nivel 1 y la ''\hyperref[table:METerminosExperiencia1]{experiencia otorgada}'' que darán todas las actividades.\\
    Esta interfaz también activa y desactiva el funcionamiento del módulo de experiencia.\\
    
    \addfigure{1}{IU/IU_E03_config_parte1}{fig:IUE03_1}{Interfaz donde se  configura el esquema de experiencia parte 1.}
    
    
    \noindent {\bf Elementos importantes:}
    \begin{quote}
    \begin{description}
    	\item[Opción \#1] Permite activar y desactivar el módulo de experiencia.
    	\item[Opción \#2] Permite seleccionar entre los 2 tipos de incremento 'Lineal' y 'Porcentual'.
    	\item[Opción \#3] Permite asignar cuanta experiencia habrá de diferencia entre la ''\hyperref[table:METerminosExperiencia1]{experiencia del nivel}'' del nivel \textit{A} y la ''\hyperref[table:METerminosExperiencia1]{experiencia del nivel}'' del nivel \textit{B}.
    	\item[Opción \#4] Permite asignar la ''\hyperref[table:METerminosExperiencia1]{experiencia del nivel}'' 1.
    	\item[Opción \#5] Permite asignar la ''\hyperref[table:METerminosExperiencia1]{experiencia otorgada}'' que entregarán todas las actividades.
    \end{description}
    \end{quote}
    
\clearpage
    \addfigure{1}{IU/IU_E03_config_parte2}{fig:IUE03_2}{Interfaz donde se  configura el esquema de experiencia parte 2.}
    
    
    \noindent {\bf Elementos importantes:}
    \begin{quote}
    \begin{description}
    	\item[Opción \#6] Permite asignar un nombre por defecto a los niveles.
    	\item[Opción \#7] Permite asignar un mensaje de felicitaciones por defecto al subir de nivel.
    	\item[Opción \#8] Permite asignar una descripción por defecto al subir de nivel.
    	\item[Opción \#9] Permite asignar el color por defecto del número de nivel.
    \end{description}
    \end{quote}
    
    
\clearpage
    
    \addfigure{1}{IU/IU_E03_config_parte3}{fig:IUE03_3}{Interfaz donde se  configura el esquema de experiencia parte 3.}
    
    \noindent {\bf Elementos importantes:}
    \begin{quote}
    \begin{description}
    	\item[Opción \#10] Permite asignar el color por defecto de la barra de progreso.
    	\item[Opción \#11] Permite asignar una imagen por defecto a los niveles.
    	\item[Botón \#1 'Guardar cambios'] Con este botón el administrador puede guardar los cambios que haya hecho en las opciones de la interfaz.
    \end{description}
    \end{quote}
    
\clearpage

\subsection*{IU-M01 Ver intento de examen}
\label{IUM01}

    Esta interfaz es de Moodle, sin embargo, es utilizada para el caso de uso \hyperref[CU-E01]{CU-E01 Recibir experiencia}.\\
    
    \noindent En esta interfaz los alumnos pueden ver un resumen de su intento para resolver el examen, así como poder reanudar el intento para terminar o corregir, o para enviar las respuestas para su revisión.   

    \addfigure{1}{IU/IU_M01_VerIntento}{fig:IUM01}{IU-M01: Ver intento.}
    
    \noindent {\bf Elementos importantes:}
    \begin{quote}
    \begin{description}
    	\item[Botón \# 1 'Regresar al intento'] Con este botón el usuario puede seguir contestando el examen.
    	\item[Botón \# 2 'Enviar todo y terminar'] Con este botón el usuario indica que terminó de responder el examen y que quiere enviar las respuestas para su revisión.
    \end{description}
    \end{quote}
	\clearpage

\subsection*{IU-M02 Confirmación de envío de intento}
\label{IUM02}

    Esta interfaz es de Moodle, sin embargo, es utilizada para el caso de uso  \hyperref[CU-E01]{CU-E01 Recibir experiencia}.\\
    
    \noindent Esta interfaz es una ventana emergente que permite al usuario pensárselo una segunda vez antes de subir sus respuestas para ser calificadas. Esto inclusive por si le da por error.\\   

    \addfigure{1}{IU/IU_M02_PopUp_Confirmacion}{fig:IUM02}{IU-M02: Confirmación de envío de intento.}
    
    \noindent {\bf Elementos importantes:}
    \begin{quote}
    \begin{description}
    	\item[Botón \#1 'Enviar todo y terminar'] Con este botón el usuario puede reafirmar que quiere mandar sus respuestas para ser calificadas.
    	\item[Botón \#2 'Cancelar'] Con este botón el usuario puede cancelar el subir sus respuestas.
    	\item[Botón \#3 'X'] Este botón tiene el mismo efecto que el Botón \# 2 'Cancelar'.
    \end{description}
    \end{quote}
	\clearpage

\subsection*{IU-M03 Intento calificado}
\label{IUM03}

    Esta interfaz es de Moodle, sin embargo, es utilizada para el caso de uso   \hyperref[CU-E01]{CU-E01 Recibir experiencia}.\\
    
    \noindent Si un usuario envía su intento para ser calificado 
    %y dicho intento puede ser calificado por el sistema,
    se muestra esta interfaz con la calificación de su intento. Algo importante es que la interfaz  \textbf{IU-E01 Bloque de experiencia} está visible, haciendo posible mostrar la actualización de los datos del usuario.\\

    \addfigure{1}{IU/IU_M03_FinalizarIntento}{fig:IUM03}{IU-M03: Intento calificado.}
    
    
	\clearpage

\subsection*{IU-M04 Sección de plugins}
\label{IUM04}

    Esta interfaz es de Moodle, sin embargo, es utilizada para el caso de uso \hyperref[CU-E02]{CU-E02 Configuración esquema de experiencia} .\\
    
    \noindent Esta interfaz contiene la lista de todos los plugins que tienen configuraciones globales, donde cada uno de sus elementos es un enlace a la página de configuración respectiva a cada plugin. Además en su primera sección tiene las opciones para ver, manejar e instalar plugins.\\

    \addfigure{1}{IU/IU_M04_SeccionPlugins}{fig:IUM04}{IU-M04: Sección donde están todos los plugins con configuraciones globales.}
	\clearpage
	

\subsection*{Moodle: IU-M05 Crear Curso}

    El objetivo de esta pantalla (Figura \ref{moodle:nuevoCurso}) es permitirle al profesor crear un nuevo curso, especificando los datos generales del curso, la descripción, apariencia, tamaño de los archivos, el seguimiento de finalización, los grupos, renombre de roles y las marcas vinculadas al curso.
    
    \addfigureB{1}{IU/mCrearCurso}{moodle:nuevoCurso}{Moodle IU-M05 Crear curso}
    
    {\bf Elementos}
    \begin{itemize}
        \item \bb{Datos Generales} es un formulario que contiene el nombre completo y corto del curso (obligatorios), su identificador, categoría, visibilidad, así como las fechas de inicio y término del curso.
        \item \bb{Descripción} 
        \item \bb{Formato de curso} 
        \item \bb{Apariencia}
        \item \bb{Archivos y subidas}
        \item \bb{Seguimiento de finalización}
        \item \bb{Grupos}
        \item \bb{Renombrar rol}
        \item \bb{Marcas}
    \end{itemize}

\section{Diseño}

\subsection{Diagrama de Clases}
    
    En la figura \ref{fig:classesXP} se muestra el diagrama de clases, los archivos {\it lib, events, settings, version} y los {\it módulos AMD} son representados mediante el uso de clases. Para facilitar la lectura del diagrama se representa a móodle como un paquete completo, el cual lee los distintos archivos y clases que requiere el plugin para funcionar.

    \addfigure{1}{diagrams/classesExp}{fig:classesXP}{Diagrama de clases del Módulo de Experiencia}
\clearpage

\subsection{Diagrama de componentes}

    En la figura \ref{fig:bloques1} se muestra el diagrama de componentes del Módulo de experiencia que contiene como interactúa el Módulo con la plataforma Moodle.

    \addfigure{1}{diagrams/bloques1}{fig:bloques1}{Diagrama de componentes del Módulo de Experiencia}

\clearpage
\subsection{Diagramas de Secuencia}
\subsection*{DS-E2: Crear curso con experiencia}

    Para diseñar la forma en que se ejecuta el caso de uso CU-E2, se tomó en consideración el flujo normal de eventos emitidos cuando se crea un curso en moodle. Los eventos emitidos en orden cronológico son {\it course\_created}, {\it course\_section\_created} y {\it enrol\_instance\_created}.\\
    
    \noindent En la figura \ref{ds:e2} se detalla la interacción entre el core de moodle, los eventos emitidos, y las clases del plugin {\bf Format Gamedle}.
    
    