\chapter{Módulo de Competencias}
\label{mod:competencias}

    %Este módulo contiene la especificación de las herramientas que involucran duelos/retos entre personas, así como, persona contra máquina.
    Este módulo contiene la especificación de los herramientas que permiten realizar competencias en dos modalidades, entre alumnos o alumno contra la computadora (CPU).

\section{Submódulo de Competencia 1 vs Sistema}

    Un alumno puede desafiar a la máquina (CPU) y establecer un nivel de dificultad (Fácil, Normal, Difícil e Imposible), los ejercicios se eligen de un conjunto de ejercicios propuestos por el profesor, la selección se realiza considerando el avance del alumno en el curso.  
    
    \begin{quote}
    \begin{description}
        \item[Objetivo:] \hfill\\
            Permitirle al alumno practicar y superarse en los conocimientos que tiene del curso.
        \item[Principios a los que da soporte:] \hfill
            \begin{itemize}
                \item 5 \principioV
                \item 7 \principioVII
            \end{itemize}
    \end{description}
    \end{quote}
    
\clearpage
    
    \noindent En la tabla \ref{tbl:probab} se muestra la probabilidad de que el sistema responda correctamente por cada uno de los niveles propuestos. Éste valor de probabilidad estará oculto al alumno.\\
    
    \addtable{|l|c|}{tbl:probab}{
        {\bf Nivel} & {\bf Probabilidad} \\\hline
        Fácil &     25 \%\\
        Normal &    50 \%\\
        Difícil &   60 \%\\
        Imposible & 90 \%\\\hline
    }{Probabilidad de que el Sistema responda de forma correcta}
    
    
\section{Submódulo de Competencias 1 vs 1}

    Un alumno de un curso puede desafiar a otro en una competencia 1 contra 1, los ejercicios se eligen aleatoriamente del conjunto de ejercicios propuestos por el profesor del curso, la selección de ejercicios de realiza considerando el progreso del curso que los alumnos tienen en común.
    
    \begin{quote}
    \begin{description}
    \item[Objetivo] \hfill\\
        Permitirle al alumno interactuar con sus compañeros de clase en una competencia amistosa y así ayudarles a practicar los temas vistos en clase.
    
    \item[Principios a los que da soporte:] \hfill
        \begin{itemize}
            \item 5 \principioV
            \item 7 \principioVII 
        \end{itemize}
    \end{description}
    \end{quote}
 
\section{Submódulo de Competencias 1 vs 1 - Con apuestas}

    Esta herramienta extiende la funcionalidad de la ''Herramienta de Competencias 1 vs 1'' permitiendo a los alumnos aportar una cantidad de monedas de plata en mutuo acuerdo.
    %Ahora los alumnos pueden desafiarse entre sí y apostar una cantidad de monedas de plata que les parezca justa  a ambos.
    
    \begin{quote}
    \begin{description}
    \item[Objetivo] \hfill\\
        Permitirle al alumno poder conseguir monedas de plata mientras práctica los conocimientos del curso y convive con sus compañeros del mismo.
    
    \item[Principios a los que da soporte:] \hfill
        \begin{itemize}
            \item 5 \principioV
            \item 6 \principioVI
            \item 7 \principioVII
            \item 8 \principioVIII
        \end{itemize}
    \end{description}
    \end{quote}
    
\clearpage
    
\section{Submódulo de Torneo}

    En la creación de un curso el profesor tendrá la opción de incluir un torneo, este se divide en dos fases; la primera consiste en la clasificación de alumnos que competirán; la segunda, en las rondas del torneo. Dependiendo cuántas rondas un alumno logre avanzar será su ''posicionamiento''. 
    
    \begin{quote}
    \begin{description}
    \item[Objetivo] \hfill\\
        Permitirle el alumno competir con sus compañeros mientras practica y demuestra sus conocimientos. Además de permitirle comparar sus conocimientos con el resto de sus compañeros.
    
    \item[Principios a los que da soporte:] \hfill
        \begin{itemize}
            \item 5 \principioV
            \item 7 \principioVII 
        \end{itemize}
    \end{description}
    \end{quote}
    
    \noindent Las rondas del torneo se realizarán mediante la herramienta de competencias 1 vs 1, en donde no se realizarán desempates.
    
\section{Submódulo de Torneo con ganancias}

    Esta herramienta extiende la funcionalidad de la ''Herramienta de Torneo'', dando recompensas al finalizar el torneo según el posicionamiento que tuvieron los participantes.
    
    \begin{quote}
    \begin{description}
    \item[Objetivo] \hfill\\
        Recompensar al alumno por su desempeño y participación en el torneo del curso. 
    \item[Principios a los que da soporte:] \hfill
        \begin{itemize}
            \item 5 \principioV
            \item 6 \principioVI
            \item 7 \principioVII 
        \end{itemize}
    \end{description}
    \end{quote}
    
\section{Submódulo de competencias poker}
    El profesor puede habilitar que haya este tipo de competencias, esto permite a un alumno proponer una competencia de poker a sus compañeros de curso. Se propone que el número de integrantes sea de 2 a 10 como máximo.
    
    \begin{quote}
    \begin{description}
    \item[Objetivo] \hfill\\
        Permitirle al alumno reforzar sus conocimientos mientras compite con el resto de sus compañeros. 
    \item[Principios a los que da soporte:] \hfill
        \begin{itemize}
            \item 5 \principioV
            \item 6 \principioVI
            \item 7 \principioVII 
            \item 8 \principioVIII 
        \end{itemize}
    \end{description}
    \end{quote}
    
    \noindent Para el poker se han diseñado tres fases, la primer fase consiste en la apuesta inicial, donde a los participantes se les otorga la cantidad mínima para apostar; la segunda fase contiene las rondas de preguntas; y en la tercer fase se realizan las apuestan. Las rondas 2 y 3 se repetirán hasta que nadie apueste, solo quede una persona o se haya llegado al límite de rondas.
    
    %La secuencia con ronda de preguntas y con ronda de apuesta (no iniciales) se repetirá hasta que nadie apueste, solo quede una persona o que se haya llegado al límite de rondas.
    %\begin{quote}
    %\begin{enumerate}
        %\item Ronda de apuesta inicial
        %\item Ronda de pregunta
        %\item Ronda de apuesta
    %\end{enumerate}
    %\end{quote}