\chapter{Análisis general}
\label{ch:analisis}

\section{Implementación de la metodología}


\section{Implementación de los marcos de trabajo}

En nuestro caso nos enfocaremos en solo 5 pasos puesto que el paso de la diversión es abstracto y no se puede definir de manera concreta.

\begin{enumerate}
    \item Definir los objetivos del sistema: en este caso sería aumentar el tiempo que pasan los estudiantes en Moodle siendo productivos.
    
    \item Definir los comportamientos que queremos que tengan los usuarios: En nuestro caso sería que queremos que realicen las actividades propuestas por los profesores, realizar las competencias programadas. Podemos medir esto al dar puntos por cada actividad realizada y por participar en las acciones interactivas.
    
    \item Describir nuestros usuarios:
    Nos apegamos a la división propuesta de Richard Bartle \cite{TiposDeUsuario}, los divide en 4 grupos, triunfadores, exploradores, socializadores y asesinos. En nuestro caso no nos enfocaremos en los exploradores puesto que los cursos de moodle son lineales.
    
    \begin{itemize}
        \item Los triunfadores son aquellos que les gusta recibir premios, en nuestro caso los logros que tenemos contemplados.
        \item Socializadores, les gusta trabajar en equipo, por lo tanto nuestra propuesta de tener grupos que estén compitiendo unos con otros va enfocada con este tipo de usuario.
        \item  Los asesinos son los usuarios competidores que sienten motivación al ganarle a otras personas, las competencias 1 vs 1 y los otros tipos que tenemos contemplados motivarían a este tipo de usuario.
    \end{itemize}
    
    Es importante recalcar que en este momento las divisiones de los usuarios son mapeados en los principios del marco de trabajo Octalysis de la siguiente manera en el cuadro \ref{table:usuariosvprincipios} según el autor de Octalysis\cite[p. 414]{libro2}.
    
    \begin{table}[h!]
\centering
\begin{tabular}{|c|c|} 
 \hline
 Triunfadores & Principio II, Principio VI \\
 \hline
  Socializadores &  Principio V, Principio III, Principio VII\\
 \hline
 Asesinos & Principio II, Principio V, Principio VIII, Principio IV \\
 \hline
\end{tabular}
\caption{Tabla de mapeo de tipos de usuario y principios de Octalysis}
\label{table:usuariosvprincipios}
\end{table}
    
    \item Desarrollar los ciclos de las actividades:
    Este punto trata acerca de ver los pasos que se necesitan para que los usuarios se mantengan motivados, principalmente por medio de la retroalimentación. Esta retroalimentación se puede mostrar por medio de puntos, o logros, y el ciclo general es:\par\hfill motivación -> acción -> retroalimentación -> motivación. \hfill
    
    Es importante tomar en cuenta que estos ciclos pueden ser repetitivos y aburridos por lo cual es necesario agregar escaleras de progreso, es decir que cada que se complete un ciclo, el siguiente sea más difícil de completar, por ejemplo, al subir de nivel que el siguiente necesite más puntos para obtenerse.
    
    \item Elije los componentes apropiados para el sistema:
    Hasta este punto es en el que decidimos que queremos utilizar puntos, tablas de lideres, niveles, experiencia, competencias, narrativas. 
    
       
\end{enumerate}

\section{Definición del alcánce}

El alcance de este proyecto es representado mediante el Product Backlog (artefacto de Scrum). El product backlog incluye dos tipos de items: los items de documentación, denotados por ls clave {\it {\bf A}x}; y los items de desarrollo del proyecto, denotados por las claves {\it {\bf RF}x} y {\it{\bf RNF}x}.\\

\noindent A continuación se menciona la lista completa de actividades y requerimientos recopilados, debido 
%debido a la larga lista de requerimientos no es posible cumplir para el término del trabajo terminal, 
se realizarán los que tienen mayor prioridad. Esta lista puede ser ampliada o reducida bajo indicaciones estrictas de los directores del trabajo terminal.\\

\noindent Al final de este documento se incluye el documento de Metodología como anexo, el cual detalla las características que deben tener los artefactos de Scrum, así como la configuración de Scrum para este proyecto.\\

\section{Product Backlog}

    \begin{multicols}{2}
    
\newcounter{theActivity}\stepcounter{theActivity}
\newcounter{theRF}\stepcounter{theRF}
\newcounter{theRNF}\stepcounter{theRNF}


\newenvironment{Actividad}[2]{\begin{itemize}\item[\bf A\arabic{theActivity}] {\bf #1:} {#2}}{\end{itemize}\stepcounter{theActivity}}

\newenvironment{RF}[2]{
    \begin{itemize}
        \item[\bf \hypertarget{RF\arabic{theRF}}{RF\arabic{theRF}}] 
        %\item[\bf RF\arabic{theRF}] 
        {\bf #1:} {#2}}{
    \end{itemize}
    \stepcounter{theRF}
    }

\newenvironment{RNF}[2]{\begin{itemize}\item[\bf RNF\arabic{theRNF}] {\bf #1:} {#2}}{\end{itemize}\stepcounter{theRNF}}
\newcommand{\Sprint}[1]{{\color{color:first}\fbox{Sprint #1}}}
\newcommand{\PBitem}{\item[] \quad}
%\noindent\bb{Sprint 1: Marco Teorico, Metodología, Alcance TT-I}
\begin{Actividad}{Investigar Scrum}{% 
    Redactar el capítulo de I del documento de metodología el cual describe el marco de trabajo Scrum, basándose en la guía oficial.} 
    \PBitem \Sprint{1}%Estimación 2 dias. \Sprint{1} 
\end{Actividad}

\begin{Actividad}{Adaptación de Scrum}{%
    Especificar como es configurado Scrum para el proyecto, definir roles, eventos, y artefactos.}
    \PBitem \Sprint{1}%Estimación 2 dias. \Sprint{1} 
\end{Actividad}

\begin{Actividad}{Adquirir Actionable Gamificación}{%
    Adquisición del libro de Yu-kai Chou, {\it Actionable Gamification: Beyond Points, Badges, and Leaderboards}}
    \PBitem \Sprint{1}%Estimación 2 dias. \Sprint{1} 
\end{Actividad}

\vfill\null\columnbreak

\begin{Actividad}{Investigar Gamificación}{%
    Ampliar la investigación de gamificación, definiciones, sus inicios, uso en la educación.}
    \PBitem \Sprint{1}%Estimación 2 dias. \Sprint{1} 
\end{Actividad}

%\begin{Actividad}{Principios de Gamificación}{%
    %Investigar cada uno de los principios de gamificación de acuerdo con el marco de referencia Octalysis,  buscar técnicas para poder soportar los principios.}
    %\PBitem Estimación 2 dias. \Sprint{1} 
%\end{Actividad}

\begin{Actividad}{Estado del arte}{%
    Investigar el estado del arte en relación a desarrollo de funcinalidades a una plataforma de aprendizaje.}
    \PBitem \Sprint{1}%Estimación 2 dias. \Sprint{1} 
\end{Actividad}

%\begin{Actividad}{Redactar Marco Teórico}{%
    %Invetigar distintos papers que definan la gamificación, describan sus inicios}
    %\PBitem Estimación 2 dias. \Sprint{1} 
%\end{Actividad}

\begin{Actividad}{Establecer los módulos}{%
    Plantear una propuesta integral la cual divida en módulos principales las funcionalidades que tendrá el producto final.}
    \PBitem \Sprint{1}%Estimación 2 dias. \Sprint{1} 
\end{Actividad}

\pagebreak

\begin{Actividad}{Alcance TT-I}{%
    Definir el alcance que tendrá el proyecto para la presentación del trabajo terminal I.}
    \PBitem \Sprint{2}%Estimación 2 dias. \Sprint{1} 
\end{Actividad}

%\hfill\bigskip\\\noindent\bb{Sprint 2: Investigación de Implementación}

\begin{Actividad}{Módulo I y II}{%
    Especificar el funcionamiento y el análisis inicial que se realiza en el módulo de Recompensa.}
    \PBitem \Sprint{2}%Estimación 2 dias. \Sprint{2} 
\end{Actividad}

%\begin{Actividad}{Módulo II}{%
    %Especificar el funcionamiento y el análisis inicial que se realiza en el módulo Financiero.}
    %\PBitem Estimación 2 dias. \Sprint{2} 
%\end{Actividad}

\begin{Actividad}{Módulo III y IV}{%
    Especificar el funcionamiento y el análisis inicial que se realiza en el módulo de Seguimiento.}
    \PBitem \Sprint{2}%Estimación 2 dias. \Sprint{2} 
\end{Actividad}

%\begin{Actividad}{Módulo IV}{%
    %Especificar el funcionamiento y el análisis inicial que se realiza en el módulo de Competencia.}
    %\PBitem Estimación 2 dias. \Sprint{2} 
%\end{Actividad}

\begin{Actividad}{Módulo V y VI}{%
    Especificar el funcionamiento y el análisis inicial que se realiza en el módulo de Personalización.}
    \PBitem \Sprint{2}%Estimación 2 dias. \Sprint{2} 
\end{Actividad}

%\begin{Actividad}{Modulo VI}{%
    %De la propuesta de solución, describir cada uno de los módulos y herramientas (submódulos) que contienen.}
    %\PBitem Estimación 2 dias. \Sprint{2} 
%\end{Actividad}

\begin{Actividad}{Alternativas a Moodle}{%
    Investigar otros sistemas gestores de aprendizaje en los que se puedan desarrollar nuevas funcionalidades.}
    \PBitem \Sprint{2}%Estimación 2 días \Sprint{2}
\end{Actividad}


\begin{Actividad}{Implementación Gamificación}{%
    Investigar distintas publicaciónes (papers) en donde se describa la forma en que se implementó gamificación y los resultados obtenidos}
    \PBitem \Sprint{2}%Estimación 2 días 
\end{Actividad}
%trabajo los desarrollos, investigaciones y trabajos ya existentes acerca de la gamificación en una plataforma de aprendizaje.


%\hfill\bigskip\\\noindent\bb{Sprint 3: Reporte Técnico del trabajo Terminal}

\begin{Actividad}{Problema}{%
    Redactar el problema que se pretende atacar con este trabajo terminal.}
    \PBitem \Sprint{3}%Estimación 2 días \Sprint{3}
\end{Actividad}

\begin{Actividad}{Propuesta de Solución}{%
    redactar la propuesta de solución, que se pretendar dar ante el problema}
    \PBitem \Sprint{3}%Estimación 2 días \Sprint{3}
\end{Actividad}

\begin{Actividad}{Justificación}{%
    Redactar por que la justificación de porque surge el proyecto y porque se optó por esa propuesta de solución.}
    \PBitem \Sprint{3}%Estimación 2 días \Sprint{3}
\end{Actividad}

\vfill\null\columnbreak

\begin{Actividad}{Alcances y Limitaciones}{%
    Establecer los alcances y limitaciones que tiene el trabajo terminal}
    \PBitem \Sprint{3}%Estimación 2 días \Sprint{3}
\end{Actividad}

\begin{Actividad}{Instalar Moodle}{%
    Realizar la instalación de Moodle de forma local, en las computadoras de los miembros del equipo.}
    \PBitem \Sprint{3}%Estimación 2 días \Sprint{3}
\end{Actividad}

\begin{Actividad}{Usar Moodle}{%
    Familiarizarse con el uso de Moodle en especifico con las funcioalidades de un administrador, gestionar cursos, gestionar grupos, crear usuarios, etc}
    \PBitem \Sprint{3}%Estimación 2 días \Sprint{3}
\end{Actividad}

%\hfill\bigskip\\\noindent\bb{Sprint 4: Pruebas de Concepto}

\begin{Actividad}{Entorno de desarrollo}{%
    Establecer el entorno de desarrollo sobre el cual se trabajará, incluyendo características de instalación}
    \PBitem \Sprint{4}%Estimación 3 hrs \Sprint{4}
\end{Actividad}

\begin{Actividad}{Filtrar plugins}{%
    Escoger los plugins de los cuales se realizarán las pruebas de concepto y documentar los criterior de discrimnación ocupados.}
    \PBitem \Sprint{4}%Estimación 3 hrs \Sprint{4}
\end{Actividad}

\begin{Actividad}{P1: Database Fields}{%
    Realizar la prueba de database fields}
    \PBitem \Sprint{4}%Estimación 3 hrs \Sprint{4}
\end{Actividad}

\begin{Actividad}{P2: Database Presets}{%
    Realizar la prueba de database presets}
    \PBitem \Sprint{4}%Estimación 3 hrs \Sprint{4}
\end{Actividad}

\begin{Actividad}{P3: User Profile Fields}{%
    Realizar la prueba de user profile fields}
    \PBitem \Sprint{4}%Estimación 3 hrs \Sprint{4}
\end{Actividad}

\begin{Actividad}{P4: Block}{%
    Realizar la prueba de block}
    \PBitem \Sprint{4}%Estimación 3 hrs \Sprint{4}
\end{Actividad}

\begin{Actividad}{Reporte de Pruebas}{%
    Realizar el reporte de pruebas de concepto para entregar al profesor de seguimiento. }
    \PBitem \Sprint{4}%Estimación 3 hrs \Sprint{4}
\end{Actividad}
    
\clearpage

\begin{RF}{Logros en curso}{%
    El sistema debe permitir premiar o otorgar un elemento distintivo (logro) cuando un alumno realice alguna acción positiva dentro de un curso}
    \item[] Prior. MA. %Estimación 2 días. 
\end{RF}

\begin{RF}{Logros en Plataforma}{%
    El sistema debe permitir premiar o otorgar un elemento distintivo (logro) cuando un alumno realice alguna acción positiva a nivel plataforma}
    \item[] Prior. A. %Estimación 2 días. 
\end{RF}

\begin{RF}{Advertencias}{%
    El sistema debe permitir premiar o otorgar un elemento distintivo cuando un alumno realice alguna acción negativa, como si fuera una advertencia.}
    \item[] Prior. M. %Estimación 2 días.
\end{RF}

\begin{RF}{Marcadores}{%
    El sistema deberá permitir a los usuarios visualizar la lista de los mejores alumnos respecto al uso en la plataforma (cuantificado mediante los puntos de experiencia), la mejor calificación ponderada, el mayor numero de preguntas diarias, ...}
    \item[] Prior. MA. %Estimación 2 días. 
\end{RF}

\begin{RF}{Configurar Logros}{%
    El sistema deberá permitir al administrador cambiar el título, imagen y mensaje de los logros y advertencias que se otorgan a los alumnos.}
    \item[] Prior. B. %Estimación 2 días. 
\end{RF}

\begin{RF}{Habilitar Logros}{%
    El sistema deberá permitir al administrador habilitar y deshabilitar los logros y advertencias que el sistema pone a disposición}
    \item[] Prior. M. %Estimación 2 días. 
\end{RF}

\begin{RF}{Experiencia}{%
    El sistema deberá cuantificar como puntos de experiencia, qué tanto usan la plataforma de acuerdo con las actividades/acciones dentro y fuera de los cursos. }
    \item[] Prior. MA. \Sprint{5}%Estimación 2 días. 
\end{RF}

\begin{RF}{Configurar Experiencia}{%
    El sistema deberá contar con un mecanismo mediante el cual el administrador defina la cantidad de experiencia que se otorga al terminar un curso y al realizar distintas actividades/acciones.}
    \item[] Prior. A. \Sprint{5}%Estimación 2 días. 
\end{RF}

\begin{RF}{Niveles}{%
    El sistema deberá asignar a los alumnos un nivel de experiencia correspondiente a los incrementos de experiencia configurados y a la cantidad de experiencia recibida. }
    \item[] Prior. A. \Sprint{5}%Estimación 2 días. 
\end{RF}

\begin{RF}{Incremento de Niveles}{%
    El sistema deberá permitir al administrador configurar la forma en que incrementan los niveles (lineal o porcentual) y el valor de incremento. }
    \item[] Prior. M. \Sprint{5}%Estimación 2 días. 
\end{RF}

\begin{RF}{Configurar Niveles}{%
    El sistema deberá permitir al administrador configurar la imagen, título, descripción y mensaje de los niveles. }
    \item[] Prior. A. \Sprint{5}%Estimación 2 días. 
\end{RF}

\begin{RF}{Incrementar Nivel}{%
    El sistema deberá notificar a un alumno cuando aumente su nivel de experiencia.}
    \item[] Prior. MA. \Sprint{5}%Estimación 2 días. 
\end{RF}

\begin{RF}{Progreso}{%
    El sistema deberá mostrarle al un estudiante el progreso que el mismo tiene del curso, mediante una barra que indique el porcentaje que lleva realizado de un curso.}
    \item[] Prior. A. %Estimación 2 días.  
\end{RF}

\begin{RF}{Configurar Progreso}{%
    El sistema deberá permitir al administrador o al profesor elegir el color de la barra de progreso para los alumnos dentro de un curso.}
    \item[] Prior. M. %Estimación 2 días. 
\end{RF}

\begin{RF}{Narrativa}{%
    El sistema deberá permitir al administrador y profesores incluir una narrativa que se vaya contando conforme el curso vaya avanzando}
    \item[] Prior. MA. %Estimación 2 días. 
\end{RF}

\begin{RF}{Personalización de Curso}{%
    El sistema deberá permitir al administrador o profesor elegir el tema  o visualización del curso que está diseñando. }
    \item[] Prior. M. %Estimación 2 días. 
\end{RF}

\begin{RF}{Plantillas de Narrativas}{%
    El sistema deberá brindar al administrador plantillas de narrativas. }
    \item[] Prior. M. %Estimación 2 días. 
\end{RF}

\begin{RF}{Personaje de Narrativa}{%
    El sistema deberá permitir al administrador o profesor especificar los datos (nombre,  imagen, etc) de los personajes principales que forman parte de la narrativa de un curso}
    \item[] Prior. A. %Estimación 2 días. 
\end{RF}

\begin{RF}{Monedas}{%
    El sistema deberá de contar una moneda principal y otra secundaría para la adquisición de items mediante la tienda. }
    \item[] Prior. MA. %Estimación 2 días. 
\end{RF}

\begin{RF}{Configurar Esquema Financiero}{%
    El sistema deberá permitir al administrador indicar la cantidad de monedas que es otorgada en determinadas acciones, el precio que tienen los items de la tienda y las equivalencias entre la moneda principal y secundaria.}
    \item[] Prior. A. %Estimación 2 días. 
\end{RF}

\begin{RF}{Tienda}{%
    El sistema deberá de contar con una tienda virtual mediante la cual se puedan adquirir items utilizando las monedas }
    \item[] Prior. MA. %Estimación 2 días. 
\end{RF}

\begin{RF}{Añadir Item}{%
    El sistema deberá permitir al administrador añadir items para que estén disponibles en la plataforma, precio de moneda irreal, su categoría y demás atributos. }
    \item[] Prior. A. %Estimación 2 días. 
\end{RF}

\begin{RF}{Modificar Item}{%
    El sistema deberá permitir al administrador modificar si el precio de moneta irreal, categoría y demás atributos de los items disponibles en la plataforma. }
    \item[] Prior. A. %Estimación 2 días. 
\end{RF}

\begin{RF}{Bloquear Items}{%
    El sistema deberá permitir al administrador bloquear los items para que, posterior a ese momento no se pueda acceder a ellos.}
    \item[] Prior. M. %Estimación 2 días. 
\end{RF}

\begin{RF}{Desbloquear Items}{%
    El sistema deberá permitir al administrador desbloquear los items bloqueados para que estos vuelvan a estar disponibles en la plataforma y se pueda acceder a ellos.}
    \item[] Prior. M. %Estimación 2 días. 
\end{RF}

\begin{RF}{Exportar Items}{%
    El sistema deberá permitir al administrador exportar los items que ha creado con el propósito de guardarlos para posteriormente ser incluidos en otra plataforma con gamificación}
    \item[] Prior. B. %Estimación 2 días. 
\end{RF}

\begin{RF}{Avatar inicial}{%
    El sistema deberá brindarle a los usuarios un avatar inicial y genérico}
    \item[] Prior. MB. %Estimación 2 días. 
\end{RF}

\begin{RF}{Configuración Avatar inicial}{%
    El sistema deberá permitir al administrador establecer la apariencia del avatar que se otorga inicialmente a los usuarios }
    \item[] Prior. MB. %Estimación 2 días. 
\end{RF}

\begin{RF}{Item: Temas}{%
    El sistema deberá contar con items de tipo tema, los cuales permitan cambiar la visualización que un usuario tiene de la plataforma siempre y cuando tenga dicho item}
    \item[] Prior. MA. %Estimación 2 días. 
\end{RF}

\begin{RF}{Item: Skin Avatar}{%
    El sistema deberá contar con items de tipo Skin los cuales contengan un conjunto de elementos que cambien la apariencia del avatar. }
    \item[] Prior. MB. %Estimación 2 días. 
\end{RF}

\begin{RF}{Item: Ropa del Avatar}{%
    El sistema deberá contar con items de tipo Ropa, los cuales permitan cambiar una prenda al avatar que los usuarios tienen.}
    \item[] Prior. MB. %Estimación 2 días. 
\end{RF}

\begin{RF}{Item: Loot-Box}{%
    El sistema deberá con un tipo de item LootBox el cual otorge cualquier otro items utilizando la probabilidad y aleatoridad de acuerdo con las categorias de los items}
    \item[] Prior. A. %Estimación 2 días. 
\end{RF}

\begin{RF}{Configuración de Loot Boxes}{%
    El sistema deberá permitir al administrador cambiar los valores de probabilidad de obtener items de una categoría en especifico mediante un lootBox}
    \item[] Prior. M. %Estimación 2 días. 
\end{RF}

\begin{RF}{Monedas en Curso}{%
    El sistema deberá permitir al administrador/profesor ponerle un alias (nombre e imagen) a las monedas (principal y secundaria) dentro de un curso.}
    \item[] Prior. M. %Estimación 2 días. 
\end{RF}

\begin{RF}{Personalización}{%
    El sistema debe contar con una página de personalización donde el usuario pueda configurar su avatar, además de la visualización que el tiene de la plataforma.}
    \item[] Prior. M. %Estimación 2 días. 
\end{RF}

\begin{RF}{Retar a compañero}{%
    El sistema deberá permitir a un alumno desafiar a otro a una sesión de preguntas acerca de los temas de un curso que tengan en común. }
    \item[] Prior. M. %Estimación 2 días. 
\end{RF}

\begin{RF}{Apuestas en retos}{%
    El sistema deberá permitir a los competidores %que participan 
    apostar una cantidad en mutuo acuerdo entre los alumnos que forma parte de un reto. }
    \item[] Prior. A. %Estimación 2 días. 
\end{RF}

\begin{RF}{Retar al sistema}{%
    El sistema deberá permitir a un alumno desafiar al sistema a una sesión de preguntas, escogiendo un nivel de dificultad }
    \item[] Prior. A. %Estimación 2 días. 
\end{RF}

\begin{RF}{Recompensas en retos con el sistema}{%
    El sistema deberá dar recompensas de acuerdo con el nivel de dificulta elegido por el alumno.}
    \item[] Prior. MA. %Estimación 2 días. 
\end{RF}

\begin{RF}{Torneos}{%
    El sistema deberá permitir al profesor organizar un torneo entre los estudiantes de un curso, con el propósito de comparar el aprovechamiento de los estudiantes}
    \item[] Prior. A. %Estimación 2 días. 
\end{RF}

\begin{RF}{Recompensa en Torneos}{%
    El sistema deberá otorgar una recompensa al primer, segundo y tercer lugar, distintiva. }
    \item[] Prior. A. %Estimación 2 días. 
\end{RF}

\begin{RF}{Poker}{%
    El sistema deberá permitir a los usuarios iniciar una sesión de poker entre distintos alumnos, donde los mismos puedan apostar las monedas ficticias del sistema.}
    \item[] Prior. M. %Estimación 2 días. 
\end{RF}



\begin{RF}{Animación de Personajes}{%
    El sistema deberá de contener animaciones para los distintos elementos de interfaz de usuario. }
    \item[] Prior. MB. %Estimación 2 días. 
\end{RF}

\begin{RNF}{Bajo Acoplamiento}{%
    El sistema deberá trabajar con el menor acoplamiento}
    \item[Tipo] Propiedad de Software
    \item[] Prior. A. %Estimación 2 días. \Sprint{*}
\end{RNF}

\begin{comment}
\begin{RNF}{Robustez}{%
    El sistema debe, INSERTAR AQUI MÉTRICA DE ROBUSTEZ. }
    \item[Tipo] Propiedad de Software
    \item[] Prior. MA. %Estimación 2 días. \Sprint{*}
\end{RNF}
\end{comment}

\begin{RNF}{Modularidad}{%
    El sistema deberá permitir al administrador habilitar únicamente las herramientas que el decida incluir en la plataforma, y deshabilitar las que no requiera.}
    \item[Tipo.] Regla de Negocio
    \item[] Prior. A. %Estimación 2 días. \Sprint{*}
\end{RNF}

\begin{RF}{Preguntas diarias}{%
    El sistema deberá contar con un ejercicio que podrá ser contestado diariamente por el estudiante.}
    \item[] Prior. A. %Estimación 2 días.  
\end{RF}
    \end{multicols}

    %\begin{quote}
    %\noindent {\bf Nota:} El número de {\it Sprint} debe estar presente en todos los items correspondientes al sprint corriente y a los sprints anteriores a este. El atributo {\it Sprint} puede no estar presente en los items que no han sido vinculados a un Sprint. 
    \noindent {\bf Nota:} El número de {\it Sprint} debe estar presente en todos los ítems que ya hayan sido agregados a un Sprint Backlog.
    %\end{quote}
    
    
\clearpage    
\section{Módulos del proyecto}
    
    Los requerimientos presentes en el Product Backlog fueron agrupados en 6 módulos (ver figura \ref{fig:modulos}): el módulo de competencia, módulo financiero, módulo de personalización, módulo de seguimiento, módulo de experiencia y módulo de recompensa. Fueron identificados 19 submódulos distribuidos en los módulos anteriormente mencionados.\\
    
    \noindent Como se comentó en la sección \hyperrefx{subsec:plugins}, la manera más recomendable para extender las funcionalidades de moodle es desarrollando o incluyendo plugins, debido a esta razón, el análisis y diseño es realizado tomando en consideración de que se trabajará desarrollando plugins.
    
    \addfigure{1}{modulos/diagrams/modulosTT}{fig:modulos}{Módulos del proyecto}

\clearpage
\section{Relación módulos-principios}
    La figura \ref{fig:modulosP} muestra la relación que cada submódulo tiene con los principios de gamificación.
    
    \addfigure{0.89}{modulos/diagrams/modulosTTyP}{fig:modulosP}{Relación entre los principios de Gamificación y los submódulos identificados}
    
\section{Relación módulos-requerimientos}

En la tabla \ref{tab:modreq} se relacionan lo módulos definidos contra los requerimientos encontrados en el Backlog, se muestran sólo los identificadores de los requerimientos para una mayor legibilidad.

\newcommand{\Refr}[1]{{\hyperlink{#1}{#1}}}
\begin{table}[h!]
    \centering
    \begin{tabular}{|c|c|}
    \hline
        Competencia & \Refr{RF36}, \Refr{RF37}, \Refr{RF38}, \Refr{RF40}, \Refr{RF41}, \Refr{RF39}, \Refr{RF42}\\
    \hline
        Seguimiento & \Refr{RF13}, \Refr{RF14}, \Refr{RF44}\\
    \hline
        Financiero & \Refr{RF19}, \Refr{RF20}, \Refr{RF21}, \Refr{RF22}, \Refr{RF23}, \Refr{RF24}, \Refr{RF25}, \Refr{RF26}, \Refr{RF32}, \Refr{RF33}\\
    \hline
        Experiencia & \Refr{RF9}, \Refr{RF7}, \Refr{RF8}, \Refr{RF10}, \Refr{RF11}, \Refr{RF12}\\
    \hline
        Recompensa & \Refr{RF1}, \Refr{RF2}, \Refr{RF4}, \Refr{RF3}, \Refr{RF5}, \Refr{RF6}\\
    \hline
        Personalización & \Refr{RF15}, \Refr{RF16}, \Refr{RF17}, \Refr{RF18}, \Refr{RF27}, \Refr{RF28}, \Refr{RF29}, \Refr{RF30}, \Refr{RF31}, \Refr{RF34}, \Refr{RF35}\\
    \hline
    \end{tabular}
    \caption{Relación entre los módulos y requerimientos}
    \label{tab:modreq}
\end{table}

\section{Curva de aprendizaje}

\subsection{Instalación}
\subsection{Block Simple HTML}
\subsection{XMLCreator}
\subsection{Events API}
\subsection{DML API y DDL API}
\subsection{Javascript AMD}
\subsection{External Pages}
\subsection{Forms Validation}
\subsection{Tables}
\subsection{AJAX AlongSide Moodle}

\subsection{Estableciendo el entorno de desarrollo}

 Al final de este documento se incluye como anexo el documento que detalla el desarrollo de las pruebas de concepto. A continuación se muestran los resultados de dicho documento.

Para llevar a cabo desarrollo sobre la plataforma moodle recomienda considerar el uso de un entorno de desarrollo integrado o IDE (Integrated Development Environment), para facilitar las tareas de programación. Las opciones que brinda moodle en su documentación son los IDEs: Eclipse, Netbeans y PHPStorm.\\

\noindent La primer prueba fue realizada con Eclipse, se incluyeron los archivos del directorio de moodle como parte del proyecto, lamentablemente, los enlaces a los demás archivos, y la depuración de código arrojaban errores debido a que había archivos que no podía vincular correctamente. Por lo que Eclipse fue descartado posterior a la prueba.

\subsubsection{NetBeans}

Netbeans proporciona un buen soporte a PHP, este IDE tiene integración de un sistema de control de versiones, atajos de teclas, lista de funciones, completación de código, soporte para HTML, CSS y Javascript, renombre de archivos/clases instantáneo, búsqueda rápida, entre otros. \cite{NetBeans}, \cite{moodleNetbeans}
% https://netbeans.org/features/
% https://docs.moodle.org/dev/Setting_up_Netbeans

% En la figura \ref{fig:netbeans} se muestra el entorno instalado con el proyecto de ''moodle'' abierto

\subsubsection{PHPStorm}

PHPStorm es un IDE comercial desarrollado por JetBrains, es considerado uno de los mejores IDE para desarrolladores que trabajan con PHP, tiene características como completación e inspección de código, soporte para PHPUnit, soporte para BeHat, editor de base de datos, depurador, entre otras funcionalidades \cite{PHPStorm},\cite{  moodlePHPStorm}.
% https://www.jetbrains.com/phpstorm/features/
% https://docs.moodle.org/dev/Setting_up_PhpStorm

%En la figura \ref{fig:PHPStorm} se puede ver una captura del IDE con el proyecto con los archivos de Moodle, abierto.\\

\begin{quote}
Finalmente, después de haber realizado la prueba con los  tres IDEs, se eligió a PHPStorm como entorno de desarrollo considerando los siguientes puntos:
    \begin{itemize}
    \item Moodle considera que PHPStorm es uno de los mejores entornos de desarrollo para PHP.
    \item PHPStorm está diseñado desde un inicio para trabajar con PHP, a diferencia de NetBeans que dan soporte a PHP y a otros lenguajes de programación.
    \item PHPStorm tiene soporte para las versiónes más recientes para PHP, mientras que NetBeans soporta actualmente hasta la versión 5.6 de PHP.
    %\item Por recomendación de un profesor de la ESCOM, debido a su experiencia como usuario.
    % \item Consultando a algunos docentes nos recomendaron utilizar PHPStorm.
    %\item Se cuenta con una licencia gratis por pertenecer al IPN/ESCOM.
    \item El equipo de desarrollo en proyectos anterior ha utilizado anteriormente herramientas de JetBrains y se ha tenido una experiencia agradable. % Android Studio
    \end{itemize}
\end{quote}

\subsection{Desarrollo de las pruebas}

    De los 54 plugins listados en la sección \hyperrefx{subsec:plugins} se decidió priorizar el desarrollo de aquellos tipos de plugins que nos permitieran extender el esquema de base de datos de moodle, y de aquellos que nos permitieran desplegar la información en la interfaz de usuario, razón por la cual se realizaron las pruebas de concepto de los tipos de plugins Database Fields, Database Presets, User Profile Fields, y Blocks.\\

    \noindent A continuación el cuadro \ref{tbl:pruebasC} resume el propósito de cada prueba y los resultados obtenidos.

    \addtable{|c|p{0.30\textwidth}|p{0.45\textwidth}|}{tbl:pruebasC}{
        {\bf Tipo de\par Plugin} & {\bf Objetivo} & {\bf Resultados} \\\hline

        Database Fields &
            Saber si este plugin nos ayudaría a guardar valores en la base de datos, y si fuera capaz, saber la forma en que lo hace. &

            Database Fields nos permite, en caso de que requiriéramos crear un nuevo tipo de dato, que puede ser usado mediante el plugin ''Database Presets''. \\\hline

        Database Presets &
            Saber si este plugin nos permite modificar el esquema de la base de datos, y si fuera capaz, saber la forma en que lo hace. &

            Database Presets nos permite crear y guardar datos en la base de datos, las restricciones es que únicamente nos permite definir formularios. El plugin puede ser usado a  nivel plataforma o a nivel curso. \\\hline

        User Profile Fields &
            Saber si este plugin nos permite guardar valores relacionados al usuario, en la base de datos. &

            User Profile Fields permite crear nuestro propio tipo de dato y agregarlo como un campo más a los datos que el usuario debe introducir. Al incluirse un plugin de este tipo todos los usuarios de la plataforma podrán editar el dato especificado por este plugin. \\\hline

        Blocks &
            Ver cómo desplegar información mediante el uso de este tipo de plugin y asegurar que un mismo block se pueda ver en las vistas principales de la plataforma. &

            Los block/blocks pueden ser instanciados más de una vez y están ligados al usuario.\newline
            Cada plugin puede definir su propio esquema de tablas, atributos e índices.\newline
            Los plugins pueden habilitar/deshabilitar configuraciones generales para el administrador o locales para el usuario. \newline
            Un plugin puede suscribir una clase para capturar los eventos que arroja moodle. \\\hline

    }{Objetivos y resultados de las pruebas de concepto realizadas}
