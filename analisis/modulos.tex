
\section{Módulos identificados}
\label{analisis:modulos}

 Como se comentó en el capítulo de introducción, nuestra propuesta de solución consiste en
 desarrollar componentes que permitan implementar gamificación dentro de una plataforma de
 aprendizaje web. Las características más importantes que debe brindar nuestra propuesta
 de solución son las siguientes:

    \begin{itemize}
    \item {\bf\color{primary} Altamente configurable}

    \item {\bf\color{primary} Solo instala/usa lo que se desea.}

        Una de las principales caracteristicar que deben proporcionar estos
        componentes hacia los profesores es que les permitan escoger que elementos
        de gamificación desean incluir en sus cursos y cuales no.\\

    \item {\bf\color{primary} Los módulos pueden trabajar independientemente uno de otro}

        Respecto el poder elegir que características se desean incluir y cuales no,
        implica que los componentes que brindan dichas caracterñisticas hayan sido
        diseñados para que puedan trabajar tanto independientemente como
        colaborativamente

    \item {\bf\color{primary} Conección lógica de eventos}

        Que las cosas que pasen en módulo puedan permitir realizar acciones con
        otros módulos para que cuando se deseém ocupar ambos, estos puedan trabajar
        en conjunto.

    \end{itemize}
 
 \noindent 
 Además de las características presentes anteriormente se ha acordado que el conjunto
 de características debe set lo suficientemente grande para poder brindarle soporte a
 la implementación de los principios de gamificación dentro de la plataforma web de
 aprendizaje (ver objetivo).

 % TAMBIEN SE DESEA QUE SEAN VARIAS OPCIONES PARA QUE LOS PROFES TENGAN UN ABANICO
 % AMPLIO DE OPCIONES


 % DESCRIBIR POR QUE HAY UNO DE EXPERIENCIA, RECOMPENSA, SEGUIMIENTO, ETC.

 En esta sección se describen los distintos módulos identificados, el propósito de cada uno
 y los submódulos de los cuales estan constituidos

    Los requerimientos presentes en el Product Backlog fueron agrupados en 6 módulos (ver figura \ref{fig:modulos}): el módulo de competencia, módulo financiero, módulo de personalización, módulo de seguimiento, módulo de experiencia y módulo de recompensa. Fueron identificados 19 submódulos distribuidos en los módulos anteriormente mencionados.\\

    %    \noindent Como se comentó en la sección \hyperrefx{subsec:plugins}, la manera más recomendable para extender las funcionalidades de moodle es desarrollando o incluyendo plugins, debido a esta razón, el análisis y diseño es realizado tomando en consideración de que se trabajará desarrollando plugins.
 
    \addfigure{1}{analisis/diagrams/modulosTT}{fig:modulos}{Módulos del proyecto}
