
\section{Módulos identificados}
\label{analisis:modulos}

 Como se comentó en el capítulo de \nameref{ch:introduccion}, nuestro
 objetivo es crear una herramienta que permita implementar los principios de
 gamificación dentro de una plataforma web de aprendizaje. Con base en el objetivo
 se identificarón seís módulos principales a desarrollar, cada uno con sus
 respectivos submódulos (ver figura \ref{fig:modulos}).\\

 \noindent
 Los módulos identificados permitiran extender la funcionalidad de la plataforma web
 de aprendizaje Moodle, la cual fue elegida con base en la investiación realizada en
 la seccion \hyperrefx{sec:sistemasaprendizaje}. A continuación se describe cada uno de los módulos y
 submódulos identificados.\\

    \addfigure{1}{analisis/diagrams/modulosTT}{fig:modulos}{Diseño modular del sistema}

\subsection{Módulo de Experiencia}
 \newcommand{\itemx}[1]{\item{\bf\color{primary}#1}}

 Este módulo brindará un mecanismo que permite a los usuarios medir su progreso
 como puntos de experiencia a nivel plataforma, además define la forma en que se
 obtendrán los puntos de experiencia, la forma en que se visualizará la información,
 el número requerido para superar cada nivel y la barra de progreso del nivel
 actual.

\subsubsection{Esquema de experiencia.}

 El esquema de experiencia es la configuración sobre cómo funciona el sistema
 de puntos de experiencia, incluyendo la cantidad de experiencia que tiene
 cada nivel, el tipo de incremento en los puntos de experiencia nivel a nivel,
 las restricciones sobre la experiencia y la forma en que se otorgarán los puntos.

\subsubsection{Niveles.}

 Es el mecanismo que permite mostrarle a los alumnos el progreso que han tenido
 a nivel plataforma mediante el nivel y los puntos de experiencia obtenidos en 
 los cursos, además contiene la configuración para establecer el cómo se vera el
 nivel y la experiencia obtenida de dicho nivel.

 % Los principios de gamificación a los cuales permite brindarles soporte son:
 % \principioII
 % \principioVI

\subsection{Módulo de Recompensa}

\subsubsection{Marcadores}
\subsubsection{Logros}
\subsubsection{Advertencias}

\subsection{Módulo de Personalización}

\subsubsection{Items de personalización}
\subsubsection{Personalización}
\subsubsection{Narrativa}

\subsection{Módulo Financiero}

\subsubsection{Esquema Financiero}
\subsubsection{Tienda}
\subsubsection{Loot Boxes}

\subsection{Módulo de Competencias}

\subsubsection{Retos}
\subsubsection{Retos al sistema (1 vs CPU)}
\subsubsection{Torneo}
\subsubsection{Apuestas}
\subsubsection{Torneo con ganancias}
\subsubsection{Competencias Poker}

\subsection{Módulo de Seguimiento}

\subsubsection{Preguntas diarias}
\subsubsection{Barras de progreso}

\noindent 

\section{Relacion entre los módulos y principios}

 Los módulos descritos anteriormente contemplan todas las funcionalidades que se desarrollarán
 durante el desarrollo de este trabajo terminal, en conjunto todos los módulos planteados contienen
 19 submódulos. Los módulos y submódulos se presentan organizados en la figura \ref{fig:modulos}.\\

 \clearpage

 \noindent 
 % TAMBIEN SE DESEA QUE SEAN VARIAS OPCIONES PARA QUE LOS PROFES TENGAN UN ABANICO
 % AMPLIO DE OPCIONES

\section{Características principales}

 Como se comentó en el capítulo de \hyperref[ch:introduccion]{introducción}, nuestra
 propuesta de solución consiste en desarrollar componentes que permitan implementar
 gamificación dentro de una plataforma de aprendizaje web con el propósito de
 al objetivo planteado el cual es

 Las características más
 importantes que debe brindar nuestra propuesta de solución son las siguientes:

    \begin{multicols}{2}
    \begin{itemize}
    \itemx{ Altamente configurable }

    \item[] Hace referencia a que se debe proporcionar al administrador, profesores
            y alumnos la flexbilidad para que puedan configurar los valores por defecto
            y visualización de la herramienta que se desarrollará.

    \itemx{ Escoge que quieres incluir }

    \item[] Una de las principales caracteristicar que deben proporcionar estos
            componentes hacia los profesores es que les permitan escoger que elementos
            de gamificación desean incluir en sus cursos y cuales no.\\

    %\vfill\null
    %\columnbreak
    %\vfill\null

    \itemx{ Bajo acoplamiento }

    \item[] Respecto el poder elegir que características se desean incluir y cuales no,
            implica que los componentes que brindan dichas caracterñisticas hayan sido
            diseñados para que puedan trabajar tanto independientemente como
            colaborativamente.

    \itemx{ Comunicación entre módulos }

    \item[] Que las cosas que pasen en módulo puedan permitir realizar acciones con
            otros módulos para que cuando se deseém ocupar ambos, estos puedan trabajar
            en conjunto.

    \end{itemize}
    %\vfill\null
    \end{multicols}
 

% =========================================
%  N O T E S
% =========================================

% DAN:
%   I THINK, IT'S A GOOD TO REMOVE THE SYSTEM REQUIREMENTS AND
%   USE THE 19 ''SUBMODULES'' AS THE PRODUCT BACKLOG ITEMS.
%

 % Los requerimientos presentes en el Product Backlog fueron agrupados en 6 módulos (ver
 % figura \ref{fig:modulos}): el módulo de competencia, módulo financiero, módulo de personalización,
 % módulo de seguimiento, módulo de experiencia y módulo de recompensa. Fueron identificados
 % 19 submódulos distribuidos en los módulos anteriormente mencionados.\\

 % \noindent Como se comentó en la sección \hyperrefx{subsec:plugins}, la manera más recomendable
 % para extender las funcionalidades de moodle es desarrollando o incluyendo plugins, debido a
 % esta razón, el análisis y diseño es realizado tomando en consideración de que se trabajará
 % desarrollando plugins.
 
