
\section{Relacion entre los módulos y principios}
\section{Soporte para los principios de gamificación}

 Los módulos descritos anteriormente contemplan todas las funcionalidades que se desarrollarán
 durante el desarrollo de este trabajo terminal, en conjunto todos los módulos planteados contienen
 19 submódulos. Los módulos y submódulos se presentan organizados en la figura \ref{fig:modulos}.\\

 \noindent 
 % TAMBIEN SE DESEA QUE SEAN VARIAS OPCIONES PARA QUE LOS PROFES TENGAN UN ABANICO
 % AMPLIO DE OPCIONES

\section{Relación entre los módulos principios y tipos de usuario}
\label{analisis:principios}
    
    La figura \ref{fig:modulosP} muestra la relación que cada submódulo tiene con los principios de gamificación.
    
    \addfigure{0.89}{analisis/diagrams/modulosTTyP}{fig:modulosP}{Relación entre los principios de Gamificación y los submódulos identificados}

    Es importante recalcar que en este momento las divisiones de los usuarios son mapeados 
    en los principios del marco de trabajo Octalysis de la siguiente manera en el cuadro
    \ref{table:usuariosvprincipios} según el autor de Octalysis\cite[p. 414]{Octalysis}.
    
    \begin{table}[h!]
    \centering
    \begin{tabular}{|c|c|} \hline
        Triunfadores & Principio II, Principio VI \\ \hline
        Socializadores &  Principio V, Principio III, Principio VII\\\hline
        Asesinos & Principio II, Principio V, Principio VIII, Principio IV \\\hline
    \end{tabular}
    \caption{Tabla de mapeo de tipos de usuario y principios de Octalysis}
    \label{table:usuariosvprincipios}
    \end{table}
