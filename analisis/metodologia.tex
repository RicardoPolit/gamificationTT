

\section{Adaptación de la metodología}

 A continuación se presenta la forma en que ha sido configurado el marco de trabajo {\it Scrum}
 para trabajar en concordancia con los marcos de trabajo {\it Octalysis} y {\it For The Win}
 usados cómo guía para la diseño de los componentes creados para la implementación de
 Gamificación en una plataforma en línea.

\subsection{Roles}

 \noindent 
 El marco de trabajo {\it Scrum} define los roles: scrum master, dueño del producto, stakeholders,
 y equipo de desarrollo. A continuación se especifica las personas que estarán cumpliendo los
 distintos roles en Scrum.

\subsubsection{Product Owner}

 Durante este proyecto el rol del dueño del producto (o {\it product owner}) lo llevarán a cabo
 los directores del trabajo terminal, ya que son a máxima autoridad entorno a la definición del
 alcance, ambos directores son listados a continuación:

    \begin{quote}
    \begin{itemize}
        \item {\it M. en C.} Sandra Ivette Bautista
        \item {\it M. en C.} Edgar Armando Catalán Salgado
    \end{itemize}
    \end{quote}

 \noindent 
 A pesar de que en la guía oficial de scrum \cite{TheScrumGuide} se especifica que el {\it
 product owner} debe ser una persona, se decidió que este rol fuera llevado acabo mediante los
 directores del trabajo terminal, con la presima de que para la toma de decisiones ambos
 directores deben estár de acuerdo.

\subsubsection{Equipo de desarrollo}

 Durante este trabajo el equipo de desarrollo estará conformado por un total de
 tres integrantes, siento estos los alumnos que presentan el trabajo terminal. Los miembros del
 equipo de desarrollo team son listados a continuación:

    \begin{quote}
    \begin{itemize}
        \item David Flores Casanova
        \item Ricardo Naranjo Polit
        \item Daniel Isaí Ortega Zúñiga
    \end{itemize}
    \end{quote}

% ROLES - PRODUCT OWNER

 % \noindent
 % Para que las funciones del Product Owner sean exitosas, la organización debe respetar sus
 % decisiones. Nadie puede forzar al Development Team para trabajar en un conjunto distinto
 % de requerimientos establecidos en el Product Backlog.

% ROLES - EQUIPO DE DESARROLLO

 % durante el desarrollo de este proyecto el equipo de desarrollo estará conformado por:

 % \noindent El tamaño del equipo debe ser lo suficientemente pequeño para permanecer ágil y lo
 % suficientemente grande para realizar entregas significativas al final de cada {\em Sprint}.\\
 % Un equipo con menos de tres miembros disminuiría la interacción y por lo tanto la productividad,
 % por otro lado, con más de nueve miembros se requiere mayor coordinación con más complejidad.

\subsubsection{Maestro Scrum}

 El rol del {\it Scrum Master} durante este proyecto se llevará a cabo mediante
 dos personas con la finalidad de dividir las responsabilidades y no sobrecargar de trabajo
 a una persona. El {\it Scrum Master} estará conformado por:

    \begin{quote}
    \begin{itemize}
        \item M. en C. Edgar Armando Catalán {\it (Responsabilidades hacia el Product Owner)}
        \item Daniel Isaí Ortega {\it(Responsabilidades hacia el equipo de desarrollo)}
    \end{itemize}                        
    \end{quote}

% ROLES - STAKEHOLDERS

\subsubsection{Stakeholders}

 Los Stakeholders son personas externas al equipo Scrum con un interés y/o conocimientos
 específicos del producto \cite{ScrumGlosary}. En relación a la naturaleza del proyecto se
 considera a los sinodales cómo los Stakeholders oficiales, los cuales son listados a
 continuación:

    \begin{quote}
    \begin{itemize}
        \item Dra. Fabiola Ocampo Botello
        \item M. en C. María del Socorro Téllez Reyes
        \item M. en C. José David Ortega Pacheco
    \end{itemize}
    \end{quote}

\subsection{Eventos}

\subsubsection{Sprints}
\subsubsection{Planeación}
\subsubsection{Deunión diaria}
\subsubsection{Revisión}
\subsubsection{Retroalimentación}
% EVENTOS
% EVENTOS - SPRINTS

 % \begin{quote}
 %   \noindent Para este proyecto los Sprints están configurados a una duración de 14 días con una
 %   estimación de 18 iteraciones, la duración de dos semanas se estableció con el propósito de:
    
    % \begin{itemize}
    %    \item Incrementar la retroalimentación y detectar los impedimentos
    %          en la forma de trabajo lo más pronto posible, y

    %    \item Realizar incrementos más cortos y continuos considerando que el
    %          equipo de desarrollo está conformado por tres integrantes.
    % \end{itemize} 
 % \end{quote}

% EVENTOS - PLANEACiÓN (SPRINT PLANNING)

 % \noindent El día acordado para llevar a cabo esta reunión son los {\bf martes} cada dos semanas
 % {\bf a la 1:30pm} en las instalaciones de la ESCOM. El horario fue acordado tomando en cuenta
 % la disponibilidad de todos los miembros del equipo Scrum.
    
    % \begin{quote}
    % {\bf Nota:} En caso de que, por algun evento extraordinario, no se pueda
    %            llevar a cabo el Sprint Planning este reunión se reagendará para
    %            que ocurra lo más pronto posible.
    % \end{quote}

% EVENTOS - DAILY SCRUM

 % En la tabla \ref{tbl:daily} muestra los días acordados, lugar y hora pre-establecidos para la reunión.

    % \addtable{|c|c|c|}{tbl:daily}{
    %    {\bf Día de Trabajo} & {\bf Lugar} & {\bf Hora Inicio} \\\hline
    %    Lunes     & ESCOM Sala 21 N & 10:00am \\\hline
    %    Martes    & ESCOM Sala 21 N & 10:00am \\\hline
    %    Miércoles & ESCOM Sala 21 N & 10:00am \\\hline
    %    Jueves    & ESCOM Sala 21 N & 10:00am \\\hline
    %    Viernes   & ESCOM Sala 21 N & 10:00am \\\hline
    %    Domingo   & -               & 12:00pm \\\hline
    %}{Horario de Daily Scrum}
    
 % \noindent Debido a la dificultad de hacer coincidir los horarios del equipo de desarrollo  con los
 % del {\it Product Owner}, cuando se requiera de sus decisiones, opinión o retroalimentación se le contactará
 % a través de mensajería instantánea.

% SPRINT REVIEW

 % \noindent Debido a que en este proyecto, los stakeholders y el equipo de scrum tienen distintos horarios
 % de disponibilidad, la revisión del {\it sprint} se divide en cuatro fases, aplicando la primer fase a los sprints
 % impares y las cuatro fases para sprints pares. Las fases de describen a continuación:
 
 %   \begin{quote}
 %   \begin{itemize}
 %   \item[\it Fase 1]
 %       Consiste en realizar una primer reunión con el equipo scrum para obtener una
 %       retroalimentación y revisar el incremento entregado.

 %   \item[\it Fase 2]
 %       En esta fase el equipo de desarrollo tiene reuniones con los Stakeholders con la
 %       finalidad de obtener retroalimentación y observaciones acerca de la forma de
 %       trabajo y del incremento.

 %   \item[\it Fase 3] 
 %       En esta fase los miembros del equipo scrum revisarán las observaciones y
 %       comentarios de los {\it Stakeholders} para saber cuales proceden.

 %   \item[\it Fase 4]
 %       Se avisa a los Stakeholders acerca de cuales observaciones procedieron y cuales no.\\
 %   \end{itemize}    
    
 %   {\bf Nota:} Las reuniones de la fase 2, dependen de la disponibilidad que cada {\it stakeholder} tenga,
 %               en caso de que ningún stakeholder tenga disponibilidad para llevar a cabo la fase 2,
 %               el proceso de la revisión del {\it Sprint} terminará.
 %   \end{quote}

% ARTEFACTOS 

% PRODUCT BACKLOG

 % \noindent Debido a que el proyecto requería una etapa de investigación, se optó por tener dos tipos
 % de {\it items} en el product backlog, los items de documentación/preparación del proyecto  y los {\it items}
 % para desarrollo del mismo.\\
    
 % \noindent{\bf Items de Documentación}\\
 % Los items de preparación del proyecto y documentación deben ser especificados
 % mediante los atributos presentes en la tabla \ref{attrPBpre}:
    
 %   \addtable{|l|l|}{attrPBpre}{
 %       {\bf Atributo} & {\bf Descripción}                                                             \\\hline
 %       id           &  Es una identificador de la forma ``Ax'' donde {\it x} es un número consecutivo \\\hline
 %       nombre       &  Nombre representativo de la actividad                                          \\\hline
 %       descripción  &  Detalle de lo que hay que hacer para llevar a cabo esta actividad.             \\\hline
 %       sprint       &  Indica el número de Sprint al cual ha sido asignada esta tarea.                \\\hline
 %       %estado      & Indica el estado ({\it por hacer, en proceso o concluida} de una actividad. \\\hline
 %       %estimación  & Especifica el periodo de tiempo estimado para la liberación de dicha actividad. \\\hline
 %   }{Atributos de los Items del P.B de Documentación}


 %   \noindent{\bf Items de Desarrollo del Proyecto}\\
 %   Describen las características del software que se desarrollará, estos items deben
 %   ser redactados de manera objetiva y como requerimientos del sistema, y deben contener
 %   los atributos presentes en la tabla \ref{attrPB}:
 %   
 %   \addtable{|l|p{0.62\textwidth}|}{attrPB}{
 %       {\bf Atributo} & {\bf Descripción}\\\hline
 %       id           &  Es una identificador de la forma '{\bf RFx}' o '{\bf RNFx}' para requerimientos    \par
 %                       funcionales y no funcionales respectivamente. {\it x} es un número consecutivo     \\\hline
%
 %       nombre       &  Nombre representativo del requerimiento del sistema.                               \\\hline
 %       descripción  &  Descripción concisa y objetiva acerca del requerimiento.                           \\\hline
 %       prioridad    &  Indica la prioridad de un requerimiento, los valores posibles son:                 \par
 %                       \qquad MA (muy alta), A (alta), M (Media), B (baja) y MB (muy baja)                \\\hline

 %       sprint       &  Indica el número de Sprint al cual ha sido asignado este requerimiento.            \\\hline
 %       tipo         &  Tipo de requerimiento no funcional según la clasificación propuesta por Frank Tsui \\\hline
        %estimación  & Especifica el periodo de tiempo estimado para la liberación de dicho requerimiento. \\\hline
 %   }{Atributos de los Items del P.B de Desarrollo del Proyecto}
    
  %  \begin{quote}
  %  {\bf Nota:} El atributo {\it Sprint} debe estar presente en todos los items correspondientes
  %              al sprint corriente y a los sprints anteriores a este. El atributo {\it Sprint}
  %              puede no estar presente en los items que no han sido vinculados a un Sprint. 
  %  \end{quote}
 

% SPRINT BACKLOG

 % \noindent Conforme los {\it items} del {\it product backlog} vayan siendo seleccionados para
 % tratarse en un sprint, se les añadirá una etiqueta que indique a qué sprint pertenecen.
    
    % \addtable{|l|l|}{SBItems}{
    %    {\bf Atributo} & {\bf Descripción}\\\hline
    %    sprint   &  Indica el número de Sprint al cual ha sido asignado el item.             \\\hline
    %    pruebas  &  (Opcional) Sentencia de cómo se evaluara que dicho ítem esté completado. \\\hline
    %
    % }{Atributos del Sprint Backlog }

