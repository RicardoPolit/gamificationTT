\section{Usuarios}
\label{analisis:usuarios}


 La división propuesta de Richard Bartle \cite{TiposDeUsuario}, los divide en 4 grupos, triunfadores,
 exploradores, socializadores y asesinos. En nuestro caso no nos enfocaremos en los exploradores
 puesto que los cursos de moodle son lineales.
    
    \begin{itemize}
    \item Los triunfadores son aquellos que les gusta recibir premios, en nuestro caso
          los logros que tenemos contemplados.

    \item Socializadores, les gusta trabajar en equipo, por lo tanto nuestra propuesta
          de tener grupos que estén compitiendo unos con otros va enfocada con este tipo
          de usuario.

    \item Los asesinos son los usuarios competidores que sienten motivación al ganarle
          a otras personas, las competencias 1 vs 1 y los otros tipos que tenemos
          contemplados motivarían a este tipo de usuario.
    \end{itemize}
    
 Es importante recalcar que en este momento las divisiones de los usuarios son mapeados en los principios
 del marco de trabajo Octalysis de la siguiente manera en el cuadro \ref{table:usuariosvprincipios} según
 el autor de Octalysis\cite[p. 414]{libro2}.
    
    \begin{table}[h!]
    \centering
    \begin{tabular}{|c|c|} \hline
        Triunfadores & Principio II, Principio VI \\ \hline
        Socializadores &  Principio V, Principio III, Principio VII\\\hline
        Asesinos & Principio II, Principio V, Principio VIII, Principio IV \\\hline
    \end{tabular}
    \caption{Tabla de mapeo de tipos de usuario y principios de Octalysis}
    \label{table:usuariosvprincipios}
    \end{table}
