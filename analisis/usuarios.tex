\section{Usuarios}
\label{analisis:usuarios}

 En esta sección se presentan los actores a los que va destinada la propuesta de
 solución, así como sus responsabilidades y perfil recomendado. Posteriormente se
 presentan por cada módulo el conjunto de funcionalidades que se le brindarán a
 cada actor especificando a que módulo pertenecen las funcionalidades.

 % =====================================
 %    A D M I N I S T R A D O R
 % =====================================

    \begin{actor}{aAdmininstrador}{Administrador del sitio}{%
        El adminsitrador del sitio es la persona con mayor jerarquía respecto
        a los permisos y funcionalidades que brinda moodle, desde configurar la
        visualización de la página principal del sitio hasta editar las políticas
        de seguridad. A continuación se describen las responsabilidades y cualidades
        que debe tener un administrador.}

    \item[Responsabilidades:] \hfill
        \begin{itemize}
        \item Instalar/desinstalar plugins
        \item configuración general de los plugins
        \end{itemize}

    \item[Perfil:] \hfill
        \begin{itemize}
        \item Contar con experiencia administrando moodle.
        \item Conocimientos de permisos en LAMP (Linux, apache, MySQL and PHP).
        \item Conocimientos básicos de gamificación.\\
        \end{itemize}
    \end{actor}

 \noindent
 Las funcionalidades extras que se le brindarán al adminitrador son las siguientes:

    \begin{quote}

    {\bf Del módulo de experiencia:}
        \begin{itemize}
        \item Establecer la cantidad de experiencia que brindarán los cursos.
        \item Especificar la experiencia correspondiente al primer nivel.
        \item Elegir el tipo de incremento de la cantidad de experiencia en los niveles.
        \item Especificar el factor o valor de incremento asociado al tipo de incremento.
        \end{itemize}

    % TODO expecificar las funcionalidades de los otros módulos
    \end{quote}

 % =====================================
 %    P R O F E S O R
 % =====================================

    \begin{actor}{aProfesor}{Profesor}{%
        ...}

    \item[Responsabilidades:] \hfill
        \begin{itemize}
        \item ...
        \end{itemize}

    \item[Perfil:] \hfill
        \begin{itemize}
        \item ...
        \end{itemize}
    \end{actor}

 \noindent
 Las funcionalidades extras que se le brindarán al profesor son las siguientes:

    \begin{quote}

    {\bf Del módulo de experiencia:}
        \begin{itemize}
        \item ...
        \end{itemize}

    % TODO expecificar las funcionalidades de los otros módulos
    \end{quote}

 % =====================================
 %    A L U M N O
 % =====================================

    \begin{actor}{aAlumno}{Alumno}{%
        ...}

    \item[Responsabilidades:] \hfill
        \begin{itemize}
        \item ...
        \end{itemize}

    \item[Perfil:] \hfill
        \begin{itemize}
        \item ...
        \end{itemize}
    \end{actor}

 \noindent
 Las funcionalidades extras que se le brindarán al profesor son las siguientes:

    \begin{quote}

    {\bf Del módulo de experiencia:}
        \begin{itemize}
        \item ...
        \end{itemize}

    % TODO expecificar las funcionalidades de los otros módulos
    \end{quote}

 La división propuesta de Richard Bartle \cite{TiposDeUsuario}, los divide en 4 grupos, triunfadores,
 exploradores, socializadores y asesinos. En nuestro caso no nos enfocaremos en los exploradores
 puesto que los cursos de moodle son lineales.
    
    \begin{itemize}
    \item Los triunfadores son aquellos que les gusta recibir premios, en nuestro caso
          los logros que tenemos contemplados.

    \item Socializadores, les gusta trabajar en equipo, por lo tanto nuestra propuesta
          de tener grupos que estén compitiendo unos con otros va enfocada con este tipo
          de usuario.

    \item Los asesinos son los usuarios competidores que sienten motivación al ganarle
          a otras personas, las competencias 1 vs 1 y los otros tipos que tenemos
          contemplados motivarían a este tipo de usuario.
    \end{itemize}
    
 Es importante recalcar que en este momento las divisiones de los usuarios son mapeados en los principios
 del marco de trabajo Octalysis de la siguiente manera en el cuadro \ref{table:usuariosvprincipios} según
 el autor de Octalysis\cite[p. 414]{Octalysis}.
    
    \begin{table}[h!]
    \centering
    \begin{tabular}{|c|c|} \hline
        Triunfadores & Principio II, Principio VI \\ \hline
        Socializadores &  Principio V, Principio III, Principio VII\\\hline
        Asesinos & Principio II, Principio V, Principio VIII, Principio IV \\\hline
    \end{tabular}
    \caption{Tabla de mapeo de tipos de usuario y principios de Octalysis}
    \label{table:usuariosvprincipios}
    \end{table}


\section{Características principales}

 % TAMBIEN SE DESEA QUE SEAN VARIAS OPCIONES PARA QUE LOS PROFES TENGAN UN ABANICO
 % AMPLIO DE OPCIONES

 Las características más
 importantes que debe brindar nuestra propuesta de solución son las siguientes:

    \begin{multicols}{2}
    \begin{itemize}
    \itemx{ Altamente configurable }

    \item[] Hace referencia a que se debe proporcionar al administrador, profesores
            y alumnos la flexbilidad para que puedan configurar los valores por defecto
            y visualización de la herramienta que se desarrollará.

    \itemx{ Escoge que quieres incluir }

    \item[] Una de las principales caracteristicar que deben proporcionar estos
            componentes hacia los profesores es que les permitan escoger que elementos
            de gamificación desean incluir en sus cursos y cuales no.\\

    %\vfill\null
    %\columnbreak
    %\vfill\null

    \itemx{ Bajo acoplamiento }

    \item[] Respecto el poder elegir que características se desean incluir y cuales no,
            implica que los componentes que brindan dichas caracterñisticas hayan sido
            diseñados para que puedan trabajar tanto independientemente como
            colaborativamente.

    \itemx{ Comunicación entre módulos }

    \item[] Que las cosas que pasen en módulo puedan permitir realizar acciones con
            otros módulos para que cuando se deseém ocupar ambos, estos puedan trabajar
            en conjunto.

    \end{itemize}
    %\vfill\null
    \end{multicols}
