\chapter{Análisis general}
\label{ch:analisis}

    \section{Adaptación de la metodología}

% INTRODUCCIÓN

 % la forma en que han sido configurado el marco de trabajo scrum para este proyecto.\\

% ROLES 
% ROLES - PRODUCT OWNER

 % \noindent Para que las funciones del Product Owner sean exitosas, la organización debe respetar sus
 % decisiones. Nadie puede forzar al Development Team para trabajar en un conjunto distinto de requerimientos
 % establecidos en el Product Backlog.

 % \noindent En este proyecto el rol del Product Owner lo llevarán a cabo los directores del trabajo terminal:
     
    % \begin{quote}
    % \begin{itemize}
    %    \item M. en C. Sandra Ivette Bautista, y el
    %    \item M. en C. Edgar Armando Catalán
    % \end{itemize}
    % \end{quote}
                                         
 % \noindent A pesar de que en la guía oficial de Scrum \cite{TheScrumGuide} se especifica que el {\it Product
 % Owner} debe ser una persona, se decidió que este rol fuera llevado a cabo mediante los directores del trabajo
 % terminal con la premisa de que para la toma de decisiones ambos directores deben de estar de acuerdo.

% ROLES - EQUIPO DE DESARROLLO

 % durante el desarrollo de este proyecto el equipo de desarrollo estará conformado por:

 % \noindent El tamaño del equipo debe ser lo suficientemente pequeño para permanecer ágil y lo
 % suficientemente grande para realizar entregas significativas al final de cada {\em Sprint}.\\
 % Un equipo con menos de tres miembros disminuiría la interacción y por lo tanto la productividad,
 % por otro lado, con más de nueve miembros se requiere mayor coordinación con más complejidad.

 % \noindent El equipo de desarrollo estará conformado por los estudiantes que cursan el trabajo terminal:
        
    % \begin{quote}
    % \begin{itemize}
    %    \item David Flores Casanova
    %    \item Ricardo Naranjo Polit
    %    \item Daniel Isaí Ortega Zúñiga
    % \end{itemize}
    % \end{quote}

% ROLES - SCRUM MASTER

 % \noindent El rol del {\it Scrum Master} durante este proyecto se llevará a cabo mediante
 % dos personas con la finalidad de dividir las responsabilidades y no sobrecargar de trabajo
 % a una persona. El {\it Scrum Master} estará conformado por:

    % \begin{quote}
    % \begin{itemize}
    %    \item M. en C. Edgar Armando Catalán {\it (Responsabilidades hacia el Product Owner)}
    %    \item Daniel Isaí Ortega {\it(Responsabilidades hacia el equipo de desarrollo)}
    % \end{itemize}                        
    % \end{quote}

% ROLES - STAKEHOLDERS

 % Los Stakeholders son personas externas al equipo Scrum con un interés y/o conocimientos
 % específicos del producto \cite{ScrumGlosary}. 

 % Durante el desarrollo del trabajo terminal
 % se consideran a los sinodales cómo los Stakeholders oficiales, los cuales son listados a
 % continuación:

    % \begin{quote}
    % \begin{itemize}
    %    \item Dra. Fabiola Ocampo Botello
    %    \item M. en C. María del Socorro Téllez Reyes
    %    \item M. en C. José David Ortega Pacheco
    % \end{itemize}
    % \end{quote}

% EVENTOS
% EVENTOS - SPRINTS

 % \begin{quote}
 %   \noindent Para este proyecto los Sprints están configurados a una duración de 14 días con una
 %   estimación de 18 iteraciones, la duración de dos semanas se estableció con el propósito de:
    
    % \begin{itemize}
    %    \item Incrementar la retroalimentación y detectar los impedimentos
    %          en la forma de trabajo lo más pronto posible, y

    %    \item Realizar incrementos más cortos y continuos considerando que el
    %          equipo de desarrollo está conformado por tres integrantes.
    % \end{itemize} 
 % \end{quote}

% EVENTOS - PLANEACiÓN (SPRINT PLANNING)

 % \noindent El día acordado para llevar a cabo esta reunión son los {\bf martes} cada dos semanas
 % {\bf a la 1:30pm} en las instalaciones de la ESCOM. El horario fue acordado tomando en cuenta
 % la disponibilidad de todos los miembros del equipo Scrum.
    
    % \begin{quote}
    % {\bf Nota:} En caso de que, por algun evento extraordinario, no se pueda
    %            llevar a cabo el Sprint Planning este reunión se reagendará para
    %            que ocurra lo más pronto posible.
    % \end{quote}

% EVENTOS - DAILY SCRUM

 % En la tabla \ref{tbl:daily} muestra los días acordados, lugar y hora pre-establecidos para la reunión.

    % \addtable{|c|c|c|}{tbl:daily}{
    %    {\bf Día de Trabajo} & {\bf Lugar} & {\bf Hora Inicio} \\\hline
    %    Lunes     & ESCOM Sala 21 N & 10:00am \\\hline
    %    Martes    & ESCOM Sala 21 N & 10:00am \\\hline
    %    Miércoles & ESCOM Sala 21 N & 10:00am \\\hline
    %    Jueves    & ESCOM Sala 21 N & 10:00am \\\hline
    %    Viernes   & ESCOM Sala 21 N & 10:00am \\\hline
    %    Domingo   & -               & 12:00pm \\\hline
    %}{Horario de Daily Scrum}
    
 % \noindent Debido a la dificultad de hacer coincidir los horarios del equipo de desarrollo  con los
 % del {\it Product Owner}, cuando se requiera de sus decisiones, opinión o retroalimentación se le contactará
 % a través de mensajería instantánea.

% SPRINT REVIEW

 % \noindent Debido a que en este proyecto, los stakeholders y el equipo de scrum tienen distintos horarios
 % de disponibilidad, la revisión del {\it sprint} se divide en cuatro fases, aplicando la primer fase a los sprints
 % impares y las cuatro fases para sprints pares. Las fases de describen a continuación:
 
 %   \begin{quote}
 %   \begin{itemize}
 %   \item[\it Fase 1]
 %       Consiste en realizar una primer reunión con el equipo scrum para obtener una
 %       retroalimentación y revisar el incremento entregado.

 %   \item[\it Fase 2]
 %       En esta fase el equipo de desarrollo tiene reuniones con los Stakeholders con la
 %       finalidad de obtener retroalimentación y observaciones acerca de la forma de
 %       trabajo y del incremento.

 %   \item[\it Fase 3] 
 %       En esta fase los miembros del equipo scrum revisarán las observaciones y
 %       comentarios de los {\it Stakeholders} para saber cuales proceden.

 %   \item[\it Fase 4]
 %       Se avisa a los Stakeholders acerca de cuales observaciones procedieron y cuales no.\\
 %   \end{itemize}    
    
 %   {\bf Nota:} Las reuniones de la fase 2, dependen de la disponibilidad que cada {\it stakeholder} tenga,
 %               en caso de que ningún stakeholder tenga disponibilidad para llevar a cabo la fase 2,
 %               el proceso de la revisión del {\it Sprint} terminará.
 %   \end{quote}

% ARTEFACTOS 

% PRODUCT BACKLOG

 % \noindent Debido a que el proyecto requería una etapa de investigación, se optó por tener dos tipos
 % de {\it items} en el product backlog, los items de documentación/preparación del proyecto  y los {\it items}
 % para desarrollo del mismo.\\
    
 % \noindent{\bf Items de Documentación}\\
 % Los items de preparación del proyecto y documentación deben ser especificados
 % mediante los atributos presentes en la tabla \ref{attrPBpre}:
    
 %   \addtable{|l|l|}{attrPBpre}{
 %       {\bf Atributo} & {\bf Descripción}                                                             \\\hline
 %       id           &  Es una identificador de la forma ``Ax'' donde {\it x} es un número consecutivo \\\hline
 %       nombre       &  Nombre representativo de la actividad                                          \\\hline
 %       descripción  &  Detalle de lo que hay que hacer para llevar a cabo esta actividad.             \\\hline
 %       sprint       &  Indica el número de Sprint al cual ha sido asignada esta tarea.                \\\hline
 %       %estado      & Indica el estado ({\it por hacer, en proceso o concluida} de una actividad. \\\hline
 %       %estimación  & Especifica el periodo de tiempo estimado para la liberación de dicha actividad. \\\hline
 %   }{Atributos de los Items del P.B de Documentación}


 %   \noindent{\bf Items de Desarrollo del Proyecto}\\
 %   Describen las características del software que se desarrollará, estos items deben
 %   ser redactados de manera objetiva y como requerimientos del sistema, y deben contener
 %   los atributos presentes en la tabla \ref{attrPB}:
 %   
 %   \addtable{|l|p{0.62\textwidth}|}{attrPB}{
 %       {\bf Atributo} & {\bf Descripción}\\\hline
 %       id           &  Es una identificador de la forma '{\bf RFx}' o '{\bf RNFx}' para requerimientos    \par
 %                       funcionales y no funcionales respectivamente. {\it x} es un número consecutivo     \\\hline
%
 %       nombre       &  Nombre representativo del requerimiento del sistema.                               \\\hline
 %       descripción  &  Descripción concisa y objetiva acerca del requerimiento.                           \\\hline
 %       prioridad    &  Indica la prioridad de un requerimiento, los valores posibles son:                 \par
 %                       \qquad MA (muy alta), A (alta), M (Media), B (baja) y MB (muy baja)                \\\hline

 %       sprint       &  Indica el número de Sprint al cual ha sido asignado este requerimiento.            \\\hline
 %       tipo         &  Tipo de requerimiento no funcional según la clasificación propuesta por Frank Tsui \\\hline
        %estimación  & Especifica el periodo de tiempo estimado para la liberación de dicho requerimiento. \\\hline
 %   }{Atributos de los Items del P.B de Desarrollo del Proyecto}
    
  %  \begin{quote}
  %  {\bf Nota:} El atributo {\it Sprint} debe estar presente en todos los items correspondientes
  %              al sprint corriente y a los sprints anteriores a este. El atributo {\it Sprint}
  %              puede no estar presente en los items que no han sido vinculados a un Sprint. 
  %  \end{quote}
 

% SPRINT BACKLOG

 % \noindent Conforme los {\it items} del {\it product backlog} vayan siendo seleccionados para
 % tratarse en un sprint, se les añadirá una etiqueta que indique a qué sprint pertenecen.
    
    % \addtable{|l|l|}{SBItems}{
    %    {\bf Atributo} & {\bf Descripción}\\\hline
    %    sprint   &  Indica el número de Sprint al cual ha sido asignado el item.             \\\hline
    %    pruebas  &  (Opcional) Sentencia de cómo se evaluara que dicho ítem esté completado. \\\hline
    %
    % }{Atributos del Sprint Backlog }


    % Explicar cómo es que está configurado scrum en este proyecto
    % detallar los entregables y las forma de trabajar.

    % Mencionar que se siguierón los pasos de ''For The Win'' y
    % que cada una de las secciones siguientes está vinculada a
    % los pasos de For The Win.

    \clearpage
    
\section{Módulos identificados}

    % Describir cuales son los módulos que fueron identificados
    % cual es el propósito de cada uno, y de que sub módulos
    % estan constituidos, solo poner la imagen principal.

    \clearpage
    \section{Usuarios}


 La división propuesta de Richard Bartle \cite{TiposDeUsuario}, los divide en 4 grupos, triunfadores,
 exploradores, socializadores y asesinos. En nuestro caso no nos enfocaremos en los exploradores
 puesto que los cursos de moodle son lineales.
    
    \begin{itemize}
    \item Los triunfadores son aquellos que les gusta recibir premios, en nuestro caso
          los logros que tenemos contemplados.

    \item Socializadores, les gusta trabajar en equipo, por lo tanto nuestra propuesta
          de tener grupos que estén compitiendo unos con otros va enfocada con este tipo
          de usuario.

    \item Los asesinos son los usuarios competidores que sienten motivación al ganarle
          a otras personas, las competencias 1 vs 1 y los otros tipos que tenemos
          contemplados motivarían a este tipo de usuario.
    \end{itemize}
    
 Es importante recalcar que en este momento las divisiones de los usuarios son mapeados en los principios
 del marco de trabajo Octalysis de la siguiente manera en el cuadro \ref{table:usuariosvprincipios} según
 el autor de Octalysis\cite[p. 414]{libro2}.
    
    \begin{table}[h!]
    \centering
    \begin{tabular}{|c|c|} \hline
        Triunfadores & Principio II, Principio VI \\ \hline
        Socializadores &  Principio V, Principio III, Principio VII\\\hline
        Asesinos & Principio II, Principio V, Principio VIII, Principio IV \\\hline
    \end{tabular}
    \caption{Tabla de mapeo de tipos de usuario y principios de Octalysis}
    \label{table:usuariosvprincipios}
    \end{table}

    % Detallar el perfil de nuestros usuarios y los beneficios
    % que se brindarán a cada uno de ellos.

    % Además mencionar que se puede brindar soporte a distintos
    % tipos de jugadores a través de los principios de gamificación.
    % Presentar imagen de principios con módulos.

    \clearpage
    
\section{Procesos}

    % Detallar cual es el proceso que se tiene planificado para
    % el uso de nuestra herramienta, describir el procesos desde
    % la instalación y configuración de plugins, hasta la aparición
    % de elementos de gamificación cuando los alumnos realizan una
    % acción.

    % Enaltecer la interoperabilidad.

    \clearpage
    
\section{Relación entre los módulos y los principios de gamificación}

    % Brindar un aspecto más amplio acerca de como se le brinda
    % soporte a distintas cosas a través de los principios del marco
    % de trabajo de octalysis.

    \clearpage
    
\section{Estudio de factibilidad de implementación sobre moodle}

    % Detalla el estudio de fatibilidad que se realizó inicialmente
    % para saber si era viable usar moodle o no.

    \clearpage
    \chapter{Modelo de Dominio de Datos}
\label{ch:dominioDatos}

 Este capítulo describe las decisiones más relevantes tomadas en relación al esquema de
 base de datos de moodle, el modelo de información que le brinda soporte a la persistencia
 de los datos usados y requeridos por los módulos y submódulos previamente establecidos, 
 finalmente se presenta el diccionario de datos que contiene la especificación de los
 atributos de las distintas relaciones requeridas.\\

 \noindent Debido a que el esquema de base de datos de moodle contiene alrededor de 400
 relaciones, solo se hace mención de aquellas relevantes para el desarrollo de este proyecto,
 cabe recalcar que la información recopilada de la base de datos de moodle fue obtenida
 directamente del esquema de base de datos, ya que no existe documentación oficial acerca
 del esquema.

\section{Esquema de la base de datos}

 % DONDE SE DOCUMENTA QUE SE VA A USAR MYSQL ???

 En la figura \ref{fig:EGBD} se muestran aquellas relaciones relevantes para el desarrollo
 de este proyecto, tanto del esquema de la base de datos de moodle cómo de las relaciones
 que fueron requeridas crear, el diagrama presentado contiene las relaciones agrupadas
 de la siguiente forma:\\
  
    \begin{quote}
    \begin{description}
        \item[Relaciones de Moodle] Contiene un subconjunto de las relaciones del esquema de
            base de datos de moodle que son relevantes para el desarrollo de este proyecto.\\

        \item[General] Contiene a todas las relaciones que no pertenecen propiamente a un
            módulo pero que son esenciales para el funcionamiento de los distintos módulos
            propuestos.\\

    \end{description}
    \end{quote}

 \noindent Las demás relaciones se encuentran organizadas en seís grupos correspondientes
 a los módulos de este proyecto (\nameref{mod:exp}, \nameref{mod:recomp}, \nameref{mod:financ},
 \nameref{mod:pers}, \nameref{mod:comp} y \nameref{mod:seguim}).

 \clearpage

 \noindent Todas las relaciones creadas para este proyecto fueron diseñadas tomando en
 consideración que se debía extender el esquema de base de datos sin afectar las distintas
 funcionalidades que brinda de forma nativa. En los casos en los que se requería añadir
 atributos de moodle a una relación en particular, se agregaría una relación con los atributos
 requeridos, más la referencia a la relación del esquema de moodle.

    \addfigure{1}{analisis/diagrams/db}{fig:EGBD}{Esquema de la base de datos}

\clearpage
\section{Diccionario de datos}

 En esta sección se detalla el diccionario de datos del proyecto, en él se describe cada una
 de las entidades pertenecientes al modelo de información, así como el nombre, tipo de dato,
 descripción y restricciones sobre los atributos de las entidades.\\

 A continuación se listan y describen los tipos de
 datos usados en el diccionario.

    \begin{quote}
    \begin{bGlosario}
        \bTerm{tVarchar}{varchar} Cadena de caracteres de longitud variable
        \bTerm{tInt}{int}         Numero enteros incluyendo, positivos y negativos.
    \end{bGlosario}
    \end{quote}

 Además de los tipos de dato, se definen las
 siguientes restriccioness:
    
    \begin{quote}
    \begin{bGlosario}
    
        \bTerm{tRequerido}{Requerido}
            Indica que el dato es obligatorio y no puede existir un registro sin que tenga
            un valor.
            
        \bTerm{tUnico}{Único}
            Indica que que los valores para un atributo o combinacion de estos no se deben
            repetir en los diferentes registros.
            
        \bTerm{tPrimaryKey}{Llave primaria}
            Indica el atributo o combinacion de estos que seran la llave primaria. La llave
            primaria es el identificador unico de cada registro dentro de la tabla, por lo
            que de no puede haber duplicados y es obligatorio que tenga un valor.
            
        \bTerm{tForeignKey}{Llave foranea}
            Indica el atributo o conjunto de estos que son llaver foranea y apuntan a otra
            tabla, esto es, sus valores posibles Sólo pueden ser aquellos que existan en la
            llave primaria a la que apuntan. FK(Tabla.atributo).
            
        \bTerm{tRango}{Rango}(a, b)
            Indica que el numero de valores esta restringido del valor $a$ al $b$.

        \bTerm{tAutoIncrement}{Auto-incremental}
            Indica que el número se incrementa en uno al insertarse un nuevo registro.

        \bTerm{tDefault}{Default} $:\ valor$
            Indica el valor por defecto que tendrá el atributo
            
    \end{bGlosario}
    \end{quote}
    
\clearpage

\section{Relaciones de moodle}

 A continuacíón se presenta la especificación de las relaciones del esquema de base
 de datos de moodle que son relevantes para el desarrollo de los módulos y submódulos
 de proyecto.

    \begin{cdtEntidad}{Mdl-configuration}{Configuración de Plugin}{%
    Es una tabla del nucleo de moodle que almacena todas las configuraciones globales
    relacionadas a los plugins instalados, al iniciar moodle las configuraciones de los
    plugins instalados y habilitados se cargan en memoria.}

	    \brAttr{id}{Id}{tInt}{%
	        Es el dígito que representa el identificador único para una configuración
            específica de un plugin.\par

            \it Restricciones:
            \refElem{tPrimaryKey},
            \refElem{tAutoIncrement}.
        }

        \brAttr{plugin}{Plugin}{tVarchar}{%
            Cadena de caracteres del nombre identificador del plugin al cual pertenece
            la configuración.\par

            \it Restricciones:
            \refElem{tRequerido},
            \refElem{tRango} (0,100),
            \refElem{tUnico}
        }
		
        \brAttr{name}{Nombre}{tVarchar}{%
            Cadena de caracteres que representa el nombre de la configuración de un
            plugin en específico.\par

            \it Restricciones:
            \refElem{tUnico},
            \refElem{tRango} (0,100),
            \refElem{tRequerido}
        }

        \brAttr{value}{Valor}{tVarchar}{%
            Cadena que almacena el valor de una configuración perteneciente a alguno
            de los plugins instalados.\par
            
            \it Restricciones:
            \refElem{tRango} (0,4294967295),
            \refElem{tRequerido}
        }
	%\cdtEntityRelSection
	%\brRel{\brRelComposition}{Ejemplo}{Escribe el enunciado que denota la composición.}
	%\brRel{\brRelAgregation}{Ejemplo}{Tambien puede ser con agregaciones.}
	%\brRel{\brRelGeneralization}{Ejemplo}{De la misma forma con Generalizaciones.}
    \end{cdtEntidad}
    \vspace{-1em}\hfill Nombre en el esquema: {\it mdl\_configuration}\ \ \par%


    \begin{cdtEntidad}{mdl-usuario}{Usuario de moodle}{%
    Es una tabla del núcleo de moodle que contiene toda la información que se
    almacena de los usuarios en la plataforma, independientemente del rol que
    estos contenga, esta relación contiene más de 53 atributos, sin embargo solo
    se detallan aquellos relevantes.}

	    \brAttr{id}{Id}{tInt}{%
	        Es el dígito que representa el identificador único para cada uno 
            de los usuarios en moodle.\par

            \it Restricciones:
            \refElem{tPrimaryKey},
            \refElem{tAutoIncrement}.
        }

    \end{cdtEntidad}
    \vspace{-1em}\hfill Nombre en el esquema: {\it mdl\_configuration}\ \ \par%

\begin{comment}

    \begin{Entidad}{usuario}{%
            \attrM{id\_usuario}
            {Atributo que relaciona un usuario de Moodle con uno nuestro}
                \Titem{ Llave foránea 
                \Titem{ No nulo}
            {General}
            
            \attr{nivel\_actual}
            {Nivel actual del usuario.}
                \Titem{ No nulo.}
            {Experiencia}
            
            \attr{experiencia\_ actual}
            {Experiencia actual del nivel del usuario.}
                \Titem{ Su valor es $ \geq 0$}
                \Titem{ No nulo.}
            {Experiencia}
    \end{Entidad}
    
    \begin{Entidad}
        {alumno}
            \attrG{id\_ usuario}
            {Atributo que relaciona un usuario con un papel de alumno. }
                \Titem{ Llave foránea a \textbf{gmdl\_usuario (mdl\_id\_ usuario)}}
                \Titem{ No nulo}
            {General}
            
            \attrM{id\_ curso}
            {Atributo que relaciona un curso con un papel de alumno.}
                \Titem{ Llave foránea  a \textbf{mdl\_course (id)}.}
                \Titem{ No nulo.}
            {General}
            
            \attr{experiencia\_ total\_ recibida}
            {Toda la experiencia que ha recibido un usuario haciendo las actividades de un determinado curso.}
                \Titem{ Su valor es $ \geq 0$}
                \Titem{ No nulo.}
            {Experiencia}
    \end{Entidad}

 \noindent La combinación de los atributos ( \textbf{gmdl\_id\_usuario } y \textbf{mdl\_id\_curso} ) es un índice único. 
    
    \begin{Entidad}
        {nivel\_categoria\_curso}
            \attrG{id\_ usuario}
            {Atributo que relaciona un usuario con un nivel. }
                \Titem{ Llave foránea  a \textbf{gmdl\_usuario (mdl\_id\_usuario)}}
                \Titem{ No nulo}
            {General}
            
            \attrM{id\_ categoria\_ curso}
            {Atributo que relaciona un nivel con una categoría de curso de Moodle.}
                \Titem{ Llave foránea  a \textbf{mdl\_course\_categories (id)}.}
                \Titem{ No nulo.}
            {General}
            
            \attr{nivel\_actual}
            {Nivel actual del usuario en una categoría de cursos.}
                \Titem{ No nulo.}
            {Experiencia}
            
            \attr{experiencia\_ actual}
            {Número entero  \newline \textbf{Tamaño:}\newline 10 dígitos }
            {Experiencia actual del usuario de su nivel actual en una categoría de cursos.}
                \Titem{ Su valor es $ \geq 0$}
                \Titem{ No nulo.}
            {Experiencia}
    \end{Entidad}

 \noindent La combinación de los atributos ( \textbf{gmdl\_id\_usuario } y \textbf{mdl\_categoria\_curso} ) es un índice único. 
    
    \begin{Entidad}
        {rango\_nivel}
            
            \attr{nombre}
            {Nivel actual del usuario en una categoría de cursos.}
                \Titem{ No nulo.}
            {Experiencia}
            
            \attr{nivel\_inferior}
            {Nivel mínimo requerido que puede tener un usuario para estar dentro del rango.}
                \Titem{ No nulo.}
            {Experiencia}
            
            \attr{nivel\_superior}
            {Nivel máximo que puede tener un usuario para estar dentro del rango.}
                \Titem{ No nulo.}
            {Experiencia}
            
            \attr{imagen}
            {Ruta donde se guarda la imagen asociada al rango.}
                \Titem{ No nulo.}
            {Experiencia}
            
            \attr{mensaje}
            {Mensaje de felicitaciones que se le muestra al usuario al subir de nivel.}
                \Titem{ No nulo.}
            {Experiencia}
            
            \attr{descripcion}
            {Descripción del rango, que se le muestra al usuario al subir de nivel.}
                \Titem{ No nulo.}
            {Experiencia}
            
    \end{Entidad}

    \begin{Entidad}
        {logro}
            \attr{icono}
            {Ruta donde se guardará el icono del logro..}
                \Titem{ No nulo.}
            {Recompensa}
            
            \attr{nombre}
            { Nombre  \textbf{único} del logro.}
                \Titem{ Índice único.}
                \Titem{ No nulo.}
            {Recompensa}
            
            \attr{descripción}
            {Descripción que le indica al usuario, cómo se desbloquea el logro.}
                \Titem{ No nulo.}
            {Recompensa}
            
            \attr{modulo}
            {Nombre del módulo al que pertenece el logro.}
                \Titem{ No nulo.}
            {Recompensa}
            
            \attr{experiencia\_ de\_logro}
            {Experiencia que otorga el logro al ser desbloqueado.}
                \Titem{ No nulo.}
            {Experiencia}
            
            \attr{tipo}
                Caracter que nos indica que tipo de logro es.
                    \newline ''A'': El logro es a nivel curso.
                    \newline ''B'': El logro es a nivel curso y es una advertencia.
                    \newline ''C'': El logro es a nivel plataforma.
                    \newline ''D'': El logro es a nivel plataforma y es una advertencia.
                \Titem{ Sus únicos valores posibles son: \{'A','B','C','D'\}.}
                \Titem{ No nulo.}
            {Recompensa}
            
    \end{Entidad}
    
    \begin{Entidad}
        {evento}
            
            \attr{nombre\_ evento}
            { Nombre  \textbf{único} del evento.}
                \Titem{ Índice único.}
                \Titem{ No nulo.}
            {Recompensa}
            
    \end{Entidad}
    
    \begin{Entidad}
        {evento\_logro}
            
            \attrG{id\_ logro}
            { Atributo que relaciona un logro con una lista de eventos. }
                \Titem{ Llave foránea a \textbf{gmdl\_logro(id)}}
                \Titem{ No nulo}
            {General}
            
            \attrG{id\_ evento}
            { Atributo que relaciona un evento con una lista de logros.}
                \Titem{ Llave foránea a \textbf{gmdl\_evento(id)}}
                \Titem{ No nulo}
            {General}
            
    \end{Entidad}
    
    \noindent La combinación de los atributos ( \textbf{gmdl\_id\_logro } y \textbf{gmdl\_id\_evento } ) es un índice único. 
    
    \begin{Entidad}
        {condicion}
            
            \attr{tabla}
            { Nombre de la tabla, la cual contiene la información para saber si la condición se cumple o no.}
                \Titem{ No nulo.}
            {Recompensa}
            
            \attr{atributo}
            { Nombre del atributo que se usará para ver si la condición se cumple o no.}
                \Titem{ No nulo.}
            {Recompensa}
            
            \attr{expresión}
            { Expresión a usar con el atributo y el valor.}
                \Titem{ No nulo.}
            {Recompensa}
            
            \attr{valor}
            {Valor que establece la referencia para saber si la condición de cumple o no. }
                \Titem{ No nulo.}
            {Recompensa}
            
    \end{Entidad}
    
    \begin{Entidad}
        {condicion\_logro}
            
            \attrG{id\_ logro}
            {Atributo que relaciona a un logro con una lista de condiciones. }
                \Titem{ Llave foránea a \textbf{gmdl\_logro(id)}}
                \Titem{ No nulo}
            {General}
            
            \attrG{id\_ condicion}
            {Atributo que relaciona a una condición con una lista de logros. }
                \Titem{ Llave foránea a \textbf{gmdl\_condicion(id)}}
                \Titem{ No nulo}
            {General}
            
    \end{Entidad}

    \noindent La combinación de los atributos ( \textbf{gmdl\_id\_logro } y \textbf{gmdl\_id\_condicion } ) es un índice único. 
    
    \begin{Entidad}
        {usuario\_logro\_curso}
            
            \attrG{id\_ logro}
            {Atributo que relaciona un logro con usuarios y cursos. }
                \Titem{ Llave foránea a \textbf{gmdl\_logro(id)}.}
                \Titem{ No nulo.}
            {General}
            
            \attrG{id\_ usuario}
            {Atributo que relaciona a un usuario con logros y cursos. }
                \Titem{ Llave foránea a \textbf{gmdl\_usuario(mdl\_ id\_usuario)} .}
                \Titem{ No nulo.}
            {General}
            
            \attrM{id\_ curso}
            {Atributo que relaciona a un curso con usuarios y logros. }
                \Titem{ Llave foránea a \textbf{mdl\_curso(id)}.}
                \Titem{ No nulo.}
            {General}
            
            \attr{cuando}
            {  Fecha que nos indica cuándo el logro fue desbloqueado.        }
                \Titem{ No nulo.}
            {Recompensa}
            
    \end{Entidad}
    
    \begin{Entidad}
        {usuario\_logro\_global}
            
            \attrG{id\_ logro}
            {Número entero positivo. \newline {\bf Tamaño:}\newline 10 dígitos}
            {Atributo que relaciona un logro con usuarios y cursos. }
                \Titem{ Llave foránea a \textbf{gmdl\_logro(id)}.}
                \Titem{ No nulo.}
            {General}
            
            \attrG{id\_ usuario}
            {Atributo que relaciona a un usuario con logros y cursos. }
                \Titem{ Llave foránea a \textbf{gmdl\_usuario(mdl\_ id\_usuario)} .}
                \Titem{ No nulo.}
            {General}
            
            \attr{cuando}
            {  Fecha que nos indica cuándo el logro fue desbloqueado.        }
                \Titem{ No nulo.}
            {Recompensa}
            
    \end{Entidad}

\section{Formas normales}

    %En esta sección se analiza el cumplimiento de las seis formas normales propuestas por Frank Codd y de la forma normal Boyce-Codd, con el propósito a que se quiere reducir en lo mayor posible la redundancia de datos. Si existe alguna forma normal que no sea cumplida, %se analizarán los cambios requeridos para su cumplimiento. En caso de que no se pueda, se especificará el por qué.  
%
    Debido a que las pautas de Moodle entorpecen el diseño de la base de datos, se analizará si afectan en redundancia. Para ello se tomarán en cuenta las 6+2 formas normales con el esquema de la base de datos propuesto.
%   
\subsection*{Primera forma normal}
    
    Cada tabla en un esquema de base de datos se considera en primera forma normal si y solo si, cumple las siguientes condiciones \cite[pág. 154]{libroBaseDeDatosIngles}: 
    
    \begin{enumerate}
        \item Tiene un dominio atómico. Un dominio es atómico si y solo si, cada elemento del dominio es indivisible \cite[pág. 161]{libroBaseDeDatosEspaniol}.
        \item Cada registro debe de tener el mismo número de valores.
        \item Cada registro debe ser único.
    \end{enumerate}  
    
    \noindent Un ejemplo de un dominio no atómico es tener todos los teléfonos de un cliente en un mismo elemento. "\textit{Teléfono\_A},  \textit{Teléfono\_B}".  Haciendo lo anterior el elemento puede albergar más de un valor, haciendo que ya no sea indivisible.\\
    
    \noindent Continuando con el ejemplo anterior, si en lugar de guardar todos los teléfonos en un atributo, se crean más atributos para poder guardar todos los teléfonos del cliente que tiene más teléfonos. Se incumple con el segundo punto de la primera forma normal.\\
    
    
    \noindent \textbf{Se cumple} con esta forma normal en nuestro esquema de la base de datos porqué no existe ningún atributo en ninguna tabla que no sea atómico.
    
    
\subsection*{Segunda forma normal}
    
    Cada tabla en un esquema de base de datos se considera en segunda forma normal si y solo si, cumple las siguientes 2 condiciones \cite[pág. 159]{libroBaseDeDatosIngles}:
    \begin{enumerate}
        \item Cumple con la primera forma normal.
        \item Todo atributo que no forma parte de la clave primaria debe depender funcionalmente de toda la clave primaria.
    \end{enumerate}
    
    \noindent \textbf{Se cumple} con esta forma normal gracias a que no se cuenta con ninguna clave primaria compuesta. Esto gracias a las pautas de moodle, que nos hacen diferenciar un registro únicamente con un número entero positivo. 
    
\clearpage    
\subsection*{Tercera forma normal}
    
    Cada tabla en un esquema de base de datos se considera en tercera forma normal si y solo si, cumple con las siguientes 2 condiciones \cite[pág. 163]{libroBaseDeDatosIngles}:
    
    \begin{enumerate}
        \item Cumple con la segunda forma normal.
        \item No existen dependencias funcionales transitivas a la clave primaria.
    \end{enumerate}
    
    \noindent \textbf{No se cumple} con esta forma normal en las siguientes tablas de la base de datos.
    \begin{itemize}
        \item \textbf{gmdl\_usuario}
        \item \textbf{gmdl\_alumno}
        %\item \textbf{gmdl\_logro}
        %\item \textbf{gmdl\_condicion}
        %\item \textbf{gmdl\_usuario\_logro\_curso}
        %\item \textbf{gmdl\_usuario\_logro\_global}
        %\item \textbf{gmdl\_evento\_logro}
        %\item \textbf{gmdl\_condicion\_logro}
    \end{itemize}
  
    \noindent Se tiene el mismo problema en cada una de las tablas anteriores, porque en sus atributos contienen una llave candidata que terminó no formando parte de la llave primaria. Gracias a esto se crea una dependencia funcional a la llave candidata y la llave candidata depende funcionalmente de la llave primaria, generando así el no cumplimiento de esta forma normal.
    
    \noindent Dicho problema se puede solventar tomando en consideración la forma normal de Boyce-Codd, es por eso que no se realizarán cambios a dichas tablas de la base de datos.
 
\subsection*{Forma normal de Boyce-Codd (FNBC)}
    
    Cada tabla en un esquema de base de datos se considera en la forma normal de Boyce-Codd si y solo si, cumple con las siguientes condiciones \cite[pág. 168]{libroBaseDeDatosIngles}:
    
    \begin{enumerate}
        \item Todos los atributos no claves de la tabla deben depender funcionalmente de toda una clave candidata.
        \item Toda dependencia funcional de la tabla debe ser hacia una clave candidata.
    \end{enumerate}
    
    \noindent Esta forma normal extiende las anteriores formas normales diciendo que en una entidad puede haber más de una clave candidata y que todos los atributos de esa entidad deben depender de una de esas claves. Gracias a esto se considera a la forma normal de Boyce-Codd como una alternativa a la segunda y tercer forma normal.\\
    
    
    \noindent \textbf{Se cumple} con la forma normal de Boyce-Codd. Si recordamos las tablas que no cumplieron con la tercera forma normal, tienen el mismo problema de tener una dependencia funcional hacia la llave candidata, y con esta forma normal no existe ese problema. \\
    
     
     
    
\clearpage    
\subsection*{Cuarta forma normal}
    
    Cada tabla en un esquema de base de datos se considera en cuarta forma normal si y solo si, cumple con las siguientes 2 condiciones \cite[pág. 182]{libroBaseDeDatosIngles}:
    
    \begin{enumerate}
        \item Cumple con la tercera forma normal o con la forma normal de Boyce-Codd.
        \item No existen dependencias multivaloradas.
    \end{enumerate}
    
    \noindent Una dependencia multivalorada $ A \twoheadrightarrow B$ se cumple si en una relación legal \textit{r(R)}, para todo par de tuplas $t_1$ y $t_2$    tales que $t_1[A] = t_2[A]$, existen unas tuplas $t_3$ y $t_4$ de \textit{r} tales que \cite[pág. 180]{libroBaseDeDatosEspaniol}:
    
    \begin{center}
           $t_1[A] = t_2[A] = t_3[A] = t_4[A]$\\
           $t_3[B] = t_1[B]$\\
           $t_3[R - B] = t_2[R - B]$\\
           $t_4[B] = t_2[B]$\\
           $t_4[R - B] = t_1[R - B]$
    \end{center}
    
    \noindent En otras palabras, si existe una relación ternaria cuyas claves foráneas A, B y C, donde C y B están relacionadas con A, pero son independientes entre sí, dicha relación tiene una dependencia multivalorada \cite[pág. 118]{libroBaseDeDatosInglesCuarteEnAdelante}.\\
    
    \noindent \textbf{Se cumple} con esta forma normal, debido a que no se cuentan con relaciones ternarias en la base de datos.
    
\end{comment}

\begin{comment}
    \noindent Si se ve nuestro esquema de la base de datos, solo se cuenta con una relación que contempla 3 tablas que es \textbf{gmdl\_usuario\_logro\_curso}. Sus claves foráneas son \textbf{gmdl\_id\_curso} que hace referencia a un curso, \textbf{gmdl\_id\_logro} que hace referencia a un logro y \textbf{gmdl\_id\_usuario} que hace referencia a un usuario.\\
    
    \noindent En dicha tabla, las claves se relacionan de la siguiente manera:
    \begin{itemize}
        \item Un usuario desbloquea logros de un curso.
        \item Un curso cuenta con logros que fueron desbloqueados por usuarios.
        \item Un logro fue desbloqueado por usuarios en uno o más cursos.
    \end{itemize}
    
    \noindent \textbf{Se cumple} con la cuarta forma normal. Observando las relaciones anteriores podemos observar que no existe una dependencia multivalorada porque cada atributo se relaciona/depende de los otros 2.
\end{comment}    
    
 
\begin{comment}
\subsection*{Quinta forma normal}

    Cada tabla en un esquema de base de datos se considera en quinta forma normal si y solo si, cumple con las siguientes 2 condiciones \cite[pág. 124]{libroBaseDeDatosInglesCuarteEnAdelante}:
    
    \begin{enumerate}
        \item Cumple con la cuarta forma normal.
        \item Todas las dependencias de unión ''JOIN'' están implicadas por las claves candidatas.
    \end{enumerate}
    
    \noindent Una dependencia de unión ''JOIN'' ocurre cuando una tabla puede volver a ser formada correctamente (refiriéndonos a que los registros permanezcan intactos) uniendo 2 o más tablas cuyos atributos sean de la tabla original \cite[pág. 124]{libroBaseDeDatosInglesCuarteEnAdelante}.\\
    
    \noindent Respecto a que las uniones ''JOIN'' estén implicadas por las claves candidatas, se refiere a que los atributos que se utilicen como pivote (para hacer dicha separación) no sean otros que los que son claves candidatas.\\
    
\end{comment}
    
\begin{comment}
    \noindent  La única tabla que puede llegar a darnos problemas con esta forma normal es la misma que se analizó en la cuarta forma normal. Para analizar si esta tabla (\textbf{gmdl\_usuario\_logro\_curso}) cumple con esta forma tenemos que listar sus atributos.
    \begin{itemize}
        \item id
        \item gmdl\_id\_logro
        \item gmdl\_id\_usuario
        \item gmdl\_id\_curso
        \item cuando
    \end{itemize}
    
    \noindent  Las únicas 2 formas en las que se puede dividir la anterior tabla, conservando los mismos registros son las siguientes:
    \begin{multicols}{2}
    
    \noindent Utilizando como pivote el atributo ''id''. 
    \begin{itemize}
        \item A1(id, gmdl\_id\_usuario)
        \item A2(id, gmdl\_id\_logro)
        \item A3(id, gmdl\_id\_curso)
        \item A4(id, cuando)
    \end{itemize}
    \noindent \textbf{Nota:} Se pueden desarrollar distintas combinaciones con lo anterior, sin embargo, siempre se usa como pivote al atributo ''id'' es por eso que no se contemplan.
    \columnbreak
    
    \noindent Utilizando como pivote la llave candidata ''gmdl\_id\_usuario, gmdl\_id\_logro , gmdl\_id\_curso''.
    \begin{itemize}
        \item B1(id, gmdl\_id\_usuario, gmdl\_id\_logro , gmdl\_id\_curso)
        \item B2(gmdl\_id\_usuario, gmdl\_id\_logro , gmdl\_id\_curso, cuando)
    \end{itemize}
    
    \end{multicols}
\end{comment}    
    
\begin{comment}
    \noindent \textbf{Se cumple} con esta forma normal, debido a que ninguna de las tablas en el esquema puede ser dividida sin contemplar como pivote a una clave candidata.
    
    
\clearpage
\subsection*{Sexta forma normal}
    
    No se cuenta con una definición formal de esta forma normal.\\
    
    \noindent La sexta forma normal trata de datos que son temporales, refiriéndose a que estos pueden cambiar en un futuro. Siendo más específicos, si la base de datos necesita llevar un historial para poder hacer estadísticas, dichas estadísticas no deberían presentar diferentes resultados sin importar cuándo se consulten \cite[pág. 125-126]{libroBaseDeDatosInglesCuarteEnAdelante}.\\
    
    \noindent Lo anterior se puede ver más a detalle con el ejemplo de; ¿cuánto se ha vendido el mes ''X''?.
    
    \noindent Si el único atributo que se tiene para calcular lo anterior es el ''precio'' en una tabla ''producto''. Esto incumple la sexta forma normal, ya que, cuando el precio de dicho producto cambie, la consulta para determinar cuánto se ha ganado el mes ''X'' con ese producto también cambiaría.\\
    
    
    \noindent \textbf{Se cumple} con la sexta forma normal, debido a que no tenemos datos temporales que puedan llegar a afectar de esa forma a nuestra interpretación de los datos. \\
    
    \noindent Los únicos datos temporales que se tienen son; El nivel\_actual y la experiencia\_actual tanto del usuario como del tipo de categoría del curso. Ambos en sus sendas tablas \textbf{gmdl\_usuario} y \textbf{gmdl\_nivel\_categoria\_curso}
    
    
\subsection*{Forma normal de dominio clave}

    Una tabla cumple con la forma normal de dominio/clave si todas las restricciones que se tienen en los datos están asociados con una clave o un dominio \cite[p. 193]{libroBaseDeDatosIngles}.
    
    \begin{itemize}
        \item Un dominio es cualquier limitación que se tengan en los datos para ser guardados en una cierta columna en una base de datos. Esto refiriéndonos a las limitaciones como las llaves foráneas o el tipo de dato.
        \item La clave es una clave única.
    \end{itemize}
    
    \noindent Esta forma normal es considerada como la perfecta forma normal.\\
    
    \noindent \textbf{No se cumple} con esta formal, gracias a los atributos nivel\_actual y la experiencia\_actual tanto del usuario como del tipo de categoría del curso. Ambos en sus sendas tablas \textbf{gmdl\_usuario} y \textbf{gmdl\_nivel\_categoria\_curso}.\\ 
    
    \noindent El atributo experiencia\_actual tiene un límite superior dependiendo del valor del atributo nivel\_actual. La forma para reparar este problema es creando una tabla ''nivel'', sin embargo, se desea que los niveles sean un infinito simbólico.
    
    \noindent \textbf{No se cumple} con esta forma normal en la tabla \textbf{gmdl\_condicion} debido a que  los valores del atributo ''atributo'' dependen del valor del atributo ''tabla''.
    
    \noindent Se considera que el incumplimiento anterior no será atendido, esto por que para resolverlo se necesitaría crear 2 tablas extras que contengan todas las tablas relacionadas con Moodle y todos los atributos asociados con esas tablas. Estas tablas actuarían como un catálogo para la tabla \textbf{gmdl\_condicion} y se tendrían que estar actualizando si se quiere que los plugins a desarrollar tengan compatibilidad con versiones posteriores de Moodle.
    
    \noindent Otro motivo por el cual no se atiende es que los datos de la tabla \textbf{gmdl\_condicion} nunca cambiarán una vez que se haya creado la base de datos y llenado los valores por defecto.
\end{comment}



% ==================================================================
%   La lista de configuraciones guardadas no pertenece a analisis
%   sino a diseño.
%                   To Do: Mover a diseño en el módulo de experiencia
% ===================================================================
    
    %\noindent Por el momento se están guardando las siguientes configuraciones:
    %\begin{center}
    %\begin{tabular}{| m{0.25\textwidth} | m{0.30\textwidth} | m{0.15\textwidth} | m{0.18\textwidth}|}\hline
        %{\bf Nombre} & {\bf Descripción} & {\bf Módulo} & {\bf Sub-módulo}  \\\hline
        
        %Modulo de exp. activo &
        %    Bandera que nos indica si el módulo está activado o no &
        %    Experiencia & Esquema de \newline experiencia\\\hline 
     
        %Tipo de incremento &
        %    Si el incremento de experiencia necesaria por nivel es lineal o porcentual &
        %    Experiencia & Esquema de \newline experiencia \\\hline 
         
        %Cantidad de incremento &
        %    Número decimal que indica cuánto se incrementa la experiencia necesaria por cada nivel &
        %    Experiencia & Esquema de \newline experiencia \\\hline
        
        %Experiencia por actividad &
        %    Cantidad de experiencia que da cualquier actividad al ser completada &
        %    Experiencia & Esquema de \newline experiencia \\\hline 
        
        %Experiencia del nivel 1 &
        %    Cantidad de experiencia asociada al nivel 1 &
        %    Experiencia & Esquema de \newline experiencia \\\hline 
        
        %Nombre del nivel &
        %    Nombre que reciben los niveles por defecto. &
        %    Experiencia & Niveles \\\hline 
        
        %Mensaje de felicitaciones &
        %    Mensaje que aparece cuando un usuario sube de nivel. &
        %    Experiencia & Niveles \\\hline 
        
        %Descripción del rango &
        %    Mensaje que aparece por defecto, si el nivel no pertenece a ningún rango. &
        %    Experiencia & Niveles \\\hline 
            
        %Imagen del nivel &
        %    Imagen del nivel por defecto (Se almacena la ruta de la imagen, después de haber sido copiada). &
        %    Experiencia & Niveles \\\hline 
        
        %Color del número del nivel &
        %    Color que tendrá el número del nivel por defecto. &
        %    Experiencia & Niveles \\\hline 
        
        %Color de la barra de progreso &
        %    Color que tendrá la barra de progreso de los niveles por defecto. &
        %    Experiencia & Niveles \\\hline 
    %\end{tabular}%
    %\end{center}%


 % PASAR EL ANALISIS COMO ANEXO LAS FORMAS NORMALES

 % y finalmente un análisis de las formas normales sobre la base de datos propuesta.\\
     
 % Con lo anteriormente especificado y de acuerdo a las 2 primeras formas normales
 % especificadas por Edgar Frank Codd \cite[ paǵ. 161-182]{libroBaseDeDatosEspaniol}
 % se diseñó el esquema de la base de datos. A continuación se presentan de manera
 % muy resumida dichas formas normales.


 % CONCLUSIÓN: PASAR AL CAPITULO DE CURVA DE APRENDIZAJE, CÓMO CONCLUSION.

 % \noindent Debido a la capa de abstracción que moodle tiene con respecto al acceso a datos
 % y a que las nuevas funcionalidades se desarrollaran mediante desarrollo de plugins. se
 % decidió utilizar las herramientas que proporciona moodle para la creación de las tablas
 % requeridas para implementar Gamificación.

    %[V] Libro que me pasó CAT (David)   

    % Una vez que se decidio moodle y que se realizó el estudio de
    % factibilidad se procedio a documentar el modelo de datos que
    % se usará contemplando los módulos analisados diseñados hasta
    % el momento.


\begin{comment}
% =================================================================
%   Pasar las siguientes lineas en los archivos presentes en la
%   carpeta de analisis
% =================================================================

% Y SI SE DOCUMENTA COMO FUE??:
%   TOP-DOWN => PRIMERO SE PLANEARON LOS MÓDULOS Y DESPUES SE BAJO A REQUERIMIENTOS

El alcance de este proyecto es representado mediante el Product Backlog (artefacto de Scrum). El product backlog incluye dos tipos de items: los items de documentación, denotados por ls clave {\it {\bf A}x}; y los items de desarrollo del proyecto, denotados por las claves {\it {\bf RF}x} y {\it{\bf RNF}x}.\\

\noindent A continuación se menciona la lista completa de actividades y requerimientos recopilados, debido 
%debido a la larga lista de requerimientos no es posible cumplir para el término del trabajo terminal, 
se realizarán los que tienen mayor prioridad. Esta lista puede ser ampliada o reducida bajo indicaciones estrictas de los directores del trabajo terminal.\\

\noindent Al final de este documento se incluye el documento de Metodología como anexo, el cual detalla las características que deben tener los artefactos de Scrum, así como la configuración de Scrum para este proyecto.\\

\section{Product Backlog}

    \begin{multicols}{2}
    
\newcounter{theActivity}\stepcounter{theActivity}
\newcounter{theRF}\stepcounter{theRF}
\newcounter{theRNF}\stepcounter{theRNF}


\newenvironment{Actividad}[2]{\begin{itemize}\item[\bf A\arabic{theActivity}] {\bf #1:} {#2}}{\end{itemize}\stepcounter{theActivity}}

\newenvironment{RF}[2]{
    \begin{itemize}
        \item[\bf \hypertarget{RF\arabic{theRF}}{RF\arabic{theRF}}] {\bf #1:} {#2}
}{
    \end{itemize}
    \stepcounter{theRF}
}

\newenvironment{RNF}[2]{\begin{itemize}\item[\bf RNF\arabic{theRNF}] {\bf #1:} {#2}}{\end{itemize}\stepcounter{theRNF}}
\newcommand{\Sprint}[1]{{\color{primary}\fbox{Sprint #1}}}
\newcommand{\PBitem}{\item[] \quad}

%\noindent\bb{Sprint 1: Marco Teorico, Metodología, Alcance TT-I}
\begin{Actividad}{Investigar Scrum}{% 
    Redactar el capítulo de I del documento de metodología el cual describe el marco de trabajo Scrum, basándose en la guía oficial.} 
    \PBitem \Sprint{1}%Estimación 2 dias. \Sprint{1} 
\end{Actividad}

\begin{Actividad}{Adaptación de Scrum}{%
    Especificar como es configurado Scrum para el proyecto, definir roles, eventos, y artefactos.}
    \PBitem \Sprint{1}%Estimación 2 dias. \Sprint{1} 
\end{Actividad}

\begin{Actividad}{Adquirir Actionable Gamificación}{%
    Adquisición del libro de Yu-kai Chou, {\it Actionable Gamification: Beyond Points, Badges, and Leaderboards}}
    \PBitem \Sprint{1}%Estimación 2 dias. \Sprint{1} 
\end{Actividad}

\vfill\null\columnbreak

\begin{Actividad}{Investigar Gamificación}{%
    Ampliar la investigación de gamificación, definiciones, sus inicios, uso en la educación.}
    \PBitem \Sprint{1}%Estimación 2 dias. \Sprint{1} 
\end{Actividad}

%\begin{Actividad}{Principios de Gamificación}{%
    %Investigar cada uno de los principios de gamificación de acuerdo con el marco de referencia Octalysis,  buscar técnicas para poder soportar los principios.}
    %\PBitem Estimación 2 dias. \Sprint{1} 
%\end{Actividad}

\begin{Actividad}{Estado del arte}{%
    Investigar el estado del arte en relación a desarrollo de funcinalidades a una plataforma de aprendizaje.}
    \PBitem \Sprint{1}%Estimación 2 dias. \Sprint{1} 
\end{Actividad}

%\begin{Actividad}{Redactar Marco Teórico}{%
    %Invetigar distintos papers que definan la gamificación, describan sus inicios}
    %\PBitem Estimación 2 dias. \Sprint{1} 
%\end{Actividad}

\begin{Actividad}{Establecer los módulos}{%
    Plantear una propuesta integral la cual divida en módulos principales las funcionalidades que tendrá el producto final.}
    \PBitem \Sprint{1}%Estimación 2 dias. \Sprint{1} 
\end{Actividad}

\pagebreak

\begin{Actividad}{Alcance TT-I}{%
    Definir el alcance que tendrá el proyecto para la presentación del trabajo terminal I.}
    \PBitem \Sprint{2}%Estimación 2 dias. \Sprint{1} 
\end{Actividad}

%\hfill\bigskip\\\noindent\bb{Sprint 2: Investigación de Implementación}

\begin{Actividad}{Módulo I y II}{%
    Especificar el funcionamiento y el análisis inicial que se realiza en el módulo de Recompensa.}
    \PBitem \Sprint{2}%Estimación 2 dias. \Sprint{2} 
\end{Actividad}

%\begin{Actividad}{Módulo II}{%
    %Especificar el funcionamiento y el análisis inicial que se realiza en el módulo Financiero.}
    %\PBitem Estimación 2 dias. \Sprint{2} 
%\end{Actividad}

\begin{Actividad}{Módulo III y IV}{%
    Especificar el funcionamiento y el análisis inicial que se realiza en el módulo de Seguimiento.}
    \PBitem \Sprint{2}%Estimación 2 dias. \Sprint{2} 
\end{Actividad}

%\begin{Actividad}{Módulo IV}{%
    %Especificar el funcionamiento y el análisis inicial que se realiza en el módulo de Competencia.}
    %\PBitem Estimación 2 dias. \Sprint{2} 
%\end{Actividad}

\begin{Actividad}{Módulo V y VI}{%
    Especificar el funcionamiento y el análisis inicial que se realiza en el módulo de Personalización.}
    \PBitem \Sprint{2}%Estimación 2 dias. \Sprint{2} 
\end{Actividad}

%\begin{Actividad}{Modulo VI}{%
    %De la propuesta de solución, describir cada uno de los módulos y herramientas (submódulos) que contienen.}
    %\PBitem Estimación 2 dias. \Sprint{2} 
%\end{Actividad}

\begin{Actividad}{Alternativas a Moodle}{%
    Investigar otros sistemas gestores de aprendizaje en los que se puedan desarrollar nuevas funcionalidades.}
    \PBitem \Sprint{2}%Estimación 2 días \Sprint{2}
\end{Actividad}


\begin{Actividad}{Implementación Gamificación}{%
    Investigar distintas publicaciónes (papers) en donde se describa la forma en que se implementó gamificación y los resultados obtenidos}
    \PBitem \Sprint{2}%Estimación 2 días 
\end{Actividad}
%trabajo los desarrollos, investigaciones y trabajos ya existentes acerca de la gamificación en una plataforma de aprendizaje.


%\hfill\bigskip\\\noindent\bb{Sprint 3: Reporte Técnico del trabajo Terminal}

\begin{Actividad}{Problema}{%
    Redactar el problema que se pretende atacar con este trabajo terminal.}
    \PBitem \Sprint{3}%Estimación 2 días \Sprint{3}
\end{Actividad}

\begin{Actividad}{Propuesta de Solución}{%
    redactar la propuesta de solución, que se pretendar dar ante el problema}
    \PBitem \Sprint{3}%Estimación 2 días \Sprint{3}
\end{Actividad}

\begin{Actividad}{Justificación}{%
    Redactar por que la justificación de porque surge el proyecto y porque se optó por esa propuesta de solución.}
    \PBitem \Sprint{3}%Estimación 2 días \Sprint{3}
\end{Actividad}

\vfill\null\columnbreak

\begin{Actividad}{Alcances y Limitaciones}{%
    Establecer los alcances y limitaciones que tiene el trabajo terminal}
    \PBitem \Sprint{3}%Estimación 2 días \Sprint{3}
\end{Actividad}

\begin{Actividad}{Instalar Moodle}{%
    Realizar la instalación de Moodle de forma local, en las computadoras de los miembros del equipo.}
    \PBitem \Sprint{3}%Estimación 2 días \Sprint{3}
\end{Actividad}

\begin{Actividad}{Usar Moodle}{%
    Familiarizarse con el uso de Moodle en especifico con las funcioalidades de un administrador, gestionar cursos, gestionar grupos, crear usuarios, etc}
    \PBitem \Sprint{3}%Estimación 2 días \Sprint{3}
\end{Actividad}

%\hfill\bigskip\\\noindent\bb{Sprint 4: Pruebas de Concepto}

\begin{Actividad}{Entorno de desarrollo}{%
    Establecer el entorno de desarrollo sobre el cual se trabajará, incluyendo características de instalación}
    \PBitem \Sprint{4}%Estimación 3 hrs \Sprint{4}
\end{Actividad}

\begin{Actividad}{Filtrar plugins}{%
    Escoger los plugins de los cuales se realizarán las pruebas de concepto y documentar los criterior de discrimnación ocupados.}
    \PBitem \Sprint{4}%Estimación 3 hrs \Sprint{4}
\end{Actividad}

\begin{Actividad}{P1: Database Fields}{%
    Realizar la prueba de database fields}
    \PBitem \Sprint{4}%Estimación 3 hrs \Sprint{4}
\end{Actividad}

\begin{Actividad}{P2: Database Presets}{%
    Realizar la prueba de database presets}
    \PBitem \Sprint{4}%Estimación 3 hrs \Sprint{4}
\end{Actividad}

\begin{Actividad}{P3: User Profile Fields}{%
    Realizar la prueba de user profile fields}
    \PBitem \Sprint{4}%Estimación 3 hrs \Sprint{4}
\end{Actividad}

\begin{Actividad}{P4: Block}{%
    Realizar la prueba de block}
    \PBitem \Sprint{4}%Estimación 3 hrs \Sprint{4}
\end{Actividad}

\begin{Actividad}{Reporte de Pruebas}{%
    Realizar el reporte de pruebas de concepto para entregar al profesor de seguimiento. }
    \PBitem \Sprint{4}%Estimación 3 hrs \Sprint{4}
\end{Actividad}

    
\clearpage

\begin{RF}{Logros en curso}{%
    El sistema debe permitir premiar o otorgar un elemento distintivo (logro) cuando un alumno realice alguna acción positiva dentro de un curso}
    \item[] Prior. MA. %Estimación 2 días. 
\end{RF}

\begin{RF}{Logros en Plataforma}{%
    El sistema debe permitir premiar o otorgar un elemento distintivo (logro) cuando un alumno realice alguna acción positiva a nivel plataforma}
    \item[] Prior. A. %Estimación 2 días. 
\end{RF}

\begin{RF}{Advertencias}{%
    El sistema debe permitir premiar o otorgar un elemento distintivo cuando un alumno realice alguna acción negativa, como si fuera una advertencia.}
    \item[] Prior. M. %Estimación 2 días.
\end{RF}

\begin{RF}{Marcadores}{%
    El sistema deberá permitir a los usuarios visualizar la lista de los mejores alumnos respecto al uso en la plataforma (cuantificado mediante los puntos de experiencia), la mejor calificación ponderada, el mayor numero de preguntas diarias, ...}
    \item[] Prior. MA. %Estimación 2 días. 
\end{RF}

\begin{RF}{Configurar Logros}{%
    El sistema deberá permitir al administrador cambiar el título, imagen y mensaje de los logros y advertencias que se otorgan a los alumnos.}
    \item[] Prior. B. %Estimación 2 días. 
\end{RF}

\begin{RF}{Habilitar Logros}{%
    El sistema deberá permitir al administrador habilitar y deshabilitar los logros y advertencias que el sistema pone a disposición}
    \item[] Prior. M. %Estimación 2 días. 
\end{RF}

\begin{RF}{Experiencia}{%
    El sistema deberá cuantificar como puntos de experiencia, qué tanto usan la plataforma de acuerdo con las actividades/acciones dentro y fuera de los cursos. }
    \item[] Prior. MA. \Sprint{5}%Estimación 2 días. 
\end{RF}

\begin{RF}{Configurar Experiencia}{%
    El sistema deberá contar con un mecanismo mediante el cual el administrador defina la cantidad de experiencia que se otorga al terminar un curso y al realizar distintas actividades/acciones.}
    \item[] Prior. A. \Sprint{5}%Estimación 2 días. 
\end{RF}

\begin{RF}{Niveles}{%
    El sistema deberá asignar a los alumnos un nivel de experiencia correspondiente a los incrementos de experiencia configurados y a la cantidad de experiencia recibida. }
    \item[] Prior. A. \Sprint{5}%Estimación 2 días. 
\end{RF}

\begin{RF}{Incremento de Niveles}{%
    El sistema deberá permitir al administrador configurar la forma en que incrementan los niveles (lineal o porcentual) y el valor de incremento. }
    \item[] Prior. M. \Sprint{5}%Estimación 2 días. 
\end{RF}

\begin{RF}{Configurar Niveles}{%
    El sistema deberá permitir al administrador configurar la imagen, título, descripción y mensaje de los niveles. }
    \item[] Prior. A. \Sprint{5}%Estimación 2 días. 
\end{RF}

\begin{RF}{Incrementar Nivel}{%
    El sistema deberá notificar a un alumno cuando aumente su nivel de experiencia.}
    \item[] Prior. MA. \Sprint{5}%Estimación 2 días. 
\end{RF}

\begin{RF}{Progreso}{%
    El sistema deberá mostrarle al un estudiante el progreso que el mismo tiene del curso, mediante una barra que indique el porcentaje que lleva realizado de un curso.}
    \item[] Prior. A. %Estimación 2 días.  
\end{RF}

\begin{RF}{Configurar Progreso}{%
    El sistema deberá permitir al administrador o al profesor elegir el color de la barra de progreso para los alumnos dentro de un curso.}
    \item[] Prior. M. %Estimación 2 días. 
\end{RF}

\begin{RF}{Narrativa}{%
    El sistema deberá permitir al administrador y profesores incluir una narrativa que se vaya contando conforme el curso vaya avanzando}
    \item[] Prior. MA. %Estimación 2 días. 
\end{RF}

\begin{RF}{Personalización de Curso}{%
    El sistema deberá permitir al administrador o profesor elegir el tema  o visualización del curso que está diseñando. }
    \item[] Prior. M. %Estimación 2 días. 
\end{RF}

\begin{RF}{Plantillas de Narrativas}{%
    El sistema deberá brindar al administrador plantillas de narrativas. }
    \item[] Prior. M. %Estimación 2 días. 
\end{RF}

\begin{RF}{Personaje de Narrativa}{%
    El sistema deberá permitir al administrador o profesor especificar los datos (nombre,  imagen, etc) de los personajes principales que forman parte de la narrativa de un curso}
    \item[] Prior. A. %Estimación 2 días. 
\end{RF}

\begin{RF}{Monedas}{%
    El sistema deberá de contar una moneda principal y otra secundaría para la adquisición de items mediante la tienda. }
    \item[] Prior. MA. %Estimación 2 días. 
\end{RF}

\begin{RF}{Configurar Esquema Financiero}{%
    El sistema deberá permitir al administrador indicar la cantidad de monedas que es otorgada en determinadas acciones, el precio que tienen los items de la tienda y las equivalencias entre la moneda principal y secundaria.}
    \item[] Prior. A. %Estimación 2 días. 
\end{RF}

\begin{RF}{Tienda}{%
    El sistema deberá de contar con una tienda virtual mediante la cual se puedan adquirir items utilizando las monedas }
    \item[] Prior. MA. %Estimación 2 días. 
\end{RF}

\begin{RF}{Añadir Item}{%
    El sistema deberá permitir al administrador añadir items para que estén disponibles en la plataforma, precio de moneda irreal, su categoría y demás atributos. }
    \item[] Prior. A. %Estimación 2 días. 
\end{RF}

\begin{RF}{Modificar Item}{%
    El sistema deberá permitir al administrador modificar si el precio de moneta irreal, categoría y demás atributos de los items disponibles en la plataforma. }
    \item[] Prior. A. %Estimación 2 días. 
\end{RF}

\begin{RF}{Bloquear Items}{%
    El sistema deberá permitir al administrador bloquear los items para que, posterior a ese momento no se pueda acceder a ellos.}
    \item[] Prior. M. %Estimación 2 días. 
\end{RF}

\begin{RF}{Desbloquear Items}{%
    El sistema deberá permitir al administrador desbloquear los items bloqueados para que estos vuelvan a estar disponibles en la plataforma y se pueda acceder a ellos.}
    \item[] Prior. M. %Estimación 2 días. 
\end{RF}

\begin{RF}{Exportar Items}{%
    El sistema deberá permitir al administrador exportar los items que ha creado con el propósito de guardarlos para posteriormente ser incluidos en otra plataforma con gamificación}
    \item[] Prior. B. %Estimación 2 días. 
\end{RF}

\begin{RF}{Avatar inicial}{%
    El sistema deberá brindarle a los usuarios un avatar inicial y genérico}
    \item[] Prior. MB. %Estimación 2 días. 
\end{RF}

\begin{RF}{Configuración Avatar inicial}{%
    El sistema deberá permitir al administrador establecer la apariencia del avatar que se otorga inicialmente a los usuarios }
    \item[] Prior. MB. %Estimación 2 días. 
\end{RF}

\begin{RF}{Item: Temas}{%
    El sistema deberá contar con items de tipo tema, los cuales permitan cambiar la visualización que un usuario tiene de la plataforma siempre y cuando tenga dicho item}
    \item[] Prior. MA. %Estimación 2 días. 
\end{RF}

\begin{RF}{Item: Skin Avatar}{%
    El sistema deberá contar con items de tipo Skin los cuales contengan un conjunto de elementos que cambien la apariencia del avatar. }
    \item[] Prior. MB. %Estimación 2 días. 
\end{RF}

\begin{RF}{Item: Ropa del Avatar}{%
    El sistema deberá contar con items de tipo Ropa, los cuales permitan cambiar una prenda al avatar que los usuarios tienen.}
    \item[] Prior. MB. %Estimación 2 días. 
\end{RF}

\begin{RF}{Item: Loot-Box}{%
    El sistema deberá con un tipo de item LootBox el cual otorge cualquier otro items utilizando la probabilidad y aleatoridad de acuerdo con las categorias de los items}
    \item[] Prior. A. %Estimación 2 días. 
\end{RF}

\begin{RF}{Configuración de Loot Boxes}{%
    El sistema deberá permitir al administrador cambiar los valores de probabilidad de obtener items de una categoría en especifico mediante un lootBox}
    \item[] Prior. M. %Estimación 2 días. 
\end{RF}

\begin{RF}{Monedas en Curso}{%
    El sistema deberá permitir al administrador/profesor ponerle un alias (nombre e imagen) a las monedas (principal y secundaria) dentro de un curso.}
    \item[] Prior. M. %Estimación 2 días. 
\end{RF}

\begin{RF}{Personalización}{%
    El sistema debe contar con una página de personalización donde el usuario pueda configurar su avatar, además de la visualización que el tiene de la plataforma.}
    \item[] Prior. M. %Estimación 2 días. 
\end{RF}

\begin{RF}{Retar a compañero}{%
    El sistema deberá permitir a un alumno desafiar a otro a una sesión de preguntas acerca de los temas de un curso que tengan en común. }
    \item[] Prior. M. %Estimación 2 días. 
\end{RF}

\begin{RF}{Apuestas en retos}{%
    El sistema deberá permitir a los competidores %que participan 
    apostar una cantidad en mutuo acuerdo entre los alumnos que forma parte de un reto. }
    \item[] Prior. A. %Estimación 2 días. 
\end{RF}

\begin{RF}{Retar al sistema}{%
    El sistema deberá permitir a un alumno desafiar al sistema a una sesión de preguntas, escogiendo un nivel de dificultad }
    \item[] Prior. A. %Estimación 2 días. 
\end{RF}

\begin{RF}{Recompensas en retos con el sistema}{%
    El sistema deberá dar recompensas de acuerdo con el nivel de dificulta elegido por el alumno.}
    \item[] Prior. MA. %Estimación 2 días. 
\end{RF}

\begin{RF}{Torneos}{%
    El sistema deberá permitir al profesor organizar un torneo entre los estudiantes de un curso, con el propósito de comparar el aprovechamiento de los estudiantes}
    \item[] Prior. A. %Estimación 2 días. 
\end{RF}

\begin{RF}{Recompensa en Torneos}{%
    El sistema deberá otorgar una recompensa al primer, segundo y tercer lugar, distintiva. }
    \item[] Prior. A. %Estimación 2 días. 
\end{RF}

\begin{RF}{Poker}{%
    El sistema deberá permitir a los usuarios iniciar una sesión de poker entre distintos alumnos, donde los mismos puedan apostar las monedas ficticias del sistema.}
    \item[] Prior. M. %Estimación 2 días. 
\end{RF}



\begin{RF}{Animación de Personajes}{%
    El sistema deberá de contener animaciones para los distintos elementos de interfaz de usuario. }
    \item[] Prior. MB. %Estimación 2 días. 
\end{RF}

\begin{RNF}{Bajo Acoplamiento}{%
    El sistema deberá trabajar con el menor acoplamiento}
    \item[Tipo] Propiedad de Software
    \item[] Prior. A. %Estimación 2 días. \Sprint{*}
\end{RNF}

\begin{comment}
\begin{RNF}{Robustez}{%
    El sistema debe, INSERTAR AQUI MÉTRICA DE ROBUSTEZ. }
    \item[Tipo] Propiedad de Software
    \item[] Prior. MA. %Estimación 2 días. \Sprint{*}
\end{RNF}
\end{comment}

\begin{RNF}{Modularidad}{%
    El sistema deberá permitir al administrador habilitar únicamente las herramientas que el decida incluir en la plataforma, y deshabilitar las que no requiera.}
    \item[Tipo.] Regla de Negocio
    \item[] Prior. A. %Estimación 2 días. \Sprint{*}
\end{RNF}

\begin{RF}{Preguntas diarias}{%
    El sistema deberá contar con un ejercicio que podrá ser contestado diariamente por el estudiante.}
    \item[] Prior. A. %Estimación 2 días.  
\end{RF}
    \end{multicols}

    %\begin{quote}
    %\noindent {\bf Nota:} El número de {\it Sprint} debe estar presente en todos los items correspondientes al sprint corriente y a los sprints anteriores a este. El atributo {\it Sprint} puede no estar presente en los items que no han sido vinculados a un Sprint. 
    \noindent {\bf Nota:} El número de {\it Sprint} debe estar presente en todos los ítems que ya hayan sido agregados a un Sprint Backlog.
    %\end{quote}
    
\section{Relación módulos-requerimientos}

En la tabla \ref{tab:modreq} se relacionan lo módulos definidos contra los requerimientos encontrados en el Backlog, se muestran sólo los identificadores de los requerimientos para una mayor legibilidad.

\newcommand{\Refr}[1]{{\hyperlink{#1}{#1}}}
\begin{table}[h!]
    \centering
    \begin{tabular}{|c|c|}
    \hline
        Competencia & \Refr{RF36}, \Refr{RF37}, \Refr{RF38}, \Refr{RF40}, \Refr{RF41}, \Refr{RF39}, \Refr{RF42}\\
    \hline
        Seguimiento & \Refr{RF13}, \Refr{RF14}, \Refr{RF44}\\
    \hline
        Financiero & \Refr{RF19}, \Refr{RF20}, \Refr{RF21}, \Refr{RF22}, \Refr{RF23}, \Refr{RF24}, \Refr{RF25}, \Refr{RF26}, \Refr{RF32}, \Refr{RF33}\\
    \hline
        Experiencia & \Refr{RF9}, \Refr{RF7}, \Refr{RF8}, \Refr{RF10}, \Refr{RF11}, \Refr{RF12}\\
    \hline
        Recompensa & \Refr{RF1}, \Refr{RF2}, \Refr{RF4}, \Refr{RF3}, \Refr{RF5}, \Refr{RF6}\\
    \hline
        Personalización & \Refr{RF15}, \Refr{RF16}, \Refr{RF17}, \Refr{RF18}, \Refr{RF27}, \Refr{RF28}, \Refr{RF29}, \Refr{RF30}, \Refr{RF31}, \Refr{RF34}, \Refr{RF35}\\
    \hline
    \end{tabular}
    \caption{Relación entre los módulos y requerimientos}
    \label{tab:modreq}
\end{table}


\end{comment}
