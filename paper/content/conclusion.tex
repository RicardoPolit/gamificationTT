
\section{Conclusiones}

\noindent Con este trabajo terminal se dio solución al problema presentado de que los sistemas de aprendizaje en línea como moodle, no proporcionan un entorno de trabajo donde las funcionalidades dedicadas a la gamificación, sean lo suficientemente flexibles
para brindar un mayor soporte a los objetivos del curso. Los módulos de este proyecto han sido programados para poder brindar una flexibilidad suficiente a los usuarios de moodle para que estos puedan cumplir con sus objetivos del curso, a la vez que utilizan las técnicas de gamificación, y así obtener los beneficios que ésta aporta a los estudiantes.  Por otra parte, la creación de nuevos complementos para moodle es un proceso más complejo de lo que parece debido a varias razones, entre las cuales, la principal es la falta de documentación y la curva de aprendizaje para entender cómo utilizar las distintas herramientas o APIs que moodle brinda.\\

\noindent Con base en el desarrollo de este trabajo terminal, se puede asegurar que
los módulos desarrollados son soportados en un servidor linux que tenga instalado
moodle 3.5 o 3.7 cumpliendo todos sus requisitos, haciendo de este proyecto y sus
módulos una herramienta que se puede utilizar en más de 5 mil servidores a lo largo
del territorio mexicano (según estadísticas proporcionadas por stats.moodle.org) y con la posibilidad, al ser traducido a otros idiomas, de ampliarlo a más de 100 mil servidores a lo largo del mundo.
\vfill\null
