
\section{Introducción}

    \noindent
    La idea de utilizar mecánicas de juegos para resolver problemas y atraer
    distintas audiencias ha sido utilizada a lo largo del tiempo \cite{GamByDesign},
%
    Una manera de hacer más atractivas las tareas realizadas por un grupo de
    personas es midiendo y comparando los resultados de dichas tareas, utilizando
    principios de gamificación para motivar a un conjunto de personas.
    \cite[p. 7]{Octalysis}.
%
    De acuerdo con distintos autores como Zichermann y Cunningham la gamificación es
    el proceso del pensamiento de juegos y realización de mecánicas para involucrar
    a los usuarios y resolver problemas \cite{GamByDesign}, por otro lado Deterding
    {\it et al.} mencionan que la gamificación es el uso de elementos presentes en
    el diseño de juegos en contextos distintos a los mismos \cite{DeterdingDefinition},
    sim embargo no existe una definición que sea ampliamente aceptada o que esté
    establecida formalmente \cite{Seaborn}, por lo que en lo correspondiente a este
    proyecto se refiere a la gamificación cómo ``el uso de mecánicas  de juegos en
    un entorno no lúdico''.\par
%
    \noindent 
    El término ``gamificación'' se originó en la industria de los medios digitales,
    el primer uso documentado se remonta a 2008, pero no fue hasta 2010 que el
    término tuvo una adopción generalizada. \cite{DeterdingGamefulness}. Muchos
    investigadores creen que la gamificación tiene el potencial de motivar y activar
    comportamientos específicos al mismo tiempo que fomenta la lealtad a la
    experiencia gamificada. Además, puede hacer las actividades no lúdicas más
    divertidas, así como impulsar a las personas a realizar tareas de forma
    constante \cite{Aldemir}.\par
%
    % GAMIFICATION IN EDUCATION (GOODS)
    \noindent
    La gamificacion implementada en la educación se centra en incrementar la
    motivación, experiencia y compromiso de los estudiantes, haciendo que estos
    aprendan de una mejor forma. \cite{GamInE-Learning}, \cite{Lee}.
%
    % WHAT IS NEED FOR GAMIFICATION
    Sin embargo, el realizar una correcta implementación requiere de dos tipos
    de habilidades, el diseño basado en juegos y el entendimiento de las técnicas
    del entorno bajo el cual se desea implementar \cite[p. 7]{ForTheWin}.
%
    Si bien es posible incorporar la gamificación sin soporte tecnológico,
    en distintos entornos como la educación, se ha demostrado que es un desafío
    lograrlo en la práctica. Con la ayuda del soporte tecnológico la integración
    de mecánicas de juegos a un curso se puede realizar de una forma más eficiente
    \cite{Wood-Reiners}.
